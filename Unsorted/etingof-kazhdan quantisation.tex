\documentclass[a4paper, 11pt]{article}

%\usepackage[center]{titlesec}

\usepackage{amsfonts, amssymb, amsmath, amsthm, amsxtra}

\usepackage{foekfont}

\usepackage{MnSymbol}

\usepackage{pdfrender, xcolor}
%\pdfrender{StrokeColor=black,LineWidth=.4pt,TextRenderingMode=2}

%\usepackage{minitoc}
%\setcounter{section}{-1}
%\setcounter{tocdepth}{}
%\setcounter{minitocdepth}{}
%\setcounter{secnumdepth}{}

\usepackage{graphicx}

\usepackage[english]{babel}
\usepackage[utf8]{inputenc}
%\usepackage{mathpazo}
%\usepackage{eucal}
\usepackage{eufrak}
\usepackage{bbm}
\usepackage{bm}
\usepackage{csquotes}
\usepackage[nottoc]{tocbibind}
\usepackage{appendix}
\usepackage{float}
\usepackage[T1]{fontenc}
\usepackage[
    left = \flqq{},% 
    right = \frqq{},% 
    leftsub = \flq{},% 
    rightsub = \frq{} %
]{dirtytalk}

\usepackage{imakeidx}
\makeindex

%\usepackage[dvipsnames]{xcolor}
\usepackage{hyperref}
    \hypersetup{
        colorlinks=true,
        linkcolor=teal,
        filecolor=pink,      
        urlcolor=teal,
        citecolor=magenta
    }
\usepackage{comment}

% You would set the PDF title, author, etc. with package options or
% \hypersetup.

\usepackage[backend=biber, style=alphabetic, sorting=nty]{biblatex}
    \addbibresource{bibliography.bib}
\renewbibmacro{in:}{}

\raggedbottom

\usepackage{mathrsfs}
\usepackage{mathtools} 
\mathtoolsset{showonlyrefs} 
%\usepackage{amsthm}
\renewcommand\qedsymbol{$\blacksquare$}

\usepackage{tikz-cd}
\tikzcdset{scale cd/.style={every label/.append style={scale=#1},
    cells={nodes={scale=#1}}}}
\usepackage{tikz}
\usepackage{setspace}
\usepackage[version=3]{mhchem}
\parskip=0.1in
\usepackage[margin=25mm]{geometry}

\usepackage{listings, lstautogobble}
\lstset{
	language=matlab,
	basicstyle=\scriptsize\ttfamily,
	commentstyle=\ttfamily\itshape\color{gray},
	stringstyle=\ttfamily,
	showstringspaces=false,
	breaklines=true,
	frameround=ffff,
	frame=single,
	rulecolor=\color{black},
	autogobble=true
}

\usepackage{todonotes,tocloft,xpatch,hyperref}

% This is based on classicthesis chapter definition
\let\oldsec=\section
\renewcommand*{\section}{\secdef{\Sec}{\SecS}}
\newcommand\SecS[1]{\oldsec*{#1}}%
\newcommand\Sec[2][]{\oldsec[\texorpdfstring{#1}{#1}]{#2}}%

\newcounter{istodo}[section]

% http://tex.stackexchange.com/a/61267/11984
\makeatletter
%\xapptocmd{\Sec}{\addtocontents{tdo}{\protect\todoline{\thesection}{#1}{}}}{}{}
\newcommand{\todoline}[1]{\@ifnextchar\Endoftdo{}{\@todoline{#1}}}
\newcommand{\@todoline}[3]{%
	\@ifnextchar\todoline{}
	{\contentsline{section}{\numberline{#1}#2}{#3}{}{}}%
}
\let\l@todo\l@subsection
\newcommand{\Endoftdo}{}

\AtEndDocument{\addtocontents{tdo}{\string\Endoftdo}}
\makeatother

\usepackage{lipsum}

%   Reduce the margin of the summary:
\def\changemargin#1#2{\list{}{\rightmargin#2\leftmargin#1}\item[]}
\let\endchangemargin=\endlist 

%   Generate the environment for the abstract:
%\newcommand\summaryname{Abstract}
%\newenvironment{abstract}%
    %{\small\begin{center}%
    %\bfseries{\summaryname} \end{center}}

\newtheorem{theorem}{Theorem}[section]
    \numberwithin{theorem}{subsection}
\newtheorem{proposition}{Proposition}[section]
    \numberwithin{proposition}{subsection}
\newtheorem{lemma}{Lemma}[section]
    \numberwithin{lemma}{subsection}
\newtheorem{claim}{Claim}[section]
    \numberwithin{claim}{subsection}
\newtheorem{question}{Question}[section]
    \numberwithin{question}{subsection}

\theoremstyle{definition}
    \newtheorem{definition}{Definition}[section]
        \numberwithin{definition}{subsection}

\theoremstyle{remark}
    \newtheorem{remark}{Remark}[section]
        \numberwithin{remark}{subsection}
    \newtheorem{example}{Example}[section]
        \numberwithin{example}{subsection}    
    \newtheorem{convention}{Convention}[section]
        \numberwithin{convention}{subsection}
    \newtheorem{corollary}{Corollary}[section]
        \numberwithin{corollary}{subsection}

\numberwithin{equation}{section}

\setcounter{section}{-1}

\renewcommand{\cong}{\simeq}
\newcommand{\ladjoint}{\dashv}
\newcommand{\radjoint}{\vdash}
\newcommand{\<}{\langle}
\renewcommand{\>}{\rangle}
\newcommand{\ndiv}{\hspace{-2pt}\not|\hspace{5pt}}
\newcommand{\cond}{\blacktriangle}
\newcommand{\decond}{\triangle}
\newcommand{\solid}{\blacksquare}
\newcommand{\ot}{\leftarrow}
\renewcommand{\-}{\text{-}}
\renewcommand{\mapsto}{\leadsto}
\renewcommand{\leq}{\leqslant}
\renewcommand{\geq}{\geqslant}
\renewcommand{\setminus}{\smallsetminus}
\makeatletter
\DeclareRobustCommand{\cev}[1]{%
  {\mathpalette\do@cev{#1}}%
}
\newcommand{\do@cev}[2]{%
  \vbox{\offinterlineskip
    \sbox\z@{$\m@th#1 x$}%
    \ialign{##\cr
      \hidewidth\reflectbox{$\m@th#1\vec{}\mkern4mu$}\hidewidth\cr
      \noalign{\kern-\ht\z@}
      $\m@th#1#2$\cr
    }%
  }%
}
\makeatother

\newcommand{\N}{\mathbb{N}}
\newcommand{\Z}{\mathbb{Z}}
\newcommand{\Q}{\mathbb{Q}}
\newcommand{\R}{\mathbb{R}}
\newcommand{\bbC}{\mathbb{C}}
\NewDocumentCommand{\x}{e{_^}}{%
  \mathbin{\mathop{\times}\displaylimits
    \IfValueT{#1}{_{#1}}
    \IfValueT{#2}{^{#2}}
  }%
}
\NewDocumentCommand{\pushout}{e{_^}}{%
  \mathbin{\mathop{\sqcup}\displaylimits
    \IfValueT{#1}{_{#1}}
    \IfValueT{#2}{^{#2}}
  }%
}
\newcommand{\supp}{\operatorname{supp}}
\newcommand{\im}{\operatorname{im}}
\newcommand{\coker}{\operatorname{coker}}
\newcommand{\id}{\mathrm{id}}
\newcommand{\chara}{\operatorname{char}}
\newcommand{\trdeg}{\operatorname{trdeg}}
\newcommand{\rank}{\operatorname{rank}}
\newcommand{\trace}{\operatorname{tr}}
\newcommand{\length}{\operatorname{length}}
\newcommand{\height}{\operatorname{ht}}
\renewcommand{\span}{\operatorname{span}}
\newcommand{\e}{\epsilon}
\newcommand{\p}{\mathfrak{p}}
\newcommand{\q}{\mathfrak{q}}
\newcommand{\m}{\mathfrak{m}}
\newcommand{\n}{\mathfrak{n}}
\newcommand{\calF}{\mathcal{F}}
\newcommand{\calG}{\mathcal{G}}
\newcommand{\calO}{\mathcal{O}}
\newcommand{\F}{\mathbb{F}}
\DeclareMathOperator{\lcm}{lcm}
\newcommand{\gr}{\operatorname{gr}}
\newcommand{\vol}{\mathrm{vol}}
\newcommand{\ord}{\operatorname{ord}}
\newcommand{\projdim}{\operatorname{proj.dim}}
\newcommand{\injdim}{\operatorname{inj.dim}}
\newcommand{\flatdim}{\operatorname{flat.dim}}
\newcommand{\globdim}{\operatorname{glob.dim}}
\renewcommand{\Re}{\operatorname{Re}}
\renewcommand{\Im}{\operatorname{Im}}
\newcommand{\sgn}{\operatorname{sgn}}
\newcommand{\coad}{\operatorname{coad}}

\newcommand{\Ad}{\mathrm{Ad}}
\newcommand{\GL}{\mathrm{GL}}
\newcommand{\SL}{\mathrm{SL}}
\newcommand{\PGL}{\mathrm{PGL}}
\newcommand{\PSL}{\mathrm{PSL}}
\newcommand{\Sp}{\mathrm{Sp}}
\newcommand{\GSp}{\mathrm{GSp}}
\newcommand{\GSpin}{\mathrm{GSpin}}
\newcommand{\rmO}{\mathrm{O}}
\newcommand{\SO}{\mathrm{SO}}
\newcommand{\SU}{\mathrm{SU}}
\newcommand{\rmU}{\mathrm{U}}
\newcommand{\rmu}{\mathrm{u}}
\newcommand{\rmV}{\mathrm{V}}
\newcommand{\gl}{\mathfrak{gl}}
\renewcommand{\sl}{\mathfrak{sl}}
\newcommand{\diag}{\mathfrak{diag}}
\newcommand{\pgl}{\mathfrak{pgl}}
\newcommand{\psl}{\mathfrak{psl}}
\newcommand{\fraksp}{\mathfrak{sp}}
\newcommand{\gsp}{\mathfrak{gsp}}
\newcommand{\gspin}{\mathfrak{gspin}}
\newcommand{\frako}{\mathfrak{o}}
\newcommand{\so}{\mathfrak{so}}
\newcommand{\su}{\mathfrak{su}}
%\newcommand{\fraku}{\mathfrak{u}}
\newcommand{\Spec}{\operatorname{Spec}}
\newcommand{\Spf}{\operatorname{Spf}}
\newcommand{\Spm}{\operatorname{Spm}}
\newcommand{\Spv}{\operatorname{Spv}}
\newcommand{\Spa}{\operatorname{Spa}}
\newcommand{\Spd}{\operatorname{Spd}}
\newcommand{\Proj}{\operatorname{Proj}}
\newcommand{\Gr}{\mathrm{Gr}}
\newcommand{\Hecke}{\mathrm{Hecke}}
\newcommand{\Sht}{\mathrm{Sht}}
\newcommand{\Quot}{\mathrm{Quot}}
\newcommand{\Hilb}{\mathrm{Hilb}}
\newcommand{\Pic}{\mathrm{Pic}}
\newcommand{\Div}{\mathrm{Div}}
\newcommand{\Jac}{\mathrm{Jac}}
\newcommand{\Alb}{\mathrm{Alb}} %albanese variety
\newcommand{\Bun}{\mathrm{Bun}}
\newcommand{\loopspace}{\mathbf{\Omega}}
\newcommand{\suspension}{\mathbf{\Sigma}}
\newcommand{\tangent}{\mathrm{T}} %tangent space
\newcommand{\Eig}{\mathrm{Eig}}
\newcommand{\Cox}{\mathrm{Cox}} %coxeter functors
\newcommand{\rmK}{\mathrm{K}} %Killing form
\newcommand{\km}{\mathfrak{km}} %kac-moody algebras
\newcommand{\Dyn}{\mathrm{Dyn}} %associated Dynkin quivers
\newcommand{\Car}{\mathrm{Car}} %cartan matrices of finite quivers

\newcommand{\Ring}{\mathrm{Ring}}
\newcommand{\Cring}{\mathrm{CRing}}
\newcommand{\Alg}{\mathrm{Alg}}
\newcommand{\Leib}{\mathrm{Leib}} %leibniz algebras
\newcommand{\Fld}{\mathrm{Fld}}
\newcommand{\Sets}{\mathrm{Sets}}
\newcommand{\Equiv}{\mathrm{Equiv}} %equivalence relations
\newcommand{\Cat}{\mathrm{Cat}}
\newcommand{\Grp}{\mathrm{Grp}}
\newcommand{\Ab}{\mathrm{Ab}}
\newcommand{\Sch}{\mathrm{Sch}}
\newcommand{\Coh}{\mathrm{Coh}}
\newcommand{\QCoh}{\mathrm{QCoh}}
\newcommand{\Perf}{\mathrm{Perf}} %perfect complexes
\newcommand{\Sing}{\mathrm{Sing}} %singularity categories
\newcommand{\Desc}{\mathrm{Desc}}
\newcommand{\Sh}{\mathrm{Sh}}
\newcommand{\Psh}{\mathrm{PSh}}
\newcommand{\Fib}{\mathrm{Fib}}
\renewcommand{\mod}{\-\mathrm{mod}}
\newcommand{\comod}{\-\mathrm{comod}}
\newcommand{\bimod}{\-\mathrm{bimod}}
\newcommand{\Vect}{\mathrm{Vect}}
\newcommand{\Rep}{\mathrm{Rep}}
\newcommand{\Grpd}{\mathrm{Grpd}}
\newcommand{\Arr}{\mathrm{Arr}}
\newcommand{\Esp}{\mathrm{Esp}}
\newcommand{\Ob}{\mathrm{Ob}}
\newcommand{\Mor}{\mathrm{Mor}}
\newcommand{\Mfd}{\mathrm{Mfd}}
\newcommand{\Riem}{\mathrm{Riem}}
\newcommand{\RS}{\mathrm{RS}}
\newcommand{\LRS}{\mathrm{LRS}}
\newcommand{\TRS}{\mathrm{TRS}}
\newcommand{\TLRS}{\mathrm{TLRS}}
\newcommand{\LVRS}{\mathrm{LVRS}}
\newcommand{\LBRS}{\mathrm{LBRS}}
\newcommand{\Spc}{\mathrm{Spc}}
\newcommand{\Top}{\mathrm{Top}}
\newcommand{\Topos}{\mathrm{Topos}}
\newcommand{\Nil}{\mathfrak{nil}}
\newcommand{\J}{\mathfrak{J}}
\newcommand{\Stk}{\mathrm{Stk}}
\newcommand{\Pre}{\mathrm{Pre}}
\newcommand{\simp}{\mathbf{\Delta}}
\newcommand{\Res}{\mathrm{Res}}
\newcommand{\Ind}{\mathrm{Ind}}
\newcommand{\Pro}{\mathrm{Pro}}
\newcommand{\Mon}{\mathrm{Mon}}
\newcommand{\Comm}{\mathrm{Comm}}
\newcommand{\Fin}{\mathrm{Fin}}
\newcommand{\Assoc}{\mathrm{Assoc}}
\newcommand{\Semi}{\mathrm{Semi}}
\newcommand{\Co}{\mathrm{Co}}
\newcommand{\Loc}{\mathrm{Loc}}
\newcommand{\Ringed}{\mathrm{Ringed}}
\newcommand{\Haus}{\mathrm{Haus}} %hausdorff spaces
\newcommand{\Comp}{\mathrm{Comp}} %compact hausdorff spaces
\newcommand{\Stone}{\mathrm{Stone}} %stone spaces
\newcommand{\Extr}{\mathrm{Extr}} %extremely disconnected spaces
\newcommand{\Ouv}{\mathrm{Ouv}}
\newcommand{\Str}{\mathrm{Str}}
\newcommand{\Func}{\mathrm{Func}}
\newcommand{\Crys}{\mathrm{Crys}}
\newcommand{\LocSys}{\mathrm{LocSys}}
\newcommand{\Sieves}{\mathrm{Sieves}}
\newcommand{\pt}{\mathrm{pt}}
\newcommand{\Graphs}{\mathrm{Graphs}}
\newcommand{\Lie}{\mathrm{Lie}}
\newcommand{\Env}{\mathrm{Env}}
\newcommand{\Ho}{\mathrm{Ho}}
\newcommand{\rmD}{\mathrm{D}}
\newcommand{\Cov}{\mathrm{Cov}}
\newcommand{\Frames}{\mathrm{Frames}}
\newcommand{\Locales}{\mathrm{Locales}}
\newcommand{\Span}{\mathrm{Span}}
\newcommand{\Corr}{\mathrm{Corr}}
\newcommand{\Monad}{\mathrm{Monad}}
\newcommand{\Var}{\mathrm{Var}}
\newcommand{\sfN}{\mathrm{N}} %nerve
\newcommand{\Diam}{\mathrm{Diam}} %diamonds
\newcommand{\co}{\mathrm{co}}
\newcommand{\ev}{\mathrm{ev}}
\newcommand{\bi}{\mathrm{bi}}
\newcommand{\Nat}{\mathrm{Nat}}
\newcommand{\Hopf}{\mathrm{Hopf}}
\newcommand{\Dmod}{\mathrm{D}\mod}
\newcommand{\Perv}{\mathrm{Perv}}
\newcommand{\Sph}{\mathrm{Sph}}
\newcommand{\Moduli}{\mathrm{Moduli}}
\newcommand{\Pseudo}{\mathrm{Pseudo}}
\newcommand{\Lax}{\mathrm{Lax}}
\newcommand{\Strict}{\mathrm{Strict}}
\newcommand{\Opd}{\mathrm{Opd}} %operads
\newcommand{\Shv}{\mathrm{Shv}}
\newcommand{\Char}{\mathrm{Char}} %CharShv = character sheaves
\newcommand{\Huber}{\mathrm{Huber}}
\newcommand{\Tate}{\mathrm{Tate}}
\newcommand{\Affd}{\mathrm{Affd}} %affinoid algebras
\newcommand{\Adic}{\mathrm{Adic}} %adic spaces
\newcommand{\Rig}{\mathrm{Rig}}
\newcommand{\An}{\mathrm{An}}
\newcommand{\Perfd}{\mathrm{Perfd}} %perfectoid spaces
\newcommand{\Sub}{\mathrm{Sub}} %subobjects
\newcommand{\Ideals}{\mathrm{Ideals}}
\newcommand{\Isoc}{\mathrm{Isoc}} %isocrystals
\newcommand{\Ban}{\-\mathrm{Ban}} %Banach spaces
\newcommand{\Fre}{\-\mathrm{Fr\acute{e}}} %Frechet spaces
\newcommand{\Ch}{\mathrm{Ch}} %chain complexes
\newcommand{\Pure}{\mathrm{Pure}}
\newcommand{\Mixed}{\mathrm{Mixed}}
\newcommand{\Hodge}{\mathrm{Hodge}} %Hodge structures
\newcommand{\Mot}{\mathrm{Mot}} %motives
\newcommand{\KL}{\mathrm{KL}} %category of Kazhdan-Lusztig modules
\newcommand{\Pres}{\mathrm{Pres}} %presentable categories
\newcommand{\Noohi}{\mathrm{Noohi}} %category of Noohi groups
\newcommand{\Inf}{\mathrm{Inf}}
\newcommand{\LPar}{\mathrm{LPar}} %Langlands parameters
\newcommand{\ORig}{\mathrm{ORig}} %overconvergent sites
\newcommand{\Quiv}{\mathrm{Quiv}} %quivers
\newcommand{\Def}{\mathrm{Def}} %deformation functors
\newcommand{\Root}{\mathrm{Root}}
\newcommand{\gRep}{\mathrm{gRep}}
\newcommand{\Higgs}{\mathrm{Higgs}}
\newcommand{\BGG}{\mathrm{BGG}}

\newcommand{\Aut}{\mathrm{Aut}}
\newcommand{\Inn}{\mathrm{Inn}}
\newcommand{\Out}{\mathrm{Out}}
\newcommand{\der}{\mathfrak{der}} %derivations on Lie algebras
\newcommand{\frakend}{\mathfrak{end}}
\newcommand{\aut}{\mathfrak{aut}}
\newcommand{\inn}{\mathfrak{inn}} %inner derivations
\newcommand{\out}{\mathfrak{out}} %outer derivations
\newcommand{\Stab}{\mathrm{Stab}}
\newcommand{\Cent}{\mathrm{Cent}}
\newcommand{\Norm}{\mathrm{Norm}}
\newcommand{\stab}{\mathfrak{stab}}
\newcommand{\cent}{\mathfrak{cent}}
\newcommand{\norm}{\mathfrak{norm}}
\newcommand{\Rad}{\operatorname{Rad}}
\newcommand{\Transporter}{\mathrm{Transp}} %transporter between two subsets of a group
\newcommand{\Conj}{\mathrm{Conj}}
\newcommand{\Diag}{\mathrm{Diag}}
\newcommand{\Gal}{\mathrm{Gal}}
\newcommand{\bfG}{\mathbf{G}} %absolute Galois group
\newcommand{\Frac}{\mathrm{Frac}}
\newcommand{\Ann}{\mathrm{Ann}}
\newcommand{\Val}{\mathrm{Val}}
\newcommand{\Chow}{\mathrm{Chow}}
\newcommand{\Sym}{\mathrm{Sym}}
\newcommand{\End}{\mathrm{End}}
\newcommand{\Mat}{\mathrm{Mat}}
\newcommand{\Diff}{\mathrm{Diff}}
\newcommand{\Autom}{\mathrm{Autom}}
\newcommand{\Artin}{\mathrm{Artin}} %artin maps
\newcommand{\sk}{\mathrm{sk}} %skeleton of a category
\newcommand{\eqv}{\mathrm{eqv}} %functor that maps groups $G$ to $G$-sets
\newcommand{\Inert}{\mathrm{Inert}}
\newcommand{\Fil}{\mathrm{Fil}}
\newcommand{\Prim}{\mathfrak{Prim}}
\newcommand{\Nerve}{\mathrm{N}}
\newcommand{\Hol}{\mathrm{Hol}} %holomorphic functions %holonomy groups
\newcommand{\Bi}{\mathrm{Bi}} %Bi for biholomorphic functions
\newcommand{\chev}{\mathfrak{chev}} %chevalley relations
\newcommand{\bfLie}{\mathbf{Lie}} %non-reduced lie algebra associated to generalised cartan matrices
\newcommand{\frakLie}{\mathfrak{Lie}} %reduced lie algebra associated to generalised cartan matrices
\newcommand{\frakChev}{\mathfrak{Chev}} 
\newcommand{\Rees}{\operatorname{Rees}}
\newcommand{\Dr}{\mathrm{Dr}} %Drinfeld's quantum double 

\renewcommand{\projlim}{\varprojlim}
\newcommand{\indlim}{\varinjlim}
\newcommand{\colim}{\operatorname{colim}}
\renewcommand{\lim}{\operatorname{lim}}
\newcommand{\toto}{\rightrightarrows}
%\newcommand{\tensor}{\otimes}
\NewDocumentCommand{\tensor}{e{_^}}{%
  \mathbin{\mathop{\otimes}\displaylimits
    \IfValueT{#1}{_{#1}}
    \IfValueT{#2}{^{#2}}
  }%
}
\NewDocumentCommand{\singtensor}{e{_^}}{%
  \mathbin{\mathop{\odot}\displaylimits
    \IfValueT{#1}{_{#1}}
    \IfValueT{#2}{^{#2}}
  }%
}
\NewDocumentCommand{\hattensor}{e{_^}}{%
  \mathbin{\mathop{\hat{\otimes}}\displaylimits
    \IfValueT{#1}{_{#1}}
    \IfValueT{#2}{^{#2}}
  }%
}
\NewDocumentCommand{\semidirect}{e{_^}}{%
  \mathbin{\mathop{\rtimes}\displaylimits
    \IfValueT{#1}{_{#1}}
    \IfValueT{#2}{^{#2}}
  }%
}
\newcommand{\eq}{\operatorname{eq}}
\newcommand{\coeq}{\operatorname{coeq}}
\newcommand{\Hom}{\mathrm{Hom}}
\newcommand{\Maps}{\mathrm{Maps}}
\newcommand{\Tor}{\mathrm{Tor}}
\newcommand{\Ext}{\mathrm{Ext}}
\newcommand{\Isom}{\mathrm{Isom}}
\newcommand{\stalk}{\mathrm{stalk}}
\newcommand{\RKE}{\operatorname{RKE}}
\newcommand{\LKE}{\operatorname{LKE}}
\newcommand{\oblv}{\mathrm{oblv}}
\newcommand{\const}{\mathrm{const}}
\newcommand{\free}{\mathrm{free}}
\newcommand{\adrep}{\mathrm{ad}} %adjoint representation
\newcommand{\NL}{\mathbb{NL}} %naive cotangent complex
\newcommand{\pr}{\operatorname{pr}}
\newcommand{\Der}{\mathrm{Der}}
\newcommand{\Frob}{\mathrm{Fr}} %Frobenius
\newcommand{\frob}{\mathrm{f}} %trace of Frobenius
\newcommand{\bfpt}{\mathbf{pt}}
\newcommand{\bfloc}{\mathbf{loc}}
\DeclareMathAlphabet{\mymathbb}{U}{BOONDOX-ds}{m}{n}
\newcommand{\0}{\mymathbb{0}}
\newcommand{\1}{\mathbbm{1}}
\newcommand{\2}{\mathbbm{2}}
\newcommand{\Jet}{\mathrm{Jet}}
\newcommand{\Split}{\mathrm{Split}}
\newcommand{\Sq}{\mathrm{Sq}}
\newcommand{\Zero}{\mathrm{Z}}
\newcommand{\SqZ}{\Sq\Zero}
\newcommand{\lie}{\mathfrak{lie}}
\newcommand{\y}{\mathrm{y}} %yoneda
\newcommand{\Sm}{\mathrm{Sm}}
\newcommand{\AJ}{\phi} %abel-jacobi map
\newcommand{\act}{\mathrm{act}}
\newcommand{\ram}{\mathrm{ram}} %ramification index
\newcommand{\inv}{\mathrm{inv}}
\newcommand{\Spr}{\mathrm{Spr}} %the Springer map/sheaf
\newcommand{\Refl}{\mathrm{Refl}} %reflection functor]
\newcommand{\HH}{\mathrm{HH}} %Hochschild (co)homology
\newcommand{\Poinc}{\mathrm{Poinc}}
\newcommand{\Simpson}{\mathrm{Simpson}}

\newcommand{\bbU}{\mathbb{U}}
\newcommand{\V}{\mathbb{V}}
\newcommand{\calU}{\mathcal{U}}
\newcommand{\calW}{\mathcal{W}}
\newcommand{\rmI}{\mathrm{I}} %augmentation ideal
\newcommand{\bfV}{\mathbf{V}}
\newcommand{\C}{\mathcal{C}}
\newcommand{\D}{\mathcal{D}}
\newcommand{\T}{\mathscr{T}} %Tate modules
\newcommand{\calM}{\mathcal{M}}
\newcommand{\calN}{\mathcal{N}}
\newcommand{\calP}{\mathcal{P}}
\newcommand{\calQ}{\mathcal{Q}}
\newcommand{\A}{\mathbb{A}}
\renewcommand{\P}{\mathbb{P}}
\newcommand{\calL}{\mathcal{L}}
\newcommand{\E}{\mathcal{E}}
\renewcommand{\H}{\mathbf{H}}
\newcommand{\scrS}{\mathscr{S}}
\newcommand{\calX}{\mathcal{X}}
\newcommand{\calY}{\mathcal{Y}}
\newcommand{\calZ}{\mathcal{Z}}
\newcommand{\calS}{\mathcal{S}}
\newcommand{\calR}{\mathcal{R}}
\newcommand{\scrX}{\mathscr{X}}
\newcommand{\scrY}{\mathscr{Y}}
\newcommand{\scrZ}{\mathscr{Z}}
\newcommand{\calA}{\mathcal{A}}
\newcommand{\calB}{\mathcal{B}}
\renewcommand{\S}{\mathcal{S}}
\newcommand{\B}{\mathbb{B}}
\newcommand{\bbD}{\mathbb{D}}
\newcommand{\G}{\mathbb{G}}
\newcommand{\horn}{\mathbf{\Lambda}}
\renewcommand{\L}{\mathbb{L}}
\renewcommand{\a}{\mathfrak{a}}
\renewcommand{\b}{\mathfrak{b}}
\renewcommand{\c}{\mathfrak{c}}
\renewcommand{\t}{\mathfrak{t}}
\renewcommand{\r}{\mathfrak{r}}
\newcommand{\fraku}{\mathfrak{u}}
\newcommand{\bbX}{\mathbb{X}}
\newcommand{\frakw}{\mathfrak{w}}
\newcommand{\frakG}{\mathfrak{G}}
\newcommand{\frakH}{\mathfrak{H}}
\newcommand{\frakE}{\mathfrak{E}}
\newcommand{\frakF}{\mathfrak{F}}
\newcommand{\g}{\mathfrak{g}}
\newcommand{\h}{\mathfrak{h}}
\renewcommand{\k}{\mathfrak{k}}
\newcommand{\z}{\mathfrak{z}}
\newcommand{\fraki}{\mathfrak{i}}
\newcommand{\frakj}{\mathfrak{j}}
\newcommand{\del}{\partial}
\newcommand{\bbE}{\mathbb{E}}
\newcommand{\scrO}{\mathscr{O}}
\newcommand{\bbO}{\mathbb{O}}
\newcommand{\scrA}{\mathscr{A}}
\newcommand{\scrB}{\mathscr{B}}
\newcommand{\scrF}{\mathscr{F}}
\newcommand{\scrG}{\mathscr{G}}
\newcommand{\scrM}{\mathscr{M}}
\newcommand{\scrN}{\mathscr{N}}
\newcommand{\scrP}{\mathscr{P}}
\newcommand{\frakS}{\mathfrak{S}}
\newcommand{\frakT}{\mathfrak{T}}
\newcommand{\calI}{\mathcal{I}}
\newcommand{\calJ}{\mathcal{J}}
\newcommand{\scrI}{\mathscr{I}}
\newcommand{\scrJ}{\mathscr{J}}
\newcommand{\scrK}{\mathscr{K}}
\newcommand{\calK}{\mathcal{K}}
\newcommand{\scrV}{\mathscr{V}}
\newcommand{\scrW}{\mathscr{W}}
\newcommand{\bbS}{\mathbb{S}}
\newcommand{\scrH}{\mathscr{H}}
\newcommand{\bfA}{\mathbf{A}}
\newcommand{\bfB}{\mathbf{B}}
\newcommand{\bfC}{\mathbf{C}}
\renewcommand{\O}{\mathbb{O}}
\newcommand{\calV}{\mathcal{V}}
\newcommand{\scrR}{\mathscr{R}} %radical
\newcommand{\rmZ}{\mathrm{Z}} %centre of algebra
\newcommand{\rmC}{\mathrm{C}} %centralisers in algebras
\newcommand{\bfGamma}{\mathbf{\Gamma}}
\newcommand{\scrU}{\mathscr{U}}
\newcommand{\rmW}{\mathrm{W}} %Weil group
\newcommand{\frakM}{\mathfrak{M}}
\newcommand{\frakN}{\mathfrak{N}}
\newcommand{\frakB}{\mathfrak{B}}
\newcommand{\frakX}{\mathfrak{X}}
\newcommand{\frakY}{\mathfrak{Y}}
\newcommand{\frakZ}{\mathfrak{Z}}
\newcommand{\frakU}{\mathfrak{U}}
\newcommand{\frakR}{\mathfrak{R}}
\newcommand{\frakP}{\mathfrak{P}}
\newcommand{\frakQ}{\mathfrak{Q}}
\newcommand{\sfX}{\mathsf{X}}
\newcommand{\sfY}{\mathsf{Y}}
\newcommand{\sfZ}{\mathsf{Z}}
\newcommand{\sfS}{\mathsf{S}}
\newcommand{\sfT}{\mathsf{T}}
\newcommand{\sfOmega}{\mathsf{\Omega}} %drinfeld p-adic upper-half plane
\newcommand{\rmA}{\mathrm{A}}
\newcommand{\rmB}{\mathrm{B}}
\newcommand{\calT}{\mathcal{T}}
\newcommand{\sfA}{\mathsf{A}}
\newcommand{\sfD}{\mathsf{D}}
\newcommand{\sfE}{\mathsf{E}}
\newcommand{\frakL}{\mathfrak{L}}
\newcommand{\K}{\mathrm{K}}
\newcommand{\rmT}{\mathrm{T}}
\newcommand{\bfv}{\mathbf{v}}
\newcommand{\bfg}{\mathbf{g}}
\newcommand{\frakV}{\mathfrak{V}}
\newcommand{\frakv}{\mathfrak{v}}
\newcommand{\bfn}{\mathbf{n}}
\renewcommand{\o}{\mathfrak{o}}

\newcommand{\aff}{\mathrm{aff}}
\newcommand{\ft}{\mathrm{ft}} %finite type
\newcommand{\fp}{\mathrm{fp}} %finite presentation
\newcommand{\fr}{\mathrm{fr}} %free
\newcommand{\tft}{\mathrm{tft}} %topologically finite type
\newcommand{\tfp}{\mathrm{tfp}} %topologically finite presentation
\newcommand{\tfr}{\mathrm{tfr}} %topologically free
\newcommand{\aft}{\mathrm{aft}}
\newcommand{\lft}{\mathrm{lft}}
\newcommand{\laft}{\mathrm{laft}}
\newcommand{\cpt}{\mathrm{cpt}}
\newcommand{\cproj}{\mathrm{cproj}}
\newcommand{\qc}{\mathrm{qc}}
\newcommand{\qs}{\mathrm{qs}}
\newcommand{\lcmpt}{\mathrm{lcmpt}}
\newcommand{\red}{\mathrm{red}}
\newcommand{\fin}{\mathrm{fin}}
\newcommand{\fd}{\mathrm{fd}} %finite-dimensional
\newcommand{\gen}{\mathrm{gen}}
\newcommand{\petit}{\mathrm{petit}}
\newcommand{\gros}{\mathrm{gros}}
\newcommand{\loc}{\mathrm{loc}}
\newcommand{\glob}{\mathrm{glob}}
%\newcommand{\ringed}{\mathrm{ringed}}
%\newcommand{\qcoh}{\mathrm{qcoh}}
\newcommand{\cl}{\mathrm{cl}}
\newcommand{\et}{\mathrm{\acute{e}t}}
\newcommand{\fet}{\mathrm{f\acute{e}t}}
\newcommand{\profet}{\mathrm{prof\acute{e}t}}
\newcommand{\proet}{\mathrm{pro\acute{e}t}}
\newcommand{\Zar}{\mathrm{Zar}}
\newcommand{\fppf}{\mathrm{fppf}}
\newcommand{\fpqc}{\mathrm{fpqc}}
\newcommand{\orig}{\mathrm{orig}} %overconvergent topology
\newcommand{\smooth}{\mathrm{sm}}
\newcommand{\sh}{\mathrm{sh}}
\newcommand{\op}{\mathrm{op}}
\newcommand{\cop}{\mathrm{cop}}
\newcommand{\open}{\mathrm{open}}
\newcommand{\closed}{\mathrm{closed}}
\newcommand{\geom}{\mathrm{geom}}
\newcommand{\alg}{\mathrm{alg}}
\newcommand{\sober}{\mathrm{sober}}
\newcommand{\dR}{\mathrm{dR}}
\newcommand{\rad}{\mathfrak{rad}}
\newcommand{\discrete}{\mathrm{discrete}}
%\newcommand{\add}{\mathrm{add}}
%\newcommand{\lin}{\mathrm{lin}}
\newcommand{\Krull}{\mathrm{Krull}}
\newcommand{\qis}{\mathrm{qis}} %quasi-isomorphism
\newcommand{\ho}{\mathrm{ho}} %homotopy equivalence
\newcommand{\sep}{\mathrm{sep}}
\newcommand{\unr}{\mathrm{unr}}
\newcommand{\tame}{\mathrm{tame}}
\newcommand{\wild}{\mathrm{wild}}
\newcommand{\nil}{\mathrm{nil}}
\newcommand{\defm}{\mathrm{defm}}
\newcommand{\Art}{\mathrm{Art}}
\newcommand{\Noeth}{\mathrm{Noeth}}
\newcommand{\affd}{\mathrm{affd}}
%\newcommand{\adic}{\mathrm{adic}}
\newcommand{\pre}{\mathrm{pre}}
\newcommand{\coperf}{\mathrm{coperf}}
\newcommand{\perf}{\mathrm{perf}}
\newcommand{\perfd}{\mathrm{perfd}}
\newcommand{\rat}{\mathrm{rat}}
\newcommand{\cont}{\mathrm{cont}}
\newcommand{\dg}{\mathrm{dg}}
\newcommand{\almost}{\mathrm{a}}
%\newcommand{\stab}{\mathrm{stab}}
\newcommand{\heart}{\heartsuit}
\newcommand{\proj}{\mathrm{proj}}
\newcommand{\qproj}{\mathrm{qproj}}
\newcommand{\pd}{\mathrm{pd}}
\newcommand{\crys}{\mathrm{crys}}
\newcommand{\prisma}{\mathrm{prisma}}
\newcommand{\FF}{\mathrm{FF}}
\newcommand{\sph}{\mathrm{sph}}
\newcommand{\lax}{\mathrm{lax}}
\newcommand{\weak}{\mathrm{weak}}
\newcommand{\strict}{\mathrm{strict}}
\newcommand{\mon}{\mathrm{mon}}
\newcommand{\sym}{\mathrm{sym}}
\newcommand{\lisse}{\mathrm{lisse}}
\newcommand{\an}{\mathrm{an}}
\newcommand{\ad}{\mathrm{ad}}
\newcommand{\sch}{\mathrm{sch}}
\newcommand{\rig}{\mathrm{rig}}
\newcommand{\pol}{\mathrm{pol}}
\newcommand{\plat}{\mathrm{flat}}
\newcommand{\proper}{\mathrm{proper}}
\newcommand{\compl}{\mathrm{compl}}
\newcommand{\non}{\mathrm{non}}
\newcommand{\access}{\mathrm{access}}
\newcommand{\comp}{\mathrm{comp}}
\newcommand{\tstructure}{\mathrm{t}} %t-structures
\newcommand{\pure}{\mathrm{pure}} %pure motives
\newcommand{\mixed}{\mathrm{mixed}} %mixed motives
\newcommand{\num}{\mathrm{num}} %numerical motives
\newcommand{\ess}{\mathrm{ess}}
\newcommand{\topological}{\mathrm{top}}
\newcommand{\convex}{\mathrm{cvx}}
\newcommand{\locconvex}{\mathrm{lcvx}}
\newcommand{\ab}{\mathrm{ab}} %abelian extensions
\newcommand{\inj}{\mathrm{inj}}
\newcommand{\surj}{\mathrm{surj}} %coverage on sets generated by surjections
\newcommand{\eff}{\mathrm{eff}} %effective Cartier divisors
\newcommand{\Weil}{\mathrm{Weil}} %weil divisors
\newcommand{\lex}{\mathrm{lex}}
\newcommand{\rex}{\mathrm{rex}}
\newcommand{\AR}{\mathrm{A\-R}}
\newcommand{\cons}{\mathrm{c}} %constructible sheaves
\newcommand{\tor}{\mathrm{tor}} %tor dimension
\newcommand{\semisimple}{\mathrm{ss}}
\newcommand{\connected}{\mathrm{connected}}
\newcommand{\cg}{\mathrm{cg}} %compactly generated
\newcommand{\nilp}{\mathrm{nilp}}
\newcommand{\isg}{\mathrm{isg}} %isogenous
\newcommand{\qisg}{\mathrm{qisg}} %quasi-isogenous
\newcommand{\irr}{\mathrm{irr}} %irreducible represenations
\newcommand{\simple}{\mathrm{simple}} %simple objects
\newcommand{\indecomp}{\mathrm{indecomp}}
\newcommand{\preproj}{\mathrm{preproj}}
\newcommand{\preinj}{\mathrm{preinj}}
\newcommand{\reg}{\mathrm{reg}}
\renewcommand{\ss}{\mathrm{ss}}

%prism custom command
\usepackage{relsize}
\usepackage[bbgreekl]{mathbbol}
\usepackage{amsfonts}
\DeclareSymbolFontAlphabet{\mathbb}{AMSb} %to ensure that the meaning of \mathbb does not change
\DeclareSymbolFontAlphabet{\mathbbl}{bbold}
\newcommand{\prism}{{\mathlarger{\mathbbl{\Delta}}}}

\begin{document}

    \title{Etingof-Kazhdan Quantisation}
    
    \author{Dat Minh Ha}
    \maketitle
    
    \begin{abstract}
    
    \end{abstract}
    
    {
    \hypersetup{} 
    %\dominitoc
    \tableofcontents %sort sections alphabetically
    }

    \begin{convention}
        We fix once and for all a field $k$. All Hopf algebras will be assumed to have invertible antipodes. All modules will be left-modules unless mentioned explicitly to be otherwise.
    \end{convention}
    
    \section{Braidings and R-matrices}
        \subsection{Braided monoidal categories}
    
        \subsection{R-matrices}
            \begin{convention}
                Throughout, suppose that $(\C, \tensor, \1)$ is a \textit{braided} $k$-linear monoidal category.
            \end{convention}
        
            \begin{definition}[R-matrices] \label{def: R_matrices}
                An \textbf{R-matrix} on an object $V \in \Ob(\C)$ is an automorphism $c \in \Aut(V \tensor V)$ satisfying the so-called \textbf{Yang-Baxter equation}:
                    $$(c \tensor \id_V) \circ (\id_V \tensor c) \circ (c \tensor \id_V) = (\id_V \tensor c) \circ (c \tensor \id_V) \circ (\id_V \tensor c) \in \Aut(V \tensor V \tensor V)$$
            \end{definition}
            \begin{example}
                For each object $V \in \Ob(\C)$, the twist map:
                    $$\tau_{V, V}: V \tensor V \xrightarrow[]{\cong} V \tensor V$$
                will be an example of an R-matrix. For a proof, consider the following Coxeter relation in the symmetric group $S_3 \cong \Aut(\{1, 2, 3\})$:
                    $$(12)(23)(12) = (23)(12)(23)$$    
            \end{example}
            \begin{remark}
                Finding all solutions to the Yang-Baxter equation is a very difficult task. To see why, specialise to the case $\C := k\mod^{\fr}$ and write the equation out in terms of a fixed basis. 
                
                Nevertheless, it is known how one might generate new solutions from a previously known one (e.g. $c = \tau_{V, V}$). In particular, if $V \in \Ob(\C)$ is an object and $c \in \Aut(V \tensor V)$ is an R-matrix then the following automorphisms on $V \tensor V$ will also be R-matrices:
                    $$c^{-1}, \lambda c, \tau_{V, V} \circ c \circ \tau_{V, V}$$
                wherein $\lambda \in k$ is some scalar.
            \end{remark}
            \begin{example}[The Yang-Baxter equation for finite-dimensional simple $\rmU_q(\sl_2)$-modules]
                Let us write $\rmU_q := \rmU_q(\sl_2(k))$ and for any root lattice element $\lambda = \e q^n\in k$ (with $\e = \pm 1$), write $V_q(n) = V_q(\lambda)$ to mean the corresponding $(n + 1)$-dimensional simple $\rmU_q$-module of highest-weight $\lambda$. For details on the theory of highest-weight $\rmU_q$-modules, we refer the reader to \cite[Chapters VI and VII]{kassel_quantum_groups}. Also, for convenience, we shall also assume for this example that $k$ is algebraically closed and of characteristic $0$. 
                
                We now analyse certain R-matrices in the braided monoidal category of finite-dimensional $k$-linear $\rmU_q$-representations. We caution the reader that it is important to view these $\rmU_q$-representations as certain $\rmU_q$-modules, since R-matrices thereon are not simply certain $k$-linear automorphisms but $\rmU_q$-linear automorphisms. To that end, recall the quantum Clebsch-Gordon formula (cf. \cite[Theorem VII.7.1]{kassel_quantum_groups}), which tells us that the tensor product $V_q(n) \tensor_k V_q(m)$ is a semi-simple $\rmU_q$-module in the following manner:
                    $$V_q(n) \tensor_k V_q(m) \cong \bigoplus_{0 \leq p \leq m} V_q(n + m - 2p)$$
                By specialising to the case wherein $m = n$, one gets that:
                    $$V_q(n) \tensor_k V_q(n) \cong \bigoplus_{0 \leq p \leq n} V_q(2(n - p))$$
                and so $\rmU_q$-module automorphisms on $V_q(n) \tensor_k V_q(n)$ are nothing but $\rmU_q$-module automorphisms on $\bigoplus_{0 \leq p \leq m} V_q(2(n - p))$. Such automorphisms can thus be computed via examining their actions on the generators $v^{\lambda}_0 \in V_q(\lambda)$ of the simple (hence cyclic) $\rmU_q$-modules $V_q(\lambda)$ (for $\lambda \in \{\pm q^{ 2(n - p) }\}_{0 \leq p \leq n}$). In particular, one sees that any $\rmU_q$-module automorphism on $V_q(n) \tensor_k V_q(n)$ is necessarily diagonalisable. 
                
                Now, let us use the quantum Clebsch-Gordon formula again to obtain the following direct sum decomposition of $V_q(n) \tensor V_q(n) \tensor V_q(n)$:
                    $$
                        \begin{aligned}
                            V_q(n) \tensor_k V_q(n) \tensor_k V_q(n) & \cong V_q(n) \tensor_k \left( \bigoplus_{0 \leq p \leq n} V_q(2(n - p)) \right)
                            \\
                            & \cong \bigoplus_{0 \leq p \leq n} ( V_q(n) \tensor_k V_q(2(n - p)) )
                            \\
                            & \cong \bigoplus_{0 \leq p \leq n} \bigoplus_{0 \leq j \leq 2(n - p)} V_q(n + 2(n - p - j))
                        \end{aligned}
                    $$
            \end{example}
            
    \section{Reconstructing bialgebras from their representation categories}
        \subsection{Fibre functors}
            For the sake of fixing terminologies, let us recall a few relevant definitions.
            \begin{definition}[Rigid monoidal categories] \label{def: rigid_monoidal_categories}
                A monoidal category is said to be \textbf{rigid} if and only if every of its objects has two-sided duals. 
            \end{definition}
            \begin{definition}[(Multi-)ring categories] \label{def: ring_categories}
                \cite[Definition 4.1.1]{EGNO} A \textbf{(multi-)ring category} is a locally finite $k$-linear monoidal category $(\C, \tensor, \1)$ on which the tensor product bifunctor $\tensor: \C \x \C \to \C$ is $k$-bilinear on morphisms and biexact, i.e. for all $V, V', V'' \in \Ob(\C)$, the natural map:
                    $$\C(V, V') \tensor_k \C(V', V'') \to \C(V, V'')$$
                is $k$-linear. Without the local finiteness condition, we shall say \say{infinite (multi-)ring categories}.
            \end{definition}
            \begin{definition}[Tensor functors] \label{def: tensor_functors}
                Let $\C, \D$ be multi-ring categories over $k$ and $F: \C \to \D$ be a monoidal functor. One says that $F$ is a \textbf{tensor functor} if and only if there is a monoidal natural isomorphism:
                    $$J_{-, -}: F(- \tensor -) \xrightarrow[]{\cong} F(-) \tensor F(-)$$
                along with an isomorphism:
                    $$F(\1_{\C}) \cong \1_{\D}$$
            \end{definition}
            \begin{definition}[(Multi-)tensor categories] \label{def: tensor_categories}
                \cite[Definition 4.1.1]{EGNO} A  \textbf{multi-tensor category} over a field $k$ is a locally finite $k$-linear abelian rigid monoidal category $(\C, \tensor, \1)$ on which the bifunctor:
                    $$\tensor: \C \x \C \to \C$$
                is $k$-bilinear on morphisms. If, in addition, we have that $\C(\1, \1) \cong k$ then $\C$ will be called a \textbf{tensor category}. Without the local finiteness condition, we shall say \say{infinite (multi-)tensor categories}.
            \end{definition}
            \begin{proposition}[Biexactness of tensor products] \label{prop: biexactness_of_tensor_products}
                \cite[Proposition 4.2.1]{EGNO} Let $(\C, \tensor, \1)$ be a multi-tensor category. Then the bifunctor $\tensor: \C \x \C \to \C$ will always be biexact. 
            \end{proposition}
                \begin{proof}
                    This is a direct consequence of the fact that multi-tensor categories are rigid as monoidal categories. 
                \end{proof}
            \begin{corollary}
                (Infinite) (multi-)tensor categories are special cases of (infinite) (multi-)ring categories.
            \end{corollary}
            
            Now that all the etymological background materials have been put into place, let us actually begin discussing the so-called \say{Tannakian reconstruction theory}. 
            \begin{convention}
                From now on until the end of this subsection, let $(\C, \tensor, \1)$ be a ring category over $k$. 
            \end{convention}
            The following definition bears significant resemblances with the notion of fibre functors in Grothendieck's interpretation of Galois theory (cf. \cite[Expos\'e V]{SGA1}).
            \begin{definition}[Fibre functors] \label{def: fibre_functors}
                A fibre functor on $(\C, \tensor, \1)$ is an \textit{exact} tensor functor:
                    $$F: \C \to k\mod^{\fr}$$
                (wherein $k\mod^{\fr}$ is a tensor category over $k$ in the usual manner).
            \end{definition}
            The following proposition, though very important, is more-or-less self-evident. 
            \begin{proposition}[Fibre functors for representation categories of associative algebras] \label{prop: fibre_functors_for_representation_categories_of_associative_algebras}
                Let $A$ be an associative $k$-algebra. Then, the forgetful functor:
                    $$A\mod \xrightarrow[]{F_A := \Hom_A(A, -)} k\mod^{\fr}$$
                is a fibre functor. Furthermore, we have that:
                    $$A \cong \End_{\Mon\Nat}(F_A)$$
                as associative $k$-algebras. 
            \end{proposition}
            Let us see what happens when we consider $\C$ to be the representation category of a bialgebra over $k$. Again, the proposition is somewhat self-evident. For details, see the discussion immediately preceding \cite[Section 5.2]{EGNO} as well as \cite[Section 5.3]{EGNO}.
            \begin{remark}[Tensor products of hom-sets]
                One thing that we would like to remind the reader about is that, should $(H, \mu, \eta, \Delta, \e)$ be a $k$-bialgebra and if $V, W, V', W'$ are $H$-modules then the canonical map:
                    $$\Hom_H(V, W) \tensor_k \Hom_H(V', W') \to \Hom_H(V \tensor_k W, V' \tensor_k W')$$
                is generally a monomorphism of $H \tensor_k H^{\op}$-modules, and is furthermore an isomorphism whenever $V, V'$ are \textit{free} over $H$. When $H$ is furthermore a Hopf algebra with (invertible) antipode $\sigma: H \to H^{\op}$ then the above becomes an monomorphism (respectively, isomorphism) of $H$-modules via the composite algebra homomorphism:
                    $$H \tensor_k H^{\op} \xrightarrow[\cong]{\id_H \tensor \sigma^{-1}} H \tensor_k H \xrightarrow[]{\mu} H$$
            \end{remark}
            \begin{proposition}[Fibre functors for representation categories of bialgebras] \label{prop: fibre_functors_for_representation_categories_of_bialgebras}
                Let $(H, \mu, \eta, \Delta, \e)$ be a $k$-bialgebra and consider the forgetful functor:
                    $$H\mod \xrightarrow[]{F_H := \Hom_H(H, -)} k\mod^{\fr}$$
                which we know by proposition \ref{prop: fibre_functors_for_representation_categories_of_associative_algebras} above to be tensorial by virtue of being a fibre functor; denote this tensor structure on $F_H$ by:
                    $$F_H(- \tensor_H -) \xrightarrow[\cong]{J_{-, -}} F_H(-) \tensor_k F_H(-)$$
                Then:
                    $$H \cong \End_{\Mon\Nat}(F_H)$$
                not simply as $k$-algebras, but furthermore as $k$-bialgebras; the $k$-coalgebra structure on $\End_{\Mon\Nat}(F_H)$ is induced by that of $H$, via the Deligne tensor product $F_H \boxtimes F_H$, in the sense that there are the following $k$-algebra homomorphisms:
                    $$\End_{\Mon\Nat}(F_H) \xrightarrow[]{\cong} H \xrightarrow[]{\Delta} H \tensor_k H \xrightarrow[]{\cong} \End_{\Mon\Nat}(F_H) \tensor_k \End_{\Mon\Nat}(F_H) \xrightarrow[]{J_{-, -}^{-1}} \End_{\Mon\Nat}(F_H \boxtimes_k F_H)$$
                    $$\End_{\Mon\Nat}(F_H) \xrightarrow[]{\cong} H \xrightarrow[]{\e} k$$
            \end{proposition}
            \begin{corollary}[Fibre functors for representation categories of Hopf algebras] \label{coro: fibre_functors_for_representation_categories_of_hopf_algebras}
                Let $(H, \mu, \eta, \Delta, \e, \sigma)$ be a Hopf $k$-algebra and consider the forgetful functor:
                    $$H\mod \xrightarrow[]{F_H := \Hom_H(H, -)} k\mod^{\fr}$$
                Then there will be an isomorphism of Hopf $k$-algebra isomorphism:
                    $$H \cong \End_{\Mon\Nat}(F_H)$$
                Particularly, the antipode on $H$ is given by:
                    $$\End_{\Mon\Nat}(F_H) \xrightarrow[]{\cong} H \xrightarrow[]{\sigma} H^{\op} \xrightarrow[]{\cong} \End_{\Mon\Nat}(F_H)^{\op}$$
            \end{corollary}
                \begin{proof}
                    True because Hopf $k$-algebras form a full subcategory of that of $k$-bialgebras. 
                \end{proof}
        
        \subsection{Quasi-cocommutative Hopf algebras}
        
        \subsection{Quantum doubles of finite-dimensional Hopf algebras and Yetter-Drinfeld modules}
        
    \section{Quantisation of finite-dimensional Lie bialgebras}
        \begin{convention}
            We fix once and for all a field $k$ of characteristic $0$. 
        \end{convention}
        
        \subsection{(Co-)Poisson algebras and deformation quantisation}
            \begin{definition}[Flat deformations] \label{def: flat_deformations}
                Let $A$ be an associative $k$-algebra.
                
                A \textbf{$n^{th}$ order flat algebra/coalgebra/bialgebra/Hopf algebra deformation}\footnote{Algebraic geometers might call these \say{$n^{th}$ order thickenings}.} (for some $n \geq 1$) of $A$ is a \textit{flat} algebra/coalgebra/bialgebra/Hopf algebra $\tilde{A}$ over $k[\hbar]/\hbar^n$ such that there exist an isomorphism of algebra/coalgebra/bialgebra/Hopf algebra over $k$ as follows:
                    $$\tilde{A} \tensor_{k[\hbar]/\hbar^n} k \xrightarrow[]{\cong} A$$
                    
                A \textbf{formal flat deformation} of $A$ is a flat algebra/coalgebra/bialgebra/Hopf algebra $\tilde{A}$ over $k[\![\hbar]\!]$ such that there eixst an isomorphism of algebra/coalgebra/bialgebra/Hopf algebra over $k$ as follows:
                    $$\tilde{A} \tensor_{k[\![\hbar]\!]} k \xrightarrow[]{\cong} A$$
            \end{definition}
            \begin{remark}
                Definition \ref{def: flat_deformations} actually works for $k$ being any commutative ring, provided that one requires that $A$ is flat as a $k$-module. 
                
                It is also clear that any cofiltered diagram of $n^{th}$ order flat deformations $\{A_n\}_{n \geq 1}$ of some (flat) $k$-algebra gives rise to a universal formal flat deformation:
                    $$A_{\infty} := \projlim_{n \geq 1} A_n$$
                as a result of Lazard's Theorem, which tells us that cofiltered limits of flat modules are once more flat, as well as the fact that the categories of algebra/coalgebra/bialgebra/Hopf algebra over $k$ has all small cofiltered limits. 
            \end{remark}
            \begin{example}[Rees algebras as formal flat deformations] \label{example: rees_algebras_as_formal_flat_deformations}
                Let $A := \{A_n\}_{n \geq 0}$ be an $\N$-filtered associative $k$-algebra recall that its associated \textbf{Rees algebra} is $\N$-graded associative $k[\![\hbar]\!]$-algebra given by:
                    $$\Rees(A) := \bigoplus_{n \geq 0} A_n \hbar^n$$
                Since we have that:
                    $$\Rees(A)/\hbar \cong \gr(A)$$
                and since $\Rees(A)$ is flat over $k[\![\hbar]\!]$ (if $k$ is replaced by a more general commutative ring, we can guarantee this flatness by assuming, e.g. that each $A_n$ is flat over each $k[\hbar]/\hbar^n$), the algebra $\Rees(A)$ can be thought of as a formal flat ($\N$-graded) deformation of $\gr(A)$ (which itself is $\N$-graded). 
            \end{example}
    
            Let us now introduce Poisson algebras, which in a sense are \say{dual} to Lie algebras; we will explain how this duality occurs momentarily. 
            \begin{definition}[Poisson algebras] \label{def: poisson_algebras}
                A \textbf{Poisson $k$-algebra} is a triple $(A, \cdot, \{-, -\})$ wherein the pair $(A, \cdot)$ is an associative $k$-algebra, and $\{-, -\}: A \tensor_k A \to A$ is a Lie bracket such that, for each $a \in A$, the map:
                    $$\{-, a\}: A \to A$$
                is a derivation on $(A, \cdot)$, i.e. for all $f, g \in A$, one has that:
                    $$\{f \cdot g, a\} = f \cdot \{g, a\} + \{f, a\} \cdot g$$
            \end{definition}
            \begin{definition}[Deformation quantisations] \label{def: deformation_quantisation}
                Let $(A, \cdot, \{-, -\})$ be a Poisson algebra over $k$. A \textbf{formal deformation quantisation} of $A$ is a formal flat deformation $\tilde{A}$ (over $k[\![\hbar]\!]$) of $A$, endowed with the Poisson structure given by the commutator $[-, -]_{\tilde{A}}$, such that:
                    $$\{f, g\} = \frac{1}{\hbar}[\tilde{f}, \tilde{g}]_{\tilde{A}} \pmod{\hbar}$$
                for any lifts $\tilde{f}, \tilde{g} \in \tilde{A}$ of $f, g \in A$ (i.e. $\tilde{f} \equiv f, \tilde{g} \equiv g \pmod{\hbar}$).
            \end{definition}
            \begin{lemma}[Poisson brackets from formal flat deformations of commutative algebras] \label{lemma: poisson_brackets_from_formal_flat_deformations_of_commutative_algebras}
                Let $(A, \cdot)$ be an associative $k$-algebra with a formal flat deformation $\tilde{A}$ (over $k[\![\hbar]\!]$). Then, the following - for all $f, g \in \rmZ(A)$ and all lifts $\tilde{f}, \tilde{g} \in \tilde{A}$ (i.e. $\tilde{f}, \tilde{g} \equiv f, g \pmod{\hbar}$) - makes the centre $\rmZ(A) \subseteq A$ a commutative Poisson $k$-algebra:
                    $$\{f, g\} := \frac{1}{\hbar}[\tilde{f}, \tilde{g}]_{\tilde{A}} \pmod{\hbar}$$
            \end{lemma}
                \begin{proof}
                    As $f, g \in \rmZ(A)$, we have that $[f, g]_A = 0$ and hence $[\tilde{f}, \tilde{g}]_{\tilde{A}} \in \hbar \tilde{A}$; we thus see firstly that the expression $\frac{1}{\hbar}[\tilde{f}, \tilde{g}]_{\tilde{A}} \in \tilde{A}$ is well-defined. It is well-known that the commutator $[-, -]_{\tilde{A}}$ is a Lie bracket on $\tilde{A}$, so $\{-, -\}$ given by $\{f, g\} := \frac{1}{\hbar}[\tilde{f}, \tilde{g}]_{\tilde{A}} \pmod{\hbar}$ for all $f, g \in \rmZ(A)$ is therefore a well-defined Lie bracket on $\rmZ(A)$. It is also easy to check that $\{f, g\}$ is depends not on the choices of lifts $\tilde{f}, \tilde{g}$. We also have that $\{f, -\}: \rmZ(A) \to \rmZ(A)$ is a derivation with respect to the multiplication $\cdot$ on $A$ for every $f \in \rmZ(A)$, owing to the fact that $[\varphi, -]_{\tilde{A}}$ is a derivation with respect to the multiplication on $\tilde{A}$, for any $\varphi \in \tilde{A}$ (one can then take $\varphi := \tilde{f}$ for any $\tilde{f} \equiv f \pmod{\hbar}$).
                \end{proof}
            \begin{corollary}[Deformation quantisations of commutative Poisson algebras induced by formal flat deformations] \label{coro: deformation_quantisation_of_poisson_algebras_from_formal_flat_deformations}
                Let $(A, \cdot)$ be a \textit{commutative} $k$-algebra with a formal flat deformation $\tilde{A}$ (over $k[\![\hbar]\!]$). Then, the following - for all $f, g \in A$ and all lifts $\tilde{f}, \tilde{g} \in \tilde{A}$ (i.e. $\tilde{f}, \tilde{g} \equiv f, g \pmod{\hbar}$) - makes $A = \rmZ(A)$ a commutative Poisson $k$-algebra:
                    $$\{f, g\} := \frac{1}{\hbar}[\tilde{f}, \tilde{g}]_{\tilde{A}} \pmod{\hbar}$$
                and thus making $\tilde{A}$ a deformation quantisation of $A$ in the sense of definition \ref{def: deformation_quantisation}.
            \end{corollary}
            \begin{example}[PBW deformations] \label{example: PBW_deformations}
                Let $\g$ be an arbitrary Lie algebra over $k$ and denote the PBW filtration on the universal enveloping algebra of $\g$ by $\rmU(\g) := \{\rmU(\g)_n\}_{n \geq 0}$. The PBW Theorem tells us that there is a canonical isomorphism of $\N$-graded commutative $k$-algebras:
                    $$\Sym(\g) \xrightarrow[]{\cong} \gr \rmU(\g)$$
                Using example \ref{example: rees_algebras_as_formal_flat_deformations}, we thus know that:
                    $$\Rees(\rmU(\g)) := \bigoplus_{n \geq 0} \rmU(\g)_n \hbar^n$$
                is a formal flat $\N$-graded deformation over $k[\![\hbar]\!]$ of the $\N$-graded commutative $k$-algebra $\Sym(\g)$. Lemma \ref{lemma: poisson_brackets_from_formal_flat_deformations_of_commutative_algebras} then tells us that $\Sym(\g)$ carries a canonically defined Poisson structure $\{-, -\}$ given by:
                    $$\{x, y\} := \frac{1}{\hbar}[\tilde{x}, \tilde{y}]_{\Rees(\rmU(\g))} \pmod{\hbar}$$
                for all $x, y \in \Sym(\g)$ and all lifts $\tilde{x}, \tilde{y} \in \Rees(\rmU(\g))$ thereof. 
    
                Now, recall also that when $\g$ is finite-dimensional, the symmetric algebra $\Sym(\g)$ also has a natural bialgebra structure (cf. \cite[Chapter III]{kassel_quantum_groups}) which happens to be compatible with the one on $\gr \rmU(\g)$, and since both are Hopf algebras, the PBW isomorphism upgrades to an isomorphism of $\N$-graded Hopf $k$]
                algebras:
                    $$\Sym(\g) \xrightarrow[\text{Hopf}]{\cong} \gr \rmU(\g)$$
                In other words, when $\g$ is finite-dimensional over $k$, $\Rees(\rmU(\g))$ is actually a formal flat $\N$-graded Hopf algebra deformation of $\Sym(\g)$. 
            \end{example}
        
        \subsection{Quasi-classical limits: from QUEs to finite-dimensional Lie bialgebras}
            \begin{definition}[Formal QUEs] \label{def: formal_QUEs}
                A \textbf{formal(ly) quantised universal enveloping algebra (QUE)} of a Lie algebra $\g$ is a formal flat Hopf algebra deformation of $\rmU(\g)$ in the sense of definition \ref{def: flat_deformations}.
            \end{definition}
        
            \begin{definition}[Lie co/bialgebras] \label{def: lie_co/bialgebras}
                A \textbf{Lie coalgebra} over $k$ is a pair $(\c, \delta)$ consisting of a $k$-vector space $\c$ and a $k$-linear map:
                    $$\delta: \c \to \c \wedge \c$$
                satisfying the following so-called \textbf{co-Jacobi identity}:
                    $$((123) + (231) + (312)) \circ (\delta \tensor_k \id_\c) \circ \delta = 0$$
                Now, suppose that $(\a, [-, -])$ is a Lie algebra over $k$ that is also equipped with a Lie coalgebra structure $\delta: \a \to \a \wedge \a$. If the Lie and co-Lie structures $[-, -]$ and $\delta$ are compatible in the sense below, then we will refer to the triple $(\a, [-, -], \delta)$ as a \textbf{Lie bialgebra} over $k$:
                    $$\delta \circ [-, -] = (\ad \tensor_k \id_{\a} + \id_{\a} \tensor_k \ad) \circ (\id_{\a \tensor_k \a} + (12))$$
            \end{definition}
            \begin{definition}[Manin triples] \label{def: manin_triples}
                A \textbf{Manin triple} consists of the datum of a triple of Lie algebras $(\d, \a^+, \a^-)$ along with an $\ad(\a)$-invariant symmetric and non-degenerate $k$-bilinear form $\kappa: \Sym^2(\d) \to \d$ such that:
                    \begin{itemize}
                        \item $\a^{\pm}$ are Lie subalgebras of $\d$,
                        \item $\d \cong \a^- \oplus \a^+$
                        \item with respect to the inner product $\kappa$, the Lie subalgebras $\a^{\pm}$ are isotropic to one another, i.e. $\kappa(\a^-, \a^+) = 0$, i.e. they are orthogonal complements of one another with respect to $\kappa$.
                    \end{itemize}
            \end{definition}
            \begin{proposition}[Manin triples classify Lie bialgebras] \label{prop: manin_triples_classify_lie_bialgebras}
                There is a bijection:
                    $$\left\{ \text{Finite-dimensional Lie bialgebras $(\a, [-, -], \delta)$} \right\}$$
                    $$\cong$$
                    $$\left\{ \text{Finite-dimensional Manin triples $(\d, \a^-, \a^+)$ with $\a^+ \cong \a$} \right\}$$
            \end{proposition}
                \begin{proof}
                     
                \end{proof}
            \begin{example}[Lie bialgebra structures semi-simple finite-dimensional Lie algebras]
                Let $\g$ be a semi-simple finite-dimensional Lie algebra over $k$ given by $3l$ Chevalley-Serre generators $\{h_i, e_i^{\pm}\}_{1 \leq i \leq l}$. Given the relations amongst these generators, and given the usual coalgebra structure on $\rmU(\g)$, one can write down and check the well-definiteness of the following Lie coalgebra structure on $\g$:
                    $$\delta(h_i) := 0, \delta(e_i^{\pm}) = \pm a_{ii} e_i^{\pm} \wedge h_i$$
                wherein $a_{ii}$ are the diagonal entries of the GCM associated to $\g$. 
            \end{example}
            
            For $\g$ a finite-dimensional Lie algebra over $k$, the process of taking the so-called \say{classical limit} in order to obtain a finite-dimensional Lie bialgebra from a quantisation $\widetilde{\rmU(\g)}$ of $\rmU(\g)$ amounts to constructing a \say{coproduct-like} map:
                $$\delta: \rmU(\g) \to \rmU(\g) \tensor_k \rmU(\g)$$
            whose job is to measure how non-cocommutative the comultiplication on $\widetilde{\rmU(\g)}$ is. It is thus natural to consider:
                $$\delta := \frac1\hbar(\tilde{\Delta} - \tilde{\Delta}^{\cop}) \pmod{\hbar}$$
            One readily checks that this map $\delta$ is well-defined up to a choice of representative $x \pmod{\hbar}$; this map is important, so we give it a name: the bi-Poisson structure (sometimes called the \say{co-Poisson-Hopf structure}) on $\rmU(\g)$. In order to study it, let us make the following definition for the sake of precision.
            \begin{definition}[Co/bi-Poisson structures] \label{def: co/bi_poisson_structures}
                Suppose that $(C, \Delta, \e)$ is a $k$-coalgebra. A \textbf{co-Poisson structure} (or \textbf{co-Poisson cobracket}) on this coalgebra is then a $k$-linear map:
                    $$\delta: C \to C \wedge C$$
                that is a co-Lie structure on $(C, \Delta, \e)$, and simultaneously a coderivation on $(C, \Delta, \e)$ in the following sense:
                    $$(\Delta \tensor_k \id_C) \circ \delta = (\delta \tensor_k \id_C + (23) \circ \id_C \tensor_k \delta) \circ \Delta$$
                Now, if $(H, \mu, \eta, \Delta, \e)$ is a $k$-bialgebra, then a co-Poisson structure thereon is a bi-Poisson structure if and only if it is compatible with the multiplication $\mu$ in the following manner:
                    $$\delta \circ \mu = (\mu \tensor_k \mu) \circ (\Delta \tensor_k \delta + \delta \tensor_k \Delta)$$
            \end{definition}
            What makes the co-Poisson structure on universal enveloping algebras $\rmU(\g)$ interesting and important is that they induce a Lie bialgebra structure on $\g$ (provided that quantisations of $\rmU(\g)$ existed in the first place). 
            \begin{lemma}[Bi-Poisson structures from Hopf algebra deformations] \label{lemma: bi_poisson_structures_from_hopf_algebra_deformations}
                Suppose that $(H, \mu, \eta, \Delta, \e)$ is a cocommutative Hopf algebra over $k$ with a formal flat Hopf algebra deformation $(\tilde{H}, \tilde{\mu}, \tilde{\eta}, \tilde{\Delta}, \tilde{\e})$ over $k[\![\hbar]\!]$. Then, there will be a bi-Poisson structure on $H$ given by:
                    $$\delta := \frac{1}{\hbar}(\tilde{\Delta} - \tilde{\Delta}^{\cop}) \pmod{\hbar}$$
            \end{lemma}
                \begin{proof}
                    Since $H$ is cocommutative, we have that $\Delta - \Delta^{\cop} = 0$ and hence $\tilde{\Delta}(y) - \tilde{\Delta}^{\cop}(y) \in \hbar \tilde{H}$ for all $y \in \tilde{H}$. The expression $\delta := \frac{1}{\hbar}(\tilde{\Delta} - \tilde{\Delta}^{\cop}) \pmod{\hbar}$ is therefore well-defined.
    
                    Let us now check the axioms in definition \ref{def: co/bi_poisson_structures}. Firstly, it is clear that $\delta$ is $k$-linear and co-alternating by construction. Secondly, let us check that the co-Jacobi identity holds. To this end, consider the following:
                        $$
                            \begin{aligned}
                                & \hbar^2 (\delta \tensor_k \id_H) \circ \delta
                                \\
                                = & ( (\tilde{\Delta} - \tilde{\Delta}^{\cop}) \tensor_{k[\![\hbar]\!]} \id_{\tilde{H}} ) \circ (\tilde{\Delta} - \tilde{\Delta}^{\cop}) \pmod{\hbar}
                                \\
                                = & ( \tilde{\Delta} \tensor_{k[\![\hbar]\!]} \id_{\tilde{H}} - \tilde{\Delta}^{\cop} \tensor_{k[\![\hbar]\!]} \id_{\tilde{H}} ) \circ (\tilde{\Delta} - \tilde{\Delta}^{\cop}) \pmod{\hbar}
                                \\
                                = & (\tilde{\Delta} \tensor_{k[\![\hbar]\!]} \id_{\tilde{H}}) \circ \tilde{\Delta} - (\tilde{\Delta}^{\cop} \tensor_{k[\![\hbar]\!]} \id_{\tilde{H}}) \circ \tilde{\Delta} - (\tilde{\Delta}^{\cop} \tensor_{k[\![\hbar]\!]} \id_{\tilde{H}}) \circ \tilde{\Delta} + (\tilde{\Delta}^{\cop} \tensor_{k[\![\hbar]\!]} \id_{\tilde{H}}) \circ \tilde{\Delta}^{\cop} \pmod{\hbar}
                            \end{aligned}
                        $$
                    It is then clear than:
                        $$((123) + (231) + (312)) \circ (\delta \tensor_k \id_H) \circ \delta = 0$$
                    Now, to check that $\delta$ is a co-derivation with respect to $\Delta$, consider the following:
                        $$
                            \begin{aligned}
                                & \hbar (\Delta \tensor_k \id_H) \circ \delta 
                                \\
                                = & (\Delta \tensor_k \id_H) \circ (\tilde{\Delta} - \tilde{\Delta}^{\cop}) \pmod{\hbar} 
                                \\
                                = & 
                            \end{aligned}
                        $$
                    Finally, in order to check that $\delta$ is compatible with the multiplication $\mu$, we will be making use of the fact that $\mu$ is an algebra homomorphism and that $\Delta$ is a coalgebra homomorphism, boththanks to $H$ being a bialgebra:
                        $$
                            \begin{aligned}
                                & \hbar( \delta \circ \mu )
                                \\
                                = & (\tilde{\Delta} - \tilde{\Delta}^{\cop}) \circ \tilde{\mu} \pmod{\hbar}
                                \\
                                = & 
                            \end{aligned}
                        $$
                \end{proof}
            \begin{theorem}[Lie bialgebra structures from bi-Poisson structures] \label{theorem: lie_bialgebra_structures_from_bi_poisson_structures}
                Let $\g$ be a Lie algebra over $k$ and suppose that there exists a formal quantisation $U$ of $\rmU(\g)$. Then, the restriction:
                    $$\delta|_{\g}: \g \to \g \tensor_k \g$$
                of the bi-Poisson structure on $\rmU(\g)$ given by:
                    $$\delta := \frac{1}{\hbar}(\Delta - \Delta^{\cop}) \pmod{\hbar}$$
                down onto the coideal $\g = \Prim(\rmU(\g))$ of primitive elements, determines a Lie bialgebra structure on $\g$.    
            \end{theorem}
                \begin{proof}
                    
                \end{proof}
            \begin{remark}
                Our proof relied heavily on the fact that, by construction of the standard coalgebra structure on $\rmU(\g)$, $\g$ can be identified with the coideal $\Prim(\rmU(\g))$ generated by primitive elements. However, we doubt that this is precisely why the theorem is true. In fact, we suspect that there is a more natural proof that somehow makes use of the identification of Hochschild cohomology and Hopf algebra cohomology; the idea here is that, $\HH^2$ ought to parametrise co/bi-Poisson structures, while $\HH^3$ parametrises obstructions to non-trivially deforming the Hopf algebra in question, and hence the existence of co/bi-Poisson structures on the classical limit can be checked via Hochschild cohomological computations. 
            \end{remark}
        
        \subsection{Quantisation: from finite-dimensional Lie bialgebras to QUEs}
            In the reverse direction, namely from finite-dimensional Lie bialgebras to formal quantisations of their universal enveloping algebras, we adopt a Tannakian perspective in order to construct these quantisations; as a bonus, this approach affords us a functorial description of quantisation. 
            
            \begin{convention}
                If $\calA$ is a $k$-linear category then we shall write $\calA[\hbar]$ for the category whose objects are those of $\calA$ (i.e. $\Ob(\calA[\hbar]) := \Ob(\calA)$) and whose hom-sets are given by:
                    $$\Hom_{\calA[\hbar]}(V, V') := \Hom_{\calA}(V, V')[\hbar] := \Hom_{\calA}(V, V') \tensor_k k[\hbar]$$
                for all $V, V' \in \Ob(\calA)$.
            \end{convention}
            
            \begin{definition}[Drinfeld categories of bialgebras] \label{def: drinfeld_categories_of_finite_type_bialgebras}
                Suppose that $H$ is a bialgebra over $k$, and consider the localisations:
                    $$H\mod[\hbar] \to H\mod[\hbar]/\hbar^n$$
                at the thick subcategories of $H\mod[\hbar]$ spanned by $\hbar^n$-torsion $H$-modules (with $n \geq 1$); in other words, the categories $H\mod[\hbar]/\hbar^n$ are those whose objects are (left-)$H$-modules and whose hom-sets are given by:
                    $$\Hom_{H\mod[\hbar]/\hbar^n}(V, V') := \Hom_{H\mod}(V, V')[\hbar]/\hbar^n$$
                for all $V, V' \in \Ob(H\mod)$. We shall also be equipping each of these categories $H\mod[\hbar]/\hbar^n$ with the fibre functor:
                    $$F_H[\hbar]/\hbar^n: H\mod[\hbar]/\hbar^n \to k[\hbar]/\hbar^n\mod^{\fr}$$
                that is the forgetful functor; recall that this functor is corepresentable by $H$, i.e.:
                    $$F_H[\hbar]/\hbar^n \cong \Hom_{H\mod[\hbar]/\hbar^n}(H, -) := \Hom_{H\mod}(H, -)/\hbar^n$$
                    
                The so-called \textbf{Drinfeld category} of $\g$, which we shall denote by $H\mod[\![\hbar]\!]$ is then the weak $2$-limit of the diagram $\{ ( H\mod[\hbar]/\hbar^n, F_H[\hbar]/\hbar^n ) \}_{n \geq 1} := \{ F_H[\hbar]/\hbar^n: H\mod[\hbar]/\hbar^n \to k[\hbar]/\hbar^n\mod^{\fr} \}_{n \geq 1}$, i.e.:
                    $$( H\mod[\![\hbar]\!], \hat{F}_H ) := 2\-\projlim_{n \geq 1} ( H\mod[\hbar]/\hbar^n, F_H[\hbar]/\hbar^n )$$
            \end{definition}
            \begin{remark}[Formal properties of Drinfeld categories] \label{remark: formal_properties_of_drinfeld_categories}
                Firstly, we note that the Drinfeld category of any bialgebra $H$ is $k[\![\hbar]\!]$-linear.
                
                Furthermore (and as the notation suggests), the category $H\mod[\![\hbar]\!]$ comes equipped with a fibre functor:
                    $$\hat{F}_H: H\mod[\![\hbar]\!] \to k[\![\hbar]\!]\mod^{\tfr}$$
                to the category of topologically free $k[\![\hbar]\!]$-modules (i.e. they are of the form $V[\![\hbar]\!]$ for some $k$-vector space $V$). It is also not hard to see, given the construction of the fibre functors $F_H[\hbar]/\hbar^n$, that:
                    $$\hat{F}_H \cong \projlim_{n \geq 1} F_H[\hbar]/\hbar^n \cong \Hom_{H\mod}(H, -)[\![\hbar]\!] \cong F_H[\hbar]^{\wedge}$$
                wherein $F_H[\hbar]: H\mod[\hbar] \to k[\hbar]\mod^{\fr}$ is the forgetful functor to the category of free\footnote{... or equivalently, projective $k[\hbar]$-modules, since $k[\hbar]$ is a PID, owing to $k$ being a field.} $k[\hbar]$-modules, and by $(-)^{\wedge}$ we meant the object-wise $\hbar$-adic completion, i.e.:
                    $$F_H[\hbar]^{\wedge}(V) := \projlim_{n \geq 1} F_H[\hbar](V)/\hbar^n$$
                
                We thus have that:
                    $$\End_{\Mon\Nat}(\hat{F}_H) \cong \End_{\Mon\Nat}(F_H[\hbar]^{\wedge}(V))$$
                and hence that there is an isomorphism of associative algebras:
                    $$\End_{\Mon\Nat}(\hat{F}_H) \cong \projlim_{n \geq 1} H[\hbar]/\hbar^n \cong H[\![\hbar]\!]$$
            \end{remark}
            Clearly, $H[\![\hbar]\!]$ is a formal flat deformation of $H$ (cf. definition \ref{def: flat_deformations}), so let us now attempt to construct a (braided) monoidal structure on the category $H\mod[\![\hbar]\!]$ so as to be able to use reconstruction theory to identify $H[\![\hbar]\!]$, firstly as a $k$-bialgebra (or even has a Hopf $k$-algebra, in the event that $H$ was Hopf to begin with), and secondly as a quantisation of $H$ in the sense of definition \ref{def: deformation_quantisation}. It is not always the case that $H$ admits a quantisation, seeing how $H$ might not have been a bialgebra to begin with\todo{What is a counter-example ?}\footnote{In fact, we need $H$ to be a Hopf algebra so that the tensor bifunctor on $H\mod$ would be $H$-bilinear. Otherwise, one may only consider tensor products of two-sided $H$-modules.}; nevertheless, this is precisely why finite-dimensional Manin triples are important to us: the fact that they correspond to finite-dimensional Lie bialgebras ensures that, if $(\g, \g_+, \g_-)$ is such a triple, then the universal enveloping algebra $\rmU(\g)$ of the Lie bialgebra $\g$ will admit $\rmU(\g)[\![\hbar]\!]$ as a quantisation.
    
    \addcontentsline{toc}{section}{References}
    \printbibliography

\end{document}