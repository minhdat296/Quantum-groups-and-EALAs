\section{The Kassel realisation of UCEs of current Lie algebras}
    \subsection{A recollection of K\"ahler differentials}
        In \cite{kassel_universal_central_extensions_of_lie_algebras}, Kassel showed that centres of UCEs of current algebras (i.e. Lie algebras of the form $\g \tensor_{\bbC} A$ for some commutative $\bbC$-algebra $A$) are identified with spaces of algebraic differential $1$-forms modulo exact forms, so before proceeding, let us take the time to recall some relevant details about algebraic differential forms. 

        There are many approaches to algebraic differential forms. Ultimately, however, we will be relying on the description of modules of differential forms by generators and relations, so let us define them that way.
        \begin{definition}[Modules of K\"ahler differentials] \label{def: kahler_differentials}
            Let $R$ be a base commutative ring and let $S$ be a commutative $R$-algebra. The $S$-module of K\"ahler differentials $\Omega^1_{S/R}$ relative to the ring map $R \to S$ is then the quotient of the $S$-module $S \tensor_R S$ by the $S$-submodule generated by the relations:
                $$ss' \tensor 1 - s' \tensor s - s \tensor s'$$
            given for all $s, s' \in S$
        \end{definition}
        \begin{remark}[Diffentials and derivations] \label{remark: differentials_and_derivations}
            The definition suggests to us that K\"ahler differentials might have something to do with derivations, and indeed they do. In fact, this relationship between algebraic $1$-forms and derivations comes from a universal property that the $S$-module $\Omega^1_{S/R}$ enjoys. Namely, for any $R$-module $M$, there exists a natural isomorphism of $S$-modules:
                $$\Der_R(S, M) \cong \Hom_S(\Omega^1_{S/R}, M)$$
            wherein LHS is the $S$-module of $R$-linear derivations from $S$ into $M$ (cf. \cite[\href{https://stacks.math.columbia.edu/tag/00RO}{Tag 00RO}]{stacks}). From this universal property, one infers that $\Omega^1_{S/R}$ is isomorphic to the $S$-module generated by the set:
                $$\{ds\}_{s \in S}$$
            whose elements are constrained by the relations:
                $$d(ss') = s' ds + s ds'$$
            given for all $s, s' \in S$. The isomorphism in question is given by:
                $$s' ds \mapsto s' \tensor s$$
            for all $s, s' \in S$.
            
            A particular instance of this phenomenon is that $R$-linear derivations from $S$ to itself are dual to differential $1$-forms relative to $R \to S$, in the sense that there is an $S$-module isomorphism:
                $$\Der_R(S) := \Der_R(S, S) \cong \Hom_A(\Omega^1_{S/R}, A)$$
        \end{remark}
        \begin{remark} 
            If $\Omega^1_{S/R}$ is finite free of rank $n$ over $R$, e.g.:
                $$\Omega^1_{S/R} \cong \bigoplus_{1 \leq i \leq n} S dv_i$$
            then we can identify:
                $$\Der_R(S) \cong \bigoplus_{1 \leq i \leq n} S \del_{v_i}$$
            where $\del_{v_i} := \frac{\del}{\del v_i} \in \Der_R(S)$ are the preimages under the isomorphism $\Der_R(S) \xrightarrow[]{\cong} \Hom_S(\Omega^1_{S/R}, S)$ of the $S$-linear duals of the generators $dv_i \in \Omega^1_{S/R}$.
        \end{remark}
        
        The following well-known lemmas are very useful. Proofs can be be found in any standard reference on general commutative algebra (e.g. \cite[\href{https://stacks.math.columbia.edu/tag/00AO}{Tag 00AO}]{stacks}).
        \begin{lemma}[$1$-forms over polynomial algebras] \label{lemma: 1_forms_over_polynomial_algebras}
            \cite[\href{https://stacks.math.columbia.edu/tag/00RX}{Tag 00RX}]{stacks} Let $R$ be a commutative ring and fix some $n \in \Z_{\geq 0}$. In this case, $\Omega^1_{R[v_1, ..., v_n]/R}$ will be free and of finite rank $n$ as an $R[v_1, ..., v_n]$-module; in particular, it admits the set $\{dv_1, ..., dv_n\}$ as a $R[v_1, ..., v_n]$-linear basis.
        \end{lemma}
        \begin{lemma}[Localisation of $1$-forms] \label{lemma: localisation_1_forms}
            \cite[\href{https://stacks.math.columbia.edu/tag/031G}{Tag 031G}]{stacks} Let $k$ be a field\footnote{... so that the only prime ideal of $k$ would be $(0)$.} and fix some $n \in \Z_{\geq 0}$, and consider the canonical ring homomorphism $k \to k[v_1, ..., v_n]$. Then, for any $1 \leq i \leq n$, there will be a $k[v_1, ..., v_n][v_i^{-1}]$-module isomorphism:
                $$\Omega^1_{k[v_1, ..., v_n][v_i^{-1}]/k} \cong \Omega^1_{k[v_1, ..., v_n]/k}[v_i^{-1}]$$
        \end{lemma}

        Let us end this subsection with the following examples, which will be useful for what comes later on.
        \begin{example}
            Let $k$ be a field.
        
            Per lemma \ref{lemma: 1_forms_over_polynomial_algebras}, we know that:
                $$\Omega^1_{k[v, t]/k} \cong k[v, t] dv \oplus k[v, t] dv$$
            Using lemma \ref{lemma: localisation_1_forms}, we then see that:
                $$\Omega^1_{k[v^{\pm 1}, t^{\pm 1}]/k} \cong k[v^{\pm 1}, t^{\pm 1}] dv \oplus k[v^{\pm 1}, t^{\pm 1}] dt$$
                $$\Omega^1_{k[v^{\pm 1}, t]/k} \cong k[v^{\pm 1}, t] dv \oplus k[v^{\pm 1}, t] dt$$
        \end{example}

        \begin{remark}[Gradings on derivations] \label{remark: gradings_on_derivations}
            If $S$ is graded as an $R$-module by some abelian group $Z$, say:
                $$S = \bigoplus_{n \in Z} S_n$$
            then $\Der_R(S)$ will naturally inherit a $Z$-grading as well. This induced grading will be given by:
                $$\Der_R(S) := \bigoplus_{n \in Z} S_n \del_{v_i}$$
            In particular, if $S := R[v_1^{\pm 1}, ..., v_n^{\pm 1}]$ - and hence $Z = \Z^n$ - then the above will imply that:
                $$\deg \del_{v_i} = (0, ..., -1, ..., 0)$$
            with $-1$ being placed in the $i^{th}$ entry.
        \end{remark}

        \begin{remark}[Gradings on $1$-forms] \label{remark: gradings_on_1_forms}
            If $\bbC$ is a field and $A$ is a commutative $\bbC$-algebra graded by an abelian group $Z$, say:
                $$A := \bigoplus_{n \in Z} A_n$$
            then $\Omega^1_{A/\bbC}$ and $\bar{\Omega}^1_{A/\bbC} := \Omega^1_{A/\bbC}/d(A)$ will carry induced $Z$-gradings. To see why this is the case, recall firstly that the $A$-module $\Omega^1_{A/\bbC}$ is isomorphic to the quotient of $A \tensor_{\bbC} A$\footnote{... which carries a grading induced by the one on $A$, with graded components given by $(A \tensor_{\bbC} A)_d \cong \bigoplus_{m + n = d} (A_m \tensor_{\bbC} A_n)$ for all $d \in Z$.} by the homogeneous\footnote{Because $\deg fg \tensor 1 = \deg g \tensor f = \deg f \tensor g = \deg f + \deg g$.} $A$-submodule generated by the relations:
                $$fg \tensor 1 - g \tensor f - f \tensor g$$
            given for all $f, g \in A$. This implies that $\Omega^1_{A/\bbC}$ inherits a $Z$-grading from the one on $A \tensor_{\bbC} A$, given by:
                $$\deg f dg = \deg f \tensor g = \deg f + \deg g$$
            for all homogeneous elements $f, g \in A$. 
        \end{remark}

    \subsection{UCEs of current Lie algebras}
        For the sake of establishing the terminology, let us make the following definition:
        \begin{definition}[Current algebras] \label{def: current_algebras}
            Let $A$ be a commutative algebra over $\bbC$. The vector space:
                $$\g \tensor_{\bbC} A$$
            with the following Lie bracket:
                $$[x f, y g]_{\g \tensor_{\bbC} A} := [x, y]_{\g} \tensor fg$$
            (given for all $x, y \in \g$ and all $f, g \in A$) shall then be referred to as a \textbf{current algebra}. 
                
            Also, we will be abbreviating:
                $$xf := x \tensor f$$
            for $x \in \g$ and $f \in A$.
        \end{definition}
        \begin{convention}[(Multi-)loop algebras]
            When $A \cong \bbC[v_1^{\pm 1}, ..., v_n^{\pm 1}]$, it is common to refer to $\g \tensor_{\bbC} A$ as a \textbf{multiloop algebra}. When $n = 1$, we will only be saying \textbf{loop algebra} (cf. e.g. \cite[Chapter 7]{kac_infinite_dimensional_lie_algebras}).
        \end{convention}

        \begin{convention} \label{conv: cyclic_1_forms}
            Let $R \to S$ be a homomorphism of commutative rings. Then, let us write:
                $$\bar{\Omega}^1_{S/R} := \Omega^1_{S/R}/d(S)$$
            where $d(S) := \im d$, with $d: S \to \Omega^1_{S/R}$ being the universal $R$-linear map resulting from the universal property mentioned in remark \ref{remark: differentials_and_derivations}. Note that this is only an $R$-module, not an $S$-module. $\Omega^1_{S/R}$ is spanned as a vector space by elements of the form $g df$, for some $f, g \in S$, so $\bar{\Omega}^1_{S/R}$ is spanned by their images under the canonical surjective linear map $\Omega^1_{S/R} \to \bar{\Omega}^1_{S/R}$. Let us denote these images by:
                $$g\bar{d}f$$
        \end{convention}

        In order to characterise centres of UCE Kassel constructed in the proof of \cite[Theorem 3.3(iii)]{kassel_universal_central_extensions_of_lie_algebras} a $\bbC$-linear map:
            $$\e: \bigwedge^2 (\g \tensor_{\bbC} A) \to \bar{\Omega}^1_{A/\bbC}$$
        by the formula\footnote{One can also take $\e(x f, y g) := -(x, y)_{\g} f \bar{d}g$, since $-f \bar{d}g \equiv g \bar{d}f \pmod{d(A)}$. This results in an isomorphic extension.}:
            $$\e(x f, y g) := (x, y)_{\g} g \bar{d}f$$
        for all $x, y \in \g$ and for all $f, g \in A$.
        \begin{lemma} \label{lemma: lie_brackets_on_UCEs_of_current_algebras}
            The linear map $\e: \bigwedge^2 (\g \tensor_{\bbC} A) \to \bar{\Omega}^1_{A/\bbC}$ constructed above is a well-defined $2$-cocycle (cf. definition \ref{def: twisted_semi_direct_products}) of $\g \tensor_{\bbC} A$ with values in $\bar{\Omega}^1_{A/\bbC}$.
        \end{lemma}
            \begin{proof}
                By construction, $\e$ is already bilinear and skew-symmetric, so the only thing to show is that it satisfies the Jacobi identity. To this end, pick $x, y, z \in \g$ and $f, g, h \in A$ and then consider the following computations in $\bar{\Omega}^1_{A/\bbC}$:
                    $$
                        \begin{aligned}
                            & \e(xf, [y, z]_{\g} gh) + \e(yg, [z, x]_{\g} hf) + \e(zh, [x, y]_{\g} fg)
                            \\
                            = & (x, [y, z]_{\g})_{\g} gh \bar{d}f + (y, [z, x]_{\g})_{\g} hf \bar{d}g + (z, [x, y]_{\g})_{\g} fg \bar{d}h
                            \\
                            = & (x, [y, z]_{\g})_{\g} ( gh \bar{d}f + hf \bar{d}g + fg\bar{d}h )
                            \\
                            = & 0
                        \end{aligned}
                    $$
                The last equality came from the fact that:
                    $$gh df + hf dg + fg dh = d(fgh)$$
                in $\Omega^1_{A/\bbC}$, per definition \ref{def: kahler_differentials} (see also remark \ref{remark: differentials_and_derivations}), which hence implies that:
                    $$gh \bar{d}f + hf \bar{d}g + fg \bar{d}h \equiv 0 \pmod{d(A)}$$
                in $\bar{\Omega}^1_{A/\bbC}$.
            \end{proof}
        Kassel then showed that the central extension $(\g \tensor_{\bbC} A) \oplus^{\e} \bar{\Omega}^1_{A/\bbC}$ is in fact universal.
        \begin{theorem}[The Kassel realisation] \label{theorem: kassel_realisation}
            \cite[Corollary 3.5]{kassel_universal_central_extensions_of_lie_algebras} For the (perfect) Lie $\bbC$-algebra $\g \tensor_{\bbC} A$, the central extension corresponding to the $2$-cocycle $\e$ as in lemma \ref{lemma: lie_brackets_on_UCEs_of_current_algebras} is universal, i.e.:
                $$\uce(\g \tensor_{\bbC} A) \cong (\g \tensor_{\bbC} A) \oplus^{\e} \bar{\Omega}^1_{A/\bbC}$$
            with Lie bracket as in lemma \ref{lemma: lie_brackets_on_UCEs_of_current_algebras}.
        \end{theorem}

        \begin{remark}[Induced gradings on UCEs] \label{remark: induced_gradings_on_UCEs}
            Let $A := \bigoplus_{n \in Z} A_n$ be a commutative $\bbC$-algebra graded by some abelian group $Z$. In that case, we shall have that:
                $$\g \tensor_{\bbC} A \cong \bigoplus_{n \in Z} (\g \tensor_{\bbC} A_n)$$
            which tells us that $\g \tensor_{\bbC} A$ is also $Z$-graded. At the same time, we now know from theorem \ref{theorem: kassel_realisation} that:
                $$\uce(\g \tensor_{\bbC} A) \cong (\g \tensor_{\bbC} A) \oplus \bar{\Omega}^1_{A/\bbC}$$
            From remark \ref{remark: gradings_on_1_forms} that the $\bbC$-vector space:
                $$\Omega^1_{A/\bbC}$$
            also carries a $Z$-grading induced by the one on $A$. Since the vector subspace $d(A)$ is also $Z$-graded ($\deg df = \deg f \tensor 1 = \deg f$ for all $f \in A$, as $d(A) \cong A \tensor 1$), the quotient vector space $\bar{\Omega}^1_{A/\bbC}$ is also naturally graded by the abelian group $Z$, with graded components given by:
                $$(\bar{\Omega}^1_{A/\bbC})_n := (\Omega^1_{A/\bbC})_n/d(A_n) \cong (\Omega^1_{A/\bbC})_n/(A_n \tensor 1)$$
            for all $n \in Z$. From these facts, we see that the $Z$-grading on $\g \tensor_{\bbC} A$ extends to the larger vector space $\uce(\g \tensor_{\bbC} A)$. Namely, the graded components are given by:
                $$\uce(\g \tensor_{\bbC} A)_n := (\g \tensor_{\bbC} A_n) \oplus (\bar{\Omega}^1_{A/\bbC})_n$$
            for all $n \in Z$.
        \end{remark}