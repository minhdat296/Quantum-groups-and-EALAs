\section{Lie algebra cohomology} \label{section: lie_algebra_cohomology_appendix}
    \subsection{A brief account of the generalities}
        Let us work over a fixed field $k$.
    
        Even though it is true that every twisted semi-direct product:
            $$0 \to \t \to \t \rtimes^{\sigma} \d \to \d \to 0$$
        gives rise to a $2$-cocycle $\sigma: \bigwedge^2 \d \to \t$, the converse statement is slightly more subtle. Namely, it is possible that two unequal $2$-cocycles:
            $$\sigma \not = \sigma'$$
        would give rise to two twisted semi-direct products that are isomorphic as extensions of $\d$ (see e.g. proposition \ref{prop: cohomological_non_triviality_of_billig_toroidal_cocycles}), and in order to capture this fact, we will need to introduce the language of Lie algebra cohomology, which also necessitates a brief discussion of homological algebra in general. We refer the reader to \cite{hilton_stammbach_homological_algebra} as a general reference on homological algebra, and particularly, Chapter VII therein as a reference on the generalities of Lie algebra cohomology. For brevity, our account of Lie algebra cohomology will have to be a bit \textit{ad hoc}. Also, in discussing homological algebra, we will have to make use of some categorical language, but we will keep this to a minimum, only mentioning what is absolutely needed. We refer the reader to \cite{maclane}, particularly Chapter VIII therein, for more details.
        
        We take for granted the fact that for any associative ring $R$, left-$R$-modules form an abelian category (cf. \cite[Section VIII.3, p. 198]{maclane}) - denoted by ${}^lR\mod$ - with all limits and colimits. We will only be working with left-modules, but everything works identically for right-modules.

        \begin{definition}[Complexes and cohomology] \label{def: complexes_and_cohomology}
            (Cf. \cite[Section IV.1]{hilton_stammbach_homological_algebra}) Let $R$ be an associative ring. A \textbf{complex} or \textbf{chain complex} of left-$R$-modules is a diagram of left-$R$-modules:
                $$\{P_i, \del_i\}_{i \in \Z} := \{ ... \xrightarrow[]{\del_{-3}} P_{-2} \xrightarrow[]{\del_{-2}} P_{-1} \xrightarrow[]{\del_{-1}} P_0 \xrightarrow[]{\del_0} P_1 \xrightarrow[]{\del_1} P_2 \xrightarrow[]{\del_2} ... \}$$
            (sometimes also abbreviated by $(P_{\bullet}, \del_{\bullet})$) such that for each $i \in \Z$, one has that:
                $$\del_i \circ \del_{i - 1} = 0$$
            The connecting maps $\del_i$ are typically called \textbf{differentials} or \textbf{coboundary maps}.
            
            For each $i \in \Z$, the \textbf{$i^{th}$ cohomology} of such a complex is given by:
                $$H^i(P_{\bullet}, \del_{\bullet}) := \frac{ \ker \del_i }{ \im \del_{i - 1} }$$
            (typically this is written as just $H^i(P_{\bullet})$, with the differentials being implicitly understood). Elements of $\ker \del_i$ are usually called \textbf{$i$-cocycles} and those of $\im \del_{i - 1}$ are usually called \textbf{$i$-coboundaries}; elements of $H^i(P_{\bullet}, \del_{\bullet})$ are usually called \textbf{$i^{th}$ cohomology classes}, and cocycles which are representatives of the same cohomology class are said to be \textbf{cohomologous}.
            
            If:
                $$H^i(P_{\bullet}, \del_{\bullet}) \cong 0$$
            for all $i \in \Z$ then we shall say that the complex $(P_{\bullet}, \del_{\bullet})$ is \textbf{exact}.
        \end{definition}
        \begin{example}
            A \textbf{short exact sequence} (cf. e.g. definition \ref{def: lie_algebra_extensions}) is nothing but an exact complex with only 3 terms. 
        \end{example}
        
        If $R$ is any associative ring and $M$ is any left-$R$-module, then the functor:
            $$\Hom_R(-, M): {}^lR\mod^{\op} \to \Z\mod$$
        is generally only left-exact, which for us shall mean that it maps kernels and finite direct sums in ${}^lR\mod^{\op}$ (i.e. cokernels and finite direct sums in ${}^lR\mod$) to kernels and finite direct sums in $\Z\mod$. Consequently, any projective resolution\footnote{It is a fact from homological algebra that any choice of projective resolution would return the same cohomologies, so the non-trivial task is to choose a resolution that would make computations convenient or in fact, even feasible in the first place. In modern terminologies, all the projective resolutions of the same module are \say{quasi-isomorphic}.} of a given left-$R$-module $P$:
            $$(P_{\bullet}, \del_{\bullet}) := \{ ... \xrightarrow[]{\del_{-3}} P_{-2} \xrightarrow[]{\del_{-2}} P_{-1} \xrightarrow[]{\del_{-1}} P_0 \xrightarrow[]{\del_0} P \to 0 \}$$
        (i.e. a chain complex of left-$R$-modules wherein each term is projective\footnote{A left-$R$-module $P$ is projective if and only if the functor $\Hom_R(P, -): {}^lR\mod \to \Z\mod$ is exact, or equivalently (because ${}^lR\mod$ is an abelian category), if and only if it preserves monomorphisms.} as a left-$R$-module) is mapped by $\Hom_R(-, M)$ to a diagram of $\Z$-modules as follows:
            $$\Hom_R(P_{\bullet}, M) := \{ 0 \to \Hom_R(P, M) \Hom_R(P^0, M) \to \Hom_R(P^{-1}, M) \to \Hom_R(P^{-2}, M) \to ... \}$$
        \begin{definition}[$\Ext$-groups] \label{def: Ext_groups}
            With notations as above, we can define:
                $$\Ext^i_R(P, M) := H^i(\Hom_R(P_{\bullet}, M))$$
            for each $i \in \Z_{\geq 0}$. By construction, these are $\Z$-modules. 
        \end{definition}
        \begin{remark}
            If $R$ is an associative algebra defined over a field $k$, then the $\Z$-modules $\Ext^i_R(P, M)$ will actually carry $k$-vector space structures as well.
        \end{remark}
        \begin{definition}[Lie algebra cohomology and the Chevalley-Eilenberg resolution] \label{def: lie_algebra_cohomology}
            If $\a$ is a Lie algebra over $k$, then its \textbf{(abelian) Lie algebra cohomology} with \textbf{coefficients} in some left-$\rmU(\a)$-module $M$ (which is typically referred to simply as an $\a$-module) shall be given by:
                $$H^i_{\Lie}(\a, M) := \Ext^i_{\rmU(\a)}(k, M) := H^i( \Hom_{\rmU(\a)}(k_{\bullet}, M) )$$
            where $k$ is regarded as a trivial left-$\rmU(\a)$-module, equipped with some projective resolution of left-$\rmU(\a)$-modules $k_{\bullet}$.

            The standard projective resolution for the left-$\rmU(\a)$-module is the \textbf{Chevalley-Eilenberg resolution} (sometimes referred to as the \textbf{Chevalley-Eilenberg complex}):
                $$\left\{ k_{-i}(\a) := \rmU(\a) \tensor_k \bigwedge^i \a \right\}_{i \in \Z_{\geq 0}}$$
            and for each $i \in \Z_{\geq 0}$, the corresponding differential/coboundary map:
                $$\del_{-(i + 1)}: k_{-(i + 1)}(\a) \to k_{-i}(\a)$$
            is given on pure tensors by:
                $$
                    \begin{aligned}
                        & \del_{-(i + 1)}\left( a \tensor (x_1 \wedge ... \wedge x_{i + 1}) \right)
                        \\
                        := &
                            \sum_{1 \leq p \leq i + 1} (-1)^p a x_p \tensor (x_1 \wedge ... \wedge \not{x_p} \wedge ... \wedge x_{i + 1})
                            \\
                            & \qquad + \sum_{1 \leq p < q \leq i + 1} (-1)^{p + q} a \tensor ( [x_p, x_q]_{\a} \wedge x_1 \wedge ... \wedge \not{x_p} \wedge ... \wedge \not{x_q} \wedge ... \wedge x_{i + 1})
                    \end{aligned}
                $$
            for all $a \in \rmU(\a)$ and all $x_1, ..., x_{i + 1} \in \a$ (cf. \cite[Section VII.4]{hilton_stammbach_homological_algebra}).
        \end{definition}
        \begin{remark}[A simplified expression of Chevalley-Eilenberg complexes] \label{remark: simplified_chevalley_eilenberg_complexes}
            Let $\a$ be a Lie algebra over $k$ and let $M$ be an arbitrary $\a$-module; let us also regard $M$ as an abelian Lie algebra. Next, consider the following complex of $k$-vector spaces\footnote{Which is sometimes also called the Chevalley-Eilenberg complex of $\a$ (with coefficients in $M$).}:
                $$\{ C_i(\a, M) := \Hom_{\rmU(\a)}(k_{-i}(\a), M) \}_{i \in \Z_{\geq 0}}$$
            and observe that its terms $C_{-i}(\a, M)$ can be described as follows:
                $$
                    \begin{aligned}
                        & C_i(\a, M)
                        \\
                        := & \Hom_{\rmU(\a)}(k_{-i}(\a), M)
                        \\
                        \cong & \Hom_{\rmU(\a)}( \rmU(\a) \tensor_k \bigwedge^i \a, M ) \ni \left( a \tensor (x_1 \wedge ... \wedge x_i) \mapsto \varphi( a \tensor (x_1 \wedge ... \wedge x_i) ) \right)
                        \\
                        \cong & \Hom_{\rmU(\a)}( \rmU(\a), \Hom_k( \bigwedge^i \a, M ) ) \ni \left( a \mapsto \varphi(a \tensor -) \right)
                        \\
                        \cong & \Hom_k( \bigwedge^i \a, M ) \ni \varphi(1 \tensor -)
                    \end{aligned}
                $$
            for all $a \in \rmU(\a)$, all $x_1, ..., x_i \in \a$, and all $\varphi \in \Hom_{\rmU(\a)}( \rmU(\a) \tensor_k \bigwedge^i \a, M )$. This permits us to think of Lie cocycles and coboundaries as certain alternating multi-linear maps (see definition \ref{def: lie_cocycles_and_coboundaries} below).
        \end{remark}
        \begin{definition}[Lie cocycles and coboundaries] \label{def: lie_cocycles_and_coboundaries}
            Let us denote by:
                $$d_i^M: C_i(\a, M) \to C_{i + 1}(\a, M)$$
            the differentials of the complex of $k$-vector spaces $\{C_i(\a, M)\}_{i \in \Z_{\geq 0}}$. These are given by:
                $$d_i^M := \Hom_{\rmU(\a)}(\del_{-(i + 1)}, M)$$
            When $M$ is understood from the context, the superscripts may be omitted. For each $i \in \Z_{\geq 0}$, we define \textbf{Lie $i$-cocycles} and \textbf{Lie $i$-coboundaries} of $\a$ with values in $M$, respectively, to be elements of:
                $$Z^i_{\Lie}(\a, M) := \ker d_i^M$$
                $$B^i_{\Lie}(\a, M) := \im d_{i - 1}^M$$
        \end{definition}
        \begin{remark}
            From remark \ref{remark: simplified_chevalley_eilenberg_complexes}, one sees that - for any $i \in \Z_{\geq 0}$, a Lie $i$-cocycle of $\a$ with values in some $\a$-module $M$ is nothing but an alternating $k$-linear map:
                $$\sigma: \bigwedge^i \a \to M$$
            such that:
                $$d_i^M(\sigma) = 0$$
            In particular, using the fact that:
                $$d_i^M := \Hom_{\rmU(\a)}(\del_{-(i + 1)}, M)$$
            by construction, together with the construction of the maps $\del_{-(i + 1)}$ as in definition \ref{def: lie_algebra_cohomology}, one then sees that indeed, when $i = 2$, one recovers the notion of Lie $2$-cocycles of $\a$ with values in $M$ as in definition \ref{def: twisted_semi_direct_products} as linear maps $\sigma: \bigwedge^2 \a \to M$ satisfying the Jacobi identity in the sense stated in \textit{loc. cit.}
        \end{remark}

        Let us now see how Lie cocycles and Lie coboundaries are given explicitly, particularly in low degrees. 

        We begin with the case of cocycles and coboundaries with values in trivial modules. 
        \begin{example}[Lie cocycles and coboundaries with trivial coefficients] \label{example: lie_cocycles_and_coboundaries_with_trivial_coefficients}
            Let $\a$ be a Lie algebra over $k$ and regard $k$ itself as a trivial $\a$-module. For each $i \in \Z_{\geq 0}$, we then have - per the discussions in remark \ref{remark: simplified_chevalley_eilenberg_complexes} that:
                $$C_i(\a, k) \cong \Hom_k(\bigwedge^i \a, k) =: (\bigwedge^i \a)^*$$
            and the differentials:
                $$d_i^k: C_i(\a, k) \to C_{i + 1}(\a, k)$$
            are then nothing but:
                $$\del_{-(i + 1)}^* := \Hom_k(\del_{-(i + 1)}, k)$$
            \begin{itemize}
                \item A Lie $i$-cocycle of $\a$ with coefficients in $k$ is then - by definition - an element of $\ker \del_{-(i + 1)}^*$, i.e. a linear map:
                    $$\sigma: \bigwedge^i \a \to k$$
                such that:
                    $$
                        \begin{aligned}
                            & 0
                            \\
                            = & \del_{-(i + 1)}^*(\sigma)( (x_1 \wedge ... \wedge x_{i + 1}) )
                            \\
                            = &
                                \sum_{1 \leq p \leq i + 1} (-1)^p \sigma(x_1 \wedge ... \wedge \not{x_p} \wedge ... \wedge x_{i + 1})
                                \\
                                & \qquad + \sum_{1 \leq p < q \leq i + 1} (-1)^{p + q} \sigma( [x_p, x_q]_{\a} \wedge x_1 \wedge ... \wedge \not{x_p} \wedge ... \wedge \not{x_q} \wedge ... \wedge x_{i + 1})
                        \end{aligned}
                    $$
                for all $x_1, ..., x_i \in \a$.
                \item A Lie $i$-coboundary is an element of $\im \del_{-i}^*$, i.e. a linear map:
                    $$\beta: \bigwedge^i \a \to k$$
                for which there exists another linear map:
                    $$\tilde{\beta}: \bigwedge^{i - 1} \a \to k$$
                i.e. a $(i - 1)$-cocycle $\tilde{\beta}$, such that:
                    $$
                        \begin{aligned}
                            & \beta(x_1 \wedge ... \wedge x_i)
                            \\
                            = & \del_{-i}^*(\tilde{\beta})(x_1 \wedge ... \wedge x_i)
                            \\
                            = &
                                \sum_{1 \leq p \leq i} (-1)^p \sigma(x_1 \wedge ... \wedge \not{x_p} \wedge ... \wedge x_i)
                                \\
                                & \qquad + \sum_{1 \leq p < q \leq i} (-1)^{p + q} \sigma( [x_p, x_q]_{\a} \wedge x_1 \wedge ... \wedge \not{x_p} \wedge ... \wedge \not{x_q} \wedge ... \wedge x_i)
                        \end{aligned}
                    $$
            \end{itemize}
        \end{example}
        \begin{example}[Lie $2$-cocycles and $2$-coboundaries with trivial coefficients] \label{example: low_degree_lie_cocycles_and_coboundaries_with_trivial_coefficients}
            Of particular usefulness to us are Lie $2$-cocycles and $2$-coboundaries with trivial coefficients. When $i = 2$, the conditions in example \ref{example: lie_cocycles_and_coboundaries_with_trivial_coefficients} reduce to the following.
            \begin{itemize}
                \item A Lie $2$-cocycle of $\a$ with coefficients in the trivial $\a$-module $k$ is a linear map $\sigma: \bigwedge^2 \a \to k$ satisfying the Jacobi identity, in the sense of definition \ref{def: twisted_semi_direct_products}.
                \item A Lie $2$-coboundary of $\a$ with coefficients in $k$ is a linear map $\beta: \bigwedge^2 \a \to k$ for which there exist an element $\tilde{\beta} \in C_1(\a, k)$ (i.e. a linear map $\tilde{\beta}: \a \to k$) such that:
                    $$\beta(x \wedge y) = \tilde{\beta}([x, y])$$
                Equivalently, this is saying that:
                    $$\beta = d_1^k(\tilde{\beta})$$
                Much like Lie $2$-cocycles, the value of a Lie $2$-boundaries $\beta$ at some $x \wedge y \in \bigwedge^2 \a$ is usually denotes by $\beta(x, y)$, i.e. we tend to think of Lie $2$-coboundaries as certain skew-symmetric bilinear maps $\a \x \a \to k$.
            \end{itemize}
        \end{example}
        \begin{example}[Lie $2$-coboundaries with \textit{non-trivial} coefficients] \label{example: low_degree_lie_coboundaries_with_non_trivial_coefficients}
            Let $\d$ be a Lie algebra. Let $\Omega$ be an $\d$-module defined by a Lie algebra action
                $$\rho: \d \to \gl(\Omega)$$
            By construction, we have that:
                $$d_i^\Omega := \Hom_{\rmU(\d)}(\del_{-(i + 1)}, \Omega)$$
            for all $i \in \Z_{\geq 0}$. By using remark \ref{remark: simplified_chevalley_eilenberg_complexes}, we shall see that a Lie $2$-coboundary of $\d$ with values in $\Omega$ shall be a linear map\footnote{Again, let us regard maps out of exterior powers as alternating multi-linear maps.} $\beta: \bigwedge^2 \d \to \Omega$ for which there is an element $\tilde{\beta} \in C_1(\d, \Omega)$ (i.e. a linear map $\tilde{\beta}: \d \to \Omega$) satisfying the following property:
                $$\beta(x, y) = \left( \rho(x) \cdot \tilde{\beta}(y) - \rho(y) \cdot \tilde{\beta}(x) \right) - \tilde{\beta}([x, y])$$
            (cf. \cite[Equation 1.6, p. 421]{kassel_quantum_groups}). In other words, $2$-coboundaries are error terms measuring how far a linear map $\d \to \Omega$ is from being a derivation of $\d$ with values in $\Omega$.
        \end{example}

        Finally, let us investigate a particular property enjoyed by graded Lie $2$-coboundaries, namely that they admit \say{lifts} along the coboundary map $d_1$ which are \textit{graded} linear maps. This is a technical preparation in service of the proof of proposition \ref{prop: sigma_1_is_not_coboundary}.
        \begin{remark}[Graded vector spaces and graded linear maps] \label{remark: grading_projections}
            For what follows, recall that if $V := \bigoplus_{n \in Z} V_n, W := \bigoplus_{n \in Z} W_n$ are vector spaces graded by some abelian group $Z$, and:
                $$\phi: V \to W$$
            is any linear map, then one can always find a \textit{graded} linear map:
                $$\bar{\phi}: V \to W$$
            given by:
                $$\bar{\phi} := \bigoplus_{n \in Z} \phi_n$$
            where $\phi_n: V_n \to W_n$ is the linear map - uniquely defined using $\phi$ for every $n \in Z$ - fitting into the following commutative diagram of vector spaces and linear maps, where the left vertical arrow is the canonical inclusion and the right vertical arrow is the canonical projection:
                $$
                    \begin{tikzcd}
                	V & W \\
                	{V_n} & {W_n}
                	\arrow["{\pr_n}", from=1-2, to=2-2]
                	\arrow["{\iota_n}", from=2-1, to=1-1]
                	\arrow["\phi_n", dashed, from=2-1, to=2-2]
                	\arrow["\phi", from=1-1, to=1-2]
                    \end{tikzcd}
                $$
            Clearly, if $\phi$ was graded to begin with, then $\bar{\phi} = \phi$.

            We caution the reader that even though $\bar{\phi}$ is given as a potentially infinite direct sum of maps, any evaluation $\bar{\phi}(v)$ at any vector $v \in V$ is actually just a finite sum of elements of $V$. This is because elements of $V := \bigoplus_{n \in Z} V_n$ are of the form $(v_n)_{n \in Z}$, with $v_n = 0$ for all but finitely many $n \in Z$.
        \end{remark}
        \begin{lemma}[Graded Lie $2$-coboundaries] \label{lemma: graded_2_coboundaries}
            Let $Z$ be an abelian group. Let $\d := \bigoplus_{n \in Z} \d_n$ be a $Z$-graded Lie algebra and $\Omega := \bigoplus_{n \in Z} \Omega_n$ be a $\d$-module, that is $Z$-graded as a vector space. Suppose also that the $\d$-action on $\Omega$ is homogeneous, i.e.:
                $$D \cdot \Omega_n \subseteq \Omega_{n + \deg D}$$
            for all $D \in \d$. 

            Next, let $\tau \in C_1(\d, \Omega)$. Then, $d_1^{\Omega}(\tau) \in C_2(\d, \Omega)$ will be a graded linear map from $\bigwedge^2 \d$ to $\Omega$ if and only if there exists some \textit{graded} linear map $\bar{\tau} \in C_1(\d, \Omega)$ such that:
                $$d_1^{\Omega}(\bar{\tau}) = d_1^{\Omega}(\tau)$$
        \end{lemma}
            \begin{proof}
                To prove the \say{if} direction, firstly take $\bar{\tau} = \tau$. Since $\tau$ is graded, the assertion is then clear from the equation defining Lie $2$-coboundaries, which is:
                    \begin{equation} \label{equation: graded_2_coboundaries}
                        d_1^{\Omega}(\tau)(D, D') = D \cdot \tau(D') - D' \cdot \tau(D) - \tau([D, D'])
                    \end{equation}
                given for all $D, D' \in \d$ (cf. example \ref{example: low_degree_lie_coboundaries_with_non_trivial_coefficients}).
            
                Let us now prove the \say{only if} direction. We claim that the sought-for $\bar{\tau}$ is nothing but:
                    $$\bar{\tau} := \bigoplus_{n \in Z} \tau_n$$
                where $\tau_n := \pr_n \circ \tau \circ \iota_n$ for each $n \in Z$ as in remark \ref{remark: grading_projections}. To prove this claim, let us firstly pick homogeneous elements $D, D' \in \d$ and for convenience, let us set:
                    $$m := \deg D, n := \deg D'$$
                Also, for each $d \in Z$, let us set:
                    $$d_1^{\Omega}(\tau)_d := \pr_d \circ d_1^{\Omega}(\tau) \circ \iota_d$$
                    $$\overline{d_1^{\Omega}(\tau)} := \bigoplus_{(m, n) \in Z^{\oplus 2} } d_1^{\Omega}(\tau)_{m + n}$$
                with notations as in remark \ref{remark: grading_projections}. Next, let us apply $\pr_{m + n}$ to both sides of equation \eqref{equation: graded_2_coboundaries}, which yields:
                    $$
                        \begin{aligned}
                            & d_1^{\Omega}(\tau)_{m + n}(D, D')
                            \\
                            = & \pr_{m + n}( d_1^{\Omega}(\tau)(D, D') )
                            \\
                            = & \pr_{m + n}\left( D \cdot \tau(D') - D' \cdot \tau(D) - \tau([D, D']) \right)
                            \\
                            = & \pr_{m + n}(D \cdot \tau(D')) - \pr_{m + n}(D' \cdot \tau(D)) - \pr_{m + n}(\tau([D, D']))
                            \\
                            = & D \cdot \pr_m( \tau(D') ) - D' \cdot \pr_n( \tau(D) ) - \pr_{m + n}(\tau([D, D']))
                            \\
                            = & D \cdot \tau_m(D') - D' \cdot \tau_n(D) - \tau_{m + n}([D, D'])
                        \end{aligned}
                    $$
                and note that the fourth and fifth equality result from the fact that $\d$ acts homogeneously on $\Omega$, as well as the fact that $\deg [D, D'] = m + n$, coming from our assumption that $\d$ is a $Z$-graded Lie algebra. From the above, we get that:
                    $$
                        \begin{aligned}
                            & \overline{d_1^{\Omega}(\tau)}(D, D')
                            \\
                            = & \bigoplus_{(m, n) \in Z^{\oplus 2} } d_1^{\Omega}(\tau)_{m + n}(D, D')
                            \\
                            = & \bigoplus_{(m, n) \in Z^{\oplus 2} } \left( D \cdot \tau_m(D') - D' \cdot \tau_n(D) - \tau_{m + n}([D, D']) \right)
                            \\
                            = & \bigoplus_{m \in Z} D \cdot \tau_m(D') - \bigoplus_{n \in Z} D' \cdot \tau_n(D) - \bigoplus_{(m, n) \in Z^{\oplus 2} } \tau_{m + n}([D, D'])
                            \\
                            = & D \cdot \bigoplus_{m \in Z} \tau_m(D') - D' \cdot \bigoplus_{n \in Z} \tau_n(D) - \bigoplus_{(m, n) \in Z^{\oplus 2} } \tau_{m + n}([D, D'])
                            \\
                            = & D \cdot \bar{\tau}(D') - D' \cdot \bar{\tau}(D) - \bar{\tau}([D, D'])
                        \end{aligned}
                    $$
                in which, to go from the third equality to the fourth one, we have made use of the assumption that, because $D, D' \in \d$ are homogeneous elements and because the $\d$-action on $\Omega$ is homogeneous, we have as a result that:
                    $$\bigoplus_{m \in Z} D \cdot \tau_m(D') = D \cdot \sum_{m \in Z} \tau_m(D') = D \cdot \bigoplus_{m \in Z} \tau_m(D')$$
                    $$\bigoplus_{n \in Z} D' \cdot \tau_n(D) = D' \cdot \sum_{n \in Z} \tau_n(D) = D' \cdot \bigoplus_{n \in Z} \tau_m(D)$$
                Now, since $d_1^{\Omega}(\tau) \in C_2(\d, \Omega)$ has been assumed to be graded to begin with, we have that:
                    $$\overline{d_1^{\Omega}(\tau)} = d_1^{\Omega}(\tau)$$
                and hence:
                    $$d_1^{\Omega}(\tau)(D, D') = D \cdot \bar{\tau}(D') - D' \cdot \bar{\tau}(D) - \bar{\tau}([D, D'])$$
                for every pair of homogeneous elements $D, D' \in \d$. 
            \end{proof}

    \subsection{Interpretations of \texorpdfstring{$H^1_{\Lie}$}{} and \texorpdfstring{$H^2_{\Lie}$}{}}
        We conclude this subsection by discussing interpretations of the two cases of low-dimensional Lie algebra cohomology that are of interest to us, namely $H^1_{\Lie}$ and $H^2_{\Lie}$. Respectively, these parametrise so-called \say{outer derivations} and abelian extensions. We will be needing these notions in the discussions leading up to proposition \ref{prop: cohomological_non_triviality_of_billig_toroidal_cocycles}, in order to be able to discern whether or not certain unequal $2$-cocycles might be cohomologous to one another. More specifically, we will be making use of the fact that isomorphism classes of abelian extensions of a given Lie algebra are in bijection with its $2^{nd}$ cohomology (with suitable coefficients, of course).
        
        \begin{definition}[Lie derivations] \label{def: lie_derivations}
            (Cf. \cite[Section VII.2, Equation 2.2, p. 234]{hilton_stammbach_homological_algebra}) Let $\d$ be a Lie algebra and let $\Omega$ be an $\d$-module defined by a Lie algebra action:
                $$\rho: \d \to \gl(\Omega)$$
            A \textbf{derivation} of $\d$ with values in $\Omega$ is a linear map:
                $$L: \d \to \Omega$$
            satisfying the following property for all $X, Y \in \d$:
                $$L( [X, Y] ) = \rho(X) \cdot L(Y) - \rho(Y) \cdot L(X)$$
            Such derivations form a vector space, for which we shall write $\der(\d, \Omega)$. For every $\omega \in \Omega$, one can define an \textbf{inner derivation} $L_{\omega}$, specified by:
                $$L_{\omega} = \rho(-) \cdot \omega$$
            Inner derivations form a vector subspace of $\der(\d, \Omega)$, which is denoted by $\inn(\d, \Omega)$. Derivations that are not inner are said to be \textbf{outer}, and the vector space of outer derivations is identified as:
                $$\out(\d, \Omega) := \der(\d, \Omega)/\inn(\d, \Omega)$$
        \end{definition}
        \begin{example}[The adjoint action]
            Let $\d$ be a Lie algebra and let $\d$ also be considered as a module over itself via the adjoint action. Then, for any $X \in \d$, the map:
                $$\ad(X): \d \to \d$$
            will be an inner derivation of $\d$ with values in itself.
        \end{example}
        \begin{example}[Derivations on current algebras] \label{example: derivations_on_current_algebras}
            Let $\a$ be a Lie algebra over $k$ and $A$ be a commutative $k$-algebra. On the current algebra $\a \tensor_k A$ (cf. definition \ref{def: current_algebras}), one can construct a natural action of $\der(A)$ on $\a \tensor_k A$ by:
                $$D \cdot xf := x D(f)$$
            for all $D \in \der(A)$ and all $x \in \a, f \in A$. This action is by Lie derivations, since the following holds for all $D \in \der(A)$ and all $x, y \in \g, f, g \in A$, merely because $D$ is a derivation on $A$:
                $$D \cdot [xf, yg]_{\a \tensor_k A} = D \cdot [x, y] fg = [x, y] ( D(f)g + f D(g) )$$
        \end{example}
        \begin{example}[Lie derivatives] \label{example: lie_derivatives}
            Another prominent example of Lie derivations is the operation of Lie derivatives. For us, knowing that this operation is well-defined will be useful for proving of lemma \ref{lemma: vector_field_action_on_toroidal_lie_algebras}. 

            To construct these derivations, consider firstly a commutative $k$-algebra $A$ and the standard action:
                $$L: \der(A) \to \gl(A)$$
            of $\der(A)$ on $A$, via evaluations of derivations $D \in \der(A)$ on \say{functions} $f \in A$, i.e.:
                $$L(D)(f) := D(f)$$
            It is clear that $L$ as above is a Lie algebra homomorphism, so the action is well-defined. Our goal is to somehow induce an action of $\der(A)$ on $\Omega^1_{A/k}$ using $L$ as above; the claim is that the sought-for action shall be given by so-called \textbf{Lie derivatives}, i.e.:
                $$D \cdot g df := D(g) df + g d(D(f))$$
            for all $D \in \der(A)$ and all $f, g \in A$.

            Now, recall that if $U$ is any bialgebra over $k$ (cf. \cite[Section III.2]{kassel_quantum_groups}) with comultiplication $\Delta: U \to U^{\tensor 2}$ and $V, W$ are $U$-modules determined by algebra homomorphisms $\pi: U \to \End_k(V)$ and $\rho: U \to \End_k(W)$, then $V \tensor_k W$ will be a $U^{\tensor 2}$-module \textit{a priori}, determined by an algebra homomorphism:
                $$\pi \tensor \rho: U^{\tensor 2} \to \End_k(V \tensor_k W)$$
            given by:
                $$(\pi \tensor \rho)(u \tensor u')(v \tensor w) := uv \tensor u' w$$
            for all $u, u' \in U$ and $v \in V, w \in W$. Then, using the fact that $\Delta$ is an algebra homomorphism per the definition of bialgebras, one can form the composite algebra homomorphism:
                $$U \xrightarrow[]{\Delta} U^{\tensor 2} \xrightarrow[]{\pi \tensor \rho} \End_k(V \tensor W)$$
            and thus can endow $V \tensor_k W$ with the structure of a $U$-module, given by:
                $$u \cdot (v \tensor w) := (\pi \tensor \rho)(\Delta(u))( v \tensor w )$$
            for all $u \in U$ and all $v \in V, w \in W$.
            
            Because the universal enveloping algebra of $\der(A)$ (or for that matter, any Lie algebra) is a Hopf algebra - and hence a bialgebra - with comultiplication given by:
                $$\Delta(X) := X \tensor 1 + 1 \tensor X$$
            for all $X \in \rmU(\der(A))$ (cf. \cite[Section III.3]{kassel_quantum_groups}), the abstract nonsense above about tensor products of bialgebra modules can be applied to the case:
                $$U := \rmU(\der(A))$$
            The aforementioned action of $\der(A)$ on $A$ thus automatically extends to $A^{\tensor 2}$ by:
                $$
                    \begin{aligned}
                        & D \cdot ( f \tensor g )
                        \\
                        := & L^{\tensor 2}( \Delta(D) )( f \tensor g )
                        \\
                        = & ( L(D) \tensor 1 + 1 \tensor L(D) )(f \tensor g)
                        \\
                        = & D(f) \tensor g + f \tensor D(g)
                    \end{aligned}
                $$
            Finally, recall from definition \ref{def: kahler_differentials} that as an $A$-module, $\Omega^1_{A/k}$ is the quotient of the $A$-module $A^{\tensor 2}$ by the $A$-submodule $I$ generated by elements of the form $fg \tensor 1 - f \tensor g - g \tensor f$, for all $f, g \in A$. We have that:
                $$
                    \begin{aligned}
                        & D \cdot ( fg \tensor 1 - f \tensor g - g \tensor f )
                        \\
                        = & D(fg) \tensor 1 - ( D(f) \tensor g + f \tensor D(g) ) - ( D(g) \tensor f + g \tensor D(f) )
                        \\
                        = & (D(f) g + f D(g)) \tensor 1 - ( D(f) \tensor g + f \tensor D(g) ) - ( D(g) \tensor f + g \tensor D(f) )
                        \\
                        = & ( D(f) g \tensor 1 - D(f) \tensor g - g \tensor D(f) ) + ( D(g) f \tensor 1 - D(g) \tensor f - f \tensor D(g) ) 
                        \in I
                    \end{aligned}
                $$
            which shows that $I$ is a $\der(A)$-submodule of $A^{\tensor 2}$. From this, we see that there is a $\der(A)$-action on $\Omega^1_{A/k}$ given by:
                $$D \cdot g df := D(g) df + g d(D(f))$$
            as claimed.

            It is also clear that Lie derivatives satsify the Leibniz rule, so they are indeed Lie derivations.
        \end{example}
        \begin{theorem}[$H^1_{\Lie}$ and derivations of Lie algebras]
            \cite[Theorem 2.1 and Proposition 2.2]{hilton_stammbach_homological_algebra} Let $\d$ be a Lie algebra and $\Omega$ be an $\d$-module. Then:
                $$Z^1_{\Lie}(\d, \Omega) \cong \der(\d, \Omega)$$
                $$B^1_{\Lie}(\d, \Omega) \cong \inn(\d, \Omega)$$
            and hence:
                $$H^1_{\Lie}(\d, \Omega) \cong \out(\d, \Omega)$$
        \end{theorem}

        \begin{theorem}[$H^2_{\Lie}$ and abelian extensions] \label{theorem: H^2_of_lie_algebras_and_abelian_extensions}
            (Cf. \cite[Theorem VII.3.3]{hilton_stammbach_homological_algebra}) Let $\d$ be a Lie algebra over $k$ and $\Omega$ be an $\d$-module, equipped with the abelian Lie algebra structure. There is then a bijection:
                $$H^2_{\Lie}(\d, \Omega) \xrightarrow[]{\cong} \{ \text{isomorphism classes of extensions of $\d$ by $\Omega$} \}$$
                $$\sigma \mapsto \Omega \rtimes^{\sigma} \d$$
        \end{theorem}
        
        \begin{remark}[Non-abelian Lie algebra cohomology ?]
            There is also a variant of the construction given in definition \ref{def: lie_algebra_cohomology}, called \textbf{non-abelian Lie algebra cohomology}, which ostensibly is for the purpose of classifying Lie algebra extensions:
                $$0 \to \t \to \frake \to \d \to 0$$
            where the kernel $\t$ is not necessarily abelian. This is harder to define and in fact, is unnecessary for our purposes, so we will make no further mention of it.
        \end{remark}