\section{Introduction}
    \subsection{Notations and conventions}
        All throughout, we work over the field of complex numbers $\bbC$, although any algebraically closed field of characteristic $0$ will suffice, as we only need this assumption so that certain operators will be diagonalisable. 

        We fix once and for all a finite-dimensional simple Lie algebra $\g$. Its Cartan matrix will be denoted by $C$, and we will use standard notations (cf. e.g. \cite{humphreys_lie_algebras} and \cite{kac_infinite_dimensional_lie_algebras}) for all of the other data that usually accompanies $\g$, e.g. a choice of simple roots, a subsequently defined root system, etc. We note that to specify these data, we will have to choose a non-degenerate and invariant symmetric bilinear form on $\g$, which shall be denoted by $\kappa$.

        The \textbf{affine Lie algebra} attached to the Lie $2$-cocycle $\kappa \in H^2_{\Lie}(\g[v^{\pm 1}], \bbC)$ shall be:
            $$\tilde{\g}_{\kappa} := \uce(\g[v^{\pm 1}])$$
        on which the Lie bracket is given by:
            $$[xf, yg]_{\tilde{\g}_{\kappa}} := [x, y]_{\g} fg + \kappa(x, y) g df$$
        Here, we have used a result by Kassel to identify:
            $$\tilde{\g}_{\kappa} \cong \g[v^{\pm 1}] \oplus \Omega^1_{\bbC[v^{\pm 1}]/\bbC}/d\bbC[v^{\pm 1}] \cong \g[v^{\pm 1}] \oplus \bbC c_{\kappa}$$
        (see \cite{kassel_universal_central_extensions_of_lie_algebras}). We also know, per a result of Garland (see \cite{garland_arithmetics_of_loop_groups}), that $\tilde{\g}_{\kappa}$ is - up to isomorphisms - the only central extension of $\g[v^{\pm 1}]$ (i.e. $H^2_{\Lie}(\g[v^{\pm 1}], \bbC)$); isomorphisms are given by rescaling $\kappa$. One also says that $\tilde{\g}_{\kappa}$ is the affine Lie algebra at \textbf{level} $\kappa$.

        The \textbf{untwisted affine Kac-Moody algebra} at level $\kappa$ (as in \cite[Chapter 7]{kac_infinite_dimensional_lie_algebras}) will then be:
            $$\hat{\g}_{\kappa} := \tilde{\g}_{\kappa} \rtimes \bbC D_{\kappa}$$
        where $D_{\kappa} \in \der(\tilde{\g}_{\kappa})$ can be shown to be equal to $v\frac{d}{dv}$. This algebra will be endowed with the non-degenerate and invariant symmetric bilinear form given by:
            $$(xf, yg)_{\hat{\g}_{\kappa}} := \kappa(x, y) \Res(g df)$$
            $$(c_{\kappa}, D_{\kappa})_{\hat{\g}_{\kappa}} := 1$$
        Using this bilinear form, one can construct the Cartan matrix $\hat{C}$ of $\hat{\g}_{\kappa}$ and related combinatorial data, like the affinisation of the root system of $\g$ and so on.
            
        Recall in particular that for $\hat{\g}$ (or indeed, for any Kac-Moody algebra) there is the \textbf{principal $\Z$-grading} of type $\vec{s} := (s_i)_{i \in \hat{\simpleroots}} \in \Z_{\geq 0} \cdot \hat{\simpleroots}$, given on the Chevalley-Serre generators by:
            $$\deg x_i^{\pm} := \pm s_i$$
            $$\deg h_i := 0$$
        for all $i \in \hat{\simpleroots}$. When $\vec{s} := \vec{1} := (1, ..., 1)$, we recover the usual root height grading. 

    \subsection{Vertex algebras}

    \subsection{Why vertex operators ?}
        From a representation-theoretic point of view, one possible motivation for the use of vertex algebras comes from simply trying to write down some kind of \say{affine Casimir element}. There are many reasons to care about such elements, though one prominent reason is that they generate the centre of some completion of $\rmU(\hat{\g}_{\kappa})$ (what we will denote by $\tilde{\rmU}_{\kappa}$ down below), and hence are useful understanding affine central characters. 
        
        Firstly, recall that for the affine Kac-Moody algebra $\hat{\g}_{\kappa}$, one shall obtain a root space decomposition of the following form, after choosing a Cartan subalgebra $\hat{\h}_{\kappa}$:
            $$\hat{\g}_{\kappa} \cong \hat{\h}_{\kappa} \oplus \bigoplus_{\alpha \in \hat{\Phi}} \hat{\g}_{\kappa, \alpha}$$
        in which:
        \begin{itemize}
            \item the real roots - of which there are finitely many - are all equally of multiplicity $1$, while
            \item the imaginary roots - of which there are infinitely many, as their root spaces are identified with $\h \tensor v^n$ for all $n \in \Z \setminus \{0\}$ - do not necessarily have multiplicity $1$ (indeed, when $\g \not \cong \sl_2$, $\dim \h > 1$).
        \end{itemize}
        Because of this, if we were to insist on writing down the \textit{na\"ive} guess for an affine (quadratic) Casimir element, i.e. something of the form:
            $$\hat{\sfr}_{\text{na\"ive}} := \sum_{(i, n) \in \simpleroots \x \Z} h_i v^n \tensor h^i v^{-n} + \sum_{\alpha \in \Re \hat{\Phi}^+} (x_{\alpha}^+ \tensor x_{\alpha}^- + x_{\alpha}^- \tensor x_{\alpha}^+)$$
        then we would have a hard time trying to make sense of the summand:
            $$\hat{\sfr}_{\text{na\"ive}}^{\Im} := \sum_{(i, n) \in \simpleroots \x (\Z \setminus \{0\})} h_i v^n \tensor h^i v^{-n}$$
        which is a sum over basis-dual basis vectors of the direct sum $\bigoplus_{n \in \Z \setminus \{0\}} \h \tensor v^n$ of all the imaginary root spaces. Worse still, this is a \say{bad} infinite sum, in the sense that the infinite in both the positive and negative directions (recall that formal Laurent series contain infinitely many terms only in either direction, not both). In other words, the expression $\hat{\sfr}_{\text{na\"ive}}^{\Im}$ is properly an element of $\hat{\g}_{\kappa}[\![v^{\pm 1}]\!]$, and this very problematic because $\bbC[\![v^{\pm 1}]\!]$ - the vector space spanned by so-called \say{formal distributions} $\sum_{n \in \Z} a_n v^n$ where, crucially, there can be infinitely many $a_n \not = 0$ - is \textit{not} an associative $\bbC$-algebra at all, so we can not regard $\hat{\g}_{\kappa}[\![v^{\pm 1}]\!]$ as a current algebra!

        If, instead, we were to regard the elements $h_i, h^i$ as operators on some $\hat{\g}_{\kappa}$-module, say $V$, then we can consider:
            $$\hat{\sfr}_{V, \text{na\"ive}}^{\Im}(z) := \sum_{(i, n) \in \simpleroots \x (\Z \setminus \{0\})} h_i z^n \tensor h^i z^{-n} \in \End(V)[\![z^{\pm 1}]\!]$$
        Even though this is still a formal distribution (this time with coefficients in the associative algebra $\End(V)$), it is known that $\End(V)[\![z^{\pm 1}]\!]$ can be used for constructing a vertex algebra structure on the underlying vector space of $V$, and hence expressions like $\hat{\sfr}_V^{\Im}(z)$ can be dealt with as vertex operators.

    \subsection{Generating series}
        To have a better sense of how to proceed from here, let us firstly introduce the physicist's perspective on infinite-dimensional Lie algebras with loop realisations, namely the apprroach via \textbf{generating series}. For untwisted affine Kac-Moody algebras, this ultimately originates from the fact that the extra \say{affinising} Chevalley-Serre generators:
            $$x_0^{\pm} := x_{\theta}^{\pm} v^{\mp 1}, h_0 := [x_{\theta}^+, x_{\theta}^-]$$
        are given in terms of the finite-type generators (elements of $\{x_i^{\pm}, h_i\}_{i \in \simpleroots}$); namely, the highest/lowest root vectors $x_{\theta}^{\pm}$ can be written as ordered monomials in $x_i^{\pm}$. The physical perspective is then, that $x_0^{\pm}, h_0$ ought be obtained as certain \say{residues} of formal distributions:
            $$J(z) := \sum_{n \in \Z} J^{(n)} z^{-n - 1}, J^{(n)} := J v^n$$
        associated to elements $J \in \g$ (and why we have decided to write $z^{-n - 1}$ instead of $z^n$ will become clear shortly); these formal distributions are usually called \textbf{associated fields} or \textbf{generating series} when $J$ is a Chevalley-Serre generator of $\g$.

        Now, to see how commutator relations are given, let us firstly fix a basis $\{J_{\alpha}\}_{\alpha = 1}^{\dim \g}$ for $\g$ and then consider the following:
            $$[J_{\alpha}^{(m)}, J_{\beta}^{(n)}] = \sum_{r = 1}^{\dim \g} C_{\alpha, \beta}^{\gamma} J_{\gamma}^{(m + n)} + m \delta_{m + n, 0} \kappa(J_{\alpha}, J_{\beta}) c_{\kappa}$$
        wherein $C_{\alpha, \beta}^{\gamma} \in \bbC$ are the structural constants. Next, let us recall that the formal Dirac distribution in a formal variable $w$ centered at $z$ is the element of $\bbC[\![w^{\pm 1}, z^{\pm 1}]\!]$ that is given by:
            $$\1(w - z) := \sum_{n \in \Z} w^n z^{-n - 1}$$
        and crucially, this formal distribution satisfies the following identity for all $J(w) \in \g[\![w^{\pm 1}]\!]$:
            $$f(z) = \Res_{w = 0} f(w) \1(w - z)$$
        We can then perform the following computations, which allows us to pass to associated fields:
            $$
                \begin{aligned}
                    & [J_{\alpha}(w), J_{\beta}(z)]
                    \\
                    := & \sum_{m \in \Z} \sum_{n \in \Z} [J_{\alpha}^{(m)} w^{-m - 1}, J_{\beta}^{(n)} z^{-n - 1}]
                    \\
                    = & \sum_{m \in \Z} \sum_{n \in \Z} \sum_{\gamma = 1}^{\dim \g} C_{\alpha, \beta}^{\gamma} J_{\gamma}^{(m + n)} w^{-m - 1} z^{-n - 1} + \kappa(J_{\alpha}, J_{\beta}) c_{\kappa} \sum_{m \in \Z} \sum_{n \in \Z} m \delta_{m + n, 0} w^{-m - 1} z^{-n - 1}
                    \\
                    = & \sum_{m \in \Z} \sum_{n \in \Z} \sum_{\gamma = 1}^{\dim \g} C_{\alpha, \beta}^{\gamma} J_{\gamma}^{(m + n)} w^{-m - n - 1} w^n z^{-n - 1} + \kappa(J_{\alpha}, J_{\beta}) c_{\kappa} \sum_{m \in \Z} m w^{-m - 1} z^{m - 1}
                    \\
                    = & \sum_{\gamma = 1}^{\dim \g} C_{\alpha, \beta}^{\gamma} J_{\gamma}(w) \1(w - z) + \kappa(J_{\alpha}, J_{\beta}) c_{\kappa} \del_z \sum_{m \in \Z} w^{-m - 1} z^m
                    \\
                    = & [J_{\alpha}, J_{\beta}](w) \1(w - z) + \kappa(J_{\alpha}, J_{\beta}) c_{\kappa} \del_z \1(z - w)
                \end{aligned}
            $$
        where $\del_z \in \End( \bbC[\![z^{\pm 1}]\!] )$ is the formal partial derivative, given by $\del_z \sum_{m \in \Z} a_m z^m := \sum_{m \in \Z} m a_m z^{m - 1}$. By applying $\Res_{w = 0}$ to both sides, we then get:
            $$
                \begin{aligned}
                    & \Res_{w = 0} [J_{\alpha}(w), J_{\beta}(z)]
                    \\
                    = & \Res_{w = 0} [J_{\alpha}, J_{\beta}](w) \1(w - z) + \kappa(J_{\alpha}, J_{\beta}) c_{\kappa} \Res_{w = 0} \del_z \1(z - w)
                    \\
                    = & [J_{\alpha}, J_{\beta}](z) + \kappa(J_{\alpha}, J_{\beta}) c_{\kappa} 
                \end{aligned}
            $$
        One can then choose the basis vectors $J_{\alpha}$ to be orthonormal to one another, and then evaluate at $z = 0$ to get $c_{\kappa}$. 

        Now, observe in particular, that for all $x, y \in \g$, the commutators of fields:
            $$[x(w), y(z)] = [x, y](w) \1(w, z) + \kappa(x, y) \del_w \1(w - z)$$
        are all \textbf{mutually local} as the formal Dirac distribution satisfies:
            $$\forall N \in \Z_{\geq 0}: N > n \implies (w - z)^N \del_w^n \1(w - z) = 0$$
        This suggests to us that vertex algebra structures on $\tilde{\g}_{\kappa}$-modules $V$ - say, specified by a Lie algebra homomorphism $\rho: \tilde{\g}_{\kappa} \to \End(V)$ - can be specified by linear maps:
            $$\rho_z: \g[\![z^{\pm 1}]\!] \to \End(V)[\![z^{\pm 1}]\!]$$
        preserving field commutators, as the mutual locality between fields that is required to happen on the $\End(V)[\![z^{\pm 1}]\!]$ side as a part of the vertex algebra definition is already satisfied on the $\tilde{\g}_{\kappa}[\![z^{\pm 1}]\!]$ side. There is also the so-called \textbf{Vacuum Axiom}, which for now can be simply understood as requiring that $V$ is a highest-weight module on which the central charge $c_{\kappa}$ acts as $1$ (this is more-or-less a normalising condition); we will be offering a physical explanation later on, when discussing the so-called \say{Heisenberg vacuum module}. Finally, the \textbf{Translation Axiom}, which requires that there is a formal derivation $T$ acting on fields:
            $$J(z) := \sum_{m \in \Z} J^{(m)} z^{-m - 1} \in \End(V)[\![z^{\pm 1}]\!]$$
        via commutators, i.e.:
            $$[T, J(z)] := [T, J^{(m)}](z)$$
        where on the RHS, we regard $T$ as a derivation on $\g[v^{\pm 1}]$ or on $\g(\!(v)\!)$, depending on the situation at hand; typically, one takes $T := \id_{\g} \tensor D$ for some derivation $D$ on $\bbC[v^{\pm 1}]$ or $\bbC(\!(v)\!)$. In combination, these requirements suggest to us that a good candidates for a $\tilde{\g}_{\kappa}$-module that can support vertex algebra structures is the induced module:
            $$\V_{\kappa} := \tilde{\rmU}_{\kappa} \tensor_{\tilde{\rmU}_{\kappa}^+} \bbC c_{\kappa}$$
        Down below, we shall see that, indeed, this module does carry a unique vertex algebra structure.