\section{Modules over VOAs and examples}
    \subsection{Modules over VOAs}

    \subsection{Lattice Heisenberg VOAs}
        \begin{definition}[Lattices] \label{def: lattices}
            A \textbf{lattice} shall be a finite free $\Z$-module $L$ equipped with a symmetric $\Z$-bilinear form:
                $$\<-, -\>: L \tensor_{\Z} L \to \Z$$
            that is \textit{positive-definite}, i.e. for all $\lambda \in L \setminus \{0\}$, one has that $\<\lambda, \lambda\> > 0$.
        \end{definition}
        \begin{example}
            The root lattice $Q$ of the finite-dimensional simple Lie algebra $\g$, on which the bilinear form is induced by the level $\kappa$, satisfies definition \ref{def: lattices}. In this case, recall also that the Cartan subalgebra $\h$ is isomorphic to the complexification of $Q$, i.e.:
                $$\h \cong Q \tensor_{\Z} \bbC$$
            because $\h$ admits the set of simple roots as a basis.
        \end{example}

        \begin{definition}[Heisenberg algebras associated to lattices] \label{def: lattice_heisenberg_algebras} 
            Given a lattice $(L, \<-, -\>)$, set $\h_L := L \tensor_{\Z} \bbC$, and let us regard this as an abelian Lie algebra. The \textbf{Heisenberg algebra} associated to said lattice, denoted by $\tilde{\h}_L$, is then the central extension:
                $$0 \to \bbC c_L \to \tilde{\h}_L \to \h_L[v^{\pm}] \to 0$$
            whose Lie bracket is given by:
                $$[ h f, h' g ]_{\tilde{\h}_L} := \<h, h'\> \Res_{v = 0} g df \cdot c_L$$
            for all $h, h' \in \h_L$ and all $f, g \in \bbC[v^{\pm 1}]$; equivalently, the Lie bracket is given by:
                $$[h v^m, h' v^n]_{\tilde{\h}_L} := \<h, h'\> \delta_{m + n, 0} n \cdot c_L$$
        \end{definition}
        \begin{definition}[Weyl algebras associated to lattices] \label{def: lattice_weyl_algebras}
            Let $L$ be a lattice. Next, let us write $\bar{\rmH}_L := \rmU(\tilde{\h}_L)/\<c_L - 1\>$. Then, we can define the \textbf{Weyl algebra} associated to the given lattice $L$ - which shall be denoted by $\tilde{\rmH}_L$ - to be the $v$-adic completion:
                $$\tilde{\rmH}_L := \projlim_{m \geq 1} \bar{\rmH}_L/\bar{\rmH}_L \cdot v^n$$
        \end{definition}

        It is clear that the Heisenberg algebra $\tilde{\h}_L$ admits a triangular decomposition of sort, as follows, which is nothing but the $\Z$-grading on $\tilde{\h}_L$:
            $$\tilde{\h}_L \cong \tilde{\h}_L^- \oplus \tilde{\h}_L^0 \oplus \tilde{\h}_L^+$$
        where:
            $$\tilde{\h}_L^{\pm} := \bigoplus_{m \in \pm \Z_{> 0}} \h_L v^m, \tilde{\h}_L^0 := \h_L \oplus \bbC c_L$$
        This, in turn, induces a triangular decomposition for $\tilde{\rmH}_L$, and we will be interested in the upper \say{Borel subalgebra} $\tilde{\rmH}_L^{\geq 0}$, which is the quotient by $c_L - 1$ of the $v$-adically completed universal enveloping algebra of $\bigoplus_{m \in \Z_{\geq 0}} \h_L v^m \oplus \bbC c_L$. This allows us to define the \textbf{vacuum module} of highest weight $\lambda \in \h_L^*$:
            $$\V_L^{\lambda} := \tilde{\rmH}_L \tensor_{\tilde{\rmH}_L^{\geq 0}} \bbC c_L$$
        A standard argument shows that, as a vector space, this is isomorphic to the underlying vector space of $\rmU(v^{-1}\h_L[v^{-1}])$, which itself is isomorphic to the vector space:
            $$\Sym( v^{-1}\h_L[v^{-1}] ) \cdot v_{\lambda} \cong \Sym(\h_L)[b_{-1}, b_{-2}, ...] \cdot v_{\lambda} \cong \rmU(\h_L)[b_{-1}, b_{-2}, ...]$$
        as $v^{-1}\h_L[v^{-1}]$ is an abelian Lie subalgebra of $\tilde{\h}_L$. As a left-$\tilde{\rmH}_L$-module, it is specified by the relations:
            $$\tilde{\h}_L^+ \cdot v_{\lambda} = 0$$
            $$\forall h \in \h_L h \cdot v_{\lambda} = \lambda(h) v_{\lambda}$$
        Note also that there is an evident $\Z_{\geq 0}$-grading on $\V_L^{\lambda}$, given by:
            $$\deg v_{\lambda} := 0$$
            $$\forall h \in \h_L: \deg h \cdot v_{\lambda} := 0$$
            $$\forall m < 0: \deg b_m \cdot v_{\lambda} := -m$$
        This is important for showing that:
        \begin{lemma}[$\V_L^{\lambda}$ is simple] \label{lemma: weyl_vacuum_modules_are_simple}
            For any $\lambda \in \h_L^*$, the left-$\tilde{\rmH}_L$-module $\V_L^{\lambda}$ is simple.
        \end{lemma}
            \begin{proof}
                Suppose $U$ is a non-zero left-$\tilde{\rmH}_L$-submodule of $\V_L^{\lambda}$, which \textit{may or may not be proper}. By construction, $\V_L^{\lambda}$ is cyclic, and so $U$ is also generated by the maximal vector $v_{\lambda}$, and hence in particular, also $\Z_{\geq 0}$-graded with the degree-$0$ component being $\bbC v_{\lambda}$. Now, because $\tilde{\h}_L^-$ acts freely via left-multiplication by the elements $b_{-1}, b_{-2}, ...$, for any homogeneous element $u \in U$, any $h \in \h_L$, and any $m < 0$, one has that:
                    $$\deg( h v^m \cdot u ) = \deg b_m + \deg u = -m + \deg u$$
                This implies that for $U$ to be a well-defined left-$\tilde{\rmH}_L$-module, we must have that:
                    $$U \cong \bigoplus_{n \in \Z_{\geq 0}} U_n$$
                in which:
                    $$\forall n \in \Z_{\geq 0}: U_n \not = 0$$
                Now, by the general theory of graded modules, we know that:
                    $$U_n = U \cap (\V_L^{\lambda})_n$$
                We also know that the underlying vector space of $\V_L^{\lambda}$ is $\rmU(\h_L)[b_{-1}, b_{-2}, ...] \cdot v_{\lambda}$, the degree-$n$ components $(\V_L^{\lambda})_n$ are nothing but spans of all (commutative) monomials of the kind:
                    $$h b_{m_r} ... b_{m_1}$$
                wherein $h \in \rmU(\h_L)$ is some monomial and the indices $m_r \geq ... \geq m_1$ make up an ordered partition of $n$, i.e.:
                    $$m_r + ... + m_1 = n$$
                (note that these monomials are linearly independent from one another). Thus, we see that each $U_n$ is spanned by certain monomials of said kind. However, given any ordered partition $m_r + ... + m_1 = n$ of some $n \geq 0$ corresponding to some monomial $b_{m_r} ... b_{m_1}$ of degree $n$, one can always construct a new monomial $b_{m_{r + 1}} b_{m_r} ... b_{m_1}$ of degree $-m_{r + 1} + n$ for some $m_{r + 1} < 0$. This implies that for every $n \geq 0$, one actually have that:
                    $$U_n = (\V_L^{\lambda})_n$$
                i.e. $U$ being a non-zero submodule automatically implies that it is the entirety of $\V_L^{\lambda}$. In other words, $\V_L^{\lambda}$ admits no non-zero proper submodule, and is therefore simple.
            \end{proof}
        \begin{lemma}
            Any $\tilde{\h}_L$-module, say $V$, on which the central charge $c_L$ acts as a scalar is necessarily infinite-dimensional, unless said scalar is $0$.
        \end{lemma}
            \begin{proof}
                Indeed, if we were to suppose for the sake of deriving a contradiction that $\dim V < +\infty$ yet $c_L$ acts as a non-zero scalar, then because:
                    $$[A, B]_{\tilde{\h}_L} \in \bbC c_L$$
                and because:
                    $$\trace([A, B]_{\tilde{\h}_L}) = \trace(AB) - \trace(BA) = 0$$
                (with both statements holding for all $A, B \in \tilde{\h}_L$), we will clearly get a contradiction. Note the subtlety that $\trace(AB) = \trace(BA)$ is an identity depending crucially on the assumption that $\dim V < +\infty$.
            \end{proof}
        \begin{corollary}
            $\tilde{\rmH}_L$-modules are necessarily infinite-dimensional. 
        \end{corollary}
        \begin{proposition}[Classification of $\Z_{\geq 0}$-graded simple left-$\tilde{\rmH}_L$-modules of countable dimensions] \label{prop: simple_lattice_weyl_modules_are_vacuum_modules}
            Any $\Z_{\geq 0}$-graded simple left-$\tilde{\rmH}_L$-module of some countable dimension is isomorphic to some $\V_L^{\lambda}$, for some $\lambda \in \h_L^*$.
        \end{proposition}
            \begin{proof}
                Since we know that $\tilde{\rmH}_L$-modules are necessarily infinite-dimensional, if $V := \bigoplus_{n \in \Z_{\geq 0}} V_n$ is a $\Z_{\geq 0}$-graded $\tilde{\rmH}_L$-module then because $\dim V$ is assumed to be countable, the graded components will be such that $V_n \not = 0$ for all $n \in \Z_{\geq 0}$. Without any loss of generality, we can pick the grading on $V$ so that $V_0$ is a $\tilde{\rmH}_L^0$-submodule of $V$. Furthermore, the assumption that $V$, as a $\tilde{\rmH}_L$-module, is only strictly $\Z_{\geq 0}$-graded instead of being $\Z$-graded, implies that $\tilde{\rmH}_L^+ \cdot V_0 = 0$.

                Let us now show that $\dim V_0 = 1$ if $V$ is furthermore simple to see that as a $\tilde{\rmH}_L$-module, $V$ is generated by a single highest-weight vector. To this end, suppose for the sake of deriving a contradiction that there exists a proper $\tilde{\rmH}_L^0$-submodule $U_0$ of $V_0$. This assumption, however, implies that $U := \tilde{\rmH}_L \cdot U_0$ shall be a proper $\tilde{\rmH}_L$-submodule of $V$. As $V$ is simple, we thus infer that $U = 0$, which in turn implies that the only proper $\tilde{\rmH}_L^0$-submodule of $V_0$ is $0$. $\tilde{\rmH}_L^0$, by construction, is a commutative algebra, so $\tilde{\rmH}_L^0$-submodules are nothing but vector subspaces, and hence the only proper vector subspace of $V_0$ is $0$. In other words, $\dim V_0 = 1$ necessarily, and hence we are done.

                Now, denote the highest-weight vector of $V$ by $v_{\max}$, i.e. we set $V_0 := \bbC v_{\max}$. It remains to determine the eigenvalue(s) of $v_{\max}$ under actions of elements of $\h_L$. Because $V$ is simple as a $\tilde{\rmH}_L$-module, and because $\h_L$ is an abelian Lie algebra by definition - and hence $\rmU(\h_L)$ is a commutative subalgebra of $\tilde{\rmH}_L$ - its image under the composition:
                    $$\rmU(\h_L) \to \tilde{\rmH}_L \to \End(V)$$
                is necessarily $1$-dimensional, per Schur's Lemma (note that we need the ground field to be algebraically closed for this, which $\bbC$ is). Elements $h \in \h_L$ acting on $v_{\max}$ thus eigenvalues $\lambda(h)$, for some common $\lambda \in \h_L^*$. 

                We have therefore proven that $V$ is a simple $\tilde{\rmH}_L$-module generated by a single vector $v_{\max}$, for which there exists some $\lambda \in \h_L^*$ such that:
                    $$\tilde{\rmH}_L^+ \cdot v_{\max} = 0$$
                    $$\forall h \in \h_L \cdot v_{\max} = \lambda(h) v_{\max}$$
                Thus, $V \cong \V_L^{\lambda}$ for some $\lambda \in \h_L^*$.
            \end{proof}

            Even though proposition \ref{prop: simple_lattice_weyl_modules_are_vacuum_modules} does not give a complete classification of $\tilde{\rmH}_L$-modules, it is sufficiently useful for our purposes, as it guarantees that \textit{any $\tilde{\rmH}_L$-module with underlying vector space satisfying the assumptions of the Reconstruction Theorem is necessary some direct sums of the modules $\V_L^{\lambda}$.} We therefore begin our study of VOAs associated to lattices by studying the VOA structures on the modules $\V_L^{\lambda}$ granted to us by the Reconstruction Theorem. 
            \begin{example}[The bosonic Fock module] \label{example: bosonic_fock_modules}
                A very important class of examples of countably dimensional $\Z_{\geq 0}$-graded simple left-$\tilde{\rmH}_L$-modules is that of \textbf{Fock modules} (or \textbf{Fock spaces}, to use the physicists' term) of highest weight $\lambda \in \h_L^*$. Before discussing the details surrounding these representations of $\tilde{\rmH}_L$, let us fix a (finite) $\Z$-linear basis $\{a_i\}_{i \in I}$ for $L$. This set is also a $\bbC$-linear basis for $\h_L := L \tensor_{\Z} \bbC$.
                
                Let us denote said $\tilde{\rmH}_L$-modules by $\scrF_L^{\lambda}$. As vector spaces, they are isomorphic to:
                    $$\rmU(\h_L)[b_{-1}, b_{-2}, ...]$$
                much like the vacuum modules $\V_L^{\lambda}$ and in particular, this means that $\dim \scrF_L^{\lambda}$ is countable and that the underlying vector space of each $\scrF_L^{\lambda}$ is $\Z_{\geq 0}$-graded, in the same way that $\V_L^{\lambda}$ is $\Z_{\geq 0}$-graded. The $\tilde{\rmH}_L$-action is given by:
                    $$
                        a_i v^m \mapsto
                        \begin{cases}
                            \text{$b_{-m} \cdot \id_{\rmU(\h_L)[b_{-1}, b_{-2}, ...]}$ if $m > 0$}
                            \\
                            \text{$-m \frac{\del}{\del b_m}$ if $m < 0$}
                            \\
                            \text{$\lambda(a_i) \cdot \id_{\rmU(\h_L)[b_{-1}, b_{-2}, ...]}$ if $m = 0$}
                        \end{cases}
                    $$
                for all $i \in I$, and clearly, this action is $\Z_{\geq 0}$-graded and of highest weight $\lambda \in \h_L^*$. Therefore, the only thing to prove in order to demonstrate that:
                    $$\scrF_L^{\lambda} \cong \V_L^{\lambda}$$
                as $\tilde{\rmH}_L$-modules, is that $\scrF_L^{\lambda}$ is simple.
                \todo[inline]{Show that the Fock modules $\tilde{\rmH}_L^{\lambda}$ are simple.}
            \end{example}

            Due to the existence of the Fock modules, there is now a need for understanding algebras of differential operators in (countably) infinitely many variables. For this, vertex operators are useful.

            As an experiment in constructing VOA structures, let us fix some $N \in \Z_{> 0}$ and then consider, firstly, the rank-$1$ lattice:
                $$L := \sqrt{N} \Z$$
            equipped with the non-degenerate $\Z$-bilinear form given by:
                $$\<\lambda, \mu\> := \frac1N \lambda \mu$$
            for all $\lambda, \mu \in \sqrt{N} \Z$. Note that in this case, we have that $\h_L := \sqrt{N}\Z \tensor_{\Z} \bbC \cong \bbC$, and therefore $\h_L^* \cong \bbC^* \cong \bbC$ as well.
            \begin{remark}
                Since the root lattice of $\sl_2$ is isomorphic to $\sqrt{N} \Z$ for some choice of $N \in \Z_{> 0}$ and since weights of $\sl_2$ are nothing but elements of $\bbC^* \cong \bbC$, the construction of a VOA structure on $\V_{\sqrt{N} \Z}^{\lambda}$ (for some $\lambda \in \bbC$) can be seen as a step towards constructing a VOA structure on the vacuum module of $\sl_2$ of highest weight $\lambda$ \textit{\`a la} Frenkel-Kac. When dealing with finite-dimensional simple Lie algebras $\g$, say with root space decomposition $\g \cong \h \oplus \bigoplus_{\alpha \in \Phi} \g_{\alpha}$, more general than $\sl_2$, different choices of $N$ will correspond to different choices of isomorphisms:
                    $$\sl_2 \xrightarrow[]{\cong} \bbC x_{-\alpha} \oplus \bbC \check{\alpha} \oplus \bbC x_{\alpha}$$
                wherein:
                    $$\alpha \in \Phi^+, \height(\alpha) = N$$
                and $x_{\pm \alpha} \in \g_{\pm \alpha}$ are root vectors such that $[x_{\alpha}, x_{-\alpha}] = \check{\alpha}$.
            \end{remark}
            \begin{proposition}[Rank-$1$ lattice simple (super) VOAs] \label{prop: rank_1_lattice_(super)_VOAs}
                Let $N \in \Z_{> 0}$ and fix some $\lambda \in \bbC$. Denote the highest-weight vector of the $\rmH_{\sqrt{N} \Z}$-module $\V_{\sqrt{N} \Z}^{\lambda}$ by $v_{\lambda}$.
                \begin{enumerate}
                    \item If $N$ is even, then on $\V_{\sqrt{N} \Z}^{\lambda}$, there will be a $\Z_{\geq 0}$-graded VOA structure:
                        $$Y: \V_{\sqrt{N} \Z}^{\lambda} \to \End(\V_{\sqrt{N} \Z}^{\lambda})[\![z^{\pm 1}]\!]$$
                    which is of conformal dimension $1$ and specified on the highest-weight vector $v_{\lambda}$ by:
                        $$Y(v_{\lambda}, z) := $$
                    \item On the other hand, if $N$ is odd, then $\V_{\sqrt{N} \Z}^{\lambda}$ will admit a $\Z_{\geq 0}$-graded \textit{super} VOA structure of conformal dimension $1$ given by:
                \end{enumerate}
            \end{proposition}
                \begin{proof}
                    \begin{enumerate}
                        \item Let us firstly explain why it is enough to only specify the VOA structure on the highest-weight vector $v_{\lambda}$. For this, set:
                            $$\beta := \sqrt{N} \cdot 1$$
                        so that:
                            $$\sqrt{N} \Z \cong \Z \beta$$
                        Firstly, $\V_{\sqrt{N} \Z}^{\lambda}$ is isomorphic to the left-$\rmH_{\sqrt{N} \Z}$-module:
                            $$\rmH_{\sqrt{N} \Z}^- \cong \rmU(\h_{\sqrt{N} \Z})[b_{-1}, b_{-2}, ...] \cong \bbC[\beta][b_{-1}, b_{-2}, ...]$$
                        generated by $b_{-1}$, which is mapped to $b_{\lambda}$ under the isomorphism. Secondly $\bbC[\beta, b_{-1}, b_{-2}, ...]$ is spanned by monomials of the form $b_{m_r} ... b_{m_1}$ (where $m_r \geq ... \geq m_1$), and since:
                            $$Y(b_{m_r} ... b_{m_1}, z) = \: : Y(b_{m_r}, z) ... Y(b_{m_1}, z) :$$
                        it is enough to specify:
                            $$Y(v_{\lambda}, z) := Y(b_{-1}, z)$$

                        Let us now describe $Y(b_{-1}, z)$. We would like $Y(-, z)$ to be a $\Z_{\geq 0}$-graded linear map, so because $\deg b_{-1} = -1$ (recall that $\deg b_m := -m$ for all $m < 0$), let us consider:
                            $$Y(b_{-1}, z) := b(z) := \sum_{m \in \Z} b_m z^{-m - 1}$$
                        Let us then verify that, together with the translation operator $T \in \End(\V_{\sqrt{N}\Z}^{\lambda})$ given by:
                            $$[T, \beta] := 0$$
                            $$\forall m \in \Z_{< 0}: [T, b_m] := -m b_{m - 1}$$
                        and $v_{\lambda}$ as the vacuum vector, the assignment $Y(-, z)$ forms a $\Z_{\geq 0}$-graded VOA whenever $N$ is even; that this is of conformal dimension $1$ is automatic from the construction. To this end, let us check the axioms defining VOAs in sequence:
                        \begin{enumerate}
                            \item 
                            \item 
                            \item 
                        \end{enumerate}
                        \item 
                    \end{enumerate}
                \end{proof}
                
            \begin{theorem}[Bosonic vertex operators] \label{theorem: bosonic_vertex_operators}
                
            \end{theorem}
                \begin{proof}
                    
                \end{proof}

    \subsection{The Frenkel-Kac vertex realisation of basic representations of affine Lie algebras}
        There is a construction, due to Frenkel-Kac, which extends the bosonic vertex operators from theorem \ref{theorem: bosonic_vertex_operators} from the Weyl algebra $\rmH_Q$ attached to the root lattice of the finite-dimensional simple Lie algebra $\g$ to the entirety of the completed enveloping algebra $\tilde{\rmU}_{\kappa}$ of the affine Lie algebra $\tilde{\g}_{\kappa}$ associated to $\g$ at level $\kappa$. In performing this construction, one yields an explicit description of the basic representation $\simple^{\omega_1}$, i.e. the finite-dimensional simple $\hat{\g}_{\kappa}$-module attached to the first fundamental weight $\varpi_1$, in terms of the bosonic vertex operators, as well as a VOA structure on the vacuum representation $\V_{\kappa}$ of $\tilde{\g}_{\kappa}$.

    \subsection{The Virasoro VOAs and conformal modules}

    \subsection{Rational VOAs}