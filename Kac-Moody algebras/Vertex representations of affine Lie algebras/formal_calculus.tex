\section{Formal calculus}
    \subsection{Formal distributions and fields}
        \begin{definition}[Formal distributions] \label{def: formal_distributions}
            A \textbf{formal distribution} in $n$ variables $z_1, ..., z_n$ with coefficients in a vector space $U$ is an element of the vector space $U[\![z_1^{\pm}, ..., z_n^{\pm}]\!]$.
        \end{definition}
        It is very important to note that even when $U$ is an associative algebra, $U[\![z_1^{\pm}, ..., z_n^{\pm}]\!]$ is \textit{not} an algebra in general. This is because the coefficients of monomials in a product of the form:
            $$\left( \sum_{(i_1, ..., i_n) \in \Z^n} a_{i_1, ..., i_n} z_1^{i_1} ... z_n^{i_n} \right) \left( \sum_{(j_1, ..., j_n) \in \Z^n} b_{j_1, ..., j_n} z_1^{j_1} ... z_n^{j_n} \right)$$
        (wherein $a_{i_1, ..., i_n}, b_{j_1, ..., j_n} \in U$) are of the form:
            $$\sum_{(i_1, ..., i_n) \in \Z^n} \sum_{(j_1, ..., j_n) \in \Z^n} a_{i_1, ..., i_n} b_{j_1, ..., j_n}$$
        which, generally speaking, shall be an infinite sum.

        \begin{convention}
            Given any formal distribution:
                $$a(z) := \sum_{m \in \Z} a_m z^{-m - 1}$$
            let us write:
                $$a(z)^- := \sum_{m \in \Z_{\geq 0}} a_m z^{-m - 1}, a(z)^+ := \sum_{m \in \Z_{< 0}} a_m z^{-m - 1}$$
            with the point being that we would then have:
                $$(\del a(z))^{\pm} = \del( a(z)^{\pm} )$$
        \end{convention}
        \begin{definition}[Quantum fields] \label{def: quantum_fields}
            Let $V$ be a vector space, to be thought of as a vector space of \say{quantum states}. A \textbf{(quantum) field} with values in $V$ is then a formal distribution:
                $$a(z) := \sum_{m \in \Z} a_m z^{-m - 1} \in \End(V)[\![z^{\pm 1}]\!]$$
            such that:
                $$\forall v \in V: a(z) \cdot v \in V(\!(z)\!)$$
            in the sense that $a_m \cdot v = 0$ for all $v \in V$ when $m \gg -\infty$. The subspace of $\End(V)[\![z^{\pm 1}]\!]$ spanned by fields shall be denoted by $\Fld(V)$, and note that this is strictly larger than $\End(V)(\!(z)\!)$.
        \end{definition}
        \begin{definition}[Normal ordering] \label{def: normal_ordering}
            Given two formal distributions $a(w) \in U[\![w^{\pm 1}]\!], b(z) \in U[\![z^{\pm 1}]\!]$, let us define their \textbf{normally ordered product} to be:
                $$\order{ a(w) b(z) } := a(w)^+ b(z) + b(z) a(w)^-$$
            We will also be writing:
                $$a(w) b(z) \sim a(w) b(z) - \order{ a(w) b(z) }$$
            to mean the \textbf{singular part} of the product $a(w) b(z)$. It is easy to see that $\sim$ is a linear relation.
        \end{definition}
        \begin{remark}
            If $\a$ is  $\Z$-graded Lie algebra and $a(w) := \sum_{m \in \Z} a_m z^{-m - 1}, b(z) := \sum_{n \in \Z} b_n z^{-n - 1}$ are formal distribution with values in $\rmU(\a)$, such that:
                $$\deg a_m := m, \deg b_n := n$$
            for all $m, n \in \Z$, then note that the normally ordered product $\order{ a(w) b(z) }$ is nothing but the sub-sum of the \say{full} product $a(w) b(z)$ consisting of terms whose coefficients:
                $$a_m b_n$$
            exist as PBW monomials in $\rmU(\a)$.

            Furthermore, if $U$ is any associative algebra, then we shall have that:
                $$
                    \begin{aligned}
                        [a(w), b(z)] & := a(w) b(z) - b(z) a(w)
                        \\
                        & = ( a(w)^+ + a(w)^- ) b(z) - b(z) ( a(w)^+ + a(w)^- )
                        \\
                        & = ( a(w)^+ b(z) + b(z) a(w)^- ) + a(w)^- b(z) - b(z) a(w)^+
                        \\
                        & = \order{ a(w) b(z) } + a(w)^- b(z) - b(z) a(w)^+
                    \end{aligned}
                $$
            from which we gather that:
                $$
                    \begin{aligned}
                        a(w) b(z) & = \order{ a(w) b(z) } + a(w)^- b(z) - b(z) a(w)^+ + b(z) a(w)
                        \\
                        = & \order{ a(w) b(z) } + a(w)^- b(z) - b(z) ( -a(w)^+ + a(w) )
                        \\
                        = & \order{ a(w) b(z) } + a(w)^- b(z) - b(z) a(w)^-
                        \\
                        = & \order{ a(w) b(z) } + [ a(w)^-, b(z) ]
                    \end{aligned}
                $$
            and likewise, that:
                $$
                    b(z) a(w) = \order{ a(w) b(z) } - [ a(w)^+, b(z) ]
                $$
            In turn, this implies that:
                \begin{equation} \label{equation: OPEs}
                    a(w) b(z) \sim [ a(w)^-, b(z) ]
                \end{equation}
                $$b(z) a(w) \sim -[ a(w)^+, b(z) ]$$
        \end{remark}
        \begin{lemma}[Normally ordered products of fields] \label{lemma: normally_ordered_products_of_fields}
            Let $V$ be a vector space. Then $\Fld(V)$ with the normally ordered product will be an associative algebra.
        \end{lemma}
            \begin{proof}[Proof sketch]
                Note that for fields $a(w), b(z)$, only finitely many of the the coefficients of $\order{ a(w) b(z) }$ are non-zero.
            \end{proof}

    \subsection{Locality and Lie (super)algebras of formal distributions}
        Let us begin this subsection by introducing some computational tools, namely formal residues and Dirac distributions.
        \begin{definition}[Formal residues] \label{def: formal_residues}
            Let $U$ be vector space. Given a formal distribution $a(z_1, ..., z_n) := \sum_{(i_1, ..., i_n) \in \Z^n} a_{i_1, ..., i_n} z_1^{i_1} ... z_n^{i_n} \in U[\![z_1^{\pm}, ..., z_n^{\pm}]\!]$, let us define its \textbf{formal residue} to be:
                $$\Res_{z_1, ..., z_n} a(z_1, ..., z_n) := a_{-1, ..., -1}$$
        \end{definition}
        
        \begin{convention}
            As a shorthand, we will usually be writing:
                $$\del a(z_1, ..., z_n) := \sum_{i = 1}^n \frac{\del a(z_1, ..., z_n)}{\del z_i} dz_i$$
            for the total derivative of a formal distribution $a(z_1, ..., z_n)$.
        \end{convention}
        \begin{lemma}[Integration by parts] \label{lemma: integration_by_parts}
            Let $U$ be a vector space.
            \begin{enumerate}
                \item If $a(z_1, ..., z_n) \in U[z_1^{\pm 1}, ..., z_n^{\pm 1}]$ is a formal distribution, then:
                    $$\Res_{z_1, ..., z_n} \del a(z_1, ..., z_n) = 0$$
                \item If $a(z_1, ..., z_n), b(z_1, ..., z_n) \in U[z_1^{\pm 1}, ..., z_n^{\pm 1}]$ are formal distribution, then we will also have the following \say{integration by parts} formula:
                    $$\Res_{z_1, ..., z_n} (\del a(z_1, ..., z_n)) b(z_1, ..., z_n) = -\Res_{z_1, ..., z_n} (\del b(z_1, ..., z_n)) a(z_1, ..., z_n)$$
            \end{enumerate}
        \end{lemma}

        Formal residues allow us to define a non-degenerate bilinear pairing:
            $$\<-, -\>: U[\![z_1^{\pm}, ..., z_n^{\pm}]\!] \x \bbC[z_1^{\pm}, ..., z_n^{\pm}] \to U$$
        given by:
            $$\< a(z_1, ..., z_n), f(z_1, ..., z_n) \> := \Res_{z_1, ..., z_n} \left( a(z_1, ..., z_n) f(z_1, ..., z_n) \right)$$
        As Laurent polynomials $f(z_1, ..., z_n) \in \bbC[z_1^{\pm 1}, ..., z_n^{\pm 1}]$ have only finitely many summands with non-zero coefficients, the bilinear form $\<-, -\>$ is indeed well-defined. Non-degeneracy follows from the fact $\bbC$, being a field, contains no non-zero annihilator of elements of $U$. In any event, this means that we can view Laurent polynomials $f(z_1, ..., z_n) \in \bbC[z_1^{\pm 1}, ..., z_n^{\pm 1}]$ as test functions for formal distributions. In particular, these test functions are very useful for dealing with \textbf{formal Dirac distribution} (also called \textbf{formal delta distributions}). These are two-variable formal distributions given by:
            $$\1(w, z) := \sum_{m \in \Z} w^m z^{-m - 1} \in \bbC[w^{\pm 1}, z^{\pm 1}]$$

        \begin{lemma}
            Let $U$ be a vector space and let us regard $U[\![w^{\pm 1}, z^{\pm 1}]\!]$ as a $\bbC[w^{\pm 1}, z^{\pm 1}]$-module via (left-)multiplication. Then, $U[\![w^{\pm 1}, z^{\pm 1}]\!]$ is faithful, i.e. multiplication by each $f(w, z) \in \bbC(w, z)$ is an injective operator.
        \end{lemma}
            \begin{proof}
                Suppose that there are formal distributions $a(w, z), b(w, z) \in U[\![w^{\pm 1}, z^{\pm 1}]\!]$ which are annihilated by the same rational function $f(w, z) \in \bbC(w, z)$, i.e.:
                    $$f(w, z) a(w, z) = f(w, z) b(w, z) = 0$$
                But this implies that:
                    $$\Res_{w, z} f(w, z) a(w, z) = \Res_{w, z} f(w, z) b(w, z) = 0$$
                i.e.:
                    $$\< f(w, z), a(w, z) \> = \< f(w, z), b(w, z) \> = 0$$
                The pairing $\<-, -\>$, however, is non-degenerate, meaning that the equation above holds true for all $f(w, z) \in \bbC(w, z)$ if and only if:
                    $$a(w, z) = b(w, z) = 0$$
                or in other words, if and only if $U[\![w^{\pm 1}, z^{\pm 1}]\!]$ is faithful as a $\bbC(w, z)$-module.
            \end{proof}
        
        \begin{convention}
            If $A$ is any operator, let us write:
                $$A^{[m]} := \frac{1}{m!} A^m$$
            for its \textbf{divided $m^{th}$ power}.
        \end{convention}
        \begin{lemma}[Basic properties of Dirac distributions] \label{lemma: basic_properties_of_dirac_distributions}
            Dirac distributions enjoy some very important properties.
            \begin{enumerate}
                \item For any Laurent polynomial $f(z) \in \bbC[z^{\pm 1}]$ (regarded by its image under the canonical embedding $\bbC[z^{\pm 1}] \subset \bbC[w^{\pm 1}, z^{\pm 1}]$), we have that:
                    $$\Res_z \1(w, z) f(z) = f(w)$$
                \item We have that:
                    $$\1(w, z) = \1(z, w)$$
                Furthermore, for any vector space $U$ and any formal distribution $a(z) \in U[z^{\pm}]$ (regarded by its image under the canonical embedding $\bbC[\![z^{\pm 1}]]\!] \subset \bbC[\![w^{\pm 1}, z^{\pm 1}]\!]$), we have that:
                    $$\1(w, z) a(z) = \1(z, w) a(w)$$
                \item We also have that:
                    $$\1(w, z) \1(z, t) = \1(t, w) \1(w, z)$$
                with identifications $\bbC[\![z^{\pm 1}, t^{\pm 1}]\!] \cong \bbC[\![t^{\pm 1}, w^{\pm 1}]\!] \cong \bbC[\![w^{\pm 1}, z^{\pm 1}]\!]$ being made.
                \item The Dirac distribution has vanishing total derivative:
                    $$\del \1(w, z) = 0$$
                or, phrased differently, we have that:
                    $$\del_w \1(w, z) = -\del_z \1(w, z)$$
                \item Fix some $j \in \N$. Then:
                    $$
                        (z - w)^n \del_w^{[j]} \1(w, z) =
                        \begin{cases}
                            \text{$\del_w^{[j - n]} \1(w, z)$ if $j \leq n$}
                            \\
                            \text{$0$ if $n > j$}
                        \end{cases}
                    $$
            \end{enumerate}
        \end{lemma}

        \begin{remark}[OPE for the Dirac distribution]
            It is also worth noting that:
                $$
                    \begin{aligned}
                        \1(w, z) & = \sum_{m \in \Z} w^m z^{-m - 1}
                        \\
                        & = \frac1z \sum_{m \in \Z_{\geq 0}} \left(\frac{w}{z}\right)^m + \frac1z \sum_{m \in \Z_{< 0}} \left(\frac{w}{z}\right)^m
                        \\
                        & = \frac1z \sum_{m \in \Z_{\geq 0}} \left(\frac{w}{z}\right)^m + \frac1z \sum_{m \in \Z_{> 0}} \left(\frac{z}{w}\right)^m
                        \\
                        & = \frac{1}{z - w}\bigg|_{|w| < |z|} + \frac1z\left( 1 - \frac{1}{1 - \frac{z}{w}} \right)\bigg|_{|w| > |z|}
                        \\
                        & = \frac{1}{z - w}\bigg|_{|w| < |z|} - \frac1z \frac{\frac{z}{w}}{1 - \frac{z}{w}}\bigg|_{|w| > |z|}
                        \\
                        & = \frac{1}{z - w}\bigg|_{|w| < |z|} - \frac1z \frac{1}{\frac{w}{z} - 1}\bigg|_{|w| > |z|}
                        \\
                        & = \frac{1}{z - w}\bigg|_{|w| < |z|} - \frac{1}{z - w}\bigg|_{|w| > |z|}
                    \end{aligned}
                $$
            and that, when we regard $\1(w, z)$ as an element of $\bbC[\![w^{\pm 1}]\!][\![z^{\pm 1}]\!]$ (i.e. as a formal distribution in the variable $z$ and with coefficients in $\bbC[\![w^{\pm 1}]\!]$), we have that:
                $$\1(w, z)^+ = \frac1z \sum_{m \in \Z_{\geq 0}} \left(\frac{w}{z}\right)^m, \1(w, z)^- = \frac1z \sum_{m \in \Z_{< 0}} \left(\frac{w}{z}\right)^m$$
            By putting the two observations together, we see that:
                $$\1(w, z) = \order{ \1(w, z) \cdot 1 } \sim \frac{1}{z - w}$$
            This is an important identity that we will constantly be making use of.
        \end{remark}
        
        \begin{definition}[Locality] \label{def: locality}
            Let $U$ be a vector space. A formal distribution $a(w, z) \in U[\![w^{\pm 1}, z^{\pm 1}]\!]$ is said to be \textbf{local} of degree $N$ if and only if there exists some sufficiently large $N \in \N$ such that:
                $$(z - w)^N a(w, z) = 0$$
            or in other words, if:
                $$a(w, z) \in \ker (z - w)^N \cdot$$
            where $(z - w)^N \cdot \in \End( U[\![w^{\pm 1}, z^{\pm 1}]\!]$ is the operator of left-multiplication by $(z - w)^N$.
                
            If $U$ is furthermore an associative algebra then we shall say that two formal distributions\footnote{We are implicitly identifying these with their images under the embeddings $U[\![w^{\pm 1}]\!] \subset U[\![w^{\pm 1}, z^{\pm 1}]\!] \supset U[\![z^{\pm 1}]\!]$.} $a(w) \in U[\![w^{\pm 1}]\!], b(z) \in U[\![z^{\pm 1}]\!]$ are \textbf{mutually local} if and only if the formal distribution $[a(w), b(z)] \in U[\![w^{\pm 1}, z^{\pm 1}]\!]$ is local in the sense above.
        \end{definition}
        
        \begin{lemma}[Cauchy's integral formula] \label{lemma: cauchy_integration_formula}
            Let $U$ be a vector space. Then, a given formal distribution $u(w, z) \in U[\![w^{\pm 1}, z^{\pm 1}]\!]$ is local if and only if there exists some $N \in \N$ such that:
                $$u(w, z) \in \sum_{j = 0}^{N - 1} \left( \del_w^{[j]} \1(w, z) \cdot U[\![w^{\pm}]\!] \right)$$
            Moreover, if:
                $$u(w, z) \in \sum_{j = 0}^{N - 1} ( \del_w^{[j]} \1(w, z) ) U[\![w^{\pm 1}]\!]$$
            then we even have the following explicit identification:
                $$u(w, z) = \sum_{j = 0}^{N - 1} ( \del_w^{[j]} \1(w, z) ) \Res_z( (z - w)^j u(w, z) )$$
        \end{lemma}
            \begin{proof}
                We know from \ref{lemma: basic_properties_of_dirac_distributions} that:
                    $$
                        (z - w)^n \del_w^{[j]} \1(w, z) =
                        \begin{cases}
                            \text{$\del_w^{[j - n]} \1(w, z)$ if $j \leq n$}
                            \\
                            \text{$0$ if $n > j$}
                        \end{cases}
                    $$
                for all $j \in \N$. The first assertion then follows from the faithfulness of $U[\![w^{\pm 1}, z^{\pm 1}]\!]$ as a $\bbC(w, z)$-module, as well as the definition of locality.

                To show the second assertion, let us begin by setting:
                    $$u(w, z) := \sum_{j = 0}^{N - 1} ( \del_w^{[j]} \1(w, z) ) u_j(w)$$
                wherein $u_j(w) \in U[\![w^{\pm 1}]\!]$. Multiplying both sides by $(z - w)^j$ then gives:
                    $$
                        \begin{aligned}
                            (z - w)^j u(w, z) & = \sum_{j = 0}^{N - 1} (z - w)^j ( \del_w^{[j]} \1(w, z) ) u_j(w)
                            \\
                            & = ( \del_w^{[j - j]} \1(w, z) ) u_j(w)
                            \\
                            & = \1(w, z) u_j(w)
                            \\
                            & = \1(z, w) u_j(w)
                            \\
                            & = \1(w, z) u_j(z)
                        \end{aligned}
                    $$
                Applying $\Res_z$ to both sides then yields:
                    $$\Res_z( (z - w)^j u(w, z) ) = \Res_z( \1(w, z) u_j(z) ) = u_j(w)$$
                and so we indeed have that:
                    $$u(w, z) = \sum_{j = 0}^{N - 1} ( \del_w^{[j]} \1(w, z) ) \Res_z( (z - w)^j u(w, z) )$$
                when we suppose that $u(w, z) \in \sum_{j = 0}^{N - 1} ( \del_w^{[j]} \1(w, z) ) U[\![w^{\pm 1}]\!]$.
            \end{proof}
        \begin{definition}[Holomorphic distributions] \label{def: holomorphic_formal_distributions}
            A formal distribution $u(w, z) \in U[\![w^{\pm 1}, z^{\pm 1}]\!]$ is said to be \textbf{holomorphic} if:
                $$u(w, z) = \sum_{j = 0}^{+\infty} ( \del_w^{[j]} \1(w, z) ) \Res_z( (z - w)^j u(w, z) )$$
            In other words, $u(w, z)$ is holomorphic if and only if we can write:
                $$u(w, z) := \sum_{N = 0}^{+\infty} u^{(N)}(w, z)$$
            wherein each $u^{(N)}(w, z) \in U[\![w^{\pm 1}, z^{\pm 1}]\!]$ is local.

            The subspace of $U[\![w^{\pm 1}, z^{\pm 1}]\!]$ spanned by holomorphic elements shall be denoted by $U[\![w^{\pm 1}, z^{\pm 1}]\!]^{\circ}$, and we note that this is in fact a subspace of $\sum_{N = 1}^{+\infty} \sum_{j = 0}^{N - 1} ( \del_w^{[j]} \1(w, z) ) U[\![w^{\pm 1}]\!]$.
        \end{definition}
        \begin{lemma}[Taylor series of holomorphic distributions] \label{lemma: taylor_seres_of_holomorphic_distributions}
            Let $U$ be a vector space and $u(z) \in U[\![z^{\pm 1}]\!]$ (regarded by its image under the embedding $\bbC[\![z^{\pm 1}]\!] \subset \bbC[\![w^{\pm 1}, z^{\pm 1}]\!]$) be a holomorphic distribution. We have the following formal version of Taylor's formula, taking the form of an equality in $U[\![w^{\pm 1}, z^{\pm 1}]\!]$:
                $$\del_w^{[N]} \1(w, z) u(z) = \del_w^{[N]} \1(w, z) \sum_{j = 0}^N (\del_w^{(j)} u(w)) (z - w)^j$$
        \end{lemma} 
            \begin{proof}
                
            \end{proof}
        
        \begin{example}[Laurent series expansion of the Dirac distribution] \label{example: laurent_series_expansion_of_dirac_distribution}
            The Dirac distribution is an important example of a holomorphic distribution. To see why, note that:
                $$
                    (z - w)^j \1(w, z) = (z - w)^j \del_w^{[0]} \1(w, z)
                    \begin{cases}
                        \text{$\1(w, z)$ if $j = 0$}
                        \\
                        \text{$0$ if $j > 0$}
                    \end{cases}
                $$
            and then consider the following:
                $$
                    \begin{aligned}
                        \sum_{j = 0}^{+\infty} ( \del_w^{[j]} \1(w, z) ) \Res_z( (z - w)^j \1(w, z) ) & = ( \del_w^{[0]} \1(w, z) ) \Res_z( (z - w)^0 \1(w, z) )
                        \\
                        & =  \1(w, z) \Res_z \1(w, z)
                        \\
                        & = \1(w, z) \Res_z \sum_{m \in \Z} w^m z^{-m - 1}
                        \\
                        & = \1(w, z) w^0
                        \\
                        & = \1(w, z)
                    \end{aligned}
                $$
            
        \end{example}
            
        \begin{proposition}[Commutator OPEs for mutually local distributions] \label{prop: commutator_OPEs_for_mutually_local_distributions}
            Let $U$ be an associative algebra. Two formal distributions $a(w) \in U[\![w^{\pm 1}]\!], b(z) \in U[\![z^{\pm 1}]\!]$ are then mutually local of degree $N$ if and only if:
                $$[a(w), b(z)] \sim \sum_{j = 0}^{N - 1} \frac{1}{(z - w)^{j + 1}} \Res_z( (z - w)^j [a(w), b(z)] )$$
            for some $N \in \N$.
        \end{proposition}
            \begin{proof}
                
            \end{proof}