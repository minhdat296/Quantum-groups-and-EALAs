\section{Enveloping algebras and reconstruction}
    \subsection{(Completed) enveloping Lie algebras of vertex algebras}
        Suppose throughout that $(V, v_0, T, Y)$ be a vertex algebra.
    
        Consider the following operator on the vector space $V[t^{\pm 1}] := V \tensor_{\bbC} \bbC[t^{\pm 1}]$:
            $$\del_V := T \tensor \id_V + \id_V \tensor \del_t$$
        To the vertex algebra $V$, one can then associate the following vector space
            $$\fraku(V) := V[t^{\pm 1}]/\im \del_V$$
        By construction, this is spanned by the elements of the kind $A^{[m]} := A \tensor t^m$ modulo the subspace spanned by the relations:
            $$0 = \del_V \cdot A^{[m]} = (T \cdot A) \tensor t^m + A \tensor m t^{m - 1} = (T \cdot A)^{[m]} + m A^{[m - 1]}$$
        or equivalently:
            $$(T \cdot A)^{[m]} = -m A^{[m - 1]}$$
        which holds for all $A \in V$. Given some $A^{[m]}$, let us denote its lift modulo $\im \del_V$ by $A^{(m)}$ (which, to be thorough, we shall note to be an element of $V[t^{\pm 1}]$). Now, by the Translation Axiom in the definition of vertex algebras, we also have the following similar relations in $\End(V)$:
            $$(T \cdot A)^{(m)} = -m A^{(m - 1)}$$
        holding for all $A \in V$. This suggests to us that there exists a well-defined linear map:
            $$\fraku(V) \to \End(V)$$
            $$A^{[m]} \mapsto A^{(m)}$$
        This allows us to define a bilinear map:
            $$\widetilde{[-, -]}_{\fraku(V)}: V[t^{\pm 1}]^{\tensor 2} \to \fraku(V)$$
            $$A^{(m)} \tensor B^{(n)} \mapsto \sum_{k \in \Z_{\geq 0}} \binom{m}{k} (A^{(k)} \cdot B)^{[m + n - k]}$$
        wherein $A^{(k)}$ is regarded as an element of $\End(V)$. One can then show that this map admits a factorisation:
            $$
                \begin{tikzcd}
                {V[t^{\pm 1}]^{\tensor 2}} & {\fraku(V)} \\
                {\fraku(V)}
                \arrow["{\widetilde{[-, -]}_{\fraku(V)}}", from=1-1, to=1-2]
                \arrow[two heads, from=1-1, to=2-1]
                \arrow["{[-, -]_{\fraku(V)}}"', dashed, from=2-1, to=1-2]
                \end{tikzcd}
            $$
        Subsequently, it is a routine (albeit tedious) check that $[-, -]_{\fraku(V)}$ is a Lie bracket given by:
            $$A^{[m]} \tensor B^{[n]} \mapsto \sum_{k \in \Z_{\geq 0}} \binom{m}{k} (A^{(k)} \cdot B)^{[m + n - k]}$$
        (see \cite[Theorem 4.1.2]{frenkel_ben_zvi_vertex_algebras_and_algebraic_curves}). We refer to $\fraku(V)$ as the \textbf{(uncompleted) enveloping Lie algebra} of $V$.

        As the name suggests, $\fraku(V)$ admits a natural completion, namely with respect to the $t$-adic topology on $V[t^{\pm 1}]$. Doing so gives rise to a topological Lie algebra $\tilde{\fraku}(V)$, called the \textbf{completed enveloping Lie algebra} of $V$, whose underlying topological vector space is $V(\!(t)\!)/\im \del_V$; the Lie bracket is given in the same way as the one on $\fraku(V)$ (again, see \cite[Theorem 4.1.2]{frenkel_ben_zvi_vertex_algebras_and_algebraic_curves}).

        \begin{remark}
            The assignment of (completed) enveloping Lie algebras to vertex algebras is furthermore functorial. Indeed, given a morphism of vertex algebras:
                $$\varphi: (V, v_0, T, Y) \to (V', v'_0, T', Y')$$
            one immediately gets an induced linear map:
                $$V[t^{\pm 1}]/\im \del_V \to V'[t^{\pm 1}]/\im \del_{V'}$$
            via extensions of scalars, and then one can check that this upgades to a Lie algebra homomorphism:
                $$\fraku(V) \to \fraku(V')$$
            $t$-adic completion is functorial also, so we get a homomorphism of topological Lie algebras:
                $$\tilde{\fraku}(V) \to \tilde{\fraku}(V')$$
            as well. One thus obtains functors $\fraku(-), \tilde{\fraku}(-)$ from the category of vertex algebras to those of Lie algebras and complete topological Lie algebras, respectively. 
        \end{remark}

        \begin{remark}
            Let $Z$ be an abelian group and suppose that $V$ is $Z$-graded, say:
                $$V := \bigoplus_{d \in Z} V_d$$
            This then induces a $Z$-grading on $\fraku(V)$ (and hence also on $\tilde{\fraku}(V)$), thanks to the fact that finite tensor products, quotients, and filtered limits commute with direct sums:
                $$\fraku(V) \cong \bigoplus_{d \in Z} V_d[t^{\pm 1}]/\im \del_{V_d} =: \bigoplus_{d \in Z} \fraku(V)_d$$
                $$\tilde{\fraku}(V) := \projlim_{n \geq 1} \fraku(V)/t^n \cong \bigoplus_{d \in Z} \projlim_{n \geq 1} \fraku(V)_d/t^n := \bigoplus_{d \in Z} \tilde{\fraku}(V)_d$$
            wherein $\del_{V_d} := T|_{V_d} \tensor \id_{V_d} + \id_{V_d} \tensor \del_t$.

            This fact is very useful when dealing with Lie algebras with loop realisations (which naturally carry some grading by some abelian groups), e.g. $\tilde{\g}_{\kappa}$, the Heisenberg algebra, or the Virasoro algebra, which are all $\Z$-graded.
        \end{remark}

    \subsection{Associative enveloping algebras of vertex algebras and Reconstruction}
        Now, given any Lie algebra, it is usually a good idea to consider its universal enveloping algebra, not least because the latter is associative and so notions such as modules and so on are defined depending on it. $\fraku(V)$ is no exception. Let us write:
            $$\frakU(V) := \rmU(\fraku(V))$$
        for its universal enveloping algebra. Since $\fraku(V)$ is spanned by elements of the form $A^{[n]}$ (where $A \in V$ and $n \in \Z$), the associative algebra $\frakU(V)$ is generated by the elements $A^{[n]}$. This implies that one can also consider the $t$-adic completion of $\frakU(V)$, which we shall denote by $\tilde{\frakU}(V)$, which is the same as:
            $$\tilde{\frakU}(V) := \projlim_{N \geq 1} \frakU(V)/\bigoplus_{n > N} \frakU(V) \cdot A^{[n]}$$
        Regardless, one obtains a functor $\tilde{\frakU}(-)$ going from the category of vertex algebras to that of complete associative algebras.