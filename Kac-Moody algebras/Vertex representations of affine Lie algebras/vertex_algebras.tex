\section{Vertex algebras}
    \subsection{State-field correspondences}
        \begin{definition}[Meromorphic algebras] \label{def: meromorphic_algebras}
            A \textbf{meromorphic algebra}\footnote{These things are apparently also known as \say{$z$-algebras}, but I would like a name that doesn't depend on a chocie of variable. Borcherd likes to describe vertex algebras as associative algebras with a kind of \say{singular} multiplication, but because \say{singular} already has an established meaning for commutative algebras, I opted for \say{meromorphic}. The locality axiom will also help us justify this choice.} is a triple $(V, Y_z, T)$ consisting of:
            \begin{itemize}
                \item a vector space $V$,
                \item a bilinear map:
                    $$Y_z: V \tensor_{\bbC} V \to V(\!(z)\!)$$
                usually called a \textbf{state-field correspondence}, given by:
                    $$Y_z(a \tensor b) := Y(a, z) \cdot b$$
                for all $a, b \in V$, where $Y(a, z) \in \Fld(V)$ is a field in the sense of definition \ref{def: quantum_fields} (and hence each $Y(a, z) \cdot b$ is indeed an element of $V(\!(z)\!)$), and
                \item a \textbf{translation operator} $T \in \End( V(\!(z)\!) )$ that is a derivation with respect to the product $Y_z$ in the sense that:
                    $$T \cdot Y_z(a \tensor b) = Y_z( T \cdot a \tensor b ) + Y_z( a \tensor T \cdot b )$$
                and also such that:
                    $$T \cdot Y(a, z) = [T, Y(a, z)] = \del_z Y(a, z)$$
                both for all $a, b \in V$ (and by $[T, Y(a, z)]$, we mean $\sum_{m \in \Z} [T, a^{(m)}] z^{-m - 1}$ if we set $Y(a, z) := \sum_{m \in \Z} a^{(m)} z^{-m - 1}$).
            \end{itemize}
        \end{definition}
        Equivalently, the data of a meromorphic algebra is that of a linear map (a state-field correpsondence):
            $$Y(-, z): V \to \End(V)[\![z^{\pm 1}]\!]$$
        along with a translation operator $T$.
        \begin{remark}
            If we set:
                $$Y(a, z) := \sum_{m \in \Z} a^{(m)} z^{-m - 1}$$
            for every $a \in V$, then note that:
                $$
                    \begin{aligned}
                        [T, Y(a, z)] & = \del_z Y(a, z)
                        \\
                        & = \del_z \sum_{m \in \Z} a^{(m)} z^{-m - 1}
                        \\
                        & = \sum_{m \in \Z} (-m - 1) a^{(m)} z^{-m - 2}
                        \\
                        & =: \sum_{m \in \Z} [T, a^{(m)}] z^{-m - 1}
                    \end{aligned}
                $$
            From this, we gather that:
                $$[T, a^{(m)}] = -m a^{(m - 1)}$$
            for all $m \in \Z$.
        \end{remark}

        \begin{definition}[Meromorphic units] \label{def: meromorphic_units}
            A \textbf{meromorphic unit} is an element $\ket{0}$ (often also referred to as a \textbf{vacuum vector}) of a meromorphic algebra $(V, Y_z, T)$ such that:
                $$Y_z(a) \cdot \ket{0}, Y_z( \ket{0}, z ) \in V[\![z]\!]$$
            and:
                $$Y_z( a \tensor \ket{0} ) \equiv Y_z( \ket{0} \tensor a ) \pmod{ zV[\![z]\!] }$$
            both for all $a \in V$.
        \end{definition}
        To justify this definition somewhat, let us remark that usually, vacuum vectors are usually highest-weight vectors in some highest-weight module $V$ of a Lie algebra of formal distributions $\a$ with a triangular decomposition $\a \cong \a^- \oplus \a^0 \oplus \a^+$, and such vacuum vectors are - by definition - annihilated by elements of $\a^+$, whose elements give rise to fields that are elements of $z^{-1}V[\![z^{-1}]\!]$. As such, we require that $Y_z(a) \cdot \ket{0} = Y(a, z) \cdot \ket{0}$ as well as $Y(\ket{0}, z)$ to lie entirely within $V[\![z]\!]$. To justify the second condition, observe that:
            $$Y_z( a \tensor \ket{0} ) - Y_z( \ket{0} \tensor a ) = Y(a, z) \cdot \ket{0} - Y(\ket{0}, z) \cdot a$$
        and then that the RHS is an element of $zV[\![z]\!]$ (as a consequence of the previous requirement that $Y_z(a) \cdot \ket{0}, Y_z( \ket{0}, z ) \in V[\![z]\!]$), since:
            $$Y(a, z)^{(0)} \cdot \ket{0} = Y(\ket{0})^{(0)} \cdot a \in V$$
        where $u(z)^{(m)} := \Res_z (z^m u(z))$ is the $m^{th}$ coefficient of a given formal distribution $u(z) := \sum_{m \in \Z} u^{(m)} z^{-m - 1}$. We thus require that $Y_z( a \tensor \ket{0} ) \equiv Y_z( \ket{0} \tensor a ) \pmod{ zV[\![z]\!] }$ to make sure that vacuum vectors are unique up to scalar multiplication.

        \begin{lemma}
            For any unital meromorphic algebra $(V, \ket{0}, Y_z, T)$ such that:
                $$T \cdot \ket{0} = 0$$
            we have that:
                $$Y(a, z) \cdot \ket{0} = \exp(zT) \cdot a$$
            for all $a \in V$.
        \end{lemma}
            \begin{proof}
                By definition, we have that:
                    $$Y_z( T \cdot a \tensor \ket{0} ) = \del_z Y(a, z) \cdot \ket{0}$$
                and from the assumption that $T \cdot \ket{0} = 0$, we get that:
                    $$T \cdot Y_z(a \tensor \ket{0}) = Y_z(T \cdot a \tensor \ket{0}) + Y_z(a \tensor 0) = Y_z(T \cdot a \tensor \ket{0})$$
                By combining the two observations, we see thus that:
                    $$T \cdot Y_z(a \tensor \ket{0}) = T \cdot Y(a, z) \cdot \ket{0} = \del_z Y(a, z) \cdot \ket{0}$$
                which tells us that $Y(a, z)$ is a solution to the ODE:
                    $$\del_z - T = 0$$
                It is well-known that this ODE admits a unique solution, that being $\exp(zT)$, so we indeed have that:
                    $$Y(a, z) \cdot \ket{0} = \exp(zT) \cdot a$$
                for all $a \in V$.
            \end{proof}
        As a sanity check, note that $\exp(zT) \cdot a$ is indeed an element of $V[\![z]\!]$.

        This lemma will be especially important later on when we attempt to explicitly construct examples of vertex algebras, as it tells us that should we somehow know abstractly of the existence of a meromorphic algebra structure such that $T \cdot \ket{0} = 0$ on a given vector space $V$, then we will immediately be able to specify the actions of $Y(a, z)$ on the vacuum vector $\ket{0}$ as $\exp(zT) \cdot a$, for all $a \in V$. When $V$ is a cyclic highest-weight module in particular, the vacuum vector will generate the entirety of $V$, and so knowing that $Y(a, z) \cdot \ket{0} = \exp(zT) \cdot a$ will be the first step in explicitly computing general actions $Y(a, z) \cdot b$. Additionally, we will see that for vertex algebras, the state-field correspondence such that $Y(a, z) \cdot \ket{0} = \exp(zT) \cdot a$ is actually unique (see theorem \ref{theorem: goddard_uniqueness_theorem}).

        \begin{corollary}[Vacuum fields] \label{coro: vacuum_fields}
            For any unital meromorphic algebra $(V, \ket{0}, Y_z, T)$ such that:
                $$T \cdot \ket{0} = 0$$
            we have that:
                $$Y(\ket{0}, z) \cdot a \equiv \id_V \cdot a \pmod{zV[\![z]\!]}$$
        \end{corollary}
            \begin{proof}
                Consider the following:
                    $$Y(\ket{0}, z) \cdot a \equiv Y(a, z) \cdot \ket{0} = \exp(zT) \cdot a \equiv \id_V \cdot a \pmod{zV[\![z]\!]}$$
            \end{proof}

        The following corollary will also be useful when we discuss the concept of conformal vectors.
        \begin{corollary} \label{coro: translation_operator_formula}
            For any unital meromorphic algebra $(V, \ket{0}, Y_z, T)$ such that:
                $$T \cdot \ket{0} = 0$$
            we have that:
                $$T \cdot a = a^{(-2)} \cdot \ket{0}$$
        \end{corollary}
            \begin{proof}
                Consider the following:
                    $$T \cdot a = \Res_z ( z^{-1} \exp(zT) \cdot a)  = \Res_z ( z^{-1} Y(a, z) \cdot \ket{0} ) = a^{(-2)} \cdot \ket{0}$$
            \end{proof}

        \begin{definition}[Field algebras and vertex algebras] \label{def: field_algebras_and_vertex_algebras}
            A \textbf{field algebra} is a unital meromorphic algebra $(V, \ket{0}, Y_z, T)$ such that for all $a, b \in V$, the field $Y(a, w) Y(b, z) \in V[\![w^{\pm 1}, z^{\pm 1}]\!]$ is local in the sense of definition \ref{def: locality}. A \textbf{vertex algebra} is a field algebra for which:
                $$T \cdot \ket{0} = 0$$
                $$Y(\ket{0}, z) = \id_V$$
                $$\forall a \in V: [ T, Y(a, z) ] = T \cdot Y(a, z)$$
        \end{definition}

    \subsection{Structure of vertex algebras}
        \begin{theorem}[Goddard's Uniqueness Theorem] \label{theorem: goddard_uniqueness_theorem}
            Let $(V, Y, \ket{0}, T)$ be a vertex algebra and consider an arbitrary field $B(z) \in \Fld(V)$ which is mutually local with all fields $Y(a, z) \in \Fld(V)$, for all $a \in V$. Suppose also that there exists some $b \in V$ such that:
                $$B(z) \cdot \ket{0} = \exp(zT) b$$
            Then:
                $$B(z) = Y(b, z)$$
        \end{theorem}
            \begin{proof}
                
            \end{proof}
        \begin{corollary}
            Let $(V, Y, \ket{0}, T)$ be a vertex algebra.
            \begin{enumerate}
                \item For any collection of vectors $a_1, ..., a_n \in V$ along with $j_1, ..., j_n \in \N$, we have that:
                    $$Y( a_1^{(j_1)} ... a_n^{j_n} \cdot \ket{0}, z ) = \order{ \del_z^{[-j_1 - 1]} Y(a_1, z) \cdot ... \cdot \del_z^{[-j_n - 1]} Y(a_n, z) }$$
                \item For any $a \in V$, we have that:
                    $$Y( T \cdot a, z ) = \del_z Y(a, z)$$
            \end{enumerate}
        \end{corollary}
            \begin{proof}
                
            \end{proof}

        \begin{theorem}[The Existence/Reconstruction Theorem] \label{theorem: vertex_algebra_reconstruction_theorem}

        \end{theorem}
            \begin{proof}
                
            \end{proof}

        \begin{theorem}[Borcherd's OPE formula] \label{theorem: borcherd_OPE_formula}

        \end{theorem}
            \begin{proof}
                
            \end{proof}

    \subsection{(Completed) enveloping Lie algebras of vertex algebras}
        Suppose throughout that $(V, v_0, T, Y)$ be a vertex algebra.
    
        Consider the following operator on the vector space $V[t^{\pm 1}] := V \tensor_{\bbC} \bbC[t^{\pm 1}]$:
            $$\del_V := T \tensor \id_V + \id_V \tensor \del_t$$
        To the vertex algebra $V$, one can then associate the following vector space
            $$\fraku(V) := V[t^{\pm 1}]/\im \del_V$$
        By construction, this is spanned by the elements of the kind $A^{[m]} := A \tensor t^m$ modulo the subspace spanned by the relations:
            $$0 = \del_V \cdot A^{[m]} = (T \cdot A) \tensor t^m + A \tensor m t^{m - 1} = (T \cdot A)^{[m]} + m A^{[m - 1]}$$
        or equivalently:
            $$(T \cdot A)^{[m]} = -m A^{[m - 1]}$$
        which holds for all $A \in V$. Given some $A^{[m]}$, let us denote its lift modulo $\im \del_V$ by $A^{(m)}$ (which, to be thorough, we shall note to be an element of $V[t^{\pm 1}]$). Now, by the Translation Axiom in the definition of vertex algebras, we also have the following similar relations in $\End(V)$:
            $$(T \cdot A)^{(m)} = -m A^{(m - 1)}$$
        holding for all $A \in V$. This suggests to us that there exists a well-defined linear map:
            $$\fraku(V) \to \End(V)$$
            $$A^{[m]} \mapsto A^{(m)}$$
        This allows us to define a bilinear map:
            $$\widetilde{[-, -]}_{\fraku(V)}: V[t^{\pm 1}]^{\tensor 2} \to \fraku(V)$$
            $$A^{(m)} \tensor B^{(n)} \mapsto \sum_{k \in \Z_{\geq 0}} \binom{m}{k} (A^{(k)} \cdot B)^{[m + n - k]}$$
        wherein $A^{(k)}$ is regarded as an element of $\End(V)$. One can then show that this map admits a factorisation:
            $$
                \begin{tikzcd}
                {V[t^{\pm 1}]^{\tensor 2}} & {\fraku(V)} \\
                {\fraku(V)}
                \arrow["{\widetilde{[-, -]}_{\fraku(V)}}", from=1-1, to=1-2]
                \arrow[two heads, from=1-1, to=2-1]
                \arrow["{[-, -]_{\fraku(V)}}"', dashed, from=2-1, to=1-2]
                \end{tikzcd}
            $$
        Subsequently, it is a routine (albeit tedious) check that $[-, -]_{\fraku(V)}$ is a Lie bracket given by:
            $$A^{[m]} \tensor B^{[n]} \mapsto \sum_{k \in \Z_{\geq 0}} \binom{m}{k} (A^{(k)} \cdot B)^{[m + n - k]}$$
        (see \cite[Theorem 4.1.2]{frenkel_ben_zvi_vertex_algebras_and_algebraic_curves}). We refer to $\fraku(V)$ as the \textbf{(uncompleted) enveloping Lie algebra} of $V$.

        As the name suggests, $\fraku(V)$ admits a natural completion, namely with respect to the $t$-adic topology on $V[t^{\pm 1}]$. Doing so gives rise to a topological Lie algebra $\tilde{\fraku}(V)$, called the \textbf{completed enveloping Lie algebra} of $V$, whose underlying topological vector space is $V(\!(t)\!)/\im \del_V$; the Lie bracket is given in the same way as the one on $\fraku(V)$ (again, see \cite[Theorem 4.1.2]{frenkel_ben_zvi_vertex_algebras_and_algebraic_curves}).

        \begin{remark}
            The assignment of (completed) enveloping Lie algebras to vertex algebras is furthermore functorial. Indeed, given a morphism of vertex algebras:
                $$\varphi: (V, v_0, T, Y) \to (V', v'_0, T', Y')$$
            one immediately gets an induced linear map:
                $$V[t^{\pm 1}]/\im \del_V \to V'[t^{\pm 1}]/\im \del_{V'}$$
            via extensions of scalars, and then one can check that this upgades to a Lie algebra homomorphism:
                $$\fraku(V) \to \fraku(V')$$
            $t$-adic completion is functorial also, so we get a homomorphism of topological Lie algebras:
                $$\tilde{\fraku}(V) \to \tilde{\fraku}(V')$$
            as well. One thus obtains functors $\fraku(-), \tilde{\fraku}(-)$ from the category of vertex algebras to those of Lie algebras and complete topological Lie algebras, respectively. 
        \end{remark}

        \begin{remark}
            Let $Z$ be an abelian group and suppose that $V$ is $Z$-graded, say:
                $$V := \bigoplus_{d \in Z} V_d$$
            This then induces a $Z$-grading on $\fraku(V)$ (and hence also on $\tilde{\fraku}(V)$), thanks to the fact that finite tensor products, quotients, and filtered limits commute with direct sums:
                $$\fraku(V) \cong \bigoplus_{d \in Z} V_d[t^{\pm 1}]/\im \del_{V_d} =: \bigoplus_{d \in Z} \fraku(V)_d$$
                $$\tilde{\fraku}(V) := \projlim_{n \geq 1} \fraku(V)/t^n \cong \bigoplus_{d \in Z} \projlim_{n \geq 1} \fraku(V)_d/t^n := \bigoplus_{d \in Z} \tilde{\fraku}(V)_d$$
            wherein $\del_{V_d} := T|_{V_d} \tensor \id_{V_d} + \id_{V_d} \tensor \del_t$.

            This fact is very useful when dealing with Lie algebras with loop realisations (which naturally carry some grading by some abelian groups), e.g. $\tilde{\g}_{\kappa}$, the Heisenberg algebra, or the Virasoro algebra, which are all $\Z$-graded.
        \end{remark}

    \subsection{Associative enveloping algebras of vertex algebras and Reconstruction}
        Now, given any Lie algebra, it is usually a good idea to consider its universal enveloping algebra, not least because the latter is associative and so notions such as modules and so on are defined depending on it. $\fraku(V)$ is no exception. Let us write:
            $$\frakU(V) := \rmU(\fraku(V))$$
        for its universal enveloping algebra. Since $\fraku(V)$ is spanned by elements of the form $A^{[n]}$ (where $A \in V$ and $n \in \Z$), the associative algebra $\frakU(V)$ is generated by the elements $A^{[n]}$. This implies that one can also consider the $t$-adic completion of $\frakU(V)$, which we shall denote by $\tilde{\frakU}(V)$, which is the same as:
            $$\tilde{\frakU}(V) := \projlim_{N \geq 1} \frakU(V)/\bigoplus_{n > N} \frakU(V) \cdot A^{[n]}$$
        Regardless, one obtains a functor $\tilde{\frakU}(-)$ going from the category of vertex algebras to that of complete associative algebras.