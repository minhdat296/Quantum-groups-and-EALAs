\section{Modules over vertex algebras and examples}
    \subsection{Modules over vertex algebras}

    \subsection{Heisenberg algebras, lattice vertex algebras, and the boson-fermion correspondence}
        \begin{definition}[Lattices] \label{def: lattices}
            A \textbf{lattice} shall be a finite free $\Z$-module $L$ equipped with a symmetric $\Z$-bilinear form:
                $$(-, -): L \tensor_{\Z} L \to \Z$$
            that is \textit{positive-definite}, i.e. for all $\lambda \in L \setminus \{0\}$, one has that $(\lambda, \lambda) > 0$.
        \end{definition}
        \begin{example}
            The root lattice $Q$ of the finite-dimensional simple Lie algebra $\g$, on which the bilinear form is induced by the level $\kappa$, satisfies definition \ref{def: lattices}. In this case, recall also that the Cartan subalgebra $\h$ is isomorphic to the dual of the complexification of $Q$, i.e.:
                $$\h \cong (Q \tensor_{\Z} \bbC)^*$$
            because $\h$ admits the set of simple roots as a basis.
        \end{example}

        \begin{definition}[Heisenberg algebras associated to lattices] \label{def: lattice_heisenberg_algebras} 
            Given a lattice $(L, (-, -))$, set $\h_L := (L \tensor_{\Z} \bbC)^*$, and let us regard this as an abelian Lie algebra. The \textbf{Heisenberg algebra} associated to said lattice, denoted by $\tilde{\h}_L$, is then the central extension:
                $$0 \to \bbC c_L \to \tilde{\h}_L \to \h_L[v^{\pm}] \to 0$$
            whose Lie bracket is given by:
                $$[ h f, h' g ]_{\tilde{\h}_L} := (h, h') \Res_{v = 0} g df \cdot c_L$$
            for all $h, h' \in \h_L$ and all $f, g \in \bbC[v^{\pm 1}]$; equivalently, the Lie bracket is given by:
                $$[h v^m, h' v^n]_{\tilde{\h}_L} := (h, h') \delta_{m + n, 0} n \cdot c_L$$
        \end{definition}
        \begin{definition}[Weyl algebras associated to lattices] \label{def: lattice_weyl_algebras}
            Let $L$ be a lattice. Next, let us write $\bar{\scrH}_L := \rmU(\tilde{\h}_L)/(c_L - 1)$. Then, we can define the \textbf{Weyl algebra} associated to the given lattice $L$ - which shall be denoted by $\tilde{\scrH}_L$ - to be the $v$-adic completion:
                $$\tilde{\scrH}_L := \projlim_{m \geq 1} \bar{\scrH}_L/\bar{\scrH}_L \cdot v^n$$
        \end{definition}

        It is clear that the Heisenberg algebra $\tilde{\h}_L$ admits a triangular decomposition of sort, as follows, which is nothing but the $\Z$-grading on $\tilde{\h}_L$:
            $$\tilde{\h}_L \cong \tilde{\h}_L^- \oplus \tilde{\h}_L^0 \oplus \tilde{\h}_L^+$$
        where:
            $$\tilde{\h}_L^{\pm} := \bigoplus_{m \in \pm \Z_{> 0}} \h_L v^m, \tilde{\h}_L^0 := \h_L \oplus \bbC c_L$$
        This, in turn, induces a triangular decomposition for $\tilde{\scrH}_L$, and we will be interested in the upper \say{Borel subalgebra} $\tilde{\scrH}_L^{\geq 0}$, which is the quotient by $c_L - 1$ of the $v$-adically completed universal enveloping algebra of $\bigoplus_{m \in \Z_{\geq 0}} \h_L v^m \oplus \bbC c_L$. This allows us to define the \textbf{vacuum module} of highest weight $\lambda \in \h_L^*$:
            $$\V_L^{\lambda} := \tilde{\scrH}_L \tensor_{\tilde{\scrH}_L^{\geq 0}} \bbC c_L$$
            
        A standard argument shows that, as a vector space, $\V_L^{\lambda}$ is isomorphic to the underlying vector space of $\rmU(v^{-1}\h_L[v^{-1}])$, which itself is isomorphic to the vector space:
            $$\rmU(v^{-1}\h_L[v^{-1}]) \cdot \ket{\lambda} \cong \Sym( v^{-1}\h_L[v^{-1}] ) \cdot \ket{\lambda}$$
        as $v^{-1}\h_L[v^{-1}]$ is an abelian Lie subalgebra of $\tilde{\h}_L$. One can also explicitly compute a basis for the vector space $\Sym( v^{-1}\h_L[v^{-1}] )$, namely:
            $$\Sym( v^{-1}\h_L[v^{-1}] ) \cong \bbC[ \{ h_{i, m} \}_{(i, m) \in I \x \Z_{< 0}} ]$$
        should we choose a basis $\{\alpha_i\}_{i \in I}$ for the lattice $L$. Explicitly, the isomorphism is given by:
            $$h_i^{(m)} \mapsto h_{i, m}$$
        \begin{remark}
            It is useful to keep in mind, for later, that when $L$ is of rank $1$ (i.e. $L \cong \Z$), the underlying vector space of $\V_L^{\lambda}$ shall be isomorphic to $\bbC[\{h_{1, m}\}_{m \in \Z_{< 0}}]$ (say, we are fixing a basis $L := \Z \alpha_1$). In \cite{frenkel_ben_zvi_vertex_algebras_and_algebraic_curves}, the authors chose the identification $\bbC[\{h_{1, m}\}_{m \in \Z_{< 0}}] \cong \bbC[b_{-1}, b_{-2}, ...]$.
        \end{remark} 
        
       In any event, the underlying vector space of $\V_L^{\lambda}$ is then acted on by $\tilde{\scrH}_L$ (from the left!) in the following manner: 
            $$\tilde{\h}_L^+ \cdot \ket{\lambda} = 0$$
            $$\forall h \in \h_L h \cdot \ket{\lambda} = \lambda(h) \ket{\lambda}$$
        Note also that there is an evident $\Z_{\geq 0}$-grading on $\V_L^{\lambda}$, given by:
            $$\deg \ket{\lambda} := 0$$
            $$\forall (i, m) \in I \x \Z_{< 0}: \deg h_{i, m} \cdot \ket{\lambda} := -m$$
        This is important for showing that:
        \begin{lemma}[$\V_L^{\lambda}$ is simple] \label{lemma: weyl_vacuum_modules_are_simple}
            For any $\lambda \in \h_L^*$, the left-$\tilde{\scrH}_L$-module $\V_L^{\lambda}$ is simple.
        \end{lemma}
            \begin{proof}
                Suppose $U$ is a non-zero left-$\tilde{\scrH}_L$-submodule of $\V_L^{\lambda}$, which \textit{may or may not be proper}. By construction, $\V_L^{\lambda}$ is cyclic, and so $U$ is also generated by the maximal vector $\ket{\lambda}$, and hence in particular, also $\Z_{\geq 0}$-graded with the degree-$0$ component being $\bbC \ket{\lambda}$. Now, because $\tilde{\h}_L^-$ acts freely via left-multiplication by the elements of $\{ h_{i, m} \}_{(i, m) \in I \x \Z_{< 0}}$, one has that:
                    $$\deg( h_{i, m} \cdot u ) = -m + \deg u$$
                This implies that for $U$ to be a well-defined left-$\tilde{\scrH}_L$-module, we must have that:
                    $$U \cong \bigoplus_{n \in \Z_{\geq 0}} U_n$$
                in which:
                    $$\forall n \in \Z_{\geq 0}: U_n \not = 0$$
                Now, by the general theory of graded modules, we know that:
                    $$U_n = U \cap (\V_L^{\lambda})_n$$
                We also know that the underlying vector space of $\V_L^{\lambda}$ is $\bbC[\{ h_{i, m} \}_{(i, m) \in I \x \Z_{< 0}}] \cdot \ket{\lambda}$, the degree-$n$ components $(\V_L^{\lambda})_n$ are nothing but spans of all (commutative) monomials of the kind:
                    $$h_{i_r, m_r} ... h_{i_1, m_1}$$
                wherein the indices $i_r, ..., i_1 \in I$ are arbitrary and the indices $m_r \geq ... \geq m_1$ make up an ordered partition of $n$, i.e.:
                    $$m_r + ... + m_1 = n$$
                (note that these monomials are linearly independent from one another). Thus, we see that each $U_n$ is spanned by certain monomials of said kind. However, given any ordered partition $m_r + ... + m_1 = n$ of some $n \geq 0$ corresponding to some monomial $h_{i_r, m_r} ... h_{i_1, m_1}$ of degree $n$, one can always construct a new monomial $h_{i_{r + 1}, m_{r + 1}} h_{i_r, m_r} ... h_{i_1, m_1}$ of degree $-m_{r + 1} + n$ for some $m_{r + 1} < 0$. This implies that for every $n \geq 0$, one actually have that:
                    $$U_n = (\V_L^{\lambda})_n$$
                i.e. $U$ being a non-zero submodule automatically implies that it is the entirety of $\V_L^{\lambda}$. In other words, $\V_L^{\lambda}$ admits no non-zero proper submodule, and is therefore simple.
            \end{proof}
        \begin{lemma}
            Any $\tilde{\h}_L$-module, say $V$, on which the central charge $c_L$ acts as a scalar is necessarily infinite-dimensional, unless said scalar is $0$.
        \end{lemma}
            \begin{proof}
                Indeed, if we were to suppose for the sake of deriving a contradiction that $\dim V < +\infty$ yet $c_L$ acts as a non-zero scalar, then because:
                    $$[A, B]_{\tilde{\h}_L} \in \bbC c_L$$
                and because:
                    $$\trace([A, B]_{\tilde{\h}_L}) = \trace(AB) - \trace(BA) = 0$$
                (with both statements holding for all $A, B \in \tilde{\h}_L$), we will clearly get a contradiction. Note the subtlety that $\trace(AB) = \trace(BA)$ is an identity depending crucially on the assumption that $\dim V < +\infty$.
            \end{proof}
        \begin{corollary}
            $\tilde{\scrH}_L$-modules are necessarily infinite-dimensional. 
        \end{corollary}
        \begin{proposition}[Classification of $\Z_{\geq 0}$-graded simple left-$\tilde{\scrH}_L$-modules of countable dimensions] \label{prop: simple_lattice_weyl_modules_are_vacuum_modules}
            Any $\Z_{\geq 0}$-graded simple left-$\tilde{\scrH}_L$-module of some countable dimension is isomorphic to some $\V_L^{\lambda}$, for some $\lambda \in \h_L^*$.
        \end{proposition}
            \begin{proof}
                Since we know that $\tilde{\scrH}_L$-modules are necessarily infinite-dimensional, if $V := \bigoplus_{n \in \Z_{\geq 0}} V_n$ is a $\Z_{\geq 0}$-graded $\tilde{\scrH}_L$-module then because $\dim V$ is assumed to be countable, the graded components will be such that $V_n \not = 0$ for all $n \in \Z_{\geq 0}$. Without any loss of generality, we can pick the grading on $V$ so that $V_0$ is a $\tilde{\scrH}_L^0$-submodule of $V$. Furthermore, the assumption that $V$, as a $\tilde{\scrH}_L$-module, is only strictly $\Z_{\geq 0}$-graded instead of being $\Z$-graded, implies that $\tilde{\scrH}_L^+ \cdot V_0 = 0$.

                Let us now show that $\dim V_0 = 1$ if $V$ is furthermore simple to see that as a $\tilde{\scrH}_L$-module, $V$ is generated by a single highest-weight vector. To this end, suppose for the sake of deriving a contradiction that there exists a proper $\tilde{\scrH}_L^0$-submodule $U_0$ of $V_0$. This assumption, however, implies that $U := \tilde{\scrH}_L \cdot U_0$ shall be a proper $\tilde{\scrH}_L$-submodule of $V$. As $V$ is simple, we thus infer that $U = 0$, which in turn implies that the only proper $\tilde{\scrH}_L^0$-submodule of $V_0$ is $0$. $\tilde{\scrH}_L^0$, by construction, is a commutative algebra, so $\tilde{\scrH}_L^0$-submodules are nothing but vector subspaces, and hence the only proper vector subspace of $V_0$ is $0$. In other words, $\dim V_0 = 1$ necessarily, and hence we are done.

                Now, denote the highest-weight vector of $V$ by $\ket{\lambda}$, i.e. we set $V_0 := \bbC \cdot \ket{\lambda}$. It remains to determine the eigenvalue(s) of $\ket{\lambda}$ under actions of elements of $\h_L$. Because $V$ is simple as a $\tilde{\scrH}_L$-module, and because $\h_L$ is an abelian Lie algebra by definition - and hence $\rmU(\h_L)$ is a commutative subalgebra of $\tilde{\scrH}_L$ - its image under the composition:
                    $$\rmU(\h_L) \to \tilde{\scrH}_L \to \End(V)$$
                is necessarily $1$-dimensional, per Schur's Lemma (note that we need the ground field to be algebraically closed for this, which $\bbC$ is). Elements $h \in \h_L$ acting on $\ket{\lambda}$ thus eigenvalues $\lambda(h)$, for some common $\lambda \in \h_L^*$. 

                We have therefore proven that $V$ is a simple $\tilde{\scrH}_L$-module generated by a single vector $\ket{\lambda}$, for which there exists some $\lambda \in \h_L^*$ such that:
                    $$\tilde{\h}_L^+ \cdot \ket{\lambda} = 0$$
                    $$\forall h \in \h_L \cdot \ket{\lambda} = \lambda(h) \ket{\lambda}$$
                Thus, $V \cong \V_L^{\lambda}$ for some $\lambda \in \h_L^*$.
            \end{proof}

        Even though proposition \ref{prop: simple_lattice_weyl_modules_are_vacuum_modules} does not give a complete classification of $\tilde{\scrH}_L$-modules, it is sufficiently useful for our purposes, as it guarantees that \textit{any $\tilde{\scrH}_L$-module with underlying vector space satisfying the assumptions of the Reconstruction Theorem is necessary some direct sums of the modules $\V_L^{\lambda}$.} We therefore begin our study of vertex algebras associated to lattices by studying the vertex algebra structures on the modules $\V_L^{\lambda}$ granted to us by the Reconstruction Theorem. 
        \begin{example}[The bosonic Fock module] \label{example: bosonic_fock_modules}
            A very important class of examples of countably dimensional $\Z_{\geq 0}$-graded simple left-$\tilde{\scrH}_L$-modules is that of \textbf{(bosonic\footnote{We will return to this point later when we discuss the boson-fermion correspondence.}) Fock modules} (or \textbf{bosonic Fock spaces}, to use the physicists' term) of highest weight $\lambda \in \h_L^*$. Before discussing the details surrounding these representations of $\tilde{\scrH}_L$, let us fix a basis:
                $$\{\alpha_i\}_{i \in I}$$
            for $L$. The corresponding basis for $\h_L := (L \tensor_{\Z} \bbC)^*$ shall then be denoted by:
                $$\{h_i\}_{i \in I}$$
            
            Let us denote said $\tilde{\scrH}_L$-modules by $\scrB_L^{\lambda}$ ($\scrB$ for \say{bosonic}). Firstly, let us define the underlying vector spaces of these modules to be:
                $$\bbC[ \{h_{i, m}\}_{(i, m) \in I \x \Z_{< 0}} ] \cdot \ket{\lambda}$$
            much like the underlying vector spaces of the vacuum modules $\V_L^{\lambda}$. In particular, this means that $\dim \scrB_L^{\lambda}$ is countable and that the underlying vector space of each $\scrB_L^{\lambda}$ is $\Z_{\geq 0}$-graded, in the same way that $\V_L^{\lambda}$ is $\Z_{\geq 0}$-graded. The $\tilde{\scrH}_L$-action is given by:
                $$
                    h_i^{(m)} := h_i v^m \mapsto
                    \begin{cases}
                        \text{$h_{i, -m} \cdot$ if $m > 0$}
                        \\
                        \text{$-m \frac{\del}{\del h_{i, m}}$ if $m < 0$}
                        \\
                        \text{$\lambda(\alpha_i) \cdot$ if $m = 0$}
                    \end{cases}
                $$
            and clearly, this action is $\Z_{\geq 0}$-graded and of highest weight $\lambda \in \h_L^*$. Therefore, the only thing to prove in order to demonstrate that:
                $$\scrB_L^{\lambda} \cong \V_L^{\lambda}$$
            as $\tilde{\scrH}_L$-modules, is that $\scrB_L^{\lambda}$ is simple. Suppose for the sake of deriving a contradiction that $\scrB_L^{\lambda}$ does admit a non-zero proper $\tilde{\scrH}_L$-submodule, say $U$. It is evident that $U$ is spanned by elements of the form:
                $$f \ket{\lambda}$$
            for some:
                $$f \in \bbC[ \{h_{i, m}\}_{(i, m) \in I \x \Z_{< 0}} ]$$
            If $f$ is a non-constant polynomial, one can then find some:
                $$\del := (-1)^r \prod_{m = 1}^r m \frac{\del}{\del h_{i, m}} \in \Diff(\bbC[ \{h_{i, m}\}_{(i, m) \in I \x \Z_{< 0}} ])$$
            such that $\del f \in \bbC[ \{h_{i, m}\}_{(i, m) \in I \x \Z_{< 0}} ]$ is constant and \textit{non-zero}, and as such $(\del f) \cdot \ket{\lambda} \in (\scrB_L^{\lambda})_0$. But $U$ is $\tilde{\scrH}_L$-stable by assumption, and since $\scrB_L^{\lambda}$ is generated by degree-$0$ elements, the previous statement implies that $(\del f) \cdot \ket{\lambda} \in U$ actually generates all of $\scrB_L^{\lambda}$. We therefore have a contradiction, and hence $\scrB_L^{\lambda}$ is simple over $\tilde{\scrH}_L$. Per proposition \ref{prop: simple_lattice_weyl_modules_are_vacuum_modules}, we thus have that:
                $$\scrB_L^{\lambda} \cong \V_L^{\lambda}$$
        \end{example}

        The existence of the Fock module of $\tilde{\scrH}_L$ thus suggests to us that there ought to be a way to represent $\tilde{\scrH}_L$ on $\Diff(\bbC[ \{h_{i, m}\}_{(i, m) \in I \x \Z_{< 0}} ])$, the algebra of $\bbC$-linear differential operators in the variables $h_{i, m}$. However, to be able to realise this vision, we will have to understand how to explicitly write down products of such differential operators. Ideally, we would like to realise $\tilde{\scrH}_L$ as a Lie subalgebra of $\Diff(\bbC[ \{h_{i, m}\}_{(i, m) \in I \x \Z_{< 0}} ])$. This is the point at which it is natural to bring in the toolkit of vertex operators and vertex algebras. Doing so will simultaneous be an exercise in constructing a non-trivial vertex algebra structure.

        Let us firstly tackle the case $L \cong \Z$, i.e. the rank-$1$ case, and let us say right away that even in this highly specialised case, one can already derive some very deep and interesting consequences from the construction of so-called \say{bosonic vertex operators} (see theorem \ref{theorem: rank_1_bosonic_vertex_operators}), e.g. the boson-fermion correspondence (see theorem \ref{theorem: boson_fermion_correspondence}).
        
        For what follows, let us fix some $N \in \Z_{> 0}$ and then consider, firstly, the rank-$1$ lattice:
            $$L := \sqrt{N} \Z$$
        equipped with the non-degenerate $\Z$-bilinear form given by:
            $$(\lambda, \mu) := \lambda \mu$$
        for all $\lambda, \mu \in \sqrt{N} \Z$. Note that in this case, we have that $\h_L := (\sqrt{N}\Z \tensor_{\Z} \bbC)^* \cong \bbC$, and therefore $\h_L^* \cong \bbC^*$. Note also, that:
            $$(\lambda, \mu) \in N \Z$$
        and so (in the sense of definition \ref{def: even_and_odd_lattices}), \textit{the lattice $\sqrt{N} \Z$ is even/odd if and only if $N$ is even/odd respectively}. 
        \begin{remark}
            Since the root lattice of $\sl_2$ is isomorphic to $\sqrt{N} \Z$ for some choice of $N \in \Z_{> 0}$ and since weights of $\sl_2$ are nothing but elements of $\bbC^*$, the construction of a vertex algebra structure on $\V_{\sqrt{N} \Z}^{\lambda}$ (for some $\lambda \in \bbC^*$) can be seen as a step towards constructing a vertex algebra structure on the vacuum module of $\sl_2$ of highest weight $\lambda$ \textit{\`a la} Frenkel-Kac-Segal. When dealing with finite-dimensional simple Lie algebras $\g$, say with root space decomposition $\g \cong \h \oplus \bigoplus_{\alpha \in \Phi} \g_{\alpha}$, more general than $\sl_2$, different choices of $N$ will correspond to different choices of isomorphisms:
                $$\sl_2 \xrightarrow[]{\cong} \bbC x_{-\alpha} \oplus \bbC \check{\alpha} \oplus \bbC x_{\alpha}$$
            wherein:
                $$\alpha \in \Phi^+, \height(\alpha) = N$$
            and $x_{\pm \alpha} \in \g_{\pm \alpha}$ are root vectors such that $[x_{\alpha}, x_{-\alpha}] = \check{\alpha}$.
        \end{remark}

        For the vertex algebra structure on each $\V_{\sqrt{N} \Z}^{\lambda}$ that will be constructed in proposition \ref{prop: VA_structures_on_rank_1_weyl_vacuum_modules} to make sense, let us firstly discuss some bosonic string theory, from where these vertex operators originate. For a moment, let us for a moment return to the setting of a general lattice $L$ instead of focusing only on $L := \sqrt{N} \Z$. The rank of $L$ as a finite free $\Z$-module will be thought of as the dimension of the spacetime that our bosonic string theory shall be taking place in. 

        The construction of the bosonic Fock modules $\scrB_L^{\lambda} \cong \V_L^{\lambda}$ as in example \ref{example: bosonic_fock_modules} suggests to us that (for every $i \in I$) when $m < 0$, the operator $h_i^{(m)} \in \End(\V_L^{\lambda})$ ought to be thought of as a \say{creation} operator, while when $m > 0$, $h_i^{(m)}$ ought to be thought of instead as an \say{annihilation operator}. The field:
            $$\bfH_i(z) := h_i(z) := \sum_{m \in \Z} h_i^{(m)} z^{-m - 1} \in \bbC h_i[v^{\pm 1}][\![z^{\pm 1}]\!]$$
        can then be thought of a field of momenta. We can subsequently define the following field of positions:
            $$
                \begin{aligned}
                    \bfX_i(z) & := -\int \bfH_i(z) dz
                    \\
                    & = -\int h_i^{(0)} z^{-1} dz + \sum_{m \in \Z \setminus \{0\}} \frac1m h_i^{(m)} z^{-m}
                    \\
                    & = -h_i^{(0)} \log(z) + \sum_{m \in \Z \setminus \{0\}} \frac1m h_i^{(m)} z^{-m}
                    \\
                    & = -h_i \log(z) + \sum_{m \in \Z \setminus \{0\}} \frac1m h_i^{(m)} z^{-m} \in \bbC h_i[v^{\pm 1}][\![z^{\pm 1}]\!]
                \end{aligned}
            $$
        wherein both the integrals and the expression $\log(z)$ are understood in the formal sense. The field $\bfX_i(z)$ is to be interpreted as the field of positions of our bosonic string within our $\rank L$-dimensional spacetime, and these positions depend on coordinates $z$ of the worldsheet that the string sweeps out.
        
        Let us now write:
            $$\bfH_i(z) := \sum_{m \in \Z} \bfH_i^{(m)} z^{-m - 1}, \bfX_i(z) := \sum_{m \in \Z} \bfX_i^{(m)} z^{-m - 1}$$
        for the Fourier expansion of $\bfH_i(z)$ and $\bfX_i(z)$; by construction, we have that $\bfH_i^{(m)} = h_i^{(m)}$, and let us set $\bfX_i^{(m)} := \xi_i v^m$. Let us also write:
            $$\bfX(z) := ( \bfX_i(z) )_{i \in I} \in \h_L[\![z^{\pm 1}]\!]$$
        Then, let us consider the following \textbf{vertex operator}, for each $\lambda \in \h_L^*$:
            $$\Gamma_{\lambda}(z) := \order{ \exp( \lambda \cdot \bfX(z) ) } \in \End(\V_L^{\lambda})[\![z^{\pm 1}]\!]$$
        wherein:
            $$\lambda \cdot \bfX(z) := \sum_{i \in I} \sum_{m \in \Z} \lambda_i(\xi_i) v^m z^{-m - 1} \in \End(\V_L^{\lambda})[\![z^{\pm 1}]\!]$$
        This vertex operator is modelled after the wave functions which are solutions to the Schr\"odinger's equations. The point here is that, in the event of string interactions - i.e. finite (normally ordered) products of the form, say:
            $$\order{ \Gamma_{\lambda_r}(z) \cdot ... \cdot \Gamma_{\lambda_1}(z) }$$
        - one can extract correlation functions out as matrix elements:
            $$\bra{w} \order{ \Gamma_{\lambda_r}(z) \cdot ... \cdot \Gamma_{\lambda_1}(z) } \ket{v}$$
        Finally, $\lambda$ is to be interpreted as a quantum number (e.g. mass, spin, charge, etc.) of the bosonic string in question.

        \begin{convention}
            For convenience, we will also be writing:
                $$\bfX_i^{\pm}(z) := \sum_{m \in \pm \Z_{> 0}} \frac1m h_i^{(m)} z^{-m}, \bfX_i^0(z) := -h_i \log(z)$$
        \end{convention}
        
        \begin{proposition}[Vertex algebra structures on rank-$1$ Weyl vacuum modules] \label{prop: VA_structures_on_rank_1_weyl_vacuum_modules}
            Let $N \in \Z_{> 0}$ and fix some $\lambda \in \bbC^*$. Denote the highest-weight vector of the $\tilde{\scrH}_{\sqrt{N} \Z}$-module $\V_{\sqrt{N} \Z}^{\lambda}$ by $\ket{\lambda}$. On $\V_{\sqrt{N} \Z}^{\lambda}$, there is a $\Z_{\geq 0}$-graded vertex algebra structure:
                $$Y_{\lambda}: \V_{\sqrt{N} \Z}^{\lambda} \to \End(\V_{\sqrt{N} \Z}^{\lambda})[\![z^{\pm 1}]\!]$$
            which is of conformal dimension $1$ and specified by:
                $$Y_{\lambda}(h_{1, -1} \cdot \ket{\lambda}, z) := \Gamma_{\lambda}(z) \cdot \ket{\lambda}$$
        \end{proposition}
            \begin{proof}
                Let us firstly explain why it is enough to only specify the vertex algebra structure on the highest-weight vector, which we denote by $\ket{\lambda}$.
                    
                Firstly, $\V_{\sqrt{N} \Z}^{\lambda}$ is isomorphic to the left-$\tilde{\scrH}_{\sqrt{N} \Z}$-module:
                    $$\tilde{\scrH}_{\sqrt{N} \Z}^- \cdot \ket{\lambda} \cong \bbC[ \{h_{1, m}\}_{m \x \Z_{< 0}} ] \cdot \ket{\lambda}$$
                generated by $b_{-1}$, which is mapped to $b_{\lambda}$ under the isomorphism. Secondly, since $\bbC[ \{h_{1, m}\}_{m \Z_{< 0}} ]$ is spanned by monomials of the form $h_{1, m_r} ... h_{1, m_1} \cdot \ket{\lambda}$ (where $m_r \geq ... \geq m_1$), and since:
                    $$Y_{\lambda}(h_{1, m_r} ... h_{1, m_1} \cdot \ket{\lambda}, z) = \order{ Y_{\lambda}(h_{1, m_r} \cdot \ket{\lambda}, z) ... Y_{\lambda}(h_{1, m_1} \cdot \ket{\lambda}, z) }$$
                it is enough to specify:
                    $$Y_{\lambda}(h_{1, -1} \cdot \ket{\lambda}, z) := Y(h_1^{(-1)}, z)$$

                Let us now describe $Y(h_{1, -1}, z)$. We would like $Y(-, z)$ to be a $\Z_{\geq 0}$-graded linear map, so because $\deg h_{1, -1} = 1$ (recall that $\deg h_{1, m} := -m$ for all $m < 0$), let us consider:
                    $$Y_{\lambda}(h_{1, -1} \cdot \ket{\lambda}, z) := h_1(z) := \sum_{m \in \Z} h_i^{(m)} z^{-m - 1}$$
                Let us then verify that, together with the translation operator $T \in \End(\V_{\sqrt{N}\Z}^{\lambda})$ given by:
                    $$[T, \ket{\lambda}] := 0$$
                    $$\forall (1, m) \in I \x \Z_{< 0}: [T, h_{1, m}] := -m h_{1, m - 1}$$
                and $\ket{\lambda}$ as the vacuum vector, the assignment $Y(-, z)$ forms a $\Z_{\geq 0}$-graded vertex algebra whenever $N$ is even; that this is of conformal dimension $1$ is automatic from the construction. To this end, let us check the axioms defining vertex algebras in sequence:
                \begin{itemize}
                    \item 
                    \item 
                    \item 
                \end{itemize}
            \end{proof}

        \begin{convention}[Group algebras]
            If $\Gamma$ is a group, say generated by a set $\{\gamma_i\}_{i \in I}$, then we will be denoting the basis elements of the group algebra $\bbC\Gamma$ by $\exp(\gamma_i)$.
        \end{convention}

        From now on, let us write:
            $$\scrB_{\sqrt{N}\Z} := \bigoplus_{\lambda \in \sqrt{N} \Z} \V_{\sqrt{N} \Z}^{\lambda}$$
        Keeping the notations from example \ref{example: bosonic_fock_modules}, let us fix the basis:
            $$\sqrt{N} \Z := \Z \alpha_1$$
        which induces:
            $$\h_{\sqrt{N} \Z} \cong \bbC h_1$$
        Since the underlying vector space of each $\V_{\sqrt{N} \Z}^{\lambda}$ is isomorphic to $\bbC[\{h_{1, m}\}_{m \in \Z_{< 0}}] \cdot \ket{\lambda}$, we note that the underlying vector space of $\scrB_{\sqrt{N} \Z}$ is isomorphic to:
            $$\bigoplus_{\lambda \in \sqrt{N} \Z} \bbC[\{h_{1, m}\}_{m \in \Z_{< 0}}] \cdot \ket{\lambda} \cong \bbC[\{h_{1, m}\}_{m \in \Z_{< 0}}] \tensor_{\bbC} \bbC \sqrt{N} \Z$$
        where $\bbC \sqrt{N} \Z$ is the group algebra of the rank-$1$ abelian group $\sqrt{N} \Z$. Some authors use the identification:
            $$\bbC[\{h_{1, m}\}_{m \in \Z_{< 0}}] \tensor_{\bbC} \bbC \sqrt{N} \Z \cong \bbC[\{h_{1, m}\}_{m \in \Z_{< 0}}, \exp(\pm \alpha_1)]$$
        
        \begin{theorem}[Rank-$1$ bosonic vertex operators] \label{theorem: rank_1_bosonic_vertex_operators}
            (Cf. \cite[Proposition 5.2.5]{frenkel_ben_zvi_vertex_algebras_and_algebraic_curves}) Let $N \in \Z_{> 0}$, and for each $\lambda \in \sqrt{N} \Z$, let $S_{\lambda} \in \End(\V_{\sqrt{N} \Z})$ be the operator given by:
                $$\forall \mu \in L: S_{\mu} \ket{\lambda} := \ket{\lambda + \mu}$$
            and then let us set:
                $$V_{\lambda}(z) := S_{\lambda} \Gamma_{\lambda}(z)$$
            (cf. \cite[Equation 5.2.8, p. 83]{frenkel_ben_zvi_vertex_algebras_and_algebraic_curves}).
            \begin{enumerate}
                \item If $N$ is even, there will be a vertex algebra structure on $\scrB_{\sqrt{N} \Z}$:
                    $$Y: \scrB_{\sqrt{N} \Z} \to \End(\scrB_{\sqrt{N} \Z})[\![z^{\pm 1}]\!]$$
                that is specified by:
                    $$\forall \lambda \in \sqrt{N} \Z: Y(h_{1, -1} \cdot \ket{\lambda} \tensor 1, z) := V_{\lambda}(z) \cdot \ket{\lambda}$$
                    $$Y( 1 \tensor \exp(\pm \alpha_1), z ) := \alpha_1(h_1) \cdot (1 \tensor \exp(\pm \alpha_1))$$
                \item On the other hand, if $N$ is odd then $\phi$ as above will endow $\scrB_{\sqrt{N} \Z}$ with a super vertex algebra structure.
            \end{enumerate}
        \end{theorem}
            \begin{proof}
                \begin{enumerate}
                    \item 
                    \item 
                \end{enumerate}
            \end{proof}

        The physically-minded reader may now wonder as to why there are no \say{fermionic} counterparts to the vertex operators given in theorem \ref{theorem: bosonic_vertex_operators}. This is surely strange, because after all, bosons are the force carriers through which fermions interact with one another, meaning that the two types of particles must go hand-in-hand. Of course, we do know this to be experimentally true, so the mathematical task is to somehow capture this in the language of vertex operators and vertex algebras. The result is known as the \textbf{boson-fermion correspondence}, and it takes the form of a vertex algebra isomorphism.

        \begin{convention}
            If $x, y$ are elements of an associative algebra, then we will be writing:
                $$\{x, y\} := xy + yx$$
        \end{convention}

        \todo[inline]{Fermions are described by super vertex algebras because of the Pauli Exclusion Principle, which postulates that half-integer spins, i.e. fermions, must be coupled.}

        \begin{definition}[Even/odd lattices] \label{def: even_and_odd_lattices}
            A lattice $(L, (-, -))$ is said to be \textbf{even} if and only if $(\lambda, \lambda) \in 2\Z$ for all $\lambda \in L$. Otherwise, it is said to be \textbf{odd}; note that this does \textit{not} mean that $(\lambda, \lambda) \not \in 2\Z$ for all $\lambda \in L$.
        \end{definition}
        \begin{definition}[The Clifford (super-)algebra associated to a lattice] \label{def: lattice_clifford_(super)_algebras}
             
        \end{definition}
        
        \begin{theorem}[Boson-fermion correspondence (and fermionic vertex operators)]  \label{theorem: boson_fermion_correspondence}
            When $N = 1$, there is a super vertex algebra isomorphism:
                $$\scrF_{\Z} \cong \scrB_{\Z}$$
        \end{theorem}
            \begin{proof}
                
            \end{proof}
                
        Let us now explain how the vertex operators constructed in theorem \ref{theorem: rank_1_bosonic_vertex_operators} can be adapted to the case of a general lattice $L$. We will then study the case where $L$ is the root lattice $Q$ of the finite-dimensional simple Lie algebra $\g$ in more details.

        \begin{proposition}[Vertex algebra structures on general Weyl vacuum modules] \label{prop: VA_structures_on_general_weyl_vacuum_modules}
            There is a $\Z_{\geq 0}$-graded vertex algebra structure:
                $$Y_{\lambda}: \V_L^{\lambda} \to \End( \V_L^{\lambda} )[\![z^{\pm 1}]\!]$$
            which is of conformal dimension $1$ and specified by:
                $$Y_{\lambda}( h_{i, -1} \cdot \ket{\lambda}, z ) := \Gamma_{\lambda}(z) \cdot \ket{\lambda}$$
        \end{proposition}
            \begin{proof}
                
            \end{proof}

        Like before, the construction of bosonic vertex operators attached to a general lattice $L := \bigoplus_{i \in I} \Z \alpha_i$ starts with the following enlargement of the Fock modules $\V_L^{\lambda}$:
            $$\scrB_L := \bigoplus_{\lambda \in L} \V_L^{\lambda}$$
        Since the underlying vector space of each $\V_L^{\lambda}$ is isomorphic to:
            $$\Sym( v^{-1}\h[v^{-1}] ) \cdot \ket{\lambda} \cong \bbC[ \{h_{i, m}\}_{(i, m) \in I \x \Z_{< 0}} ] \cdot \ket{\lambda}$$
        we consequently see that as the underlying vector space of $\scrB_L$ is given by:
            $$\bigoplus_{\lambda \in L} \bbC[ \{h_{i, m}\}_{(i, m) \in I \x \Z_{< 0}} ] \cdot \ket{\lambda} \cong \Sym( v^{-1}\h[v^{-1}] ) \tensor_{\bbC} \bbC L$$
        as a consequence of the group $L$ being abelian. The $\tilde{\scrH}_L$-action on each $\V_L^{\lambda}$ can then be extended to the whole direct sum $\scrB$ only by means of specifying an additional $\tilde{\scrH}_L$-action on $\bbC L$. Let us denote each of the former by $\pi_{\lambda}$ and the latter by $\rho$; the enlarged action will then be given by:
            $$\pi := \bigoplus_{\lambda \in L} \pi_{\lambda} \tensor 1 + 1 \tensor \rho$$
        Since $\tilde{\scrH}_L$ is topologically generated by $\tilde{\h}_L$ (by construction), it is enough to give the values of $\rho$ on elements of the basis $\{h_i^{(m)}\}_{(i, m) \in I \x \Z} \cup \{c_L\}$ of $\tilde{\h}_L$.

        Let us now define the action:
            $$\rho: \tilde{\scrH}_L \to \End( \bbC L )$$
        by letting it be given as follows:
            $$\rho( h_i^{(m)} ) \cdot \exp(\alpha_j) := \alpha_j(h_i) \delta_{m, 0} e^{\alpha_j} = \delta_{i, j} \delta_{m, 0} \exp(\alpha_j)$$
            $$\rho(c_L) := 0$$
        It is also important to note that the codomain of $\rho$ is actually $\der(\bbC L)$, and hence the codomain of $\pi$ shall lie inside $\Diff( \Sym( v^{-1}\h[v^{-1}] ) \tensor_{\bbC} \bbC L )$, so we will indeed be able to extend the vertex operator constructions on each Fock module $\V_L^{\lambda}$ to the entirety of $\V_L$.

        Next, let us recall that in theorem \ref{theorem: rank_1_bosonic_vertex_operators}, there is a disparity between when $N$ is even and when $N$ is odd. Namely, in the former case, we get a $\Z_{\geq 0}$-graded vertex algebra structure on $\scrB_{\sqrt{N} \Z}$, while in the latter case, we get an additional compatible $\Z/2$-grading on the vertex algebra structure, making it a $\Z_{\geq 0}$-graded vertex \textit{super}algebra structure. Because of this, let us introduce the following final ingredient for the formulation of theorem \ref{theorem: bosonic_vertex_operators} down below, which shall be able to detect the parity of $L$, i.e. whether it is an even or odd lattice. This is to be a group $2$-cocycle:
            $$\e \in Z^2_{\Grp}( L, \Z/2 )$$
        specified by:
            $$\e(\alpha, \alpha) = (-1)^{\frac12 (\alpha, \alpha)}$$
            $$\e(\alpha, \beta) = (-1)^{(\alpha, \beta)} \e(\beta, \alpha)$$
            $$\e(\alpha, 0) = \e(0, \alpha) = 1$$
        for all $\alpha, \beta \in L$; here, we have made the identification $\Z/2 \cong \{\pm 1\}$. Such a $2$-cocycle gives rise to a twisted semi-direct product:
            $$1 \to \Z/2 \to L \rtimes^{\e} \Z/2 \to L \to 1$$
        For convenience, let us abbreviate:
            $$L^{\e} := L \rtimes^{\e} \Z/2$$
        and for the sake of thoroughness, let us recall that the group multipliation of $L^{\e}$ is given by:
            $$(\alpha + \e_a) \cdot^{\e} (\beta + \e_b) := (\alpha + \beta) + \e(\alpha, \beta) \e_a \e_b$$
        for all $\alpha, \beta \in L$ and all $\e_a, \e_b \in \Z/2$; this means that in the group algebra $\bbC L^{\e}$, the multiplication is given by:
            $$\exp(\alpha + \e_a) \exp(\beta + \e_b) = \exp( (\alpha + \beta) + \e(\alpha, \beta) \e_a \e_b )$$
        Now, consider the following two-sided ideal of $\bbC L^{\e}$:
            $$J^{\e} := \sum_{\alpha \in L} \exp$$
        \todo[inline]{Finish up}

        \begin{theorem}[General bosonic vertex operators] \label{theorem: bosonic_vertex_operators}
            
        \end{theorem}
            \begin{proof}
                
            \end{proof}

    \subsection{Universal affine vertex algebras and the Frenkel-Kac-Segal (FKS) construction}
        All throughout, we work over the field of complex numbers $\bbC$, although any algebraically closed field of characteristic $0$ will suffice, as we only need this assumption so that certain operators will be diagonalisable. 

        We fix once and for all a finite-dimensional simple Lie algebra $\g$. Its Cartan matrix will be denoted by $C$, and we will use standard notations (cf. e.g. \cite{humphreys_lie_algebras} and \cite{kac_infinite_dimensional_lie_algebras}) for all of the other data that usually accompanies $\g$, e.g. a choice of simple roots, a subsequently defined root system, etc. We note that to specify these data, we will have to choose a non-degenerate and invariant symmetric bilinear form on $\g$, which shall be denoted by $\kappa$.

        The \textbf{affine Lie algebra} attached to the Lie $2$-cocycle $\kappa \in H^2_{\Lie}(\g[v^{\pm 1}], \bbC)$ shall be:
            $$\tilde{\g}_{\kappa} := \uce(\g[v^{\pm 1}])$$
        on which the Lie bracket is given by:
            $$[xf, yg]_{\tilde{\g}_{\kappa}} := [x, y]_{\g} fg + \kappa(x, y) g df$$
        Here, we have used a result by Kassel to identify:
            $$\tilde{\g}_{\kappa} \cong \g[v^{\pm 1}] \oplus \Omega^1_{\bbC[v^{\pm 1}]/\bbC}/d\bbC[v^{\pm 1}] \cong \g[v^{\pm 1}] \oplus \bbC c_{\kappa}$$
        (see \cite{kassel_universal_central_extensions_of_lie_algebras}). We also know, per a result of Garland (see \cite{garland_arithmetics_of_loop_groups}), that $\tilde{\g}_{\kappa}$ is - up to isomorphisms - the only central extension of $\g[v^{\pm 1}]$ (i.e. $H^2_{\Lie}(\g[v^{\pm 1}], \bbC)$); isomorphisms are given by rescaling $\kappa$. One also says that $\tilde{\g}_{\kappa}$ is the affine Lie algebra at \textbf{level} $\kappa$.

        The \textbf{untwisted affine Kac-Moody algebra} at level $\kappa$ (as in \cite[Chapter 7]{kac_infinite_dimensional_lie_algebras}) will then be:
            $$\hat{\g}_{\kappa} := \tilde{\g}_{\kappa} \rtimes \bbC D_{\kappa}$$
        where $D_{\kappa} \in \der(\tilde{\g}_{\kappa})$ can be shown to be equal to $v\frac{d}{dv}$. This algebra will be endowed with the non-degenerate and invariant symmetric bilinear form given by:
            $$(xf, yg)_{\hat{\g}_{\kappa}} := \kappa(x, y) \Res_v (g df)$$
            $$(c_{\kappa}, D_{\kappa})_{\hat{\g}_{\kappa}} := 1$$
        Using this bilinear form, one can construct the Cartan matrix $\hat{C}$ of $\hat{\g}_{\kappa}$ and related combinatorial data, like the affinisation of the root system of $\g$ and so on.
            
        Recall in particular that for $\hat{\g}$ (or indeed, for any Kac-Moody algebra) there is the \textbf{principal $\Z$-grading} of type $\vec{s} := (s_i)_{i \in \hat{\simpleroots}} \in \Z_{\geq 0} \cdot \hat{\simpleroots}$, given on the Chevalley-Serre generators by:
            $$\deg x_i^{\pm} := \pm s_i$$
            $$\deg h_i := 0$$
        for all $i \in \hat{\simpleroots}$. When $\vec{s} := \vec{1} := (1, ..., 1)$, we recover the usual root height grading. 
    
        \begin{convention}
            Suppose now that $c_{\kappa}$ acts as $1$ on standard modules $\standard^{\lambda}(\hat{\g})$ (and hence also as $1$ on finite-dimensional simple modules and in particular, fundamental modules). In other words, we are stipulating that $\standard^{\lambda}(\hat{\g})$ carries also the structure of a left-module over $\bar{\scrU}_{\kappa} := \rmU(\tilde{\g}_{\kappa})/\rmU(\tilde{\g}_{\kappa}) \cdot (c_{\kappa} - 1)$.
        \end{convention}

        \begin{remark}[Smooth modules ?]
            There exists also an analogue of $\tilde{\g}_{\kappa}$ constructed using $\bbC(\!(v)\!)$ in place of $\bbC[v^{\pm 1}]$. For a moment, let us denote the affine Kac-Moody algebra that we have been considering up until this point by $\tilde{\g}_{\kappa}^{\pol}$ and the other version by $\tilde{\g}_{\kappa}^{\loc}$. It can be easily shown that the categorry:
                $$\tilde{\g}_{\kappa}^{\pol}\mod$$
            is equivalent to the category:
                $$\tilde{\g}_{\kappa}^{\loc}\mod^{\smooth}$$
            of so-called \textbf{smooth modules} over $\tilde{\g}_{\kappa}^{\loc}$; said smooth modules are those $\tilde{\g}_{\kappa}^{\loc}$-modules on which the Lie subalgebra $\g[\![v]\!] \subset \tilde{\g}_{\kappa}^{\loc}$ acts locally nilpotently. Moreover, this equivalence maps standard modules to standard modules, and hence preserves simplicity of objects, etc. Furthermore, non-smooth modules will never be considered, so we will be using the same notation $\tilde{\g}_{\kappa}$ for both version. Clarifications will be provided if necessary.
        \end{remark}

        \begin{definition}
            The completed universal enveloping algebra of $\tilde{\g}_{\kappa}$ is given by:
                $$\tilde{\scrU}_{\kappa} := \projlim_{n \geq 1} \bar{\scrU}_{\kappa}/\bar{\scrU}_{\kappa} \cdot v^n$$
            Should the need arise, we may also write $\tilde{\scrU}_{\kappa}(\g)$ instead of just $\tilde{\scrU}_{\kappa}$ to be explicit.
        \end{definition}
        \begin{remark}
            Observe that:
                $$\tilde{\scrU}_{\kappa}\mod$$
            is equivalent to both the category of $\tilde{\g}_{\kappa}^{\pol}$-modules on which $c_{\kappa}$ acts as $1$, as well as the category of smooth $\tilde{\g}_{\kappa}^{\loc}$-modules on which $c_{\kappa}$ acts as $1$.
        \end{remark}
        There is a $\Z$-grading on the affine Lie algebra $\tilde{\g}_{\kappa}$ given by:
            $$\deg x v^m := m, \deg c_{\kappa} := 0$$
        for all $x \in \g$ and all $m \in \Z$ (for more details, see \cite{kassel_universal_central_extensions_of_lie_algebras} and \cite[Subsection 2.3.2]{msc_thesis_gamma_extended_toroidal_lie_algebras}), which induces the following triangular decomposition for $\tilde{\g}_{\kappa}$:
            $$\tilde{\g}_{\kappa} := \tilde{\g}_{\kappa}^- \oplus \tilde{\g}_{\kappa}^0 \oplus \tilde{\g}_{\kappa}^+$$
        wherein:
            $$\tilde{\g}_{\kappa}^{\pm} := \bigoplus_{m \in \pm \Z_{> 0}} \g v^m, \tilde{\g}_{\kappa}^0 := \g \oplus \bbC c_{\kappa}$$
        Let us then set:
            $$\tilde{\g}_{\kappa}^{\geq 0} := \tilde{\g}_{\kappa}^0 \oplus \tilde{\g}_{\kappa}^+$$
            $$\bar{\scrU}_{\kappa}^{\geq 0} := \rmU(\tilde{\g}_{\kappa}^{\geq 0})/\tilde{\g}_{\kappa}^{\geq 0} \cdot (c_{\kappa} - 1)$$
        and then define:
            $$\tilde{\scrU}_{\kappa}^{\geq 0} := \projlim_{n \geq 1} \bar{\scrU}_{\kappa}^{\geq 0}/\bar{\scrU}_{\kappa}^{\geq 0} \cdot v^n, \tilde{\scrU}_{\kappa}^- := \rmU(\tilde{\g}_{\kappa}^-)$$
        \begin{definition}[Vacuum modules for $\tilde{\g}_{\kappa}$]
            Like for the Weyl algebra (cf. definition \ref{def: lattice_weyl_algebras}), let us define the \textbf{vacuum module} of $\tilde{\g}_{\kappa}$ of highest weight $\lambda \in \hat{\h}_{\kappa}^*$ to be induced left-$\tilde{\scrU}_{\kappa}$-module:
                $$\V_{\kappa}^{\lambda} := \tilde{\scrU}_{\kappa} \tensor_{\tilde{\scrU}_{\kappa}^{\geq 0}} \bbC c_{\kappa}$$
            The highest-weight vector is:
                $$\ket{\lambda} := 1 \tensor c_{\kappa}$$
            and the left-$\tilde{\scrU}_{\kappa}$-action on $\V_{\kappa}^{\lambda}$ is given by:
                $$\forall h \in \h: h \cdot \ket{\lambda} := \lambda(h) \ket{\lambda}$$
                $$\tilde{\g}_{\kappa}^+ \cdot \ket{\lambda} := 0$$
            
            Sometimes, for clarity, we may write $\V_{\kappa}^{\lambda}(\g)$.
        \end{definition}
        A standard argument shows that as a left-$\tilde{\scrU}_{\kappa}$-module, the vacuum module $\V_{\kappa}^{\lambda}$ is identified with $\tilde{\scrU}_{\kappa}^-$. Using PBW, one can also explicitly compute the underlying vector space of $\tilde{\scrU}_{\kappa}^-$ to be spanned by ordered monomials of the form:
            $$x_{\alpha_r}^{(m_r)} ... x_{\alpha_1}^{(m_1)} \cdot \ket{\lambda}$$
        where $\alpha_r \leq ... \leq \alpha_1$ are roots of $\g$, $x_{\alpha_r}, ... x_{\alpha_1} \in \g_{\alpha_r}, ..., \g_{\alpha_1}$ are root vectors, and $m_r, ..., m_1 \in \Z_{< 0}$ are arbitrary. 

        \begin{theorem}[Universal affine vertex algebras] \label{theorem: universal_affine_VOAs}
            
        \end{theorem}
            \begin{proof}
                
            \end{proof}
    
        There is a construction, due to Frenkel-Kac and also Segal, which extends the bosonic vertex operators from theorem \ref{theorem: bosonic_vertex_operators} from the Weyl algebra $\tilde{\scrH}_Q$ attached to the root lattice of the finite-dimensional simple Lie algebra $\g$ to the entirety of the completed enveloping algebra $\tilde{\scrU}_{\kappa}$ of the affine Lie algebra $\tilde{\g}_{\kappa}$ associated to $\g$ at level $\kappa$. In performing this construction, one yields an explicit description of the basic representation $\simple^{\varpi_0}(\hat{\g})$, i.e. the (infinite-dimensional) integrable simple $\hat{\g}_{\kappa}$-module attached to the $0^{th}$ fundamental weight $\varpi_0$, in terms of the bosonic vertex operators, as well as a vertex algebra structure on the vacuum representation $\V_{\kappa}^{\lambda}$ of $\tilde{\g}_{\kappa}$.

        \begin{definition}[Level of highest-weight affine Kac-Moody modules] \label{def: levels_of_affine_weights}
            Define the \textbf{level} of a \textit{simple} $\hat{\g}_{\kappa}$-module of highest-weight $\lambda \in \hat{\h}_{\kappa}$ (or equivalently, the level of the weight $\lambda$ itself, thanks to the bijective correspondence between weights and simple highest-weight modules) as the number:
                $$\lambda(c_{\kappa})$$
        \end{definition}
        \begin{remark}[Levels vs. invariant non-degenerate bilinear forms on $\g$]
            Given a simple $\hat{\g}_{\kappa}$-module $V$ of highest-weight $\lambda$, let us note that:
                $$c_{\kappa} \cdot V_{\lambda} = \lambda(c_{\kappa}) \cdot V_{\lambda}$$
            which means that to specify the level of $V$ is the same as to specify the scalar\footnote{$c_{\kappa}$ necessarily acts as by scalar multiplication, as it is a central element of $\hat{\g}_{\kappa}$ (one argues using Schur's Lemma).} by which the central charge $c_{\kappa}$ acts on $V$. In light of this, and of the fact, that the explicit value of the level of a weight is entirely reliant on our choice of the invariant and non-degenerate symmetric bilinear form $\kappa \in \Sym^2(\g)^*$, it thus suffices to pretend as if the bilinear form $\kappa$ itself is actually the level of $V$, provided that $\lambda$ is fixed.

            Furthermore, note that definition \ref{def: levels_of_affine_weights} only works for Kac-Moody algebras of affine type, as only such Kac-Moody possess unique central charges.
        \end{remark}
        \begin{example}
            Suppose that $\g$ is simply laced (i.e. of one of the types $\sfA_l, \sfD_l, \sfE_l$) and let $\theta \in \Phi$ denote the highest root. \textit{A priori}, $\theta$ is of maximal length, and since $\g$ is simply laced, we have that:
                $$\kappa(\theta, \theta) = 2$$
            Recall also (from \cite[p. 100]{kac_infinite_dimensional_lie_algebras}) that:
                $$\check{\alpha}_0 := \frac{2}{\kappa(\theta, \theta)} c_{\kappa} - \check{\theta} \tensor 1 = c_{\kappa} - \check{\theta} \tensor 1$$
            By definition, the $0^{th}$ fundamental weight $\varpi_0$ is specified by:
                $$\varpi_0( \check{\alpha}_j) = \delta_{0, j}$$
            and thus:
                $$1 = \varpi_0(c_{\kappa}) - \varpi_0(\check{\theta} \tensor 1) = \varpi_0(c_{\kappa})$$
            where the last equality will become clear if one writes $\check{\theta} := \sum_{i = 1}^{\dim \g} a_i \check{\alpha}_i$ for some $a_i \in \Z$. We thus see that the $0^{th}$ fundamental weight is of level $1$. 
        \end{example}

        Suppose that $V$ is a $\hat{\g}_{\kappa}$-module with a vertex algebra structure:
            $$Y: V \to \End(V)[\![z^{\pm 1}]\!]$$
        

        \begin{theorem}[The level-$1$ FKS realisation] \label{theorem: level_1_FKS_realisation}
            Let $\g$ be simply laced.
            \begin{enumerate}
                \item The vertex algebra $\scrB_Q$ (as in theorem \ref{theorem: bosonic_vertex_operators}) is isomorphic to the universal affine vertex algebra $\V_{\kappa}^{\varpi_0}(\g)$.
                \item The simple $\hat{\g}_{\kappa}$-module $\simple^{\varpi_0}(\hat{\g}_{\kappa})$ carries a $\V_{\kappa}^{\varpi_0}(\g)$-module structure.
            \end{enumerate}
        \end{theorem}
            \begin{proof}
                
            \end{proof}