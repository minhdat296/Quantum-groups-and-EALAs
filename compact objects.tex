\begin{definition}[Compact objects] \label{def: compact_objects}
                An object $c \in \Ob(\C)$ of a locally small category $\C$ is said to be \textbf{compact} if and only if the representable copresheaf:
                    $$\C(c, -): \C \to \Sets$$
                preserves all small filtered colimits that exist in $\C$. The (\textit{a priori} full) subcategory of $\C$ spanned by such objects is denoted by $\C^{\comp}$ (or sometimes $\C^{\omega}$) and shall be known as the \textbf{maximal compact subcategory} of $\C$. 
            \end{definition}
            \begin{remark}[Compact objects in full subcategories] \label{remark: compact_objects_in_full_subcategories}
                Suppsoe that $\C$ is a category and $\C_0 \subseteq \C$ is a full subcategory thereof. Then, should $c \in \Ob(\C_0)$ be any object of $\C_0$ that is compact as an object of the larger ambient category $\C$, then it is clear from definition \ref{def: compact_objects} that it would also be compact as an object of $\C_0$; that is:
                    $$\C_0 \cap \C^{\comp} = \C_0^{\comp}$$
            \end{remark}
            \begin{proposition}[Finite-length objects are compact] \label{prop: finite_length_objects_are_compact}
                In any abelian category $\calA$, an object is of finite length (cf. definition \ref{def: lengths_of_objects_and_jordan_holder_series}) if and only if it is compact\footnote{We should note that within the context of the theory of (locally small) additive or triangulated categories $\calA$, compact objects $X \in \Ob(\calA)$ are usually defined to be those such that the representable copresheaf $\calA(X, -): \calA \to \Sets$ preserves only arbitrary coproducts that exist in $\calA$. If that is the definition of compactness that one adopts, then corollary \ref{prop: finite_length_objects_are_compact} shall have to state that an object of an abelian category is of finite length if and only if it is compact and \textit{projective}.}. 
            \end{proposition}
                \begin{proof}
                    Let $X \in \Ob(\calA^{\fin})$ be an object of $\calA$ that is of some finite length $n$ and fix a Jordan-H\"older filtration:
                        $$0 =: X_0 \subseteq X_1 \subseteq ... \subseteq X_n \subseteq X$$
                    thereof. Said filtration induces a 
                    
                    Converly, let $X \in \Ob(\calA^{\comp})$ be a compact object of $\calA$. 
                \end{proof}
            \begin{lemma}[Projective indecomposable modules over finite algebras are simple] \label{lemma: projective_indecomposable_modules_over_finite_algebras_are_simple}
                \footnote{One interesting consequence of this lemma (which is not too relevant to the current discussion about Dynkin quivers, as path algebras of quivers - even over fields - are generally not semi-simple, only hereditary) is that it implies via lemma \ref{lemma: projective_and_injective_modules_over_semi_simple_rings} that over a semi-simple finite-dimensional algebra over a field, a module is indecomposable if and only if it is simple.} Let $k$ be a field and $A$ be a finite-dimensional $k$-algebra. Then, there exists a surjection:
                    $$
                        \left\{\text{projective left/right-$A$-modules}\right\}
                        \to
                        \left\{\text{simple left/right-$A$-modules}\right\}
                    $$
                whose pre-images over each simple left/right-$A$-module $M$ is a choice of projective cover $\e: P \to M$ wherein $P$ is indecomposable as an $A$-module, which is unique up to isomorphisms. 
            \end{lemma}
                \begin{proof}
                    First of all, because the category of left/right-$A$-modules \textit{a priori} has enough projectives, meaning that for all left/right-$A$-modules $M$ there exists a projective cover $\e: P \to M$, we certainly have a surjection:
                        $$
                            \left\{\text{projective left/right-$A$-modules}\right\}
                            \to
                            \left\{\text{simple left/right-$A$-modules}\right\}
                        $$
                    Now, in order to show that the pre-image of the class of simple left/right-$A$-modules under this surjection is that of indecomposable projective left/right-$A$-modules, start by recalling that simple left/right-modules are cyclic\footnote{The proof is simple (pun not intended!): if we were to suppose to the contrary that there existed a simple module $M$ with $\geq 1$ generators, then said module will admit non-zero (cyclic) proper submodules generated by the generators, and therefore $M$ can not be simple, contradicting our initial assumption.}. This tells us that every simple left/right-$A$-module admits a projective cover by a free\footnote{We assume the Axiom of Choice, so that free module would be projective.} left/right-$A$-module on $1$ generator, which is of course indecomposable and unique up to isomorphisms, per the universal property of left-adjoints.
                \end{proof}
            \begin{proposition}[Compact indecomposable objects are simple] \label{prop: compact_indecomposable_objects_are_simple}
                Let $\calA$ be a locally small abelian category. Then, an indecomposable object is compact if and only if it is simple.
            \end{proposition}
                \begin{proof}
                    Straightforward from the fact that simple objects are precisely those indecomposable objects which are of finite lengths (one can then simply apply proposition \ref{prop: finite_length_objects_are_compact}). 
                \end{proof}