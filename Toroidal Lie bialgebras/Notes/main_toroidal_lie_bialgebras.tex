\documentclass[a4paper, 11pt]{article}

%\usepackage[center]{titlesec}

\usepackage{amsfonts, amssymb, amsmath, amsthm, amsxtra}

\usepackage{foekfont}

\usepackage{MnSymbol}

\usepackage{pdfrender, xcolor}
%\pdfrender{StrokeColor=black,LineWidth=.4pt,TextRenderingMode=2}

%\usepackage{minitoc}
%\setcounter{section}{-1}
%\setcounter{tocdepth}{}
%\setcounter{minitocdepth}{}
%\setcounter{secnumdepth}{}

\usepackage{graphicx}

\usepackage[english]{babel}
\usepackage[utf8]{inputenc}
%\usepackage{mathpazo}
%\usepackage{eucal}
\usepackage{eufrak}
\usepackage{bbm}
\usepackage{bm}
\usepackage{csquotes}
\usepackage[nottoc]{tocbibind}
\usepackage{appendix}
\usepackage{float}
\usepackage[T1]{fontenc}
\usepackage[
    left = \flqq{},% 
    right = \frqq{},% 
    leftsub = \flq{},% 
    rightsub = \frq{} %
]{dirtytalk}

\usepackage{imakeidx}
\makeindex

%\usepackage[dvipsnames]{xcolor}
\usepackage{hyperref}
    \hypersetup{
        colorlinks=true,
        linkcolor=teal,
        filecolor=pink,      
        urlcolor=teal,
        citecolor=magenta
    }
\usepackage{comment}

% You would set the PDF title, author, etc. with package options or
% \hypersetup.

\usepackage[backend=biber, style=alphabetic, sorting=nty]{biblatex}
    \addbibresource{bibliography.bib}
\renewbibmacro{in:}{}

\raggedbottom

\usepackage{mathrsfs}
\usepackage{mathtools} 
\mathtoolsset{showonlyrefs} 
%\usepackage{amsthm}
\renewcommand\qedsymbol{$\blacksquare$}

\usepackage{tikz-cd}
\tikzcdset{scale cd/.style={every label/.append style={scale=#1},
    cells={nodes={scale=#1}}}}
\usepackage{tikz}
\usepackage{setspace}
\usepackage[version=3]{mhchem}
\parskip=0.1in
\usepackage[margin=25mm]{geometry}

\usepackage{listings, lstautogobble}
\lstset{
	language=matlab,
	basicstyle=\scriptsize\ttfamily,
	commentstyle=\ttfamily\itshape\color{gray},
	stringstyle=\ttfamily,
	showstringspaces=false,
	breaklines=true,
	frameround=ffff,
	frame=single,
	rulecolor=\color{black},
	autogobble=true
}

\usepackage{todonotes,tocloft,xpatch,hyperref}

% This is based on classicthesis chapter definition
\let\oldsec=\section
\renewcommand*{\section}{\secdef{\Sec}{\SecS}}
\newcommand\SecS[1]{\oldsec*{#1}}%
\newcommand\Sec[2][]{\oldsec[\texorpdfstring{#1}{#1}]{#2}}%

\newcounter{istodo}[section]

% http://tex.stackexchange.com/a/61267/11984
\makeatletter
%\xapptocmd{\Sec}{\addtocontents{tdo}{\protect\todoline{\thesection}{#1}{}}}{}{}
\newcommand{\todoline}[1]{\@ifnextchar\Endoftdo{}{\@todoline{#1}}}
\newcommand{\@todoline}[3]{%
	\@ifnextchar\todoline{}
	{\contentsline{section}{\numberline{#1}#2}{#3}{}{}}%
}
\let\l@todo\l@subsection
\newcommand{\Endoftdo}{}

\AtEndDocument{\addtocontents{tdo}{\string\Endoftdo}}
\makeatother

\usepackage{lipsum}

%   Reduce the margin of the summary:
\def\changemargin#1#2{\list{}{\rightmargin#2\leftmargin#1}\item[]}
\let\endchangemargin=\endlist 

%   Generate the environment for the abstract:
%\newcommand\summaryname{Abstract}
%\newenvironment{abstract}%
    %{\small\begin{center}%
    %\bfseries{\summaryname} \end{center}}

\newtheorem{theorem}{Theorem}[section]
    \numberwithin{theorem}{subsection}
\newtheorem{proposition}{Proposition}[section]
    \numberwithin{proposition}{subsection}
\newtheorem{lemma}{Lemma}[section]
    \numberwithin{lemma}{subsection}
\newtheorem{claim}{Claim}[section]
    \numberwithin{claim}{subsection}
\newtheorem{question}{Question}[section]
    \numberwithin{question}{subsection}

\theoremstyle{definition}
    \newtheorem{definition}{Definition}[section]
        \numberwithin{definition}{subsection}

\theoremstyle{remark}
    \newtheorem{remark}{Remark}[section]
        \numberwithin{remark}{subsection}
    \newtheorem{example}{Example}[section]
        \numberwithin{example}{subsection}    
    \newtheorem{convention}{Convention}[section]
        \numberwithin{convention}{subsection}
    \newtheorem{corollary}{Corollary}[section]
        \numberwithin{corollary}{subsection}

\numberwithin{equation}{section}

\setcounter{section}{-1}

\renewcommand{\cong}{\simeq}
\newcommand{\ladjoint}{\dashv}
\newcommand{\radjoint}{\vdash}
\newcommand{\<}{\langle}
\renewcommand{\>}{\rangle}
\newcommand{\ndiv}{\hspace{-2pt}\not|\hspace{5pt}}
\newcommand{\cond}{\blacktriangle}
\newcommand{\decond}{\triangle}
\newcommand{\solid}{\blacksquare}
\newcommand{\ot}{\leftarrow}
\renewcommand{\-}{\text{-}}
\renewcommand{\mapsto}{\leadsto}
\renewcommand{\leq}{\leqslant}
\renewcommand{\geq}{\geqslant}
\renewcommand{\setminus}{\smallsetminus}
\makeatletter
\DeclareRobustCommand{\cev}[1]{%
  {\mathpalette\do@cev{#1}}%
}
\newcommand{\do@cev}[2]{%
  \vbox{\offinterlineskip
    \sbox\z@{$\m@th#1 x$}%
    \ialign{##\cr
      \hidewidth\reflectbox{$\m@th#1\vec{}\mkern4mu$}\hidewidth\cr
      \noalign{\kern-\ht\z@}
      $\m@th#1#2$\cr
    }%
  }%
}
\makeatother

\newcommand{\N}{\mathbb{N}}
\newcommand{\Z}{\mathbb{Z}}
\newcommand{\Q}{\mathbb{Q}}
\newcommand{\R}{\mathbb{R}}
\newcommand{\bbC}{\mathbb{C}}
\NewDocumentCommand{\x}{e{_^}}{%
  \mathbin{\mathop{\times}\displaylimits
    \IfValueT{#1}{_{#1}}
    \IfValueT{#2}{^{#2}}
  }%
}
\NewDocumentCommand{\pushout}{e{_^}}{%
  \mathbin{\mathop{\sqcup}\displaylimits
    \IfValueT{#1}{_{#1}}
    \IfValueT{#2}{^{#2}}
  }%
}
\newcommand{\supp}{\operatorname{supp}}
\newcommand{\im}{\operatorname{im}}
\newcommand{\coker}{\operatorname{coker}}
\newcommand{\id}{\mathrm{id}}
\newcommand{\chara}{\operatorname{char}}
\newcommand{\trdeg}{\operatorname{trdeg}}
\newcommand{\rank}{\operatorname{rank}}
\newcommand{\trace}{\operatorname{tr}}
\newcommand{\length}{\operatorname{length}}
\newcommand{\height}{\operatorname{ht}}
\renewcommand{\span}{\operatorname{span}}
\newcommand{\e}{\epsilon}
\newcommand{\p}{\mathfrak{p}}
\newcommand{\q}{\mathfrak{q}}
\newcommand{\m}{\mathfrak{m}}
\newcommand{\n}{\mathfrak{n}}
\newcommand{\calF}{\mathcal{F}}
\newcommand{\calG}{\mathcal{G}}
\newcommand{\calO}{\mathcal{O}}
\newcommand{\F}{\mathbb{F}}
\DeclareMathOperator{\lcm}{lcm}
\newcommand{\gr}{\operatorname{gr}}
\newcommand{\vol}{\mathrm{vol}}
\newcommand{\ord}{\operatorname{ord}}
\newcommand{\projdim}{\operatorname{proj.dim}}
\newcommand{\injdim}{\operatorname{inj.dim}}
\newcommand{\flatdim}{\operatorname{flat.dim}}
\newcommand{\globdim}{\operatorname{glob.dim}}
\renewcommand{\Re}{\operatorname{Re}}
\renewcommand{\Im}{\operatorname{Im}}
\newcommand{\sgn}{\operatorname{sgn}}
\newcommand{\coad}{\operatorname{coad}}

\newcommand{\Ad}{\mathrm{Ad}}
\newcommand{\GL}{\mathrm{GL}}
\newcommand{\SL}{\mathrm{SL}}
\newcommand{\PGL}{\mathrm{PGL}}
\newcommand{\PSL}{\mathrm{PSL}}
\newcommand{\Sp}{\mathrm{Sp}}
\newcommand{\GSp}{\mathrm{GSp}}
\newcommand{\GSpin}{\mathrm{GSpin}}
\newcommand{\rmO}{\mathrm{O}}
\newcommand{\SO}{\mathrm{SO}}
\newcommand{\SU}{\mathrm{SU}}
\newcommand{\rmU}{\mathrm{U}}
\newcommand{\rmu}{\mathrm{u}}
\newcommand{\rmV}{\mathrm{V}}
\newcommand{\gl}{\mathfrak{gl}}
\renewcommand{\sl}{\mathfrak{sl}}
\newcommand{\diag}{\mathfrak{diag}}
\newcommand{\pgl}{\mathfrak{pgl}}
\newcommand{\psl}{\mathfrak{psl}}
\newcommand{\fraksp}{\mathfrak{sp}}
\newcommand{\gsp}{\mathfrak{gsp}}
\newcommand{\gspin}{\mathfrak{gspin}}
\newcommand{\frako}{\mathfrak{o}}
\newcommand{\so}{\mathfrak{so}}
\newcommand{\su}{\mathfrak{su}}
%\newcommand{\fraku}{\mathfrak{u}}
\newcommand{\Spec}{\operatorname{Spec}}
\newcommand{\Spf}{\operatorname{Spf}}
\newcommand{\Spm}{\operatorname{Spm}}
\newcommand{\Spv}{\operatorname{Spv}}
\newcommand{\Spa}{\operatorname{Spa}}
\newcommand{\Spd}{\operatorname{Spd}}
\newcommand{\Proj}{\operatorname{Proj}}
\newcommand{\Gr}{\mathrm{Gr}}
\newcommand{\Hecke}{\mathrm{Hecke}}
\newcommand{\Sht}{\mathrm{Sht}}
\newcommand{\Quot}{\mathrm{Quot}}
\newcommand{\Hilb}{\mathrm{Hilb}}
\newcommand{\Pic}{\mathrm{Pic}}
\newcommand{\Div}{\mathrm{Div}}
\newcommand{\Jac}{\mathrm{Jac}}
\newcommand{\Alb}{\mathrm{Alb}} %albanese variety
\newcommand{\Bun}{\mathrm{Bun}}
\newcommand{\loopspace}{\mathbf{\Omega}}
\newcommand{\suspension}{\mathbf{\Sigma}}
\newcommand{\tangent}{\mathrm{T}} %tangent space
\newcommand{\Eig}{\mathrm{Eig}}
\newcommand{\Cox}{\mathrm{Cox}} %coxeter functors
\newcommand{\rmK}{\mathrm{K}} %Killing form
\newcommand{\km}{\mathfrak{km}} %kac-moody algebras
\newcommand{\Dyn}{\mathrm{Dyn}} %associated Dynkin quivers
\newcommand{\Car}{\mathrm{Car}} %cartan matrices of finite quivers

\newcommand{\Ring}{\mathrm{Ring}}
\newcommand{\Cring}{\mathrm{CRing}}
\newcommand{\Alg}{\mathrm{Alg}}
\newcommand{\Leib}{\mathrm{Leib}} %leibniz algebras
\newcommand{\Fld}{\mathrm{Fld}}
\newcommand{\Sets}{\mathrm{Sets}}
\newcommand{\Equiv}{\mathrm{Equiv}} %equivalence relations
\newcommand{\Cat}{\mathrm{Cat}}
\newcommand{\Grp}{\mathrm{Grp}}
\newcommand{\Ab}{\mathrm{Ab}}
\newcommand{\Sch}{\mathrm{Sch}}
\newcommand{\Coh}{\mathrm{Coh}}
\newcommand{\QCoh}{\mathrm{QCoh}}
\newcommand{\Perf}{\mathrm{Perf}} %perfect complexes
\newcommand{\Sing}{\mathrm{Sing}} %singularity categories
\newcommand{\Desc}{\mathrm{Desc}}
\newcommand{\Sh}{\mathrm{Sh}}
\newcommand{\Psh}{\mathrm{PSh}}
\newcommand{\Fib}{\mathrm{Fib}}
\renewcommand{\mod}{\-\mathrm{mod}}
\newcommand{\comod}{\-\mathrm{comod}}
\newcommand{\bimod}{\-\mathrm{bimod}}
\newcommand{\Vect}{\mathrm{Vect}}
\newcommand{\Rep}{\mathrm{Rep}}
\newcommand{\Grpd}{\mathrm{Grpd}}
\newcommand{\Arr}{\mathrm{Arr}}
\newcommand{\Esp}{\mathrm{Esp}}
\newcommand{\Ob}{\mathrm{Ob}}
\newcommand{\Mor}{\mathrm{Mor}}
\newcommand{\Mfd}{\mathrm{Mfd}}
\newcommand{\Riem}{\mathrm{Riem}}
\newcommand{\RS}{\mathrm{RS}}
\newcommand{\LRS}{\mathrm{LRS}}
\newcommand{\TRS}{\mathrm{TRS}}
\newcommand{\TLRS}{\mathrm{TLRS}}
\newcommand{\LVRS}{\mathrm{LVRS}}
\newcommand{\LBRS}{\mathrm{LBRS}}
\newcommand{\Spc}{\mathrm{Spc}}
\newcommand{\Top}{\mathrm{Top}}
\newcommand{\Topos}{\mathrm{Topos}}
\newcommand{\Nil}{\mathfrak{nil}}
\newcommand{\J}{\mathfrak{J}}
\newcommand{\Stk}{\mathrm{Stk}}
\newcommand{\Pre}{\mathrm{Pre}}
\newcommand{\simp}{\mathbf{\Delta}}
\newcommand{\Res}{\mathrm{Res}}
\newcommand{\Ind}{\mathrm{Ind}}
\newcommand{\Pro}{\mathrm{Pro}}
\newcommand{\Mon}{\mathrm{Mon}}
\newcommand{\Comm}{\mathrm{Comm}}
\newcommand{\Fin}{\mathrm{Fin}}
\newcommand{\Assoc}{\mathrm{Assoc}}
\newcommand{\Semi}{\mathrm{Semi}}
\newcommand{\Co}{\mathrm{Co}}
\newcommand{\Loc}{\mathrm{Loc}}
\newcommand{\Ringed}{\mathrm{Ringed}}
\newcommand{\Haus}{\mathrm{Haus}} %hausdorff spaces
\newcommand{\Comp}{\mathrm{Comp}} %compact hausdorff spaces
\newcommand{\Stone}{\mathrm{Stone}} %stone spaces
\newcommand{\Extr}{\mathrm{Extr}} %extremely disconnected spaces
\newcommand{\Ouv}{\mathrm{Ouv}}
\newcommand{\Str}{\mathrm{Str}}
\newcommand{\Func}{\mathrm{Func}}
\newcommand{\Crys}{\mathrm{Crys}}
\newcommand{\LocSys}{\mathrm{LocSys}}
\newcommand{\Sieves}{\mathrm{Sieves}}
\newcommand{\pt}{\mathrm{pt}}
\newcommand{\Graphs}{\mathrm{Graphs}}
\newcommand{\Lie}{\mathrm{Lie}}
\newcommand{\Env}{\mathrm{Env}}
\newcommand{\Ho}{\mathrm{Ho}}
\newcommand{\rmD}{\mathrm{D}}
\newcommand{\Cov}{\mathrm{Cov}}
\newcommand{\Frames}{\mathrm{Frames}}
\newcommand{\Locales}{\mathrm{Locales}}
\newcommand{\Span}{\mathrm{Span}}
\newcommand{\Corr}{\mathrm{Corr}}
\newcommand{\Monad}{\mathrm{Monad}}
\newcommand{\Var}{\mathrm{Var}}
\newcommand{\sfN}{\mathrm{N}} %nerve
\newcommand{\Diam}{\mathrm{Diam}} %diamonds
\newcommand{\co}{\mathrm{co}}
\newcommand{\ev}{\mathrm{ev}}
\newcommand{\bi}{\mathrm{bi}}
\newcommand{\Nat}{\mathrm{Nat}}
\newcommand{\Hopf}{\mathrm{Hopf}}
\newcommand{\Dmod}{\mathrm{D}\mod}
\newcommand{\Perv}{\mathrm{Perv}}
\newcommand{\Sph}{\mathrm{Sph}}
\newcommand{\Moduli}{\mathrm{Moduli}}
\newcommand{\Pseudo}{\mathrm{Pseudo}}
\newcommand{\Lax}{\mathrm{Lax}}
\newcommand{\Strict}{\mathrm{Strict}}
\newcommand{\Opd}{\mathrm{Opd}} %operads
\newcommand{\Shv}{\mathrm{Shv}}
\newcommand{\Char}{\mathrm{Char}} %CharShv = character sheaves
\newcommand{\Huber}{\mathrm{Huber}}
\newcommand{\Tate}{\mathrm{Tate}}
\newcommand{\Affd}{\mathrm{Affd}} %affinoid algebras
\newcommand{\Adic}{\mathrm{Adic}} %adic spaces
\newcommand{\Rig}{\mathrm{Rig}}
\newcommand{\An}{\mathrm{An}}
\newcommand{\Perfd}{\mathrm{Perfd}} %perfectoid spaces
\newcommand{\Sub}{\mathrm{Sub}} %subobjects
\newcommand{\Ideals}{\mathrm{Ideals}}
\newcommand{\Isoc}{\mathrm{Isoc}} %isocrystals
\newcommand{\Ban}{\-\mathrm{Ban}} %Banach spaces
\newcommand{\Fre}{\-\mathrm{Fr\acute{e}}} %Frechet spaces
\newcommand{\Ch}{\mathrm{Ch}} %chain complexes
\newcommand{\Pure}{\mathrm{Pure}}
\newcommand{\Mixed}{\mathrm{Mixed}}
\newcommand{\Hodge}{\mathrm{Hodge}} %Hodge structures
\newcommand{\Mot}{\mathrm{Mot}} %motives
\newcommand{\KL}{\mathrm{KL}} %category of Kazhdan-Lusztig modules
\newcommand{\Pres}{\mathrm{Pres}} %presentable categories
\newcommand{\Noohi}{\mathrm{Noohi}} %category of Noohi groups
\newcommand{\Inf}{\mathrm{Inf}}
\newcommand{\LPar}{\mathrm{LPar}} %Langlands parameters
\newcommand{\ORig}{\mathrm{ORig}} %overconvergent sites
\newcommand{\Quiv}{\mathrm{Quiv}} %quivers
\newcommand{\Def}{\mathrm{Def}} %deformation functors
\newcommand{\Root}{\mathrm{Root}}
\newcommand{\gRep}{\mathrm{gRep}}
\newcommand{\Higgs}{\mathrm{Higgs}}
\newcommand{\BGG}{\mathrm{BGG}}

\newcommand{\Aut}{\mathrm{Aut}}
\newcommand{\Inn}{\mathrm{Inn}}
\newcommand{\Out}{\mathrm{Out}}
\newcommand{\der}{\mathfrak{der}} %derivations on Lie algebras
\newcommand{\frakend}{\mathfrak{end}}
\newcommand{\aut}{\mathfrak{aut}}
\newcommand{\inn}{\mathfrak{inn}} %inner derivations
\newcommand{\out}{\mathfrak{out}} %outer derivations
\newcommand{\Stab}{\mathrm{Stab}}
\newcommand{\Cent}{\mathrm{Cent}}
\newcommand{\Norm}{\mathrm{Norm}}
\newcommand{\stab}{\mathfrak{stab}}
\newcommand{\cent}{\mathfrak{cent}}
\newcommand{\norm}{\mathfrak{norm}}
\newcommand{\Rad}{\operatorname{Rad}}
\newcommand{\Transporter}{\mathrm{Transp}} %transporter between two subsets of a group
\newcommand{\Conj}{\mathrm{Conj}}
\newcommand{\Diag}{\mathrm{Diag}}
\newcommand{\Gal}{\mathrm{Gal}}
\newcommand{\bfG}{\mathbf{G}} %absolute Galois group
\newcommand{\Frac}{\mathrm{Frac}}
\newcommand{\Ann}{\mathrm{Ann}}
\newcommand{\Val}{\mathrm{Val}}
\newcommand{\Chow}{\mathrm{Chow}}
\newcommand{\Sym}{\mathrm{Sym}}
\newcommand{\End}{\mathrm{End}}
\newcommand{\Mat}{\mathrm{Mat}}
\newcommand{\Diff}{\mathrm{Diff}}
\newcommand{\Autom}{\mathrm{Autom}}
\newcommand{\Artin}{\mathrm{Artin}} %artin maps
\newcommand{\sk}{\mathrm{sk}} %skeleton of a category
\newcommand{\eqv}{\mathrm{eqv}} %functor that maps groups $G$ to $G$-sets
\newcommand{\Inert}{\mathrm{Inert}}
\newcommand{\Fil}{\mathrm{Fil}}
\newcommand{\Prim}{\mathfrak{Prim}}
\newcommand{\Nerve}{\mathrm{N}}
\newcommand{\Hol}{\mathrm{Hol}} %holomorphic functions %holonomy groups
\newcommand{\Bi}{\mathrm{Bi}} %Bi for biholomorphic functions
\newcommand{\chev}{\mathfrak{chev}} %chevalley relations
\newcommand{\bfLie}{\mathbf{Lie}} %non-reduced lie algebra associated to generalised cartan matrices
\newcommand{\frakLie}{\mathfrak{Lie}} %reduced lie algebra associated to generalised cartan matrices
\newcommand{\frakChev}{\mathfrak{Chev}} 
\newcommand{\Rees}{\operatorname{Rees}}
\newcommand{\Dr}{\mathrm{Dr}} %Drinfeld's quantum double 

\renewcommand{\projlim}{\varprojlim}
\newcommand{\indlim}{\varinjlim}
\newcommand{\colim}{\operatorname{colim}}
\renewcommand{\lim}{\operatorname{lim}}
\newcommand{\toto}{\rightrightarrows}
%\newcommand{\tensor}{\otimes}
\NewDocumentCommand{\tensor}{e{_^}}{%
  \mathbin{\mathop{\otimes}\displaylimits
    \IfValueT{#1}{_{#1}}
    \IfValueT{#2}{^{#2}}
  }%
}
\NewDocumentCommand{\singtensor}{e{_^}}{%
  \mathbin{\mathop{\odot}\displaylimits
    \IfValueT{#1}{_{#1}}
    \IfValueT{#2}{^{#2}}
  }%
}
\NewDocumentCommand{\hattensor}{e{_^}}{%
  \mathbin{\mathop{\hat{\otimes}}\displaylimits
    \IfValueT{#1}{_{#1}}
    \IfValueT{#2}{^{#2}}
  }%
}
\NewDocumentCommand{\semidirect}{e{_^}}{%
  \mathbin{\mathop{\rtimes}\displaylimits
    \IfValueT{#1}{_{#1}}
    \IfValueT{#2}{^{#2}}
  }%
}
\newcommand{\eq}{\operatorname{eq}}
\newcommand{\coeq}{\operatorname{coeq}}
\newcommand{\Hom}{\mathrm{Hom}}
\newcommand{\Maps}{\mathrm{Maps}}
\newcommand{\Tor}{\mathrm{Tor}}
\newcommand{\Ext}{\mathrm{Ext}}
\newcommand{\Isom}{\mathrm{Isom}}
\newcommand{\stalk}{\mathrm{stalk}}
\newcommand{\RKE}{\operatorname{RKE}}
\newcommand{\LKE}{\operatorname{LKE}}
\newcommand{\oblv}{\mathrm{oblv}}
\newcommand{\const}{\mathrm{const}}
\newcommand{\free}{\mathrm{free}}
\newcommand{\adrep}{\mathrm{ad}} %adjoint representation
\newcommand{\NL}{\mathbb{NL}} %naive cotangent complex
\newcommand{\pr}{\operatorname{pr}}
\newcommand{\Der}{\mathrm{Der}}
\newcommand{\Frob}{\mathrm{Fr}} %Frobenius
\newcommand{\frob}{\mathrm{f}} %trace of Frobenius
\newcommand{\bfpt}{\mathbf{pt}}
\newcommand{\bfloc}{\mathbf{loc}}
\DeclareMathAlphabet{\mymathbb}{U}{BOONDOX-ds}{m}{n}
\newcommand{\0}{\mymathbb{0}}
\newcommand{\1}{\mathbbm{1}}
\newcommand{\2}{\mathbbm{2}}
\newcommand{\Jet}{\mathrm{Jet}}
\newcommand{\Split}{\mathrm{Split}}
\newcommand{\Sq}{\mathrm{Sq}}
\newcommand{\Zero}{\mathrm{Z}}
\newcommand{\SqZ}{\Sq\Zero}
\newcommand{\lie}{\mathfrak{lie}}
\newcommand{\y}{\mathrm{y}} %yoneda
\newcommand{\Sm}{\mathrm{Sm}}
\newcommand{\AJ}{\phi} %abel-jacobi map
\newcommand{\act}{\mathrm{act}}
\newcommand{\ram}{\mathrm{ram}} %ramification index
\newcommand{\inv}{\mathrm{inv}}
\newcommand{\Spr}{\mathrm{Spr}} %the Springer map/sheaf
\newcommand{\Refl}{\mathrm{Refl}} %reflection functor]
\newcommand{\HH}{\mathrm{HH}} %Hochschild (co)homology
\newcommand{\Poinc}{\mathrm{Poinc}}
\newcommand{\Simpson}{\mathrm{Simpson}}

\newcommand{\bbU}{\mathbb{U}}
\newcommand{\V}{\mathbb{V}}
\newcommand{\calU}{\mathcal{U}}
\newcommand{\calW}{\mathcal{W}}
\newcommand{\rmI}{\mathrm{I}} %augmentation ideal
\newcommand{\bfV}{\mathbf{V}}
\newcommand{\C}{\mathcal{C}}
\newcommand{\D}{\mathcal{D}}
\newcommand{\T}{\mathscr{T}} %Tate modules
\newcommand{\calM}{\mathcal{M}}
\newcommand{\calN}{\mathcal{N}}
\newcommand{\calP}{\mathcal{P}}
\newcommand{\calQ}{\mathcal{Q}}
\newcommand{\A}{\mathbb{A}}
\renewcommand{\P}{\mathbb{P}}
\newcommand{\calL}{\mathcal{L}}
\newcommand{\E}{\mathcal{E}}
\renewcommand{\H}{\mathbf{H}}
\newcommand{\scrS}{\mathscr{S}}
\newcommand{\calX}{\mathcal{X}}
\newcommand{\calY}{\mathcal{Y}}
\newcommand{\calZ}{\mathcal{Z}}
\newcommand{\calS}{\mathcal{S}}
\newcommand{\calR}{\mathcal{R}}
\newcommand{\scrX}{\mathscr{X}}
\newcommand{\scrY}{\mathscr{Y}}
\newcommand{\scrZ}{\mathscr{Z}}
\newcommand{\calA}{\mathcal{A}}
\newcommand{\calB}{\mathcal{B}}
\renewcommand{\S}{\mathcal{S}}
\newcommand{\B}{\mathbb{B}}
\newcommand{\bbD}{\mathbb{D}}
\newcommand{\G}{\mathbb{G}}
\newcommand{\horn}{\mathbf{\Lambda}}
\renewcommand{\L}{\mathbb{L}}
\renewcommand{\a}{\mathfrak{a}}
\renewcommand{\b}{\mathfrak{b}}
\renewcommand{\c}{\mathfrak{c}}
\renewcommand{\t}{\mathfrak{t}}
\renewcommand{\r}{\mathfrak{r}}
\newcommand{\fraku}{\mathfrak{u}}
\newcommand{\bbX}{\mathbb{X}}
\newcommand{\frakw}{\mathfrak{w}}
\newcommand{\frakG}{\mathfrak{G}}
\newcommand{\frakH}{\mathfrak{H}}
\newcommand{\frakE}{\mathfrak{E}}
\newcommand{\frakF}{\mathfrak{F}}
\newcommand{\g}{\mathfrak{g}}
\newcommand{\h}{\mathfrak{h}}
\renewcommand{\k}{\mathfrak{k}}
\newcommand{\z}{\mathfrak{z}}
\newcommand{\fraki}{\mathfrak{i}}
\newcommand{\frakj}{\mathfrak{j}}
\newcommand{\del}{\partial}
\newcommand{\bbE}{\mathbb{E}}
\newcommand{\scrO}{\mathscr{O}}
\newcommand{\bbO}{\mathbb{O}}
\newcommand{\scrA}{\mathscr{A}}
\newcommand{\scrB}{\mathscr{B}}
\newcommand{\scrF}{\mathscr{F}}
\newcommand{\scrG}{\mathscr{G}}
\newcommand{\scrM}{\mathscr{M}}
\newcommand{\scrN}{\mathscr{N}}
\newcommand{\scrP}{\mathscr{P}}
\newcommand{\frakS}{\mathfrak{S}}
\newcommand{\frakT}{\mathfrak{T}}
\newcommand{\calI}{\mathcal{I}}
\newcommand{\calJ}{\mathcal{J}}
\newcommand{\scrI}{\mathscr{I}}
\newcommand{\scrJ}{\mathscr{J}}
\newcommand{\scrK}{\mathscr{K}}
\newcommand{\calK}{\mathcal{K}}
\newcommand{\scrV}{\mathscr{V}}
\newcommand{\scrW}{\mathscr{W}}
\newcommand{\bbS}{\mathbb{S}}
\newcommand{\scrH}{\mathscr{H}}
\newcommand{\bfA}{\mathbf{A}}
\newcommand{\bfB}{\mathbf{B}}
\newcommand{\bfC}{\mathbf{C}}
\renewcommand{\O}{\mathbb{O}}
\newcommand{\calV}{\mathcal{V}}
\newcommand{\scrR}{\mathscr{R}} %radical
\newcommand{\rmZ}{\mathrm{Z}} %centre of algebra
\newcommand{\rmC}{\mathrm{C}} %centralisers in algebras
\newcommand{\bfGamma}{\mathbf{\Gamma}}
\newcommand{\scrU}{\mathscr{U}}
\newcommand{\rmW}{\mathrm{W}} %Weil group
\newcommand{\frakM}{\mathfrak{M}}
\newcommand{\frakN}{\mathfrak{N}}
\newcommand{\frakB}{\mathfrak{B}}
\newcommand{\frakX}{\mathfrak{X}}
\newcommand{\frakY}{\mathfrak{Y}}
\newcommand{\frakZ}{\mathfrak{Z}}
\newcommand{\frakU}{\mathfrak{U}}
\newcommand{\frakR}{\mathfrak{R}}
\newcommand{\frakP}{\mathfrak{P}}
\newcommand{\frakQ}{\mathfrak{Q}}
\newcommand{\sfX}{\mathsf{X}}
\newcommand{\sfY}{\mathsf{Y}}
\newcommand{\sfZ}{\mathsf{Z}}
\newcommand{\sfS}{\mathsf{S}}
\newcommand{\sfT}{\mathsf{T}}
\newcommand{\sfOmega}{\mathsf{\Omega}} %drinfeld p-adic upper-half plane
\newcommand{\rmA}{\mathrm{A}}
\newcommand{\rmB}{\mathrm{B}}
\newcommand{\calT}{\mathcal{T}}
\newcommand{\sfA}{\mathsf{A}}
\newcommand{\sfD}{\mathsf{D}}
\newcommand{\sfE}{\mathsf{E}}
\newcommand{\frakL}{\mathfrak{L}}
\newcommand{\K}{\mathrm{K}}
\newcommand{\rmT}{\mathrm{T}}
\newcommand{\bfv}{\mathbf{v}}
\newcommand{\bfg}{\mathbf{g}}
\newcommand{\frakV}{\mathfrak{V}}
\newcommand{\frakv}{\mathfrak{v}}
\newcommand{\bfn}{\mathbf{n}}
\renewcommand{\o}{\mathfrak{o}}

\newcommand{\aff}{\mathrm{aff}}
\newcommand{\ft}{\mathrm{ft}} %finite type
\newcommand{\fp}{\mathrm{fp}} %finite presentation
\newcommand{\fr}{\mathrm{fr}} %free
\newcommand{\tft}{\mathrm{tft}} %topologically finite type
\newcommand{\tfp}{\mathrm{tfp}} %topologically finite presentation
\newcommand{\tfr}{\mathrm{tfr}} %topologically free
\newcommand{\aft}{\mathrm{aft}}
\newcommand{\lft}{\mathrm{lft}}
\newcommand{\laft}{\mathrm{laft}}
\newcommand{\cpt}{\mathrm{cpt}}
\newcommand{\cproj}{\mathrm{cproj}}
\newcommand{\qc}{\mathrm{qc}}
\newcommand{\qs}{\mathrm{qs}}
\newcommand{\lcmpt}{\mathrm{lcmpt}}
\newcommand{\red}{\mathrm{red}}
\newcommand{\fin}{\mathrm{fin}}
\newcommand{\fd}{\mathrm{fd}} %finite-dimensional
\newcommand{\gen}{\mathrm{gen}}
\newcommand{\petit}{\mathrm{petit}}
\newcommand{\gros}{\mathrm{gros}}
\newcommand{\loc}{\mathrm{loc}}
\newcommand{\glob}{\mathrm{glob}}
%\newcommand{\ringed}{\mathrm{ringed}}
%\newcommand{\qcoh}{\mathrm{qcoh}}
\newcommand{\cl}{\mathrm{cl}}
\newcommand{\et}{\mathrm{\acute{e}t}}
\newcommand{\fet}{\mathrm{f\acute{e}t}}
\newcommand{\profet}{\mathrm{prof\acute{e}t}}
\newcommand{\proet}{\mathrm{pro\acute{e}t}}
\newcommand{\Zar}{\mathrm{Zar}}
\newcommand{\fppf}{\mathrm{fppf}}
\newcommand{\fpqc}{\mathrm{fpqc}}
\newcommand{\orig}{\mathrm{orig}} %overconvergent topology
\newcommand{\smooth}{\mathrm{sm}}
\newcommand{\sh}{\mathrm{sh}}
\newcommand{\op}{\mathrm{op}}
\newcommand{\cop}{\mathrm{cop}}
\newcommand{\open}{\mathrm{open}}
\newcommand{\closed}{\mathrm{closed}}
\newcommand{\geom}{\mathrm{geom}}
\newcommand{\alg}{\mathrm{alg}}
\newcommand{\sober}{\mathrm{sober}}
\newcommand{\dR}{\mathrm{dR}}
\newcommand{\rad}{\mathfrak{rad}}
\newcommand{\discrete}{\mathrm{discrete}}
%\newcommand{\add}{\mathrm{add}}
%\newcommand{\lin}{\mathrm{lin}}
\newcommand{\Krull}{\mathrm{Krull}}
\newcommand{\qis}{\mathrm{qis}} %quasi-isomorphism
\newcommand{\ho}{\mathrm{ho}} %homotopy equivalence
\newcommand{\sep}{\mathrm{sep}}
\newcommand{\unr}{\mathrm{unr}}
\newcommand{\tame}{\mathrm{tame}}
\newcommand{\wild}{\mathrm{wild}}
\newcommand{\nil}{\mathrm{nil}}
\newcommand{\defm}{\mathrm{defm}}
\newcommand{\Art}{\mathrm{Art}}
\newcommand{\Noeth}{\mathrm{Noeth}}
\newcommand{\affd}{\mathrm{affd}}
%\newcommand{\adic}{\mathrm{adic}}
\newcommand{\pre}{\mathrm{pre}}
\newcommand{\coperf}{\mathrm{coperf}}
\newcommand{\perf}{\mathrm{perf}}
\newcommand{\perfd}{\mathrm{perfd}}
\newcommand{\rat}{\mathrm{rat}}
\newcommand{\cont}{\mathrm{cont}}
\newcommand{\dg}{\mathrm{dg}}
\newcommand{\almost}{\mathrm{a}}
%\newcommand{\stab}{\mathrm{stab}}
\newcommand{\heart}{\heartsuit}
\newcommand{\proj}{\mathrm{proj}}
\newcommand{\qproj}{\mathrm{qproj}}
\newcommand{\pd}{\mathrm{pd}}
\newcommand{\crys}{\mathrm{crys}}
\newcommand{\prisma}{\mathrm{prisma}}
\newcommand{\FF}{\mathrm{FF}}
\newcommand{\sph}{\mathrm{sph}}
\newcommand{\lax}{\mathrm{lax}}
\newcommand{\weak}{\mathrm{weak}}
\newcommand{\strict}{\mathrm{strict}}
\newcommand{\mon}{\mathrm{mon}}
\newcommand{\sym}{\mathrm{sym}}
\newcommand{\lisse}{\mathrm{lisse}}
\newcommand{\an}{\mathrm{an}}
\newcommand{\ad}{\mathrm{ad}}
\newcommand{\sch}{\mathrm{sch}}
\newcommand{\rig}{\mathrm{rig}}
\newcommand{\pol}{\mathrm{pol}}
\newcommand{\plat}{\mathrm{flat}}
\newcommand{\proper}{\mathrm{proper}}
\newcommand{\compl}{\mathrm{compl}}
\newcommand{\non}{\mathrm{non}}
\newcommand{\access}{\mathrm{access}}
\newcommand{\comp}{\mathrm{comp}}
\newcommand{\tstructure}{\mathrm{t}} %t-structures
\newcommand{\pure}{\mathrm{pure}} %pure motives
\newcommand{\mixed}{\mathrm{mixed}} %mixed motives
\newcommand{\num}{\mathrm{num}} %numerical motives
\newcommand{\ess}{\mathrm{ess}}
\newcommand{\topological}{\mathrm{top}}
\newcommand{\convex}{\mathrm{cvx}}
\newcommand{\locconvex}{\mathrm{lcvx}}
\newcommand{\ab}{\mathrm{ab}} %abelian extensions
\newcommand{\inj}{\mathrm{inj}}
\newcommand{\surj}{\mathrm{surj}} %coverage on sets generated by surjections
\newcommand{\eff}{\mathrm{eff}} %effective Cartier divisors
\newcommand{\Weil}{\mathrm{Weil}} %weil divisors
\newcommand{\lex}{\mathrm{lex}}
\newcommand{\rex}{\mathrm{rex}}
\newcommand{\AR}{\mathrm{A\-R}}
\newcommand{\cons}{\mathrm{c}} %constructible sheaves
\newcommand{\tor}{\mathrm{tor}} %tor dimension
\newcommand{\semisimple}{\mathrm{ss}}
\newcommand{\connected}{\mathrm{connected}}
\newcommand{\cg}{\mathrm{cg}} %compactly generated
\newcommand{\nilp}{\mathrm{nilp}}
\newcommand{\isg}{\mathrm{isg}} %isogenous
\newcommand{\qisg}{\mathrm{qisg}} %quasi-isogenous
\newcommand{\irr}{\mathrm{irr}} %irreducible represenations
\newcommand{\simple}{\mathrm{simple}} %simple objects
\newcommand{\indecomp}{\mathrm{indecomp}}
\newcommand{\preproj}{\mathrm{preproj}}
\newcommand{\preinj}{\mathrm{preinj}}
\newcommand{\reg}{\mathrm{reg}}
\renewcommand{\ss}{\mathrm{ss}}

%prism custom command
\usepackage{relsize}
\usepackage[bbgreekl]{mathbbol}
\usepackage{amsfonts}
\DeclareSymbolFontAlphabet{\mathbb}{AMSb} %to ensure that the meaning of \mathbb does not change
\DeclareSymbolFontAlphabet{\mathbbl}{bbold}
\newcommand{\prism}{{\mathlarger{\mathbbl{\Delta}}}}

\begin{document}

    \title{Classical limits and universal quantum R-matrices of affine Yangians}
    
    \author{Dat Minh Ha}
    \maketitle
    
    \begin{abstract}
        These are some notes on the construction of certain topological Lie bialgebras - so-called \say{toroidal Lie bialgebras} - which shall be demonstrated to be the classical limits of formal Yangians associated to affine Kac-Moody algebras (outside of types $\sfA_1^{(1)}$ and $\sfA_1^{(2)}$). As toroidal Lie algebras have non-trivial centres, thereby forcing any invariant bilinear form thereon to be degenerate, the construction of the aforementioned toroidal Lie bialgebra structures necessitates enlarging the toroidal Lie algebras to so-called \say{extended toroidal Lie algebras}, on which invariant bilinear form can be non-degenerate, which permits the construction of Manin triples from which Lie bialgebra structures can arise.
    \end{abstract}
    
    {
    \hypersetup{} 
    %\dominitoc
    \tableofcontents %sort sections alphabetically
    \listoftodos
    }

    \section{Introduction}
        Suppose that $\g$ is a finite-dimenisonal simple Lie algebra over $\bbC$.
        
        \subsection{Yangian extended toroidal Lie algebras}
            The first goal for these notes is to establish, in a certain sense, that the formal Yangian $\rmY_{\hbar}(\hat{\g}_{[1]})$ associated to the untwisted affine Kac-Moody algebra $\hat{\g}_{[1]}$ associated to $\g$ (in the sense of \cite[Chapter 7]{kac_infinite_dimensional_lie_algebras}) has a \say{Hopf classical limit}. While it is known that $\rmY_{\hbar}(\hat{\g}_{[1]})$ carries a kind of \say{coproduct} (cf. \cite{guay_nakajima_wendlandt_affine_yangian_coproduct}), behaving very similarly to the Hopf coproduct on the finite-type formal Yangian $\rmY_{\hbar}(\g)$, and also that $\rmY_{\hbar}(\hat{\g}_{[1]})$ is a graded flat deformation of the universal enveloping algebra $\rmU(\tilde{\g}_{[2]}^+)$ of the universal central extension $\tilde{\g}_{[2]}^+$ of $\g_{[2]}^+ := \g[v^{\pm}, t]$, much like how $\rmY_{\hbar}(\g)$ is a graded flat deformation of $\rmU(\g[t])$ (cf. \cite{chari_pressley_quantum_groups}), the problem that we must tackle here stems from the fact that it is not at all obvious how one might endow $\g_{[2]}^+$ with a Lie bialgebra structure quantising to the aforementioned \say{coproduct} on $\rmY_{\hbar}(\hat{\g}_{[1]})$. This issue, in turn, originates directly from the structure of the Lie algebra $\tilde{\g}_{[2]}^+$ itself: unlike $\g[t]$, it has a centre, meaning that any invariant bilinear form on $\tilde{\g}_{[2]}^+$ is necessarily degenerate (cf. question \ref{question: extending_invariant_inner_products_on_multi_loop_to_universal_central_extensions}).

            To remedy this issue, the natural thing to attempt is to enlarge $\tilde{\g}_{[2]}^+$ by an orthogonal complement of the centre $\z_{[2]}^+$ of $\tilde{\g}_{[2]}^+$; the resulting enlarged Lie algebra is denoted by $\hat{\g}_{[2]}^+$ (cf. theorem \ref{theorem: extended_toroidal_lie_algebras}). Interestingly, said orthogonal complement turns out to be spanned by certain derivations on $\bbC[v^{\pm 1}, t]$ (cf. remark \ref{remark: dual_of_toroidal_centres_contains_derivations}), thus making $\hat{\g}_{[2]}^+$ strongly resemble the so-called \say{extended affine Lie algebras} of nullity $2$ in the sense of say, \cite{neher_lectures_on_EALAs}. Furthermore, we demonstrate that the Lie algebra structure on $\hat{\g}_{[2]}^+$ is highly constrained, with the only degree of freedom lying in what the $\z_{[2]}$-summands of the commutators between the aforementioned derivations are (\textit{a priori}, these commutators are elements of $\hat{\g}_{[2]}$, but we shall show that they in fact lie within $\z_{[2]} \oplus \d_{[2]}$; cf. remark \ref{remark: dual_of_toroidal_centres_contains_derivations}). Afterwards, this freedom will be regarded as $2$-cocycles:
                $$\tau \in H^2_{\Lie}(\d_{[2]}, \z_{[2]})$$
            where now we view $\d_{[2]}$ as a Lie subalgebra of $\der_{\bbC}(\bbC[v^{\pm 1}, t^{\pm 1}])$ with the usual commutator bracket. 

            We have also computed the centre of $\hat{\g}_{[2]}$ to show that - somewhat expectedly - this centre is $2$-dimensional, analogous to how the centre of an affine Kac-Moody algebra (to be regarded as an extended affine Lie algebra of nullity $1$) is $1$-dimensional (cf. \cite[Chapter 7]{kac_infinite_dimensional_lie_algebras}). 

        \subsection{Toroidal Lie bialgebras as classical limits of affine Yangians}
            Given the construction of $\hat{\g}_{[2]}^+$, we see that this Lie algebra can support \textit{non-degenerate} invariant bilinear forms on it, meaning that it can carry Lie bialgebra structures, and such a Lie cobracket is constructed in corollary \ref{coro: extended_toroidal_lie_bialgebras}. Now, what is rather interesting is that this cobracket restricts down to a well-defined Lie bialgebra structure on $\tilde{\g}_{[2]}$, which is then shown in theorem \ref{theorem: toroidal_lie_algebras_as_classical_limits_of_formal_affine_yangians} to be the classical limit of the \say{coproduct} on $\rmY_{\hbar}(\hat{\g}_{[1]})$ constructed in \cite{guay_nakajima_wendlandt_affine_yangian_coproduct}. We caution the reader that this is not exactly the same notion of a classical limit as in \cite{etingof_kazhdan_quantisation_1}, since the aforementioned \say{coproduct} $\rmY_{\hbar}(\hat{\g}_{[1]})$ is merely an algebra homomorphism from $\rmY_{\hbar}(\hat{\g}_{[1]})$ to another associative algebra that resembles (a topological completion of) the algebraic tensor product $\rmY_{\hbar}(\hat{\g}_{[1]}) \tensor_{\bbC} \rmY_{\hbar}(\hat{\g}_{[1]})$. 

            One subtlety that must be dealt with when dealing with affine Yangians - as opposed to finite-type Yangians - is the matter of topological completeness. The Lie (bi)algebra $\tilde{\g}_{[2]}^+$ is infinite-dimensional, but luckily it is graded, so one strategy is to attempt to complete the algebraic tensor product $\tilde{\g}_{[2]}^+ \tensor_{\bbC} \tilde{\g}_{[2]}^+$ with respect to the induced grading thereon. The codomain of the Lie cobracket on $\tilde{\g}_{[2]}^+$ ought then to be contained in this grading completion. We should note, also, that unlike $\tilde{\g}_{[2]}^+$, the Lie bialgebra $\hat{\g}_{[2]}^+$ is actually \textit{not} graded, which is why we have not taken up the task of quantising it within these notes. 

        \subsection{Universal R-matrices for affine Yangians}

        \subsection{What have we \textit{not} done ?}
            In \cite{guay_nakajima_wendlandt_affine_yangian_coproduct}, the authors constructed the \say{Hopf coproduct} on $\rmY_{\hbar}(\hat{\g}_{[1]})$ by essentially taking a limit over binary tensor products of representations in the category $\calO$ of $\rmY_{\hbar}(\hat{\g}_{[1]})$; recall that one has tensor products of modules over any Hopf algebra, after all. The decision to involve the category $\calO$ may seem arbitrary, but it is ultimately necessary because objects therein are acted upon locally nilpotently by $\rmY_{\hbar}(\hat{\g}_{[1]})$, which helps us do away with having to topologically complete anything. 
        
            One can therefore make an argument for - instead of pursuing a genuine Hopf coproduct on $\rmY_{\hbar}(\hat{\g}_{[1]})$ - seeking a notion of a category $\calO$ for $\tilde{\g}_{[2]}^+$ that:
            \begin{itemize}
                \item is closed under tensor products over $\bbC$
                \item is a \say{classical limit} of the category $\calO$ of $\rmY_{\hbar}(\hat{\g}_{[1]})$.
            \end{itemize}
            For technical reasons, it might perhaps be better (and even necessary) to seek for such a category $\calO$ for the universal central extension $\tilde{\g}_{[2]}$ of $\g_{[2]} := \g[v^{\pm 1}, t^{\pm 1}]$, which ought somehow to be the \say{classical limit} of some kind of a category $\calO$ for the so-called \say{restricted quantum double} of $\rmY_{\hbar}(\hat{\g}_{[1]})$ (cf. \cite{wendlandt_restricted_quantum_doubles_of_yangians}). In any event, the point here is that the Lie bialgebra structure on $\tilde{\g}_{[2]}^+$ can perhaps be realised as a limit over tensor products of representations in the category $\calO$ thereof. An outstanding difficulty in accomplishing this task is the current lack of understanding of weight modules for either $\tilde{\g}_{[2]}^+$ or $\tilde{\g}_{[2]}$ (or for that matter, for affine Yangians and their restricted quantum doubles).

            Yet another argument in favour of pursuing the aforementioned notion of a category $\calO$ for $\tilde{\g}_{[2]}^+$ comes from symplectic geometry. \todo{Something about quantisations of Grassmannian slices ...}

    \section{Extended toroidal Lie (bi)algebras}
    \begin{convention} \label{conv: a_fixed_finite_dimensional_simple_lie_algebra}
        Throughout this section, we fix a finite-dimensional simple Lie algebra $\g$ over $\bbC$, equipped with a symmetric and non-degenerate invariant $\bbC$-bilinear form $(-, -)_{\g}$. It is known that such a bilinear form is unique up to $\bbC^{\x}$-multiples, so for all intents and purposes, it can be assumed to be the Killing form, though this assumption is not necessary. 

        Suppose also that $\g$ is equipped with a basis $\{x_i\}_{1 \leq i \leq \dim_{\bbC} \g}$ and with respect to $(-, -)_{\g}$, we identify a dual basis $\{x_i^*\}_{1 \leq i \leq \dim_{\bbC} \g}$. Recall that the universal Casimir element/canonical element of $\g$ is:
            $$\sfr_{\g} := \sum_{1 \leq i \leq \dim_{\bbC} \g} x_i \tensor x_i^* \in \g \tensor_{\bbC} \g$$
        and recall that $\sfr_{\g}$ is independent of what we choose the basis vectors $x_i$ to be.

        Eventually, we will also be concerned with the Dynkin diagram associated to the root system of $\g$. Let us denote this by:
            $$\Gamma := (\Gamma_0, \Gamma_1)$$
        wherein $\Gamma_0$ means the (finite) set of vertices and $\Gamma_1$ means the set of undirected edges between said vertices. 

        The set of all roots and respectively, positive/negative roots, and simple roots of $\g$ shall be denoted by:
            $$\Phi, \Phi^{\pm}, \Phi^{\circ}$$
        The set $\Phi^{\circ}$ is $\Z$-linearly independent and its $\Z$-span:
            $$Q := \Z \Phi^{\circ}$$
        is typically referred to as the root lattice of $\g$. Recall also that there is a set of fundamental weights: if we write:
            $$\check{\Phi}$$
        for the set of coroots of $\g$, then the so-called weight lattice of $\g$ shall be given by\footnote{We avoid the usual $\Delta$ notation, as we would like to reserve this symbol for a coproduct construction on affine Yangians.}:
            $$\Pi := \Hom_{\Z}(\check{Q}, \Z), \check{Q} := \Z\check{\Phi}$$
        inside which lies the set of fundamental weights, whose elements are dual to those of $\check{\Phi}^{\circ}$ (i.e. dual to simple coroots) with respect to $(-, -)_{\g}$\footnote{Which we might as well normalise so that $(\alpha_j, \check{\alpha}_j)_{\g} = 2$ for every $j \in \Gamma_0$, and hence the fundamental weights $\lambda_i$ will be simply be subjected to the relation $\delta_{ij} = 2 \frac{(\lambda_i, \check{\alpha}_i)_{\g}}{(\alpha_j, \check{\alpha}_j)_{\g}} = (\lambda_i, \check{\alpha}_i)_{\g}$.}.
    \end{convention}

    \begin{convention}
        Throughout, we shall use $(-)^{\star}$ to denote graded duals. 
    \end{convention}

    \begin{convention}
        If $k$ is a commutative ring and $A$ is a $k$-algebra, and if $L$ is a Lie algebra over $k$, then the default Lie algebra structure on the $k$-module $L \tensor_k A$ shall be the one given by extension of scalars, i.e.:
            $$[x \tensor a, y \tensor b]_{L \tensor_k A} := [x, y]_L \mu_A(a \tensor b)$$
        $L \tensor_k A$ is usually regarded as Lie algebra over $k$ instead of over $A$.  
    \end{convention}

    \begin{convention} \label{conv: multiloop_algebras}
        We fix once and for all the following notations:
        \begin{itemize}
            \item $A_{[n]} := \bbC[v_1^{\pm 1}, ..., v_n^{\pm 1}]$, $A_{[2]}^+ := \bbC[v_1^{\pm 1}, ..., v_n], A_{[n]}^- := v_n^{-1}\bbC[v_1^{\pm 1}, ..., v_n^{-1}]$, and in particular, when $n \leq 2$, let us set $v_1 := v$ and $v_2 := t$ per the usual Yangian conventions (cf. e.g. \cite{wendlandt_formal_shift_operators_on_yangian_doubles});
            \item $\g_{[n]} := \g \tensor_{\bbC} A_{[n]}$, $\g_{[n]}^{\pm} := \g \tensor_{\bbC} A_{[n]}^{\pm}, \tilde{\g}_{[n]}^{\pm} := \uce(\g_{[n]}^{\pm})$ (with $\uce$ meaning \say{universal central extension}\footnote{These are indeed universal \textit{a priori}, thanks to the fact that the Lie algebras $\g_{[n]}, \g_{[n]}^{\pm}$ are perfect.});
            \item $\z_{[n]}^{\pm} := \z(\tilde{\g}_{[n]}^{\pm})$ (cf. remark \ref{remark: centres_of_dual_toroidal_lie_algebras}).
        \end{itemize}
    \end{convention}

    \subsection{A Lie bialgebra structure on the \texorpdfstring{$2$}{}-loop Lie algebra \texorpdfstring{$\g_{[2]}$}{}}
        \begin{definition} \label{def: residue_form_on_loop_algebra}
            For $v$ a formal variable, we can extend the invariant inner product $(-, -)_{\g}$ to the following pairing on $\g_{[1]}$ by defining:
                $$(x f(v), y g(v))_{\g_{[1]}} := (x, y)_{\g} \Res_{v = 0}( v^{-1} f(v) g(v) )$$
            for all $x, y \in \g$ and all $f(v), g(v) \in A_{[1]}$, and recall that:
                $$\Res_{v = 0}\left( \sum_{n \in \Z} a_n v^n \right) := a_{-1}$$
            More algebraically\footnote{So that the definition would work still when we replace $\bbC$ with a general algebraically closed field of characteristic $0$.}, we can define this as:
                $$(x v^m, y v^n)_{\g_{[1]}} := (x, y)_{\g} \delta_{m + n, 0}$$
        \end{definition}
        \begin{definition} \label{def: residue_form_on_multiloop_algebra}
            Now, let $(-, -)_{\g_{[1]}}$ be as in definition \ref{def: residue_form_on_loop_algebra}. This can be extended furthermore to $\g_{[2]}$ in the following manner: for all $X(v), Y(v) \in \g_{[1]}$ and all $f(t), g(t) \in \bbC[t^{\pm 1}]$, define:
                $$(X(v) f(t), Y(v) g(t))_{\g_{[2]}} := -(X(v), Y(v))_{\g_{[1]}} \Res_{t = 0}( f(t) g(t) )$$
            More algebraically, we can define this as:
                $$(X(v) f(t), Y(v) g(t))_{\g_{[2]}} := -(X(v), Y(v))_{\g_{[1]}} \delta_{m + n, -1}$$
            In both cases, the appearance of the minus sign is a crucial choice for our purposes. 
        \end{definition}

        \begin{question} \label{question: multiloop_lie_bialgebras}
            \begin{enumerate}
                \item Verify that $(-, -)_{\g_{[2]}}$ is an invariant and non-degenerate symmetric $\bbC$-bilinear form on $\g_{[2]}$.
                \item Show that by equpping $\g_{[2]}$ with the invariant inner product $(-, -)_{\g_{[2]}}$, the following triple of Lie algebras becomes a well-defined Manin triple:
                    $$(\g_{[2]}, \g_{[2]}^+, \g_{[2]}^-)$$
                \item Find a formula for the canonical element $\sfr_{\g[v^{\pm 1}, t} \in \g_{[2]}^+ \hattensor_{\bbC} \g_{[2]}^+$ with respect to the restriction of $(-, -)_{\g_{[2]}}$ to $\g[v^{\pm 1}, t^{-1}] \x \g_{[2]}^+$.
                \item Find the Lie bialgebra structure on $\g_{[2]}^+$ arising from the Manin triple in 2.
            \end{enumerate}
        \end{question}
            \begin{proof}
                \begin{enumerate}
                    \item The symmetry and bilinearity of $(-, -)_{\g_{[2]}}$ are clear from the construction of this bilinear pairing as in definition \ref{def: residue_form_on_multiloop_algebra}. $(-, -)_{\g_{[2]}}$-invariance follows from the $\g$-invariance of $(-, -)_{\g}$, which is by hypothesis. Finally, non-degeneracy follows from the non-degeneracy of $(-, -)_{\g}$ (also by hypothesis) as well as the non-degeneracy of the residual pairings on $A_{[1]}$ (as in definition \ref{def: residue_form_on_loop_algebra}) and on $\bbC[t^{\pm 1}]$ (as in definition \ref{def: residue_form_on_multiloop_algebra}); to see that the latter point holds, simply note that there exists no $m \in \Z$ such that $\delta_{m + n, - 1} = 0$ (respectively, such that $\delta_{m + n, 0}$) for all $n \in \Z$.
                    \item It is not hard to see that: with respect to $(-, -)_{\g_{[2]}}$ as in 1, one has that:
                        $$(\g_{[2]}^+)^{\star} \cong \bigoplus_{m \in \Z, p \in \Z_{\geq 0}} (\g v^m t^p)^* \cong \bigoplus_{m \in \Z, p \in \Z_{\geq 0}} \g v^{-m} t^{-p - 1} \cong \g_{[2]}^-$$
                    with respect to the invariant inner product $(-, -)_{\g_{[2]}}$. It is also easy to see that:
                        $$\g_{[2]} \cong \g_{[2]}^+ \oplus \g_{[2]}^-$$
                    Note also that $\g_{[2]} \supset \g_{[2]}^+, \g_{[2]}^-$ are Lie subalgebras. Finally, to prove that $(-, -)_{\g_{[2]}}$ pairs the vector subspaces $\g_{[2]}^+, \g_{[2]}^-$ isotropically, simply that there does not exist any $p, q \geq 0$ or $p, q \leq -1$ simultaneously so that:
                        $$\delta_{p + q, -1} = 0$$
                    which means that:
                        $$(\g_{[2]}^-, \g_{[2]}^-)_{\g_{[2]}} = (\g_{[2]}^+, \g_{[2]}^+)_{\g_{[2]}} = 0$$
                    \item It will be convenient for us to make the identification of topological vector spaces:
                        $$\g_{[2]}^+ \hattensor_{\bbC} \g[v^{\pm 1}, t^{-1}] \cong \g[v_2^{\pm 1}, t_1] \hattensor_{\bbC} \g[v^{\pm 1}, t_2^{-1}]$$
                    Also, let us fix the basis:
                        $$\{X_{i, m, p}\}_{1 \leq i \leq \dim_{\bbC} \g, (m, p) \in \Z^2} := \{x_i v^m t^p\}_{1 \leq i \leq \dim_{\bbC} \g, (m, p) \in \Z^2}$$
                    for $\g_{[2]}$. It is easy to see that the graded dual of this basis with respect to the invariant inner product $(-, -)_{\g_{[2]}}$ is:
                        $$\{X_{i, m, p}^{\star}\}_{1 \leq i \leq \dim_{\bbC} \g, (m, p) \in \Z^2} := \{x_i^* v^{-m} t^{-p - 1}\}_{1 \leq i \leq \dim_{\bbC} \g, (m, p) \in \Z^2}$$
                    
                    By definition, the canonical element $\sfr_{\g_{[2]}^+} \in \g[v_2^{\pm 1}, t_1] \hattensor_{\bbC} \g[v^{\pm 1}, t_2^{-1}]$ is given by:
                        $$\sfr_{\g_{[2]}^+} := \sum_{1 \leq i \leq \dim_{\bbC} \g} \sum_{(m, p) \in \Z \x \Z_{\geq 0}} X_{i, m, p} \tensor X_{i, m, p}^{\star}$$
                    As such, we have that:
                        $$
                            \begin{aligned}
                                \sfr_{\g_{[2]}^+} & := \sum_{1 \leq i \leq \dim_{\bbC} \g} \sum_{m \in \Z} \sum_{p \in \Z_{\geq 0}} x_i v_1^m t_1^p \tensor x_i^* v_2^{-m} t_2^{-p - 1}
                                \\
                                & = -\left( \sum_{1 \leq i \leq \dim_{\bbC} \g} x_i \tensor x_i^* \right) \left( v_2 \sum_{m \in \Z} v_1^m v_2^{-m - 1} \right) \left( t_2^{-1} \sum_{p \in \Z_{\geq 0}} (t_1 t_2^{-1})^p \right)
                                \\
                                & = -\sfr_{\g} v_2 \1(v_1, v_2) \1^+(t_1, t_2)
                            \end{aligned}
                        $$
                    wherein:
                        $$\1(z, w) := \sum_{m \in \Z} z^m w^{-m - 1} \in \bbC[\![z^{\pm 1}, w^{\pm 1}]\!]$$ 
                    is the $2$-variable formal Dirac distribution, and we obtained:
                        $$t_2^{-1} \sum_{p \in \Z_{\geq 0}} (t_1 t_2^{-1})^p = \frac{1}{t_2 - t_1} =: \1^+(t_1, t_2)$$
                    simply by formally evaluating the geometric series. 
                    \item Let us keep the identification:
                        $$\g_{[2]}^+ \hattensor_{\bbC} \g[v^{\pm 1}, t^{-1}] \cong \g[v_2^{\pm 1}, t_1] \hattensor_{\bbC} \g[v^{\pm 1}, t_2^{-1}]$$
                    From \cite[pp. 5]{etingof_kazhdan_quantisation_1}\footnote{Actually, this citation is not quite right, since the result was stated for finite-dimensional Manin triples only. However, since we're dealing with graded duals with finite-dimensional graded components, I believe the analogous result still holds. Of course, I should write this down carefully at some point.}, we know that the Lie bialgebra structure (say, $\delta_{\g_{[2]}^+}$) on $\g_{[2]}$ is given at any $X(v, t) \in \g_{[2]}$ by:
                        $$\delta_{\g_{[2]}^+}( X(v, t) ) = [X(v_1, t_1) \tensor 1 + 1 \tensor X(v_2, t_2), \sfr_{\g_{[2]}^+}]$$
                    When $X(v, t) := x v^m t^p$ for some $x \in \g, m \in \Z, t \in \Z_{\geq 0}$, this can written out more explicitly as follows:
                        $$
                            \begin{aligned}
                                \delta_{\g_{[2]}^+}( x v^m t^p ) & = -\left[x v_1^m t_1^p \tensor 1 + 1 \tensor x v_2^m t_2^p, \sfr_{\g} v_2 \1(v_1, v_2) \1^+(t_1, t_2)\right]
                                \\
                                & = -[x \tensor 1 + 1 \tensor x, \sfr_{\g}] \cdot v_1^m t_1^p \cdot v_2^m t_2^p \cdot v_2 \1(v_1, v_2) \1^+(t_1, t_2)
                            \end{aligned}
                        $$
                \end{enumerate}
            \end{proof}

    \subsection{Failing to extend the Lie bialgebra structure to the universal central extension \texorpdfstring{$\tilde{\g}_{[2]}$}{}}
        \begin{question} \label{question: extending_invariant_inner_products_on_multi_loop_to_universal_central_extensions}
            \begin{enumerate}
                \item Prove that there is a unique invariant symmetric bilinear form $(-, -)_{\tilde{\g}_{[2]}}$ on $\t$ whose restriction to $\g_{[2]}$ coincides with $(-, -)_{\g_{[2]}}$.
                \item Find a Lie subalgebra $\tilde{\g}_{[2]}^- \subset \t$ such that:
                    $$\tilde{\g}_{[2]}\cong \tilde{\g}_{[2]}^+\oplus \tilde{\g}_{[2]}^-$$
                and such that $\tilde{\g}_{[2]}^-$ is paired isotropically with $\s$ by $(-, -)_{\tilde{\g}_{[2]}}$. 
                \item Why is the triple:
                    $$(\tilde{\g}_{[2]}, \tilde{\g}_{[2]}^+, \tilde{\g}_{[2]}^-)$$
                with $\tilde{\g}_{[2]}$ being equipped with $(-, -)_{\tilde{\g}_{[2]}}$ not a Manin triple ?
            \end{enumerate}
        \end{question}
            \begin{proof}
                \begin{enumerate}
                    \item Suppose that $B$ is any invariant inner product on $\t$ and fix an element $Z \in \z_{[2]}$. This gives us:
                        $$B([X, Y], Z) = B(X, [Y, Z]) = B(X, 0) = 0$$
                    for all $X, Y \in \t$. As such, the sought-for unique invariant inner product on $\t$ induced by $(-, -)_{\g_{[2]}}$, whose restriction to $\g_{[2]} \subset \t$ coincides with $(-, -)_{\g_{[2]}}$, must be determined by:
                        $$(X, Z)_{\tilde{\g}_{[2]}} = 0, (Z, Z)_{\tilde{\g}_{[2]}} = 0$$
                    for all $X \in \t$ and all $Z \in \z_{[2]}$.
                    \item One thing that we are able to gather from 1 is that, with respect to $(-, -)_{\tilde{\g}_{[2]}}$, the centre $\z_{[2]}$ is orthogonally complementary to $\g_{[2]}$. With this in mind, we claim that:
                        $$\tilde{\g}_{[2]}^- \cong \g_{[2]}^- \oplus \z_{[2]}^-$$
                    wherein $\z_{[2]}^-$ is such that:
                        $$\z_{[2]} \cong \z_{[2]}^+ \oplus \z_{[2]}^-$$
                    and note that $\z_{[2]}^-$ must exist due to $\s$ being a Lie subalgebra of $\t$ and hence $\z_{[2]}^+$ being a Lie subalgebra of $\z_{[2]}$ (namely, one has that $\z_{[2]}^+ = \z_{[2]} \cap \s$). To see that this is indeed that the Lie subalgebra of $\t$ that we are after, firstly note that because:
                        $$(-, -)_{\tilde{\g}_{[2]}}|_{\g_{[2]}} = (-, -)_{\g_{[2]}}$$
                    and because it has been shown that $(-, -)_{\g_{[2]}}$ pairs $\g_{[2]}^+$ and $\g_{[2]}^-$ isotropically as subspaces of $\g_{[2]}$, the only thing to demonstrate is that $(-, -)_{\tilde{\g}_{[2]}}$ pairs elements of $\z_{[2]}^+$ and $\z_{[2]}^-$ isotropically with one another in the sense that:
                        $$(\z_{[2]}^+, \z_{[2]}^+)_{\tilde{\g}_{[2]}} = (\z_{[2]}^-, \z_{[2]}^-)_{\tilde{\g}_{[2]}} = 0$$
                    This is directly due to the fact that elements of $\z_{[2]}^+$ and likewise, those of $\z_{[2]}^-$, are central as elements of $\t$. Lastly, one verifies that, one indeed has that:
                        $$\tilde{\g}_{[2]}^+\oplus \tilde{\g}_{[2]}^- \cong ( \g_{[2]}^+ \oplus \z_{[2]}^+ ) \oplus ( \g_{[2]}^- \oplus \z_{[2]}^- ) \cong \g_{[2]} \oplus \z_{[2]} \cong \t$$
                    \item $(\tilde{\g}_{[2]}, \tilde{\g}_{[2]}^+, \tilde{\g}_{[2]}^-)$ is not a Manin triple (nor a graded Manin triple, for that matter) due to the simple fact that the non-zero vector space $\z_{[2]}$ is contained entirely in $\Rad (-, -)_{\tilde{\g}_{[2]}} := \{Z \in \tilde{\g}_{[2]}\mid \forall X \in \t: (X, Z)_{\tilde{\g}_{[2]}} = 0\}$. This implies that the invariant inner product $(-, -)_{\tilde{\g}_{[2]}}$ on $\t$ is \textit{degenerate}, thereby violating the definition of Manin triples. 

                    Note that we have not even checked whether or not $\tilde{\g}_{[2]}^-$ is actually a Lie subalgebra of $\t$ or merely a vector subspace. This will turn out to be true, but we defer this discussion to question \ref{question: toroidal_dual}. 
                \end{enumerate}
            \end{proof}
        \begin{remark}[What exactly is $\z_{[2]}^-$ ?] \label{remark: centres_of_dual_toroidal_lie_algebras}
            In attempting to answer question \ref{question: extending_invariant_inner_products_on_multi_loop_to_universal_central_extensions}, we relied on the existence of an abstract vector subspace $\z_{[2]}^-$ of $\z_{[2]}$ specified by the condition that:
                $$\z_{[2]} \cong \z_{[2]}^+ \oplus \z_{[2]}^-$$
            Let us now spend a bit of time on giving an explicit description of $\z_{[2]}^-$. 

            Suppose that $k$ is an arbitrary commutative ring. Recall firstly that, should $\a$ be a perfect Lie algebra over $k$ (i.e. a Lie algebra such that $\a = [\a, \a]$) with a non-degenerate invariant inner product $(-, -)_{\a}$, then not only does $\a_A := \a \tensor_k A$ admit a universal central extension $\uce(\a_A)$ for any commutative $k$-algebra $A$ (i.e. one that is initial in the category of all central extensions of $\a$) - and recall also that any universal central extension must split - but also, that there is the following explicit description of $\uce(\a_A)$ due to Kassel (see \cite[Corollary 3.5]{kassel_universal_central_extensions_of_lie_algebras}):
                $$\uce(\a_A) \cong \a_A \oplus \bar{\Omega}^1_{A/k}$$
            with $\bar{\Omega}^1_{A/k} := \coim d_{A/k} := \Omega^1_{A/k}/d_{A/k}(A)$ being the coimage of the universal K\"ahler differential map $d_{A/k}: A \to \Omega^1_{A/k}$; if we denote:
                $$
                    \begin{tikzcd}
                    A & {\Omega^1_{A/k}} \\
                    & {\bar{\Omega}^1_{A/k}}
                    \arrow[two heads, from=1-2, to=2-2]
                    \arrow["{d_{A/k}}", from=1-1, to=1-2]
                    \arrow["{\bar{d}_{A/k}}"', from=1-1, to=2-2]
                    \end{tikzcd}
                $$
            then the Lie bracket on $\uce(\a_A)$ with respect to Kassel's realisation shall be given by:
                $$
                    \begin{aligned}
                        [ x \tensor a, y \tensor b ]_{\uce(\a_A)} & = [ X \tensor a, Y \tensor b ]_{\a_A} + (x, y)_{\a} b \bar{d}_{A/k}(a)
                        \\
                        & = [X, Y]_{\a} ab + (x, y)_{\a} a \bar{d}_{A/k}(b)
                    \end{aligned}
                $$
            for all $x, y \in \a$ and all $a, b \in A$.

            We now specialise to the case wherein $k \cong \bbC$, $\a = \g$, and $(-, -)_{\a} = (-, -)_{\g}$ as in convention \ref{conv: a_fixed_finite_dimensional_simple_lie_algebra} and for the moment, let us consider:
                $$A \in \{ A_{[n]}^+ := \bbC[v_1, ..., v_n], A_{[n]}^{\pm} := \bbC[v_1^{\pm 1}, ..., v_n^{\pm 1}] \}$$
            and also, let us abbreviate:
                $$\Omega_{[n]} := \Omega^1_{A_{[n]}/\bbC}, \Omega^{\pm}_{[n]} := \Omega^1_{A_{[n]}^{\pm}/\bbC}$$
                $$\bar{\Omega}_{[n]} := \bar{\Omega}^1_{A_{[n]}/\bbC}, \bar{\Omega}_{[n]}^{\pm} := \bar{\Omega}^1_{A_{[n]}^{\pm}/\bbC}$$
                $$d := d_{A/k}, \bar{d} := \bar{d}_{A/k}$$
            Eventually, we will specialise to the case $n = 2$. \textit{A priori}, both $\Omega_{[n]}$ and $\Omega^{\pm}_{[n]}$ are free and of rank $n$ over $A_{[n]}^+$ and $A_{[n]}^{\pm}$ respectively, specifically generated by the basis elements:
                $$d(v_j)$$
            In turn, this implies that the $A_{[n]}^+$-module $\bar{\Omega}_{[n]}$ and the $A_{[n]}^{\pm}$-module $\bar{\Omega}_{[n]}^{\pm}$ are both generated by the basis elements:
                $$\bar{d}(v_j)$$
            that are subjected to the following relation:
                $$0 = \bar{d}( v_1^{m_1} ... v_n^{m_n} ) = \sum_{1 \leq j \leq n} m_j v_1^{m_1} ... v_j^{m_j - 1} ... v_n^{m_n} \bar{d}(v_j)$$
            From this, one infers that the elements:
                $$m_j^{-1} v_1^{m_1} ... v_j^{m_j - 1} ... v_n^{m_n} \bar{d}(v_j)$$
            form a basis for $\bar{\Omega}^+_{[n]}$ and $\bar{\Omega}_{[n]}^{\pm}$ as $\bbC$-vector spaces. 

            When $n = 2$, we can write things out more explicitly: $\z_{[2]} \cong \bar{\Omega}_{[2]}$ now decomposes as a $\bbC$-vector space in the following manner:
                $$\z_{[2]} \cong ( \bigoplus_{(r, s) \in \Z^2} \bbC K_{r, s}) \oplus \bbC c_v \oplus \bbC c_t$$
            and $\z_{[2]}^+ \cong \bar{\Omega}_{[2]}^+$ decomposes in the following manner:
                $$\z_{[2]}^+ \cong ( \bigoplus_{(r, s) \in \Z \x \Z_{> 0}} \bbC K_{r, s}) \oplus \bbC c_v$$
            wherein:
                $$
                    K_{r, s} :=
                    \begin{cases}
                        \text{$\frac1s v^{r - 1} t^s \bar{d}(v)$ if $(r, s) \in \Z \x (\Z \setminus \{0\})$}
                        \\
                        \text{$-\frac1r v^r t^{-1} \bar{d}(t)$ if $(r, s) \in (\Z \setminus \{0\}) \x \{0\}$}
                        \\
                        \text{$0$ if $(r, s) = (0, 0)$}
                    \end{cases}
                $$
                $$c_v := v^{-1} \bar{d}(v), c_t := t^{-1} \bar{d}(t)$$
            In fact, any element of the form:
                $$v^m t^p \bar{d}(v^n t^q) \in \z_{[2]}$$
            can be written in terms of the basis vectors $K_{r, s}, c_v, c_t$ in the following manner:
                $$v^m t^p \bar{d}(v^n t^q) = \delta_{(m, p) + (n, q), (0, 0)} ( n c_v + q c_t ) + (np - mq) K_{m + n, p + q}$$

            From the above and from the requirement on $\z_{[2]}^-$ that:
                $$\z_{[2]} \cong \z_{[2]}^+ \oplus \z_{[2]}^-$$
            one sees immediately that:
                $$\z_{[2]}^- \cong ( \bigoplus_{(r, s) \in \Z \x \Z_{\leq 0}} \bbC K_{r, s}) \oplus \bbC c_t$$
        \end{remark}
        \begin{question} \label{question: toroidal_dual}
            Verify that $\tilde{\g}_{[2]}^-$ is a well-defined Lie subalgebra of $\t$.
        \end{question}
            \begin{proof}
                We now know that:
                    $$\tilde{\g}_{[2]}^- \cong \g_{[2]}^- \oplus \left( ( \bigoplus_{(r, s) \in \Z \x \Z_{\leq 0}} \bbC K_{r, s}) \oplus \bbC c_t \right)$$
                (with notations as in remark \ref{remark: centres_of_dual_toroidal_lie_algebras}), so the verification can be carried out by firstly considering the following, for any $X(v, t), Y(v, t) \in \g_{[2]}^-$ and any $Z, Z' \in \z_{[2]}^-$:
                    $$
                        \begin{aligned}
                            [ X(v, t) + Z, Y(v, t) + Z' ]_{\tilde{\g}_{[2]}} & = [ X(v, t), Y(v, t) ]_{\tilde{\g}_{[2]}} + [ Z, Y(v, t) ]_{\tilde{\g}_{[2]}} + [X(v, t) + Z, Z']_{\tilde{\g}_{[2]}}
                            \\
                            & = [ X(v, t), Y(v, t) ]_{\tilde{\g}_{[2]}}
                        \end{aligned}
                    $$
                wherein the equalities hold thanks to the elements $Z, Z'$ being central inside $\t$, and then, without any loss of generality, we consider secondly the following for:
                    $$X(v, t) := x f(v, t), Y(v, t) := y g(v, t)$$
                for some $x, y \in \g$ and $f(v, t), g(v, t) \in t^{-1}\bbC[v^{\pm 1}, t^{-1}]$:
                    $$
                        \begin{aligned}
                            [ X(v, t), Y(v, t) ]_{\tilde{\g}_{[2]}} & = [ x f(v, t), y g(v, t) ]_{\tilde{\g}_{[2]}}
                            \\
                            & = [x, y]_{\g} f(v, t) g(v, t) + (x, y)_{\g} f(v, t) \bar{d}( g(v, t) )
                        \end{aligned}
                    $$
                This is clearly an element of $\t$, in light of how the Lie bracket $[-, -]_{\tilde{\g}_{[2]}}$ is given, so we are done. 
            \end{proof}

        \begin{remark}[The $\Z$-grading on $\tilde{\g}_{[2]}$] \label{remark: Z_gradings_on_toroidal_lie_algebras}
            Throughout, the $\Z$-grading on $\tilde{\g}_{[2]}$ as well as those on its Lie subalgebras $\tilde{\g}_{[2]}^{\pm}$ will play crucial roles in many computations that we will end up performing (cf. remark \ref{remark: total_degrees_of_classical_yangian_R_matrices} and theorem \ref{theorem: toroidal_lie_bialgebras} in particular), so let us spend some time describing it in details before moving on.

            If $k$ is an arbitrary commutative ring and $A$ is a $\Z$-graded commutative $k$-algebra, say:
                $$A := \bigoplus_{n \in \Z} A_n$$
            and if $\a$ is a perfect Lie algebra over $k$, then $\a_A$ will also be $\Z$-graded, specifically in the following manner:
                $$\a_A := \a \tensor_k A \cong \bigoplus_{n \in \Z} \a \tensor_k A_n$$
            and for convenience, let us write $\a_{A_n} := \a \tensor_k A_n$ for each $n \in \Z$. This grading on $\a_A$ actually extends to the whole of $\uce(\a_A)$, though to be able to describe this extension in details, let us firstly how the $A$-module $\Omega^1_{A/k}$ itself is constructed. To this end, recall that the $A$-module $\Omega^1_{A/k}$ is generated by the set:
                $$\{d_{A/k}(a)\}_{a \in A}$$
            subjected to the relations:
                $$d_{A/k}(ab) - a d_{A/k}(b) - d_{A/k}(a) b = 0$$
            defined for all $a, b \in A$. From this, one infers that there is an induced $\Z$-grading on $\Omega^1_{A/k}$ given by:
                $$\deg d_{A/k}(ab) = \deg a d_{A/k}(b) = \deg d_{A/k}(a) b = \deg a + \deg b - 1$$
            for all $a, b \in A$. Inside $\Omega^1_{A/k}$, now viewed as a $k$-module, one has the $k$-submodule $\im d_{A/k}$, which is also $\Z$-graded: the grading is given like above, namely:
                $$\deg d(a) = \deg a - 1$$
            This $\Z$-grading induces another one on $\bar{\Omega}^1_{A/k}$, given by:
                $$\deg \bar{d}_{A/k}(ab) = \deg a \bar{d}_{A/k}(b) = \deg \bar{d}_{A/k}(a) b = \deg a + \deg b - 1$$
            for all $a, b \in A$.

            Now, let us focus once more on the case:
                $$A := A_{[2]}$$
            (cf. remark \ref{remark: centres_of_dual_toroidal_lie_algebras}) wherein the relevant $\Z$-grading is given by:
                $$\deg v := 0, \deg t := 1$$
            Since we know that the basis elements of $\z_{[2]}$ are given by:
                $$
                    K_{r, s} :=
                    \begin{cases}
                        \text{$\frac1s v^{r - 1} t^s \bar{d}(v)$ if $(r, s) \in \Z \x (\Z \setminus \{0\})$}
                        \\
                        \text{$-\frac1r v^r t^{-1} \bar{d}(t)$ if $(r, s) \in (\Z \setminus \{0\}) \x \{0\}$}
                        \\
                        \text{$0$ if $(r, s) = (0, 0)$}
                    \end{cases}
                $$
                $$c_v := v^{-1} \bar{d}(v), c_t := t^{-1} \bar{d}(t)$$
            (cf. \textit{loc. cit.}) their respective degrees with respect to the $\Z$-grading on $\bar{\Omega}_{[2]} \cong \z_{[2]}$ are:
                $$
                    \deg K_{r, s} =
                    \begin{cases}
                        \text{$s - 1$ if $(r, s) \in \Z \x (\Z \setminus \{0\})$}
                        \\
                        \text{$-1$ if $(r, s) \in (\Z \setminus \{0\}) \x \{0\}$}
                        \\
                        \text{$0$ if $(r, s) = (0, 0)$}
                    \end{cases}
                $$
                $$\deg c_v = \deg c_t = -1$$
        \end{remark}

    \subsection{Extending \texorpdfstring{$\tilde{\g}_{[2]}$}{} to fix degeneracy; Yangian extended toroidal Lie algebras} \label{subsection: extended_toroidal_lie_algebras}
        We now attempt to fix the issue whereby any bilinear form on $\tilde{\g}_{[2]}:= \uce(\g_{[2]})$ is necessarily degenerate. We do this by formally introducing a \say{complementary} vector space $\d_{[2]}$ whose elements shall pair non-degenerately with those of $\z_{[2]}$. 
        \begin{convention} \label{conv: orthogonal_complement_of_toroidal_centres}
            From now on, $\d_{[2]}$ shall be the $\bbC$-vector space:
                $$\d_{[2]} \cong ( \bigoplus_{(r, s) \in \Z^2} \bbC D_{r, s} ) \oplus \bbC D_v \oplus \bbC D_t$$
            such that we can endow:
                $$\hat{\g}_{[2]} := \tilde{\g}_{[2]}\oplus \d_{[2]}$$
            with a $\bbC$-bilinear form $(-, -)_{\hat{\g}_{[2]}}$ such that:
            \begin{itemize}
                \item the elements $D_{r, s}, D_v, D_t$ are graded-dual with respect to $(-, -)_{\hat{\g}_{[2]}}$ to the elements $K_{r, s}, c_v, c_t$, respectively;
                \item $(\g_{[2]}, \z_{[2]} \oplus \d_{[2]})_{\hat{\g}_{[2]}} := 0$;
                \item $(\z_{[2]}, \z_{[2]})_{\hat{\g}_{[2]}} = (\d_{[2]}, \d_{[2]})_{\hat{\g}_{[2]}} := 0$;
                \item $(-, -)_{\hat{\g}_{[2]}}|_{\Sym^2_{\bbC}(\g_{[2]})} := (-, -)_{\g_{[2]}}$
            \end{itemize}
        \end{convention}
        \begin{convention}
            Let us assume also that, should there be a Lie algebra structure $[-, -]_{\hat{\g}_{[2]}}$ on $\hat{\g}_{[2]}$ with respect to which $\t$ becomes a Lie subalgebra of $\hat{\g}_{[2]}$, i.e.
                $$[-, -]_{\hat{\g}_{[2]}}|_{\bigwedge^2 \tilde{\g}_{[2]}} := [-, -]_{\tilde{\g}_{[2]}}$$
            then the bilinear form $(-, -)_{\hat{\g}_{[2]}}$ will be \textit{invariant} with respect to $[-, -]_{\hat{\g}_{[2]}}$.

            Even though this will turn out to be the case, we do not assume from the beginning that:
                $$\hat{\g}_{[2]} \cong \tilde{\g}_{[2]}\rtimes \d_{[2]}$$
            i.e. we do not need the assumption that $\t$ is a $\d_{[2]}$-module. This is because \textit{a priori}, we have no knowledge of the Lie algebra structure on $\d_{[2]}$.
        \end{convention}

        \begin{remark}[How does $\d_{[2]}$ act on $\g_{[2]}$ ?] \label{remark: derivation_action_on_multiloop_algebras}
            Let us firstly see how elements of $\d_{[2]}$ might act on those of $\g_{[2]}$, with respect to some Lie bracket $[-, -]_{\hat{\g}_{[2]}}$. 

            To this end, fix $x, y \in \g$, $(m, p), (n, q) \in \Z^2$, along with some $D \in \d_{[2]}$, and then consider the following:
                $$
                    \begin{aligned}
                        ( D, [x v^m t^p, y v^n t^q]_{\tilde{\g}_{[2]}} )_{\hat{\g}_{[2]}} & = ( D, [x, y]_{\g} v^{m + n} t^{p + q} + (x, y)_{\g} v^m t^p \bar{d}( v^n t^q ) )_{\hat{\g}_{[2]}}
                        \\
                        & = (x, y)_{\g} ( D, v^m t^p \bar{d}( v^n t^q ) )_{\hat{\g}_{[2]}}
                        \\
                        & = (x, y)_{\g} ( D, \delta_{(m, p) + (n, q), (0, 0)} ( n c_v + q c_t ) + (np - mq) K_{m + n, p + q} )_{\hat{\g}_{[2]}}
                    \end{aligned}
                $$
            Now, without any loss of generality, let us suppose that $D \in \d_{[2]}$ is some basis element, i.e.:
                $$D \in \{ D_{r, s}, D_v, D_t \}$$
            and consider these cases separately, for the sake of clarity:
            \begin{enumerate}
                \item \textbf{(Case 1: $D := D_{r, s}$):} Fix some $(r, s) \in \Z^2$ and consider the following: 
                    $$
                        \begin{aligned}
                            ( D_{r, s}, [x v^m t^p, y v^n t^q]_{\tilde{\g}_{[2]}} )_{\hat{\g}_{[2]}} & = (x, y)_{\g} ( D_{r, s}, \delta_{(m, p) + (n, q), (0, 0)} ( n c_v + q c_t ) + (np - mq) K_{m + n, p + q} )_{\hat{\g}_{[2]}}
                            \\
                            & = (x, y)_{\g} (np - mq) \delta_{(r, s), (m + n, p + q)}
                        \end{aligned}
                    $$
                The assumption that $(-, -)_{\hat{\g}_{[2]}}$ is invariant with respect to $[-, -]_{\hat{\g}_{[2]}}$ then implies that:
                    $$( [D_{r, s}, x v^m t^p]_{\hat{\g}_{[2]}}, y v^n t^q )_{\hat{\g}_{[2]}} = (x, y)_{\g} (np - mq) \delta_{(r, s), (m + n, p + q)}$$
                Now, suppose that:
                    $$[D_{r, s}, x v^m t^p]_{\hat{\g}_{[2]}} := \sum_{(a, b) \in \Z^2} \lambda_{a, b}(x) v^a t^b + K_{(m, p), (r, s)}(x) + \xi_{(m, p), (r, s)}(x)$$
                for some $\lambda_{a, b}(x) \in \g$, $K_{(m, p), (r, s)}(x) \in \z_{[2]}$, and $\xi_{(m, p), (r, s)}(x) \in \d_{[2]}$, depending on our choices of $x \in \g$ and $(m, p) \in \Z^2$. Next, consider the following:
                    $$
                        \begin{aligned}
                            ( [D_v, x v^m t^p]_{\hat{\g}_{[2]}}, y v^n t^q )_{\hat{\g}_{[2]}} & = \left( \sum_{(a, b) \in \Z^2} \lambda_{a, b}(x) v^a t^b + K_{(m, p), (r, s)}(x) + \xi_{(m, p), (r, s)}(x), y v^n t^q \right)_{\hat{\g}_{[2]}}
                            \\
                            & = \sum_{(a, b) \in \Z^2} \left( \lambda_{a, b}(x) v^a t^b, y v^n t^q \right)_{\g_{[2]}}
                            \\
                            & = -\sum_{(a, b) \in \Z^2} (\lambda_{a, b}(x), y)_{\g} \delta_{ (a, b) + (n, q), (0, -1) }
                            \\
                            & = -(\lambda_{-n, -q - 1}(x), y)_{\g}
                        \end{aligned}
                    $$
                which tells us that:
                    $$(x, y)_{\g} (np - mq) \delta_{(r, s), (m + n, p + q)} = -(\lambda_{-n, -q - 1}(x), y)_{\g}$$
                The non-degeneracy of the inner product $(-, -)_{\g}$ as well as the arbitrariness of the choices of $y \in \g$ and $(n, q) \in \Z^2$ then together imply that:
                    $$\lambda_{-n, -q - 1}(x) = -(np - mq) \delta_{(r, s), (m + n, p + q)} = (mq - np) \delta_{(r, s), (m + n, p + q)}$$
                for any fixed choices of $x \in \g$ and $(m, p) \in \Z^2$. From this, we infer that:
                    $$
                        \begin{aligned}
                            [D_{r, s}, x v^m t^p]_{\hat{\g}_{[2]}} & = \sum_{(n, q) \in \Z^2} -(np - mq) \delta_{(r, s), (m + n, p + q)} v^{-n} t^{-q - 1} + K_{(m, p), (r, s)}(x) + \xi_{(m, p), (r, s)}(x)
                            \\
                            & = ( m(s - p) - (r - m)p ) x v^{m - r} t^{p - s - 1} + K_{(m, p), (r, s)}(x) + \xi_{(m, p), (r, s)}(x)
                            \\
                            & = ( ms - rp ) x v^{m - r} t^{p - s - 1} + K_{(m, p), (r, s)}(x) + \xi_{(m, p), (r, s)}(x)
                        \end{aligned}
                    $$
                    
                We now claim that:
                    $$\xi_{(m, p), (r, s)}(x) = 0$$
                To this end, consider firstly the following, wherein $Z \in \z_{[2]}$ is an arbitrary choice:
                    $$
                        \begin{aligned}
                            ( [D_{r, s}, x v^m t^p]_{\hat{\g}_{[2]}}, Z )_{\hat{\g}_{[2]}} & = ( D_{r, s}, [x v^m t^p, Z]_{\tilde{\g}_{[2]}} )_{\hat{\g}_{[2]}}
                            \\
                            & = (D, 0)_{\hat{\g}_{[2]}}
                            \\
                            & = 0
                        \end{aligned}
                    $$
                Simultaneously, consider the following:
                    $$
                        \begin{aligned}
                            ( [D_{r, s}, x v^m t^p]_{\hat{\g}_{[2]}}, Z )_{\hat{\g}_{[2]}} & = \left( \sum_{(a, b) \in \Z^2} \lambda_{a, b}(x) v^a t^b + K_{(m, p), (r, s)}(x) + \xi_{(m, p), (r, s)}(x), Z \right)_{\hat{\g}_{[2]}}
                            \\
                            & = ( \xi_{(m, p), (r, s)}(x), Z )_{\hat{\g}_{[2]}}
                        \end{aligned}
                    $$
                The previous observation along with this one imply that:
                    $$( \xi_{(m, p), (r, s)}(x), Z )_{\hat{\g}_{[2]}} = 0$$
                for \textit{any} $Z \in \z_{[2]}$, but since $\d_{[2]}$ is graded-dual to $\z_{[2]}$ by construction, the above implies via the non-degeneracy of the inner product $(-, -)_{\hat{\g}_{[2]}}$ that:
                    $$\xi_{(m, p), (r, s)}(x) = 0$$
                necessarily. 

                We can now conclude that:
                    $$[D_{r, s}, x v^m t^p]_{\hat{\g}_{[2]}} = ( rp - ms ) x v^{m - r} t^{p - s - 1} + K_{(m, p), (r, s)}(x)$$
                \item \textbf{(Case 2: $D := D_v$):} In this case, it is easy to see that:
                    $$
                        \begin{aligned}
                            ( D_v, [x v^m t^p, y v^n t^q]_{\tilde{\g}_{[2]}} )_{\hat{\g}_{[2]}} & = (x, y)_{\g} ( D_v, \delta_{(m, p) + (n, q), (0, 0)} ( n c_v + q c_t ) + (np - mq) K_{m + n, p + q} )_{\hat{\g}_{[2]}}
                            \\
                            & = (x, y)_{\g} \delta_{(m, p) + (n, q), (0, 0)} n
                        \end{aligned}
                    $$
                Using invariance, we then see that:
                    $$( [D_v, x v^m t^p]_{\hat{\g}_{[2]}}, y v^n t^q )_{\hat{\g}_{[2]}} = (x, y)_{\g} \delta_{(m, p) + (n, q), (0, 0)} n$$
                Now, suppose that:
                    $$[D_v, x v^m t^p]_{\hat{\g}_{[2]}} := \sum_{(a, b) \in \Z^2} \lambda_{a, b}(x) v^a t^b + K_{m, p}(x) + \xi_{m, p}(x)$$
                for some $\lambda_{a, b}(x) \in \g$, $K_{m, p}(x) \in \z_{[2]}$, and $\xi_{m, p}(x) \in \d_{[2]}$, depending on our choices of $x \in \g$ and $(m, p) \in \Z^2$. Then, consider the following:
                    $$
                        \begin{aligned}
                            ( [D_v, x v^m t^p]_{\hat{\g}_{[2]}}, y v^n t^q )_{\hat{\g}_{[2]}} & = \left( \sum_{(a, b) \in \Z^2} \lambda_{a, b}(x) v^a t^b + K_{m, p}(x) + \xi_{m, p}(x), y v^n t^q \right)_{\hat{\g}_{[2]}}
                            \\
                            & = \sum_{(a, b) \in \Z^2} \left( \lambda_{a, b}(x) v^a t^b, y v^n t^q \right)_{\g_{[2]}}
                            \\
                            & = -\sum_{(a, b) \in \Z^2} (\lambda_{a, b}(x), y)_{\g} \delta_{ (a, b) + (n, q), (0, -1) }
                            \\
                            & = -(\lambda_{-n, -q - 1}(x), y)_{\g}
                        \end{aligned}
                    $$
                From this, we are able to conclude that:
                    $$(x, y)_{\g} \delta_{(m, p) + (n, q), (0, 0)} n = -(\lambda_{-n, -q - 1}(x), y)_{\g}$$
                As this holds for all $y \in \g$ and all $(n, q) \in \Z^2$, we can infer from the above and from the non-degeneracy of the inner product $(-, -)_{\g}$ that:
                    $$\lambda_{-n, -q - 1}(x) = \delta_{(m, p) + (n, q), (0, 0)} n x$$
                for any $x \in \g$ and any $(m, p) \in \Z^2$ (both fixed!), and hence:
                    $$
                        \begin{aligned}
                            [D_v, x v^m t^p]_{\hat{\g}_{[2]}} & = \sum_{(n, q) \in \Z^2} \delta_{(m, p) + (n, q), (0, 0)} n x v^{-n} t^{-q - 1} + K_{m, p}(x) + \xi_{m, p}(x)
                            \\
                            & = -m x v^m t^{p - 1} + K_{m, p}(x) + \xi_{m, p}(x)
                        \end{aligned}
                    $$

                Now, by arguing as in \textbf{Case 1}, we will see that:
                    $$\xi_{m, p}(x) = 0$$
                and afterwards we will be able to conclude that:
                    $$[D_v, x v^m t^p]_{\hat{\g}_{[2]}} = -m x v^m t^{p - 1} + K_{m, p}(x)$$
                \item \textbf{(Case 3: $D := D_t$)} Arguing as when $D = D_v$, we will obtain:
                    $$[D_t, x v^m t^p]_{\hat{\g}_{[2]}} = -p x v^m t^{p - 1} + K_{m, p}(x)$$
                for some $K_{m, p}(x) \in \z_{[2]}$.
            \end{enumerate}
        \end{remark}
        \begin{remark}
            We see now also that $\tilde{\g}_{[2]}$ is a Lie algebra ideal of $\hat{\g}_{[2]}$ with respect to $[-, -]_{\hat{\g}_{[2]}}$.
        \end{remark}

        \begin{remark}[$\d_{[2]}$ acts by derivations] \label{remark: dual_of_toroidal_centres_contains_derivations}
            \begin{enumerate}
                \item We can identify the derivations $D_{r, s}, D_v, D_t$ explicitly in terms of $\del_v := \frac{\del}{\del v}, \del_t := \frac{\del}{\del t}$. For this, let us firstly equip $\der_{\bbC}(A_{[2]})$ - the $\bbC$-vector space of all $\bbC$-linear derivations on $A_{[2]}$ - with the following basis:
                    $$\{ v^m t^p \del_v, v^n t^q \del_t \}_{(m, p), (n, q) \in \Z^2}$$
                \begin{enumerate}
                    \item To compute $D_{r, s}$ in terms of $\del_v, \del_t$, suppose firstly that:
                        $$D_{r, s} := f(v, t) \del_v + g(v, t) \del_t$$
                    with $f(v, t), g(v, t) \in A_{[2]}$. Next, fix some $(m, p) \in \Z^2$ and then consider the following:
                        $$
                            \begin{aligned}
                                D_{r, s}( v^m t^p ) & = f(v, t) \del_v( v^m t^p ) + g(v, t) \del_t( v^m t^p )
                                \\
                                & = f(v, t) m v^{m - 1} t^p + g(v, t) p v^m t^{p - 1}
                            \end{aligned}
                        $$
                    At the same time, we also have that:
                        $$D_{r, s}(v^m t^p) := ( ms - rp ) v^{m - r} t^{p - s - 1}$$
                    and hence:
                        $$f(v, t) m v^{m - 1} t^p + g(v, t) p v^m t^{p - 1} = ( ms - rp ) v^{m - r} t^{p - s - 1}$$
                    From this, one infers that:
                        $$f(v, t) = s v^{-r + 1} t^{-s - 1}, g(v, t) = -r v^{-r} t^{-s}$$
                    and therefore:
                        $$D_{r, s} = s v^{-r + 1} t^{-s - 1} \del_v - r v^{-r} t^{-s} \del_t$$
                    \item One easily checks that:
                        $$D_v = -v t^{-1} \del_v$$
                    \item Likewise:
                        $$D_t = -\del_t$$
                \end{enumerate}
                Consequently, we see that elements of $\d_{[2]}$ are derivations on $A_{[2]}$.
    
                \item Now that we know that the basis elements $D_{r, s}, D_v, D_t \in \d_{[2]}$ are actually certain derivations on $A_{[2]}$, we can also check that the commutators of the elements $D_{r, s}, D_v, D_t$ are still elements of $\d_{[2]}$. This ensures us that we can \textit{choose} to endow $\d_{[2]}$ with the structure of a Lie subalgebra of $\der_{\bbC}(A_{[2]})$, i.e. the Lie algebra structure such that:
                    $$[D, D']_{\hat{\g}_{[2]}} \in \d_{[2]}$$
                for any $D, D' \in \d_{[2]}$. In general, however, we are only guaranteed that:
                    $$[D, D']_{\hat{\g}_{[2]}} = \z_{[2]} \oplus \d_{[2]}$$
                We note also that it is not even guaranteed \textit{a priori} that the $\d_{[2]}$-summand of the commutators of the form $[D, D']_{\hat{\g}_{[2]}}$ has to be the usual commutator inherited from $\der_{\bbC}(A_{[2]})$; this turns out to be true, but is a somewhat non-trivial fact (cf. proposition \ref{prop: lie_bracket_on_orthogonal_complement_of_toroidal_centre}). 
            \end{enumerate}
        \end{remark}
        
        \begin{remark}[How does $\d_{[2]}$ act on $\z_{[2]}$ ?] \label{remark: derivation_action_on_toroidal_centres}
            We can now use what we know about how $\d_{[2]}$ acts on $\g_{[2]}$ in conjunction with the Jacobi identity in order to compute Lie brackets of the form:
                $$[D, Z]_{\hat{\g}_{[2]}}$$
            for any $D \in \d_{[2]}$ and any $Z \in \z_{[2]}$. 
        
            \begin{enumerate}
                \item For the computations that follow, \textit{we will need to assume that}:
                    $$\hat{\g}_{[2]} \cong \tilde{\g}_{[2]}\rtimes \d_{[2]}$$
                in which case it can be easily shown that:
                    $$[\d_{[2]}, \g_{[2]}]_{\hat{\g}_{[2]}} \subseteq \g_{[2]}$$
                Different methods will have to be employed in the absence of this assumption.
    
                The idea here is to use the fact that:
                    $$[-, -]_{\tilde{\g}_{[2]}} = [-, -]_{\g_{[2]}} + \e$$
                wherein $\e: \bigwedge^2 \g_{[2]} \to \z_{[2]}$ is given by:
                    $$\e(x v^m t^p, y v^n t^q) := (x, y)_{\g} v^m t^p \bar{d}(v^n t^q)$$
                for any $x, y \in \g$ and any $(m, p), (n, q) \in \Z^2$. With this in mind, consider the following for any $D \in \d_{[2]}$, which holds thanks to the Jacobi identity:
                    $$
                        \begin{aligned}
                            & [ D, [x v^m t^p, y v^n t^q]_{\tilde{\g}_{[2]}} ]_{\hat{\g}_{[2]}}
                            \\
                            = & [ [D, x v^m t^p]_{\hat{\g}_{[2]}}, y v^n t^q ]_{\tilde{\g}_{[2]}} + [ x v^m t^p, [D, y v^n t^q]_{\hat{\g}_{[2]}} ]_{\tilde{\g}_{[2]}}
                            \\
                            = & [x D(v^m t^p), y v^n t^q]_{\hat{\g}_{[2]}} + [x v^m t^p, y D(v^n t^q)]_{\hat{\g}_{[2]}}
                            \\
                            = & [x, y]_{\g} (D(v^m t^p) v^n t^q + v^m t^p D(v^n t^q)) + (x, y)_{\g} ( D(v^m t^p) \bar{d}(v^n t^q) + v^m t^p \bar{d}(D(v^n t^q)) )
                            \\
                            = & [x, y]_{\g} D(v^{m + n} t^{p + q}) + (x, y)_{\g} ( D(v^m t^p) \bar{d}(v^n t^q) + v^m t^p \bar{d}(D(v^n t^q)) )
                        \end{aligned}
                    $$
                Note that for the second equality, we implicitly invoked the fact that the basis elements of $\d_{[2]}$ (and hence all elements thereof) act on $\g_{[2]}$ in the following manner:
                    $$[D, x v^m t^p] = x D(v^m t^p), D \in \{D_{r, s}\}_{(r, s) \in \Z^2} \cup \{D_v, D_t\}$$
                (cf. remark \ref{remark: derivation_action_on_multiloop_algebras}), and it makes sense to write this because we now know (after remark \ref{remark: dual_of_toroidal_centres_contains_derivations}) that elements of $\d_{[2]}$ are certain derivations on $A_{[2]}$. At the same time, we have that:
                    $$
                        \begin{aligned}
                            [ D, [x v^m t^p, y v^n t^q]_{\tilde{\g}_{[2]}} ]_{\hat{\g}_{[2]}} & = [ D, [x, y]_{\g} v^{m + n} t^{p + q} + (x, y)_{\g} v^m t^p \bar{d}(v^n t^q) ]_{\hat{\g}_{[2]}}
                            \\
                            & = [x, y]_{\g} D(v^{m + n} t^{p + q}) + (x, y)_{\g} [D, v^m t^p \bar{d}(v^n t^q)]_{\hat{\g}_{[2]}}
                        \end{aligned}
                    $$
                By putting the two computations together, one yields:
                    $$[D, v^m t^p \bar{d}(v^n t^q)]_{\hat{\g}_{[2]}} = D(v^m t^p) \bar{d}(v^n t^q) + v^m t^p \bar{d}(D(v^n t^q))$$
                Since we know how the basis elements of $\d_{[2]}$ act on $A_{[2]}$ (cf. remark \ref{remark: dual_of_toroidal_centres_contains_derivations}), the above is enough to determine how $\d_{[2]}$ acts on $\z_{[2]}$. 

                Let us also note the similarity between the formulae:
                    $$[D, f \bar{d}(g)]_{\hat{\g}_{[2]}} = D(f) \bar{d}(g) + f \bar{d}(D(g))$$
                and those for Lie derivatives.
                
                \item Now, let us \textit{not} assume that:
                    $$\hat{\g}_{[2]} \cong \tilde{\g}_{[2]}\rtimes \d_{[2]}$$
                i.e. that $\tilde{\g}_{[2]}$ might not be a $\d_{[2]}$-module from the start. Without any loss of generality, let us consider the following for any $h, h' \in \h$ so that\footnote{We can make this assumption because ultimately, elements of $\z_{[2]}$ do not depend on those of $\g$.}:
                    $$(h, h')_{\g} = 1$$
                any $f(v, t), g(v, t) \in A_{[2]}$, and any $D \in \d_{[2]}$:
                    $$[ D, [h f(v, t), h' g(v, t)]_{\tilde{\g}_{[2]}} ]_{\hat{\g}_{[2]}} = [ D, f(v, t) \bar{d}( g(v, t) ) ]_{\hat{\g}_{[2]}}$$
                At the same time, we have via the Jacobi identity that:
                    $$
                        \begin{aligned}
                            [ D, [h f(v, t), h' g(v, t)]_{\tilde{\g}_{[2]}} ]_{\hat{\g}_{[2]}} & = [ h f(v, t), [D, h' g(v, t)]_{\hat{\g}_{[2]}} ]_{\tilde{\g}_{[2]}} + [ [D, h f(v, t)]_{\hat{\g}_{[2]}}, h' g(v, t) ]_{\tilde{\g}_{[2]}}
                            \\
                            & = [ h f(v, t), h' D( g(v, t) ) ]_{\tilde{\g}_{[2]}} + [ h D( f(v, t) ), h' g(v, t) ]_{\tilde{\g}_{[2]}}
                            \\
                            & = f(v, t) \bar{d}( D( g(v, t) ) ) + D( f(v, t) ) \bar{d}(g(v, t))
                        \end{aligned}
                    $$
                One thus sees that:
                    $$[ D, f(v, t) \bar{d}( g(v, t) ) ]_{\hat{\g}_{[2]}} = f(v, t) \bar{d}( D( g(v, t) ) ) + D( f(v, t) ) \bar{d}(g(v, t))$$
                and since the element $f(v, t) \bar{d}( g(v, t) )$ is central (via the map $\e$ mentioned earlier), this gives another description of:
                    $$[ \d_{[2]}, \z_{[2]} ]_{\hat{\g}_{[2]}}$$
                With this in mind, we return quickly to remark \ref{remark: derivation_action_on_multiloop_algebras}; there, we previously demonstrated that:
                    $$[ \d_{[2]}, \g_{[2]} ]_{\hat{\g}_{[2]}} \subseteq \g_{[2]} \oplus \z_{[2]}$$
                but we claim now that the following stronger fact holds:
                    $$[ \d_{[2]}, \g_{[2]} ]_{\hat{\g}_{[2]}} \subseteq \g_{[2]}$$
                To see why this is the case, suppose firstly that for any $D \in \d_{[2]}$, any $X := x f(v, t) \in \g_{[2]}$ (for some $f(v, t) \in A_{[2]}$), there is $K(X) \in \z_{[2]}$ depending on $X$ (and indeed, such a $K(X)$ exists by remark \ref{remark: derivation_action_on_multiloop_algebras}) such that:
                    $$[ D, X ]_{\hat{\g}_{[2]}} = x D( f(v, t) ) + K(X)$$
                Next, pick an arbitrary element $\xi \in \d_{[2]}$ and then consider the following:
                    $$( [ D, X ]_{\hat{\g}_{[2]}}, \xi )_{\hat{\g}_{[2]}} = (D(X) + K(X), \xi)_{\hat{\g}_{[2]}} = (K(X), \xi)_{\hat{\g}_{[2]}}$$
                wherein the last equality holds as a consequence of the fact that:
                    $$( \g_{[2]}, \d_{[2]} )_{\hat{\g}_{[2]}} = 0$$
                per the construction of the bilnear form $(-, -)_{\hat{\g}_{[2]}}$ as in convention \ref{conv: orthogonal_complement_of_toroidal_centres}. At the same time, using invariance yields us:
                    $$( [ D, X ]_{\hat{\g}_{[2]}}, \xi )_{\hat{\g}_{[2]}} = (X, [\xi, D]_{\hat{\g}_{[2]}})_{\hat{\g}_{[2]}} = 0$$
                wherein the last equality is due to the fact that:
                    $$[\xi, D]_{\hat{\g}_{[2]}} \in \z_{[2]} \oplus \d_{[2]}$$
                (cf. remark \ref{remark: dual_of_toroidal_centres_contains_derivations}) and the fact that:
                    $$( \g_{[2]}, \z_{[2]} \oplus \d_{[2]} )_{\hat{\g}_{[2]}} = 0$$
                per the construction of the bilnear form $(-, -)_{\hat{\g}_{[2]}}$ as in convention \ref{conv: orthogonal_complement_of_toroidal_centres}. We thus have that:
                    $$(K(X), \xi)_{\hat{\g}_{[2]}} = (X, [\xi, D]_{\hat{\g}_{[2]}})_{\hat{\g}_{[2]}} = 0$$
                for every $\xi \in \d_{[2]}$. The non-degeneracy of $(-, -)_{\hat{\g}_{[2]}}$ then implies through this fact that:
                    $$K(X) = 0$$
                As such, we have that:
                    $$[ \d_{[2]}, \g_{[2]} ]_{\hat{\g}_{[2]}} \subseteq \g_{[2]}$$
                as claimed. 
            \end{enumerate}

        \end{remark}

        Remarks \ref{remark: derivation_action_on_multiloop_algebras}, \ref{remark: dual_of_toroidal_centres_contains_derivations}, and \ref{remark: derivation_action_on_toroidal_centres} can now be packaged together into the following result.
        \begin{proposition}[$\tilde{\g}_{[2]}$ as a $\der_{\bbC}(A_{[2]})$-module] \label{prop: toroidal_lie_algebras_as_modules_over_vector_field_lie_algebras}
            $\tilde{\g}_{[2]}$ is a $\der_{\bbC}(A_{[2]})$-module, decomposing into a direct sum of the submodules $\g_{[2]}$ and $\z_{[2]}$. 
        \end{proposition}

        Now, let us see if and how $\d_{[2]}$ might be endowed with a Lie algebra structure of its own, so that we might have that:
            $$\hat{\g}_{[2]} \cong \tilde{\g}_{[2]} \rtimes \d_{[2]}$$
        The upshot is that in general, the vector space $\d_{[2]}$ fails to be a Lie algebra, namely due to the commutators $[D, D']_{\hat{\g}_{[2]}}$ (for any two $D, D' \in \d_{[2]}$) having a non-vanishing $\z_{[2]}$-summand in general, and their $\d_{[2]}$-summand is actually nothing but the usual commutator of derivations inherited from $\der_{\bbC}(A_{[2]})$, and $\d_{[2]}$ will only be a Lie algebra if we choose to endow it with said commutator bracket (so that the $\z_{[2]}$ would be made to vanish). 
        
        \begin{proposition}[How does $\d_{[2]}$ act on itself] \label{prop: lie_bracket_on_orthogonal_complement_of_toroidal_centre}
            Let $\d_{[2]}$ be given as in convention \ref{conv: orthogonal_complement_of_toroidal_centres}. Then:
                $$[ \d_{[2]}, \d_{[2]} ]_{\hat{\g}_{[2]}} \subset \z_{[2]} \oplus \d_{[2]}$$
            i.e. the $\g_{[2]}$-summand of any commutator of the kind $[D, D']_{\hat{\g}_{[2]}}$ (for any two $D, D' \in \d_{[2]}$) actually vanishes. Furthermore, neither the $\z_{[2]}$- nor the $\d_{[2]}$-summand of those commutators $[D, D']_{\hat{\g}_{[2]}}$ necessarily vanish in general. 
        \end{proposition}
            \begin{proof}
                Pick arbitrary elements $D, D' \in \d_{[2]}$ and set:
                    $$[D, D']_{\hat{\g}_{[2]}} := X(D, D') + K(D, D') + \xi(D, D')$$
                for some $X(D, D') \in \g_{[2]}, K(D, D') \in \z_{[2]}$, and $\xi(D, D') \in \d_{[2]}$ depending on $D, D'$. Pick also a test element $y g(v, t) \in \g_{[2]}$, for some arbitrary $y \in \g$ and $g(v, t) \in A_{[2]}$ and set:
                    $$[D, y g(v, t)]_{\hat{\g}_{[2]}} := y D( g(v, t) ) + K_{D, Y}$$
                    $$[D', y g(v, t)]_{\hat{\g}_{[2]}} := y D'( g(v, t) ) + K_{D', Y}$$
                for some $K_{D, Y} \in \z_{[2]}$ depending on $Y$ (cf. remark \ref{remark: derivation_action_on_multiloop_algebras}).
                
                Via the Jacobi identity, we get that:
                    $$
                        \begin{aligned}
                            & [ [D, D']_{\hat{\g}_{[2]}}, y g(v, t) ]_{\hat{\g}_{[2]}}
                            \\
                            = & [ D, [ D', y g(v, t) ]_{\hat{\g}_{[2]}} ]_{\hat{\g}_{[2]}} + [ D', [ y g(v, t), D ]_{\hat{\g}_{[2]}} ]_{\hat{\g}_{[2]}}
                            \\
                            = & [ D, y D'( g(v, t) ) + K_{D', Y} ]_{\hat{\g}_{[2]}} - [ D', y D( g(v, t) ) + K_{D, Y} ]_{\hat{\g}_{[2]}}
                            \\
                            = & \left( y D( D'(g(v, t)) ) + K_{DD', Y} + [ D, K_{D', Y} ]_{\hat{\g}_{[2]}} \right) - \left( y D'( D(g(v, t)) ) + K_{D'D, Y} + [ D', K_{D, Y} ]_{\hat{\g}_{[2]}} \right)
                            \\
                            = & y (DD' - D'D)( g(v, t) ) + ( K_{DD', Y} - K_{D'D, Y} ) + ( [ D, K_{D', Y} ]_{\hat{\g}_{[2]}} - [ D', K_{D, Y} ]_{\hat{\g}_{[2]}} )
                        \end{aligned}
                    $$
                for some $K_{DD', Y}, K_{D'D, Y} \in \z_{[2]}$ such that:
                    $$[ D, y D'( g(v, t) ) ]_{\hat{\g}_{[2]}} := y D( D'( g(v, t) ) ) + K_{DD', Y}$$
                    $$[ D', y D( g(v, t) ) ]_{\hat{\g}_{[2]}} := y D( D'( g(v, t) ) ) + K_{D'D, Y}$$
                At the same time, we have that:
                    $$
                        \begin{aligned}
                            & [ [D, D']_{\hat{\g}_{[2]}}, y g(v, t) ]_{\hat{\g}_{[2]}}
                            \\
                            = & [ X(D, D') + K(D, D') + \xi(D, D') , y g(v, t) ]_{\hat{\g}_{[2]}}
                            \\
                            = & [ X(D, D') + \xi(D, D') , y g(v, t) ]_{\hat{\g}_{[2]}}
                            \\
                            = & [ X(D, D') , y g(v, t) ]_{\hat{\g}_{[2]}} + \left( y \xi(D, D')(g(v, t)) + K_{\xi(D, D'), Y} \right)
                        \end{aligned}
                    $$
                wherein the second equality holds thanks to the fact that $[\z_{[2]}, \g_{[2]}]_{\hat{\g}_{[2]}} = 0$, and $K_{\xi(D, D'), Y} \in \z_{[2]}$ is some element (cf. remark \ref{remark: derivation_action_on_multiloop_algebras}). Combining the two observations together then yields:
                    $$
                        \begin{aligned}
                            & [ X(D, D') , y g(v, t) ]_{\hat{\g}_{[2]}} + \left( y \xi(D, D')(g(v, t)) + K_{\xi(D, D'), Y} \right)
                            \\
                            = & y (DD' - D'D)( g(v, t) ) + ( K_{DD', Y} - K_{D'D, Y} ) + ( [ D, K_{D', Y} ]_{\hat{\g}_{[2]}} - [ D', K_{D, Y} ]_{\hat{\g}_{[2]}} )
                        \end{aligned}
                    $$
                From remark \ref{remark: centres_of_dual_toroidal_lie_algebras}, we know that there exists $K_{X(D, D'), Y} \in \z_{[2]}$ such that:
                    $$[ X(D, D') , y g(v, t) ]_{\hat{\g}_{[2]}} = [ X(D, D') , Y ]_{\hat{\g}_{[2]}} = [X(D, D'), Y]_{\g_{[2]}} + K_{X(D, D'), Y}$$
                using which we can write:
                    $$
                        \begin{aligned}
                            & [X(D, D'), Y]_{\g_{[2]}} - y \left( ( DD' - D'D) - \xi(D, D') \right)( g(v, t) )
                            \\
                            = & \left( [ D, K_{D', Y} ]_{\hat{\g}_{[2]}} - [ D', K_{D, Y} ]_{\hat{\g}_{[2]}} \right) - \left( K_{X(D, D'), Y} + K_{\xi(D, D'), Y} \right)
                        \end{aligned}
                    $$
                    
                We note right away that the LHS lies entirely in $\g_{[2]}$, whereas the RHS is an element of $\z_{[2]}$ due to the fact that $[\d_{[2]}, \z_{[2]}]_{\hat{\g}_{[2]}} \subseteq \z_{[2]}$ (cf. remark \ref{remark: derivation_action_on_toroidal_centres}), which tells us that $[ D, K_{D', Y} ]_{\hat{\g}_{[2]}}, [ D', K_{D, Y} ]_{\hat{\g}_{[2]}} \in \z_{[2]}$ in particular. Because $\g_{[2]}$ is centreless (as $\g$ is simple and the Lie bracket on $\g_{[2]}$ is given by extension of scalars), this observation subsequently implies that the LHS must vanish, i.e.:
                    $$[X(D, D'), Y]_{\g_{[2]}} - y \left( ( DD' - D'D) - \xi(D, D') \right)( g(v, t) ) = 0$$
                Because we have by construction that:
                    $$DD' - D'D - \xi(D, D') \in \d_{[2]}$$
                we now make the following claim: \textit{if we fix some arbitrary $E \in \g_{[2]}$ and some $P \in \d_{[2]}$ then:}
                    $$\forall H := h \varphi \in \g_{[2]}: [E, H]_{\g_{[2]}} = h P( \varphi ) \implies E = 0$$

                Using the root space decomposition for $\g$, we see that if $h \in \h$ then we then will have that $[E, H]_{\g_{[2]}} \in \n^{\pm}_{[2]}$, but at the same time, that $h P(\varphi) \in \h_{[2]}$. The only way for this to be true is that $[E, H]_{\g_{[2]}} = 0$, which is the case if and only if $E = 0$. If $h \in \n^{\pm}$, then $[E, H]_{\g_{[2]}} \in \n^{\pm}_{[2]} \oplus \h_{[2]}$ and the $\h_{[2]}$-summand will be non-zero in general; at the same time, $h P(\varphi) \in \n^{\pm}_{[2]}$ in this case, and again, the only way for these to facts to be true simultaneously is that $E = 0$ necessarily. 

                Apply the claim to the fact that:
                    $$[X(D, D'), Y]_{\g_{[2]}} = y \left( ( DD' - D'D) - \xi(D, D') \right)( g(v, t) )$$
                - and again, note that $( DD' - D'D) - \xi(D, D') \in \d_{[2]}$ - then yields:
                    $$X(D, D') = 0$$
                precisely as desired. 
            \end{proof}
        \begin{corollary}
            For any $D, D' \in \d_{[2]}$, the $\d_{[2]}$-summand of $[D, D']_{\hat{\g}_{[2]}}$ is nothing but the commutator $DD' - D'D$.
        \end{corollary} 
        In light of the above, it is also valuable to know how the $\d_{[2]}$-summand of the commutators:
            $$[D, D']_{\hat{\g}_{[2]}}$$
        are given explicitly. Later on, these computations will be used to prove that the centre of $\hat{\g}_{[2]}$ is $2$-dimensional, namely given by $\bbC c_v \oplus \bbC c_v$.
        \begin{lemma}[Explicitly commutators between basis elements of $\d_{[2]}$] \label{lemma: explicit_commutators_between_basis_elements_of_toroidal_central_orthogonal_complement}
            The usual commutator (i.e. $[D, D'] := DD' - D'D$) between the basis elements $D_{r, s}, D_v, D_t \in \d_{[2]}$ are given as follows:
                $$[D_v, D_t] = 0$$
                $$[D_v, D_{r, s}] = r D_{r, s + 1}$$
                $$[D_t, D_{r, s}] = D_{r, s + 1}$$
                $$[D_{a, b}, D_{r, s}] = (br - sa) D_{a + r, b + s + 1}$$
        \end{lemma}
            \begin{proof}
                \begin{enumerate}
                    \item Since we know that:
                        $$D_v = -vt^{-1} \del_v, D_t = -\del_t$$
                    (cf. remark \ref{remark: dual_of_toroidal_centres_contains_derivations}), it is therefore trivial that:
                        $$[D_v, D_t] = 0$$
                    \item From remark \ref{remark: derivation_action_on_multiloop_algebras}, we know that:
                        $$D_v(v^m t^p) = -m v^m t^{p - 1}$$
                        $$D_{r, s}(v^m t^p) = ( ms - rp ) v^{m - r} t^{p - s - 1}$$
                    From this, we infer that:
                        $$
                            \begin{aligned}
                                [D_v, D_{r, s}](v^m t^p) & = D_v( D_{r, s}(v^m t^p) ) - D_{r, s}( D_v(v^m t^p) )
                                \\
                                & = (ms - rp) D_v( v^{m - r} t^{p - s - 1} ) + m D_{r, s}( v^m t^{p - 1} )
                                \\
                                & = -(m - r)(ms - rp) v^{m - r} t^{p - s - 2} + (ms - r(p - 1)) m v^{m - r} t^{p - s - 2}
                                \\
                                & = r(m(s + 1) - rp) v^{m - r} t^{p - (s + 1) - 1}
                                \\
                                & = r D_{r, s + 1}(v^m t^p)
                            \end{aligned}
                        $$
                    and hence:
                        $$[D_v, D_{r, s}] = r D_{r, s + 1}$$
                    \item Likewise, we can show that:
                        $$[D_t, D_{r, s}] = D_{r, s + 1}$$
                    \item Lastly, we can also show, using a completely analogous argument, that:
                        $$[D_{a, b}, D_{r, s}] = (br - sa) D_{a + r, b + s + 1}$$
                \end{enumerate}
            \end{proof}
        By exploiting invariance again, we are able to obtain the following corollary to the lemma above, concerning commutators between elements of $\z_{[2]}$ and those of $\d_{[2]}$ (which are already known to be elements of $\z_{[2]}$; cf. remark \ref{remark: derivation_action_on_toroidal_centres}). Due to its usefulness, we nevertheless grant it lemma-hood.
        \begin{lemma}[Explicit commutators between basis elements of $\d_{[2]}$ and $\z_{[2]}$] \label{lemma: explicit_commutators_between_central_basis_elements_and_derivations}
            In the Lie algebra $\hat{\g}_{[2]}$, one has the following non-trivial relations between elements of $\z_{[2]}$ and those of $\d_{[2]}$:
                $$
                    \forall (a, b) \in \Z^2: [D, K_{a, b}]_{\hat{\g}_{[2]}} =
                    \begin{cases}
                        \text{$((b - 1)r - sa) D_{a - r, b - s - 1}$ if $D = D_{r, s}$}
                        \\
                        \text{$-r K_{a, b - 1}$ if $D = D_v$}
                        \\
                        \text{$- D_{a, b - 1}$ if $D = D_t$}
                    \end{cases}
                $$
                $$[\d_{[2]}, c_v]_{\hat{\g}_{[2]}} = [\d_{[2]}, c_t]_{\hat{\g}_{[2]}} = 0 = 0$$
        \end{lemma}
            \begin{proof}
                Firstly, let us note that we need not assume that:
                    $$\hat{\g}_{[2]} \cong \tilde{\g}_{[2]}$$
            
                For a moment, consider two arbitrary derivations $D', D \in \d_{[2]}$ along with an arbitrary element $K \in \z_{[2]}$. Suppose also that:
                    $$[D', D]_{\hat{\g}_{[2]}} = [D', D] + K(D', D)$$
                for some $K(D', D) \in \z_{[2]}$ (cf. proposition \ref{prop: lie_bracket_on_orthogonal_complement_of_toroidal_centre}). Using the invariance property of the bilinear form $(-, -)_{\hat{\g}_{[2]}}$ as well as the fact that:
                    $$(\z_{[2]}, \z_{[2]})_{\hat{\g}_{[2]}} = 0$$
                per the construction of $(-, -)_{\hat{\g}_{[2]}}$ (cf. convention \ref{conv: orthogonal_complement_of_toroidal_centres}), one gets:
                    $$([D', D], K)_{\hat{\g}_{[2]}} = ([D', D] + K(D', D), K)_{\hat{\g}_{[2]}} = ([D', D]_{\hat{\g}_{[2]}}, K)_{\hat{\g}_{[2]}} = (D', [D, K]_{\hat{\g}_{[2]}})_{\hat{\g}_{[2]}}$$
                Since we know how the ordinary commutators:
                    $$[D', D]$$
                are given and particularly, how if $D', D \in \d_{[2]}$ are basis elements then their commutator $[D', D]$ will be in the span of a single basis element of $\d_{[2]}$ (cf. lemma \ref{lemma: explicit_commutators_between_basis_elements_of_toroidal_central_orthogonal_complement}), as well as how $\d_{[2]}$ is the orthogonal complement of $\z_{[2]}$ with respect to $(-, -)_{\hat{\g}_{[2]}}$ by construction, it remains now to simply specialise to the case wherein $K$ is some basis element of $\z_{[2]}$. 
                \begin{enumerate}
                    \item If:
                        $$K = K_{a, b}$$
                    for some fixed $(a, b) \in \Z^2$ then:
                        $$([D', D], K_{a, b})_{\hat{\g}_{[2]}} = 1 \iff [D', D] = D_{a, b}$$
                    Hence, we have that:
                        $$1 = (D_{a, b}, K_{a, b})_{\hat{\g}_{[2]}} = ([D', D], K_{a, b})_{\hat{\g}_{[2]}} = (D', [D, K_{a, b}]_{\hat{\g}_{[2]}})_{\hat{\g}_{[2]}}$$
                    Using the commutators computed in lemma \ref{lemma: explicit_commutators_between_basis_elements_of_toroidal_central_orthogonal_complement}, we then see that:
                        $$
                            [D, K_{a, b}]_{\hat{\g}_{[2]}} =
                            \begin{cases}
                                \text{$((b - 1)r - sa) D_{a - r, b - s - 1}$ if $D = D_{r, s}$}
                                \\
                                \text{$-r K_{a, b - 1}$ if $D = D_v$}
                                \\
                                \text{$- D_{a, b - 1}$ if $D = D_t$}
                            \end{cases}
                        $$
                    \item If:
                        $$K = c_v$$
                    then we have that:
                        $$([D', D], c_v)_{\hat{\g}_{[2]}} = 1 \iff [D', D] = D_v$$
                    which in turn implies that:
                        $$1 = (D_v, K_v)_{\hat{\g}_{[2]}} = ([D', D], c_v)_{\hat{\g}_{[2]}} = (D', [D, c_v]_{\hat{\g}_{[2]}})_{\hat{\g}_{[2]}}$$
                    Once again, by using the commutators computed in lemma \ref{lemma: explicit_commutators_between_basis_elements_of_toroidal_central_orthogonal_complement}, we then see that:
                        $$\forall D \in \{D_{r, s}\}_{(r, s) \in \Z^2} \cup \{D_v, D_t\}: [D, c_v]_{\hat{\g}_{[2]}} = 0$$
                    and since the set $\{D_{r, s}\}_{(r, s) \in \Z^2} \cup \{D_v, D_t\}$ is a basis for $\d_{[2]}$, one thus has that:
                        $$[\d_{[2]}, c_v]_{\hat{\g}_{[2]}} = 0$$
                    as none of said commutators are elements of $\bbC D_v$.
                    \item Likewise, one can show that:
                        $$[\d_{[2]}, c_t]_{\hat{\g}_{[2]}} = 0$$
                \end{enumerate}
            \end{proof}
        We will also see that in fact, the centre of $\hat{\g}_{[2]}$ is actually just spanend by $c_v$ and $c_t$. See proposition \ref{prop: centres_of_extended_toroidal_lie_algebras} and the discussion preceding it for more details. 
        
        \begin{question}[A different Lie bracket on $\der_{\bbC}(A_{[2]})$ ?] \label{question: alternative_derivation_lie_bracket}
            Is there a Lie algebra structure $\der_{\bbC}(A_{[2]})$ (different from the standard one given by commutators $[D, D'] := DD' - D'D$) with respect to which $\d_{[2]}$ is always a Lie subalgebra ? 
        \end{question}
        
        \begin{proposition}[$\tilde{\g}_{[2]}$ as a $\d_{[2]}$-module] \label{prop: toroidal_lie_algebras_as_modules_over_div_0_vector_field_lie_algebras}
            When we assume that:
                $$\hat{\g}_{[2]} \cong \tilde{\g}_{[2]} \rtimes \d_{[2]}$$
            i.e. when $\d_{[2]}$ is a Lie subalgebra of $\der_{\bbC}(A_{[2]})$ with the usual commutator bracket, we will have that $\tilde{\g}_{[2]}$ is a $\d_{[2]}$-module, decomposing into a direct sum of the submodules $\g_{[2]}$ and $\z_{[2]}$. 
        \end{proposition}
        
        In summary, we have yielded the following result:
        \begin{theorem}[Yangian extended toroidal Lie algebras] \label{theorem: extended_toroidal_lie_algebras}
            There is a Lie bracket on the vector space:
                $$\hat{\g}_{[2]} := \tilde{\g}_{[2]} \oplus \d_{[2]}$$
            given as in remarks \ref{remark: derivation_action_on_multiloop_algebras} and \ref{remark: derivation_action_on_toroidal_centres}, uniquely determined by the choice of an invariant non-degenerate symmetric $\bbC$-bilinear form $(-, -)_{\hat{\g}_{[2]}}$ as in convention \ref{conv: orthogonal_complement_of_toroidal_centres}, up to choices of central summands for the elements of $[\d_{[2]}, \d_{[2]}]_{\hat{\g}_{[2]}}$ (cf. remark \ref{remark: dual_of_toroidal_centres_contains_derivations}).

            When said central summands are chosen to vanish, i.e. when $\d_{[2]}$ is a Lie subalgebra of $\der_{\bbC}(A_{[2]})$ with the usual commutator bracket, one shall obtain:
                $$\hat{\g}_{[2]} := \tilde{\g}_{[2]} \rtimes \d_{[2]}$$
        \end{theorem}
        \begin{definition}[Yangian extended toroidal Lie algebras] \label{def: extended_toroidal_lie_algebras}
            We refer to $\hat{\g}_{[2]}$ as in theorem \ref{theorem: extended_toroidal_lie_algebras} as a \textbf{Yangian extended ($2$-)toroidal Lie algebra}. 
        \end{definition}
        \begin{remark}
            Our notion of Yangian extended $2$-toroidal Lie algebras does not quite coincide with the very similar notion of an \say{extended affine Lie algebra of nullity $2$}. Ultimately, this is because the bilinear form that we have endowed $\g_{[2]}$ with is of degree $-1$ (as opposed to $0$) in the second variable. For the latter, $\d_{[2]}$ would be the Lie algebra of divergence-free vector fields on the (smooth) affine $\bbC$-scheme $\Spec A_{[2]}$.
        \end{remark}
        \begin{convention}
            Since we are not concerned here with nullity-$2$ extended affine Lie algebras, but rather only the Yangian analogue thereof, we will drop the descriptor \say{Yangian}. 
        \end{convention}

        Even though we are now perfectly ready to endow $\hat{\g}_{[2]}^+$ with a Lie bialgebra structure (which eventually will be shown to descend down to a Lie bialgebra structure on $\tilde{\g}_{[2]}^+$; cf. theorem \ref{theorem: toroidal_lie_bialgebras}) arising from some Manin triple of the kind:
            $$(\hat{\g}_{[2]}, \hat{\g}_{[2]}^+, \hat{\g}_{[2]}^-)$$
        wherein $\hat{\g}_{[2]}$ is equipped with the non-degenerate invariant bilinear form $(-, -)_{\hat{\g}_{[2]}}$, let us defer this construction to theorem \ref{theorem: extended_toroidal_manin_triples} for organisational purposes.

        One question that is not too relevant to our eventual study of affine Yangians, but is nevertheless natural and interesting in its own right, is the following:
        \begin{question}[Uniqueness of extended toroidal Lie algebras] \label{question: uniqueness_of_extended_toroidal_lie_algebras}
            How freely does the $\z_{[2]}$-summand of the commutators of the kind:
                $$[D, D']_{ \hat{\g}_{[2]} } \in \z_{[2]} \oplus \d_{[2]}$$
            vary ? Alternatively (and perhaps not equivalently), what is the dimension of the $\bbC$-vector space:
                $$H^2_{\Lie}( \der_{\bbC}(A_{[2]}), \bar{\Omega}_{[2]} )$$
            wherein the $\der_{\bbC}(A_{[2]})$-action on $\bar{\Omega}_{[2]}$ is determined by how $\d_{[2]}$ acts on $\z_{[2]}$ (cf. remark \ref{remark: centres_of_dual_toroidal_lie_algebras}).
    
            This question ought to also be equivalent to question \ref{question: alternative_derivation_lie_bracket}. 
        \end{question}
        \begin{question}
            More generally, one can pose the following problem about the cyclic cohomology of smooth varieties. Suppose that $X$ is a smooth variety over an algebraically closed field $k$ of characteristic $0$ and write:
                $$\bar{\Omega}_{X/k}^n := \Omega_{X/k}^n/d( \Omega_{X/k}^{n - 1} )$$
            as well as:
                $$\scrD_{X/k}^1$$
            for the tangent sheaf of $X$ (which is naturally a sheaf of Lie algebras via local commutators of vector fields). $\scrD_{X/k}^1$ acts naturally on $\bar{\Omega}_{X/k}^n$ via Lie derivatives, so it is natural to inquire about the dimension of the $k$-vector spaces:
                $$H^{\bullet}_{\Lie}( \scrD_{X/k}^1, \bar{\Omega}_{X/k}^n )$$
            and furthermore, if there is any dependence thereof on $n \geq 1$.  
        \end{question}
        \begin{remark}[Some explicit elements of $H^2_{\Lie}(\d_{[2]}, \z_{[2]})$] \label{remark: non_uniqueness_of_yangian_extended_lie_algebras}
            From proposition \ref{prop: lie_bracket_on_orthogonal_complement_of_toroidal_centre}, we know that:
                $$[\d_{[2]}, \d_{[2]}]_{\hat{\g}_{[2]}} \subset \z_{[2]} \oplus \d_{[2]}$$
            we can obtain some $2$-cocyles $\bar{\sigma} \in H^2_{\Lie}(\d_{[2]}, \z_{[2]})$ by restricting elements $\sigma \in H^2_{\Lie}(\der_{\bbC}(A_{[2]}), \z_{[2]})$, some of which are known.

            It pays to abstract the situation out to the $n$-variable case for a moment, mostly for us to make the point that the dimension of the vector space $H^2_{\Lie}(\der_{\bbC}(A_{[n]}), \bar{\Omega}_{[n]})$ depends not on the number of variables. In \cite[pp. 5, below Equation 1.3]{billig_energy_momentum_tensor}, it was noted that\todo{Find a proper citation for this. \cite{billig_energy_momentum_tensor} does not provide one.}:
                $$H^2_{\Lie}(\der_{\bbC}(A_{[n]}), \bar{\Omega}_{[n]}) \cong \bbC \sigma_1 \oplus \bbC \sigma_2$$
            with the $2$-cocyles $\sigma_1, \sigma_2$ twisting the Lie brackets:
                $$[v_1^{m_1} ... v_n^{m_n} v_p \del_{v_p}, v_1^{r_1} ... v_n^{r_n} v_q \del_{v_q}] \in \der_{\bbC}(A_{[n]})$$
            being given by:
                $$\sigma_1(v_1^{m_1} ... v_n^{m_n} v_p \del_{v_p}, v_1^{r_1} ... v_n^{r_n} v_q \del_{v_q}) = r_p m_q \sum_{1 \leq i \leq n} r_i v_1^{m_1 + r_1} ... v_n^{m_n + r_n} v_i^{-1} \bar{d}(v_i)$$
                $$\sigma_2(v_1^{m_1} ... v_n^{m_n} v_p \del_{v_p}, v_1^{r_1} ... v_n^{r_n} v_q \del_{v_q}) = m_p r_q \sum_{1 \leq i \leq n} r_i v_1^{m_1 + r_1} ... v_n^{m_n + r_n} v_i^{-1} \bar{d}(v_i)$$
            for every $1 \leq p, q \leq n$ and every $(m_1, ..., m_n), (r_1, ..., r_n) \in \Z^n$. 

            Now, back to the $2$-variable case. Here, we know how the basis elements $D_{r, s}, D_v, D_t$ of $\d_{[2]}$ are given in terms of $A_{[2]}$-multiples of the partial derivatives $\del_v, \del_t$ (cf. remark \ref{remark: dual_of_toroidal_centres_contains_derivations}), so we can exploit the bilinearity of $2$-cocycles in order to see how $\sigma_1, \sigma_2$ act on elements of $\d_{[2]}$. Knowing that the aforementioned basis elements are given by:
                $$D_{r, s} = s v^{-r + 1} t^{-s - 1} \del_v - r v^{-r} t^{-s} \del_t$$
                $$D_v = -v t^{-1} \del_v$$
                $$D_t = -\del_t$$
            we see thus that:
                $$\sigma_a(D_t, -) = 0$$
                $$\sigma_a(D_{r, s}, -), \sigma_a(D_v, -) \not = 0$$
            as elements of $\Hom_{\bbC}(\d_{[2]}, \z_{[2]})$, where $a \in \{1, 2\}$; one proves this by looking at the powers of $v, t$ in the multiples of $\del_v, \del_t$ in the expressions for $D_{r, s}, D_v, D_t$. Subsequently, we can conclude that neither of the $2$-cocycles $\sigma_1, \sigma_2$ vanish entirely on the Lie subalgebra $\d_{[2]}$ of $\der_{\bbC}(A_{[2]})$, and hence:
                $$H^2_{\Lie}(\d_{[2]}, \z_{[2]}) \cong \bbC \sigma_1 \oplus \bbC \sigma_2$$
            We delegate the relevant detailed computations to the proof of theorem \ref{theorem: non_uniqueness_of_yangian_extended_lie_algebras} down below.  
        \end{remark}
        \begin{theorem}[Non-uniqueness of Yangian extended Lie algebras] \label{theorem: non_uniqueness_of_yangian_extended_lie_algebras}
            Let the vector space:
                $$\d_{[2]} := (\bigoplus_{(r, s) \in \Z^2} \bbC D_{r, s}) \oplus \bbC D_v \oplus \bbC D_t$$
            as in convention \ref{conv: orthogonal_complement_of_toroidal_centres} (see remark \ref{remark: dual_of_toroidal_centres_contains_derivations} also, for how the elements $D_{r, s}, D_v, D_t$ are given in terms of $\del_v, \del_t$) be viewed as a Lie subalgebra of $\der_{\bbC}(A_{[2]})$ (possible thanks to proposition \ref{prop: lie_bracket_on_orthogonal_complement_of_toroidal_centre}), which we endow with the usual commutator bracket. Then, there are two isomorphism classes of Lie algebra extensions:
                $$0 \to \tilde{\g}_{[2]} \to \hat{\g}_{[2]} \to \d_{[2]} \to 0$$
            or, in cohomological terms, one has that:
                $$\dim_{\bbC} H^2_{\Lie}(\d_{[2]}, \tilde{\g}_{[2]}) = 2$$
        \end{theorem}
            \begin{proof}
                As indicated in remark \ref{remark: non_uniqueness_of_yangian_extended_lie_algebras}, it only remains to prove that:
                    $$\sigma_a(D_t, -) = 0$$
                    $$\sigma_a(D_v, -), \sigma_a(D_{r, s}, -) \not = 0$$
                (where $a \in \{1, 2\}$) via explicit computations. For our own convenience, let us temporarily relabel the variables as:
                    $$v := v_1, t := v_2$$
                \begin{itemize}
                    \item Firstly, note that:
                        $$\sigma_1(\del_{v_2}, v_1^{r_1} v_2^{r_2} \del_{v_q}) = \sigma_1(v_1^0 v_2^0 \del_{v_2}, -) = r_1 \cdot 0 \cdot \sum_{1 \leq i \leq 2} (...) = 0$$
                        $$\sigma_2(\del_{v_2}, v_1^{r_1} v_2^{r_2} \del_{v_q}) = \sigma_2(v_1^0 v_2^0 \del_{v_2}, -) = 0 \cdot r_2 \cdot \sum_{1 \leq i \leq 2} (...) = 0$$
                    and hence we indeed have that:
                        $$\sigma_a(D_t, -) = \sigma_a(-\del_{v_2}, -) = -\sigma_a(\del_{v_2}, -) = 0$$
                    \item Secondly, to show that:
                        $$\sigma_a(D_{r, s}, -) \not = 0$$
                    it suffices to only show that:
                        $$\sigma_a(D_{r, s}, D_v) \not = 0$$
                    To this end, recall that:
                        $$D_{r, s} = s v_1^{-r + 1} v_2^{-s - 1} \del_{v_1} - r v_1^{-r} v_2^{-s} \del_{v_2}$$
                        $$D_v = -v_1 v_2^{-1} \del_{v_1}$$
                    which tells us that:
                        $$
                            \begin{aligned}
                                \sigma_1(D_{r, s}, D_v) & = ( s \cdot 1 \cdot (-r + 1) - r \cdot (-1) \cdot (-s) ) \sum_{1 \leq i \leq 2} r_i v_1^{(-r + 1) + 1} v_2^{(-s - 1) - 1} v_i^{-1} \bar{d}(v_i)
                                \\
                                & = (-2sr + 1) \sum_{1 \leq i \leq 2} r_i v_1^{-r} v_2^{-s - 2} v_i^{-1} \bar{d}(v_i)
                            \end{aligned}
                        $$
                    \item Lastly, to show that:
                        $$\sigma_a(D_v, -) \not = 0$$
                    simply note that:
                        $$\sigma_a(D_{r, s}, D_v) = -\sigma_a(D_v, D_{r, s})$$
                \end{itemize}
            \end{proof}
        \begin{corollary}
            More compactly, and given how $\d_{[2]}$ acts on $\g_{[2]} \subset \tilde{\g}_{[2]} \cong \g_{[2]} \oplus \z_{[2]}$ (cf. remark \ref{remark: derivation_action_on_multiloop_algebras}), the formula in theorem \ref{theorem: non_uniqueness_of_yangian_extended_lie_algebras} can be equivalently stated as:
                $$\dim_{\bbC} H^2_{\Lie}(\d_{[2]}, \z_{[2]}) = 2$$
        \end{corollary}
        \begin{corollary}
            One also has that:
                $$\dim_{\bbC} H^2_{\Lie}(\der_{\bbC}(A_{[2]}), \tilde{\g}_{[2]}) = \dim_{\bbC} H^2_{\Lie}(\der_{\bbC}(A_{[2]}), \z_{[2]}) = 2$$
            as a consequence of a combination of proposition \ref{prop: lie_bracket_on_orthogonal_complement_of_toroidal_centre} and theorem \ref{theorem: non_uniqueness_of_yangian_extended_lie_algebras}.
        \end{corollary}

    \subsection{The centre of \texorpdfstring{$\hat{\g}_{[2]}$}{}}
        Let us conclude this section with the following question, which is natural now that we have a solid handle on how the Lie bracket on $\hat{\g}_{[2]}$ is given:
        \begin{question}
            What is the centre $\hat{\z}_{[2]} := \z( \hat{\g}_{[2]} )$ ? This ought to be smaller than $\z_{[2]}$ somehow, since elements of $\z_{[2]}$ need not be central in $\hat{\g}_{[2]}$. 
        \end{question}
        \begin{remark}[Computing the centre without computing all the brackets ...]
            Since $\g_{[2]}$ is centreless, we have that:
                $$\hat{\z}_{[2]} = \z( \z_{[2]} \oplus \d_{[2]} )$$
            As $\z_{[2]}$ is an abelian Lie algebra, this implies that in order to compute $\hat{\z}_{[2]}$, it suffices to explicitly compute the commutators of the form:
                $$[D, K]_{\hat{\g}_{[2]}}, [D, D']_{\hat{\g}_{[2]}}$$
            for $D, D' \in \d_{[2]}$ and $K \in \z_{[2]}$, to see which ones vanish. However, this is rather tedious and not very insightful.
            
            An alternative method is as follows: exploiting the fact that the symmetric bilinear form $(-, -)_{\hat{\g}_{[2]}}$ is both invariant and non-degenerate (by construction; cf. convention \ref{conv: orthogonal_complement_of_toroidal_centres}), we can characterise the centre $\hat{\z}_{[2]}$ as the Lie ideal of $\hat{\g}_{[2]}$ containing elements $Z$ such that:
                $$0 = ([Z, X]_{\hat{\g}_{[2]}}, Y)_{\hat{\g}_{[2]}} = (Z, [X, Y]_{\hat{\g}_{[2]}})_{\hat{\g}_{[2]}}$$
            for any $X, Y \in \hat{\g}_{[2]}$, with the first equality holding thanks to the fact that $Z$ is supposed to commute with every other element of $\hat{\g}_{[2]}$ by assumption of being central. We are thus left with the task of finding elements:
                $$Z \in \hat{\g}_{[2]}$$
            such that:
                $$(Z, [\hat{\g}_{[2]}, \hat{\g}_{[2]}]_{\hat{\g}_{[2]}})_{\hat{\g}_{[2]}} = 0$$
            Since $\g_{[2]}$ and $\d_{[2]}$ is are both non-abelian Lie subalgebras of $\hat{\g}_{[2]}$, their elements can not be central in $\hat{\g}_{[2]}$. As such, we have narrowed the scope of our search down to:
                $$\hat{\z}_{[2]} \subset \z_{[2]}$$

            Another way to see that:
                $$\hat{\z}_{[2]} \subset \z_{[2]}$$
            is to use the fact that there are two isomorphism classes of Lie algebra extensions:
                $$0 \to \tilde{\g}_{[2]} \to \hat{\g}_{[2]} \to \d_{[2]} \to 0$$
            (cf. theorem \ref{theorem: non_uniqueness_of_yangian_extended_lie_algebras}). This tells us that:
                $$\hat{\z}_{[2]} \subset \z(\tilde{\g}_{[2]}) = \z_{[2]}$$
        \end{remark}
        \begin{proposition}[Centres of extended toroidal Lie algebras] \label{prop: centres_of_extended_toroidal_lie_algebras}
            The centre $\hat{\z}_{[2]}$ is a two-dimensional Lie subalgebra of $\z_{[2]}$, namely spanned by $c_v$ and $c_t$. 
        \end{proposition}
            \begin{proof}
                Since we know that:
                    $$\hat{\z}_{[2]} \subset \z_{[2]}$$
                and that the only possibly non-zero bracket with elements of $\z_{[2]}$ are elements of $[\d_{[2]}, \z_{[2]}]_{\hat{\g}_{[2]}}$, and since we also know from lemma \ref{lemma: explicit_commutators_between_central_basis_elements_and_derivations} that:
                    $$[\d_{[2]}, K]_{\hat{\g}_{[2]}} = 0 \iff K \in \bbC c_v \oplus \bbC c_v$$
                we can conclude immediately that:
                    $$\hat{\z}_{[2]} = \bbC c_v \oplus \bbC c_t$$
            \end{proof}
        \begin{remark}
            It is rather interesting that:
                $$\hat{\z}_{[2]} \cong \bbC c_v \oplus \bbC c_t$$
            as this is in good analogy with the affine Kac-Moody case, where the centre of $\hat{\g}_{[1]}$ is $1$-dimensional, namely spanned by the element $c_v$. 
        \end{remark}

    \section{Toroidal Lie bialgebras as classical limits of formal Yangians}
    \subsection{Chevalley-Serre and Levendorskii presentations for toroidal Lie algebras}
        \begin{convention} \label{conv: a_fixed_untwisted_affine_kac_moody_algebra}
            We introduce the following notations, which deviate slightly from those of e.g. \cite[Chapter 7]{kac_infinite_dimensional_lie_algebras}, so as to be consistent with the rest of our notes.
        
            Let us write:
                $$\hat{\g}_{[1]} := \tilde{\g}_{[1]} \rtimes \d_{[1]}$$
            to mean the untwisted affine Kac-Moody algebra attached to the finite-dimensional simple Lie algebra $\g$ from convention \ref{conv: a_fixed_finite_dimensional_simple_lie_algebra} (cf. \cite[Chapter 7]{kac_infinite_dimensional_lie_algebras}). Here, $\g_{[1]} := \g[v^{\pm 1}]$ is equipped with the invariant inner product given by:
                $$(x v^m, y v^n)_{\g_{[1]}} := (x, y)_{\g} \delta_{m + n = 0}$$
            for all $x, y \in \g$ and all $m, n \in \Z$, and $\d_{[1]}$ is the $1$-dimensional Lie algebra spanned by the derivation (with notations as in remark \ref{remark: dual_of_toroidal_centres_contains_derivations}):
                $$D_{0, -1} \in \der_{\bbC}(\tilde{\g}_{[1]})$$
            on $\tilde{\g}_{[1]} := \uce(\g_{[1]})$ acting as $\id_{\g} \tensor \left( -v \frac{d}{dv} \right)$ on $\g_{[1]}$ and as zero on $\z_{[1]} := \z(\tilde{\g}_{[1]})$; recall also that this centre is $1$-dimensional, and in particular, it is identifiable as:
                $$\z_{[1]} \cong \bbC K_{0, -1}$$
            (see remark \ref{remark: centres_of_dual_toroidal_lie_algebras} for how the elements $K_{r, s}$ are given).

            Fix a Cartan subalgebra $\hat{\h}_{[1]}$ of $\hat{\g}_{[1]}$. The affine Dynkin diagram associated to $\hat{\g}_{[1]}$ (cf. \cite[Chapter 4]{kac_infinite_dimensional_lie_algebras}) shall be denoted by:
                $$\hat{\Gamma} := ( \hat{\Gamma}_0, \hat{\Gamma}_1 )$$
            with $\hat{\Gamma}_0$ denoting the set of vertices and $\hat{\Gamma}_1$ denoting the set of undirected edges.

            Also, let us denote the highest root by $\theta$. Note that we have a bijection (but not literally an equality; cf. \cite[Chapters 4, 5, 7]{kac_infinite_dimensional_lie_algebras}):
                $$\hat{\Gamma}_0 \cong \Gamma_0 \cup \{\theta\}$$
        \end{convention}
    
        Let us now give a demonstration of the existence of a Chevalley-Serre-type presentation for the Lie algebras $\tilde{\g}_{[2]}$ and $\tilde{\g}_{[2]}^{\pm}$. The point of doing this is two-fold:
        \begin{enumerate}
            \item The formal Yangian $\rmY_{\hbar}(\hat{\g}_{[1]})$ (respectively, Yangian $\rmY(\hat{\g}_{[1]})$) associated to $\hat{\g}_{[1]}$ is given as a certain associative $\bbC$-algebra defined by a similar Chevalley-Serre-type presentation, and the goal is usually to somehow prove that once one reduces modulo $\hbar$ (respectively, take the associated graded algebra), one shall obtain a $\bbC$-algebra isomorphism with the universal enveloping algebra $\rmU(\tilde{\g}_{[2]}^+)$.
            \item The (formal) Yangian assoicated to $\hat{\g}_{[1]}$ enjoys a more convenient presentation by generators of degrees $0$ and $1$ (in the variable $t$) and hence so does $\tilde{\g}_{[2]}^+$ (and in fact, the Lie algebras $\tilde{\g}_{[2]}$ and $\tilde{\g}_{[2]}^-$; see proposition \ref{prop: levendorskii_presentation__for_central_extensions_of_multiloop_algebras} and corollary \ref{coro: levendorskii_presentation__for_central_extensions_of_multiloop_algebras}) which eventually, will become useful in the proof of theorem \ref{theorem: toroidal_lie_bialgebras} - wherein we establish the existence of a (topological\footnote{We will be clearer about what this means later.}) Lie bialgebra structure on $\tilde{\g}_{[2]}^+$. These presentations in terms of low-degree generators will be known as the Levendorskii-type presentations.
        \end{enumerate}
        
        \begin{lemma}[Chevalley-Serre presentation for $\tilde{\g}_{[2]}$] \label{lemma: chevalley_serre_presentation_for_central_extensions_of_multiloop_algebras}
            (Cf. \cite[Proposition 6.6]{wendlandt_formal_shift_operators_on_yangian_doubles}) The Lie algebra $\tilde{\g}_{[2]}$ is isomorphic to the Lie algebra $\t$ generated by the set:
                $$\{ E_{i, r}^{\pm}, H_{i, r} \}_{(i, r) \in \hat{\Gamma}_0 \x \Z} \cup \{ K \}$$
            whose elements are subjected to the following relations, given for all $(i, r), (j, s) \in \hat{\Gamma}_0 \x \Z$:
                $$[ H_{i, r}, H_{j, s} ] = 0$$
                $$[ H_{i, r}, E_{j, s}^{\pm} ] = \pm (\alpha_j, \check{\alpha}_i) E_{j, r + s}^{\pm}$$
                $$[ E_{i, r}^+, E_{j, s}^- ] = \delta_{ij} H_{i, r + s}$$
                $$[ E_{i, r + 1}^{\pm}, E_{j, s}^{\pm} ] - [ E_{i, r}^{\pm}, E_{j, s + 1}^{\pm} ] = 0$$
                $$[K, \t] = 0$$
            The isomorphism $\t \xrightarrow[]{\cong} \tilde{\g}_{[2]}$ in question is given as follows, for all $(i, r) \in \hat{\Gamma}_0 \x \Z$:
                $$\forall (i, r) \in \Gamma_0 \x \Z: E_{i, r}^{\pm} \mapsto e_i^{\pm} t^r, H_{i, r} \mapsto h_i t^r$$
                $$\forall (i, r) \in \{\theta\} \x \Z: E_{\theta, r}^{\pm} \mapsto e_{\theta}^{\mp} v^{\pm 1} t^r, H_{\theta, r} \mapsto h_{\theta} t^r + c_v t^r$$
                $$K \mapsto c_t$$
        \end{lemma}
        \begin{corollary} \label{coro: chevalley_serre_presentation_for_central_extensions_of_multiloop_algebras}
            The Lie algebras $\tilde{\g}_{[2]}^{\pm}$ is isomorphic to the Lie algebras $\t^{\pm}$ generated, respectively, by the sets:
                $$\{ E_{i, r}^{\pm}, H_{i, r} \}_{(i, r) \in \Gamma_0 \x \Z_{\geq 0}}$$
                $$\{ E_{i, r}^{\pm}, H_{i, r} \}_{(i, r) \in \Gamma_0 \x \Z_{< 0}} \cup \{K\}$$
            whose elements are subjected to the same relations as in lemma \ref{lemma: chevalley_serre_presentation_for_central_extensions_of_multiloop_algebras}. The isomorphisms $\t^{\pm} \xrightarrow[]{\cong} \tilde{\g}_{[2]}^{\pm}$ in question are just codomain restrictions of the isomorphism $\t \xrightarrow[]{\cong} \tilde{\g}_{[2]}$ from lemma \ref{lemma: chevalley_serre_presentation_for_central_extensions_of_multiloop_algebras}.
        \end{corollary}

        Let us now demonstrate how a low-degree presentation for the Lie algebras $\tilde{\g}_{[2]}$ and $\tilde{\g}_{[2]}^{\pm}$ may be obtained, as eluded to earlier. This necessitates introducing the (formal) Yangian associated to the untwisted affine Kac-Moody algebra $\hat{\g}_{[1]}$ because as explained above, we will be relying on the existence of such a low-degree presentation for those (formal) Yangians.
        \begin{convention}[Yangians associated to symmetrisable Kac-Moody algebras]
            If $\fraku$ a general symmetrisable Kac-Moody algebra whose associated Cartan matrix is indecomposable. We refer the reader to \cite[Section 2]{guay_nakajima_wendlandt_affine_yangian_coproduct} for the definition of the \textbf{formal Yangian} $\rmY_{\hbar}(\fraku)$ and \textbf{Yangian} $\rmY(\fraku)$, as well as all relevant discussions about the various \say{basic} presentations of these associative algebras (living over $\bbC[\hbar]$ and $\bbC$ respectively). The only thing that we will note is that we will be denoting the Chevalley-Serre generators by:
                $$E_{i, r}^{\pm}, H_{i, r}$$
        \end{convention}
        \begin{convention}
            From now on, let us write:
                $$T_{i, 1}(\hbar) := H_{i, 1} - \frac12 \hbar H_{i, 0}^2$$
                $$T_{i, 1} := T_{i, 1}(1) = H_{i, 1} - \frac12 H_{i, 0}^2$$
        \end{convention}

        One key property of formal (affine) Yangians that we will be relying on is the fact that 
        \begin{lemma}[Formal Yangians as Rees algebras] \label{lemma: formal_yangians_as_rees_algebras}
            \cite[Theorem 6.10]{guay_nakajima_wendlandt_affine_yangian_vertex_representations_and_PBW} If $\fraku$ is a general indecomposable symmetrisable Kac-Moody algebra. If $\fraku$ is either of finite type but not $\sfA_1$ or of untwisted affine type but not $\sfA_1^{(1)}$ and not $\sfA_1^{(2)}$ then the natural \textit{graded} $\bbC$-algebra homomoprhism:
                $$\rmY_{\hbar}(\fraku) \to \Rees_{\hbar} \rmY(\fraku)$$
            will be an isomorphism. 
        \end{lemma}
         \begin{corollary}[Formal affine Yangians as flat graded deformations] \label{coro: formal_affine_yangians_as_flat_graded_deformations}
            Suppose that $\g \not \cong \sl_2(\bbC)$. Then the $\bbC[\hbar]$-algebra $\rmY_{\hbar}(\hat{\g}_{[1]})$ will be a flat $\Z$-graded deformation of the $\Z$-graded $\bbC$-algebra $\rmU(\tilde{\g}_{[2]}^+)$. 
         \end{corollary}
         
        The hypotheses of the following lemma are satisfied at least when $\g$ is either of finite type or of affine type, save for the types $\sfA_1^{(1)}$ and $\sfA_1^{(2)}$.
        \begin{lemma}[A Levendorskii-type presentation for Yangians of Kac-Moody algebras] \label{lemma: levendorskii_presentation}
            \cite[Theorem 2.13]{guay_nakajima_wendlandt_affine_yangian_coproduct} Suppose for a moment that $\fraku$ is a general symmetrisable Kac-Moody algebra whose Cartan matrix is:
            \begin{itemize}
                \item indecomposable,
                \item such that, for any $i < j \in \Gamma_0$ (with respect to some choice of total ordering on $\Gamma_0$) the following $2 \x 2$ matrix is invertible:
                    $$
                        \begin{pmatrix}
                            c_{ii} & c_{ij}
                            \\
                            c_{ji} & c_{ji}
                        \end{pmatrix}
                    $$
            \end{itemize}
            The formal Yangian $\rmY_{\hbar}(\fraku)$ of $\fraku$ will then be isomorphic to the associative $\bbC$-algebra generated by the set:
                $$\{ H_{i, r}, E_{i, r}^{\pm} \}_{(i, r) \in \Gamma_0 \x \N}$$
            whose elements are subjected to the following relations\footnote{... and it is understood that the elements $H_{i, 0} = h_i, E_{i, 0}^{\pm} = e_i^{\pm}$ satisfy the Chevalley-Serre relations defining $\fraku$; cf. \cite[Chapter 1]{kac_infinite_dimensional_lie_algebras}.}:
                $$H_{i, 0} = h_i, E_{i, 0}^{\pm} = e_i^{\pm}$$
                $$[ H_{i, r}, H_{j, s} ] = 0$$
                $$[ H_{i, 0}, E_{j, s}^{\pm} ] = \pm c_{ij} E_{j, s}^{\pm}$$
                $$[ E_{i, r}^+, E_{j, s}^- ] = \pm \delta_{ij} H_{i, r + s}$$
                $$\left[ T_{i, 1}(\hbar), E_{j, 0}^{\pm} \right] = \pm \hbar c_{ij} E_{j, 1}^{\pm}$$
                $$[ E_{i, 1}^{\pm}, E_{j, 0}^{\pm} ] - [ E_{i, 0}^{\pm}, E_{j, 1}^{\pm} ] = \pm \frac12 \hbar c_{ij} \{E_{i, 0}^{\pm}, E_{j, 0}^{\pm}\}$$
        \end{lemma}
        \begin{proposition}[Levendorskii presentation for $\tilde{\g}_{[2]}$] \label{prop: levendorskii_presentation__for_central_extensions_of_multiloop_algebras}
            The Lie algebra $\tilde{\g}_{[2]}$ is isomorphic to the Lie algebra $\t$ generated by the set:
                $$\{ E_{i, r}^{\pm}, H_{i, r} \}_{(i, r) \in \hat{\Gamma}_0 \x \Z} \cup \{ K \}$$
            whose elements are subjected to the following relations, given for all $(i, r), (j, s) \in \hat{\Gamma}_0 \x \Z$:
                $$[\t, K] = 0$$
                $$H_{i, 0} = h_i, E_{i, 0}^{\pm} = e_i^{\pm}$$
                $$[ H_{i, r}, H_{j, s} ] = 0$$
                $$[ H_{i, 0}, E_{j, s}^{\pm} ] = \pm (\alpha_j, \check{\alpha}_i) E_{j, s}^{\pm}$$
                $$[ E_{i, r}^+, E_{j, s}^- ] = \delta_{ij} H_{i, r + s}$$
                $$[ E_{i, 1}^{\pm}, E_{j, 0}^{\pm} ] - [ E_{i, 0}^{\pm}, E_{j, 1}^{\pm} ] = 0$$
            The isomorphism $\t \xrightarrow[]{\cong} \tilde{\g}_{[2]}$ in question is as in lemma \ref{lemma: chevalley_serre_presentation_for_central_extensions_of_multiloop_algebras}.
        \end{proposition}
            \begin{proof}
                Combine corollary \ref{coro: formal_affine_yangians_as_flat_graded_deformations} with lemma \ref{lemma: levendorskii_presentation}. Note that flatness (in particular, $\hbar$-torsion-freeness) is crucial for our application of corollary \ref{coro: formal_affine_yangians_as_flat_graded_deformations}.
            \end{proof}
        \begin{corollary} \label{coro: levendorskii_presentation__for_central_extensions_of_multiloop_algebras}
            The Lie algebras $\tilde{\g}_{[2]}^{\pm}$ are, respectively, isomorphic to the Lie algebra $\t^{\pm}$ generated by the set:
                $$\{ E_{i, r}^{\pm}, H_{i, r} \}_{(i, r) \in \Gamma_0 \x \Z_{\geq 0}}$$
                $$\{ E_{i, r}^{\pm}, H_{i, r} \}_{(i, r) \in \Gamma_0 \x \Z_{< 0}} \cup \{K\}$$
            whose elements are subjected to the relations as in proposition \ref{prop: levendorskii_presentation__for_central_extensions_of_multiloop_algebras}. The isomorphism $\t^{\pm} \xrightarrow[]{\cong} \tilde{\g}_{[2]}^{\pm}$ in question is as in lemma \ref{lemma: chevalley_serre_presentation_for_central_extensions_of_multiloop_algebras}.
        \end{corollary}

    \subsection{Toroidal Lie bialgebras}
        \begin{convention} \label{conv: positive_and_negative_parts_of_derivation_subalgebras_of_extended_toroidal_algebras}
            Let us now adopt the following notations:
            \begin{itemize}
                \item
                    $$\d_{[2]}^+ := ( \bigoplus_{(r, s) \in \Z \x \Z_{\leq 0} } \bbC D_{r, s} ) \oplus \bbC D_t$$
                    $$\d_{[2]}^- := ( \bigoplus_{(r, s) \in \Z \x \Z_{> 0} } \bbC D_{r, s} ) \oplus \bbC D_v$$
                shall respectively be the Lie subalgebras of $\d_{[2]}$ which are graded-dual to $\z_{[2]}^{\pm}$ with respect to $(-, -)_{\hat{\g}_{[2]}}$;
                \item $\hat{\g}_{[2]}^{\pm} := \tilde{\g}_{[2]}^{\pm} \rtimes \d_{[2]}^{\pm}$.
            \end{itemize}
        \end{convention}    
        \begin{theorem} \label{theorem: extended_toroidal_manin_triples}
            There is a complete topological Manin triple:
                $$(\hat{\g}_{[2]}, \hat{\g}_{[2]}^+, \hat{\g}_{[2]}^-)$$
            wherein $\hat{\g}_{[2]}$ is equipped with the non-degenerate invariant inner product $(-, -)_{\hat{\g}_{[2]}}$ (cf. convention \ref{conv: orthogonal_complement_of_toroidal_centres}).
        \end{theorem}
            \begin{proof}
                Given how $\d_{[2]}$ acts on $\g_{[2]}$ and on $\z_{[2]}$ (cf. remarks \ref{remark: derivation_action_on_multiloop_algebras} and \ref{remark: derivation_action_on_toroidal_centres} respectively), which implies in particular that:
                    $$[\d_{[2]}, \tilde{\g}_{[2]}^{\pm}] \subseteq \tilde{\g}_{[2]}^{\pm}$$
                it shall suffice to only demonstrate that the vector spaces $\d_{[2]}^{\pm}$ are Lie subalgebras of $\d_{[2]}$ in order to show that $\hat{\g}_{[2]}^{\pm}$ are Lie subalgebras of $\hat{\g}_{[2]}$. To do this, the only computations that we will need to make are of the commutators:
                    $$[D_{r, s}, D_t], (r, s) \in \Z \x \Z_{\leq 0}$$
                    $$[D_{r, s}, D_v], (r, s) \in \Z \x \Z_{> 0}$$
                which is entirely possible since we know how the elements $D_{r, s}, D_v, D_t$ are given in terms of the partial derivatives $\del_v, \del_t$ (cf. remark \ref{remark: dual_of_toroidal_centres_contains_derivations}).

                It now remains to show that $(-, -)_{\hat{\g}_{[2]}}$ pairs the subalgebras $\hat{\g}_{[2]}^{\pm}$ isotropically, but this is true entirely due to how this invariant bilinear form was constructed in convention \ref{conv: orthogonal_complement_of_toroidal_centres}.
            \end{proof}
        \begin{corollary}[Lie cobracket on $\hat{\g}_{[2]}^+$] \label{coro: extended_toroidal_lie_bialgebras}
            On the extended toroidal Lie algebra $\hat{\g}_{[2]}^+$, there is a continuous Lie cobracket\footnote{Note the completion!}, making $\hat{\g}_{[2]}^+$ a complete topological Lie bialgebra:
                $$\delta_{\hat{\g}_{[2]}^+}: \hat{\g}_{[2]}^+ \to \hat{\g}_{[2]}^+ \hattensor_{\bbC} \hat{\g}_{[2]}^+$$
            given for any $X \in \hat{\g}_{[2]}^+$ by the following formula (cf. \cite{etingof_kazhdan_quantisation_1}):
                $$\delta_{\hat{\g}_{[2]}^+}(X) = [ X \tensor 1 + 1 \tensor X, \sfr_{\hat{\g}_{[2]}^+} ]$$
            wherein:
                $$\sfr_{\hat{\g}_{[2]}^+} := \sfr_{\g_{[2]}^+} + \sfr_{\z_{[2]}^+} + \sfr_{\d_{[2]}^+} \in \hat{\g}_{[2]}^+ \hattensor_{\bbC} \hat{\g}_{[2]}^-$$
            with $\sfr_{\g_{[2]}^+} \in \g_{[2]}^+ \hattensor_{\bbC} \g_{[2]}^-$ being as in question \ref{question: multiloop_lie_bialgebras} and\footnote{Note how we are simply summing over tensor products of dual basis elements.} $\sfr_{\z_{[2]}^+} \in \z_{[2]}^+ \hattensor_{\bbC} \d_{[2]}^+$ and $\sfr_{\d_{[2]}^+} \in \d_{[2]}^+ \hattensor_{\bbC} \z_{[2]}^+$ being given by the following formulae:
                $$\sfr_{\z_{[2]}^+} := \sum_{(r, s) \in \Z \x \Z_{> 0}} K_{r, s} \tensor D_{r, s} + c_v \tensor D_v$$
                $$\sfr_{\d_{[2]}^+} := \sum_{(r, s) \in \Z \x \Z_{\leq 0}} D_{r, s} \tensor K_{r, s} + D_t \tensor c_t$$
        \end{corollary}
    
        \begin{convention}[Formal Dirac distributions] \label{conv: formal_dirac_distributions}
            We will be using the following shorthands:
                $$\1(z, w) = \sum_{m \in \Z} z^m w^{-m - 1}$$
                $$\1^+(z, w) = \sum_{m \in \Z_{\geq 0}} z^m w^{-m - 1}$$
            as well as:
                $$\{ X_1, ..., X_n \} := \sum_{\sigma \in S_n} X_{\sigma(1)} \cdot ... \cdot X_{\sigma(n)}$$
                $$\bar{\Delta}(X) := X \tensor 1 + 1 \tensor X$$
        \end{convention}

        \begin{remark}[Total degrees of \say{Yangian} canonical elements] \label{remark: total_degrees_of_classical_yangian_R_matrices}
            One property of the R-matrix $\sfr_{\hat{\g}_{[2]}^+}$ from corollary \ref{coro: extended_toroidal_lie_bialgebras} that will help simplify some computations later on (see the proof of theorem \ref{theorem: toroidal_lie_bialgebras}) is that they are of total degree $-1$. 

            Recall that if $V := \bigoplus_{m \in \Z} V_m, W := \bigoplus_{n \in \Z} W_n$ are $\Z$-graded vector spaces then for any $k \in \Z$, we have that:
                $$(V \tensor_{\bbC} W)_k \cong \bigoplus_{m + n = k} V_m \tensor_{\bbC} W_n$$
                
            If we now take $V = W = \rmU(\tilde{\g}_{[2]})$ then the claim from above would read:
                $$\sfr_{\tilde{\g}_{[2]}^+} \in ( \rmU(\tilde{\g}_{[2]}^+) \tensor_{\bbC} \rmU(\tilde{\g}_{[2]}^-) )_{-1}$$
            with the $\Z$-grading on $\tilde{\g}_{[2]}^{\pm}$ (and hence on $\rmU(\tilde{\g}_{[2]}^{\pm})$) being the one on the second variable $t$ (cf. remark \ref{remark: Z_gradings_on_toroidal_lie_algebras}), and actually, this is entirely due to:
                $$\sfr_{\g_{[2]}^+} \in ( \rmU(\g_{[2]}^+) \tensor_{\bbC} \rmU(\g_{[2]}^-) )_{-1}$$
            and this can already be inferred from the computations done in question \ref{question: multiloop_lie_bialgebras}, where we showed that:
                $$\sfr_{\g_{[2]}^+} = \sfr_{\g} v_2 \1(v_1, v_2) \1^+(t_1, t_2)$$
            (the crucial detail to notice here is that $\deg \1^+(t_1, t_2) = -1$ since $\1^+(t_1, t_2) := \sum_{p \in \Z_{\geq 0}} t_1^p t_2^{-p - 1}$).

            What this means for us is that, should we have $X \in \tilde{\g}_{[2]}^+$ such that:
                $$\deg X \leq 0$$
            then it will automatically be the case that:
                $$\delta_{\tilde{\g}_{[2]}^+}(X) = 0$$
        \end{remark}
        
        We are now finally able to put a Lie cobracket on the toroidal Lie algebra $\tilde{\g}_{[2]}^+$, compatible with the Lie bracket thereon in a manner that produces a Lie bialgebra structure. This Lie bialgebra structure is the classical limit of the coproduct on the formal Yangian $\rmY_{\hbar}(\hat{\g}_{[1]})$. 
        \begin{theorem}[Toroidal Lie bialgebras] \label{theorem: toroidal_lie_bialgebras}
            Assume convention \ref{conv: a_fixed_untwisted_affine_kac_moody_algebra} and let us abbreviate:
                $$\hat{\delta}^+ := \delta_{\hat{\g}_{[2]}^+}$$
            with $\delta_{\hat{\g}_{[2]}^+}$ as in corollary \ref{coro: extended_toroidal_lie_bialgebras}. Let:
                $$\tilde{\delta}^+ := \hat{\delta}^+|_{\tilde{\g}_{[2]}}$$
            Then $(\tilde{\g}_{[2]}^+, \tilde{\delta}^+)$ will be a complete topological Lie sub-bialgebra of $(\hat{\g}_{[2]}^+, \hat{\delta}^+)$ as given in corollary \ref{coro: extended_toroidal_lie_bialgebras}. Thanks to corollary \ref{coro: levendorskii_presentation__for_central_extensions_of_multiloop_algebras}, we know that it is enough to specify how $\tilde{\delta}^+$ is given on the set of generators:
                $$\{E_{i, 0}^{\pm}\}_{i \in \hat{\Gamma}_0} \cup \{H_{i, r}\}_{ (i, r) \in \hat{\Gamma}_0 \x \{0, 1\} }$$
            and since we know that under the isomorphism $\t^+ \xrightarrow[]{\cong} \tilde{\g}_{[2]}^+$ in \textit{loc. cit.}, we have the following assignments:
                $$\forall i \in \hat{\Gamma}_0: E_{i, 0}^{\pm} \mapsto e_i^{\pm}, H_{i, 0} \mapsto h_i$$
                $$\forall i \in \Gamma_0: H_{i, 1} \mapsto h_i t$$
                $$H_{\theta, 1} \mapsto h_{\theta} t + t c_v$$
            it is enough to specify the following, wherein $h \in \h$ is arbitrary:
                $$\tilde{\delta}^+(h) = 0$$
                $$\tilde{\delta}^+(ht) = [h_1 \tensor 1, \sfr_{\g} v_2 \1(v_1, v_2)]$$
                $$\tilde{\delta}^+(t c_v) = 0$$
        \end{theorem}
            \begin{proof}
                \begin{enumerate}
                    \item Since $\deg x = 0$ for all $x \in \g$, we get via remark \ref{remark: total_degrees_of_classical_yangian_R_matrices} that:
                        $$\hat{\delta}^+(x) = 0$$
                    and in particular, we have that:
                        $$\hat{\delta}^+(h) = 0$$

                    \item Let us now compute $\hat{\delta}^+(ht)$ for an arbitrary $h \in \h$. 
                    \begin{enumerate}
                        \item \textbf{($\g_{[2]}^+$-component):} Firstly, to compute:
                            $$[\bar{\Delta}(ht), \sfr_{\g_{[2]}^+}]$$
                        let us firstly recall from question \ref{question: multiloop_lie_bialgebras} that:
                            $$\sfr_{\g_{[2]}^+} = \sfr_{\g} v_2\1(v_1, v_2) \1^+(t_1, t_2)$$
                        (with notations as in \textit{loc. cit.}); let us also choose a root basis for $\g$ for writing out $\sfr_{\g}$ explicitly: this is to say that for each positive root $\alpha \in \Phi^+$, we choose corresponding basis vectors $e_{\alpha}^{\pm} \in \g_{\pm \alpha}$ normalised so that:
                            $$(e_{\alpha}^-, e_{\alpha}^+)_{\g} = 1$$
                        to get the following basis for $\g$:
                            $$\{h_i\}_{i \in \Gamma_0} \cup \{e_{\alpha}^-, e_{\alpha}^+\}_{\alpha \in \Phi^+}$$
                        From this, we see that:
                            $$
                                \begin{aligned}
                                    & [\bar{\Delta}(ht), \sfr_{\g_{[2]}^+}]
                                    \\
                                    = & -\sum_{i \in \Gamma_0} [\bar{\Delta}(ht), h_i \tensor h_i v_2\1(v_1, v_2) \1^+(t_1, t_2)] - \sum_{\alpha \in \Phi^+} [\bar{\Delta}(ht), (e_{\alpha}^- \tensor e_{\alpha}^+ + e_{\alpha}^+ \tensor e_{\alpha}^-) v_2\1(v_1, v_2) \1^+(t_1, t_2)]
                                \end{aligned}
                            $$

                        Now, for each $i \in \Gamma_0$, observe that:
                            $$
                                \begin{aligned}
                                    & [h t_1 \tensor 1, h_i \tensor h_i v_2\1(v_1, v_2) \1^+(t_1, t_2)]
                                    \\
                                    = & \sum_{(m, p) \in \Z \x \Z_{\geq 0}} [ht_1 \tensor 1, h_i v_1^m t_1^p \tensor h_i v_2^{-m} t_2^{-p - 1}]
                                    \\
                                    = & \sum_{(m, p) \in \Z \x \Z_{\geq 0}} [ht_1, h_i v_1^m t_1^p]_{\tilde{\g}_{[2]}^+} \tensor h_i v_2^{-m} t_2^{-p - 1}
                                    \\
                                    = & \sum_{(m, p) \in \Z \x \Z_{\geq 0}} (h, h_i)_{\g} v_1^m t_1^p \bar{d}(t_1) \tensor h_i v_2^{-m} t_2^{-p - 1}
                                \end{aligned}
                            $$
                        and likewise, that:
                            $$[1 \tensor h t_2, h_i \tensor h_i v_2\1(v_1, v_2) \1^+(t_1, t_2)] = \sum_{(m, p) \in \Z \x \Z_{\geq 0}} h_i v_1^m t_1^p \tensor (h, h_i)_{\g} v_2^{-m} t_2^{-p - 1} \bar{d}(t_2)$$
                        Adding the two summands together then yields:
                            $$
                                \begin{aligned}
                                    & [\bar{\Delta}(ht), h_i \tensor h_i v_2\1(v_1, v_2) \1^+(t_1, t_2)]
                                    \\
                                    = & (h, h_i)_{\g} \sum_{(m, p) \in \Z \x \Z_{\geq 0}} \left( v_1^m t_1^p \bar{d}(t_1) \tensor h_i v_2^{-m} t_2^{-p - 1} + h_i v_1^m t_1^p \tensor v_2^{-m} t_2^{-p - 1} \bar{d}(t_2) \right)
                                    \\
                                    = & (h, h_i)_{\g} ( \bar{d}(t_1) \tensor h_i + h_i \tensor \bar{d}(t_2) ) v_2\1(v_1, v_2) \1^+(t_1, t_2)
                                \end{aligned}
                            $$
                        
                        Next, consider the following:
                            $$
                                \begin{aligned}
                                    & [ht_1 \tensor 1, e_{\alpha}^- \tensor e_{\alpha}^+ v_2\1(v_1, v_2) \1^+(t_1, t_2)]
                                    \\
                                    = & \sum_{(m, p) \in \Z \x \Z_{\geq 0}} [ht_1 \tensor 1, e_{\alpha}^- v_1^m t_1^p \tensor e_{\alpha}^+ v_2^{-m} t_2^{-p - 1}]
                                    \\
                                    = & \sum_{(m, p) \in \Z \x \Z_{\geq 0}} [ht_1, e_{\alpha}^- v_1^m t_1^p]_{\tilde{\g}_{[2]}^+} \tensor e_{\alpha}^+ v_2^{-m} t_2^{-p - 1}
                                    \\
                                    = & \sum_{(m, p) \in \Z \x \Z_{\geq 0}} \left( -\alpha(h) e_{\alpha}^- v_1^m t_1^{p + 1} + (h, e_{\alpha}^-)_{\g} t_1 \bar{d}(v_1^m t_1^p) \right) \tensor e_{\alpha}^+ v_2^{-m} t_2^{-p - 1}
                                    \\
                                    = & \sum_{(m, p) \in \Z \x \Z_{\geq 0}} -\alpha(h) e_{\alpha}^- v_1^m t_1^{p + 1} \tensor e_{\alpha}^+ v_2^{-m} t_2^{-p - 1}
                                    \\
                                    & = -\alpha(h) ( e_{\alpha}^- \tensor e_{\alpha}^+ ) v_2 \1(v_1, v_2) t_1 \1^+(t_1, t_2)
                                \end{aligned}    
                            $$
                        wherein the second-to-last identity comes from the fact that\footnote{This can be proven easily by passing to the vector representation of $\g$, wherein $h$ is represented by a diagonal matrix while $e^{\pm}$ is represented by an upper/lower triangular matrix, and then using the fact that $(-, -)_{\g}$ differs from the trace form only by a non-zero constant.}:
                            $$(h, e^{\pm})_{\g} = 0$$
                        for every $h \in \h$ and every $e^{\pm} \in \n^{\pm}$. Similarly, we find that:
                            $$[ht_1 \tensor 1, e_{\alpha}^+ \tensor e_{\alpha}^- v_2\1(v_1, v_2) \1^+(t_1, t_2)] = \alpha(h) ( e_{\alpha}^+ \tensor e_{\alpha}^- ) v_2 \1(v_1, v_2) t_1 \1^+(t_1, t_2)$$
                        By putting the two together, one obtains:
                            $$[h t_1 \tensor 1, (e_{\alpha}^- \tensor e_{\alpha}^+ + e_{\alpha}^+ \tensor e_{\alpha}^-) v_2\1(v_1, v_2) \1^+(t_1, t_2)] = -\alpha(h) ( e_{\alpha}^- \tensor e_{\alpha}^+ - e_{\alpha}^+ \tensor e_{\alpha}^- ) v_2 \1(v_1, v_2) t_1 \1^+(t_1, t_2)$$
                        Likewise, we find that:
                            $$[1 \tensor h t_2, (e_{\alpha}^- \tensor e_{\alpha}^+ + e_{\alpha}^+ \tensor e_{\alpha}^-) v_2\1(v_1, v_2) \1^+(t_1, t_2)] = \alpha(h) ( e_{\alpha}^- \tensor e_{\alpha}^+ - e_{\alpha}^+ \tensor e_{\alpha}^- ) v_2 \1(v_1, v_2) t_2 \1^+(t_1, t_2)$$
                        and hence:
                            $$
                                \begin{aligned}
                                    & [\bar{\Delta}(ht), \sfr_{\g_{[2]}^+}]
                                    \\
                                    = & -\left( \sum_{i \in \Gamma_0} (h, h_i)_{\g} ( \bar{d}(t_1) \tensor h_i + h_i \tensor \bar{d}(t_2) ) + \sum_{\alpha \in \Phi^+} \alpha(h) ( e_{\alpha}^- \tensor e_{\alpha}^+ - e_{\alpha}^+ \tensor e_{\alpha}^- )(t_2 - t_1) \right) v_2 \1(v_1, v_2) \1^+(t_1, t_2)
                                    \\
                                    & = -\left( \bar{d}(t_1) \tensor h + h \tensor \bar{d}(t_2) + [h_1 \tensor 1, \sfr_{\g}] (t_2 - t_1) \right) v_2 \1(v_1, v_2) \1^+(t_1, t_2)
                                    \\
                                    & = -\left( \bar{d}(t_1) \tensor h + h \tensor \bar{d}(t_2) \right) v_2 \1(v_1, v_2) \1^+(t_1, t_2) + [h_1 \tensor 1, \sfr_{\g}] v_2 \1(v_1, v_2)
                                \end{aligned}
                            $$
                        We note that the last equality holds thanks to the fact that:
                            $$(t_2 - t_1) \1^+(t_1, t_2) = (t_2 - t_1) \sum_{p \in \Z_{\geq 0}} t_1^p t_2^{-p - 1} = (t_2 - t_1) \frac{1}{t_2 - t_1} = 1$$
                            
                        \item \textbf{($\z_{[2]}^+$-component):} Recall from corollary \ref{coro: extended_toroidal_lie_bialgebras} that:
                            $$\sfr_{\z_{[2]}^+} := \sum_{(r, s) \in \Z \x \Z_{> 0}} K_{r, s} \tensor D_{r, s} + c_{v_1} \tensor D_{v_2}$$
                        and so:
                            $$
                                \begin{aligned}
                                    & [\bar{\Delta}(ht), \sfr_{\z_{[2]}^+}]
                                    \\
                                    = & \sum_{(r, s) \in \Z \x \Z_{> 0}} [\bar{\Delta}(ht), K_{r, s} \tensor D_{r, s}] + [\bar{\Delta}(ht), c_{v_1} \tensor D_{v_2}]
                                    \\
                                    = & -\sum_{(r, s) \in \Z \x \Z_{> 0}} K_{r, s} \tensor h D_{r, s}(t) - c_{v_1} \tensor h D_{v_2}(t_2)
                                    \\
                                    = & -\sum_{(r, s) \in \Z \x \Z_{> 0}} K_{r, s} \tensor r h v_2^{-r} t_2^{-s}
                                \end{aligned}
                            $$
                        where the minus sign in the third equation appeared because:
                            $$[ht, D_{r, s}] = -[D_{r, s}, ht] = -h D_{r, s}(t)$$
                            $$[ht, D_v] = -[D_v, ht] = -h D_v(t)$$
                        (cf. remark \ref{remark: derivation_action_on_multiloop_algebras}) and the last equality is due to the fact that:
                            $$D_{r, s} = -s v^{-r + 1} t^{-s - 1} \del_v + r v^{-r} t^{-s} \del_t$$
                            $$D_v = -v t^{-1} \del_v$$
                        (cf. remark \ref{remark: dual_of_toroidal_centres_contains_derivations}). 
                        
                        \item \textbf{($\d_{[2]}^+$-component):} Recall from corollary \ref{coro: extended_toroidal_lie_bialgebras} that:
                            $$\sfr_{\z_{[2]}^+} := \sum_{(r, s) \in \Z \x \Z_{\leq 0}} D_{r, s} \tensor K_{r, s} + D_{t_1} \tensor c_{t_2}$$
                        and so:
                            $$
                                \begin{aligned}
                                    & [\bar{\Delta}(ht), \sfr_{\z_{[2]}^+}]
                                    \\
                                    = & \sum_{(r, s) \in \Z \x \Z_{\leq 0}} [\bar{\Delta}(ht), D_{r, s} \tensor K_{r, s}] + [\bar{\Delta}(ht), D_{t_1} \tensor c_{t_2}]
                                    \\
                                    = & -\sum_{(r, s) \in \Z \x \Z_{\leq 0}} h D_{r, s}(t_1) \tensor K_{r, s} - h D_{t_1}(t_1) \tensor c_{t_2}
                                    \\
                                    = & -\sum_{(r, s) \in \Z \x \Z_{\leq 0}} r h v_1^{-r} t_1^{-s} \tensor K_{r, s} + h \tensor c_{t_2}
                                \end{aligned}
                            $$
                        where the minus sign in the third equation appeared because:
                            $$[ht, D_{r, s}] = -[D_{r, s}, ht] = -h D_{r, s}(t)$$
                            $$[ht, D_t] = -[D_t, ht] = -h D_t(t)$$
                        (cf. remark \ref{remark: derivation_action_on_multiloop_algebras}) and the the last equality is due to the fact that:
                            $$D_{r, s} = -s v^{-r + 1} t^{-s - 1} \del_v + r v^{-r} t^{-s} \del_t$$
                            $$D_t = -\del_t$$
                        (cf. remark \ref{remark: dual_of_toroidal_centres_contains_derivations}). 
                    \end{enumerate}

                    Since we know that:
                        $$[\bar{\Delta}(ht), \sfr_{\g_{[2]}^+}] = -\left( \bar{d}(t_1) \tensor h + h \tensor \bar{d}(t_2) \right) v_2 \1(v_1, v_2) \1^+(t_1, t_2) + [h_1 \tensor 1, \sfr_{\g}] v_2 \1(v_1, v_2)$$
                    we now claim that:
                        $$[\bar{\Delta}(ht), \sfr_{\z_{[2]}^+} + \sfr_{\d_{[2]}^+}] = \left( \bar{d}(t_1) \tensor h + h \tensor \bar{d}(t_2) \right) v_2 \1(v_1, v_2) \1^+(t_1, t_2)$$
                    (since ultimately, we would like to show that $\hat{\delta}^+(ht) = \sfr_{\g} v_2 \1(v_1, v_2)$), and to prove that this is the case, let us first note that we now have that:
                        $$
                            \begin{aligned}
                                & [\bar{\Delta}(ht), \sfr_{\z_{[2]}^+} + \sfr_{\d_{[2]}^+}]
                                \\
                                = & -\sum_{(r, s) \in \Z \x \Z_{> 0}} \left( K_{r, s} \tensor r h v_2^{-r} t_2^{-s} + r h v_1^{-r} t_1^s \tensor K_{r, -s} \right) - \sum_{r \in \Z} r h v_1^{-r} \tensor K_{r, 0} + h \tensor c_{t_2}
                            \end{aligned}
                        $$
                    wherein the first summand corresponds to the indices $(r, 0) \in \Z \x \Z_{\leq 0}$. From remark \ref{remark: centres_of_dual_toroidal_lie_algebras}, we know that:
                        $$
                            K_{r, s} :=
                            \begin{cases}
                                \text{$\frac1s v^{r - 1} t^s \bar{d}(v)$ if $(r, s) \in \Z \x (\Z \setminus \{0\})$}
                                \\
                                \text{$-\frac1r v^r t^{-1} \bar{d}(t)$ if $(r, s) \in (\Z \setminus \{0\}) \x \{0\}$}
                                \\
                                \text{$0$ if $(r, s) = (0, 0)$}
                            \end{cases}
                        $$
                    from which one infers that:
                        $$
                            \begin{aligned}
                                & -\sum_{r \in \Z} r h v_1^{-r} \tensor K_{r, 0}
                                \\
                                = & -\sum_{r \in \Z} r h v_1^{-r} \tensor \left( -\frac1r v_2^r t_2^{-1} \bar{d}(t_2) \right)
                                \\
                                = & \sum_{r \in \Z} h v_1^{-r} \tensor v_2^r t_2^{-1} \bar{d}(t_2)
                                \\
                                = & \sum_{r \in \Z} h v_1^{-r} \tensor v_2^r t_2^{-1} \bar{d}(t_2)
                            \end{aligned}
                        $$
                        
                    Next, recall again from remark \ref{remark: centres_of_dual_toroidal_lie_algebras} that:
                        $$(r, s) \in \Z^2 \implies K_{r, s} = \frac1s v^{r - 1} t^s \bar{d}(v) = -\frac1r v^r t^{s - 1} \bar{d}(t) \in \bar{\Omega}_{[2]}$$
                    and then consider the following:
                        $$
                            \begin{aligned}
                                & -\sum_{(r, s) \in \Z \x \Z_{> 0}} \left( K_{r, s} \tensor r h v_2^{-r} t_2^{-s} + r h v_1^{-r} t_1^s \tensor K_{r, -s} \right)
                                \\
                                = & \sum_{(r, s) \in \Z \x \Z_{> 0}} \left( v_1^r t_1^{s - 1} \bar{d}(t_1) \tensor h v_2^{-r} t_2^{-s} - h v_1^{-r} t_1^s \tensor v_2^r t_2^{-s - 1} \bar{d}(t_2) \right)
                            \end{aligned}
                        $$
                    wherein we note that for all $s \in \Z_{> 0}$, the summands corresponding to the indices $(0, s)$ vanish.

                    We now have that:
                        $$
                            \begin{aligned}
                                & [\bar{\Delta}(ht), \sfr_{\z_{[2]}^+} + \sfr_{\d_{[2]}^+}]
                                \\
                                = & \sum_{(r, s) \in \Z \x \Z_{> 0}} \left( K_{r, s} \tensor r h v_2^{-r} t_2^{-s} + r h v_1^{-r} t_1^s \tensor K_{r, -s} \right) - \sum_{r \in \Z} r h v_1^{-r} \tensor K_{r, 0} + h \tensor c_{t_2}
                                \\
                                = & \sum_{(r, s) \in \Z \x \Z_{> 0}} \left( v_1^r t_1^{s - 1} \bar{d}(t_1) \tensor h v_2^{-r} t_2^{-s} - h v_1^{-r} t_1^s \tensor v_2^r t_2^{-s - 1} \bar{d}(t_2) \right) + \sum_{r \in \Z} h v_1^{-r} \tensor v_2^r t_2^{-1} \bar{d}(t_2) + h \tensor t_2^{-1} \bar{d}(t_2)
                                \\
                                = & \sum_{(r, s) \in \Z \x \Z_{> 0}} \left( v_1^r t_1^{s - 1} \bar{d}(t_1) \tensor h v_2^{-r} t_2^{-s} + h v_1^r t_1^s \tensor v_2^{-r} t_2^{-s - 1} \bar{d}(t_2) \right) + \sum_{r \in \Z} h v_1^{-r} \tensor v_2^r t_2^{-1} \bar{d}(t_2)
                                \\
                                = & \sum_{(r, s) \in \Z \x \Z_{> 0}} \left( v_1^r t_1^{s - 1} \bar{d}(t_1) \tensor h v_2^{-r} t_2^{-s} + h v_1^{-r} t_1^s \tensor v_2^r t_2^{-s - 1} \bar{d}(t_2) \right) + \sum_{r \in \Z} h v_1^r \tensor v_2^{-r} t_2^{-1} \bar{d}(t_2)
                                \\
                                = & ( \bar{d}(t_1) \tensor h ) \sum_{(r, s) \in \Z \x \Z_{> 0}} v_1^r t_1^{s - 1} \tensor v_2^{-r} t_2^{-s} + ( h \tensor \bar{d}(t_2) ) \left( \sum_{(r, s) \in \Z \x \Z_{> 0}} v_1^r t_1^s \tensor v_2^{-r} t_2^{-s - 1} + \sum_{r \in \Z} v_1^r \tensor v_2^{-r} t_2^{-1} \right)
                                \\
                                = & ( \bar{d}(t_1) \tensor h ) \sum_{(r, s) \in \Z \x \Z_{\geq 0}} v_1^r t_1^s \tensor v_2^{-r} t_2^{-s - 1} + ( h \tensor \bar{d}(t_2) ) \left( \sum_{(r, s) \in \Z \x \Z_{> 0}} v_1^r t_1^s \tensor v_2^{-r} t_2^{-s - 1} + \sum_{r \in \Z} v_1^r \tensor v_2^{-r} t_2^{-1} \right)
                                \\
                                = & ( \bar{d}(t_1) \tensor h + h \tensor \bar{d}(t_2) ) \sum_{(r, s) \in \Z \x \Z_{\geq 0}} v_1^r t_1^s \tensor v_2^{-r} t_2^{-s - 1}
                                \\
                                = & ( \bar{d}(t_1) \tensor h + h \tensor \bar{d}(t_2) ) v_2 \1(v_1, v_2) \1^+(t_1, t_2)
                            \end{aligned}
                        $$

                    We can now add the three components together to yield:
                        $$[\bar{\Delta}(ht), \sfr_{\hat{\g}_{[2]}^+}] = [ \bar{\Delta}(ht), \sfr_{\g_{[2]}^+} + (\sfr_{\z_{[2]}^+} + \sfr_{\d_{[2]}^+}) ] =  [h_1 \tensor 1] v_2 \1(v_1, v_2)$$
                    precisely as claimed. 
                    
                    \item Finally, in order to compute $\hat{\delta}^+(t c_v)$, let us simply note that because:
                        $$\deg t = 1, \deg c_v = -1$$
                    in $\z_{[2]}$ (cf. remark \ref{remark: Z_gradings_on_toroidal_lie_algebras}), we have that:
                        $$\deg t c_v = 0$$
                    and hence:
                        $$\hat{\delta}^+(t c_v) = 0$$
                    per remark \ref{remark: total_degrees_of_classical_yangian_R_matrices}.
                \end{enumerate}
            \end{proof}

        One issue that we have neglected to discuss so far is the fact that $\tilde{\delta}^+$ is somehow inherently topological: e.g. we now know that:
            $$\tilde{\delta}^+(ht) = [ht \tensor 1, \sfr_{\g} v_2 \1(v_1, v_2)]$$
        which means that \textit{a priori}, $\tilde{\delta}^+(ht)$ is not an element of $\tilde{\g}_{[2]}^+ \tensor_{\bbC} \tilde{\g}_{[2]}^+$ but rather of some topological completion thereof. To make sense of this, we can rely on the natural $\Z_{\geq 0}$-grading on $\tilde{\g}_{[2]}^+$ on the second variable $t$, which induces a $\Z_{\geq 0}$-grading on the algebraic tensor product $\tilde{\g}_{[2]}^+ \tensor_{\bbC} \tilde{\g}_{[2]}^+$, which can then be topologically completed somehow. 

        The following definition can be viewed as an adaptation of the analysis done in  \cite[Appendix A]{wendlandt_formal_shift_operators_on_yangian_doubles}.
        \begin{definition}[Grading-completions of $\Z_{\geq 0}$-graded vector spaces] \label{def: grading_comnpletions_of_positively_graded_vector_spaces}
            Suppose that $V := \bigoplus_{n \in \Z_{\geq 0}} V_n$ is a $\Z_{\geq 0}$-graded vector space over a general field. Then, the \textbf{grading-completion} of $V$ shall be the following cofiltered limit in $k\-\Vect$:
                $$\breve{V} := \projlim_{m \in \Z_{\geq 0}} V/\bigoplus_{m \geq n} V_n$$
        \end{definition}
        To make sure that this definition is a sensible one, consider the following simple example:
        \begin{example}
            Let $k$ be a field and consider the graded $k$-vector space:
                $$k[t] \cong \bigoplus_{n \in \Z_{\geq 0}} \frac{(t^n)}{(t^{n + 1})}$$
            Its grading-completion is then given by:
                $$\breve{k[t]} \cong \projlim_{m \in \Z_{\geq 0}} k[t]/\bigoplus_{m \geq n} \frac{(t^n)}{(t^{n + 1})} \cong \projlim_{m \in \Z_{\geq 0}} k[t]/t^{m + 1} \cong k[\![t]\!]$$
            which is to be expected.
        \end{example}
        \begin{lemma}[Grading-completions are topologically complete] \label{lemma: grading_completions_are_topologically_complete}
            Suppose that $V := \bigoplus_{n \in \Z_{\geq 0}} V_n$ is a $\Z_{\geq 0}$-graded vector space over a generaly field $k$. Then the grading-completion $\breve{V}$ as in definition \ref{def: grading_comnpletions_of_positively_graded_vector_spaces} is a topologically complete $\Z_{\geq 0}$-graded vector space over $k$.
        \end{lemma}
            \begin{proof}
                
            \end{proof}
        \begin{theorem}[Codomain of $\tilde{\delta}^+$] \label{theorem: toroidal_topological_lie_bialgebras}
            Consider the Lie algebra $\tilde{\g}_{[2]}^+$ with its natural $\Z_{\geq 0}$-grading given by:
                $$\deg v := 0, \deg t := 1$$
            (cf. remark \ref{remark: Z_gradings_on_toroidal_lie_algebras}) and recall from theorem \ref{theorem: toroidal_lie_bialgebras} the Lie cobracket:
                $$\tilde{\delta}^+: \tilde{\g}_{[2]}^+ \to \tilde{\g}_{[2]}^+ \tensor_{\bbC} \tilde{\g}_{[2]}^+$$
            To make this Lie cobracket well-defined, one must enlarge the codomain to the grading-completion $\tilde{\g}_{[2]}^+ \brevetensor_{\bbC} \tilde{\g}_{[2]}^+$, and one thus obtains a non-trivially \textit{continuous} Lie cobracket:
                $$\breve{\delta}^+: \breve{\g}_{[2]}^+ \to \tilde{\g}_{[2]}^+ \brevetensor_{\bbC} \tilde{\g}_{[2]}^+$$
            wherein $\breve{\g}_{[2]}^+$ is the grading-completion of $\tilde{\g}_{[2]}^+$.
        \end{theorem}
    
    \subsection{Hopf coproducts and classical limits of completed affine Yangians}
        \begin{convention}
            In this subsection, we assume that $\g$ is simply laced, excluding the case where $\g$ is of type $\sfA_1$. 
        \end{convention}

        We begin this subsection by reviewing the construction of what we shall call the \say{Hopf coproduct} $\Delta$ on the Yangian $\rmY_{\hbar}(\hat{\g}_{[1]})$, as was done in \cite[Sections 4 and 5]{guay_nakajima_wendlandt_affine_yangian_coproduct}. We will then lift this map to the formal Yangian $\rmY_{\hbar}(\hat{\g}_{[1]})$ to get another \say{Hopf coproduct} $\Delta_{\hbar}$ thereon. The point of doing this is so that ultimately, we would obtain:
            $$\frac{1}{\hbar}(\Delta_{\hbar} - \Delta_{\hbar}^{\cop}) \pmod{\hbar} \equiv \tilde{\delta}^+$$
        and hence be able to realise the topological Lie bialgebra $(\tilde{\g}_{[2]}^+, \tilde{\delta}^+)$ from theorem \ref{theorem: toroidal_lie_bialgebras} as the classical limit of the formal Yangian $\rmY_{\hbar}(\hat{\g}_{[1]})$ in some sense (which, let us caution, is not exactly the same as in \cite{etingof_kazhdan_quantisation_1}).
        
        \begin{lemma}[The category $\calO$ for the affine Yangian $\rmY(\hat{\g}_{[1]})$] \label{lemma: category_O_affine_yangian}
            (Cf. \cite[Theorem 4.9]{guay_nakajima_wendlandt_affine_yangian_coproduct}).
        
            There is a full subcategory of ${}^l\rmY(\hat{\g}_{[1]})\mod$, called the \textbf{category $\calO$}. This category satisfies the following properties:
            \begin{itemize}
                \item Every object $V \in \Ob(\calO)$ is $\hat{\h}_{[1]}$-diagonalisable and with finite-dimensional ($\h$-)weight spaces, and
                \item For every object $V \in \Ob(\calO)$, there exist \textbf{maximal weights} $\lambda_1, ..., \lambda_k \in \hat{\h}_{[1]}^*$ such that, for any $\mu \in \Pi(V)$, one has that:
                    $$\forall 1 \leq i \leq k: \lambda_i - \mu \in \hat{Q}^+$$
            \end{itemize}

            The aforementioned category $\calO$ of $\rmY(\hat{\g}_{[1]})$ is closed under tensor products over $\bbC$, i.e. if $V_1, V_2$ are any two objects of the category $\calO$, then there will be a $\bbC$-algebra homomorphism:
                $$\Delta_{V_1, V_2}: \rmY(\hat{\g}_{[1]}) \to \End_{\bbC}(V_1 \tensor_{\bbC} V_2)$$
            Furthermore, these tensor products are coassociative in the sense that any $\bbC$-vector space isomorphism:
                $$(V_1 \tensor_{\bbC} V_2) \tensor_{\bbC} V_3 \xrightarrow[]{\cong} V_1 \tensor_{\bbC} (V_2 \tensor_{\bbC} V_3)$$
            between objects $V_1, V_2, V_3 \in \Ob(\calO)$ upgrades to an isomorphism of left-$\rmY(\hat{\g}_{[1]})$-modules.

            Explicitly, for each $V_1, V_2 \in \Ob(\calO)$, the map $\Delta_{V_1, V_2}$ is given on the generating set\footnote{Using the Levendorskii presentation for $\rmY(\hat{\g}_{[1]})$, one sees that this generating set suffices.} $\hat{\h}_{[1]} \cup \{T_{i, 1}, E_{i, 0}^{\pm}\}_{i \in \hat{\Gamma}_0}$ by:
                $$\forall h \in \hat{\h}_{[1]}: \Delta_{V_1, V_2}(h) := \bar{\Delta}(h)$$
                $$\forall i \in \hat{\Gamma}_0: \Delta_{V_1, V_2}(E_{i, 0}^{\pm}) := \bar{\Delta}(E_{i, 0}^{\pm})$$
                $$\forall i \in \hat{\Gamma}_0: \Delta_{V_1, V_2}(T_{i, 0}) = \bar{\Delta}(T_{i, 0}) + [H_{i, 0} \tensor 1, \sfr_{ \hat{\g}_{[1]} }^-]$$
            with $\sfr_{ \hat{\g}_{[1]} }^-$ being the Casimir tensor\footnote{This is denoted by $\Omega_+$ in \cite{guay_nakajima_wendlandt_affine_yangian_coproduct} and \cite{guay_nakajima_wendlandt_affine_yangian_vertex_representations_and_PBW}. We opted to designate this the \say{negative} half of the Casimir tensor of $\hat{\g}_{[1]}$ in accordance with the root-degree of the first tensor factor. Also, in \textit{loc. cit.}, the authors considered the Casimir tensor associated to the Kac-Moody pairing on $\hat{\h}_{[1]} \tensor_{\bbC} \hat{\h}_{[1]} \oplus \hat{\n}_{[1]}^- \hattensor_{\bbC} \hat{\n}_{[1]}^+$, but we need only the \say{triangular} component since the Cartan component will be killed by $[H_{i, 0} \tensor 1, -]$ anyway.} associated to the non-degenerate Kac-Moody pairing on $\hat{\n}_{[1]}^- \hattensor_{\bbC} \hat{\n}_{[1]}^+$.
        \end{lemma}
        \begin{remark}
            The category $\calO$ as in lemma \ref{lemma: category_O_affine_yangian} is \textit{not} monoidal, since it lacks a monoidal unit. 
        \end{remark}
        \begin{remark}[Why involve the category $\calO$ ?]
            For a moment, let us pick the root bases $\{ e_{\alpha, k}^{\pm} \}_{(\alpha, k) \in \hat{\Phi}^+ \x \{1, ..., \dim_{\bbC} (\hat{\g}_{[1]})_{\alpha} \}}$ for $\hat{\n}_{[1]}^{\pm}$ in such a way that they are dual to one another with respect to the Kac-Moody pairing on $\hat{\g}_{[1]}$. In terms of these bases, one can write:
                $$\sfr_{\hat{\g}_{[1]}}^- = \sum_{\alpha \in \hat{\Phi}^+} \sum_{k = 1}^{ \dim_{\bbC} (\hat{\g}_{[1]})_{\alpha} } e_{\alpha, k}^- \tensor e_{\alpha, k}^+$$
        
            One notable detail is the fact that the sum\footnote{The completed tensor product $\rmY(\hat{\g}_{[1]}) \hattensor_{\bbC} \rmY(\hat{\g}_{[1]})$ is only to be understood in the vague sense that it denotes some completion of the algebraic tensor product $\rmY(\hat{\g}_{[1]}) \tensor_{\bbC} \rmY(\hat{\g}_{[1]})$ wherein the sum in question converges.}:
                $$\sum_{\alpha \in \hat{\Phi}^+} \sum_{k = 1}^{ \dim_{\bbC} (\hat{\g}_{[1]})_{\alpha} } e_{\alpha, k}^- \tensor e_{\alpha, k}^+ \in \rmY(\hat{\g}_{[1]}) \hattensor_{\bbC} \rmY(\hat{\g}_{[1]})$$
            is infinite \textit{a priori}, since the affine Kac-Moody algebra $\hat{\g}_{[1]}$ has infinitely many positive roots. However, this is precisely why we have restricted our attention down to the category $\calO$: notice that for any $V \in \Ob(\calO)$ and any $\mu \in \Pi(V)$, there exists a natural number $N \in \N$ such that:
                $$\forall \alpha \in \hat{\Phi}^+: r \geq N \implies V_{\mu + r \alpha} \cong 0$$
            From this, one sees that even though it is given by an infinite sum, the operator:
                $$\sum_{\alpha \in \hat{\Phi}^+} \sum_{k = 1}^{ \dim_{\bbC} (\hat{\g}_{[1]})_{\alpha} } e_{\alpha, k}^- \tensor e_{\alpha, k}^+ \in \End_{\bbC}(V_1 \tensor_{\bbC} V_2)$$
            is ultimately locally nilpotent on the vector spaces of the kind $V_1 \tensor_{\bbC} V_2$, wherein $V_1, V_2 \in \Ob(\calO)$; as such, one sees that the infinite sum above actually becomes finite (and hence converges) after evaluation on elements of the $\rmY(\hat{\g}_{[1]})$-modules in the category $\calO$, and the maps $\Delta_{V_1, V_2}$ as in lemma \ref{lemma: category_O_affine_yangian} are therefore well-defined. 
        \end{remark}
        \begin{convention}
            If $L$ is a Kac-Moody algebra of some simply laced untwisted affine type and then we will denote by $\breve{\rmY}(L)$ the grading-completion (cf. lemma \ref{def: grading_comnpletions_of_positively_graded_vector_spaces}) of $\rmY(L)$ with respect to its root grading.
        \end{convention}
        \begin{lemma}[$\rmY(\hat{\g}_{[1]})$-modules are $\breve{\rmY}(\hat{\g}_{[1]})$-modules] \label{lemma: lifting_representations_of_affine_yangians_to_root_grading_completions}
            (Cf. \cite[Proposition 5.14]{guay_nakajima_wendlandt_affine_yangian_coproduct}) Any left-$\rmY(\hat{\g}_{[1]})$-module $V$ in the category $\calO$, given by a $\bbC$-algebra homomorphism:
                $$\rho: \rmY(\hat{\g}_{[1]}) \to \End_{\bbC}(V)$$
            gives rise to a unique left-$\breve{\rmY}(\hat{\g}_{[1]})$-module structure on $V$, which is the same as a $\bbC$-algebra homomorphism:
                $$\breve{\rho}: \breve{\rmY}(\hat{\g}_{[1]}) \to \End_{\bbC}(V)$$
            fitting into the following commutative diagram of $\bbC$-algebras and homomorphisms between them, where the vertical arrow is the canonical one as in \cite[Section 5, Lemma 5.3]{guay_nakajima_wendlandt_affine_yangian_coproduct}:
                $$
                    \begin{tikzcd}
                	{\breve{\rmY}(\hat{\g}_{[1]})} & {\End_{\bbC}(V)} \\
                	{\rmY(\hat{\g}_{[1]})}
                	\arrow[from=2-1, to=1-1]
                	\arrow["{\breve{\rho}}", dashed, from=1-1, to=1-2]
                	\arrow["\rho"', from=2-1, to=1-2]
                    \end{tikzcd}
                $$
        \end{lemma}
        \begin{proposition}[Hopf coproduct on affine Yangians] \label{prop: hopf_coproduct_on_yangians}
            (Cf. \cite[Proposition 5.18]{guay_nakajima_wendlandt_affine_yangian_coproduct}) There exists a $\bbC$-algebra homomorphism:
                $$\Delta: \rmY(\hat{\g}_{[1]}) \to \breve{\rmY}(\hat{\g}_{[1]} \oplus \hat{\g}_{[1]})$$
            satisfying:
                $$\Delta_{V_1, V_2} = (\breve{\rho}_1 \tensor \breve{\rho}_2) \circ \Delta$$
            for any objects $(V_1, \rho_1), (V_2, \rho_2)$ of the category $\calO$ of $\rmY(\hat{\g}_{[1]})$.
        \end{proposition}
        
        \begin{lemma}[The category $\calO_{\hbar}$ of the formal affine Yangian $\rmY_{\hbar}(\hat{\g}_{[1]})$] \label{lemma: category_O_formal_affine_yangian}
            For the formal affine Yangian $\rmY_{\hbar}(\hat{\g}_{[1]})$, one can define a category $\calO_{\hbar}$ in the exact same way\footnote{Ultimately, this is because we have that $[h, E_{i, r}^{\pm}] = \pm \alpha_i(h) E_{i, r}^{\pm} \in \rmY_{\hbar}(L) \setminus \hbar\rmY_{\hbar}(L) \cong \rmY^0(L)$ for all Cartan elements $h \in \hat{\h}_{[1]}$.} as how the category $\calO$ was defined for $\rmY(\hat{\g}_{[1]})$ in lemma \ref{lemma: category_O_affine_yangian}. 

            The category $\calO_{\hbar}$ is closed under $\tensor_{\bbC}$ (cf. lemma \ref{lemma: category_O_affine_yangian}): for every $V_1, V_2 \in \Ob(\calO_{\hbar})$, there is a corresponding $\bbC$-algebra homomorphism:
                $$\Delta_{V_1, V_2, \hbar}: \rmY_{\hbar}(\hat{\g}_{[1]}) \to \End_{\bbC}(V_1 \tensor_{\bbC} V_2)$$
            given by:
                $$\forall h \in \hat{\h}_{[1]}: \Delta_{V_1, V_2, \hbar}(h) := \bar{\Delta}(h)$$
                $$\forall i \in \hat{\Gamma}_0: \Delta_{V_1, V_2, \hbar}(E_{i, 0}^{\pm}) := \bar{\Delta}(E_{i, 0}^{\pm})$$
                $$\forall i \in \hat{\Gamma}_0: \Delta_{V_1, V_2, \hbar}(T_{i, 0}) = \bar{\Delta}(T_{i, 0}) + \hbar [H_{i, 0} \tensor 1, \sfr_{ \hat{\g}_{[1]} }^- ]$$
            Furthermore, the tensor products in $\calO_{\hbar}$ are coassociative in the same sense as in lemma \ref{lemma: category_O_affine_yangian}.
        \end{lemma}
            \begin{proof}
                This is a consequence of lemma \ref{lemma: category_O_affine_yangian} and the fact that we have a graded $\bbC$-algebra isomorphism:
                    $$\rmY_{\hbar}(\hat{\g}_{[1]}) \xrightarrow[]{\cong} \Rees_{\hbar} \rmY(\hat{\g}_{[1]})$$
                (cf. lemma \ref{lemma: formal_yangians_as_rees_algebras}).
            \end{proof}
        \begin{convention}
            If $\fraku$ is a Kac-Moody algebra of some simply laced untwisted affine type and then we will denote by $\breve{\rmY}_{\hbar}(\fraku)$ the completion of $\rmY_{\hbar}(\fraku)$ with respect to its root grading, with \say{completion} being understood to be in the sense of \cite[Appendix A]{wendlandt_formal_shift_operators_on_yangian_doubles}. Note that this root grading is the same as the one on $\rmY(\fraku)$ due to the fact that:
                $$\forall h \in \hat{\h}_{[1]}: [h, E_{i, r}^{\pm}] = \pm \alpha_i(h) E_{i, r}^{\pm} \in \rmY_{\hbar}(\fraku) \setminus \hbar\rmY_{\hbar}(\fraku) \cong \rmY^0(\fraku)$$
            so the construction of $\breve{\rmY}_{\hbar}(\fraku)$ from $\rmY_{\hbar}(\fraku)$ is the same as that of $\breve{\rmY}(\fraku)$ from $\rmY(\fraku)$.
        \end{convention}
        \begin{lemma}[$\rmY_{\hbar}(\hat{\g}_{[1]})$-modules are $\breve{\rmY}_{\hbar}(\hat{\g}_{[1]})$-modules] \label{lemma: lifting_representations_of_formal_affine_yangians_to_root_grading_completions}
            Any left-$\rmY_{\hbar}(\hat{\g}_{[1]})$-module $V$ in the category $\calO$, given by a $\bbC$-algebra homomorphism:
                $$\rho: \rmY_{\hbar}(\hat{\g}_{[1]}) \to \End_{\bbC}(V)$$
            gives rise to a unique left-$\breve{\rmY}_{\hbar}(\hat{\g}_{[1]})$-module structure on $V$, which is the same as a $\bbC$-algebra homomorphism:
                $$\breve{\rho}: \breve{\rmY}_{\hbar}(\hat{\g}_{[1]}) \to \End_{\bbC}(V)$$
            fitting into the following commutative diagram of $\bbC$-algebras and homomorphisms between them, where the vertical arrow is the canonical inclusion (cf. \cite[Section 5, Lemma 5.3]{guay_nakajima_wendlandt_affine_yangian_coproduct}):
                $$
                    \begin{tikzcd}
                	{\breve{\rmY}_{\hbar}(\hat{\g}_{[1]})} & {\End_{\bbC}(V)} \\
                	{\rmY_{\hbar}(\hat{\g}_{[1]})}
                	\arrow[from=2-1, to=1-1]
                	\arrow["{\breve{\rho}}", dashed, from=1-1, to=1-2]
                	\arrow["\rho"', from=2-1, to=1-2]
                    \end{tikzcd}
                $$
        \end{lemma}
            \begin{proof}
                
            \end{proof}
        \begin{theorem}[Hopf coproduct on formal affine Yangians] \label{theorem: hopf_coproduct_on_formal_yangians}
            The $\bbC$-algebra homomorphism $\Delta: \rmY(\hat{\g}_{[1]}) \to \breve{\rmY}(\hat{\g}_{[1]} \oplus \hat{\g}_{[1]})$ from proposition \ref{prop: hopf_coproduct_on_yangians} lifts\footnote{... in the sense that $\Delta_{\hbar} \pmod{(\hbar - \hbar_0)} \equiv \Delta$ for any $\hbar_0 \in \bbC^{\x}$.} to a $\bbC$-algebra homomorphism:
                $$\Delta_{\hbar}: \rmY_{\hbar}(\hat{\g}_{[1]}) \to \breve{\rmY}_{\hbar}(\hat{\g}_{[1]} \oplus \hat{\g}_{[1]})$$
            satisfying:
                $$\Delta_{V_1, V_2, \hbar} = (\breve{\rho}_1 \tensor \breve{\rho}_2) \circ \Delta_{\hbar}$$
            for any $(V_1, \rho_1), (V_2, \rho_2) \in \Ob(\calO_{\hbar})$.
        \end{theorem}
            \begin{proof}
                
            \end{proof}
        
        \begin{theorem}[Toroidal Lie algebras as classical limits of formal affine Yangians] \label{theorem: toroidal_lie_algebras_as_classical_limits_of_formal_affine_yangians}
           The topological Lie bialgebra $(\tilde{\g}_{[2]}^+, \tilde{\delta}^+)$ from theorem \ref{theorem: toroidal_topological_lie_bialgebras} (see also theorem \ref{theorem: toroidal_lie_bialgebras}) is the classical limit of the formal affine Yangian $\rmY_{\hbar}(\hat{\g}_{[1]})$ with the \say{coproduct} $\Delta_{\hbar}$ (as in theorem \ref{theorem: hopf_coproduct_on_formal_yangians}), in the sense that:
                $$\frac{1}{\hbar}( \Delta_{\hbar} - \Delta_{\hbar}^{\cop} ) \equiv \tilde{\delta}^+ \pmod{\hbar}$$
        \end{theorem}
            \begin{proof}
                Before we begin computing, let us make the preliminary observation that $T_{i, 1}(\hbar) \in \rmY_{\hbar}(\hat{\g}_{[1]})$ is a lift modulo $\hbar$ of $H_{i, 1} \in \t^+ \cong \tilde{\g}_{[2]}^+$ (cf. corollary \ref{coro: chevalley_serre_presentation_for_central_extensions_of_multiloop_algebras}):
                    $$T_{i, 1}(\hbar) := H_{i, 1} - \frac12 \hbar H_{i, 0}^2 \equiv H_{i, 1} \pmod{\hbar}$$
                Also, let us note that, we know from lemma \ref{lemma: levendorskii_presentation} and corollary \ref{coro: levendorskii_presentation__for_central_extensions_of_multiloop_algebras} that it is enough to only check the value of $\Delta_{\hbar}^{\cop}$ on the generators $H_{i, 0}, E_{i, 0}^{\pm}$, and $T_{i, 1} \equiv H_{i, 1} \pmod{\hbar}$, for all $i \in \hat{\Gamma}_0$.
            
                Firstly, from theorem \ref{theorem: hopf_coproduct_on_formal_yangians}, we know that:
                    $$\forall h \in \hat{\h}_{[1]}: \Delta_{\hbar}(h) := \bar{\Delta}(h)$$
                    $$\forall i \in \hat{\Gamma}_0: \Delta_{\hbar}(E_{i, 0}^{\pm}) := \bar{\Delta}(E_{i, 0}^{\pm})$$
                    $$\forall i \in \hat{\Gamma}_0: \Delta_{\hbar}(T_{i, 0}) = \bar{\Delta}(T_{i, 0}) + [H_{i, 0} \tensor 1, \sfr_{ \hat{\g}_{[1]} }^-]$$
                This tells us that:
                    $$\forall h \in \hat{\h}_{[1]}: \Delta_{\hbar}^{\cop}(h) := \bar{\Delta}(h)$$
                    $$\forall i \in \hat{\Gamma}_0: \Delta_{\hbar}^{\cop}(E_{i, 0}^{\pm}) := \bar{\Delta}(E_{i, 0}^{\pm})$$
                    $$\forall i \in \hat{\Gamma}_0: \Delta_{\hbar}^{\cop}(T_{i, 0}) = \bar{\Delta}(T_{i, 0}) + [1 \tensor H_{i, 0}, \sfr_{ \hat{\g}_{[1]} }^+]$$
                with $\sfr_{\hat{\g}_{[1]}}^-$ being the Casimir tensor associated to the non-degenerate Kac-Moody pairing on $\hat{\n}_{[1]}^- \hattensor_{\bbC} \hat{\n}_{[1]}^+$.

                It is then trivial that:
                    $$\frac{1}{\hbar}( \Delta_{\hbar} - \Delta_{\hbar}^{\cop} )(X) = 0$$
                for:
                    $$X \in \hat{\h}_{[1]} \cup \{E_{i, 0}^{\pm}\}_{i \in \hat{\Gamma}_0}$$
                which implies that:
                    $$\frac{1}{\hbar}( \Delta_{\hbar} - \Delta_{\hbar}^{\cop} )(X) \equiv \tilde{\delta}^+(X) \pmod{\hbar}$$
                which is because it is known from theorem \ref{theorem: toroidal_lie_bialgebras} that:
                    $$\tilde{\delta}^+(X) = 0$$
                whenever $\deg X = 0$, which is the case here.
                
                Now, let us verify that:
                    $$\frac{1}{\hbar}(\Delta_{\hbar} - \Delta_{\hbar}^{\cop})(T_{i, 1}) \equiv \tilde{\delta}^+(H_{i, 1})$$
                It is not hard to see that\footnote{One can prove this by e.g. picking the root bases foor $\hat{\n}_{[1]}^{\pm}$.}:
                    $$[1 \tensor H_{i, 0}, \sfr_{ \hat{\g}_{[1]} }^+] = -[H_{i, 0} \tensor 1, \sfr_{ \hat{\g}_{[1]} }^+]$$
                which tells us that:
                    $$\frac{1}{\hbar}( \Delta_{\hbar} - \Delta_{\hbar}^{\cop} )(T_{i, 1}) = [H_{i, 0} \tensor 1, \sfr_{ \hat{\g}_{[1]} }^- + \sfr_{ \hat{\g}_{[1]} }^+]$$
                Since $H_{i, 0}$ commutes with every element of $\hat{\h}_{[1]}$, we can equivalently rewrite the above into:
                    $$\frac{1}{\hbar}( \Delta_{\hbar} - \Delta_{\hbar}^{\cop} )(T_{i, 1}) = [H_{i, 0} \tensor 1, \sfr_{\hat{\h}_{[1]}} + \sfr_{ \hat{\g}_{[1]} }^- + \sfr_{ \hat{\g}_{[1]} }^+] = [H_{i, 0} \tensor 1, \sfr_{ \hat{\g}_{[1]} }]$$
                wherein $\sfr_{\hat{\h}_{[1]}}$ is the Casimir element associated to the Kac-Moody pairing on $\hat{\h}_{[1]} \tensor_{\bbC} \hat{\h}_{[1]}$. 

                We know from theorem \ref{theorem: toroidal_lie_bialgebras} that:
                    $$\tilde{\delta}^+(H_{i, 1}) = [ H_{i, 0} \tensor 1, \sfr_{\g} v_2 \1(v_1, v_2) ]$$
                so we will be done if we can show that:
                    $$[H_{i, 0} \tensor 1, \sfr_{ \hat{\g}_{[1]} }] = [ H_{i, 0} \tensor 1, \sfr_{\g} v_2 \1(v_1, v_2) ]$$
                From the fact that:
                    $$\hat{\g}_{[1]} \cong \g_{[1]} \oplus \z_{[1]} \oplus \d_{[1]} \cong \g_{[1]} \oplus \bbC c_v \oplus \bbC D_{0, -1}$$
                (with notations as in conventions \ref{conv: a_fixed_untwisted_affine_kac_moody_algebra} and \ref{conv: orthogonal_complement_of_toroidal_centres}) and implicitly from the solution to question \ref{question: multiloop_lie_bialgebras} (notice that the bilinear form on $\g_{[2]}$ in \textit{loc. cit.} is nothing but the Kac-Moody form; cf. \cite[Chapter 7]{kac_infinite_dimensional_lie_algebras}), which is given by:
                    $$\forall x, y \in \g: \forall m, n \in \Z: (x v^m, y v^n)_{\hat{\g}_{[1]}} = (x, y)_{\g} \delta_{m + n, 0}$$
                    $$(K_{0, -1}, D_{0, -1})_{\hat{\g}_{[1]}} = 1$$
                we infer that:
                    $$\sfr_{ \hat{\g}_{[1]} } = \sfr_{\g} v_2 \1(v_1, v_2) + \sfr_{\z_{[1]}} + \sfr_{\d_{[1]}} = \sfr_{\g} v_2 \1(v_1, v_2) + K_{0, -1} \tensor D_{0, -1} + D_{0, -1} \tensor K_{0, -1}$$
                wherein $\sfr_{\z_{[1]}}, \sfr_{\d_{[1]}}$ respectively denote the Casimir elements corresponding to the Kac-Moody form on $\z_{[1]} \tensor_{\bbC} \d_{[1]}$ and on $\d_{[1]} \tensor_{\bbC} \z_{[1]}$ respectively. The element $K_{0, -1} \in \breve{\g}_{[1]} := \g_{[1]} \oplus \z_{[1]}$ is central and therefore commutes with $H_{i, 0}$, which implies that:
                    $$[H_{i, 0} \tensor 1, \sfr_{\z_{[1]}}] = [H_{i, 0} \tensor 1, K_{0, -1} \tensor D_{0, -1}] = 0$$
                At the same time, we also know that $D_{0, -1}$ acts as $\id_{\g} \tensor \left(-v \frac{d}{dv}\right)$ on $\g_{[1]}$ and hence as zero on the elements of $\g$ (i.e. degree-$0$ elements of $\g_{[1]}$), and so:
                    $$[H_{i, 0} \tensor 1, \sfr_{\d_{[1]}}] = [H_{i, 0} \tensor 1, D_{0, -1} \tensor K_{0, -1}] = 0$$
                as well. As such, we have demonstrated that:
                    $$[H_{i, 0} \tensor 1, \sfr_{ \hat{\g}_{[1]} }] = [ H_{i, 0} \tensor 1, \sfr_{\g} v_2 \1(v_1, v_2) ]$$
                as we sought to. As mentioned above, this allows us to conclude that:
                    $$\frac{1}{\hbar}( \Delta_{\hbar} - \Delta_{\hbar}^{\cop} )(T_{i, 1}) \equiv \tilde{\delta}^+(H_{i, 1}) \pmod{\hbar}$$
            \end{proof}

    \section{R-matrices of toroidal Lie bialgebras and of affine Yangians}
    \subsection{Classical R-matrices of toroidal Lie bialgebras}
        \begin{convention}
            We assume familiarity with the fact that, for any homomorphism of commutative rings $R \to S$, one has that:
                $$\Hom_S(\Omega_{S/R}^1, N) \cong \Der_R(S, N)$$
            for all $S$-modules $N$ (cf. \cite[\href{https://stacks.math.columbia.edu/tag/00RO}{Tag 00RO}]{stacks}).
        \end{convention}
        
        \begin{remark}[Pairing of $1$-forms and vector fields] \label{remark: pairing_1_forms_and_vector_fields} 
            Suppose for a moment that $A$ is a commutative algebra over a field $k$ generated by some set:
                $$\{v_i\}_{i \in I}$$
            and suppose furthermore that $A$ is equipped with a non-degenerate and symmetric $k$-bilinear form:
                $$(-, -)_A: A \x A \to k$$
            Next, consider the following natural pairing between $\Omega^1_{A/k}$ and $\der_k(A)$, given as the interior product/contraction of differential forms by vector fields, i.e. in the following manner:
                $$\iota_A(f dv_i, g \del_{v_j}) := (f, g)_A \delta_{i, j}$$
            for all $f, g \in A$. Clearly, the pairing:
                $$\iota_A: \Omega^1_{A/k} \x \der_k(A) \to k$$
            is a non-degenerate and symmetric $k$-bilinear form.

            Now, in the particular case of $A := A_{[n]}$ (for some $n \geq 1$), note firstly that the algebra $A_{[n]}$ is naturally equipped with the non-degenerate and symmetric $k$-bilinear form:
                $$(v_1^{m_1} ... v_n^{m_n}, v_1^{m_1'} ... v_n^{m_n'})_{A_{[n]}} := \delta_{(m_1 + m_1', ..., m_{n - 1} + m_{n - 1}', m_n + m_n'), (0, ..., 0, -1)}$$
            (cf. question \ref{question: multiloop_lie_bialgebras}) and with respect to this, the pairing:
                $$\iota_{A_{[n]}}: \Omega_{[n]} \x \der_k(A_{[n]}) \to k$$
            is then given by:
                $$\iota_{A_{[n]}}(f dv_i, g \del_{v_j}) := (f, g)_{A_{[n]}} \delta_{i, j}$$
            for all $f, g \in A_{[n]}$ and all $1 \leq i, j \leq n$. 
        \end{remark}

        \begin{remark}[Pairing of cyclic $1$-forms and divergence-zero vector fields] \label{remark: pairing_cyclic_1_forms_and_div_zero_vector_fields} 
            Again, suppose for a moment that $A$ is an arbitrary \textit{smooth}\footnote{... so that $\Omega^1_{A/k}$ would be finite free as an $A$-module.} commutative algebra over a field $k$ of characteristic $0$ and that $A$ is generated by some set:
                $$\{v_i\}_{i \in I}$$
            and that $A$ carries a non-degenerate and symmetric $k$-bilinear form $(-, -)_A$. Via the canonical quotient map of $k$-vector spaces:
                $$\Omega^1_{A/k} \to \bar{\Omega}^1_{A/k}$$
            one obtains an induced bilinear form that we will denote by:
                $$\bar{\iota}_A: \bar{\Omega}^1_{A/k} \x \der_k(A) \to k$$

            When $A = A_{[n]}$, recall from remark \ref{remark: centres_of_dual_toroidal_lie_algebras} that there is a canonical basis for $\bar{\Omega}^1_{A/k}$ consisting of the elements:
                $$m_i^{-1} v_1^{m_1} ... v_i^{m_i - 1} ... v_n^{m_n} \bar{d}(v_i)$$
            wherein $1 \leq i \leq n$. The bilinear form $\bar{\iota}_{[n]} := \bar{\iota}_{A_{[n]}}$ is then given by:
                $$\bar{\iota}_{[n]}( m_i^{-1} v_1^{m_1} ... v_i^{m_i - 1} ... v_n^{m_n} \bar{d}(v_i), v_1^{m_1'} ... v_n^{m_n'} \del_{v_j} ) = m_i^{-1} \delta_{(m_1 + m_1', ..., m_i + m_i' - 1, ..., m_n + m_n'), (0, ..., 0, -1)} \delta_{i, j}$$
            for all $1 \leq i, j \leq n$. While it is true that $\bar{\iota}_{[n]}$ is non-degenerate, we caution that - based on the calculation above - $\bar{\iota}_{[n]}$ is not symmetric due to the appearance of the $m_i^{-1}$ factor: switching the two inputs would yield a $(-m'_i + 1)^{-1}$ factor instead. The upshot is that, once we restrict the second input from $\der_k(A_{[n]})$ to the \say{Yangian divergence-zero} subspace $\d_{[n]}$, the bilinear form $\bar{\iota}_{[n]}$ ought to get symmetrised, and symmetry is needed for us to be able to write down a canonical element corresponding to the resulting symmetric pairing.

            Let us now focus on the case where $k = \bbC$ and $n = 2$, partly because this is the only case wherein we have a concrete basis for $\d_{[2]}$ to work with, but also because ultimately, this is the only case of particular interest to us. Recall from remark \ref{remark: dual_of_toroidal_centres_contains_derivations} and from the construction of $\d_{[2]}$ (cf. convention \ref{conv: orthogonal_complement_of_toroidal_centres}) that:
                $$\d_{[2]} := (\bigoplus_{(r, s) \in \Z^2} \bbC D_{r, s}) \oplus \bbC D_v \oplus \bbC D_t$$
            wherein:
                $$\forall (r, s) \in \Z^2: D_{r, s} = s v^{-r + 1} t^{-s - 1} \del_v - r v^{-r} t^{-s} \del_t$$
                $$D_v = -v t^{-1} \del_v, D_t = -\del_t$$
            Let us also recall from remark \ref{remark: centres_of_dual_toroidal_lie_algebras} that the canonical basis elements of $\bar{\Omega}_{[2]}$ can be alternatively written as:
                $$
                    K_{r, s} :=
                    \begin{cases}
                        \text{$\frac1s v^{r - 1} t^s \bar{d}(v)$ if $(r, s) \in \Z \x (\Z \setminus \{0\})$}
                        \\
                        \text{$-\frac1r v^r t^{-1} \bar{d}(t)$ if $(r, s) \in (\Z \setminus \{0\}) \x \{0\}$}
                        \\
                        \text{$0$ if $(r, s) = (0, 0)$}
                    \end{cases}
                $$
                $$c_v := v^{-1} \bar{d}(v), c_t := t^{-1} \bar{d}(t)$$
            It is clear that:
                $$
                    \bar{\iota}_{[2]}( K_{r, s}, D ) =
                    \begin{cases}
                        \text{$1$ if $D = D_{r, s}$}
                        \\
                        \text{$0$ if $D \not \in \bbC D_{r, s}$}
                    \end{cases}
                $$
                $$
                    \bar{\iota}_{[2]}( c_v, D ) =
                    \begin{cases}
                        \text{$1$ if $D = D_v$}
                        \\
                        \text{$0$ if $D \not \in \bbC D_v$}
                    \end{cases}
                $$
                $$
                    \bar{\iota}_{[2]}( c_t, D ) =
                    \begin{cases}
                        \text{$1$ if $D = D_t$}
                        \\
                        \text{$0$ if $D \not \in \bbC D_t$}
                    \end{cases}
                $$
            so we have managed to show that not only is there a non-degenerate and symmetric $\bbC$-bilinear form:
                $$\bar{\iota}_{[2]}: \bar{\Omega}_{[2]} \x \d_{[2]} \to \bbC$$
            induced naturally by the canonical pairing of differential $1$-forms and derivations on $A_{[2]}$, but also that this induced bilinear form coincides with the restriction of the bilinear form $(-, -)_{\hat{\g}_{[2]}}$ (as constructed in convention \ref{conv: orthogonal_complement_of_toroidal_centres}) to $\z_{[2]} \oplus \d_{[2]} \cong \bar{\Omega}_{[2]} \oplus \d_{[2]}$, since we have per its construction that the elements $D_{r, s}, D_v, D_t$ are dual with respect to $(-, -)_{\hat{\g}_{[2]}}$ to the elements $K_{r, s}, c_v, c_t$, respectively. 
        \end{remark}

        We can package the previous two remarks into the following result:
        \begin{proposition}[Pairing of cyclic $1$-forms and divergence-zero vector fields] \label{prop: pairing_cyclic_1_forms_and_div_zero_vector_fields} 
            Denote the usual non-degenerate and symmetric bilinear pairing of differential $1$-forms and derivations on $A_{[2]}$ by:
                $$\iota_{[2]}: \Omega_{[2]} \x \der_{\bbC}(A_{[2]}) \to \bbC$$
            Via the canonical quotient map of $\bbC$-vector spaces:
                $$\Omega_{[2]} \to \bar{\Omega}_{[2]}$$
            one obtains an induced non-degenerate and symmetric bilinear pairing:
                $$\bar{\iota}_{[2]}: \bar{\Omega}_{[2]} \x \d_{[2]} \to \bbC$$
            coninciding with the restriction of the bilinear form $(-, -)_{\hat{\g}_{[2]}}$ (as constructed in convention \ref{conv: orthogonal_complement_of_toroidal_centres}) to $\z_{[2]} \oplus \d_{[2]} \cong \bar{\Omega}_{[2]} \oplus \d_{[2]}$.
            
            In other words, there is a commutative diagram of $\bbC$-vector spaces and $\bbC$-linear maps between them as follows:
                $$
                    \begin{tikzcd}
                	{\Omega_{[2]} \tensor_{\bbC} \der_{\bbC}(A_{[2]})} \\
                	& \bbC \\
                	{\bar{\Omega}_{[2]} \tensor_{\bbC} \d_{[2]}}
                	\arrow[dashed, from=1-1, to=3-1]
                	\arrow["{\iota_{[2]}}", from=1-1, to=2-2]
                	\arrow["{(-, -)_{\hat{\g}_{[2]}}}"', from=3-1, to=2-2]
                    \end{tikzcd}
                $$
        \end{proposition}
        \begin{lemma}[Canonical element for the canonical pairing of $1$-forms and vector fields] \label{lemma: canonical_elements_for_the_canonical_pairing_of_1_forms_and_vector_fields}
            Let $k$ be a field of characteristic $0$. Denote by:
                $$\iota_{[n]}^+: \Omega_{[n]}^+ \x \der_k(A_{[n]}^-) \to k$$
            the restriction of $\iota_{[n]}$ to be between differential $1$-forms and derivations on $A_{[n]}^+ \subset A_{[n]}$ and $A_{[n]}^- \subset A_{[n]}$ respectively. The canonical element attached to $\iota_{[n]}^+$ shall then be given by:
                $$\sfr_{[n]}^+ := v_1' \1(v_1, v_1') ... v_{n - 1}' \1(v_{n - 1}, v_{n - 1}') \1^+(v_n, v_n') \sum_{1 \leq i \leq n} dv_i \tensor \del_{v_i}$$
        \end{lemma}
            \begin{proof}
                A basis for $\Omega_{[n]}^+$ is given by:
                    $$\{ v_1^{m_1} ... v_{n - 1}^{m_{n - 1}} v_n^{m_n} dv_i \}_{1 \leq i \leq n}$$
                and its dual (which is a basis for $\der_k(A_{[n]}^-)$) with respect to $\iota_{[n]}^+$ is given by:
                    $$\{ v_1^{-m_1} ... v_{n - 1}^{m_{n - 1}} v_n^{-m_n - 1} \del_{v_i} \}_{1 \leq i \leq n}$$
                We then have - more-or-less tautologically - that:
                    $$
                        \begin{aligned}
                            \sfr_{[n]}^+ & := \sum_{1 \leq i \leq n} \sum_{(m_1, ..., m_{n - 1}, m_n) \in \Z^{n - 1} \x \Z_{\geq 0}} v_1^{m_1} ... v_{n - 1}^{m_{n - 1}} v_n^{m_n} dv_i \tensor v_1^{-m_1} ... v_{n - 1}^{m_{n - 1}} v_n^{-m_n - 1} \del_{v_i}
                            \\
                            & = v_1' \1(v_1, v_1') ... v_{n - 1}' \1(v_{n - 1}, v_{n - 1}') \1^+(v_n, v_n') \sum_{1 \leq i \leq n} dv_i \tensor \del_{v_i}
                        \end{aligned}
                    $$
            \end{proof}
        The lemma above suggests to us that it is possible to rewrite the canonical elements $\sfr_{\z_{[2]}^+}, \sfr_{\d_{[2]}^+}$ in terms of formal Dirac distributions. Furthermore, it implies that the problem now reduces to computing the complements of the bases of $\bar{\Omega}_{[2]}^+$ (i.e. a basis for $d(A_{[2]}^+)$) and of $\d_{[2]}^+$ inside the bases of $\Omega_{[2]}^+$ and of $\der_{\bbC}(A_{[2]}^+)$ respectively; due to the duality of $\Omega_{[2]}^+$ and $\der_{\bbC}(A_{[2]})$ via the bilinear form $\iota_{[2]}$, these complemenets ought to be in bijection with one another. This is because, the canonical element arising from those complementary basis elements, say $\sfr_{[2]}^{+ \varnothing}$, satisfies:
            $$\sfr_{\z_{[2]}^+} = \sfr_{[2]}^+ - \sfr_{[2]}^{+ \varnothing}$$
        (and likewise for $\sfr_{\d_{[2]}^+}$ after a flip map). 
        \begin{remark}[A basis for $A_{[n]}$] \label{remark: basis_for_global_functions_on_split_tori}
            Let $k$ be a field of characteristic $0$ and fix some $n \geq 1$. Set:
                $$v_0 := 1$$
            so that:
                $$A_{[0]} \cong k$$

            The $k$-algebra:
                $$A_{[n]} := k[v_1^{\pm 1}, ..., v_n^{\pm 1}]$$
            has a natural $\Z$-grading in the last variable $v_n$ given by:
                $$A_{[n]} \cong \bigoplus_{m_n \in \Z} A_{[n - 1]} v_n^{m_n}$$
            Applying the universal K\"ahler differential map $d: A_{[n]} \to \Omega_{[n]}$ then yields:
                $$d( A_{[n]} ) \cong \bigoplus_{m_n \in \Z} d( A_{[n - 1]} v_n^{m_n} ) \cong \bigoplus_{p \in \Z} ( d( A_{[n - 1]} ) v_n^{m_n} \oplus A_{[n - 1]} v_n^{m_n - 1} dv_n )$$

            When $n = 1$, the equation above specialises to:
                $$d(A_{[1]}) \cong \bigoplus_{m \in \Z} ( d( A_{[0]} ) v^m \oplus A_{[0]} v^{m - 1} dv ) \cong \bigoplus_{m \in \Z} k v^{m - 1} dv$$
            When $n = 2$, we subsequently have that:
                $$d(A_{[2]}) \cong \bigoplus_{p \in \Z} ( d(A_{[1]}) t^p \oplus A_{[1]} t^{p - 1} dt ) \cong \bigoplus_{(m, p) \in \Z^2} ( k v^{m - 1} t^p dv \oplus k v^m t^{p - 1} dt )$$
            from which it is inferable that the set:
                $$\{v^{m - 1} t^p dv\}_{(m, p) \in \Z^2} \cup \{v^m t^{p - 1} dt\}_{(m, p) \in \Z^2}$$
            forms a basis for $d(A_{[2]})$; to obtain a basis for $d(A_{[2]}^{\pm})$, simply change the indexing set from $\Z^2$ to $\Z \x \Z_{\geq 0}$ and $\Z \x \Z_{< 0}$ respectively.

            The subset of the standard basis for $\der_k(A_{[2]})$ whose elements are dual with respect to the bilinear form $\iota_{[2]}$ to the elements of the basis for $d(A_{[2]})$ (i.e. a basis for $\der_k(A_{[2]})/\d_{[2]}$) as above is thus:
                $$\{v^{-m + 1} t^{-p - 1} \del_v\}_{(m, p) \in \Z^2} \cup \{v^{-m} t^{-p} \del_t\}_{(m, p) \in \Z^2}$$
            and likewise, to obtain the corresponding subsets of the standard basis for $\der_k(A_{[2]}^{\pm})$, simply change the indexing set from $\Z^2$ to $\Z \x \Z_{\geq 0}$ and $\Z \x \Z_{< 0}$ respectively.

            Given the above, the canonical element for the restricted pairing:
                $$\iota_{[2]}^+: d(A_{[2]}^+) \x \der_k( A_{[2]}^- )/\d_{[2]}^- \to k$$
            (which is still non-degenerate and symmetric) then takes the form:
                $$
                    \begin{aligned}
                        \sfr_{[2]}^{+ \varnothing} & = \sum_{(m, p) \in \Z \x \Z_{\geq 0}} \left( v_1^{m - 1} t_1^p dv_1 \tensor v_2^{-m + 1} t_2^{-p - 1} \del_{v_2} + v_1^m t_1^{p - 1} dt_1 \tensor v_2^{-m} t_2^{-p} \del_{t_2} \right)
                        \\
                        & = v_1^{-1} v_2^2 \1(v_1, v_2) \1^+(t_1, t_2) dv_1 \tensor \del_{v_2} + v_2 \1(v_1, v_2) t_1^{-1} t_2 \1^+(t_1, t_2) dt_1 \tensor \del_{t_2}
                    \end{aligned}
                $$
            By applying the flip map, one gets $\sfr_{[2]}^{- \varnothing}$, the canonical element for the restricted pairing:
                $$\iota_{[2]}^-: \der_k( A_{[2]}^- )/\d_{[2]}^- \x d(A_{[2]}^+) \to k$$
        \end{remark}
        We have therefore arrived at the following result:
        \begin{proposition}[Toroidal classical R-matrices in terms of formal distributions] \label{prop: toroidal_classical_R_matrices_in_terms_of_formal_distributions}
            The canonical elements $\sfr_{\z_{[2]}^+}$ and $\sfr_{\d_{[2]}^+}$ (cf. corollary \ref{coro: extended_toroidal_lie_bialgebras}) can be alternatively written in the following manner, using formal Dirac distributions (cf. convention \ref{conv: formal_dirac_distributions})\footnote{Note that the total degrees of $\sfr_{\z_{[2]}^+}$ and $\sfr_{\d_{[2]}^+}$ are both $0$ in $v$ and $-1$ in $t$ (ultimately because $\deg \1(w, z) = \deg \1^+(z, w) = -1$ by construction), agreeing with remark \ref{remark: total_degrees_of_classical_yangian_R_matrices}.}:
                $$\sfr_{\z_{[2]}^+} = \sfr_{[2]}^+ - \sfr_{[2]}^{+ \varnothing}$$
                $$\sfr_{\d_{[2]}^+} = \sfr_{[2]}^- - \sfr_{[2]}^{- \varnothing}$$
            with $\sfr_{[2]}^{\pm}$ and $\sfr_{[2]}^{\pm \varnothing}$ as in lemma \ref{lemma: canonical_elements_for_the_canonical_pairing_of_1_forms_and_vector_fields} and remark \ref{remark: basis_for_global_functions_on_split_tori}. 
        \end{proposition}
        
        \begin{remark}[Toroidal classical R-matrices as meromorphic functions] \label{remark: toroidal_classical_R_matrices_as_meromorphic_functions}
            The appearance of formal Dirac distributions in the expressions for $\sfr_{\z_{[2]}^+}$ and $\sfr_{\d_{[2]}^+}$ (as well as for $\sfr_{\g_{[2]}^+}$; cf. question \ref{question: multiloop_lie_bialgebras}), and hence for $\sfr_{\tilde{\g}_{[2]}^+} = \sfr_{\hat{\g}_{[2]}^+}$ suggests to us that when regarded complex-analytically, these are meromorphic $\bbC$-valued functions in the variables $v, t$. It is thus natural to inquire into where the singularities of these functions lie. 
        \end{remark}
        \begin{question}
            Classify the singularities of $\sfr_{\tilde{\g}_{[2]}^+} = \sfr_{\hat{\g}_{[2]}^+}$ and compare them to those of the universal quantum R-matrix of $\rmY_{\hbar}(\hat{\g}_{[1]})$. 
        \end{question}

    \subsection{\textit{Interlude}: A root space decomposition for Yangian extended toroidal Lie algebras}
        As is now standard practice in infinite-dimensional Lie theory, infinite-dimensional Lie algebra induced from finite-dimensional simple Lie algebras ought to carry a grading by some kind of induced \say{higher root lattice} (e.g. affine Kac-Moody algebras are graded by the affinisations of the root lattices of the underlying finite-dimensional simple Lie algebras; cf. \cite[Chapter 6]{kac_infinite_dimensional_lie_algebras}). There are many reasons as to why one might seek to endow Lie algebras with such gradings, but one rather important reason is that without a root grading of some sort - which in turn would give rise to some kind of triangular decomposition - one would have no hope of setting up a theory of highest-weight modules which, from practical experiences with cases such as $\g$ and $\hat{\g}_{[1]}$, we know to be an extremely powerful method for attacking the problem of classifying say, simple modules over Lie algebras. Therfore, it is natural to ask the question of whether or not our Yangian extended toroidal Lie algebra $\hat{\g}_{[2]}$ can also be endowed with such an induced grading, primarily because $\hat{\g}_{[2]}$ carries a non-degenerate invariant symmetric bilinear form.

        \newpage
        
        \begin{convention}
            If $\fraku$ is a symmetrisable Kac-Moody algebra and $V$ is a $\fraku$-module, then for each weight $\lambda \in \Pi(V)$, we shall be denoting the corresponding weight space by $V[\lambda]$.
        \end{convention}

        Let us firstly recall two equivalent natural gradings on the affine Kac-Moody algebra $\hat{\g}_{[1]}$. 
        \begin{remark}[$\hat{Q}$-grading on $\hat{\g}_{[1]}$]
            The $Q$-grading on $\g$ and the natural $\Z$-grading on $A_{[1]} := \bbC[v^{\pm 1}]$ induce, together, a $Q \x \Z$-grading on $\g_{[1]}$. Explicitly, for each $\lambda \in \Phi$, each $x \in \g[\alpha]$, and each $m \in \Z$, one has that:
                $$\deg x v^m = (\alpha, m)$$
            Following \cite[Chapter 6]{kac_infinite_dimensional_lie_algebras}, we know that there is an isomorphism of $\Z$-modules:
                $$\hat{Q} \xrightarrow[]{\cong} Q \x \Z$$
                $$\alpha + m \delta \mapsto (\alpha, m)$$
            (given for all $\alpha \in \Phi$ and all $m \in \Z$), with $\delta$ denoting the lowest positive imaginary root. As such, $\g_{[1]}$ can be equivalently viewed as being $\hat{Q}$-graded in the sense that for each $\alpha \in Q$, each $x \in \g[\alpha]$, and each $m \in \Z$, one has that:
                $$\deg x v^m = \alpha + m \delta$$
        \end{remark}
        \begin{proposition}[Induced $Q \x \Z$-grading on $\hat{\g}_{[2]}$] \label{prop: root_grading_on_extended_toroidal_lie_algebras}
            For what follows, we will need to make the choice\footnote{See proposition \ref{prop: lie_bracket_on_orthogonal_complement_of_toroidal_centre} for an elaboration.} that the restriction of $[-, -]_{\hat{\g}_{[2]}}$ to the vector subspace $\d_{[2]}$ is just the ordinary commutator of derivations.
        
            Define the following grading on $\tilde{\g}_{[2]}$\footnote{Note that we can not simply define a grading on $\g_{[2]}$ alone, since $[\g_{[2]}, \g_{[2]}]_{\tilde{\g}_{[2]}} \not \subset \g_{[2]}$.}, naturally induced by the natural $Q \x \Z$-grading on $\g_{[1]}$. Firstly, let us declare that:
                $$\forall \alpha \in \Phi: \forall x \in \g[\alpha]: \deg x := (\alpha, 0)$$
                $$\deg v := (0, 1)$$
                $$\deg t := (0, 0)$$
            If we are to extend the $Q \x \Z$-grading on $\tilde{\g}_{[2]}$ as above to $\hat{\g}_{[2]}$ then the Lie bracket $[-, -]_{\hat{\g}_{[2]}}$ ought to be $Q \x \Z$-graded in a compatible manner. Given the adjoint actions of the derivations $D_{r, s}, D_v, D_t$ on the monomials $x v^m t^p \in \g_{[2]}$ (in particular, how said actions affect the $Q \x \Z$-degrees of said monomials; cf. remarks \ref{remark: derivation_action_on_multiloop_algebras} and \ref{remark: dual_of_toroidal_centres_contains_derivations}), let us declare that:
                $$\forall (r, s) \in \Z^2: \deg D_{r, s} := (0, -r)$$
                $$\deg D_v = \deg D_t := (0, 0)$$
            We would also like the bilinear form $(-, -)_{\hat{\g}_{[2]}}$ to be of total degree $(0, 0)$, which forces:
                $$\forall (r, s) \in \Z^2: \deg K_{r, s} := (0, r)$$
                $$\deg c_v = \deg c_t := (0, 0)$$
        \end{proposition}
            \begin{proof}
                Let us check whether the constructed $Q \x \Z$-grading on $\hat{\g}_{[2]}$ is well-defined.

                Firstly, let us check that the grading is well-define on $\tilde{\g}_{[2]} := \g_{[2]} \oplus \z_{[2]}$. To this end, pick $x, y \in \g$ and that $x \in \g[\alpha], y \in \g[\beta]$ for some $\alpha, \beta \in \Phi \cup \{0\}$; also, choose some arbitrary $(m, p), (n, q) \in \Z^2$. Next, consider:
                    $$
                        \begin{aligned}
                            [x v^m t^p, y v^n t^q]_{\tilde{\g}_{[2]}} & = [x, y]_{\g} v^{m + n} t^{p + q} + (x, y)_{\g} v^m t^p \bar{d}(v^n t^p)
                            \\
                            & = [x, y]_{\g} v^{m + n} t^{p + q} + (x, y)_{\g} \delta_{(m, p) + (n, q), (0, 0)} ( n c_v + q c_t ) + (np - mq) K_{m + n, p + q}
                        \end{aligned}
                    $$
                Now, note that if either:
                    $$\alpha + \beta = 0, \alpha \not = 0$$
                or:
                    $$\alpha = \beta = 0$$
                (i.e. $x, y \in \h$) then:
                    $$[x, y] \in \h$$
                and hence:
                    $$\deg [x, y]_{\g} v^{m + n} t^{p + q} = \deg K_{m + n, p + q} = (0, m + n)$$
                On the other hand, if:
                    $$\alpha + \beta \not = 0$$
                then:
                    $$[x, y] \in \n^- \oplus \n^+$$
                which means in particular that at leeast either $x$ or $y$ is nilpotent under the vector representation of $\g$, and hence:
                    $$(x, y)_{\g} = 0$$
                as $(-, -)_{\g}$ is some non-zero multiple of the trace form, and traces of nilpotent matrices are equally $0$. Hence, in this case, we have that:
                    $$\deg [x v^m t^p, y v^n t^q]_{\tilde{\g}_{[2]}} = \deg [x, y]_{\g} v^{m + n} t^{p + q} = (\alpha + \beta, m + n)$$
                Both cases together show that the constructed $Q \x \Z$-grading on $\tilde{\g}_{[2]}$ is well-defined. 
                
                Secondly, note that from how commutators of elements of $\d_{[2]} := \bigoplus_{(r, s) \in \Z^2} \bbC D_{r, s} \oplus \bbC D_v \oplus \bbC D_t$ are given (cf. lemma \ref{lemma: explicit_commutators_between_basis_elements_of_toroidal_central_orthogonal_complement}), one sees that:
                    $$\deg [D_v, D_t] = (0, 0) = \deg D_v + \deg D_t$$
                    $$\deg [D_v, D_{r, s}] = (0, -r) = \deg D_v + \deg D_{r, s}$$
                    $$\deg [D_t, D_{r, s}] = (0, -r) = \deg D_t + \deg D_{r, s}$$
                    $$\deg [D_{a, b}, D_{r, s}] = \deg D_{a + r, b + s + 1} = (0, -(a + r)) = \deg D_{a, b} + \deg D_{r, s}$$
                Thus, the constructed grading is well-defined on $\d_{[2]}$. Recall also from proposition \ref{prop: lie_bracket_on_orthogonal_complement_of_toroidal_centre} that:
                    $$[\d_{[2]}, \d_{[2]}]_{\hat{\g}_{[2]}} \subseteq \z_{[2]} \oplus \d_{[2]}$$
                with the $\d_{[2]}$-summand being the usual commutator of derivations $[-, -]$ inherited from $\der_{\bbC}(A_{[2]})$, while the $\z_{[2]}$-summand is undetermined, but can be viewed as twist of $[-, -]$ by a cocycle $\sigma \in H^2_{\Lie}(\d_{[2]}, \z_{[2]})$ (cf. theorem \ref{theorem: non_uniqueness_of_yangian_extended_lie_algebras}). For this reason, we can and must choose the restriction of $[-, -]_{\hat{\g}_{[2]}}$ down to $\d_{[2]}$ to be the usual commutator $[-, -]$ for the construction of our $Q \x \Z$-grading. 
            \end{proof}
        
        \begin{remark}
            For what follows, let us recall from \cite[Chapter 7]{kac_infinite_dimensional_lie_algebras} that the root space decomposition of the untwisted affine Kac-Moody algebra $\hat{\g}_{[1]}$ takes the form:
                $$\hat{\g}_{[1]} \cong \hat{\h}_{[1]} \oplus \bigoplus_{\beta \in \Re(\hat{\Phi})} \hat{\g}_{[1]}[\beta] \oplus \bigoplus_{\beta \in \Im(\hat{\Phi})} \hat{\g}_{[1]}[\beta]$$
            in which the untwisted affine root system $\hat{\Phi}$ decomposes into a disjoint union of the subsets of real and imaginary roots:
                $$\hat{\Phi} \cong \Re(\hat{\Phi}) \cup \Im(\hat{\Phi})$$
            where:
                $$\Re(\hat{\Phi}) \cong \Phi + \Z\delta \cong \Phi \x \Z$$
                $$\Im(\hat{\Phi}) \cong (\Z \setminus \{0\})\delta$$
            and the corresponding root spaces are given by:
                $$\forall \alpha + m\delta \in \Re(\hat{\Phi}): \hat{\g}_{[1]}[\alpha + m\delta] \cong \g[\alpha] v^m$$
                $$\forall r\delta \in \Im(\hat{\Phi}): \hat{\g}_{[1]}[r\delta] \cong \h v^r$$
            The Cartan subalgebra $\hat{\h}_{[1]}$ is as in convention \ref{conv: a_fixed_untwisted_affine_kac_moody_algebra}.
        \end{remark}    
        The following is a corollary to proposition \ref{prop: root_grading_on_extended_toroidal_lie_algebras}. One can see it to be true simply by looking at the degrees of elements of $\hat{\g}_{[2]}$. 
        \begin{theorem}[Root space decomposition for extended toroidal Lie algebras] \label{theorem: root_space_decomposition_for_extended_toroidal_lie_algebras}
            The weight spaces of the adjoint action of $\hat{\g}_{[1]}$ on $\hat{\g}_{[2]}$ can be given explicitly in terms of the basis elements of the latter in the following manner:
                $$\forall (\alpha, m) \in \Phi \x \Z: \hat{\g}_{[2]}[\alpha + m\delta] \cong \hat{\g}_{[1]}[\alpha + m\delta] \tensor_{\bbC} \bbC[t^{\pm 1}]$$
                $$
                    \forall r \in \Z \setminus \{0\}: \hat{\g}_{[2]}[r \delta] \cong \hat{\g}_{[1]}[r \delta] \tensor_{\bbC} \bbC[t^{\pm 1}] \oplus \bigoplus_{s \in \Z} (\bbC K_{r, s} \oplus \bbC D_{-r, s})
                $$
                $$\hat{\g}_{[2]}[0] \cong \h \oplus (\bbC c_v \oplus \bbC c_t) \oplus (\bbC D_v \oplus \bbC D_t)$$
            Furthermore, $\hat{\g}_{[2]}$ is a weight module of $\hat{\g}_{[1]}$, i.e.:
                $$\hat{\g}_{[2]} \cong \bigoplus_{\beta \in \hat{\Phi} \cup \{0\}} \hat{\g}_{[2]}[\beta]$$
        \end{theorem}
        \begin{corollary}
            Recall the Manin triple:
                $$(\hat{\g}_{[2]}, \hat{\g}_{[2]}^+, \hat{\g}_{[2]}^-)$$
            from theorem \ref{theorem: extended_toroidal_manin_triples}. Precisely because this is a Manin triple, one has the following root space decompositions for the Lie subalgebras $\hat{\g}_{[2]}^{\pm}$, induced by the one on $\hat{\g}_{[2]}$:
                $$
                    \forall (\alpha, m) \in \Phi \x \Z:
                    \begin{cases}
                        \hat{\g}_{[2]}^+[\alpha + m\delta] \cong \hat{\g}_{[1]}[\alpha + m\delta] \tensor_{\bbC} \bbC[t]
                        \\
                        \hat{\g}_{[2]}^-[\alpha + m\delta] \cong \hat{\g}_{[1]}[\alpha + m\delta] \tensor_{\bbC} t^{-1}\bbC[t^{-1}]
                    \end{cases}
                $$
                $$
                    \forall r \in \Z \setminus \{0\}:
                    \begin{cases}
                        \text{$\hat{\g}_{[2]}^+ \cong \hat{\g}_{[1]}[r \delta] \tensor_{\bbC} \bbC[t^{\pm 1}] \oplus \bigoplus_{s \in \Z_{\leq 0}} (\bbC K_{r, s} \oplus \bbC D_{-r, s})$ if $r > 0$}
                        \\
                        \text{$\hat{\g}_{[2]}^- \cong \hat{\g}_{[1]}[r \delta] \tensor_{\bbC} \bbC[t^{\pm 1}] \oplus \bigoplus_{s \in \Z_{> 0}} (\bbC K_{r, s} \oplus \bbC D_{-r, s})$ if $r < 0$}
                    \end{cases}
                $$
                $$\hat{\g}_{[2]}^+[0] \cong \h \oplus \bbC c_v \oplus \bbC D_t, \hat{\g}_{[2]}^-[0] \cong \h \oplus \bbC c_t \oplus \bbC D_v$$
            Of course, one has also from these constructions that:
                $$\hat{\g}_{[2]}^{\pm} \cong \bigoplus_{\beta \in \hat{\Phi} \cup \{0\}} \hat{\g}_{[2]}^{\pm}[\beta]$$
        \end{corollary}
        
        \begin{remark}[Toroidal root systems] \label{remark: toroidal_root_systems}
            The root space decomposition of $\hat{\g}_{[2]}$ as in theorem \ref{theorem: root_space_decomposition_for_extended_toroidal_lie_algebras} suggests to us that it is possible to construct a \say{toroidal root system}:
                $$\hat{\Phi}_{[2]}$$
            for $\hat{\g}_{[2]}$ with the following features.
            \begin{enumerate}
                \item \textbf{(Anisotropic roots):} Firstly, there is a set of anisotropic roots - which are in bijection with the real roots of $\hat{\g}_{[1]}$ or equivalently, the roots of $\g$. Interestingly, the corresponding root spaces:
                    $$\hat{\g}_{[2]}[\alpha + m\delta], (\alpha, m) \in \Phi \x \Z$$
                are free and of rank $1$ over $\bbC[t^{\pm 1}]$, in good analogy with how real roots of an affine Kac-Moody algebras are equally of multiplicity $1$.

                Such roots are \textbf{anisotropic} in the following sense. If we fix:
                    $$(\alpha, m, p), (\beta, n, q) \in \Phi \x \Z^2$$
                along with root vectors:
                    $$x_{\alpha} \in \g[\alpha], x_{\beta} \in \g[\beta]$$
                then:
                    $$( x_{\alpha} v^m t^p, x_{\beta} v^n t^q )_{\hat{\g}_{[2]}} = \delta_{(\alpha, m, p) + (\beta, n, q), (0, 0, -1)}$$
                This suggest to us that for each positive real root:
                    $$\alpha + m\delta \in \hat{\Phi}^+ \cong \Phi^+ \x \Z_{\geq 0}$$
                one has the following non-trivial pairing of subspaces:
                    $$\left( \hat{\g}_{[2]}^-[\pm (\alpha + m\delta)], \hat{\g}_{[2]}^+[\pm (\alpha + m\delta)] \right)_{\hat{\g}_{[2]}} \not = 0$$
                thus justifying our use of the term \say{anisotropic}.
                \item \textbf{(Isotropic roots):} Observe that the \say{Cartan subalgebra}:
                    $$\hat{\g}_{[2]}[0]$$
                is finite-dimensional, namely of dimension $\dim_{\bbC} \h + 2 + 2$, with each summand of $2$ corresponding to one of the direct summands $\bbC c_v \oplus \bbC D_v$ and $\bbC c_t \oplus \bbC D_t$ of $\hat{\g}_{[2]}[0]$, similar to how:
                    $$\dim_{\bbC} \hat{\h}_{[1]} = \dim_{\bbC} \h + 2$$
                in the affine Kac-Moody case, where the summand of $2$ corressponds to the direct summand of the $1$-dimensional centre and the subspace spanned by the canonical degree derivation. From this, we infer that $\hat{\g}_{[2]}$ ought to admit two distinct weights with respect to the adjoint action of $\hat{\g}_{[1]}$ that can be reasonably called \textbf{isotropic roots}. In particular, these anisotropic roots are to be the image of the derivations:
                    $$D_v, D_t \in \hat{\g}_{[2]}[0]$$
                under the dualising map $\hat{\g}_{[2]}[0] \xrightarrow[]{\cong} \hat{\g}_{[2]}[0]^*$. Let us denote these roots, respectively, by:
                    $$\delta_v, \delta_t \in \hat{\g}_{[2]}[0]^*$$
                and note that because:
                    $$(\d_{[2]}, \d_{[2]})_{\hat{\g}_{[2]}} = 0$$
                per the construction of the bilinear form $(-, -)_{\hat{\g}_{[2]}}$ (cf. convention \ref{conv: orthogonal_complement_of_toroidal_centres}), we have that:
                    $$(\delta_v, \delta_v)_{\hat{\g}_{[2]}} = (\delta_t, \delta_t)_{\hat{\g}_{[2]}} = 0$$
                thus justifying our use of the term \say{isotropic}.

                Note also that, once again because $(\d_{[2]}, \d_{[2]})_{\hat{\g}_{[2]}} = 0$, we also have that:
                    $$(\delta_v, \delta_t)_{\hat{\g}_{[2]}} = 0$$
                i.e. the two isotropic roots of $\hat{\g}_{[2]}$ are perpendicular to one another. 
            \end{enumerate}

            In conclusion, we have that:
                $$\hat{\Phi}_{[2]} \cong \Phi \cup (\Z \delta_v \setminus \{0\}) \cup (\Z \delta_t \setminus \{0\}) \cong \hat{\Phi}_{[1]} \cup \Z \delta_t \setminus \{0\}$$
            where $\hat{\Phi}_{[1]} := \hat{\Phi}$ is the root system of the affine Kac-Moody algebra $\hat{\g}_{[1]}$. This allows us to rewrite the root space decomposition of $\hat{\g}_{[2]}$, as discussed in theorem \ref{theorem: root_space_decomposition_for_extended_toroidal_lie_algebras}, in the following more meaningful manner:
                $$\hat{\g}_{[2]} \cong \bigoplus_{\beta \in \hat{\Phi}_{[2]} \cup \{0\}} \hat{\g}_{[2]}[\beta]$$

            Note also that the set $\hat{\Phi}_{[2]}$ satisfies also some (but not all) the properties expected of an abstract root system. 

            \todo[inline]{What are the simple roots, positive/negative roots ?}
        \end{remark}
        The following result is nothing but a formal consequence of the discussion above.
        \begin{proposition}[Triangular decomposition of extended toroidal Lie algebras] \label{prop: triangular_decomposition_of_extended_toroidal_lie_algebras}
            Set:
                $$\hat{\g}_{[2]}[\hat{\Phi}_{[2]}^{\pm}] := \bigoplus_{\beta \in \hat{\Phi}_{[2]}^{\pm}} \hat{\g}_{[2]}[\beta]$$
        
            The root space decomposition of the Lie algebra $\hat{\g}_{[2]}$ (respectively, of $\hat{\g}_{[2]}^{\pm}$) induces a triangular decomposition thereof as follows:
                $$\hat{\g}_{[2]} \cong \hat{\g}_{[2]}[\hat{\Phi}_{[2]}^-] \oplus \hat{\g}_{[2]}[0] \oplus \hat{\g}_{[2]}[\hat{\Phi}_{[2]}^+]$$
        \end{proposition}
        Remark \ref{remark: toroidal_root_systems} also prompts the following result.
        \begin{lemma}[Chevalley-Serre presentation for extended toroidal Lie algebras] \label{lemma: chevalley_serre_presentation_for_extended_toroidal_lie_algebras}
            The extended toroidal Lie algebra $\hat{\g}_{[2]}$ is isomorphic to the Lie algebra $\hat{\t}$ generated by the set:
                $$\{ H_{i, r}, E_{i, r}^{\pm} \}_{(i, r) \in \hat{\Gamma}_0 \x \Z} \cup \{K, D\}$$
            whose elements are subjected to the following relations, given for all $(i, r), (j, s) \in \hat{\Gamma}_0 \x \Z$:
                $$[ H_{i, r}, H_{j, s} ] = 0$$
                $$[ H_{i, r}, E_{j, s}^{\pm} ] = \pm (\alpha_j, \check{\alpha}_i) E_{j, r + s}^{\pm}$$
                $$[ E_{i, r}^+, E_{j, s}^- ] = \delta_{ij} H_{i, r + s}$$
                $$[ E_{i, r + 1}^{\pm}, E_{j, s}^{\pm} ] - [ E_{i, r}^{\pm}, E_{j, s + 1}^{\pm} ] = 0$$
                $$[K, \hat{\t}] = 0$$
                $$[D, H_{i, r}] = -r H_{i, r}, [D, E_{i, r}^{\pm}] = -r E_{i, r}^{\pm}$$
            The isomorphism $\hat{\t} \xrightarrow[]{\cong} \hat{\g}_{[2]}$ in question is given as follows, for all $(i, r) \in \hat{\Gamma}_0 \x \Z$:
                $$\forall (i, r) \in \Gamma_0 \x \Z: E_{i, r}^{\pm} \mapsto e_i^{\pm} t^r, H_{i, r} \mapsto h_i t^r$$
                $$\forall (i, r) \in \{\theta\} \x \Z: E_{\theta, r}^{\pm} \mapsto e_{\theta}^{\mp} v^{\pm 1} t^r, H_{\theta, r} \mapsto h_{\theta} t^r + c_v t^r$$
                $$K \mapsto c_t$$
                $$D \mapsto D_t$$
        \end{lemma}
            \begin{proof}
                Clear from a combination of lemma \ref{lemma: chevalley_serre_presentation_for_central_extensions_of_multiloop_algebras} and remark \ref{remark: toroidal_root_systems}.
            \end{proof}
        \begin{corollary} \label{coro: chevalley_serre_presentation_for_extended_toroidal_lie_algebras}
            The Lie algebras $\hat{\g}_{[2]}^{\pm}$ is isomorphic to the Lie algebras $\hat{\t}^{\pm}$ generated, respectively, by the sets:
                $$\{ E_{i, r}^{\pm}, H_{i, r} \}_{(i, r) \in \Gamma_0 \x \Z_{\geq 0}} \cup \{D\}$$
                $$\{ E_{i, r}^{\pm}, H_{i, r} \}_{(i, r) \in \Gamma_0 \x \Z_{< 0}} \cup \{K\}$$
            whose elements are subjected to the same relations as in lemma \ref{lemma: chevalley_serre_presentation_for_extended_toroidal_lie_algebras}. The isomorphisms $\hat{\t}^{\pm} \xrightarrow[]{\cong} \hat{\g}_{[2]}^{\pm}$ in question are just codomain restrictions of the isomorphism $\hat{\t} \xrightarrow[]{\cong} \hat{\g}_{[2]}$ from \textit{loc. cit.}
        \end{corollary}
        \begin{remark}[Comparison to the Chevalley-Serre presentation for affine Yangians]
            
        \end{remark}
            
        \begin{convention}
            To streamline notations, let us from now on write:
                $$\hat{\h}_{[2]} := \hat{\g}_{[2]}[0]$$
            
            Also, let us use the subscript \say{$[1]$} to denote objects related to the affine Kac-Moody algebra $\hat{\g}_{[1]}$ (e.g. its root system is to be $\hat{\Phi}_{[1]}$, its root lattice is to be $\hat{Q}_{[1]}$, etc.). The unique lowest positive imaginary root of $\hat{\g}_{[1]}$ will be denoted by $\delta_{[1]}$ from now on (note that $\delta_{[1]} \not = \delta_v$).
        \end{convention}

        \newpage

    \subsection{Classical limits of universal quantum R-matrices of affine Yangians}
        In preparation for comparing the classical R-matrix $\sfr_{ \hat{\g}_{[2]}^+ }$ and the quantum R-matrix of $\rmY_{\hbar}(\hat{\g}_{[1]})$, let us firstly study a Gaussian/LU decomposition of $\sfr_{ \hat{\g}_{[2]}^+ }$:
            $$\sfr_{ \hat{\g}_{[2]}^+ } := \sfr_{ \hat{\g}_{[2]}^+ }^- + \sfr_{ \hat{\g}_{[2]}^+ }^0 + \sfr_{ \hat{\g}_{[2]}^+ }^+$$
        This is in fact one of the reasons for studying root space decompositions for $\hat{\g}_{[2]}$ and for $\hat{\g}_{[2]}^{\pm}$, as it affords us a description of the summands:
            $$\sfr_{ \hat{\g}_{[2]}^+ }^{\pm}$$
        as canonical tensors of the pairings:
            $$\bigoplus_{(\alpha, m) \in \Phi^+ \x \Z} \hat{\g}_{[2]}^{\mp}[-\alpha + m\delta] \hattensor_{\bbC} \hat{\g}_{[2]}^{\pm}[\alpha + m\delta]$$
        \begin{convention}
            Suppose that $\fraku$ is a symmetrisable Kac-Moody algebra whose Cartan matrix is indecomposable. Fix for it a Cartan subalgebra $\fraku^0$ and denote the corresponding triangular decomposition by:
                $$\fraku := \fraku^- \oplus \fraku^0 \oplus \fraku^+$$
        
            Denote by:
                $$\sfr_{\fraku}^{\pm}$$
            the Casimir tensor associated to the non-degenerate Kac-Moody pairing on $\fraku^{\mp} \hattensor_{\bbC} \fraku^{\pm}$ (where $\hattensor_{\bbC}$ denotes an appropriate completion with respect to the root grading on $\fraku^{\pm}$), and by:
                $$\sfr_{\fraku}^0$$
            the Casimir tensor associated to the non-degenerate Kac-Moody pairing on $\fraku^0 \tensor_{\bbC} (\fraku^0)^*$.

            Note that the classical R-matrix of the usual Lie bialgebra structure on $\fraku$ is nothing but:
                $$\sfr_{\fraku} := \sfr_{\fraku}^- + \sfr_{\fraku}^0 + \sfr_{\fraku}^+$$
        \end{convention}
        \begin{theorem}[A Gaussian decomposition for $\sfr_{ \hat{\g}_{[2]}^+ }$]
            The classical R-matrix $\sfr_{ \hat{\g}_{[2]}^+ }$ from corollary \ref{coro: extended_toroidal_lie_bialgebras} admits an additive decomposition:
                $$\sfr_{ \hat{\g}_{[2]}^+ } := \sfr_{ \hat{\g}_{[2]}^+ }^- + \sfr_{ \hat{\g}_{[2]}^+ }^0 + \sfr_{ \hat{\g}_{[2]}^+ }^+$$
            corresponding to the triangular decomposition of $\hat{\g}_{[2]}^+$ (cf. proposition \ref{prop: triangular_decomposition_of_extended_toroidal_lie_algebras}) in which:
                $$\sfr_{ \hat{\g}_{[2]}^+ }^{\pm} := \sfr_{\hat{\g}_{[1]}}^{\pm} v_2 \1(v_1, v_2) \1^+(t_1, t_2)$$
                $$\sfr_{ \hat{\g}_{[2]}^+ }^0 := \sfr_{\hat{\g}_{[1]}}^0 + D_t \tensor c_t$$
            In other words, $\sfr_{\hat{\g}_{[1]}}^{\pm}$ are - respectively, the Casimir tensor associated to the non-degenerate pairing\footnote{See remark \ref{remark: toroidal_root_systems} for more details.} on $\hat{\g}_{[2]}^{\pm}[\hat{\Phi}_{[1]}^{\pm}] \hattensor_{\bbC} \hat{\g}_{[2]}^{\mp}[\hat{\Phi}_{[1]}^{\mp}]$ via $(-, -)_{\hat{\g}_{[2]}}$ and likewise, $\sfr_{ \hat{\g}_{[2]}^+ }^0$ is the Casimir tensor associated to the non-degenerate pairing on $\hat{\h}_{[2]}^+ \tensor_{\bbC} \hat{\h}_{[2]}^-$ via $(-, -)_{\hat{\g}_{[2]}}$. 
        \end{theorem}
        \begin{corollary}
            The classical R-matrix $\sfr_{ \hat{\g}_{[2]}^+ }$ from corollary \ref{coro: extended_toroidal_lie_bialgebras} can be rewritten as:
                $$\sfr_{ \hat{\g}_{[2]}^+ } = \sfr_{\hat{\g}_{[1]}} v_2 \1(v_1, v_2) \1^+(t_1, t_2) + D_t \tensor c_t$$
        \end{corollary}
            
    \addcontentsline{toc}{section}{References}
    \printbibliography

\end{document}