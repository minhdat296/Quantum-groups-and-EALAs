\begin{remark}[$\tilde{\g}_{[2]}$ as a $\tilde{\g}_{[1]}$-module]
            In light of the Chevalley-Serre presentation for $\tilde{\g}_{[2]}$ (as in lemma \ref{lemma: chevalley_serre_presentation_for_central_extensions_of_multiloop_algebras}), we see that $\tilde{\g}_{[2]}$ is a weight module for $\hat{\g}_{[1]}$ and therefore, it is reasonable to propose that there exists a $\hat{Q}_{[1]}$-grading:
                $$\tilde{\g}_{[2]} \cong \tilde{\g}_{[2]}[0] \oplus \bigoplus_{\alpha \in \hat{\Phi}} \tilde{\g}_{[2]}[\alpha]$$
            where for each positive root $\alpha \in \hat{\Phi}^+$, the corresponding weight space is given by:
                $$\tilde{\g}_{[2]}[\pm \alpha] := \bigoplus_{r \in \Z} \bbC E_{\alpha, r}^{\pm}$$
            (note that the elements $E_{\alpha, r}^{\pm}$ are yet to be defined), and:
                $$\tilde{\g}_{[2]}[0] := \bbC K \oplus \bigoplus_{(i, r) \in \hat{\Gamma}_0 \x \Z} \bbC H_{i, r}$$
            with notations as in the aforementioned lemma. In fact, since the derivation $D_{0, -1} \in \hat{\g}_{[1]}$ (cf. convention \ref{conv: a_fixed_untwisted_affine_kac_moody_algebra}) acts as $0$ on the vector space $\tilde{\g}_{[2]}$ (cf. lemma \ref{lemma: chevalley_serre_presentation_for_central_extensions_of_multiloop_algebras}), we might as well regard $\tilde{\g}_{[2]}$ as a module over $\tilde{\g}_{[1]} := \uce(\g_{[1]})$; in doing so, $\tilde{\g}_{[2]}$ remains a weight module of $\tilde{\g}_{[1]}$. 
        \end{remark}
        \begin{remark}[Root vectors for $\tilde{\g}_{[2]}$] \label{remark: root_vectors_for_toroidal_lie_algebras}
            Before one can state a proper proposition, one needs to make sense of the \say{root vectors}:
                $$E_{\alpha, r}^{\pm}$$
            \begin{enumerate}
                \item If $\alpha \in \Phi^+$ is a positive finite root, then because we have that (cf. lemma \ref{lemma: chevalley_serre_presentation_for_central_extensions_of_multiloop_algebras}):
                    $$E_{i, r}^{\pm} \mapsto e_i^{\pm} t^r$$
                under the isomorphism $\t \xrightarrow[]{\cong} \tilde{\g}_{[2]}$, we can \textit{choose}:
                    $$E_{\alpha, r}^{\pm} := \ad(E_{i_m, r_m}^{\pm}) \cdot ... \cdot \ad(E_{i_2, r_2}^{\pm}) \cdot E_{i_1, r_1}^{\pm}$$
                such that:
                    $$\alpha = \sum_{1 \leq j \leq m} \alpha_{i_j}, r = \sum_{1 \leq j \leq m} r_j$$
                so that \textit{any} such choice would result in:
                    $$E_{\alpha, r}^{\pm} = \ad(e_{i_m}^{\pm}) \cdot ... \cdot \ad(e_{i_2}^{\pm}) \cdot e_{i_1} t^r$$
                Since finite roots (i.e. real roots) of any symmetrisable Kac-Moody are identically of multiplicity $1$ \textit{a priori} (cf. \cite[Chapter 4]{kac_infinite_dimensional_lie_algebras}), meaning that the element:
                    $$\ad(e_{i_m}^{\pm}) \cdot ... \cdot \ad(e_{i_2}^{\pm}) \cdot e_{i_1}^{\pm}$$
                span the entire root space $\hat{\g}_{[1]}[\alpha]$. The aforementioned choices of parititions $\alpha = \sum_{1 \leq j \leq m} \alpha_{i_j}$ and $r = \sum_{1 \leq j \leq m} r_j$ are consequently immaterial.
                
                \item On the other hand, if $\alpha$ is a positive imaginary root then because we have that:
                    $$E_{\theta, r}^{\pm} \mapsto e_{\theta}^{\mp} v^{\pm 1} t^r$$
                under the isomorphism $\t \xrightarrow[]{\cong} \tilde{\g}_{[2]}$, and because every imaginary root $\theta'$ is a $\Z \setminus \{0\}$-multiple of the lowest positive imaginary root $\theta$ (cf. \cite[Chapter 6]{kac_infinite_dimensional_lie_algebras}), we can always write:
                    $$E_{\alpha, r}^{\pm} = \ad(e_{\theta}^+)^{n - 1} e_{\theta}^+ v^{\pm n} t^r$$
                supposing that:
                    $$\alpha = n \theta$$
                for some $n \in \Z_{> 0}$.
            \end{enumerate}
        \end{remark}
        \begin{proposition}[Root space decomposition for toroidal Lie algebras] \label{prop: root_space_decomposition_for_toroidal_lie_algebras}
            Let us use the same notations as in lemma \ref{lemma: chevalley_serre_presentation_for_central_extensions_of_multiloop_algebras} and as a shorthand, let us write:
                $$\tilde{\h}_{[1]} := \bbC K \oplus \bigoplus_{r \in \Z} \bbC H_{i, r}$$
            Let $\tilde{\h}_{[1]}$ act on $\tilde{\g}_{[2]}$ via the adjoint action (well defined because the former is a Lie subalgebra of the latter). Then, there is a direct sum decomposition of $\tilde{\h}_{[1]}$-modules:
                $$\tilde{\g}_{[2]} \cong \tilde{\g}_{[2]}[0] \oplus \bigoplus_{\alpha \in \hat{\Phi}} \tilde{\g}_{[2]}[\alpha]$$
            where for each positive root $\alpha \in \hat{\Phi}^+$, the corresponding weight space is given by:
                $$\tilde{\g}_{[2]}[\pm \alpha] := \bigoplus_{r \in \Z} \bbC E_{\alpha, r}^{\pm}$$
                $$\tilde{\g}_{[2]}[0] := \tilde{\h}_{[1]}$$
        \end{proposition}
            \begin{proof}
                Using the relations from lemma \ref{lemma: chevalley_serre_presentation_for_central_extensions_of_multiloop_algebras}, we can check that the vector spaces $\tilde{\g}_{[2]}[\alpha]$ (for $\alpha \in \hat{\Phi} \cup \{0\}$) are indeed closed under the adjoint action of $\tilde{\h}_{[1]}$:
                \begin{itemize}
                    \item Obviously, $\tilde{\g}_{[2]}[0] \cong \tilde{\h}_{[1]}$ is an abelian Lie subalgebra of $\tilde{\g}_{[2]}$ so there is nothing to check there. 
                    \item Let us now show that $\tilde{\g}_{[2]}[\alpha]$ is a $\tilde{\h}_{[1]}$-module when $\alpha$ is a real root.
                    
                    From lemma \ref{lemma: chevalley_serre_presentation_for_central_extensions_of_multiloop_algebras}, we know that:
                        $$K \mapsto c_t$$
                    under the isomorphism $\t \xrightarrow[]{\cong} \tilde{\g}_{[2]}$ and hence is central as an element of $\tilde{\g}_{[2]}$, so it acts as zero on the vector spaces $\tilde{\g}_{[2]}[\alpha]$. As such, it suffices to check how the elements $H_{i, r}$ act on these vector spaces. 
                    
                    To this end, fix a vertex:
                        $$i \in \hat{\Gamma}_0$$
                    Next, recall that we have the relations:
                        $$[H_{i, r}, E_{j, s}^{\pm}]_{\tilde{\g}_{[2]}} = \pm \frac12 (\alpha_j, \check{\alpha}_i)_{\g} E_{j, r + s}^{\pm}$$
                    If we now pick an arbitrary positive finite root $\alpha \in \Phi^+$, an integer $s \in \Z$, as well as partitions: 
                        $$\alpha = \sum_{1 \leq k \leq m} \alpha_{j_k}, s = \sum_{1 \leq k \leq m} s_k$$
                    then we will have - per remark \ref{remark: root_vectors_for_toroidal_lie_algebras} - that:
                        $$
                            \begin{aligned}
                                [H_{i, r}, E_{\alpha, s}^{\pm}]_{\tilde{\g}_{[2]}} & = [ H_{i, r}, \ad(E_{j_m, s_m}^{\pm}) \cdot ... \cdot \ad(E_{j_2, s_2}^{\pm}) \cdot E_{i_1, s_1}^{\pm} ]_{\tilde{\g}_{[2]}}
                                \\
                                & = [ H_{i, r}, \ad(e_{j_m}^{\pm}) \cdot ... \cdot \ad(e_{j_2}^{\pm}) \cdot e_{j_1}^{\pm} t^s ]_{\tilde{\g}_{[2]}}
                                \\
                                & = 
                                \begin{cases}
                                    \text{$[ h_i t^r, \ad(e_{j_m}^{\pm}) \cdot ... \cdot \ad(e_{j_2}^{\pm}) \cdot e_{j_1}^{\pm} t^s ]_{\tilde{\g}_{[2]}}$ if $i \in \Gamma_0$}
                                    \\
                                    \text{$[ h_{\theta} t^r + c_v t^r, \ad(e_{j_m}^{\pm}) \cdot ... \cdot \ad(e_{j_2}^{\pm}) \cdot e_{j_1}^{\pm} t^s ]_{\tilde{\g}_{[2]}}$ if $i = \theta$}
                                \end{cases}
                                \\
                                & = 
                                \begin{cases}
                                    \text{$[ h_i, \ad(e_{j_m}^{\pm}) \cdot ... \cdot \ad(e_{j_2}^{\pm}) \cdot e_{j_1}^{\pm} ]_{\g} t^{r + s} + K(H_{i, r}, E_{\alpha, s}^{\pm})$ if $i \in \Gamma_0$}
                                    \\
                                    \text{$[ h_{\theta} + c_v, \ad(e_{j_m}^{\pm}) \cdot ... \cdot \ad(e_{j_2}^{\pm}) \cdot e_{j_1}^{\pm} ]_{\g} t^{r + s} + K(H_{i, r}, E_{\alpha, s}^{\pm})$ if $i = \theta$}
                                \end{cases}
                                \\
                                & = [ h_i, e_{\alpha}^{\pm} ]_{\g} t^{r + s} + K(H_{i, r}, E_{\alpha, s}^{\pm}) \: \text{(since $c_v$ is central)}
                                \\
                                & = \pm \frac12 (\alpha, \check{\alpha}_i)_{\g} e_{\alpha}^{\pm} t^{r + s} + K(H_{i, r}, E_{\alpha, s}^{\pm})
                                \\
                                & = \pm \frac12 (\alpha, \check{\alpha}_i)_{\g} E_{\alpha, r + s}^{\pm} + K(H_{i, r}, E_{\alpha, s}^{\pm})
                            \end{aligned}
                        $$
                    wherein we have set:
                        $$e_{\alpha}^{\pm} := \ad(e_{j_m}^{\pm}) \cdot ... \cdot \ad(e_{j_2}^{\pm}) \cdot e_{j_1}^{\pm}$$
                    and:
                        $$K(H_{i, r}, E_{\alpha, s}^{\pm}) \in \z_{[2]}$$
                    is the central summand of the commutator $[ h_i t^r, \ad(e_{j_m}^{\pm}) \cdot ... \cdot \ad(e_{j_2}^{\pm}) \cdot e_{j_1}^{\pm} t^s ]_{\tilde{\g}_{[2]}}$. Actually, we have that:
                        $$K(H_{i, r}, E_{\alpha, s}^{\pm}) = (h_i, e_{\alpha}^{\pm})_{\g} \cdot (...) = 0 \cdot (...) = 0$$
                    so in fact:
                        $$[H_{i, r}, E_{\alpha, s}^{\pm}]_{\tilde{\g}_{[2]}} = \pm \frac12 (\alpha, \check{\alpha}_i)_{\g} E_{\alpha, r + s}^{\pm}$$
                    This shows that $\tilde{\g}_{[2]}[\alpha]$ is a well-defined $\tilde{\h}_{[1]}$-module whenever $\alpha$ is a real root. 
                    \item Since every imaginary root in $\hat{\Phi}$ is a $\Z \setminus \{0\}$-multiple of the lowest positive imaginary root, it remains now to only show that:
                        $$\tilde{\g}_{[2]}[\theta]$$
                    is a $\tilde{\h}_{[1]}$-module. To this end, fix some $E_{\theta, s}^+ \in \tilde{\g}_{[2]}[\theta]$ as well as some $H_{i, r} \in \tilde{\h}_{[1]}$ and consider the following:
                        $$
                            \begin{aligned}
                                [H_{i, r}, E_{\theta, s}^+]_{\tilde{\g}_[2]} & =
                                \begin{cases}
                                    \text{$[h_i t^r, e_{\theta} v^{-1} t^s]_{\tilde{\g}_[2]}$ if $i \in \Gamma_0$}
                                    \\
                                    \text{$[h_{\theta} t^r + c_v t^r, e_{\theta} v^{-1} t^s]_{\tilde{\g}_[2]}$ if $i = \theta$}
                                \end{cases}
                                \\
                                & =
                                \begin{cases}
                                    \text{$[h_i, e_{\theta}^+]_{\g} v^{-1} t^{r + s} + K(H_{i, r}, E_{\theta, s}^+)$ if $i \in \Gamma_0$}
                                    \\
                                    \text{$[h_{\theta}, e_{\theta}^+]_{\g} v^{-1} t^{r + s} + K(H_{\theta, r}, E_{\theta, s}^+)$ if $i = \theta$}
                                \end{cases}
                                \\
                                & = 
                                \begin{cases}
                                    \text{$(\theta, \check{\alpha}_i)_{\g} e_{\theta}^+ v^{-1} t^{r + s} + K(H_{i, r}, E_{\theta, s}^+)$ if $i \in \Gamma_0$}
                                    \\
                                    \text{$(\theta, \check{\theta})_{\g} e_{\theta}^+ v^{-1} t^{r + s} + K(H_{\theta, r}, E_{\theta, s}^+)$ if $i = \theta$}
                                \end{cases}
                                \\
                                & = 
                                \begin{cases}
                                    \text{$K(H_{i, r}, E_{\theta, s}^+)$ if $i \in \Gamma_0$}
                                    \\
                                    \text{$(\theta, \check{\theta})_{\g} e_{\theta}^+ v^{-1} t^{r + s} + K(H_{\theta, r}, E_{\theta, s}^+)$ if $i = \theta$}
                                \end{cases}
                            \end{aligned}
                        $$
                    in which $K(H_{i, r}, E_{\theta, s}^+)$ is the central summand of the commutator $[h_i t^r, e_{\theta} v^{-1} t^s]_{\tilde{\g}_[2]}$ (where $i \in \hat{\Gamma}_0$ is any vertex of the affine Dynkin diagram); also, the second equality holds thanks to the fact that $c_v \in \tilde{\g}_{[2]}$ is central. Now, by arguing as in the previous case, we can also show that:
                        $$\forall i \in \hat{\Gamma}_0: K(H_{i, r}, E_{\theta, s}^+) = 0$$
                    and so in fact, we have that:
                        $$
                            [H_{i, r}, E_{\theta, s}^+]_{\tilde{\g}_[2]}
                            =
                            \begin{cases}
                                \text{$0$ if $i \in \Gamma_0$}
                                \\
                                \text{$(\theta, \check{\theta})_{\g} e_{\theta}^+ v^{-1} t^{r + s}$ if $i = \theta$}
                            \end{cases}
                        $$
                    From this, we infer that indeed, $\tilde{\g}_{[2]}[\theta]$ is a well-defined $\tilde{\h}_{[1]}$-module just as we desired for it to be. 
                \end{itemize}

                The analysis above shows also that firstly, for each root $\alpha \in \hat{\Phi}$, the corresponding vector space $\tilde{\g}_{[2]}[\alpha]$ is indeed the weight space of the $\tilde{\g}_{[1]}$-module $\tilde{\g}_{[2]}$ that is of weight $\alpha$ and secondly, that the weight-$0$ subspace of $\tilde{\g}_{[2]}$ is precisely $\tilde{\g}_{[2]}[0] \cong \tilde{\h}_{[1]}$. Our notations are thus justified. 

                Now that we know that $\tilde{\g}_{[2]}$ is a well-defined $\tilde{\h}_{[1]}$-module and that the vector subspaces $\tilde{\g}_{[2]}[\alpha]$ (for $\alpha \in \hat{\Phi} \cup \{0\}$) are well-defined submodules, it remains to only demonstrate that:
                    $$\tilde{\g}_{[2]}[\alpha] \cap \tilde{\g}_{[2]}[\beta] = \{0\}$$
                for any $\alpha \not = \beta \in \hat{\Phi} \cup \{0\}$. This is automatically true if we assume that $\alpha \not = \beta$, as this implies that elements of $\tilde{\g}_{[2]}[\alpha], \tilde{\g}_{[2]}[\beta]$, when acted upon by those of $\tilde{\h}_{[1]}$, will be of different eigenvalues.
            \end{proof}
        \begin{corollary} \label{coro: root_space_decomposition_for_toroidal_lie_algebras}
            Set:
                $$\tilde{\h}_{[1]}^+ := \bigoplus_{r \in \Z_{\geq 0}} \bbC H_{i, r}$$
                $$\tilde{\h}_{[1]}^- := \bbC K \oplus \bigoplus_{r \in \Z_{< 0}} \bbC H_{i, r}$$
        
            The Lie algebras $\tilde{\g}_{[2]}^{\pm}$ decompose into the following direct sums of $\tilde{\h}_{[1]}^{\pm}$-modules:
                $$\tilde{\g}_{[2]}^{\pm} \cong \tilde{\g}_{[2]}^{\pm}[0] \oplus \bigoplus_{\alpha \in \hat{\Phi}} \tilde{\g}_{[2]}^{\pm}[\alpha]$$
            wherein:
                $$\tilde{\g}_{[2]}^{\pm}[0] := \tilde{\h}_{[1]}^{\pm}$$
            and for each root $\alpha \in \hat{\Phi}$:
                $$\tilde{\g}_{[2]}^{\pm}[\alpha] := \tilde{\g}_{[2]}^{\pm} \cap \tilde{\g}_{[2]}[\alpha]$$
        \end{corollary}
        \begin{corollary}[Triangular decompositions for toroidal Lie algebras] \label{coro: triangular_decompositions_for_toroidal_lie_algebras}
            The Lie algebras $\tilde{\g}_{[2]}$ and $\tilde{\g}_{[2]}^{\pm}$, respectively, admit the following triangular decompositions:
                $$\tilde{\g}_{[2]} \cong \tilde{\g}_{[2]}[\hat{\Phi}^-] \oplus \tilde{\h}_{[1]} \oplus \tilde{\g}_{[2]}[\hat{\Phi}^+]$$
                $$\tilde{\g}_{[2]}^{\pm} \cong \tilde{\g}_{[2]}^{\pm}[\hat{\Phi}^-] \oplus \tilde{\h}_{[1]}^{\pm} \oplus \tilde{\g}_{[2]}^{\pm}[\hat{\Phi}^+]$$
            wherein:
                $$\tilde{\g}_{[2]}[\hat{\Phi}^{\pm}] := \bigoplus_{\alpha \in \hat{\Phi}^{\pm}} \tilde{\g}_{[2]}[\alpha]$$
                $$\tilde{\g}_{[2]}^{\pm}[\hat{\Phi}^{\pm}] := \tilde{\g}_{[2]}[\hat{\Phi}^{\pm}] \cap \tilde{\g}_{[2]}^{\pm} \cong \bigoplus_{\alpha \in \hat{\Phi}^{\pm}} \tilde{\g}_{[2]}^{\pm}[\alpha]$$
        \end{corollary}

        \begin{remark}[Pairing positive/negative root spaces in toroidal Lie algebras] \label{remark: pairing_positive_and_negative_root_spaces_in_toroidal_lie_algebras}
            Fix a positive root $\alpha \in \hat{\Phi}^+$. Recall from lemma \ref{lemma: chevalley_serre_presentation_for_central_extensions_of_multiloop_algebras} also that:
                $$\forall (i, r) \in \Gamma_0 \x \Z: E_{i, r}^{\pm} \mapsto e_i^{\pm} t^r, H_{i, r} \mapsto h_i t^r$$
                $$\forall (i, r) \in \{\theta\} \x \Z: E_{\theta, r}^{\pm} \mapsto e_{\theta}^{\mp} v^{\pm 1} t^r, H_{\theta, r} \mapsto h_{\theta} t^r + c_v t^r$$
            \begin{enumerate}
                \item Consider, firstly, the case where $\alpha \in \Phi^+$ is of finite type. According to remark \ref{remark: root_vectors_for_toroidal_lie_algebras}, we can arbitrarily choose a partition:
                    $$\alpha = \sum_{1 \leq j \leq m} \alpha_{i_j}$$
                so that:
                    $$E_{\alpha, r}^{\pm} = \ad(e_{i_m}^{\pm}) \cdot ... \cdot \ad(e_{i_2}^{\pm}) \cdot e_{i_1}^{\pm} t^r$$
                Set:
                    $$e_{\alpha}^{\pm} := \ad(e_{i_m}^{\pm}) \cdot ... \cdot \ad(e_{i_2}^{\pm}) \cdot e_{i_1}^{\pm}$$
                and let us choose the simple root vectors $e_{i_j}^{\pm}$ so that:
                    $$(e_{\alpha}^-, e_{\alpha}^+)_{\hat{\g}_{[1]}} = 1$$
                and note that since the root spaces $\tilde{\g}_{[2]}[\pm\alpha]$ are equally $1$-dimensional, these choices do not matter. We thus get that:
                    $$(E_{\alpha, r}^-, E_{\alpha, s}^+)_{\hat{\g}_{[2]}} = (e_{\alpha}^-, e_{\alpha}^+)_{\hat{\g}_{[1]}} \delta_{r + s, -1} = \delta_{r + s, -1}$$

                Similarly, we can show that:
                    $$(E_{\theta, r}^-, E_{\theta, s}^+)_{\hat{\g}_{[2]}} = (e_{\theta}^- v, e_{\theta}^+ v^{-1})_{\hat{\g}_{[1]}} \delta_{r + s, -1} = \delta_{r + s, -1}$$
                \item Secondly, given how $(-, -)_{\hat{\g}_{[2]}}$ is constructed and how $(-, -)_{\hat{\g}_{[1]}}$ pairs $\hat{\h}_{[1]}$ with itself (cf. \cite[Chapter 2]{kac_infinite_dimensional_lie_algebras}), we have that:
                    $$\forall i,j \in \Gamma_0: (H_{i, r}, H_{j, s})_{\hat{\g}_{[2]}} = (\alpha_j, \check{\alpha}_i)_{\g} \delta_{r + s, -1}$$
                and also that:
                    $$(H_{\theta, r}, H_{\theta, s})_{\hat{\g}_{[2]}} = (h_{\theta} t^r + c_v t^r, h_{\theta} t^s + c_v t^s)_{\hat{\g}_{[2]}} = (\theta, \check{\theta})_{\g} \delta_{r + s, -1}$$
                wherein the last equality holds thanks to $c_v$ being central in $\hat{\g}_{[2]}$ (cf. proposition \ref{prop: centres_of_extended_toroidal_lie_algebras}) and the bilinear form $(-, -)_{\hat{\g}_{[2]}}$ being invariant and non-degenerate by construction. 

                The element $c_t$ is also central in $\hat{\g}_{[2]}$, according to proposition \ref{prop: centres_of_extended_toroidal_lie_algebras}, so using invariance again, we get that:
                    $$\forall H \in \tilde{\h}_{[1]}: (H, K)_{\hat{\g}_{[2]}} = 0$$
            \end{enumerate}
        \end{remark}