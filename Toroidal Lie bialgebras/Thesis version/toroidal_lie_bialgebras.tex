\section{Affine Yangians as quantisations}
    \begin{convention}
        As a shorthand, we will be writing:
            $$\bar{\Delta}(X) := X \tensor 1 + 1 \tensor X$$
    \end{convention}

    \subsection{Manin triples and quantisations of Lie bialgebras} \label{subsection: manin_triples_and_quantisations_of_lie_bialgebras}
        \begin{remark}
            Technically speaking, we do not need our field $k$ to be algebraically closed for the abstract machineries presented in this subsection to be true. This hypothesis is only needed for some of the examples. 
        \end{remark}
    
        \begin{definition}[Lie bialgebras] \label{def: lie_bialgebras}
            Let $\a$ be a Lie algebra over $k$ equipped with a $k$-linear map:
                $$\delta: \a \to \a \tensor_k \a$$
            $\a$ will then be a \textbf{Lie bialgebra} (over $k$) with \textbf{Lie cobracket} $\delta$ if the following two conditions are satisfied:
            \begin{itemize}
                \item Firstly, we insist that $\delta$ is a Lie $1$-cocycle\footnote{In cohomological terms, one can write $\delta \in H^1_{\Lie}(\a, \a \tensor_k \a)$.} of $\a$ with coefficients in $\a \tensor_k \a$, which is to say that the following identity is to hold in $\rmU(\a) \tensor_k \rmU(\a)$:
                    $$\delta( [x, y] ) = [\bar{\Delta}(x), \delta(y)] + [\delta(x), \bar{\Delta}(y)]$$
                for all $x, y \in \a$.
                \item Secondly, we require that the map:
                    $$\delta^*: \a^* \tensor_k \a^* \to \a^*$$
                is a Lie bracket on the $k$-vector space $\a^*$.
            \end{itemize}
        \end{definition}

        \begin{definition}[Manin triples] \label{def: manin_triples}
            
        \end{definition}

        \begin{definition}[Quantisations] \label{def: quantisations}
            
        \end{definition}

        \todo[inline]{Not done}

    \subsection{(Extended) toroidal Lie bialgebras}
        \begin{theorem} \label{theorem: extended_toroidal_manin_triples}
            There is a complete topological Manin triple:
                $$(\extendedtoroidal, \extendedtoroidal^{\positive}, \extendedtoroidal^{\negative})$$
            wherein $\extendedtoroidal$ is equipped with the non-degenerate invariant inner product $(-, -)_{\extendedtoroidal}$ (cf. convention \ref{conv: orthogonal_complement_of_toroidal_centres}).
        \end{theorem}
            \begin{proof}
                We know from lemma \ref{lemma: positive/negative_extended_toroidal_lie_algebras} that $\extendedtoroidal^{\positive/\negative}$ are Lie subalgebras of $\extendedtoroidal$, so it now remains to show that $(-, -)_{\extendedtoroidal}$ pairs the subalgebras $\extendedtoroidal^{\positive/\negative}$ isotropically, but this is true entirely due to how this invariant bilinear form was constructed in convention \ref{conv: orthogonal_complement_of_toroidal_centres}.
            \end{proof}
        \begin{corollary}[Lie cobracket on $\extendedtoroidal^{\positive}$] \label{coro: extended_toroidal_lie_bialgebras}
            On the extended toroidal Lie algebra $\extendedtoroidal^{\positive}$, there is a continuous Lie cobracket\footnote{Note the completion!}, making $\extendedtoroidal^{\positive}$ a complete topological Lie bialgebra:
                $$\delta_{\extendedtoroidal^{\positive}}: \extendedtoroidal^{\positive} \to \extendedtoroidal^{\positive} \hattensor_k \extendedtoroidal^{\positive}$$
            given for any $X \in \extendedtoroidal^{\positive}$ by the following formula (cf. \cite{etingof_kazhdan_quantisation_1}):
                $$\delta_{\extendedtoroidal^{\positive}}(X) = [ X \tensor 1 + 1 \tensor X, \sfr_{\extendedtoroidal^{\positive}} ]$$
            wherein:
                $$\sfr_{\extendedtoroidal^{\positive}} := \sfr_{\g} + \sfr_{\z_{[2]}^{\positive}} + \sfr_{\d_{[2]}^{\positive}} \in \extendedtoroidal^{\positive} \hattensor_k \extendedtoroidal^{\negative}$$
            with:
                $$\sfr_{\g_{[2]}^{\positive}} \in \g_{[2]}^{\positive} \hattensor_k \g_{[2]}^{\negative} := \sfr_{\g} v_2 \1(v_1, v_2) \1(t_1, t_2)$$
            and\footnote{Note how we are simply summing over tensor products of dual basis elements.} $\sfr_{\z_{[2]}^{\positive}} \in \z_{[2]}^{\positive} \hattensor_k \d_{[2]}^{\positive}$ and $\sfr_{\d_{[2]}^{\positive}} \in \d_{[2]}^{\positive} \hattensor_k \z_{[2]}^{\positive}$ being given by the following formulae:
                $$\sfr_{\z_{[2]}^{\positive}} := \sum_{(r, s) \in \Z \x \Z_{> 0}} K_{r, s} \tensor D_{r, s} + c_v \tensor D_v$$
                $$\sfr_{\d_{[2]}^{\positive}} := \sum_{(r, s) \in \Z \x \Z_{\leq 0}} D_{r, s} \tensor K_{r, s} + D_t \tensor c_t$$
        \end{corollary}
    
        \begin{convention}[Formal Dirac distributions] \label{conv: formal_dirac_distributions}
            We will be using the following shorthands:
                $$\1(z, w) = \sum_{m \in \Z} z^m w^{-m - 1}$$
                $$\1^+(z, w) = \sum_{m \in \Z_{\geq 0}} z^m w^{-m - 1}$$
            as opposed to the usual $\delta$ notation, in order to avoid confusion with Lie cobrackets.
        \end{convention}

        \begin{remark}[Total degrees of \say{Yangian} canonical elements] \label{remark: total_degrees_of_classical_yangian_R_matrices}
            One property of the R-matrix $\sfr_{\extendedtoroidal^{\positive}}$ from corollary \ref{coro: extended_toroidal_lie_bialgebras} that will help simplify some computations later on (see the proof of theorem \ref{theorem: toroidal_lie_bialgebras}) is that they are of total degree $-1$. 

            Recall that if $V := \bigoplus_{m \in \Z} V_m, W := \bigoplus_{n \in \Z} W_n$ are $\Z$-graded vector spaces then for any $k \in \Z$, we have that:
                $$(V \tensor_k W)_k \cong \bigoplus_{m + n = k} V_m \tensor_k W_n$$
                
            If we now take $V = W = \rmU(\toroidal)$ then the claim from above would read:
                $$\sfr_{\toroidal^{\positive}} \in ( \rmU(\toroidal^{\positive}) \tensor_k \rmU(\toroidal^{\negative}) )_{-1}$$
            with the $\Z$-grading on $\toroidal^{\positive/\negative}$ (and hence on $\rmU(\toroidal^{\positive/\negative})$) being the one on the second variable $t$ (cf. remark \ref{remark: Z_gradings_on_toroidal_lie_algebras}), and actually, this is entirely due to:
                $$\sfr_{\g_{[2]}^{\positive}} \in ( \rmU(\g_{[2]}^{\positive}) \tensor_k \rmU(\g_{[2]}^{\negative}) )_{-1}$$

            What this means for us is that, should we have $X \in \toroidal^{\positive}$ such that:
                $$\deg X \leq 0$$
            then it will automatically be the case that:
                $$\delta_{\toroidal^{\positive}}(X) = 0$$
        \end{remark}
        
        We are now finally able to put a Lie cobracket on the toroidal Lie algebra $\toroidal^{\positive}$, compatible with the Lie bracket thereon in a manner that produces a Lie bialgebra structure. This Lie bialgebra structure is the classical limit of the coproduct on the formal Yangian:
            $$\formalyangian := \formalyangian(\hat{\g})$$
        associated to the affine Kac-Moody algebra $\hat{\g}$. 
        \begin{theorem}[Toroidal Lie bialgebras] \label{theorem: toroidal_lie_bialgebras}
            Assume convention \ref{conv: a_fixed_untwisted_affine_kac_moody_algebra} and let us abbreviate:
                $$\hat{\delta}^{\positive} := \delta_{\extendedtoroidal^{\positive}}$$
            with $\delta_{\extendedtoroidal^{\positive}}$ as in corollary \ref{coro: extended_toroidal_lie_bialgebras}. Let:
                $$\tilde{\delta}^{\positive} := \hat{\delta}^{\positive}|_{\toroidal}$$
            Then $(\toroidal^{\positive}, \tilde{\delta}^{\positive})$ will be a complete topological Lie sub-bialgebra of $(\extendedtoroidal^{\positive}, \hat{\delta}^{\positive})$ as given in corollary \ref{coro: extended_toroidal_lie_bialgebras}. Thanks to corollary \ref{coro: levendorskii_presentation_for_central_extensions_of_multiloop_algebras}, we know that it is enough to specify how $\tilde{\delta}^{\positive}$ is given on the set of generators:
                $$\{X_{i, 0}^{\pm}\}_{i \in \hat{\simpleroots}} \cup \{H_{i, r}\}_{ (i, r) \in \hat{\simpleroots} \x \{0, 1\} }$$
            and since we know that under the isomorphism $\t^+ \xrightarrow[]{\cong} \toroidal^{\positive}$ in \textit{loc. cit.}, we have the following assignments:
                $$\forall i \in \hat{\simpleroots}: X_{i, 0}^{\pm} \mapsto x_i^{\pm}, H_{i, 0} \mapsto h_i$$
                $$\forall i \in \simpleroots: H_{i, 1} \mapsto h_i t$$
                $$H_{\theta, 1} \mapsto h_{\theta} t + t c_v$$
            it is enough to specify the following, wherein $h \in \h$ is arbitrary:
                $$\tilde{\delta}^{\positive}(h) = 0$$
                $$\tilde{\delta}^{\positive}(ht) = [h_1 \tensor 1, \sfr_{\g} v_2 \1(v_1, v_2)]$$
                $$\tilde{\delta}^{\positive}(t c_v) = 0$$
        \end{theorem}
            \begin{proof}
                \begin{enumerate}
                    \item Since $\deg x = 0$ for all $x \in \g$, we get via remark \ref{remark: total_degrees_of_classical_yangian_R_matrices} that:
                        $$\hat{\delta}^{\positive}(x) = 0$$
                    and in particular, we have that:
                        $$\hat{\delta}^{\positive}(h) = 0$$

                    \item Let us now compute $\hat{\delta}^{\positive}(ht)$ for an arbitrary $h \in \h$. 
                    \begin{enumerate}
                        \item \textbf{($\g_{[2]}^{\positive}$-component):} Firstly, to compute:
                            $$[\bar{\Delta}(ht), \sfr_{\g_{[2]}^{\positive}}]$$
                        let us firstly note that:
                            $$\sfr_{\g_{[2]}^{\positive}} = \sfr_{\g} v_2\1(v_1, v_2) \1^+(t_1, t_2)$$
                        Let us also choose a root basis for $\g$ for writing out $\sfr_{\g}$ explicitly: this is to say that for each positive root $\alpha \in \Phi^+$, we choose corresponding basis vectors $x_{\alpha}^{\pm} \in \g_{\pm \alpha}$ normalised so that:
                            $$(x_{\alpha}^-, x_{\alpha}^+)_{\g} = 1$$
                        to get the following basis for $\g$:
                            $$\{h_i\}_{i \in \simpleroots} \cup \{x_{\alpha}^-, x_{\alpha}^+\}_{\alpha \in \Phi^+}$$
                        From this, we see that:
                            $$
                                \begin{aligned}
                                    & [\bar{\Delta}(ht), \sfr_{\g_{[2]}^{\positive}}]
                                    \\
                                    = & 
                                    \begin{aligned}
                                        & -\sum_{i \in \simpleroots} [\bar{\Delta}(ht), h_i \tensor h_i v_2\1(v_1, v_2) \1^+(t_1, t_2)]
                                        \\
                                        - & \sum_{\alpha \in \Phi^+} [\bar{\Delta}(ht), (x_{\alpha}^- \tensor x_{\alpha}^+ + x_{\alpha}^+ \tensor x_{\alpha}^-) v_2\1(v_1, v_2) \1^+(t_1, t_2)]
                                    \end{aligned}
                                \end{aligned}
                            $$

                        Now, for each $i \in \simpleroots$, observe that:
                            $$
                                \begin{aligned}
                                    & [h t_1 \tensor 1, h_i \tensor h_i v_2\1(v_1, v_2) \1^+(t_1, t_2)]
                                    \\
                                    = & \sum_{(m, p) \in \Z \x \Z_{\geq 0}} [ht_1 \tensor 1, h_i v_1^m t_1^p \tensor h_i v_2^{-m} t_2^{-p - 1}]
                                    \\
                                    = & \sum_{(m, p) \in \Z \x \Z_{\geq 0}} [ht_1, h_i v_1^m t_1^p]_{\toroidal^{\positive}} \tensor h_i v_2^{-m} t_2^{-p - 1}
                                    \\
                                    = & \sum_{(m, p) \in \Z \x \Z_{\geq 0}} (h, h_i)_{\g} v_1^m t_1^p \bar{d}(t_1) \tensor h_i v_2^{-m} t_2^{-p - 1}
                                \end{aligned}
                            $$
                        and likewise, that:
                            $$[1 \tensor h t_2, h_i \tensor h_i v_2\1(v_1, v_2) \1^+(t_1, t_2)] = \sum_{(m, p) \in \Z \x \Z_{\geq 0}} h_i v_1^m t_1^p \tensor (h, h_i)_{\g} v_2^{-m} t_2^{-p - 1} \bar{d}(t_2)$$
                        Adding the two summands together then yields:
                            $$
                                \begin{aligned}
                                    & [\bar{\Delta}(ht), h_i \tensor h_i v_2\1(v_1, v_2) \1^+(t_1, t_2)]
                                    \\
                                    = & (h, h_i)_{\g} \sum_{(m, p) \in \Z \x \Z_{\geq 0}} \left( v_1^m t_1^p \bar{d}(t_1) \tensor h_i v_2^{-m} t_2^{-p - 1} + h_i v_1^m t_1^p \tensor v_2^{-m} t_2^{-p - 1} \bar{d}(t_2) \right)
                                    \\
                                    = & (h, h_i)_{\g} ( \bar{d}(t_1) \tensor h_i + h_i \tensor \bar{d}(t_2) ) v_2\1(v_1, v_2) \1^+(t_1, t_2)
                                \end{aligned}
                            $$
                        
                        Next, consider the following:
                            $$
                                \begin{aligned}
                                    & [ht_1 \tensor 1, x_{\alpha}^- \tensor x_{\alpha}^+ v_2\1(v_1, v_2) \1^+(t_1, t_2)]
                                    \\
                                    = & \sum_{(m, p) \in \Z \x \Z_{\geq 0}} [ht_1 \tensor 1, x_{\alpha}^- v_1^m t_1^p \tensor x_{\alpha}^+ v_2^{-m} t_2^{-p - 1}]
                                    \\
                                    = & \sum_{(m, p) \in \Z \x \Z_{\geq 0}} [ht_1, x_{\alpha}^- v_1^m t_1^p]_{\toroidal^{\positive}} \tensor x_{\alpha}^+ v_2^{-m} t_2^{-p - 1}
                                    \\
                                    = & \sum_{(m, p) \in \Z \x \Z_{\geq 0}} \left( -\alpha(h) x_{\alpha}^- v_1^m t_1^{p + 1} + (h, x_{\alpha}^-)_{\g} t_1 \bar{d}(v_1^m t_1^p) \right) \tensor x_{\alpha}^+ v_2^{-m} t_2^{-p - 1}
                                    \\
                                    = & \sum_{(m, p) \in \Z \x \Z_{\geq 0}} -\alpha(h) x_{\alpha}^- v_1^m t_1^{p + 1} \tensor x_{\alpha}^+ v_2^{-m} t_2^{-p - 1}
                                    \\
                                    & = -\alpha(h) ( x_{\alpha}^- \tensor x_{\alpha}^+ ) v_2 \1(v_1, v_2) t_1 \1^+(t_1, t_2)
                                \end{aligned}    
                            $$
                        wherein the second-to-last identity comes from the fact that\footnote{This can be proven easily by passing to the vector representation of $\g$, wherein $h$ is represented by a diagonal matrix while $x^{\pm}$ is represented by an upper/lower triangular matrix, and then using the fact that $(-, -)_{\g}$ differs from the trace form only by a non-zero constant.}:
                            $$(h, x^{\pm})_{\g} = 0$$
                        for every $h \in \h$ and every $x^{\pm} \in \n^{\pm}$. Similarly, we find that:
                            $$[ht_1 \tensor 1, x_{\alpha}^+ \tensor x_{\alpha}^- v_2\1(v_1, v_2) \1^+(t_1, t_2)] = \alpha(h) ( x_{\alpha}^+ \tensor x_{\alpha}^- ) v_2 \1(v_1, v_2) t_1 \1^+(t_1, t_2)$$
                        By putting the two together, one obtains:
                            $$[h t_1 \tensor 1, (x_{\alpha}^- \tensor x_{\alpha}^+ + x_{\alpha}^+ \tensor x_{\alpha}^-) v_2\1(v_1, v_2) \1^+(t_1, t_2)] = -\alpha(h) ( x_{\alpha}^- \tensor x_{\alpha}^+ - x_{\alpha}^+ \tensor x_{\alpha}^- ) v_2 \1(v_1, v_2) t_1 \1^+(t_1, t_2)$$
                        Likewise, we find that:
                            $$[1 \tensor h t_2, (x_{\alpha}^- \tensor x_{\alpha}^+ + x_{\alpha}^+ \tensor x_{\alpha}^-) v_2\1(v_1, v_2) \1^+(t_1, t_2)] = \alpha(h) ( x_{\alpha}^- \tensor x_{\alpha}^+ - x_{\alpha}^+ \tensor x_{\alpha}^- ) v_2 \1(v_1, v_2) t_2 \1^+(t_1, t_2)$$
                        and hence:
                            $$
                                \begin{aligned}
                                    & [\bar{\Delta}(ht), \sfr_{\g_{[2]}^{\positive}}]
                                    \\
                                    = &
                                    -\left(
                                    \begin{aligned}
                                        & \sum_{i \in \simpleroots} (h, h_i)_{\g} ( \bar{d}(t_1) \tensor h_i + h_i \tensor \bar{d}(t_2) )
                                        \\
                                        + & \sum_{\alpha \in \Phi^+} \alpha(h) ( x_{\alpha}^- \tensor x_{\alpha}^+ - x_{\alpha}^+ \tensor x_{\alpha}^- )(t_2 - t_1)
                                    \end{aligned}
                                    \right) v_2 \1(v_1, v_2) \1^+(t_1, t_2)
                                    \\
                                    & = -\left( \bar{d}(t_1) \tensor h + h \tensor \bar{d}(t_2) + [h_1 \tensor 1, \sfr_{\g}] (t_2 - t_1) \right) v_2 \1(v_1, v_2) \1^+(t_1, t_2)
                                    \\
                                    & = -\left( \bar{d}(t_1) \tensor h + h \tensor \bar{d}(t_2) \right) v_2 \1(v_1, v_2) \1^+(t_1, t_2) + [h_1 \tensor 1, \sfr_{\g}] v_2 \1(v_1, v_2)
                                \end{aligned}
                            $$
                        We note that the last equality holds thanks to the fact that:
                            $$(t_2 - t_1) \1^+(t_1, t_2) = (t_2 - t_1) \sum_{p \in \Z_{\geq 0}} t_1^p t_2^{-p - 1} = (t_2 - t_1) \frac{1}{t_2 - t_1} = 1$$
                            
                        \item \textbf{($\z_{[2]}^{\positive}$-component):} Recall from corollary \ref{coro: extended_toroidal_lie_bialgebras} that:
                            $$\sfr_{\z_{[2]}^{\positive}} := \sum_{(r, s) \in \Z \x \Z_{> 0}} K_{r, s} \tensor D_{r, s} + c_{v_1} \tensor D_{v_2}$$
                        and so:
                            $$
                                \begin{aligned}
                                    & [\bar{\Delta}(ht), \sfr_{\z_{[2]}^{\positive}}]
                                    \\
                                    = & \sum_{(r, s) \in \Z \x \Z_{> 0}} [\bar{\Delta}(ht), K_{r, s} \tensor D_{r, s}] + [\bar{\Delta}(ht), c_{v_1} \tensor D_{v_2}]
                                    \\
                                    = & -\sum_{(r, s) \in \Z \x \Z_{> 0}} K_{r, s} \tensor h D_{r, s}(t) - c_{v_1} \tensor h D_{v_2}(t_2)
                                    \\
                                    = & -\sum_{(r, s) \in \Z \x \Z_{> 0}} K_{r, s} \tensor r h v_2^{-r} t_2^{-s}
                                \end{aligned}
                            $$
                        where the minus sign in the third equation appeared because:
                            $$[ht, D_{r, s}] = -[D_{r, s}, ht] = -h D_{r, s}(t)$$
                            $$[ht, D_v] = -[D_v, ht] = -h D_v(t)$$
                        (cf. remark \ref{remark: derivation_action_on_multiloop_algebras}) and the last equality is due to the fact that:
                            $$D_{r, s} = -s v^{-r + 1} t^{-s - 1} \del_v + r v^{-r} t^{-s} \del_t$$
                            $$D_v = -v t^{-1} \del_v$$
                        (cf. remark \ref{remark: dual_of_toroidal_centres_contains_derivations}). 
                        
                        \item \textbf{($\d_{[2]}^{\positive}$-component):} Recall from corollary \ref{coro: extended_toroidal_lie_bialgebras} that:
                            $$\sfr_{\z_{[2]}^{\positive}} := \sum_{(r, s) \in \Z \x \Z_{\leq 0}} D_{r, s} \tensor K_{r, s} + D_{t_1} \tensor c_{t_2}$$
                        and so:
                            $$
                                \begin{aligned}
                                    & [\bar{\Delta}(ht), \sfr_{\z_{[2]}^{\positive}}]
                                    \\
                                    = & \sum_{(r, s) \in \Z \x \Z_{\leq 0}} [\bar{\Delta}(ht), D_{r, s} \tensor K_{r, s}] + [\bar{\Delta}(ht), D_{t_1} \tensor c_{t_2}]
                                    \\
                                    = & -\sum_{(r, s) \in \Z \x \Z_{\leq 0}} h D_{r, s}(t_1) \tensor K_{r, s} - h D_{t_1}(t_1) \tensor c_{t_2}
                                    \\
                                    = & -\sum_{(r, s) \in \Z \x \Z_{\leq 0}} r h v_1^{-r} t_1^{-s} \tensor K_{r, s} + h \tensor c_{t_2}
                                \end{aligned}
                            $$
                        where the minus sign in the third equation appeared because:
                            $$[ht, D_{r, s}] = -[D_{r, s}, ht] = -h D_{r, s}(t)$$
                            $$[ht, D_t] = -[D_t, ht] = -h D_t(t)$$
                        (cf. remark \ref{remark: derivation_action_on_multiloop_algebras}) and the the last equality is due to the fact that:
                            $$D_{r, s} = -s v^{-r + 1} t^{-s - 1} \del_v + r v^{-r} t^{-s} \del_t$$
                            $$D_t = -\del_t$$
                        (cf. remark \ref{remark: dual_of_toroidal_centres_contains_derivations}). 
                    \end{enumerate}

                    Since we know that:
                        $$[\bar{\Delta}(ht), \sfr_{\g_{[2]}^{\positive}}] = -\left( \bar{d}(t_1) \tensor h + h \tensor \bar{d}(t_2) \right) v_2 \1(v_1, v_2) \1^+(t_1, t_2) + [h_1 \tensor 1, \sfr_{\g}] v_2 \1(v_1, v_2)$$
                    we now claim that:
                        $$[\bar{\Delta}(ht), \sfr_{\z_{[2]}^{\positive}} + \sfr_{\d_{[2]}^{\positive}}] = \left( \bar{d}(t_1) \tensor h + h \tensor \bar{d}(t_2) \right) v_2 \1(v_1, v_2) \1^+(t_1, t_2)$$
                    (since ultimately, we would like to show that $\hat{\delta}^{\positive}(ht) = \sfr_{\g} v_2 \1(v_1, v_2)$), and to prove that this is the case, let us first note that we now have that:
                        $$
                            \begin{aligned}
                                & [\bar{\Delta}(ht), \sfr_{\z_{[2]}^{\positive}} + \sfr_{\d_{[2]}^{\positive}}]
                                \\
                                = & -\sum_{(r, s) \in \Z \x \Z_{> 0}} \left( K_{r, s} \tensor r h v_2^{-r} t_2^{-s} + r h v_1^{-r} t_1^s \tensor K_{r, -s} \right) - \sum_{r \in \Z} r h v_1^{-r} \tensor K_{r, 0} + h \tensor c_{t_2}
                            \end{aligned}
                        $$
                    wherein the first summand corresponds to the indices $(r, 0) \in \Z \x \Z_{\leq 0}$. From example \ref{example: toroidal_lie_algebras_centres}, we know that:
                        $$
                            K_{r, s} :=
                            \begin{cases}
                                \text{$\frac1s v^{r - 1} t^s \bar{d}(v)$ if $(r, s) \in \Z \x (\Z \setminus \{0\})$}
                                \\
                                \text{$-\frac1r v^r t^{-1} \bar{d}(t)$ if $(r, s) \in (\Z \setminus \{0\}) \x \{0\}$}
                                \\
                                \text{$0$ if $(r, s) = (0, 0)$}
                            \end{cases}
                        $$
                    from which one infers that:
                        $$
                            \begin{aligned}
                                & -\sum_{r \in \Z} r h v_1^{-r} \tensor K_{r, 0}
                                \\
                                = & -\sum_{r \in \Z} r h v_1^{-r} \tensor \left( -\frac1r v_2^r t_2^{-1} \bar{d}(t_2) \right)
                                \\
                                = & \sum_{r \in \Z} h v_1^{-r} \tensor v_2^r t_2^{-1} \bar{d}(t_2)
                                \\
                                = & \sum_{r \in \Z} h v_1^{-r} \tensor v_2^r t_2^{-1} \bar{d}(t_2)
                            \end{aligned}
                        $$
                        
                    Next, recall again from example \ref{example: toroidal_lie_algebras_centres} that:
                        $$(r, s) \in \Z^2 \implies K_{r, s} = \frac1s v^{r - 1} t^s \bar{d}(v) = -\frac1r v^r t^{s - 1} \bar{d}(t) \in \bar{\Omega}_{[2]}$$
                    and then consider the following:
                        $$
                            \begin{aligned}
                                & -\sum_{(r, s) \in \Z \x \Z_{> 0}} \left( K_{r, s} \tensor r h v_2^{-r} t_2^{-s} + r h v_1^{-r} t_1^s \tensor K_{r, -s} \right)
                                \\
                                = & \sum_{(r, s) \in \Z \x \Z_{> 0}} \left( v_1^r t_1^{s - 1} \bar{d}(t_1) \tensor h v_2^{-r} t_2^{-s} - h v_1^{-r} t_1^s \tensor v_2^r t_2^{-s - 1} \bar{d}(t_2) \right)
                            \end{aligned}
                        $$
                    wherein we note that for all $s \in \Z_{> 0}$, the summands corresponding to the indices $(0, s)$ vanish.

                    We now have that:
                        $$
                            \begin{aligned}
                                & [\bar{\Delta}(ht), \sfr_{\z_{[2]}^{\positive}} + \sfr_{\d_{[2]}^{\positive}}]
                                \\
                                = & \sum_{(r, s) \in \Z \x \Z_{> 0}} \left( K_{r, s} \tensor r h v_2^{-r} t_2^{-s} + r h v_1^{-r} t_1^s \tensor K_{r, -s} \right) - \sum_{r \in \Z} r h v_1^{-r} \tensor K_{r, 0} + h \tensor c_{t_2}
                                \\
                                = & \sum_{(r, s) \in \Z \x \Z_{> 0}} \left( v_1^r t_1^{s - 1} \bar{d}(t_1) \tensor h v_2^{-r} t_2^{-s} - h v_1^{-r} t_1^s \tensor v_2^r t_2^{-s - 1} \bar{d}(t_2) \right) + \sum_{r \in \Z} h v_1^{-r} \tensor v_2^r t_2^{-1} \bar{d}(t_2) + h \tensor t_2^{-1} \bar{d}(t_2)
                                \\
                                = & \sum_{(r, s) \in \Z \x \Z_{> 0}} \left( v_1^r t_1^{s - 1} \bar{d}(t_1) \tensor h v_2^{-r} t_2^{-s} + h v_1^r t_1^s \tensor v_2^{-r} t_2^{-s - 1} \bar{d}(t_2) \right) + \sum_{r \in \Z} h v_1^{-r} \tensor v_2^r t_2^{-1} \bar{d}(t_2)
                                \\
                                = & \sum_{(r, s) \in \Z \x \Z_{> 0}} \left( v_1^r t_1^{s - 1} \bar{d}(t_1) \tensor h v_2^{-r} t_2^{-s} + h v_1^{-r} t_1^s \tensor v_2^r t_2^{-s - 1} \bar{d}(t_2) \right) + \sum_{r \in \Z} h v_1^r \tensor v_2^{-r} t_2^{-1} \bar{d}(t_2)
                                \\
                                = & ( \bar{d}(t_1) \tensor h ) \sum_{(r, s) \in \Z \x \Z_{> 0}} v_1^r t_1^{s - 1} \tensor v_2^{-r} t_2^{-s} + ( h \tensor \bar{d}(t_2) ) \left( \sum_{(r, s) \in \Z \x \Z_{> 0}} v_1^r t_1^s \tensor v_2^{-r} t_2^{-s - 1} + \sum_{r \in \Z} v_1^r \tensor v_2^{-r} t_2^{-1} \right)
                                \\
                                = & ( \bar{d}(t_1) \tensor h ) \sum_{(r, s) \in \Z \x \Z_{\geq 0}} v_1^r t_1^s \tensor v_2^{-r} t_2^{-s - 1} + ( h \tensor \bar{d}(t_2) ) \left( \sum_{(r, s) \in \Z \x \Z_{> 0}} v_1^r t_1^s \tensor v_2^{-r} t_2^{-s - 1} + \sum_{r \in \Z} v_1^r \tensor v_2^{-r} t_2^{-1} \right)
                                \\
                                = & ( \bar{d}(t_1) \tensor h + h \tensor \bar{d}(t_2) ) \sum_{(r, s) \in \Z \x \Z_{\geq 0}} v_1^r t_1^s \tensor v_2^{-r} t_2^{-s - 1}
                                \\
                                = & ( \bar{d}(t_1) \tensor h + h \tensor \bar{d}(t_2) ) v_2 \1(v_1, v_2) \1^+(t_1, t_2)
                            \end{aligned}
                        $$

                    We can now add the three components together to yield:
                        $$[\bar{\Delta}(ht), \sfr_{\extendedtoroidal^{\positive}}] = [ \bar{\Delta}(ht), \sfr_{\g_{[2]}^{\positive}} + (\sfr_{\z_{[2]}^{\positive}} + \sfr_{\d_{[2]}^{\positive}}) ] =  [h_1 \tensor 1] v_2 \1(v_1, v_2)$$
                    precisely as claimed. 
                    
                    \item Finally, in order to compute $\hat{\delta}^{\positive}(t c_v)$, let us simply note that because:
                        $$\deg t = 1, \deg c_v = -1$$
                    in $\z_{[2]}$ (cf. remark \ref{remark: Z_gradings_on_toroidal_lie_algebras}), we have that:
                        $$\deg t c_v = 0$$
                    and hence:
                        $$\hat{\delta}^{\positive}(t c_v) = 0$$
                    per remark \ref{remark: total_degrees_of_classical_yangian_R_matrices}.
                \end{enumerate}
            \end{proof}

    \subsection{Topological issues}
        \todo[inline]{Not written}
    
    \subsection{Hopf coproducts and classical limits of completed affine Yangians}
        \begin{convention}
            In this subsection, we assume that $\g$ is simply laced, excluding the case where $\g$ is of type $\sfA_1$. 
        \end{convention}

        We begin this subsection by reviewing the construction of what we shall call the \say{Hopf coproduct} $\Delta$ on the Yangian $\formalyangian$ associated to the affine Kac-Moody algebra $\hat{\g}$, as was done in \cite[Sections 4 and 5]{guay_nakajima_wendlandt_affine_yangian_coproduct}. We will then lift this map to the formal Yangian $\formalyangian$ to get another \say{Hopf coproduct} $\Delta_{\hbar}$ thereon. The point of doing this is so that ultimately, we would obtain:
            $$\frac{1}{\hbar}(\Delta_{\hbar} - \Delta_{\hbar}^{\cop}) \pmod{\hbar} \equiv \tilde{\delta}^{\positive}$$
        and hence be able to realise the topological Lie bialgebra $(\toroidal^{\positive}, \tilde{\delta}^{\positive})$ from theorem \ref{theorem: toroidal_lie_bialgebras} as the classical limit of the formal Yangian $\formalyangian$ in some sense (which, let us caution, is not exactly the same as in \cite{etingof_kazhdan_quantisation_1}).
        
        \begin{lemma}[The category $\calO$ for the affine Yangian $\yangian$] \label{lemma: category_O_affine_yangian}
            (Cf. \cite[Theorem 4.9]{guay_nakajima_wendlandt_affine_yangian_coproduct}).
        
            There is a full subcategory of the category of $\yangian$-modules, called the \textbf{category $\calO$}. This category satisfies the following properties:
            \begin{itemize}
                \item Every object $V \in \Ob(\calO)$ is $\hat{\h}$-diagonalisable and with finite-dimensional ($\h$-)weight spaces, and
                \item For every object $V \in \Ob(\calO)$, there exist \textbf{maximal weights} $\lambda_1, ..., \lambda_k \in \hat{\h}^*$ such that, for any $\mu \in \Pi(V)$, one has that:
                    $$\forall 1 \leq i \leq k: \lambda_i - \mu \in \hat{Q}^+$$
            \end{itemize}

            The aforementioned category $\calO$ of $\yangian$ is closed under tensor products over $k$, i.e. if $V_1, V_2$ are any two objects of the category $\calO$, then there will be a $k$-algebra homomorphism:
                $$\Delta_{V_1, V_2}: \yangian \to \End_k(V_1 \tensor_k V_2)$$
            Furthermore, these tensor products are coassociative in the sense that any $k$-vector space isomorphism:
                $$(V_1 \tensor_k V_2) \tensor_k V_3 \xrightarrow[]{\cong} V_1 \tensor_k (V_2 \tensor_k V_3)$$
            between objects $V_1, V_2, V_3 \in \Ob(\calO)$ upgrades to an isomorphism of left-$\yangian$-modules.

            Explicitly, for each $V_1, V_2 \in \Ob(\calO)$, the map $\Delta_{V_1, V_2}$ is given on the generating set\footnote{Using the Levendorskii presentation for $\yangian$, one sees that this generating set suffices.} $\hat{\h} \cup \{T_{i, 1}, X_{i, 0}^{\pm}\}_{i \in \hat{\simpleroots}}$ by:
                $$\forall h \in \hat{\h}: \Delta_{V_1, V_2}(h) := \bar{\Delta}(h)$$
                $$\forall i \in \hat{\simpleroots}: \Delta_{V_1, V_2}(X_{i, 0}^{\pm}) := \bar{\Delta}(X_{i, 0}^{\pm})$$
                $$\forall i \in \hat{\simpleroots}: \Delta_{V_1, V_2}(T_{i, 0}) = \bar{\Delta}(T_{i, 0}) + [H_{i, 0} \tensor 1, \sfr_{ \hat{\g} }^-]$$
            with $\sfr_{ \hat{\g} }^-$ being the Casimir tensor\footnote{This is denoted by $\Omega_+$ in \cite{guay_nakajima_wendlandt_affine_yangian_coproduct} and \cite{guay_nakajima_wendlandt_affine_yangian_vertex_representations_and_PBW}. We opted to designate this the \say{negative} half of the Casimir tensor of $\hat{\g}$ in accordance with the root-degree of the first tensor factor. Also, in \textit{loc. cit.}, the authors considered the Casimir tensor associated to the Kac-Moody pairing on $\hat{\h} \tensor_k \hat{\h} \oplus \hat{\n}^- \hattensor_k \hat{\n}^+$, but we need only the \say{triangular} component since the Cartan component will be killed by $[H_{i, 0} \tensor 1, -]$ anyway.} associated to the non-degenerate Kac-Moody pairing on $\hat{\n}^- \hattensor_k \hat{\n}^+$.
        \end{lemma}
        \begin{remark}
            The category $\calO$ as in lemma \ref{lemma: category_O_affine_yangian} is \textit{not} monoidal, since it lacks a monoidal unit. 
        \end{remark}
        \begin{remark}[Why involve the category $\calO$ ?]
            For a moment, let us pick the root bases $\{ x_{\alpha, k}^{\pm} \}_{(\alpha, k) \in \hat{\Phi}^+ \x \{1, ..., \dim_k (\hat{\g})_{\alpha} \}}$ for $\hat{\n}^{\pm}$ in such a way that they are dual to one another with respect to the Kac-Moody pairing on $\hat{\g}$. In terms of these bases, one can write:
                $$\sfr_{\hat{\g}}^- = \sum_{\alpha \in \hat{\Phi}^+} \sum_{k = 1}^{ \dim_k (\hat{\g})_{\alpha} } x_{\alpha, k}^- \tensor x_{\alpha, k}^+$$
        
            One notable detail is the fact that the sum\footnote{The completed tensor product $\yangian \hattensor_k \yangian$ is only to be understood in the vague sense that it denotes some completion of the algebraic tensor product $\yangian \tensor_k \yangian$ wherein the sum in question converges.}:
                $$\sum_{\alpha \in \hat{\Phi}^+} \sum_{k = 1}^{ \dim_k (\hat{\g})_{\alpha} } x_{\alpha, k}^- \tensor x_{\alpha, k}^+ \in \yangian \hattensor_k \yangian$$
            is infinite \textit{a priori}, since the affine Kac-Moody algebra $\hat{\g}$ has infinitely many positive roots. However, this is precisely why we have restricted our attention down to the category $\calO$: notice that for any $V \in \Ob(\calO)$ and any $\mu \in \Pi(V)$, there exists a natural number $N \in \N$ such that:
                $$\forall \alpha \in \hat{\Phi}^+: r \geq N \implies V_{\mu + r \alpha} \cong 0$$
            From this, one sees that even though it is given by an infinite sum, the operator:
                $$\sum_{\alpha \in \hat{\Phi}^+} \sum_{k = 1}^{ \dim_k (\hat{\g})_{\alpha} } x_{\alpha, k}^- \tensor x_{\alpha, k}^+ \in \End_k(V_1 \tensor_k V_2)$$
            is ultimately locally nilpotent on the vector spaces of the kind $V_1 \tensor_k V_2$, wherein $V_1, V_2 \in \Ob(\calO)$; as such, one sees that the infinite sum above actually becomes finite (and hence converges) after evaluation on elements of the $\yangian$-modules in the category $\calO$, and the maps $\Delta_{V_1, V_2}$ as in lemma \ref{lemma: category_O_affine_yangian} are therefore well-defined. 
        \end{remark}
        \begin{convention}
            If $\fraku$ is a Kac-Moody algebra of some simply laced untwisted affine type and then we will denote by $\hat{\rmY}(\fraku)$ the grading-completion of $\rmY(\fraku)$ with respect to its root grading.
        \end{convention}
        \begin{lemma}[$\yangian$-modules are $\hat{\rmY}(\hat{\g})$-modules] \label{lemma: lifting_representations_of_affine_yangians_to_root_grading_completions}
            (Cf. \cite[Proposition 5.14]{guay_nakajima_wendlandt_affine_yangian_coproduct}) Any left-$\yangian$-module $V$ in the category $\calO$, given by a $k$-algebra homomorphism:
                $$\rho: \yangian \to \End_k(V)$$
            gives rise to a unique left-$\hat{\rmY}(\hat{\g})$-module structure on $V$, which is the same as a $k$-algebra homomorphism:
                $$\hat{\rho}: \hat{\rmY}(\hat{\g}) \to \End_k(V)$$
            fitting into the following commutative diagram of $k$-algebras and homomorphisms between them, where the vertical arrow is the canonical one as in \cite[Section 5, Lemma 5.3]{guay_nakajima_wendlandt_affine_yangian_coproduct}:
                $$
                    \begin{tikzcd}
                	{\hat{\rmY}(\hat{\g})} & {\End_k(V)} \\
                	{\yangian}
                	\arrow[from=2-1, to=1-1]
                	\arrow["{\hat{\rho}}", dashed, from=1-1, to=1-2]
                	\arrow["\rho"', from=2-1, to=1-2]
                    \end{tikzcd}
                $$
        \end{lemma}
        \begin{proposition}[Hopf coproduct on affine Yangians] \label{prop: hopf_coproduct_on_yangians}
            (Cf. \cite[Proposition 5.18]{guay_nakajima_wendlandt_affine_yangian_coproduct}) There exists a $k$-algebra homomorphism:
                $$\Delta: \yangian \to \hat{\rmY}(\hat{\g} \oplus \hat{\g})$$
            satisfying:
                $$\Delta_{V_1, V_2} = (\hat{\rho}_1 \tensor \hat{\rho}_2) \circ \Delta$$
            for any objects $(V_1, \rho_1), (V_2, \rho_2)$ of the category $\calO$ of $\yangian$.
        \end{proposition}
        
        \begin{lemma}[The category $\calO_{\hbar}$ of the formal affine Yangian $\formalyangian$] \label{lemma: category_O_formal_affine_yangian}
            For the formal affine Yangian $\formalyangian$, one can define a category $\calO_{\hbar}$ in the exact same way\footnote{Ultimately, this is because we have that $[h, X_{i, r}^{\pm}] = \pm \alpha_i(h) X_{i, r}^{\pm} \in \rmY_{\hbar}(L) \setminus \hbar\rmY_{\hbar}(L) \cong \rmY^0(L)$ for all Cartan elements $h \in \hat{\h}$.} as how the category $\calO$ was defined for $\yangian$ in lemma \ref{lemma: category_O_affine_yangian}. 

            The category $\calO_{\hbar}$ is closed under $\tensor_k$ (cf. lemma \ref{lemma: category_O_affine_yangian}): for every $V_1, V_2 \in \Ob(\calO_{\hbar})$, there is a corresponding $k$-algebra homomorphism:
                $$\Delta_{V_1, V_2, \hbar}: \formalyangian \to \End_k(V_1 \tensor_k V_2)$$
            given by:
                $$\forall h \in \hat{\h}: \Delta_{V_1, V_2, \hbar}(h) := \bar{\Delta}(h)$$
                $$\forall i \in \hat{\simpleroots}: \Delta_{V_1, V_2, \hbar}(X_{i, 0}^{\pm}) := \bar{\Delta}(X_{i, 0}^{\pm})$$
                $$\forall i \in \hat{\simpleroots}: \Delta_{V_1, V_2, \hbar}(T_{i, 0}) = \bar{\Delta}(T_{i, 0}) + \hbar [H_{i, 0} \tensor 1, \sfr_{ \hat{\g} }^- ]$$
            Furthermore, the tensor products in $\calO_{\hbar}$ are coassociative in the same sense as in lemma \ref{lemma: category_O_affine_yangian}.
        \end{lemma}
            \begin{proof}
                This is a consequence of lemma \ref{lemma: category_O_affine_yangian} and the fact that we have a graded $k$-algebra isomorphism:
                    $$\formalyangian \xrightarrow[]{\cong} \Rees_{\hbar} \yangian$$
                (cf. lemma \ref{lemma: formal_yangians_as_rees_algebras}).
            \end{proof}
        \begin{convention}
            If $\fraku$ is a Kac-Moody algebra of some simply laced untwisted affine type and then we will denote by $\hat{\rmY}_{\hbar}(\fraku)$ the completion of $\rmY_{\hbar}(\fraku)$ with respect to its root grading, with \say{completion} being understood to be in the sense of \cite[Appendix A]{wendlandt_formal_shift_operators_on_yangian_doubles}. Note that this root grading is the same as the one on $\rmY(\fraku)$ due to the fact that:
                $$\forall h \in \hat{\h}: [h, X_{i, r}^{\pm}] = \pm \alpha_i(h) X_{i, r}^{\pm} \in \rmY_{\hbar}(\fraku) \setminus \hbar\rmY_{\hbar}(\fraku) \cong \rmY^0(\fraku)$$
            so the construction of $\hat{\rmY}_{\hbar}(\fraku)$ from $\rmY_{\hbar}(\fraku)$ is the same as that of $\hat{\rmY}(\fraku)$ from $\rmY(\fraku)$.
        \end{convention}
        \begin{lemma}[$\formalyangian$-modules are $\hat{\rmY}_{\hbar}$-modules] \label{lemma: lifting_representations_of_formal_affine_yangians_to_root_grading_completions}
            Any left-$\formalyangian$-module $V$ in the category $\calO$, given by a $k$-algebra homomorphism:
                $$\rho: \formalyangian \to \End_k(V)$$
            gives rise to a unique left-$\hat{\rmY}_{\hbar}$-module structure on $V$, which is the same as a $k$-algebra homomorphism:
                $$\hat{\rho}: \hat{\rmY}_{\hbar} \to \End_k(V)$$
            fitting into the following commutative diagram of $k$-algebras and homomorphisms between them, where the vertical arrow is the canonical inclusion (cf. \cite[Section 5, Lemma 5.3]{guay_nakajima_wendlandt_affine_yangian_coproduct}):
                $$
                    \begin{tikzcd}
                	{\hat{\rmY}_{\hbar}} & {\End_k(V)} \\
                	{\formalyangian}
                	\arrow[from=2-1, to=1-1]
                	\arrow["{\hat{\rho}}", dashed, from=1-1, to=1-2]
                	\arrow["\rho"', from=2-1, to=1-2]
                    \end{tikzcd}
                $$
        \end{lemma}
            \begin{proof}
                
            \end{proof}
        \begin{theorem}[Hopf coproduct on formal affine Yangians] \label{theorem: hopf_coproduct_on_formal_yangians}
            The $k$-algebra homomorphism $\Delta: \yangian \to \hat{\rmY}(\hat{\g} \oplus \hat{\g})$ from proposition \ref{prop: hopf_coproduct_on_yangians} lifts\footnote{... in the sense that $\Delta_{\hbar} \pmod{(\hbar - \hbar_0)} \equiv \Delta$ for any $\hbar_0 \in k^{\x}$.} to a $k$-algebra homomorphism:
                $$\Delta_{\hbar}: \formalyangian \to \hat{\rmY}_{\hbar}(\hat{\g} \oplus \hat{\g})$$
            satisfying:
                $$\Delta_{V_1, V_2, \hbar} = (\hat{\rho}_1 \tensor \hat{\rho}_2) \circ \Delta_{\hbar}$$
            for any $(V_1, \rho_1), (V_2, \rho_2) \in \Ob(\calO_{\hbar})$.
        \end{theorem}
            \begin{proof}
                
            \end{proof}
        
        \begin{theorem}[Toroidal Lie algebras as classical limits of formal affine Yangians] \label{theorem: toroidal_lie_algebras_as_classical_limits_of_formal_affine_yangians}
           The topological Lie bialgebra $(\toroidal^{\positive}, \tilde{\delta}^{\positive})$ from theorem \ref{theorem: toroidal_topological_lie_bialgebras} (see also theorem \ref{theorem: toroidal_lie_bialgebras}) is the classical limit of the formal affine Yangian $\formalyangian$ with the \say{coproduct} $\Delta_{\hbar}$ (as in theorem \ref{theorem: hopf_coproduct_on_formal_yangians}), in the sense that:
                $$\frac{1}{\hbar}( \Delta_{\hbar} - \Delta_{\hbar}^{\cop} ) \equiv \tilde{\delta}^{\positive} \pmod{\hbar}$$
        \end{theorem}
            \begin{proof}
                Before we begin computing, let us make the preliminary observation that $T_{i, 1}(\hbar) \in \formalyangian$ is a lift modulo $\hbar$ of $H_{i, 1} \in \toroidal^{\positive}$:
                    $$T_{i, 1}(\hbar) := H_{i, 1} - \frac12 \hbar H_{i, 0}^2 \equiv H_{i, 1} \pmod{\hbar}$$
                Also, let us note that, we know by lemma \ref{lemma: levendorskii_presentation} that it is enough to only check the value of $\Delta_{\hbar}^{\cop}$ on the generators $H_{i, 0}, X_{i, 0}^{\pm}$, and $T_{i, 1} \equiv H_{i, 1} \pmod{\hbar}$, for all $i \in \hat{\simpleroots}$.
            
                Firstly, from theorem \ref{theorem: hopf_coproduct_on_formal_yangians}, we know that:
                    $$\forall h \in \hat{\h}: \Delta_{\hbar}(h) := \bar{\Delta}(h)$$
                    $$\forall i \in \hat{\simpleroots}: \Delta_{\hbar}(X_{i, 0}^{\pm}) := \bar{\Delta}(X_{i, 0}^{\pm})$$
                    $$\forall i \in \hat{\simpleroots}: \Delta_{\hbar}(T_{i, 0}) = \bar{\Delta}(T_{i, 0}) + [H_{i, 0} \tensor 1, \sfr_{ \hat{\g} }^-]$$
                This tells us that:
                    $$\forall h \in \hat{\h}: \Delta_{\hbar}^{\cop}(h) := \bar{\Delta}(h)$$
                    $$\forall i \in \hat{\simpleroots}: \Delta_{\hbar}^{\cop}(X_{i, 0}^{\pm}) := \bar{\Delta}(X_{i, 0}^{\pm})$$
                    $$\forall i \in \hat{\simpleroots}: \Delta_{\hbar}^{\cop}(T_{i, 0}) = \bar{\Delta}(T_{i, 0}) + [1 \tensor H_{i, 0}, \sfr_{ \hat{\g} }^+]$$
                with $\sfr_{\hat{\g}}^-$ being the Casimir tensor associated to the non-degenerate Kac-Moody pairing on $\hat{\n}^- \hattensor_k \hat{\n}^+$.

                It is then trivial that:
                    $$\frac{1}{\hbar}( \Delta_{\hbar} - \Delta_{\hbar}^{\cop} )(X) = 0$$
                for:
                    $$X \in \hat{\h} \cup \{X_{i, 0}^{\pm}\}_{i \in \hat{\simpleroots}}$$
                which implies that:
                    $$\frac{1}{\hbar}( \Delta_{\hbar} - \Delta_{\hbar}^{\cop} )(X) \equiv \tilde{\delta}^{\positive}(X) \pmod{\hbar}$$
                which is because it is known from theorem \ref{theorem: toroidal_lie_bialgebras} that:
                    $$\tilde{\delta}^{\positive}(X) = 0$$
                whenever $\deg X = 0$, which is the case here.
                
                Now, let us verify that:
                    $$\frac{1}{\hbar}(\Delta_{\hbar} - \Delta_{\hbar}^{\cop})(T_{i, 1}) \equiv \tilde{\delta}^{\positive}(H_{i, 1})$$
                It is not hard to see that\footnote{One can prove this by e.g. picking the root bases foor $\hat{\n}^{\pm}$.}:
                    $$[1 \tensor H_{i, 0}, \sfr_{ \hat{\g} }^+] = -[H_{i, 0} \tensor 1, \sfr_{ \hat{\g} }^+]$$
                which tells us that:
                    $$\frac{1}{\hbar}( \Delta_{\hbar} - \Delta_{\hbar}^{\cop} )(T_{i, 1}) = [H_{i, 0} \tensor 1, \sfr_{ \hat{\g} }^- + \sfr_{ \hat{\g} }^+]$$
                Since $H_{i, 0}$ commutes with every element of $\hat{\h}$, we can equivalently rewrite the above into:
                    $$\frac{1}{\hbar}( \Delta_{\hbar} - \Delta_{\hbar}^{\cop} )(T_{i, 1}) = [H_{i, 0} \tensor 1, \sfr_{\hat{\h}} + \sfr_{ \hat{\g} }^- + \sfr_{ \hat{\g} }^+] = [H_{i, 0} \tensor 1, \sfr_{ \hat{\g} }]$$
                wherein $\sfr_{\hat{\h}}$ is the Casimir element associated to the Kac-Moody pairing on $\hat{\h} \tensor_k \hat{\h}$. 

                We know from theorem \ref{theorem: toroidal_lie_bialgebras} that:
                    $$\tilde{\delta}^{\positive}(H_{i, 1}) = [ H_{i, 0} \tensor 1, \sfr_{\g} v_2 \1(v_1, v_2) ]$$
                so we will be done if we can show that:
                    $$[H_{i, 0} \tensor 1, \sfr_{ \hat{\g} }] = [ H_{i, 0} \tensor 1, \sfr_{\g} v_2 \1(v_1, v_2) ]$$
                From the fact that:
                    $$\hat{\g} \cong \g \oplus \z \oplus \d \cong \g \oplus k c_v \oplus k D_{0, -1}$$
                we infer that:
                    $$\sfr_{ \hat{\g} } = \sfr_{\g} v_2 \1(v_1, v_2) + \sfr_{\z} + \sfr_{\d} = \sfr_{\g} v_2 \1(v_1, v_2) + K_{0, -1} \tensor D_{0, -1} + D_{0, -1} \tensor K_{0, -1}$$
                wherein $\sfr_{\z}, \sfr_{\d}$ respectively denote the Casimir elements corresponding to the Kac-Moody form on $\z \tensor_k \d$ and on $\d \tensor_k \z$ respectively. The element $K_{0, -1} \in \hat{\g} := \g \oplus \z$ is central and therefore commutes with $H_{i, 0}$, which implies that:
                    $$[H_{i, 0} \tensor 1, \sfr_{\z}] = [H_{i, 0} \tensor 1, K_{0, -1} \tensor D_{0, -1}] = 0$$
                At the same time, we also know that $D_{0, -1}$ acts as $\id_{\g} \tensor \left(-v \frac{d}{dv}\right)$ on $\g$ and hence as zero on the elements of $\g$ (i.e. degree-$0$ elements of $\g$), and so:
                    $$[H_{i, 0} \tensor 1, \sfr_{\d}] = [H_{i, 0} \tensor 1, D_{0, -1} \tensor K_{0, -1}] = 0$$
                as well. As such, we have demonstrated that:
                    $$[H_{i, 0} \tensor 1, \sfr_{ \hat{\g} }] = [ H_{i, 0} \tensor 1, \sfr_{\g} v_2 \1(v_1, v_2) ]$$
                as we sought to. As mentioned above, this allows us to conclude that:
                    $$\frac{1}{\hbar}( \Delta_{\hbar} - \Delta_{\hbar}^{\cop} )(T_{i, 1}) \equiv \tilde{\delta}^{\positive}(H_{i, 1}) \pmod{\hbar}$$
            \end{proof}