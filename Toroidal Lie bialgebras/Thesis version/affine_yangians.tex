\section{Yangians associated to Kac-Moody algebras} \label{section: yangians_of_kac_moody_algebras}
    \subsection{Presentations for Yangians}
        \begin{definition}[Formal Yangians associated to symmetrisable Kac-Moody algebras] \label{def: formal_yangians_associated_to_symmetrisable_kac_moody_algebras}
            The \textbf{formal Yangian associated to the derived subalgebra of $\fraku$} is a certain $\bbC$-algebra:
                $$\rmY_{\hbar}(\fraku')$$
            generated by the set:
                $$\{X_{i, r}^{\pm}, H_{i, r}\}_{(i, r) \in \simpleroots_{\fraku} \x \Z_{\geq 0}}$$
            whose elements are subjected to the following relations, given for all $(i, r), (j, s) \in \simpleroots_{\fraku} \x \Z_{\geq 0}$:
                $$[H_{i, r}, H_{j, s}] = 0$$
                $$[H_{i, 0}, X_{j, s}^{\pm}] = \pm a_{ij} X_{j, s}^{\pm}$$
                $$[X_{i, r}^+, X_{j, s}^-] = \delta_{i, j} H_{i, r + s}$$
                $$[H_{i, r + 1}, X^{\pm}_{j, s}] - [H_{i, r}, X^{\pm}_{j, s + 1}] = \frac12 \hbar a_{ij} \{H_{i, r}, X^{\pm}_{j, s}\}$$
                $$[X^{\pm}_{i, r + 1}, X^{\pm}_{j, s}] - [X^{\pm}_{i, r}, X^{\pm}_{j, s + 1}] = \frac12 \hbar a_{ij} \{X^{\pm}_{i, r}, X^{\pm}_{j, s}\}$$
                $$\sum_{ \sigma \in S_{1 - c_{ij}} } \ad(X_{ i, r_{\sigma(1 - c_{ij})} }) \cdot ... \cdot \ad(X_{ i, r_{\sigma(2)} }) \cdot \ad(X_{ i, r_{\sigma(1)} }) \cdot X_{j, s}^{\pm} = 0$$
                
            To obtain the \textbf{formal Yangian associated to the Kac-Moody algebra $\fraku$} itself:
                $$\rmY_{\hbar}(\fraku)$$
            we must enlarge the generating set to:
                $$\{X_{i, r}^{\pm}, H_{i, r}\}_{(i, r) \in \simpleroots_{\fraku} \x \Z_{\geq 0}} \cup \fraku_0$$
            and then impose the following additional relations, given for all $h \in \fraku_0$ and all $(j, s) \in \simpleroots_{\fraku} \x \Z_{\geq 0}$:
                $$[h, H_{j, s}] = 0$$
                $$[h, X_{j, s}^{\pm}] = \pm \alpha_j(h) X_{j, s}^{\pm}$$
        \end{definition}
        \begin{remark}[Formal Yangians as \texorpdfstring{$\bbC[\hbar]$}{}-algebras and the $\Z_{\geq 0}$-grading] \label{remark: positive_Z_grading_on_formal_yangians}
            Equivalently, the formal Yangians $\rmY_{\hbar}(\fraku')$ and $\rmY_{\hbar}(\fraku)$ from definition \ref{def: formal_yangians_associated_to_symmetrisable_kac_moody_algebras} can be regarded as $\bbC[\hbar]$-algebras generated by the sets:
                $$\{X_{i, r}^{\pm}, H_{i, r}\}_{(i, r) \in \simpleroots_{\fraku} \x \{0, 1\}}$$
            and:
                $$\{X_{i, r}^{\pm}, H_{i, r}\}_{(i, r) \in \simpleroots_{\fraku} \x \{0, 1\}} \cup \fraku_0$$
            respectively, whose elements are subjected to the same relations.

            In any event, these algebras carry natural $\Z_{\geq 0}$-gradings given by:
                $$\deg \hbar = 1$$
                $$\forall (i, r) \in \simpleroots_{\fraku} \x \Z_{\geq 0}: \deg X^{\pm}_{i, r} = \deg H_{i, r} = r$$
            Later on, this grading will be used for establishing formal Yangians as Rees algebras for certain cases of the Kac-Moody algebra $\fraku$.
        \end{remark}
        \begin{remark}[The degree-$0$ components of formal Yangians] \label{remark: universal_enveloping_algebras_inside_yangians}
            When regarded as a $\bbC$-algebra, both $\rmY_{\hbar}(\fraku')$ and $\rmY_{\hbar}(\fraku)$ admit the universal enveloping algebra $\rmU(\fraku)$ as a $\bbC$-subalgebra. In particular, $\rmU(\fraku)$ is isomorphic to the degree-$0$ graded component of $\rmY_{\hbar}(\fraku')$ and $\rmY_{\hbar}(\fraku)$; this can be shown by constructing an algebra homomorphisms:
                $$\rmU(\fraku) \to \rmY_{\hbar}(\fraku'), \rmU(\fraku) \to \rmY_{\hbar}(\fraku)$$
            given on generators by:
                $$x_i^{\pm} \mapsto X_{i, 0}^{\pm}, h_i \mapsto H_{i, 0}$$
            and then showing that these maps are injective. 

            From now on, this fact will be used without explicit mention.
        \end{remark}

        Under certain mild technical restrictions, it is possible to give a presentation for the formal Yangians $\rmY_{\hbar}(\fraku)$ in terms only of generators of degrees $0$ and $1$, i.e. elements of the set:
            $$\{X_{i, r}^{\pm}, H_{i, r}\}_{(i, r) \in \simpleroots_{\fraku} \x \{0, 1\}} \cup \fraku_0$$
        This can be viewed as a generalisation of the work \cite{levendorskii_finite_type_yangians_presentation} of Levendorskii, applicable to the case where $\fraku$ is of finite-type (e.g. $\fraku \cong \g$), and is very useful for performing computations with the generators of $\rmY_{\hbar}(\fraku)$.
        
        \begin{convention}[An auxiliary generator for (formal) Yangians]
            Henceforth, let us write:
                $$T_{i, 1}(\hbar) := H_{i, 1} - \frac12 \hbar H_{i, 0}^2$$
                $$T_{i, 1} := T_{i, 1}(1) = H_{i, 1} - \frac12 H_{i, 0}^2$$
        \end{convention}
        
        The hypotheses of the following lemma are satisfied at least when $\g$ is either of finite type or of affine type, save for the types $\sfA_1^{(1)}$ and $\sfA_1^{(2)}$.
        \begin{lemma}[A Levendorskii-type presentation for Yangians of Kac-Moody algebras] \label{lemma: levendorskii_presentation}
            \cite[Theorem 2.13]{guay_nakajima_wendlandt_affine_yangian_coproduct} Choose a total ordering on the set of simple roots $\simpleroots_{\fraku}$ and suppose that the Cartan matrix $C := (c_{ij})_{i, j \in \simpleroots_{\fraku}}$ of the Kac-Moody algebra $\fraku$ is such that, for any $i < j \in \simpleroots_{\fraku}$ (with respect to some choice of total ordering on $\simpleroots_{\fraku}$) the following $2 \x 2$ submatrix is invertible:
                $$
                    \begin{pmatrix}
                        c_{ii} & c_{ij}
                        \\
                        c_{ji} & c_{ji}
                    \end{pmatrix}
                $$
            
            The formal Yangian $\rmY_{\hbar}(\fraku)$ of $\fraku$ will then be isomorphic to the associative $\bbC$-algebra generated by the set:
                $$\{ H_{i, r}, X_{i, r}^{\pm} \}_{(i, r) \in \simpleroots_{\fraku} \x \Z_{\geq 0}}$$
            whose elements are subjected to the following relations\footnote{... and it is understood that the elements $H_{i, 0} = h_i, X_{i, 0}^{\pm} = e_i^{\pm}$ satisfy the Chevalley-Serre relations defining $\fraku$; cf. \cite[Chapter 1]{kac_infinite_dimensional_lie_algebras}.}:
                $$H_{i, 0} = h_i, X_{i, 0}^{\pm} = x_i^{\pm}$$
                $$[ H_{i, r}, H_{j, s} ] = 0$$
                $$[ H_{i, 0}, X_{j, s}^{\pm} ] = \pm a_{ij} X_{j, s}^{\pm}$$
                $$[ X_{i, r}^+, X_{j, s}^- ] = \pm \delta_{ij} H_{i, r + s}$$
                $$\left[ T_{i, 1}(\hbar), X_{j, 0}^{\pm} \right] = \pm \hbar a_{ij} X_{j, 1}^{\pm}$$
                $$[ X_{i, 1}^{\pm}, X_{j, 0}^{\pm} ] - [ X_{i, 0}^{\pm}, X_{j, 1}^{\pm} ] = \pm \frac12 \hbar a_{ij} \{X_{i, 0}^{\pm}, X_{j, 0}^{\pm}\}$$
        \end{lemma}

        Let us now consider what might occur when we specialise the variable $\hbar$ to a specific value $\hbar_0 \in k$.

        Firstly, let us consider the case $\hbar_0 \not = 0$, which is much simpler than the case $\hbar_0 = 0$. 
        \begin{definition}[Yangians associated to symmetrisable Kac-Moody algebras] \label{def: yangians_associated_to_symmetrisable_kac_moody_algebras}
            By specialising $\hbar$ to some $\hbar_0 \in k^{\x}$, one obtains the \textbf{Yangian} associated to $\fraku'$ and $\fraku$, respectively:
                $$\rmY(\fraku') := \rmY_{\hbar}(\fraku')/(\hbar - \hbar_0)$$
                $$\rmY(\fraku) := \rmY_{\hbar}(\fraku)/(\hbar - \hbar_0)$$
        \end{definition}
        \begin{definition}[Rees algebras] \label{def: rees_algebras}
            Let:
                $$A := \bigoplus_{r \in \Z_{\geq 0}} A_r$$
            be a $\Z_{\geq 0}$-filtered $\bbC$-algebra. The \textbf{Rees algebra} associated to $A$ is then the $\bbC[\hbar]$-algebra given by:
                $$\Rees_{\hbar} A := \bigoplus_{r \in \Z_{\geq 0}} A_r \hbar^r$$
        \end{definition}
        \begin{lemma}[Basic properties of Rees algebras] \label{lemma: basic_properties_of_rees_algebras}
            \cite[Exercise I.9.5]{kassel_quantum_groups} Let:
                $$A := \bigoplus_{r \in \Z_{\geq 0}} A_r$$
            be a $\Z_{\geq 0}$-filtered $\bbC$-algebra. Also, fix an arbitrary element $\hbar_0 \in k$.
            \begin{enumerate}
                \item There are $\bbC$-algebra isomorphisms:
                    $$
                        (\Rees_{\hbar} A)/(\hbar - \hbar_0) \cong
                        \begin{cases}
                            \text{$A$ if $\hbar_0 \not = 0$}
                            \\
                            \text{$\gr A$ if $\hbar_0 = 0$}
                        \end{cases}
                    $$
                \item Suppose that the associated graded algebra $\gr A$ is generated by a set of homogeneous elements:
                    $$a_0, a_1, a_2, ...$$
                respectively of degree $r_0 := 0, r_1, r_2, ... \in \Z_{\geq 0}$, then $\Rees_{\hbar} A$ will be generated by the elements:
                    $$a_0, a_1 \hbar^{r_1}, a_2 \hbar^{r_2}, ...$$
            \end{enumerate}
        \end{lemma}
        \begin{convention}
            In light of lemma \ref{lemma: basic_properties_of_rees_algebras}, it is typically to just make the choice:
                $$\hbar_0 := 1$$
            and hence work with:
                $$\rmY(\fraku) := \rmY_1(\fraku)$$
        \end{convention}
        \begin{remark}[Yangians are $\Z_{\geq 0}$-filtered] \label{remark: positive_Z_filtrations_on_yangians}
            Now, the Yangian $\rmY(\fraku)$ is no longer graded but just $\Z_{\geq 0}$-filtered. The (ascending) filtration in question is given by:
                $$\deg X_{i, r}^{\pm} = \deg H_{i, r} = r$$
                
            However, whether this filtration gives - via the Rees algebra construction - a $\Z_{\geq 0}$-grading agreeing with the one on formal Yangians is a rather subtle issue, with no complete solution in general. Another difficult problem is that of explicitly computing the associated graded algebra:
                $$\gr \rmY(\fraku)$$
            When $\fraku$ is of finite type, it is known that:
                $$\gr \rmY(\fraku) \cong \rmU(\fraku[t])$$
            (in fact, Drinfeld seemed to have constructed the Yangian of a finite-dimensional simple Lie algebra specifically so that this would be true) but when $\fraku$ is a general symmetrisable Kac-Moody algebra, what the algebra $\gr \rmY(\fraku)$ is explicitly is still not known.
            
            That said, partial answers to both of these problems have been obtained in \cite[Section 6]{guay_nakajima_wendlandt_affine_yangian_vertex_representations_and_PBW}, where the authors focused on the case where $\fraku$ is of a simply laced affine type. We defer the discussion of the details of these problems to subsection \ref{subsection: yangians_as_deformations}, where the special cases relevant to our goals are examined. 
        \end{remark}

        \begin{remark}
            When $\fraku$ is of finite type, it is also known that the formal Yangian $\rmY_{\hbar}(\fraku)$ (hence also the Yangian $\rmY(\fraku)$) carries a Hopf bialgebra structure\footnote{Recall also that Hopf bialgebra structures, if they exist, are unique up to isomorphisms.}. For organisational purposes, we would like the contents of this section to remain entirely algebraic, as opposed to algebraic and coalgebraic. As such, we will not mention this fact again until subsection \ref{subsection: manin_triples_and_quantisations_of_lie_bialgebras},where we will discuss this Hopf structures as an instance of the much more general fact that Lie bialgebras admit so-called \say{quantisations}. 
        \end{remark}

    \subsection{Coproducts on Yangians}
         \todo[inline]{Not done}
         
         The following is sometimes called \textbf{Drinfeld's J-presentation}.
        \begin{lemma}[Drinfeld's presentation for finite-type Yangians] \label{lemma: drinfeld_presentation_for_finite_type_yangians}
            Set:
                $$n := \dim_{\bbC} \g$$
        
            Interestingly, the finite-type formal Yangian:
                $$\rmY_{\hbar}(\g)$$
            admits a presentation so that it is isomorphic to the $\bbC[\hbar]$-algebra generated by the set:
                $$\{ x_{\lambda}, y_{\lambda} \}_{1 \leq \lambda \leq n}$$
            whose elements subjected to the following relations:
                $$[ x_{\lambda}, x_{\mu} ] = \sum_{1 \leq \lambda \leq n} c_{\lambda \mu \nu} x_{\nu}, [ x_{\lambda}, y_{\mu} ] = \sum_{1 \leq \lambda \leq n} c_{\lambda \mu \nu} y_{\nu}$$
                $$[ y_{\lambda}, [y_{\mu}, x_{\nu}] ] - [ x_{\lambda}, [y_{\mu}, y_{\nu}] ] = \hbar^2 \sum_{1 \leq \alpha, \beta, \gamma \leq n} a_{\lambda \mu \nu \alpha \beta \gamma} \{ x_{\alpha}, x_{\beta}, x_{\gamma} \}$$
                $$[ [y_{\lambda}, y_{\mu}], [x_r, x_s] ] + [ [y_r, y_s], [x_{\lambda}, x_{\mu}] ] = \hbar^2 \sum_{1 \leq \alpha, \beta, \gamma, \nu \leq n} ( a_{\lambda \mu \nu \alpha \beta \gamma} c_{r s \nu} + a_{r s \nu \alpha \beta \gamma} c_{\lambda \mu \nu} ) \{ x_{\alpha}, x_{\beta}, x_{\gamma} \}$$
            wherein $c_{\cdot \cdot \cdot}$ are the structural constants of $\g$, and:
                $$a_{\lambda \mu \nu \alpha \beta \gamma} = \frac{1}{24} \sum_{1 \leq i, j, k \leq n} c_{\lambda \alpha i} c_{\mu \beta j} c_{\nu \gamma k} c_{i j k}$$
            and we set:
                $$\deg x_{\lambda} := 0, \deg y_{\lambda} := 1$$
            for all $1 \leq \lambda \leq n$.
            
            The isomorphism in question is given by:
                $$\varphi(h_i) = d_{ii}^{-1} H_{i, 0}, \varphi(h_i t) = d_{ii}^{-1} H_{i, 0} + \varphi(v_i)$$
                $$\varphi(x_i^{\pm}) = d_{ii}^{-1} E_{i, 0}^{\pm}, \varphi(h_i t) = d_{ii}^{-1} H_{i, 0} + \varphi(w_i^{\pm})$$
            wherein:
                $$v_i := -\frac12 d_{ii} h_i^2 + \frac14 \sum_{\alpha \in \Phi^+} \height(\alpha)^2 d_{ii}^{-1} \alpha(h_i) \{x_{\alpha}^+, x_{\alpha}^-\}$$
                $$w_i^{\pm} := -\frac12 d_{ii} \{x_i^{\pm}, h_i\} + \frac14 \sum_{\alpha \in \Phi^+} \height(\alpha)^2 d_{ii}^{-1} \alpha(h_i) \{[x_i^{\pm}, x_{\alpha}^{\pm}], x_{\alpha}^{\mp}\}$$
            with choices of roots vectors $x_{\alpha}^{\pm} \in \g_{\pm \alpha}$ such that $(x_{\alpha}^-, x_{\alpha}^+)_{\g} = 1$. One notes also that $\varphi$ respects the $\Z_{\geq 0}$-grading on both algebras.
        \end{lemma}
        \begin{theorem}[Finite-type Yangian Hopf structure] \label{theorem: finite_type_yangian_hopf_structure} 
            There is a $\Z_{\geq 0}$-graded Hopf $\bbC[\hbar]$-algebra structure:
                $$(\Delta_{\hbar}, S_{\hbar}, \e_{\hbar})$$
            on $\rmY_{\hbar}(\g)$ given by:
                $$\Delta_{\hbar}(x_{\lambda}) := \bar{\Delta}(x_{\lambda}), \Delta_{\hbar}(y_{\lambda}) := \bar{\Delta}(y_{\lambda}) + \frac12 \hbar [x_{\lambda} \tensor 1, \sfr_{\g}]$$
                $$S_{\hbar}(x_{\lambda}) = -x_{\lambda}, S_{\hbar}(y_{\lambda}) = -y_{\lambda} + \frac14 c_{} x_{\lambda}$$
                $$\e_{\hbar}(x_{\lambda}) = \e_{\hbar}(y_{\lambda}) = 0$$
            with $c$ being the eigenvalue of $\ad(\sfr_{\g})$. 
        \end{theorem}
        
        \begin{lemma}[The category $\calO$ for the affine Yangian $\rmY(\hat{\g})$] \label{lemma: category_O_affine_yangian}
            (Cf. \cite[Theorem 4.9]{guay_nakajima_wendlandt_affine_yangian_coproduct}).
        
            There is a full subcategory of the category of $\rmY(\hat{\g})$-modules, called the \textbf{category $\calO$}. This category satisfies the following properties:
            \begin{itemize}
                \item Every object $V \in \Ob(\calO)$ is $\hat{\h}$-diagonalisable and with finite-dimensional ($\hat{\h}$-)weight spaces, and
                \item For every object $V \in \Ob(\calO)$, there exist \textbf{maximal weights} $\lambda_1, ..., \lambda_k \in \hat{\h}^*$ such that, for any $\mu \in \Pi(V)$, one has that:
                    $$\forall 1 \leq i \leq k: \lambda_i - \mu \in \hat{Q}^+$$
            \end{itemize}

            The aforementioned category $\calO$ of $\rmY(\hat{\g})$ is closed under tensor products over $\bbC$, i.e. if $V_1, V_2$ are any two objects of the category $\calO$, then there will be a $\bbC$-algebra homomorphism:
                $$\Delta_{V_1, V_2}: \rmY(\hat{\g}) \to \End_k(V_1 \tensor_{\bbC} V_2)$$
            Furthermore, these tensor products are coassociative in the sense that any $\bbC$-vector space isomorphism:
                $$(V_1 \tensor_{\bbC} V_2) \tensor_{\bbC} V_3 \xrightarrow[]{\cong} V_1 \tensor_{\bbC} (V_2 \tensor_{\bbC} V_3)$$
            between objects $V_1, V_2, V_3 \in \Ob(\calO)$ upgrades to an isomorphism of left-$\rmY(\hat{\g})$-modules.

            Explicitly, for each $V_1, V_2 \in \Ob(\calO)$, the map $\Delta_{V_1, V_2}$ is given on the generating set\footnote{Using the Levendorskii presentation for $\rmY(\hat{\g})$, one sees that this generating set suffices.} $\hat{\h} \cup \{T_{i, 1}, X_{i, 0}^{\pm}\}_{i \in \hat{\simpleroots}}$ by:
                $$\forall h \in \hat{\h}: \Delta_{V_1, V_2}(h) := \bar{\Delta}(h)$$
                $$\forall i \in \hat{\simpleroots}: \Delta_{V_1, V_2}(X_{i, 0}^{\pm}) := \bar{\Delta}(X_{i, 0}^{\pm})$$
                $$\forall i \in \hat{\simpleroots}: \Delta_{V_1, V_2}(T_{i, 0}) = \bar{\Delta}(T_{i, 0}) + [H_{i, 0} \tensor 1, \sfr_{ \hat{\g} }^-]$$
            with $\sfr_{ \hat{\g} }^-$ being the Casimir tensor\footnote{This is denoted by $\Omega_+$ in \cite{guay_nakajima_wendlandt_affine_yangian_coproduct} and \cite{guay_nakajima_wendlandt_affine_yangian_vertex_representations_and_PBW}. We opted to designate this the \say{negative} half of the Casimir tensor of $\hat{\g}$ in accordance with the root-degree of the first tensor factor. Also, in \textit{loc. cit.}, the authors considered the Casimir tensor associated to the Kac-Moody pairing on $\hat{\h} \tensor_{\bbC} \hat{\h} \oplus \hat{\n}^- \hattensor_k \hat{\n}^+$, but we need only the \say{triangular} component since the Cartan component will be killed by $[H_{i, 0} \tensor 1, -]$ anyway.} associated to the non-degenerate Kac-Moody pairing on $\hat{\n}^- \hattensor_k \hat{\n}^+$.
        \end{lemma}
        \begin{remark}[Why involve the category $\calO$ ?]
            For a moment, let us pick the root bases $\{ x_{\alpha, k}^{\pm} \}_{(\alpha, k) \in \hat{\Phi}^+ \x \{1, ..., \dim_{\bbC} (\hat{\g})_{\alpha} \}}$ for $\hat{\n}^{\pm}$ in such a way that they are dual to one another with respect to the Kac-Moody pairing on $\hat{\g}$. In terms of these bases, one can write:
                $$\sfr_{\hat{\g}}^- = \sum_{\alpha \in \hat{\Phi}^+} \sum_{k = 1}^{ \dim_{\bbC} (\hat{\g})_{\alpha} } x_{\alpha, k}^- \tensor x_{\alpha, k}^+$$
        
            One notable detail is the fact that the sum\footnote{The completed tensor product $\rmY(\hat{\g}) \hattensor_k \rmY(\hat{\g})$ is only to be understood in the vague sense that it denotes some completion of the algebraic tensor product $\rmY(\hat{\g}) \tensor_{\bbC} \rmY(\hat{\g})$ wherein the sum in question converges.}:
                $$\sum_{\alpha \in \hat{\Phi}^+} \sum_{k = 1}^{ \dim_{\bbC} (\hat{\g})_{\alpha} } x_{\alpha, k}^- \tensor x_{\alpha, k}^+ \in \rmY(\hat{\g}) \hattensor_k \rmY(\hat{\g})$$
            is infinite \textit{a priori}, since the affine Kac-Moody algebra $\hat{\g}$ has infinitely many positive roots. However, this is precisely why we have restricted our attention down to the category $\calO$: notice that for any $V \in \Ob(\calO)$ and any $\mu \in \Pi(V)$, there exists a natural number $N \in \Z_{\geq 0}$ such that:
                $$\forall \alpha \in \hat{\Phi}^+: r \geq N \implies V_{\mu + r \alpha} \cong 0$$
            From this, one sees that even though it is given by an infinite sum, the operator:
                $$\sum_{\alpha \in \hat{\Phi}^+} \sum_{k = 1}^{ \dim_{\bbC} (\hat{\g})_{\alpha} } x_{\alpha, k}^- \tensor x_{\alpha, k}^+ \in \End_k(V_1 \tensor_{\bbC} V_2)$$
            is ultimately locally nilpotent on the vector spaces of the kind $V_1 \tensor_{\bbC} V_2$, wherein $V_1, V_2 \in \Ob(\calO)$; as such, one sees that the infinite sum above actually becomes finite (and hence converges) after evaluation on elements of the $\rmY(\hat{\g})$-modules in the category $\calO$, and the maps $\Delta_{V_1, V_2}$ as in lemma \ref{lemma: category_O_affine_yangian} are therefore well-defined. 
        \end{remark}
        \begin{convention}
            If $\fraku$ is a Kac-Moody algebra of some simply laced untwisted affine type and then we will denote by $\hat{\rmY}(\fraku)$ the grading-completion of $\rmY(\fraku)$ with respect to its root grading.
        \end{convention}
        \begin{lemma}[$\rmY(\hat{\g})$-modules are $\hat{\rmY}(\hat{\g})$-modules] \label{lemma: lifting_representations_of_affine_yangians_to_root_grading_completions}
            (Cf. \cite[Proposition 5.14]{guay_nakajima_wendlandt_affine_yangian_coproduct}) Any left-$\rmY(\hat{\g})$-module $V$ in the category $\calO$, given by a $\bbC$-algebra homomorphism:
                $$\rho: \rmY(\hat{\g}) \to \End_k(V)$$
            gives rise to a unique left-$\hat{\rmY}(\hat{\g})$-module structure on $V$, which is the same as a $\bbC$-algebra homomorphism:
                $$\hat{\rho}: \hat{\rmY}(\hat{\g}) \to \End_k(V)$$
            fitting into the following commutative diagram of $\bbC$-algebras and homomorphisms between them, where the vertical arrow is the canonical one as in \cite[Section 5, Lemma 5.3]{guay_nakajima_wendlandt_affine_yangian_coproduct}:
                $$
                    \begin{tikzcd}
                	{\hat{\rmY}(\hat{\g})} & {\End_k(V)} \\
                	{\rmY(\hat{\g})}
                	\arrow[from=2-1, to=1-1]
                	\arrow["{\hat{\rho}}", dashed, from=1-1, to=1-2]
                	\arrow["\rho"', from=2-1, to=1-2]
                    \end{tikzcd}
                $$
        \end{lemma}
        \begin{theorem}[Hopf coproduct on affine Yangians] \label{theorem: hopf_coproduct_on_yangians}
            (Cf. \cite[Proposition 5.18]{guay_nakajima_wendlandt_affine_yangian_coproduct}) There exists a $\bbC$-algebra homomorphism:
                $$\Delta: \rmY(\hat{\g}) \to \hat{\rmY}(\hat{\g} \oplus \hat{\g})$$
            satisfying:
                $$\Delta_{V_1, V_2} = (\hat{\rho}_1 \tensor \hat{\rho}_2) \circ \Delta$$
            for any objects $(V_1, \rho_1), (V_2, \rho_2)$ of the category $\calO$ of $\rmY(\hat{\g})$.
        \end{theorem}

        \begin{theorem} \label{theorem: hopf_coproduct_on_formal_yangians}
            \todo[inline]{How do we lift $\Delta$ to $\rmY_{\hbar}(\hat{\g})$ ? Does this depend on PBW ?}
        \end{theorem}

    \subsection{Finite-type and affine Yangians as deformations; PBW theorems} \label{subsection: yangians_as_deformations}
        Let us now discuss how, when $\fraku$ is either of a finite type or a simply laced affine type, it can be shown that $\rmY_{\hbar}(\fraku)$ is the flat deformation of the universal enveloping algebra of a certain polynomial Lie algebra. In the former case, we will see that the Lie algebra in question is:
            $$\g[t]$$
        whereas in the latter case, it is:
            $$\toroidal^{\positive} := \uce(\g[v^{\pm 1}, t])$$
        Both ought to be viewed as UCEs, and it just so happens that the Lie algebra $\g[t]$ admits itself as a UCE (cf. example \ref{example: affine_lie_algebras_centres}).

        Without any loss of generality, we can assume that the Cartan matrix of $\fraku$ is indecomposable. 
    
        \begin{definition}[Graded and PBW deformations] \label{def: graded_and_PBW_deformations}
            Fix an ($\Z_{\geq 0}$-graded) algebra $U$ over a field $k$. An \textbf{($\Z_{\geq 0}$-graded) deformation} of such an algebra $U$ is then a ($\Z_{\geq 0}$-graded) $\bbC[\hbar]$-algebra $Y_{\hbar}$, free as a $\bbC[\hbar]$-module, and such that:
                $$Y_{\hbar}/\hbar Y_{\hbar} \cong U$$
            Now, fix some $\hbar_0 \in k^{\x}$. The algebra:
                $$Y_{\hbar_0} := Y_{\hbar}/(\hbar - \hbar_0)$$
            is then called the \textbf{PBW deformation} of $U$ at $\hbar_0$. 
        \end{definition}
        \begin{lemma}[Rees algebras as graded deformations] \label{lemma: rees_algebras_as_graded_deformations}
            Let $Y$ be a $\Z_{\geq 0}$-filtered algebra. Then, the Rees algebra $\Rees_{\hbar} Y$ will be a $\Z_{\geq 0}$-graded flat deformation of the associated graded algebra $\gr Y$, while $Y$ itself will be a PBW deformation of $\gr Y$.
        \end{lemma}

        \begin{theorem}[PBW bases for finite-type Yangians] \label{theorem: finite_type_yangians_PBW}
            \cite{levendorskii_finite_type_yangians_PBW} One obtains a $\Z_{\geq 0}$-graded algebra isomorphism:
                $$\rmU(\g[t]) \hookrightarrow \rmY(\g) \to \gr \rmY(\g)$$
            by composing the natural embedding of $\rmU(\g[t])$ into $\rmY(\g)$ as the degreee-$0$ graded component (cf. remark \ref{remark: universal_enveloping_algebras_inside_yangians}), with the canonical quotient map $\rmY(\g) \to \gr \rmY(\g)$.
        \end{theorem}
        
        \begin{lemma} \label{lemma: affine_yangian_PBW_surjectivity}
            One obtains a surjective $\Z_{\geq 0}$-graded algebra homomorphism:
                $$\rmU(\toroidal^{\positive}) \hookrightarrow \rmY(\hat{\g}) \to \gr \rmY(\hat{\g})$$
            by composing the natural embedding of $\rmU(\toroidal^{\positive})$ into $\rmY(\hat{\g})$ as the degreee-$0$ graded component (cf. remark \ref{remark: universal_enveloping_algebras_inside_yangians}), with the canonical quotient map $\rmY(\hat{\g}) \to \gr \rmY(\hat{\g})$.
        \end{lemma}
        \begin{theorem}[PBW bases for simply laced affine Yangians] \label{theorem: affine_yangian_PBW}
            When $\hat{\g}$ is simply laced, the natural algebra homomorphism composition:
                $$\rmU(\toroidal^{\positive}) \hookrightarrow \rmY(\hat{\g}) \to \gr \rmY(\hat{\g})$$
            as in lemma \ref{lemma: affine_yangian_PBW_surjectivity} is also injective, i.e. there is an isomorphism of $\Z_{\geq 0}$-algebras:
                $$\rmU(\toroidal^{\positive}) \xrightarrow[]{\cong} \gr \rmY(\hat{\g})$$
            This implies the existence of PBW bases for $\rmY(\hat{\g})$.
        \end{theorem}

        \begin{proposition}[Levendorskii presentation for $\toroidal^+$] \label{prop: levendorskii_presentation__for_central_extensions_of_multiloop_algebras}
            Suppose that $\hat{\g}$ is simply laced.
        
            The Lie algebra $\toroidal^+$ is isomorphic to the Lie algebra  generated by the set:
                $$\{ X_{i, r}^{\pm}, H_{i, r} \}_{(i, r) \in \hat{\simpleroots} \x \Z_{\geq 0}}$$
            whose elements are subjected to the following relations, given for all $(i, r), (j, s) \in \hat{\simpleroots} \x \Z_{\geq 0}$:
                $$H_{i, 0} = h_i, X_{i, 0}^{\pm} = x_i^{\pm}$$
                $$[ H_{i, r}, H_{j, s} ] = 0$$
                $$[ H_{i, 0}, X_{j, s}^{\pm} ] = \pm (\alpha_i, \alpha_j) X_{j, s}^{\pm}$$
                $$[ X_{i, r}^+, X_{j, s}^- ] = \delta_{ij} H_{i, r + s}$$
                $$[ X_{i, 1}^{\pm}, X_{j, 0}^{\pm} ] - [ X_{i, 0}^{\pm}, X_{j, 1}^{\pm} ] = 0$$
            as well as the \say{Serre relations}, given for all $i \not = j \in \hat{\simpleroots}$ and all $r, s \in \Z_{\geq 0}$:
                $$\ad(X_{i, 0}^{\pm})^{1 - c_{ij}}( X_{j, s}^{\pm} ) = 0$$
            The isomorphism in question is as in lemma \ref{lemma: chevalley_serre_presentations_for_positive_toroidal_lie_algebras}.
        \end{proposition}
            \begin{proof}
                
            \end{proof}
        \begin{corollary}[Levendorskii presentation for $\toroidal$]
            Suppose that $\hat{\g}$ is simply laced.
            
            The Lie algebra $\toroidal$ is isomorphic to the Lie algebra  generated by the set:
                $$\{ X_{i, r}^{\pm}, H_{i, r} \}_{(i, r) \in \hat{\simpleroots} \x \Z} \cup \{ K \}$$
            whose elements are subjected to the same relations as in proposition \ref{prop: levendorskii_presentation__for_central_extensions_of_multiloop_algebras}, as well as the following additional set of relations:
                $$[K, \toroidal] = 0$$
            The isomorphism in question is as in lemma \ref{lemma: chevalley_serre_presentations_for_positive_toroidal_lie_algebras}.
        \end{corollary}
            \begin{proof}
                Combine proposition \ref{prop: levendorskii_presentation__for_central_extensions_of_multiloop_algebras} with lemma \ref{lemma: chevalley_serre_presentations_for_positive_toroidal_lie_algebras}. 
            \end{proof}