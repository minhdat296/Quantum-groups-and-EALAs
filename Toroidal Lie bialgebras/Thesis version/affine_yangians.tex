\section{Yangians associated to Kac-Moody algebras}
    \begin{convention}[A fixed symmetrisable Kac-Moody algebra] \label{conv: a_fixed_symmetrisable_kac_moody_algebra}
        Over the course of this section, suppose that $\fraku$ is a symmetrisable Kac-Moody algebra whose Cartan matrix is is indecomposable. All related constructions will be carried out over the previously fixed field $k$, which we recall to be algebraically closed and of characteristic $0$. 

        To avoid confusion with the Cartan subalgebra $\h$ of $\g$ (cf. convention \ref{conv: a_fixed_finite_dimensional_simple_lie_algebra}), let us write:
            $$\fraku_0$$
        to mean a choice of Cartan subalgebra of $\fraku$. 
        
        On $\fraku$, there shall be a non-degenerate and invariant symmetric bilinear form:
            $$(-, -)_{\fraku}$$
        Let us also denote the set of simple roots of $\fraku$ by:
            $$\simpleroots_{\fraku} := \{\alpha_i\}_{i \in \simpleroots_{\fraku}}$$
        We will be normalising the Chevalley-Serre generators, i.e. elements of the set:
            $$\{x_i^{\pm}, h_i\}_{i \in \simpleroots}$$
        so that:
            $$(x_i^+, x_i^-)_{\fraku} = 1$$
            $$[x_i^+, x_i^-] = h_i$$
        for all $i \in \simpleroots_{\fraku}$.

        Let us also denote the Cartan matrix of $\fraku$ by:
            $$C := (c_{ij})_{i, j \in \simpleroots_{\fraku}}$$
    \end{convention}
    Out of technical necessity, we must right away exclude the cases where $\fraku$ is either of type $\sfA_1^{(1)}$ or of type $\sfA_2^{(2)}$ (in the notations of \cite[Chapter 4]{kac_infinite_dimensional_lie_algebras}). There seem to be evidences towards the fact that in these cases, the presentations for the associated (formal) Yangian as given in definitions \ref{def: formal_yangians_associated_to_symmetrisable_kac_moody_algebras} and \ref{def: yangians_associated_to_symmetrisable_kac_moody_algebras} must include some higher-order relations. In the former case, it is also known that the formal Yangian is not a graded flat deformation of $\rmU(\toroidal^{\positive})$, which is problematic for us.

    \begin{convention}
        All non-Lie algebras will automatically be assumed to be associative and unital. 

        Also, if $A$ is an algebra and $X_1, ..., X_n \in A$ are arbitrary elements therein, then we will be using the following shorthands:
            $$\{ X_1, ..., X_n \} := \sum_{\sigma \in S_n} X_{\sigma(1)} \cdot ... \cdot X_{\sigma(n)}$$
        and for any $X \in A$, we will be writing:
            $$\ad(X) := [X, -]$$
    \end{convention}

    \subsection{Presentations for Yangians}
        \begin{definition}[Formal Yangians associated to symmetrisable Kac-Moody algebras] \label{def: formal_yangians_associated_to_symmetrisable_kac_moody_algebras}
            The \textbf{formal Yangian associated to the derived subalgebra of $\fraku$} is a certain $k$-algebra:
                $$\rmY_{\hbar}(\fraku')$$
            generated by the set:
                $$\{X_{i, r}^{\pm}, H_{i, r}\}_{(i, r) \in \simpleroots_{\fraku} \x \Z_{\geq 0}}$$
            whose elements are subjected to the following relations, given for all $(i, r), (j, s) \in \simpleroots_{\fraku} \x \Z_{\geq 0}$:
                $$[H_{i, r}, H_{j, s}] = 0$$
                $$[H_{i, 0}, X_{j, s}^{\pm}] = \pm (\alpha_i, \check{\alpha}_j)_{\fraku} X_{j, s}^{\pm}$$
                $$[X_{i, r}^+, X_{j, s}^-] = \delta_{i, j} H_{i, r + s}$$
                $$[H_{i, r + 1}, X^{\pm}_{j, s}] - [H_{i, r}, X^{\pm}_{j, s + 1}] = \frac12 \hbar (\alpha_i, \check{\alpha}_j)_{\fraku} \{H_{i, r}, X^{\pm}_{j, s}\}$$
                $$[X^{\pm}_{i, r + 1}, X^{\pm}_{j, s}] - [X^{\pm}_{i, r}, X^{\pm}_{j, s + 1}] = \frac12 \hbar (\alpha_i, \check{\alpha}_j)_{\fraku} \{X^{\pm}_{i, r}, X^{\pm}_{j, s}\}$$
                $$\sum_{ \sigma \in S_{1 - c_{ij}} } \ad(X_{ i, r_{\sigma(1 - c_{ij})} }) \cdot ... \cdot \ad(X_{ i, r_{\sigma(2)} }) \cdot \ad(X_{ i, r_{\sigma(1)} }) \cdot X_{j, s}^{\pm} = 0$$
                
            To obtain the \textbf{formal Yangian associated to the Kac-Moody algebra $\fraku$} itself:
                $$\rmY_{\hbar}(\fraku)$$
            we must enlarge the generating set to:
                $$\{X_{i, r}^{\pm}, H_{i, r}\}_{(i, r) \in \simpleroots_{\fraku} \x \Z_{\geq 0}} \cup \fraku_0$$
            and then impose the following additional relations, given for all $h \in \fraku_0$ and all $(j, s) \in \simpleroots_{\fraku} \x \Z_{\geq 0}$:
                $$[h, H_{j, s}] = 0$$
                $$[h, X_{j, s}^{\pm}] = \pm \alpha_j(h) X_{j, s}^{\pm}$$
        \end{definition}
        \begin{remark}[Formal Yangians as \texorpdfstring{$k[\hbar]$}{}-algebras and the $\Z_{\geq 0}$-grading] \label{remark: positive_Z_grading_on_formal_yangians}
            Equivalently, the formal Yangians $\rmY_{\hbar}(\fraku')$ and $\rmY_{\hbar}(\fraku)$ from definition \ref{def: formal_yangians_associated_to_symmetrisable_kac_moody_algebras} can be regarded as $k[\hbar]$-algebras generated by the sets:
                $$\{X_{i, r}^{\pm}, H_{i, r}\}_{(i, r) \in \simpleroots_{\fraku} \x \{0, 1\}}$$
            and:
                $$\{X_{i, r}^{\pm}, H_{i, r}\}_{(i, r) \in \simpleroots_{\fraku} \x \{0, 1\}} \cup \fraku_0$$
            respectively, whose elements are subjected to the same relations.

            In any event, these algebras carry natural $\Z_{\geq 0}$-gradings given by:
                $$\deg \hbar = 1$$
                $$\forall (i, r) \in \simpleroots_{\fraku} \x \Z_{\geq 0}: \deg X^{\pm}_{i, r} = \deg H_{i, r} = r$$
            Later on, this grading will be used for establishing formal Yangians as Rees algebras for certain cases of the Kac-Moody algebra $\fraku$.
        \end{remark}
        \begin{remark}[The degree-$0$ graded component]
            When regarded as a $k$-algebra, both $\rmY_{\hbar}(\fraku')$ and $\rmY_{\hbar}(\fraku)$ admit the universal enveloping algebra $\rmU(\fraku)$ as a $k$-subalgebra. In particular, $\rmU(\fraku)$ is isomorphic to the degree-$0$ graded component of $\rmY_{\hbar}(\fraku')$ and $\rmY_{\hbar}(\fraku)$; this can be shown by constructing an algebra homomorphisms:
                $$\rmU(\fraku) \to \rmY_{\hbar}(\fraku'), \rmU(\fraku) \to \rmY_{\hbar}(\fraku)$$
            given on generators by:
                $$x_i^{\pm} \mapsto X_{i, 0}^{\pm}, h_i \mapsto H_{i, 0}$$
            and then showing that these maps are injective. 

            From now on, this fact will be used without explicit mention.
        \end{remark}

        Under certain mild technical restrictions, it is possible to give a presentation for the formal Yangians $\rmY_{\hbar}(\fraku)$ in terms only of generators of degrees $0$ and $1$, i.e. elements of the set:
            $$\{X_{i, r}^{\pm}, H_{i, r}\}_{(i, r) \in \simpleroots_{\fraku} \x \{0, 1\}} \cup \fraku_0$$
        This can be viewed as a generalisation of the work \cite{levendorskii_finite_type_yangians_presentation} of Levendorskii, applicable to the case where $\fraku$ is of finite-type (e.g. $\fraku \cong \g$), and is very useful for performing computations with the generators of $\rmY_{\hbar}(\fraku)$.
        
        \begin{convention}[An auxiliary generator for (formal) Yangians]
            Henceforth, let us write:
                $$T_{i, 1}(\hbar) := H_{i, 1} - \frac12 \hbar H_{i, 0}^2$$
                $$T_{i, 1} := T_{i, 1}(1) = H_{i, 1} - \frac12 H_{i, 0}^2$$
        \end{convention}
        
        The hypotheses of the following lemma are satisfied at least when $\g$ is either of finite type or of affine type, save for the types $\sfA_1^{(1)}$ and $\sfA_1^{(2)}$.
        \begin{lemma}[A Levendorskii-type presentation for Yangians of Kac-Moody algebras] \label{lemma: levendorskii_presentation}
            \cite[Theorem 2.13]{guay_nakajima_wendlandt_affine_yangian_coproduct} Choose a total ordering on the set of simple roots $\simpleroots_{\fraku}$ and suppose that the Cartan matrix $C := (c_{ij})_{i, j \in \simpleroots_{\fraku}}$ of the Kac-Moody algebra $\fraku$ is such that, for any $i < j \in \simpleroots_{\fraku}$ (with respect to some choice of total ordering on $\simpleroots_{\fraku}$) the following $2 \x 2$ submatrix is invertible:
                $$
                    \begin{pmatrix}
                        c_{ii} & c_{ij}
                        \\
                        c_{ji} & c_{ji}
                    \end{pmatrix}
                $$
            
            The formal Yangian $\rmY_{\hbar}(\fraku)$ of $\fraku$ will then be isomorphic to the associative $\bbC$-algebra generated by the set:
                $$\{ H_{i, r}, X_{i, r}^{\pm} \}_{(i, r) \in \simpleroots_{\fraku} \x \Z_{\geq 0}}$$
            whose elements are subjected to the following relations\footnote{... and it is understood that the elements $H_{i, 0} = h_i, X_{i, 0}^{\pm} = e_i^{\pm}$ satisfy the Chevalley-Serre relations defining $\fraku$; cf. \cite[Chapter 1]{kac_infinite_dimensional_lie_algebras}.}:
                $$H_{i, 0} = h_i, X_{i, 0}^{\pm} = x_i^{\pm}$$
                $$[ H_{i, r}, H_{j, s} ] = 0$$
                $$[ H_{i, 0}, X_{j, s}^{\pm} ] = \pm (\alpha_i, \check{\alpha}_j)_{\fraku} X_{j, s}^{\pm}$$
                $$[ X_{i, r}^+, X_{j, s}^- ] = \pm \delta_{ij} H_{i, r + s}$$
                $$\left[ T_{i, 1}(\hbar), X_{j, 0}^{\pm} \right] = \pm \hbar (\alpha_i, \check{\alpha}_j)_{\fraku} X_{j, 1}^{\pm}$$
                $$[ X_{i, 1}^{\pm}, X_{j, 0}^{\pm} ] - [ X_{i, 0}^{\pm}, X_{j, 1}^{\pm} ] = \pm \frac12 \hbar (\alpha_i, \check{\alpha}_j)_{\fraku} \{X_{i, 0}^{\pm}, X_{j, 0}^{\pm}\}$$
        \end{lemma}

        Let us now consider what might occur when we specialise the variable $\hbar$ to a specific value $\hbar_0 \in k$.

        Firstly, let us consider the case $\hbar_0 \not = 0$, which is much simpler than the case $\hbar_0 = 0$. 
        \begin{definition}[Yangians associated to symmetrisable Kac-Moody algebras] \label{def: yangians_associated_to_symmetrisable_kac_moody_algebras}
            By specialising $\hbar$ to some $\hbar_0 \in k^{\x}$, one obtains the \textbf{Yangian} associated to $\fraku'$ and $\fraku$, respectively:
                $$\rmY(\fraku') := \rmY_{\hbar}(\fraku')/(\hbar - \hbar_0)$$
                $$\rmY(\fraku) := \rmY_{\hbar}(\fraku)/(\hbar - \hbar_0)$$
        \end{definition}
        \begin{definition}[Rees algebras] \label{def: rees_algebras}
            Let:
                $$A := \bigoplus_{r \in \Z_{\geq 0}} A_r$$
            be a $\Z_{\geq 0}$-filtered $k$-algebra. The \textbf{Rees algebra} associated to $A$ is then the $k[\hbar]$-algebra given by:
                $$\Rees_{\hbar} A := \bigoplus_{r \in \Z_{\geq 0}} A_r \hbar^r$$
        \end{definition}
        \begin{lemma}[Basic properties of Rees algebras] \label{lemma: basic_properties_of_rees_algebras}
            \cite[Exercise I.9.5]{kassel_quantum_groups} Let:
                $$A := \bigoplus_{r \in \Z_{\geq 0}} A_r$$
            be a $\Z_{\geq 0}$-filtered $k$-algebra. Also, fix an arbitrary element $\hbar_0 \in k$.
            \begin{enumerate}
                \item There are $k$-algebra isomorphisms:
                    $$
                        (\Rees_{\hbar} A)/(\hbar - \hbar_0) \cong
                        \begin{cases}
                            \text{$A$ if $\hbar_0 \not = 0$}
                            \\
                            \text{$\gr A$ if $\hbar_0 = 0$}
                        \end{cases}
                    $$
                \item Suppose that the associated graded algebra $\gr A$ is generated by a set of homogeneous elements:
                    $$a_0, a_1, a_2, ...$$
                respectively of degree $r_0 := 0, r_1, r_2, ... \in \Z_{\geq 0}$, then $\Rees_{\hbar} A$ will be generated by the elements:
                    $$a_0, a_1 \hbar^{r_1}, a_2 \hbar^{r_2}, ...$$
            \end{enumerate}
        \end{lemma}
        \begin{convention}
            In light of lemma \ref{lemma: basic_properties_of_rees_algebras}, it is typically to just make the choice:
                $$\hbar_0 := 1$$
            and hence work with:
                $$\rmY(\fraku) := \rmY_1(\fraku)$$
        \end{convention}
        \begin{remark}[Yangians are $\Z_{\geq 0}$-filtered] \label{remark: positive_Z_filtrations_on_yangians}
            Now, the Yangian $\rmY(\fraku)$ is no longer graded but just $\Z_{\geq 0}$-filtered. The (ascending) filtration in question is given by:
                $$\deg X_{i, r}^{\pm} = \deg H_{i, r} = r$$
                
            However, whether this filtration gives - via the Rees algebra construction - a $\Z_{\geq 0}$-grading agreeing with the one on formal Yangians is a rather subtle issue, with no complete solution in general. Another difficult problem is that of explicitly computing the associated graded algebra:
                $$\gr \rmY(\fraku)$$
            When $\fraku$ is of finite type, it is known that:
                $$\gr \rmY(\fraku) \cong \rmU(\fraku[t])$$
            (in fact, Drinfeld seemed to have constructed the Yangian of a finite-dimensional simple Lie algebra specifically so that this would be true) but when $\fraku$ is a general indecomposable symmetrisable Kac-Moody algebra, what the algebra $\gr \rmY(\fraku)$ is explicitly is still not known.
            
            That said, partial answers to both of these problems have been obtained in \cite[Section 6]{guay_nakajima_wendlandt_affine_yangian_vertex_representations_and_PBW}, where the authors focused on the case where $\fraku$ is of a simply laced affine type. We defer the discussion of the details of these problems to subsection \ref{subsection: yangians_as_deformations}, where the special cases relevant to our goals are examined. 
        \end{remark}

        \begin{remark}
            When $\fraku$ is of finite type, it is also known that the formal Yangian $\rmY_{\hbar}(\fraku)$ (hence also the Yangian $\rmY(\fraku)$) carries a Hopf bialgebra structure\footnote{Recall also that Hopf bialgebra structures, if they exist, are unique up to isomorphisms.}. For organisational purposes, we would like the contents of this section to remain entirely algebraic, as opposed to algebraic and coalgebraic. As such, we will not mention this fact again until subsection \ref{subsection: manin_triples_and_quantisations_of_lie_bialgebras},where we will discuss this Hopf structures as an instance of the much more general fact that Lie bialgebras admit so-called \say{quantisations}. 
        \end{remark}

    \subsection{Finite-type and affine Yangians as deformations} \label{subsection: yangians_as_deformations}
        Let us now discuss how, when $\fraku$ is either of finite type or affine type\footnote{... and satisfying some other mild technical restrictions.}, it can be shown that $\rmY_{\hbar}(\fraku)$ is the flat deformation of the universal enveloping algebra of a certain polynomial Lie algebra. In the former case, we will see that the Lie algebra in question is:
            $$\g[t]$$
        whereas in the latter case, it is:
            $$\toroidal^{\positive} := \uce(\g[v^{\pm 1}, t])$$
        Both ought to be viewed as UCEs; it just so happens that the Lie algebra $\g[t]$ admits itself as a UCE (cf. example \ref{example: affine_lie_algebras_centres}).
    
        \begin{definition}[Graded and PBW deformations] \label{def: graded_and_PBW_deformations}
            Fix an $\Z_{\geq 0}$-graded associative algebra:
                $$U_0 := \bigoplus_{r \geq 0} U_r$$
            over a field $k$. An \textbf{$\Z_{\geq 0}$-graded deformation} of such an algebra $U_0$ is then an $\Z_{\geq 0}$-graded associative $k[\hbar]$-algebra $U_{\hbar}$, free as a $k[\hbar]$-module, and such that:
                $$U_{\hbar}/\hbar U_{\hbar} \cong U_0$$
            Now, fix some $\hbar_0 \in k^{\x}$. The algebra:
                $$U_{\hbar_0} := U_{\hbar}/(\hbar - \hbar_0)U_{\hbar}$$
            is then called the \textbf{PBW deformation} of $U_0$ at $\hbar_0$.  
        \end{definition}