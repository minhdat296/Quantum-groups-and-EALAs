\section{Weight-modules for Yangians}
    \subsection{Standard modules}
        

    \subsection{The category \texorpdfstring{$\calO_{\hbar}$}{}}
        \begin{definition}[Category $O$] \label{def: category_0}
            Let us define the \say{\textbf{category $O$}} of $\rmY_{\hbar}$, which shall be denoted by $\calO_{\hbar}$, to be the full subcategory of ${}^l\rmY_{\hbar}\mod$ spanned by left-$\rmY_{\hbar}$-modules whose underlying vector spaces are:
            \begin{itemize}
                \item $Q \x \Z_{\geq 0}$-graded, and
                \item acted on locally nilpotently and locally finitely by elements of the set $\{X_{i, r}^+\}_{(i, r) \in \simpleroots \x \Z_{\geq 0}}$.
            \end{itemize}
        \end{definition}
        \begin{remark}
            Our definition differs with the one given in \cite[Section 3]{guay_nakajima_wendlandt_affine_yangian_coproduct}. Namely, we are working with modules over $\rmY_{\hbar}$ as opposed to those over $\rmY$. One reason for this is that, unlike the former, the latter is only $Q$-graded while carrying a compatible $\Z_{\geq 0}$-filtration. As such, if we were to work with $\rmY$, we will end up with e.g. standard modules that are not $Q \x \Z_{\geq 0}$-graded, which can make translating notions from the classical theory of the category $O$ for Kac-Moody algebras into this setting more difficult than it needs to be. Moreover, modules over $\rmY_{\hbar}$ contain more information than those over $\rmY$ anyway: for instance, a module over the latter is simply a module over the former such that on it, $\hbar$ acts as $1$, and since $\hbar \in \rmY_{\hbar}$ is central anyway, one sees thus via Schur's Lemma - which says that central elements act as scalars - that at least, any simple $\rmY_{\hbar}$-module will automatically be a $\rmY$-module.
        \end{remark}
        \begin{proposition}
            \begin{enumerate}
                \item Let $V_1, V_2$ be objects of $\calO_{\hbar}$. Then, $V_1 \tensor_{\bbC} V_2$ will be an object of $\calO_{\hbar}$ as well.
                \item If $V_1, V_2, V_3$, then there will exist an isomorphism in $\calO_{\hbar}$:
                    $$(V_1 \tensor_{\bbC} V_2) \tensor_{\bbC} V_3 \cong V_1 \tensor_{\bbC} (V_2 \tensor_{\bbC} V_3)$$
                that is natural in $V_1, V_2, V_3$.
            \end{enumerate}  
        \end{proposition}
            \begin{proof}
                When $\g$ is of finite type, this is clear from the fact that $\rmY_{\hbar}$ has a Hopf structure. 
            \end{proof}