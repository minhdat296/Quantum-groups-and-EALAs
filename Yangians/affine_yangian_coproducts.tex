\section{Coproducts for affine Yangians}
    These are notes for \cite{guay_nakajima_wendlandt_affine_yangian_coproduct}.

    \subsection{Setup}
        Everything will occur over $\bbC$, though for all intents and purposes, one can substitute in any algebraically closed field of characteristic $0$.
    
        For what follows, let $\a$ be a Kac-Moody algebra such that:
        \begin{itemize}
            \item the Cartan matrix is indecomposable
            \item $\a$ is not of the types $\sfA_1^{(1)}$ and $\sfA_1^{(2)}$ (though it is expected that what is presented below works for these Kac-Moody algebras as well, just for different reasons).
        \end{itemize}
        Denote the set of simple roots of $\a$ by $\simpleroots_{\a}$.
        
        Also, let us write:
            $$\rmY_{\hbar} := \rmY_{\hbar}(\a)$$
        for the \textbf{formal Yangian} associated to $\a$. This is the associative $\bbC$-algebra generated by the set:
            $$\{ X_{i, r}^{\pm}, H_{i, r} \}_{(i, r) \in \simpleroots_{\a} \x \Z_{\geq 0}}$$
        whose elements are subjected to certain relations, as in \cite[Definition 2.1]{guay_nakajima_wendlandt_affine_yangian_coproduct}. For the sequel, it is worth recalling that $\rmY_{\hbar}$ carries three natural gradings:
        \begin{itemize}
            \item Firstly, there is the \say{current} grading, which is a $\Z_{\geq 0}$-grading given by:
                $$\deg X_{i, r}^{\pm} = \deg H_{i, r} = r$$
                $$\deg \hbar = 1$$
            With respect to this grading, one can identify $\rmU(\a)$ with the degree-$0$ component.
            \item Secondly, there is the \say{root height} grading, which is another $\Z_{\geq 0}$-grading given inductively by:
                $$\deg X_{i, r}^{\pm 1} = 1$$
                $$\deg H_{i, r} = \deg \hbar = 0$$
            \item Finally, there is the \say{root} grading, which is a grading by the root lattice $Q_{\a} := \Z \cdot \simpleroots_{\a}$ of $\a$, and is given by:
                $$\deg X_{i, r}^{\pm} = \pm\alpha_i$$
                $$\deg H_{i, r} = \deg \hbar = 0$$
        \end{itemize}

        Finally, let us write:
            $$\rmY := \rmY_{\hbar}/(\hbar - 1)$$
        This will be called the \textbf{Yangian}. Unlike the formal Yangian, this algebra does not carry a \say{current} grading, but rather a \say{current} filtration, given by:
            $$\deg X_{i, r}^{\pm} = \deg H_{i, r} = r$$
        It is expected that:
            $$\rmY_{\hbar} := \Rees_{\hbar} \rmY := \bigoplus_{\hbar} \rmY_r \hbar^r$$
        and hence:
            $$\rmU(\a) \cong \rmY_{\hbar}/\hbar \cong \gr \rmY$$
        but so far, this has only been proven for $\a$ either of finite types and $\a$ of simply laced affine types\footnote{Which means that type $\sfA_1^{(1)}$ is excluded, as its Dynkin diagram is $\bullet \toto \bullet$.} (see \cite{guay_regelskis_wendlandt_affine_yangian_vertex_representations_and_PBW}).

    \subsection{The category \texorpdfstring{$\calO_{\hbar}$}{}}
        \todo[inline]{I think the point of using $\calO_{\hbar}$ is that - aside from finiteness - this allows us to introduce a completion of $\rmY$ that would accommodate the infinite sums that Kac-Moody Casimir tensors can contain. This is because associativity and braidedness of tensor products in $\calO_{\hbar}$ are ultimately controlled by these Casimir tensors.}
    
        Following \cite{weekes_phd_thesis_highest_weight_truncated_shifted_yangians}, let us define the category $\calO_{\hbar}$ as follows:
        \begin{definition}[Category $O$] \label{def: category_0}
            (Cf. \cite[Definition 3.2.1]{weekes_phd_thesis_highest_weight_truncated_shifted_yangians}) Let $A := \bigoplus_{n \in Z} A_n$ be an associative algebra graded by some partially ordered commutative monoid $(Z, \geq)$, and let $A_{\geq 0} := \bigoplus_{n \geq 0} A_n$. The \textbf{category $O$} of $A$ is then $A\mod^{A_{\geq}\-\fin} \cap A_0\mod^{\ss}$, where by $A\mod^{A_{\geq}\-\fin}$ we mean the full subcategory of $A\mod^{\fin}$ whose objects are acted on locally finitely by $A_{\geq 0}$, i.e. for every $M \in \Ob(A\mod^{A_{\geq}\-\fin})$ and all $v \in M$, we insist that the subspace $A_{\geq 0} \cdot v \subseteq M$ is finite-dimensional.
        \end{definition}
        When $A$ as in definition \ref{def: category_0} is the Yangian $\rmY_{\hbar}$ being graded by the partially ordered commutative monoid $Q_{\a} \x \Z_{\geq 0}$ (i.e. we are combining the current and root gradings), one obtains the category $O$ of $\rmY_{\hbar}$ as in \cite[Section 3]{guay_nakajima_wendlandt_affine_yangian_coproduct}. We denote the category $O$ of $\rmY_{\hbar}$ by $\calO_{\hbar}$. Furthermore, note that because $\rmU(\a) \cong \rmY_{\hbar, Q \x \{0\}}$, the degree-$(0, 0)$ component of $\rmY_{\hbar}$ is nothing but the universal enveloping algebra of the Cartan subalgebra of $\a$.

        \begin{proposition}
            \begin{enumerate}
                \item Let $V_1, V_2$ be objects of $\calO_{\hbar}$. Then, $V_1 \tensor_{\bbC} V_2$ will be an object of $\calO_{\hbar}$ as well.
                \item If $V_1, V_2, V_3$, then there will exist an isomorphism in $\calO_{\hbar}$:
                    $$(V_1 \tensor_{\bbC} V_2) \tensor_{\bbC} V_3 \cong V_1 \tensor_{\bbC} (V_2 \tensor_{\bbC} V_3)$$
                that is natural in $V_1, V_2, V_3$.
            \end{enumerate}  
        \end{proposition}