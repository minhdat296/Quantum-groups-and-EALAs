\section{Highest-weight modules over finite untwisted Yangians} \label{section: highest_weight_modules_for_finite_untwisted_yangians}
    \begin{convention}
        Throughout, let $\rmY(\g)^+$ be the \say{upper triangular} subalgebra of $\rmY(\g)$ generated by the set $\{ X^+_{\alpha, r} \}_{(\alpha, r) \in \Phi^+ \x \Z_{\geq 0}}$ (cf. corollary \ref{coro: triangular_decompositions_of_finite_type_yangians}).

        Also, let us recall that because $\rmY(\g)^0 \cong \rmU(\h[t]) \cong \Sym(\h[t])$, to specify how the subalgebra $\rmY(\g)^0$ acts on a $\rmY(\g)$-module is the same as to specify how $\h$ and the generator $t$ act on said module. We will often make this implicit identification.

        Finally, we will usually making use of the fact that because $\rmU(\g)$ is identified with the degree-$0$ filtrant $\rmY(\g)_0$ (which is \textit{not} the same as $\rmY(\g)^0$; see remark \ref{remark: the_degree_grading_on_finite_type_yangians}) and hence is a subalgebra of $\rmY(\g)$, modules over $\rmY(\g)$ can be regarded as $\g$-modules.
    \end{convention}

    \begin{convention}
        Let us write $Q := \bigoplus_{i \in \simpleroots} \Z \alpha_i$ and $\Lambda := \bigoplus_{i \in \simpleroots} \Z \varpi_i$ for the root and weight lattices of $\g$. Here, $\alpha_i$ and $\varpi_i$ are, respectively, the simple roots and fundamental weights.
    \end{convention}

    \subsection{Classification of finite-dimensional simple modules}
        \begin{definition}[Weight modules] \label{def: weight_modules_for_finite_untwisted_yangians}
            A $\rmY(\g)$-module $V$ is said to be a \textbf{weight module} (or to have a \textbf{weight space decomposition}) if and only if for all $(i, r) \in \simpleroots \x \Z_{\geq 0}$ and all $v \in V$, there exists some $\lambda_{i, r} \in \bbC$ such that:
                $$H_{i, r} \cdot v = \lambda_{i, r} v$$
            In other words, a weight module of $\rmY(\g)$ is a module that is acted upon semi-simply by $\h[t]$. The direct summand of $V$ on which $\h[t]$ acts by multiplication by the entries of the tuple of scalars $(\lambda_{i, r})_{(i, r) \in \simpleroots \x \Z_{\geq 0}} \in (\bbC^{\oplus \simpleroots})^{\Z_{\geq 0}}$ is called the subspace of \textbf{weight} $(\lambda_{i, r})_{(i, r) \in \simpleroots \x \Z_{\geq 0}}$.
        \end{definition}
        \begin{remark}
            Let $I$ be a set. Then, let us observe that to give a tuple:
                $$( \lambda_{i, r} )_{(i, r) \in I \x \Z_{\geq 0}} \in (\bbC^{\oplus I})^{\Z_{\geq 0}}$$
            is the same as giving a tuple of formal power series:
                $$( \lambda_i(u) )_{i \in I} \in (1 + u^{-1} \bbC[\![u^{-1}]\!])^{\oplus I}$$
            whose entries $\lambda_i(u) \in 1 + u^{-1} \bbC[\![u^{-1}]\!]$ are given by:
                $$\lambda_i(u) := 1 + \sum_{r \in \Z_{\geq 0}} \lambda_{i, r} u^{-r - 1}$$
            Choosing $u^{-1}$ instead of $u$ as the formal variable is a matter of convenience.
        \end{remark}
        \begin{definition}[Highest-weight modules] \label{def: highest_weight_modules_for_finite_untwisted_yangians}
            A $\rmY(\g)$-module $V$ is said to be of \textbf{highest-weight} $( \lambda_i(u) )_{i \in \simpleroots} \in (1 + u^{-1} \bbC[\![u^{-1}]\!])^{\oplus \simpleroots}$ if it is a weight module that is cyclic and whose generators (usually called \textbf{primitive vectors} or \textbf{highest-weight vectors}) are annihilated by $\rmY(\g)^+$ and are of weight $\lambda$, i.e. any such generator $v_{\max}$ satisfies:
                $$\rmY(\g)^+ \cdot v_{\max} = 0$$
                $$H_i(u) \cdot v_{\max} = \lambda_i(u) v_{\max}$$
            with the series $H_i(u) \in 1 + u^{-1} \bbC[\![u^{-1}]\!]$ as in proposition \ref{prop: generating_series_for_finite_untwisted_yangians} (see also proposition \ref{prop: shift_automorphisms_via_generating_series}).
        \end{definition}

        \begin{definition}[Standard modules] \label{def: standard_modules_for_finite_untwisted_yangians}
            The \textbf{standard $\rmY(\g)$-module} of highest-weight $\lambda(u) := ( \lambda_i(u) )_{i \in \simpleroots} \in (1 + u^{-1} \bbC[\![u^{-1}]\!])^{\oplus \simpleroots}$, which shall be denoted by:
                $$\standard^{\lambda(u)}$$
            is the quotient of $\rmY(\g)$ by the left-ideal generated by the coefficients of the formal series in the set:
                $$\left\{ X_i^+(u), H_i(u) - \lambda_i(u) \right\}_{i \in \simpleroots}$$
            with the series $X_i^+(u), H_i(u) \in 1 + u^{-1} \bbC[\![u^{-1}]\!]$ as in proposition \ref{prop: generating_series_for_finite_untwisted_yangians} (see also proposition \ref{prop: shift_automorphisms_via_generating_series}).
        \end{definition}
        Clearly, standard modules are higehst-weight modules in the sense of definition \ref{def: highest_weight_modules_for_finite_untwisted_yangians}. The image of $1 \in \rmY(\g)$ under the canonical quotient map $\rmY(\g) \to \standard^{\lambda(u)}$ is a primitive vector of $\standard^{\lambda(u)}$; we will usually denote it by $v_{\lambda(u)}$.

        By the PBW theorem for $\rmY(\g)$ (theorem \ref{theorem: PBW_bases_for_finite_type_yangians}), we know also that $\standard^{\lambda(u)}$ has an ordered basis consisting of monomials of the form:
            $$X^-_{i_m, r_m} ... X^-_{i_1, r_1} \cdot v_{\lambda(u)}$$
        for some $m \geq 0$ and $(i_j, r_j) \in \simpleroots \x \Z_{\geq 0}$.

    \subsection{Universal R-matrices and the RTT presentation} \label{subsection: universal_R_matrices_of_finite_untwisted_yangians}