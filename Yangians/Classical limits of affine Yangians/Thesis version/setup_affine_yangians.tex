\section{Some conventions}
    \subsection{Symmetrisable Kac-Moody algebras} \label{subection: a_fixed_symmetrisable_kac_moody_algebra}
        It will be convenient to phrase certain definitions and results in terms of general symmetrisable Kac-Moody algebras, so let us fix such a Lie algebra $\fraku$. \textit{We do not require that the Cartan matrix is indecomposable}. All related constructions will be carried out over the previously fixed field $k$, which we recall to be algebraically closed and of characteristic $0$. 

        To avoid confusion with the Cartan subalgebra $\h$ of $\g$ (cf. subsection \ref{subsection: finite_dimensional_simple_lie_algebras}), let us write:
            $$\fraku_0$$
        to mean a choice of Cartan subalgebra of $\fraku$. On $\fraku$, there shall be a non-degenerate and invariant symmetric bilinear form:
            $$(-, -)_{\fraku}$$
        whose restriction to $\fraku_0$ is non-degenerate. This allows us to construct the Cartan matrix of $\fraku$:
            $$C_{\fraku} := (c_{ij})_{i, j \in \simpleroots_{\fraku}}$$
        using the same procedure used to construct the affine Cartan matrix $\hat{C}$ in subsection \ref{subsection: a_fixed_untwisted_affine_kac_moody_algebra}. Said procedure yields us also a symmetrisation:
            $$C_{\fraku} := D_{\fraku} A_{\fraku}$$
        wherein $D_{\fraku}$ is diagonal and invertible, and $A_{\fraku}$ is symmetric (cf. \cite[Chapter 2]{kac_infinite_dimensional_lie_algebras}). From the Cartan matrix $C$, one can construct an adjacency matrix of an undirected graph with weighted edges, called the Dynkin diagram of $\fraku$ (cf. \cite[Section 4.7]{kac_infinite_dimensional_lie_algebras}), whose roots form the set:
            $$\Phi_{\fraku}$$
        of roots of $\fraku$; if between any two vertices of the Dynkin diagram of $\fraku$, there is exactly one edge, then we will say that the Dynkin diagram (and likewise, $\fraku$) is \textbf{simply laced}. Let us also denote the set of simple roots of $\fraku$ by:
            $$\{\alpha_i\}_{i \in \simpleroots_{\fraku}}$$
        We will be normalising the Chevalley-Serre generators (cf. \cite[Theorem 1.4]{kac_infinite_dimensional_lie_algebras}), i.e. elements of the set:
            $$\{x_i^{\pm}, h_i\}_{i \in \simpleroots}$$
        so that:
            $$(x_i^+, x_i^-)_{\fraku} = 1$$
        for all $i \in \simpleroots_{\fraku}$. Also, let us regard $\fraku$ as a Lie algebra graded by its root lattice:
            $$Q_{\fraku} := \Z\simpleroots_{\fraku}$$
        with the grading in question being given by:
            $$\deg x = \alpha$$
        for all $\alpha \in \Phi_{\fraku} \cup \{0\}$ and all $x \in \fraku_{\alpha}$.
        
        To avoid confusion with the \say{upper/lower triangular} nilpotent Lie subalgebras $\n^{\pm} \subset \g$ (cf. subsection \ref{subsection: finite_dimensional_simple_lie_algebras}), let us instead write:
            $$\fraku_{\up/\low} := \bigoplus_{\alpha \in \Phi_{\fraku}^{\pm}} \fraku_{\alpha}$$
        Then, let us write:
            $$\b_{\up/\low} := \fraku_0 \oplus \fraku_{\up/\low}$$
        to denote the upper/lower Borel subalgebras of $\fraku$, i.e. the direct sums of the Cartan subalgebra $\fraku_0$ with the direct sums of the positive/negative root spaces. Recall that $\fraku$ admits a triangular decomposition:
            $$\fraku \cong \fraku_{\low} \oplus \fraku_0 \oplus \fraku_{\up}$$
        (cf. \cite[Theorem 1.2]{kac_infinite_dimensional_lie_algebras}). Affine Kac-Moody algebras are symmetrisable \textit{a priori} (cf. \cite[Chapter 4]{kac_infinite_dimensional_lie_algebras} and subsection \ref{subsection: a_fixed_untwisted_affine_kac_moody_algebra}), and when:
            $$\fraku \cong \hat{\g}$$
        we will defer to the more conventional notations:
            $$\hat{\h} := \fraku_0$$
            $$\hat{\n}^{\pm} := \fraku_{\up/\low}$$
        (cf. subsection \ref{subsection: a_fixed_untwisted_affine_kac_moody_algebra}).
            
        Out of technical necessity, we must right away exclude the cases where $\fraku$ is either of type $\sfA_1^{(1)}$ or of type $\sfA_2^{(2)}$ (in the notations of \cite[Chapter 4]{kac_infinite_dimensional_lie_algebras}). There seem to be evidences towards the fact that in these cases, the presentations for the associated (formal) Yangian as given in definitions \ref{def: formal_yangians_associated_to_symmetrisable_kac_moody_algebras} and \ref{def: yangians_associated_to_symmetrisable_kac_moody_algebras} must include some higher-order relations. In the former case, it is also known that the formal Yangian is not a graded flat deformation of $\rmU(\toroidal^{\positive})$, which is problematic for us.

    \subsection{Some abbreviations}
        We will also be making use of some shorthands.
        \begin{convention}
            All non-Lie algebras will automatically be assumed to be associative and unital. 
    
            Also, if $A$ is an algebra and $X_1, ..., X_n \in A$ are arbitrary elements therein, then we will be using the following shorthands:
                $$\{ X_1, ..., X_n \} := \sum_{\sigma \in S_n} X_{\sigma(1)} \cdot ... \cdot X_{\sigma(n)}$$
            and for any $X \in A$, we will be writing:
                $$\ad(X) := [X, -]$$
        \end{convention}
        \begin{convention}
            Let $V$ be a vector space.
        
            As a shorthand, we will be writing:
                $$\bar{\Delta}(X) := X \tensor 1 + 1 \tensor X$$
            and this will be understood to be an element of $\rmT(V)^{\tensor 2}$. If we have:
                $$X := \sum_{i, j} X_i \tensor X_j \in \rmT(V)^{\tensor 2}$$
            then we will be writing:
                $$X_{12} := \sum_{i, j} X_i \tensor X_j \tensor 1 \in \rmT(V)^{\tensor 3}$$
                $$X_{23} := \sum_{i, j} 1 \tensor X_i \tensor X_j \in \rmT(V)^{\tensor 3}$$
                $$X_{13} := \sum_{i, j} X_i \tensor 1 \tensor X_j \in \rmT(V)^{\tensor 3}$$
            and likewise for the other permutations. If $V$ is a Lie algebra then instead of the tensor algebra, we will typically think of these elements as living in tensor powers of the universal enveloping algebra.
        \end{convention}