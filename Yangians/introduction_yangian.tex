\section{Introduction}
    \subsection{What are Yangians and why do we care ?}
        Until the 1980s, it was believed that most universal enveloping algebras of Lie algebras, say $\t$, coming from finite-dimensional simple Lie algebras $\g$ (over $\bbC$), e.g. $\t = \g, \g[t], \hat{\g}$, were \say{rigid}, in the sense that:
            $$H^i_{\Lie}(\t, M) \cong 0$$
        for all $\t$-modules $M$ satisfying some finitenes condition and all $i > 0$. This belief is not unfounded, as there is a classical result by Whitehead asserting that indeed, one has that:
            $$H^i_{\Lie}(\g, M) \cong 0$$    
        for all finite-dimensional $\g$-modules $M$ and all $i > 0$. However, in many other cases, it was discovered by Drinfeld that this previous belief was actually false. Namely, when $M \cong \bigwedge^2 \t$, it is entirely possible that:
            $$H^2_{\Lie}(\t, \bigwedge^2 \t) \not \cong 0$$
        in particular, thereby implying the existence of non-trivial deformations (over $\bbC[\![\hbar]\!]$) of $\rmU(\t)$, per the Hochschild theory. By imposing $\t$-invariance on such $2$-cocycles, Drinfeld and Belavin were able to furthermore single out those deformations of $\rmU(\t)$ with a reasonable representation theory, in much the same way that the existence of the invariant Killing form on finite-dimensional simple Lie algebras over $\bbC$ leads to their representation theory as we know it today. Etingof and Kazhdan (see \cite{etingof_kazhdan_quantisation_1}) then also showed that the deformations of $\rmU(\t)$ obtained via the procedure above is not only functorial, but also universal (though their result is purely existential, not constructive!). Note also, that in looking at \say{invariant} Lie $2$-cocycles:
            $$\delta: \t \to \bigwedge^2 \t$$
        one is in fact looking at so-called \textbf{Lie bialgebra} structures on $\t$; when $\t$ is infinite-dimensional, there are actually some topological subtleties at play as well, but we will not worry too much about them for now. Deformations of $\rmU(\t)$ are called \textbf{quantisations} of these Lie bialgebra structures on $\t$. These quantisations are usually known holistically as \textbf{quantum groups}; technically, they are certain topological bialgebras (or even Hopf algebras) that deform the canonical Hopf algebra structure on $\rmU(\t)$. The Lie bialgebras themselves are known as \textbf{classical limits} of the quantum groups.
    
        Now, the classification of classical R-matrices due to Drinfeld and Belavin asserts that there exist only three types of such matrices. One such type, the so-called \say{rational} R-matrices, is known to give rise to topological Lie bialgebra structures on:
            $$\t \cong \uce( \a[t] ), \: \a \in \{\g, \g[v^{\pm 1}]\}$$
        Quantisations of these Lie bialgebras are known as \textbf{Yangians}. As such, these Yangians are somehow \say{naturally occuring} quantum groups.

        Yangians had actually occured in the mathematical physics literature prior to the discovery of quantum groups. Previously, they were regarded merely as associative algebras of physical significance that happen to carry bialgebra structures. Their classical limits were known as well, as these can be computed knowing only the bialgebra structures, but what was missing was the realisation that Yangians deform $\rmU(\t)$ in the cohomological sense explained above. For such historical reasons, Yangians would usually be given in the literature by explicit generators and relations. Now, this is not to say that such presentations are not useful; in fact, they are invaluable for representation theory, and worse still, no one seems to have ever derived these relations for Yangians using only the abstract framework proposed by Etingof-Kazhdan. 

    \subsection{What are the technical difficulties ?}
        In the original article where Yangians (particularly those associated to $\a := \g$) first appeared (see \cite{drinfeld_original_yangian_paper}), Drinfeld demonstrated how the previously-known bialgebra structures on the Yangian of $\g$ originate from a classical R-matrix of $\g[t]$. The analogous result for the case $\a := \g[v^{\pm 1}]$, on the other hand, was not known until rather recently. Firstly, this is because even though one knows principally, according to Etingof-Kazhdan, that the Yangian ought to exist as a topological bialgebra in this case as well, an explicit formula for the coproduct on this Yangian was not discovered until \cite{guay_nakajima_wendlandt_affine_yangian_coproduct}. Secondly, no account of any Lie bialgebra structure on $\t := \uce(\g[v^{\pm 1}, t])$ has actually ever been recorded in the literature. This remains to be done, but an explicit formula for the corresponding Lie cobracket and classical R-matrix can be inferred from the contents of \cite[Chapter 3]{msc_thesis_gamma_extended_toroidal_lie_algebras}. Lastly, the connection between the classical R-matrix of $\t$ and the so-called \textbf{quantum R-matrix} of its Yangian remains mysterious, so it is not yet absolutely clear if the classical R-matrix of $\t$ mentioned above might be the \say{correct} one, though it is believed by experts\footnote{Which is to say, people with much more experience with this stuff than the author.} that this is the \say{correct} classical R-matrix.

    \subsection{What are our goals for these notes ?}
        Aside from serving as a place for noting down many of the fundamental aspects of the theory of Yangians, we would also like to gain an understanding of the role that Yangians play in the theory of vertex algebras, particularly that of so-called W-algebras. 

    \subsection{What have we \textit{not} done ?}
        Yangians can also be understood via geometric means. Namely, slight variations of Yangians, known as truncated shifted Yangians can be regarded as deformation quantisations of certain transversal slices inside affine Grassmannians. We do not expect to be able to cover these aspects of the theory here. The interested reader can consult \cite{quantisation_of_affine_grassmannian_slices} and subsequent works.