\section{Setups}
    \subsection{Finite-dimensional simple Lie algebras} \label{subsection: finite_dimensional_simple_lie_algebras}
        As a precursor to our main discussion, let us recall some features of the theory of finite-dimensional simple Lie algebras, particularly about their structure.

        We will be working over $\bbC$, though this can be replaced by any algebraically closed field of characteristic $0$. The fact that $\bbC$ is algebraically closed is needed so that we will have enough eigenvalues and hence, certain operators will be diagonalisable. The second fact, that $\chara \bbC = 0$, is made so that we can avoid having certain relations, e.g. as in theorem \ref{theorem: serre_theorem_for_finite_dimensional_simple_lie_algebras}, vanish.

        \begin{definition}[Simple Lie algebras]
            A Lie algebra is said to be \textbf{simple} if and only if it admits no non-zero Lie ideals. 
        \end{definition}

        Over a field $\bbC$, much is known about the structure of a simple Lie algebra $\g$ that is finite-dimensional when regarded as a $\bbC$-vector space. The bulk of the content presented above is discussed in further details in any standard textbook on Lie algebras (cf. e.g. \cite{humphreys_lie_algebras} or the first half of \cite{carter_affine_lie_algebras}). Let us give a very brief recap of this theory. Also, let us fix once and for all such a finite-dimensional simple Lie algebra:
            $$\g$$

        One begins this process by choosing a \textbf{Cartan subalgebra} $\h$ for $\g$, which is a maximal abelian Lie subalgebra $\h$\footnote{We note also that, it is well-known that all Cartan subalgebras of $\g$ are conjugate to one another.}, whose elements are diagonalisable under the adjoint representation $\ad: \g \to \gl(\g)$. Now, let $V$ be a $\g$-module. Then, one can abstractly define the vector subspace of $V$ consisting of elements of \textbf{weight} $\mu \in \h^*$ to be:
            $$V_{\mu} := \{v \in V \mid \forall h \in \h: h \cdot v = \mu(h) v\}$$
        If we have a direct sum decomposition of $\h$-module:
            $$V \cong \bigoplus_{\mu \in \h^*} V_{\mu}$$
        then we will say that $V$ is a \textbf{weight module} for $\g$. Interestingly, elements of $\g_{\alpha}$ (with $\g$ acting on itself by the adjoint action) act by raising/lowering the weights of elements of $\g$-modules $V$ in the sense that:
            $$\g_{\alpha} \cdot V_{\mu} \subseteq V_{\mu + \alpha}$$
        for all weights $\alpha, \mu \in \h^*$. 
        
        As it turns out, $\g$ is a weight module over itself via the adjoint representation. This particular weight space decomposition is usually referred to as the \textbf{root space decomposition} of $\g$. 
        \begin{theorem}[Root space decomposition for finite-dimensional simple Lie algebras] \label{theorem: root_space_decomposition_for_finite_dimensional_simple_lie_algebras}
            Let $\g$ be a module over itself via the adjoint representation.
            \begin{enumerate}
                \item $\g$ is a weight module over itself.
                \item The weight space $\g_0$ is nothing but the Cartan subalgebra $\h$.
                \item For each non-zero weight $\alpha$ of this $\g$-module, $\dim_{\bbC} \g_{\alpha} = 1$.
            \end{enumerate}
        \end{theorem}
        Typically, the non-zero weights $\alpha$ of the adjoint representations of $\g$ such that $\g_{\alpha} \not \cong 0$ are called \textbf{roots}, and the subset of $\h^*$ consisting of such roots is denoted by:
            $$\Phi$$
        It is then possible to \textit{(non-canonically) choose} a partition of $\Phi$ into two disjoint subsets $\Phi^{\pm}$, consisting of \textbf{positive} and \textbf{negative} roots respectively. There are various ways of doing so, but for our purposes, we find it \textit{convenient} to make use of invariant and non-degenerate bilinear forms on $\g$ as ultimately, we will be very often concerned with how such bilinear forms pair elements of $\Phi$ together\footnote{Strictly speaking, this sentence does not make sense yet, since $\Phi \subset \h^* \subset \g^*$, while the bilinear forms mentioned are on $\g$. We shall elaborate shortly.}. Before moving on, however, let us note that for any root $\alpha \in \Phi$, we have that:
            $$[\g_{-\alpha}, \g_{\alpha}] \subset \h$$
        Let us also observe that:
            $$\g = [\g, \g]$$
        which is true per the assumption that $\g$ is simple, and implies that any element $h \in \h$ arises as $h = [x, y]$ for some $x \in \g_{-\alpha}, y \in \g_{\alpha}$ (for some $\alpha \in \Phi$). Together with the fact that $\h$ is finite-dimensional, meaning that $\h \cong \h^*$, these facts imply that the vector space $\h^*$ is spanned by the set $\Phi$.

        The \textbf{root lattice} of $\g$ is the $\Z$-module generated by the ($\Z$-linearly \textit{depedent}) set $\Phi$:
            $$Q := \Z \Phi$$
        and in light of the root space decomposition of $\g$, one sees that $\g$ is graded by the $\Z$-module $Q$. Let us also note that because $\Phi$ generated $\h^*$, the $\Z$-module $Q$ is actually a lattice inside $\h^*$, in the sense that:
            $$Q \tensor_{\Z} \bbC \cong \h^*$$
        thus justifying the terminology \say{root lattice}.

        Now, as eluded to above, one of the most important features of $\g$ is that it posses an invariant and non-degenerate $\bbC$-bilinear form:
            $$(-, -)_{\g}$$
        which is unique up to $\bbC^{\x}$-multiples. The canonical choice is the so-called Killing form, given by $\kappa(x, y) := \trace(\ad(x) \circ \ad(y))$ for all $x, y \in \g$, but in various other context, other natural choices such as the more general trace form $\trace(\rho(x) \rho(y))$ (associated to some representation $\rho: \g \to \gl(V)$; one recovers $\kappa$ by taking $\rho := \ad$) are also very useful. What is important to us is that the Killing form is essentially unique: if $\kappa'$ is any invariant and non-degenerate symmetric $\bbC$-bilinear form on $\g$ then there will exist a \textit{unique} $c \in \bbC^{\x}$ such that $\kappa' = c \kappa$.

        Now, such a bilinear form $(-, -)_{\g}$ helps us associate to $\g$ a \say{root system} (to be defined shortly), and the upshot is that these \say{roots systems} classify finite-dimensional simple Lie algebras (over algebraically closed fields of characteristic $0$) up to isomorphisms; again, we remark that root systems can be constructed in the absence of such a bilinear form, but we find it more convenient to make use of $(-, -)_{\g}$. A choice of \textbf{simple roots}:
            $$\{\alpha_i\}_{i \in \simpleroots}$$
        can then be made: this is to be a basis for the $\Z$-module $Q$ (cf. \cite[Subsection 10.1, p. 47]{humphreys_lie_algebras}) - and hence a basis for the vector space $\h^*$ - such that any root $\alpha \in \Phi$ of the form:
            $$\alpha = \sum_{i \in \simpleroots} m_i \alpha_i \in \Z\{\alpha_i\}_{i \in \simpleroots}$$
        where the coefficients $m_i \in \Z$ are (exclusively) either non-negative or non-positive. The sets:
            $$\Phi^{\pm} := \Phi \cap \pm \Z_{\geq 0} \{\alpha_i\}_{i \in \simpleroots}$$
        shall be referred to, respectively, as the sets of \textbf{positive} and \textbf{negative} roots. Elements of $Q^+ := \Z_{\geq 0} \{\alpha_i\}_{i \in \hat{\simpleroots}}$ are typically regarded as being \textbf{positive} (and in particular, the simple roots are positive by convention) and conversely, elements of $Q^- := \Z_{\leq 0} \{\alpha_i\}_{i \in \hat{\simpleroots}}$ are typically said to be \textbf{negative}. One can also easily show that $\Phi^{\pm} = \Phi \cap Q^{\pm}$. 

        Given an element:
            $$\mu := \sum_{i \in \simpleroots} m_i \alpha_i \in Q$$
        we define its \textbf{height} to be the sum of the coefficients:
            $$\height \mu := \sum_{i \in \simpleroots} m_i$$
        $Q$ is therefore partially ordered by heights. Now, to return briefly to the notion of weight modules for $\g$, such a $\g$-module $V$ is said to be of \textbf{highest-weight} if and only if it has a weight $\lambda$ such that:
            $$\g_{\alpha} \cdot V_{\lambda} = 0$$
        for all \textit{positive} roots $\alpha \in \Phi^+$. Now, a remarkable feature of the theory of finite-dimensional simple Lie algebras is that finite-dimensional simple modules thereof all possess highest weights (cf. \cite[Theorem 21.1, p. 112]{humphreys_lie_algebras}). Since $\g$ is a simple Lie algebra, it possesses no non-zero (proper) Lie ideal, i.e. no non-zero proper submodules (via the adjoint action), and hence is a simple module over itself. Per the aforementioned fact, $\g$ therefore has a highest weight, which is the \textbf{highest root}, typically denoted by:
            $$\theta$$
        This is a maximal element of the partially ordered $\Z$-module $Q$.
        
        The \textbf{Cartan matrix} of $\g$ can then be defined to be:
            $$C := (c_{ij})_{i, j \in \simpleroots} := \left( \frac{2(\alpha_i, \alpha_j)_{\g}}{(\alpha_i, \alpha_i)_{\g}} \right)_{i, j \in \simpleroots}$$
        It can be shown that $C$ can be symmetrised, in the sense that there exists an invertible diagonal matrix:
            $$D := (d_{ij})_{i, j \in \simpleroots} = \left(\frac{2\delta_{i, j}}{(\alpha_i, \alpha_j)_{\g}}\right)_{i, j \in \simpleroots}$$
        and a symmetric matrix:
            $$A := (a_{ij})_{i, j \in \simpleroots} = \left((\alpha_i, \alpha_j)_{\g}\right)_{i, j \in \simpleroots}$$
        (which is nothing but the matrix representation of the bilinear form $(-, -)_{\g}$ with respect to the basis $\{\alpha_i\}_{i \in \simpleroots}$), such that:
            $$C = DA$$

        From theorem \ref{theorem: root_space_decomposition_for_finite_dimensional_simple_lie_algebras}, we see that for any given positive root $\alpha \in \Phi^+$ and corresponding choices of root vectors\footnote{Choices of which are unique up to non-zero scalar multiples, since subspaces of non-zero weights are equally $1$-dimensional (see theorem \ref{theorem: root_space_decomposition_for_finite_dimensional_simple_lie_algebras}).} $x_{\pm\alpha} \in \g_{\pm\alpha}$, one has that:
            $$(h, [x_{\alpha}, x_{-\alpha}])_{\g} = ([h, x_{\alpha}], x_{-\alpha})_{\g} = \alpha(h) (x_{\alpha}, x_{-\alpha})_{\g}$$
        for all $h \in \h$. Per the non-degneracy of the bilinear form $(-, -)_{\g}$, there must then exist an element:
            $$\alpha^{\vee} := [x_{\alpha}, x_{-\alpha}]$$
        such that:
            $$\frac{(-, \alpha^{\vee})_{\g}}{(x_{\alpha}, x_{-\alpha})_{\g}} = \alpha$$
        as linear functionals on $\h$, since $[\g_{\alpha}, \g_{-\alpha}] \subseteq \h$. Now, because each of the root spaces $\g_{\alpha}$ is $1$-dimensional (cf. theorem \ref{theorem: root_space_decomposition_for_finite_dimensional_simple_lie_algebras}), and again because $[\g_{\alpha}, \g_{-\alpha}] \subseteq \h$ for any root $\alpha$, it is reasonable to expect, for each positive root $\alpha \in \Phi^+$, an injective Lie algebra homomorphism:
            $$\sl_2(\bbC) \hookrightarrow \g$$
        given by:
            $$x^{\pm} \mapsto x_{\pm \alpha}, h \mapsto \alpha^{\vee}$$
        with $x^{\pm}, h$ as in example \ref{example: sl_2}. To see that such a map indeed exists, note firstly that:
            $$[\alpha^{\vee}, x_{\pm \alpha}] = \pm \alpha(\alpha^{\vee}) x_{\pm \alpha}$$
        Without loss of generality, let us insist that the root vectors $x_{\pm \alpha}$ satisfy:
            $$(x_{-\alpha}, x_{\alpha})_{\g} = \frac{2}{(\alpha^{\vee}, \alpha^{\vee})_{\g}}$$
        (cf. \cite[Proposition 8.3]{humphreys_lie_algebras}). We see also that, without any loss of generality, we can furthermore normalise the bilinear form $(-, -)_{\g}$ such that:
            $$(x_{\alpha}, x_{-\alpha})_{\g} = 1$$
        for any root $\alpha \in \Phi$. Doing so yields:
            $$(\alpha^{\vee}, \alpha^{\vee})_{\g} = 2$$
        and so:
            $$2 = \frac{(\alpha^{\vee}, \alpha^{\vee})_{\g}}{(x_{\alpha}, x_{-\alpha})_{\g}} = \alpha(\alpha^{\vee})$$
        which implies that:
            $$[\alpha^{\vee}, x_{\pm \alpha}] = \pm 2 x_{\pm \alpha}$$
        which is the missing relation for $\sl_2(\bbC)$ (cf. example \ref{example: sl_2}).
        
        The next result is a fundamental theorem in the study of finite-dimensional simple Lie algebras over algebraically closed fields of characteristic $0$. It essentially asserts that to give such a Lie algebra via a presentation by generators and relations is the same as to specify its Cartan matrix. The result is not only practically useful, but also is the mean by which one approaches Kac-Moody algebras, where the Cartan matrix is no longer required to be positive-definite; we refer the reader to \cite[Chapters 1-8]{kac_infinite_dimensional_lie_algebras} for details. 
        \begin{theorem}[Serre's Theorem] \label{theorem: serre_theorem_for_finite_dimensional_simple_lie_algebras}
            $\g$ is isomorphic to the Lie algebra generated by the set:
                $$\{x_i^{\pm}, h_i\}_{i \in \simpleroots}$$
            whose elements are subjected to the following relations, given for all $i, j \in \simpleroots$:
                $$[h_i, h_j] = 0$$
                $$[h_i, x_j^{\pm}] = \pm c_{ij} x_j^{\pm}, [x_i^+, x_j^-] = \delta_{ij} h_i$$
            and for all $i \not = j \in \simpleroots$, there are also the so-called \textbf{Serre relations}:
                $$\ad(x_i^{\pm})^{1 - c_{ij}}(x_j^{\pm}) = 0$$
            This is usually referred to as the \textbf{Chevalley-Serre} presentation for $\g$, and the relations are usually referred to collectively as the \textbf{Chevalley-Serre relations}.

            The isomorphism in question is given by:
                $$x_i^{\pm} \mapsto x_{\pm \alpha_i}, h_i \mapsto \alpha_i^{\vee}$$
            where $x_{\pm \alpha_i} \in \g_{\pm \alpha_i}$ are root vectors such that:
                $$(x_{-\alpha_i}, x_{\alpha_i})_{\g} = 1$$
        \end{theorem}

        Next, one ought to also know that due to the PBW theorem, which is true for all universal enveloping algebras of Lie algebras, one has that:
            $$\rmU(\g) \cong \rmU(\n^-) \tensor_{\bbC} \rmU(\h) \tensor_{\bbC} \rmU(\n^+)$$

        Finally, let us also note that the \textbf{Borel subalgebras}:
            $$\b^{\pm} := \h \oplus \n^{\pm}$$
        (relative to our choices of a Cartan subalgebra $\h$ and positive/negative roots that determine $\n^{\pm}$) are also very important in practice, as we shall see shortly. Amongst other things, these subalgebras are means by which one can set up the theory of \textbf{highest-weight modules} for $\g$ that was referenced earlier in this subsection. In particular, one defines the \textbf{standard modules} (also called \textbf{Verma modules}) of highest-weight $\lambda \in \h^*$ as:
            $$\standard^{\lambda} := \rmU(\g)/\< \n^+, \h - \lambda \>$$
        The following basic properties of standard modules are well-known (see \cite[Sections 20 and 21]{humphreys_lie_algebras}):
        \begin{itemize}
            \item Using the PBW theorem and the tensor-hom adjunction, one sees that:
                $$\Delta^{\lambda} \cong \rmU(\g) \tensor_{\rmU(\b^+)} \1^{\lambda} \cong \rmU(\n^-)$$
            as left-$\rmU(\g)$-modules. Here, $\1^{\lambda}$ is the $1$-dimensional representation of $\b^+$ determined by the canonical composition $\b^+ \to \h \xrightarrow[]{\lambda} \bbC$.
            \item $\standard^{\lambda}$ is a weight-module; moreover, its weight spaces are finite-dimensional. The weights of $\standard^{\lambda}$ are all of the form:
                $$\lambda - \sum_{i \in \simpleroots} m_i \alpha_i$$
            for some $m_i \in \Z_{\geq 0}$. Furthermore, if $M$ is any left-$\rmU(\g)$-submodule of $\standard^{\lambda}$, then $M$ will also be a weight-module; in particular, its weight spaces are given by:
                $$M_{\mu} := \standard^{\lambda}_{\mu} \cap M$$
            \item Standard modules are cyclic, i.e. the subspace of weight $\lambda$ inside any $\standard^{\lambda}$ is $1$-dimensional. Any highest-weight vector of $\standard^{\lambda}$, i.e. any vector of weight $\lambda$, is a generator; these are sometimes called \textbf{maximal vectors}.
            \item Any standard module is indecomposable as a left-module over $\rmU(\g)$.
            \item Any standard module $\standard^{\lambda}$ contains a unique maximal left-$\rmU(\g)$-submodule, which we shall denote by $\simple^{\lambda}$; this can be shown by firstly showing that any \textit{proper} left-$\rmU(\g)$-submodule of $\standard^{\lambda}$ can not contain maximal vectors, and hence the unique maximal left-$\rmU(\g)$-submodule of $\standard^{\lambda}$ is nothing but the sum of all the proper left-$\rmU(\g)$-submodules of $\standard^{\lambda}$.
        \end{itemize}
        There is one last property, and due to its importance, we grant it theorem-hood. Before we state the theorem, let us also recall that a weight $\varpi \in \h^*$ is said to be \textbf{dominant} if and only if:
            $$\varpi(\alpha_i^{\vee}) \in \R_{\geq 0}$$
        and \textbf{integral} if:
            $$\varpi(\alpha_i^{\vee}) \in \Z$$
        both for all $i \in \simpleroots$. A \textbf{dominant integral} weight is a weight that is simultaneously dominant and integral, i.e.:
            $$\varpi(\alpha_i^{\vee}) \in \Z_{\geq 0}$$
        for all $i \in \simpleroots$. 
        
        When dealing with dominant weights, it is also useful to know about \textbf{fundamental weights}: the set of these elements - usually denoted by $\{\varpi_i\}_{i \in \simpleroots}$ - of $\h^*$ make up a basis dual to the basis $\{\alpha_i^{\vee}\}_{i \in \simpleroots}$ of $\h$, i.e.:
            $$\varpi_i(\alpha_j^{\vee}) := \delta_{i, j}$$
        Let us write:
            $$\Lambda := \Z \cdot \{\varpi_i\}_{i \in \simpleroots}$$
        and refer to this as the \textbf{weight lattice} of $\g$. What can be shown is that a weight $\varpi \in \h^*$ is integral if and only if:
            $$\varpi \in \Lambda$$
        and that it is dominant integral if and only if:
            $$\varpi \in \Lambda^+ := \Z_{\geq 0} \cdot \{\varpi_i\}_{i \in \simpleroots}$$
        Interestingly, this coincides with $Q^+$, i.e. there is a semi-group isomorphism:
            $$\Lambda^+ \cong Q^+$$
        mapping $\varpi_i$ to $\alpha_i$, though at the same time, the quotient:
            $$\Lambda/Q$$
        is usually only a finite $\Z$-module, which may be non-zero.
        \begin{theorem}[Classification of finite-dimensional simple left-$\rmU(\g)$-modules] \label{theorem: classification_of_finite_dimensional_simple_modules_over_finite_dimensional_simple_lie_algebras}
            $\standard^{\varpi}$ has a unique simple left-$\rmU(\g)$-module quotient. This simple quotient is finite-dimensional if and only if $\varpi$ is dominant and integral. Furthermore, any simple left-$\rmU(\g)$-module arises in this way, and if $\varpi, \varpi' \in Q$ are weights, then:
                $$\simple^{\varpi} \cong \simple^{\varpi'} \iff \varpi = \varpi'$$
        \end{theorem}
        For any $i \in \simpleroots$, we call the finite-dimensional module:
            $$\simple^{\varpi_i}$$
        the \textbf{$i^{th}$ fundamental module}.

        It is also worth knowing the following two structural facts about the categories of left/right-$\rmU(\g)$-modules.
        \begin{itemize}
            \item $\rmU(\g)$ (or for that matter, the universal enveloing algebra of any Lie algebra) has a Hopf algebra structure whose comultiplication, counit, and antipode are respectively given on all primitive elements $x \in \g = \prim(\rmU(\g))$ by:
                $$\Box(x) := x \tensor 1 + 1 \tensor x$$
                $$\e(x) := 1$$
                $$S(x) := -x$$
            This is a cocommutative Hopf algebra, and hence the category ${}^l\rmU(\g)\mod$ (likewise, the category ${}^r\rmU(\g)\mod$) is symmetric monoidal with respect to $\tensor_{\bbC}$ (this is true for any cocommutative Hopf algebra; see \cite[Proposition III.5.1]{kassel_quantum_groups}). Moreover, because the antipode is invertible, these categories are also equivalent; in particular, the left and right actions of any given primitive element differ only by a sign. For this reason, we will usually abuse notations slightly and write:
                $$\rmU(\g)\mod$$
            and speak only of \say{$\g$-modules} or \say{$\rmU(\g)$-modules}.
            \item The category $\rmU(\g)\mod^{\fd}$ of finite-dimensional $\rmU(\g)$-modules is semi-simple. This is a theorem of Weyl.
        \end{itemize} 

        If $V$ is a highest-weight module for $\g$, then its set of weights will be denoted by:
            $$\Pi(V)$$
        ($\Pi$ for \say{poid}, French for \say{weight}). Observe that if $V := \bigoplus_{\lambda \in \Pi(V)} V_{\lambda}, W := \bigoplus_{\mu \in \Pi(W)} W_{\mu}$ are two weight modules for $\g$, then:
            $$V \tensor_{\bbC} W \cong  \bigoplus_{(\lambda, \mu) \in \Pi(V) \x \Pi(W)} V_{\lambda} \tensor_{\bbC} W_{\mu}$$
        and since:
            $$h \cdot (v \tensor w) = (h \cdot v) \tensor w + v \tensor (h \cdot w) = (\lambda + \mu)(h) \cdot (v \tensor w)$$
        for all $h \in \h$ and all $v \in V, w \in W$, we have that:
            $$\Pi(V \tensor_{\bbC} W) = \Pi(V) + \Pi(W)$$
        From this, we gather that finite-dimensional $\g$-modules, corresponding to dominant integral weights $\varpi \in \Lambda$, can be constructed using tensor products of fundamental modules.  

        We end this subsection with a brief analysis of the easiest possible example of a finite-dimensional simple Lie algebra. 
        \begin{example}[$\sl_2(\bbC)$] \label{example: sl_2}
            Recall that $\sl_2(\bbC)$ is the kernel of the trace map:
                $$\trace: \gl_2(\bbC) \to \bbC$$
            i.e. it is the Lie algebra of trace-zero $2 \x 2$-matrices whose Lie bracket is the usual commutator of matrices. It is of dimension $\dim_{\bbC} \gl_2(\bbC) - \dim_{\bbC} \bbC = 4 - 1 = 3$, and a common choice of basis is:
                $$\left\{ h := \begin{pmatrix} 1 & 0 \\ 0 & -1 \end{pmatrix}, x^+ := \begin{pmatrix} 0 & 1 \\ 0 & 0 \end{pmatrix}, x^- := \begin{pmatrix} 0 & 0 \\ 1 & 0 \end{pmatrix} \right\}$$
                
            Now, observe that:
                $$[h, x^{\pm}] = \pm 2 x^{\pm}, [x^+, x^-] = h$$
            From this, one sees that:
                $$\sl_2(\bbC)_0 = \bbC h, \sl_2(\bbC)_{\pm 2} = \bbC x^{\pm}$$
            which in particular, implies that $\bbC h$ is a Cartan subalgebra and that the root space decomposition of $\sl_2(\bbC)$ takes the form:
                $$\sl_2(\bbC) = \bbC x^- \oplus \bbC h \oplus \bbC x^+$$
            One notes also, that the relations $[h, x^{\pm}] = \pm 2 x^{\pm}, [x^+, x^-] = h$ are precisely the Chevalley-Serre relations for $\sl_2(\bbC)$, and so the set $\{h, x^{\pm}\}$ is a set of Chevalley-Serre generators for $\sl_2(\bbC)$.
            
            In this case, the Cartan matrix is just:
                $$\begin{pmatrix} 2 \end{pmatrix}$$

            From the above, it is easy to see that for $\sl_2(\bbC)$, the weight lattice is nothing but $\Z$, and the sub-semi-group of dominant integral weights is $\Z_{\geq 0}$.
        \end{example}

    \subsection{Drinfeld's quantum doubles and finite QUEs}
        From a purely mathematical point of view, the study of quantum groups has roots in the study of deformations of modules over Lie algebras. We have written more about this in \cite{quantum_double} (see also \cite{etingof_kazhdan_quantisation_1}), so for the sake of directly introducing quantised enveloping algebras (QUEs), let us simply say that our interest lies in relaxing the symmetric monoidal structure on $\rmU(\g)\mod$ into a \textit{braided} monoidal structure on a distinguished category $\rmU_q(\g)\mod$. From this category, the QUE $\rmU_q(\g)$ can be reconstructed (in the Tannakian sense) and shall also come automatically equipped with a \textit{non-cocommutative} Hopf structure per some abstract nonsense from the theory of braided monoidal categories (see \cite{EGNO}).

        One way of transforming symmetric monoidal structures into braided monoidal structures is to make use of the constructions of so-called Drinfeld doubles of Hopf algebras and Drinfeld centres of monoidal categories. To be slightly more precise, we are trying to realise QUEs, say $U$ for now, as Hopf algebras so that:
            $$U\mod \cong \Dr(U^{\frac12})\mod \cong \calZ(U^{\frac12}\mod)$$
        where:
        \begin{itemize}
            \item $\Dr(U^{\frac12})$ denotes the \textbf{Drinfeld quantum double} (also called the \textbf{Hopf double}) of a certain Hopf algebra $U^{\frac12}$, and
            \item $\calZ(\C)$ denotes the \textbf{Drinfeld centre} (or simple \textbf{centre}) of a monoidal category $\C$, which is abstractly \textit{braided}.
        \end{itemize}
        That said, where does the Hopf algebra $U^{\frac12}$ come from in the first place, so that we can even perform this quantum doubling procedure to begin with ? In some sense, this has to be found \say{by hand}, i.e. be written down via generators and relations. One can then check that such an algebra $U^{\frac12}$ is indeed the \say{correct} one to consider by verifying that its classical limit coincides with whatever Lie bialgebra that one is trying to quantise (see \cite{etingof_kazhdan_quantisation_1} for more details).

        For us, the \say{half-algebra} $U^{\frac12}$ in the schematic above shall be a certain deformation $\rmU_q(\b^+)$ of $\rmU(\b^+)$, i.e. a certain quantisation of a Lie bialgebra structure on $\b^+$. We begin our description of this construction by first stating the Lie bialgebra on $\b^+$ that we are trying to quantise. 

        To this end, recall firstly that a finite-dimensional \textbf{Manin triple} is a triple of Lie algebras:
            $$(\p, \p^+, \p^-)$$
        in which:
        \begin{itemize}
            \item $\p$ is a finite-dimensional Lie algebra with a non-degenerate and invariant symmetric bilinear form $(-, -)$,
            \item $\p^{\pm} \subseteq \p$ are Lie subalgebras paired \textbf{isotropically} by $(-, -)$, i.e. $(\p^{\pm}, \p^{\pm}) = 0$ while $(\p^-, \p^+) \not = 0$, and
            \item $\p^{\pm}$ are linearly dual to one another via $(-, -)$.
        \end{itemize}
        From finite-dimensional simple Lie algebras such as $\g$, one can construct canonical Manin triples:
            $$(\frakDr(\b^{\pm}), \b^+, \b^-)$$
        \begin{itemize}
            \item As vector spaces, one has that:
                $$\frakDr(\b^{\pm}) \cong \g \oplus \h$$
            and this is equipped with the pairing:
            \item 
            \item 
        \end{itemize}

        Now, onto the aforementioned quantisation of $\b^+$. Since $\b^+ \cong \b^-$ via the bilinear form $(-, -)_{\g}$ and hence $\rmU(\b^+) \cong \rmU(\b^-)$ as Hopf algebras, we can uniformly define $\rmU_q(\b^+)$ and $\rmU_q(\b^-)$ with only some minor sign differences between the relations. 
        \begin{definition}[The quantum Borel subalgebras $\rmU_q(\b^{\pm})$] \label{def: quantum_borel_subalgebras}
            Let $q$ be an indeterminate.
        
            As associative $\bbC$-algebras, $\rmU_q(\b^{\pm})$ are the algebras generated, respectively, by the sets:
                $$\{ X_i^+, K_i^+ \}_{i \in \simpleroots}, \{ X_i^-, K_i^- \}_{i \in \simpleroots}$$
            whose elements are subjected to the relations, given for all $i, j \in \simpleroots$:
                $$[K_i^{\pm}, K_j^{\pm}] = 0$$
                $$K_i^{\pm} X_j^{\pm} = q^{a_{ij}} X_j^{\pm} K_i^{\pm}$$
            and for all $i \not = j \in \simpleroots$, there are also the \textbf{quantum Serre relations}:
                $$\sum_{s = 0}^{1 - a_{ij}} (-1)^s \left[\binom{1 - a_{ij}}{s}\right]_q (X_i^{\pm})^{(1 - a_{ij}) - s} X_j^{\pm} X_i^s = 0$$
        \end{definition}
        Clearly, there is an algebra isomorphism:
            $$\rmU_q(\b^+) \xrightarrow[]{\cong} \rmU_q(\b^-)$$
        given by:
            $$X_i^+ \mapsto X_i^-, K_i^+ \mapsto K_i^-$$
        \begin{convention}
            It will also be convenient for us to consider a certain algebra:
                $${}^{\not \integrable}\rmU_q(\b^{\pm})$$
            generated by the same elements as those that generate $\rmU_q(\b^{\pm})$, but now, we do not impose the quantum Serre relations. The notation is to reflect the fact that without the Serre relations, one can not obtain so-called \say{integrable modules}.
        \end{convention}
        \begin{lemma}[Bialgebra structures on ${}^{\not \integrable}\rmU_q(\b^{\pm})$] \label{lemma: bialgebra_strructures_on_non_integrable_quantum_borel_subalgebras}
            There are $\bbC$-bialgebra structures on ${}^{\not \integrable}\rmU_q(\b^{\pm})$ whose comultiplications $\Box_q^{\pm}$ and counits $\e_q^{\pm}$ are given by:
                $$\Box_q^{\pm}(X_i^+) := X_i^+ \tensor 1 + K_i^+ \tensor X_i^+$$
                $$\Box_q^{\pm}(X_i^-) := X_i^- \tensor K_i^- + 1 \tensor X_i^-$$
                $$\Box_q(K_i^{\pm}) := K_i^{\pm} \tensor K_i^{\pm}$$
                $$\e_q(X_i^{\pm}) := 0$$
                $$\e_q(K_i^{\pm}) := 1$$
        \end{lemma}
            \begin{proof}
                
            \end{proof}
        \begin{proposition}[Bialgebra structures on $\rmU_q(\b^{\pm})$] \label{prop: bialgebra_structures_on_quantum_borel_subalgebras}
            There are $\bbC$-bialgebra structures on $\rmU_q(\b^{\pm})$ given by the same formulae as in \ref{lemma: bialgebra_strructures_on_non_integrable_quantum_borel_subalgebras}.
        \end{proposition}
            \begin{proof}
                The only thing to do is to find a bi-ideal $I^{\pm} \subseteq {}^{\not \integrable}\rmU_q(\b^{\pm})$ so that we can realise $\rmU_q(\b^{\pm})$ as a quotient bialgebra of ${}^{\not \integrable}\rmU_q(\b^{\pm})$. To this end, let $I^{\pm}$ firstly be the ideal generated by $\{X_i^{\pm}\}_{i \in \simpleroots}$. It is then clear that:
                    $$\Box_q(I^{\pm}) \subseteq {}^{\not \integrable}\rmU_q(\b^{\pm}) \tensor_{\bbC} I^{\pm} + I^{\pm} \tensor_{\bbC} {}^{\not \integrable}\rmU_q(\b^{\pm})$$
                    $$\e_q(I^{\pm}) = 0$$
                and hence we are done.
            \end{proof}
        \begin{corollary}[Classical limits of $\rmU_q(\b^{\pm})$]
            The Lie bialgebras $\b^{\pm}$ are classical limits of $\rmU_q(\b^{\pm})$, in the sense that:
                $$\delta^{\pm}(x) \equiv \frac{1}{\log q}( \Box_q^{\pm} - \Box_q^{\pm, \cop} )(X) \pmod{\log q}$$
            for all $x \in \b^{\pm}$ and all lifts $X \equiv x \pmod{\log q}$. 
        \end{corollary}
            \begin{proof}
                
            \end{proof}
        \begin{proposition}[Uniqueness of quantum Borel subalgebras] \label{prop: quantum_borel_algberas_uniqueness}
            As quantisations of $\b^{\pm}$, the bialgebras $\rmU_q(\b^{\pm})$ are unique up to isomorphisms. 
        \end{proposition}
            \begin{proof}
                By \cite{etingof_kazhdan_quantisation_1}, this assertion is equivalent to saying that:
                    $$\dim_{\bbC} H^2_{\Lie}(\b^{\pm}, \bigwedge^2 \b^{\pm}) = 1$$
            \end{proof}
        \begin{theorem}[The Drinfeld-Jimbo quantum group $\rmU_q(\g)$]
            The classical double $\frakDr(\b^{\pm})$ admits the quantum double of $\rmU_q(\b^{\pm})$, which shall be denoted by $\rmU_q(\g)$, as a quantisation.

            As a $\bbC$-algebra, $\rmU_q(\g)$ is isomorphic to the algebra generated by the set:
                $$\{X_i^{\pm}, K_i^{\pm}\}_{i \in \simpleroots}$$
            whose elements are subjected to the following relations, in addition to the ones in definition \ref{def: quantum_borel_subalgebras}:
                $$K_i^{\pm} K_i^{\mp} = 1$$
                $$[X_i^+, X_j^-] = \delta_{i, j} \frac{K_i^+ - K_i^-}{q - q^{-1}}$$
            The relations above are given for all $i, j \in \simpleroots$.

            Moreover, as a $\bbC$-bialgebra, $\rmU_q(\g)$ carrries an additional antipode:
                $$S_q: \rmU_q(\g) \to \rmU_q(\g)^{\op, \cop}$$
            given by:
                $$S_q(X_i^+) := -K_i^- X_i^+$$
                $$S_q(X_i^-) := -X_i^- K_i^+$$
                $$S_q(K_i^{\pm}) := K_i^{\mp}$$
            making it a Hopf algebra. 
        \end{theorem}
            \begin{proof}
                
            \end{proof}

        Actually, there is a Hopf algebra structure on $\rmU_q(\b^{\pm})$, provided that said Hopf algebra structure is with respect to a monoidal structure on the category of $\bbC$-vector spaces different from the usual one. This new monoidal structure, instead of being symmetric, will be \textit{braided}.

     \subsection{Basic structural properties of finite QUEs}
        For convenience, let us introduce the following abbreviations:
            $$\rmU := \rmU(\g)$$
            $$\rmU^0 := \rmU(\h), \rmU^{\pm} := \rmU(\n^{\pm})$$
            $$\rmU^{\leq 0} := \rmU^- \tensor_{\bbC} \rmU^0, \rmU^{\geq 0} := \rmU^0 \tensor_{\bbC} \rmU^+$$
        and likewise for their quantised counterparts:
            $$\rmU_q := \rmU_q(\g)$$
            $$\rmU_q^{\pm 0} := \{ \text{subalgebra generated by $K_i^+$ or $K_i^-$} \}, \rmU_q^0 := \{ \text{subalgebra generated by $K_i^+$ and $K_i^-$} \}$$
            $$\rmU_q^{\pm} := \{ \text{subalgebra generated by $X_i^{\pm}$} \}$$
            $$\rmU_q^{\leq 0} := \rmU_q^- \tensor_{\bbC} \rmU_q^0, \rmU_q^{\geq 0} := \rmU_q^0 \tensor_{\bbC} \rmU_q^+$$