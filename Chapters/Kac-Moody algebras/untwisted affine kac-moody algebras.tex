\section{Untwisted affine Kac-Moody algebras as universal central extensions}
    \begin{convention}
        From this point on, all Kac-Moody algebras shall be assumed to be reduced (cf. definition \ref{def: reduced_kac_moody_algebras}) unless stated explicitly to be otherwise.
    \end{convention}

    \begin{definition}[Untwisted affine Kac-Moody algebras] \label{def: untwisted_affine_kac_moody_algebras}
        Any Kac-Moody algebra associated to an untwisted affine GCM (cf. definition \ref{def: untwisted_affine_generalised_cartan_matrices_and_dynkin_quivers}) is also called an \textbf{untwisted affine Kac-Moody algebra}.
    \end{definition}
    \begin{remark}[Derived subalgebras and centres of untwisted affine Kac-Moody algebras] \label{remark: derived_subalgebras_and_centres_of_untwisted_affine_kac_moody_algebras}
        Let us now provide a bit of (somewhat \textit{post hoc}) justification for why an analysis of derived subalgebras of reduced Kac-Moody algebras (cf. corollary \ref{coro: derived_subalgebras_of_kac_moody_algebras} and proposition \ref{prop: centres_of_reduced_kac_moody_algebras}) might be of any interest. 
        
        To this end, suppose firstly that $\g$ is an untwisted affine Kac-Moody algebra whose corresponding GCM is $A \in \Mat_n(\Z)$. Secondly, note that proposition \ref{prop: centres_of_reduced_kac_moody_algebras} tells us that $\g$ and $[\g, \g]$ have equal centres, which are $1$-dimensional as a result of $A$ being of affine type and therefore is of rank $n - 1$ (cf. proposition \ref{prop: affineness_criterion_via_radicals}). By fixing a non-zero central element $c \in \z([\g, \g])$ and using the universal property of principal central extensions (cf. proposition \ref{prop: principal_central_extensions_of_simply_affine_lie_algbebras_are_universal}), one sees thus that $[\g, \g]$ arises as the following principal central extension of $\bar{\g}[t^{\pm 1}] := \bar{\g} \tensor_k k[t^{\pm 1}]$:
            $$0 \to \span_k \{c\} \to [\g, \g] \to \bar{\g}[t^{\pm 1}] \to 0$$
        wherein $\bar{\g} := \frakLie(\bar{A})$ is the \textit{a priori} finite-type Kac-Moody $k$-algebra (i.e. simple finite-dimensional Lie $k$-algebra) associated to an appropriate choice of an \textit{a priori} invertible $(n - 1) \x (n - 1)$ principal minor of $A$ (cf. lemma \ref{lemma: tameness_criterion_via_determinants}).
        
        Now, a combination of corollary \ref{coro: derived_subalgebras_of_kac_moody_algebras} and theorem \ref{theorem: serre_relations_generate_radical_of_non_reduced_kac_moody_algebras} informs us that the derived subalgebra $[\g, \g] \subseteq \g$ is the Lie $k$-algebra generated by the elements of $\{\alpha_i^{\vee}, e_i^-, e_i^+\}_{1 \leq i \leq n}$, which are subjected to the relations:
            $$\forall 1 \leq i, j \leq n: [\alpha_i^{\vee}, \alpha_j^{\vee}] = 0$$
            $$\forall 1 \leq i \leq n: [\alpha_i^{\vee}, e_i^{\pm}] = \mp B(\alpha_i, \alpha_i^{\vee}) e_i^{\pm} = \mp a_{ij} e_i^{\pm}$$
            $$\forall 1 \leq i, j \leq n: [e_i^-, e_j^+] = \delta_{ij} \alpha_i^{\vee}$$
            $$\forall 1 \leq i, j \leq n: \ad(e_i^{\pm})^{1 - a_{ij}}(e_j^{\pm}) = 0$$
        One thus sees that $[\g, \g]$ is generated by $3(n - 1)$ Chevalley generators This discrepancy in the number of Chevalley generators of $\g$ and $[\g, \g]$ results in there being a non-central element $\del_t \in \g \setminus \z(\g)$ such that:
            $$\g \cong [\g, \g] \oplus \span_k \{\del_t\}$$
        For cohomological reasons, $\del_t$ shall have to be a degree-preserving outer derivation on $[\g, \g]$ (cf. proposition \ref{prop: lie_algebra_outer_derivations_are_parametrised_by_HH1}).
    \end{remark}

    \subsection{(Extended) simply affine Lie algebras}
        In this subsection, we construct a very particular Lie algebra $\hat{\g}_{\kappa}$ from the original simple finite-dimensional Lie algebra $\g$. We shall see, ultimately in theorem \ref{theorem: untwisted_affinisation}, that $\hat{\g}_{\kappa}$ is in fact an affine Kac-Moody algebra. 
    
        \subsubsection{Invariant bilinear forms and universal central extensions}
            \begin{definition}[Central extensions of Lie algebras] \label{def: central_extensions_of_lie_algebras}
                Suppose that $\g$ is a Lie algebra over some commutative ring $k$. A central extension thereof is then another Lie algebra $\breve{\g}$ over $k$ fitting into a short exact exact sequence of $k$-modules as follows:
                    $$0 \to \z \to \breve{\g} \to \g \to 0$$
                wherein $\z \subset \z(\breve{\g})$ is a Lie subalgebra of the centre $\z(\breve{\g})$ of $\breve{\g}$.
            \end{definition}
            \begin{definition}[Invariant bilinear forms] \label{def: invariant_bilinear_forms}
                Suppose that $\g$ is a Lie algebra over some commutative ring $k$. A $k$-bilinear form $\psi$ on $\g$, viewed as a $k$-linear map $\psi: \g \tensor_k \g \to k$ (via the universal property of tensor products of $k$-modules), is said to be \textbf{$\g$-invariant} if $\psi([x, y], z) = \psi(x, [y, z])$ for all $x, y, z \in \g$.
            \end{definition}
            \begin{example}[Killing forms]
                The Killing form on any finite-dimensional complex Lie algebra $\g$ is an instance of an invariant $\bbC$-bilinear form.
            \end{example}
            \begin{remark}[Spaces of invariant bilinear forms] \label{remark: spaces_oe^-_invariant_bilinear_forms}
                Suppose that $\g$ is a Lie algebra over some commutative ring $k$. Then, the space of all $\g$-invariant bilinear forms on $\g$ is precisely and automatically $\Hom_k(\g \tensor_k \g, k)^{\g}$, the $k$-submodule of $\Hom_k(\g \tensor_k \g, k)$ spanned by $\g$-fixed elements. 
            \end{remark}
            \begin{proposition}[Simple finite-dimensional Lie algebras only have the Killing form] \label{prop: simple_finite_dimensional_lie_algebras_only_have_the_killing_form}
                Let $k$ be a field of characteristic $0$ and $\g$ be a simple finite-dimensional\footnote{In fact, this is true for all tame reduced Kac-Moody $k$-algebras (cf. proposition \ref{prop: uniqueness_of_generalised_killing_forms}), among which the simple finite-dimensional Lie $k$-algebras are the finite-type ones.} Lie algebra over $k$. Then every non-degenerate and symmetric $\g$-invariant $k$-bilinear form $\psi: \g \x \g \to k$ has the form $\psi := \xi \kappa$ for some $\xi \in k^{\x}$, with $\kappa$ being the Killing form (and is thus symmetric and non-degenerate as well).
            \end{proposition}
                \begin{proof}
                    This is a corollary of proposition \ref{prop: uniqueness_of_generalised_killing_forms}.
                \end{proof}
            \begin{lemma}[$\HH^2$ and invariant bilinear forms] \label{lemma: second_cohomology_parametrises_invariant_bilinear_forms}
                Let $\g$ be a Lie algebra over some commutative ring $k$. There is a canonical bijection $\HH^2_k(\g, k) \cong \Hom_k(\g \tensor_k \g, k)^{\g}$.
            \end{lemma}
                \begin{proof}
                    Straightforward from the fact that for all $V \in \Ob(\Rep_k(\g))$, the elements $\psi \in \HH^2_k(\g, V)$ are in bijection with equivalence classes of extensions of $\g$ by $V$ in $k\mod$ (cf. proposition \ref{prop: lie_algebra_extensions_are_parametrised_by_HH2}).
                \end{proof}
            \begin{lemma}[Second Whitehead Lemma] \label{lemma: second_whitehead_lemma}
                \cite[Proposition VII.6.3]{hilton_stammbach_homological_algebra} Let $k$ be a field of characteristic $0$ and $\g$ be a simple Lie algebra over $k$. Then, for any $V \in \Ob({}^l\rmU(\g)\mod)$ which is finite-dimensional over $k$, one has $\HH^2_k(\g, V) \cong 0$.
            \end{lemma}
            \begin{proposition}[Killings forms and universal central extensions] \label{prop: killing_forms_and_universal_central_extensions}
                Let $k$ be a field of characteristic $0$ and $\g$ be a simple finite-dimensional Lie algebra over $k$, and denote the Killing form thereon by $\kappa: \g \tensor_k \g \to k$. Then, one has a canonical bijection $\HH^2_k(\g, k) \cong \{\kappa\}$.
            \end{proposition}
                \begin{proof}
                    That there exists a bijection $\HH^2_k(\g, k) \cong \{\kappa\}$ is trivial after one applies lemma \ref{lemma: second_cohomology_parametrises_invariant_bilinear_forms} to lemma \ref{lemma: second_whitehead_lemma}. As for the canonicity of this bijection, apply proposition \ref{prop: simple_finite_dimensional_lie_algebras_only_have_the_killing_form}.
                \end{proof}
                
            \begin{definition}[Loop algebras] \label{def: loop_algebras}
                Let $\g$ be a Lie algebra over some commutative ring $k$. The associated \textbf{loop algebra}\footnote{Not to be confused with the \textit{formal} loop algebra $\g(\!(t)\!) := \g \tensor_k k(\!(t)\!)$.} is then the $k$-module $\g[t^{\pm 1}] := \g \tensor_k k[t^{\pm 1}]$ equipped with the $k$-bilinear brackets $[-, -]_{\g[t^{\pm 1}]}$ given by $[x \tensor f, y \tensor g]_{\g[t^{\pm 1}]} := [x, y]_{\g} \tensor fg$, for all pure tensors\footnote{It suffices to only define $[-, -]_{\g[t^{\pm 1}]}$ for pure tensors, since elements of tensor products are finite linear combinations of pure tensors; one can then make use of the bilinearity of Lie brackets.} $x \tensor f, y \tensor g \in \g \tensor_k k[t^{\pm 1}]$.
            \end{definition}
            \begin{remark}[Killing forms on loop algebras] \label{remark: killing_forms_on_loop_algebras}
                Let $\g$ be a Lie algebra over some commutative ring $k$ and denote the Killing form thereon by $\kappa: \g \tensor_k \g \to k$. Then, as a direct consequence of the definition of the Lie bracket on the loop algebra $\g[t^{\pm 1}]$, there is an induced symmetric $k$-bilinear form $\kappa[t^{\pm 1}]: \g[t^{\pm 1}] \tensor_k \g[t^{\pm 1}] \to k$ thereon, given by:
                    $$\kappa[t^{\pm 1}](x \tensor f, y \tensor g) := \kappa(x, y) fg$$
                When $k$ is a field of characteristic $0$, it is also easy to see that $\kappa[t^{\pm 1}]$ is non-degenerate. 
            \end{remark}
            \begin{proposition}[Principal central extensions of simply affine Lie algebras are universal] \label{prop: principal_central_extensions_of_simply_affine_lie_algbebras_are_universal}
                Let $k$ be a field of characteristic $0$ and $\g$ be any simple finite-dimensional Lie algebra over $k$. Then there is a canonical bijection $\HH^2_k(\g[t^{\pm 1}], k) \cong \{\kappa[t^{\pm 1}]\}$.
            \end{proposition}
                \begin{proof}
                    This is a straightforward consequence of proposition \ref{prop: killing_forms_and_universal_central_extensions} and the construction of $\kappa[t^{\pm 1}]$ (cf. remark \ref{remark: killing_forms_on_loop_algebras}).
                \end{proof}
            \begin{definition}[Simply affine Lie algebra] \label{def: simply_affine_lie_algebras}
                Let $k$ be a field of characteristic $0$ and $\g$ be any simple finite-dimensional Lie algebra over $k$, and denote the Killing form thereon by $\kappa$. Then, the universal principal central extension (with $\kappa[t^{\pm 1}]$ as in remark \ref{remark: killing_forms_on_loop_algebras}):
                    $$0 \to \span_k \{\kappa[t^{\pm 1}]\} \to \breve{\g}_{\kappa} \to \g[t^{\pm 1}] \to 0$$
                gives rise to what is known as \textit{the}\footnote{Of course, $\breve{\g}_{\kappa}$ is unique up to isomorphisms!} \textbf{simply affine Lie algebra} attached to $\g$ over $k$.
            \end{definition}
            \begin{remark}[Lie brackets on simply affine Lie algebras] \label{remark: lie_brackets_on_simply_affine_lie_algebras}
                Let $k$ be a field of characteristic $0$ and $\g$ be any simple Lie algebra over $k$; denote the Killing form thereon by $\kappa$ and the Killing form on the loop algebra $\g[t^{\pm 1}]$ by $\kappa[t^{\pm 1}]$. By applying corollary \ref{coro: lie_brackets_on_principal_central_extensions} to proposition \ref{prop: principal_central_extensions_of_simply_affine_lie_algbebras_are_universal}, one sees that the Lie bracket on $\breve{\g}_{\kappa}$ can be induced from that on $\g[t^{\pm 1}]$ (or rather, that on $\g$) in the following manner for all $x \tensor f, y \tensor g \in \g[t^{\pm 1}]$ and all $\mu, \nu \in k$:
                    $$
                        \begin{aligned}
                            [x \tensor f + \mu c, y \tensor g + \nu c]_{\breve{\g}_{\kappa}} & = [x \tensor f, y \tensor g]_{\g[t^{\pm 1}]} + c \kappa[t^{\pm 1}](x \tensor f, y \tensor g)
                            \\
                            & = [x, y]_{\g} \tensor fg + c \kappa(x, y) fg
                        \end{aligned}
                    $$
                wherein $ \in \span_k \{\kappa[t^{\pm 1}]\}$ is any fixed non-zero element (and note that $ \in \z(\breve{\g}_{\kappa})$ since all $1$-dimensional Lie algebras are \textit{a priori} abelian and isomorphic to one another). 
            \end{remark}
            \begin{remark}[Central charges and inner derivations] \label{remark: central_charges_and_inner_derivations}
                A more intrinsic (though less motivated and contextual) definition of simply affine Lie algebras is as follows: given any simple finite-dimensional Lie algebra over a field $k$ of characteristic $0$, a corresponding simply affine Lie algebra $\breve{\g}_c$ with so-called \textbf{central charge} $c \in k^{\x}$ can then arise as the following extension:
                    $$0 \to \span_k \{c\} \to \breve{\g}_c \to \g[t^{\pm 1}] \to 0$$
                Observe using proposition \ref{prop: principal_central_extensions_of_simply_affine_lie_algbebras_are_universal} that this is certainly a well-posed definition: in fact, the proposition tells us that for any choice of a central charge $c \in k^{\x}$, there is an isomorphism of extensions:
                    $$
                        \begin{tikzcd}
                        	0 & {\span_k \{c\}} & {\breve{\g}_c} & {\g[t^{\pm 1}]} & 0 \\
                        	0 & {\span_k \{\kappa[t^{\pm 1}]\}} & {\breve{\g}_{\kappa}} & {\g[t^{\pm 1}]} & 0
                        	\arrow[from=2-1, to=2-2]
                        	\arrow[from=2-2, to=2-3]
                        	\arrow[from=2-3, to=2-4]
                        	\arrow[from=2-4, to=2-5]
                        	\arrow[from=1-1, to=1-2]
                        	\arrow[from=1-2, to=1-3]
                        	\arrow[from=1-3, to=1-4]
                        	\arrow[from=1-4, to=1-5]
                        	\arrow[Rightarrow, no head, from=2-4, to=1-4]
                        	\arrow["\cong", from=1-2, to=2-2]
                        	\arrow["\cong", from=1-3, to=2-3]
                        \end{tikzcd}
                    $$
            \end{remark}
            
        \subsubsection{Extended simply affine Lie algebras}
            \begin{definition}[Extended simply affine Lie algebras] \label{def: extended_simply_affine_lie_algebras}
                Assume convention \ref{conv: simple_finite_dimensional_lie_algebras} and consider the simply affine Lie algebra $\breve{\g}_{\kappa}$. An \textbf{extended simply affine Lie algebra} can then be defined to be the Lie $k$-algebra $\hat{\g}_{\kappa}$ whose underlying $k$-vector space is $\breve{\g}_{\kappa} \oplus \span_k \{\del_t\}$, with $\del_t \in \out_k(\breve{\g}_{\kappa})$ being some outer derivation that acts as a differential operator (with respect to the variable $t$) on the $k[t^{\pm 1}]$ component of $\g \tensor_k k[t^{\pm 1}]$ and as zero on $\span_k \{c\}$, i.e.:
                    $$\del_t(x \tensor f + \mu c) := x \tensor \del_t(f)$$
            \end{definition}
            \begin{convention}
                One usually considers the outer derivation $\del_t := t\frac{d}{dt}$ (hence why the subscript \say{$t$}). 
            \end{convention}
            \begin{remark}[Lie brackets on extended simply affine Lie algebras] \label{remark: lie_brackets_on_extended_simply_affine_lie_algebras}
                Through proposition \ref{prop: lie_algebra_outer_derivations_are_parametrised_by_HH1} and using the Chevalley-Eilenberg resolution (cf. proposition \ref{prop: chevalley_eilenberg_complexes}), the Lie bracket on $\hat{\g}_{\kappa}$ can be shown to be given by:
                    $$[x \tensor f + \mu_1 v + \nu_1 \del_t, y \tensor g + \mu_2 v + \nu_2 \del_t]_{\hat{\g}_{\psi}} := [x, y]_{\g} \tensor fg + c \kappa(x, y) fg + [x, y]_{\g} \tensor \del_t(fg)$$
                for all $x \tensor f, y \tensor g \in \g[t^{\pm 1}]$ and all $\mu_1, \mu_2, \nu_1, \nu_2 \in k$.
            \end{remark}
            
            \begin{proposition}[Simply affine Lie algebras as derived subalgebras] \label{prop: simply_affine_Lie_algebras_as_derived_subalgebras}
                Assume convention \ref{conv: simple_finite_dimensional_lie_algebras}. The simply affine Lie algebra $\breve{\g}_{\kappa}$ from definition \ref{def: simply_affine_lie_algebras} is isomorphic to the derived subalgebra of the extended simply affine Lie algebra $\hat{\g}_{\kappa}$ as in definition \ref{def: extended_simply_affine_lie_algebras}.
            \end{proposition}
                \begin{proof}
                    This is a consequence of proposition \ref{prop: lie_algebra_outer_derivations_are_parametrised_by_HH1}.
                \end{proof}
          
    \subsection{Untwisted affinisation of simple finite-dimensional Lie algebras} \label{subsection: untwisted_affinisation_of_simple_finite_dimensional_lie_algebras}
        \begin{convention}
            From now on, suppose that $\Q$ is equipped with the usual absolute value $|-|$ so that $(\Q, |-|)$ would be an archimedean field, and suppose also that $k/\Q$ is an archimedean subextension of $\bbC/\Q$ so that notions such as positivity and negativity could be meaningfully discussed. Typically, $k$ is taken to either be $\Q, \R$, or $\bbC$.
        \end{convention}
        \begin{convention}[Start with a simple finite-dimensional Lie algebra ...] \label{conv: simple_finite_dimensional_lie_algebras}
            We begin with the sextuple:
                $$(\g, A, \h, \Pi, \h^{\vee}, \Pi^{\vee}, B)$$
            made up of the data of a simple finite-dimensional Lie algebra $\g$ over $k$, its finite-type GCM $A \in \Mat_l(\Z)$, along with an $r$-dimensional $k$-linear minimal root datum $(\h, \Pi, \h^{\vee}, \Pi^{\vee})$ thereof, with $r = l$ as a consequence of $A$ being of finite type. Out of such data, one can also construct a weight lattice $(\Z^{\oplus r}, \bbX, B)$ wherein $\bbX := \span_{\Z} \Pi$ of $\g$, so that $\Pi$ would be identified with the set of simple roots of $\g$. With respect to such a weight lattice, one has the following weight space decomposition of $\g$ relative to the chosen root datum $(\h, \Pi, \h^{\vee}, \Pi^{\vee}, B)$:
                $$\g \cong \bigoplus_{\lambda \in \bbX} \g_{\lambda}$$
            wherein $\g_{\lambda} := \{x \in \g \mid \forall h \in \h: \ad(h)(x) = B(\lambda^{\vee}, h) x\}$ for all $\lambda \in \bbX$ (note that $\g_0 \cong \h$).
            
            Now, let $\{h_i, e_i^-, e_i^+\}_{1 \leq i \leq l}$ be the set of Chevalley generators of $\g$, whose elements are \textit{a priori} subjected to the Chevalley relations as in definition \ref{def: reduced_kac_moody_algebras}. From the aforementioned weight space decomposition, one can also define a Chevalley-Cartan involution $\omega: \g \to \g$ given by $\omega(h_i) := -h_i, \omega(e_i^{\pm}) = -e_i^{\mp}$ for all $1 \leq i \leq l$. 
            
            Recall also that there is a natural partial ordering on the set of weight lattice $\bbX$: for all $\lambda, \lambda' \in \bbX$, one declares that $\lambda \geq \lambda'$ if and only if $\lambda - \lambda' \in \bbX^+$ (cf. proposition \ref{prop: roots_are_partially_ordered_by_heights}). This poset has maximal elements \textit{a priori} (called \say{heighest weights}), and let us choose such an element:
                $$\theta := \sum_{1 \leq i \leq l} \theta_i \alpha_i$$
            with $\Pi := \{\alpha_i\}_{1 \leq i \leq n} \subseteq \h^{\vee}$; its dual (called the \textbf{heighest coweight}) shall be denoted by:
                $$\theta^{\vee} := \sum_{1 \leq i \leq l} \theta_i^{\vee} \alpha_i^{\vee}$$
            with $\Pi^{\vee} := \{\alpha_i^{\vee}\}_{1 \leq i \leq n} \subseteq \h$.
        \end{convention}
        \begin{convention}[Generators for extended simply affine Lie algebras] \label{conv: extra_generators_for_extended_simply_affine_lie_algebras}
             Assume convention \ref{conv: simple_finite_dimensional_lie_algebras}.
             
             We begin by choosing $\e_0^+ \in \g_{\theta}$ such that $\kappa(\e_0^-, \e_0^+) = \frac{2}{\kappa(\theta, \theta)}$ with $\e_0^- := -\omega(\e_0^+)$; an easy computation to then perform immediately (using propositions \ref{prop: generalised_killing_forms_are_invariant}) is that:
                $$[\e_0^-, \e_0^+]_{\g} = -\theta^{\vee}$$
            Now, let us set $\hat{e}_0^{\pm} := \e_0^{\pm} \tensor t^{\pm 1}$ and $\hat{e}_i^{\pm} := e_i^{\pm} \tensor 1$ for all $1 \leq i \leq l$, and then make the following computation using remark \ref{remark: lie_brackets_on_extended_simply_affine_lie_algebras}:
                $$[\hat{e}_0^-, \hat{e}_0^+]_{\hat{\g}_{\kappa}} = [\e_0^- \tensor t^{-1}, \e_0^+ \tensor t]_{\hat{\g}_{\kappa}} = [\e_0^-, \e_0^+]_{\g} \tensor 1 + c \kappa(\e_0^-, \e_0^+) + \del_t(c \kappa(\e_0^-, \e_0^+)) = -\theta^{\vee} \tensor 1 + \frac{2c}{\kappa(\theta, \theta)}$$
        \end{convention}
        \begin{convention}[Root data of extended Cartan matrices] \label{conv: root_data_of_extended_cartan_matrices}
            Assume convention \ref{conv: simple_finite_dimensional_lie_algebras}. We refer the reader to definition \ref{def: extended_cartan_matrices} for the notion of extended Cartan matrices, and to lemma \ref{lemma: extended_cartan_matrices_are_affine_generalised_cartan_matrices} for a proof of the fact that $\hat{A} \in \Mat_{l + 1}(\Z)$ is a GCM of untwisted affine type (cf. definition \ref{def: untwisted_affine_generalised_cartan_matrices_and_dynkin_quivers}).
        
            Consider the direct sum of $k$-vector spaces:
                $$\hat{\h} := \h \tensor 1 \oplus \span_k \{c\} \oplus \span_k \{\del_t\}$$
            Through definitions \ref{def: simply_affine_lie_algebras} and \ref{def: extended_simply_affine_lie_algebras}, one recognises that this is an abelian Lie $k$-subalgebra of $\hat{\g}_{\kappa}$; we shall aim to establish $\hat{\h}$ as the root space $(\hat{\g}_{\kappa})_0$. As such, let us first construct a root datum $(\hat{\h}, \hat{\Pi}, \hat{\h}^{\vee}, \hat{\Pi}^{\vee}, \hat{B})$.
            
            Of course, we firstly have:
                $$\hat{\h}^{\vee} := \hat{\h}^*$$
                
            Then, we shall have $\hat{B}$ as the Cartan quadratic form of the (affine) Dynkin quiver associated to the affine GCM $\hat{A}$ (cf. definition \ref{def: cartan_quadratic_forms_of_finite_quivers}), i.e.:
                $$\hat{B} := B_{\Dyn(\hat{A})}$$ 
            Next, let us fix an element $\delta \in \hat{\h}^{\vee}$ defined via the following properties:
                $$\hat{B}|_{\h \tensor 1 \oplus \span_k \{c\}}(\delta, -) = 0, \hat{B}(\delta, \del_t) = -1$$
            Before we move on, note that the conditions above help us ensure firstly that $\delta \in \hat{\h}^{\vee} \setminus \{0\}$ and secondly that $\delta \in \Rad \hat{B}|_{\h \tensor 1 \oplus \span_k \{c\}}$ (see definition \ref{def: radicals_of_bilinear_forms} for the definition of $\Rad \hat{B}|_{\h \tensor 1 \oplus \span_k \{c\}}$). These observations, together with the fact that $\hat{A} \in \Mat_{l + 1}(\Z)$ is an affine-type GCM, thus help us infer that $\Rad \hat{B}|_{\h \tensor 1 \oplus \span_k \{c\}} \cong \span_k \{\delta\}$ (cf. proposition \ref{prop: affineness_criterion_via_radicals}); one can therefore choose $\delta$ to be the unique lowest positive imaginary root of the affine Dynkin quiver $\Dyn(\hat{A})$ (hence the notation \say{$\delta$}). Lastly, declare that the simple (co)roots are as follows:
                $$\forall 1 \leq i \leq l: \hat{\alpha}_i := \alpha_i, \hat{\alpha}_i^{\vee} := \alpha_i^{\vee} \tensor 1$$
                $$\hat{\alpha}_0^{\vee} := [\hat{e}_0^-, \hat{e}_0^+] = -\theta^{\vee} \tensor 1 + \frac{2c}{\kappa(\theta, \theta)} = -\hat{\theta}^{\vee} + \frac{2c}{\kappa(\theta, \theta)}$$
                $$\hat{\alpha}_0 := \delta - \hat{\theta}$$
        \end{convention}
    
        \subsubsection{From universal central extensions to affine Cartan matrices}
            \begin{lemma}[Cartan subalgebras of extended simply affine Lie algebras] \label{lemma: cartan_subalgebras_of_extended_simply_affine_Lie_algebras}
                \cite[Proposition 12.2.13]{perrin_kac_moody_algebras} The quintuple $(\hat{\h}, \hat{\Pi}, \hat{\h}^{\vee}, \hat{\Pi}^{\vee}, \hat{B})$ from convention \ref{conv: root_data_of_extended_cartan_matrices} is a well-defined minimal root datum of the extended GCM $\hat{A} \in \Mat_{l + 1}(\Z)$. 
            \end{lemma}
            \begin{theorem}[Untwisted affine Kac-Moody algebras as universal central extensions] \label{theorem: untwisted_affinisation}
                \cite[Theorem 7.4]{kac_infinite_dimensional_lie_algebras} Assume conventions \ref{conv: simple_finite_dimensional_lie_algebras}, \ref{conv: extra_generators_for_extended_simply_affine_lie_algebras}, and \ref{conv: root_data_of_extended_cartan_matrices}. The extended simply affine Lie algebra $\hat{\g}_{\kappa}$ (cf. definition \ref{def: extended_simply_affine_lie_algebras}) is an untwisted affine Kac-Moody algebra: in particular, there is an isomorphism of Lie $k$-algebras:
                    $$\psi: \frakLie(\hat{A}) \to \hat{\g}_{\kappa}$$
            \end{theorem}
                \begin{proof}
                    Because we have constructed a root datum for the affine GCM $\hat{A} \in \Mat_{l + 1}(\Z)$, we can proceed with verifying that the Lie $k$-algebra $\hat{\g}_{\kappa}$ admits the same presentation (cf. corollary \ref{coro: presentations_of_reduced_kac_moody_algebras}) as the affine Kac-Moody algebra $\frakLie(\hat{A})$ in order to show that the two are isomorphic.
                        \begin{itemize}
                            \item \textbf{(Chevalley relations on $\hat{\h}$):} Firstly, because $\hat{\h} := \h \tensor 1 \oplus \span_k \{c\} \oplus \span_k \{\del_t\}$ is a commutative Lie $k$-subalgebra of $\hat{\g}_{\kappa}$ by construction, one certainly has that $[h, h']_{\hat{\g}_{\kappa}} = 0$ for all $h, h' \in \hat{\h}$.
                            \item \textbf{(Chevalley relations on $\hat{e}_1^{\pm}, ..., \hat{e}_l^{\pm}$):} We begin by verifying that the Chevalley relations hold for the generators $\hat{e}_1^{\pm}, ..., \hat{e}_l^{\pm}$. To this end, let us fix an arbitrary index $1 \leq j \leq l$, observe firstly that it shall suffice to only compute $[\hat{\alpha}_i^{\vee} + c + \del_t, \hat{e}_j^{\pm}]_{\hat{\g}_{\kappa}}$, and then consider the following:
                                $$
                                    \begin{aligned}
                                        [\hat{\alpha}_i^{\vee} + c + \del_t, \hat{e}_j^{\pm}]_{\hat{\g}_{\kappa}} & = [\alpha_i^{\vee} \tensor 1 + c + \del_t, e_j^{\pm} \tensor 1 ]_{\hat{\g}_{\kappa}}
                                        \\
                                        & = [\alpha_i^{\vee} \tensor 1, e_j^{\pm} \tensor 1]_{\hat{\g}_{\kappa}} + [c, e_j^{\pm} \tensor 1]_{\hat{\g}_{\kappa}} + [\del_t, e_j^{\pm} \tensor 1]_{\hat{\g}_{\kappa}}
                                        \\
                                        & = [\alpha_i^{\vee}, e_j^{\pm}]_{\g} \tensor 1
                                        \\
                                        & = \mp B(\alpha_i, \alpha_j^{\vee}) e_j^{\pm} \tensor 1
                                        \\
                                        & = \mp \hat{B}(\hat{\alpha}_i, \hat{\alpha}_j^{\vee}) \hat{e}_j^{\pm}
                                    \end{aligned}
                                $$
                            
                            Next, let us fix a pair of arbitrarily indices $1 \leq i, j \leq l$ and then consider the following:
                                $$[\hat{e}_i^-, \hat{e}_j^+]_{\hat{\g}_{\kappa}} = [e_i^- \tensor 1, e_j^+ \tensor 1]_{\hat{\g}_{\kappa}} = [e_i^-, e_j^+]_{\g} \tensor 1 = \delta_{ij} \alpha_i^{\vee} \tensor 1 = \delta_{ij} \hat{\alpha}_i^{\vee}$$
                            We have thus that Chevalley relations as in corollary \ref{coro: presentations_of_reduced_kac_moody_algebras} hold for the generators $\hat{e}_1^{\pm}, ..., \hat{e}_l^{\pm}$.
                            
                            \item \textbf{(Chevalley relations on $\hat{e}_0^{\pm}$):} Now, let us show that $[h, \hat{e}_0^{\pm}]_{\hat{\g}_{\kappa}} = \mp \hat{B}(\hat{\alpha}_0, h) \hat{e}_0^{\pm}$ for all $h \in \hat{\h}$; in fact, since we have $\hat{\alpha}_0 := \delta - \hat{\theta}$ by convention \ref{conv: root_data_of_extended_cartan_matrices}, we shall aim to demonstrate that $[h, \hat{e}_0^{\pm}]_{\hat{\g}_{\kappa}} = \mp \hat{B}(\delta - \hat{\theta}, h) \hat{e}_0^{\pm}$, and it suffices to only perform the necessary computations for $h := \hat{\alpha}_i^{\vee} + c + \del_t$ for some arbitrary $1 \leq i \leq l$. To this end, consider the following, which comes from the construction of $\delta \in \hat{\h}^{\vee}$ (cf. convention \ref{conv: root_data_of_extended_cartan_matrices}):
                                $$\hat{B}(\delta - \hat{\theta}, \hat{\alpha}_i^{\vee} + c) = -\hat{B}(\hat{\theta}, \hat{\alpha}_i^{\vee} + c) = -B(\theta, \alpha_i^{\vee})$$
                            Since $\hat{e}_0^{\pm} := \e_0^{\pm} \tensor 1$ and since $\e_0^{\pm} \in \g_{\pm \theta}$ (cf. convention \ref{conv: extra_generators_for_extended_simply_affine_lie_algebras}), and since $\dim_k \g_{\pm \theta} = 1$ (cf. corollary \ref{coro: multiplicities_of_real_roots_of_kac_moody_algebras}), one has that:
                                $$[\alpha_i^{\vee}, \e_0^{\pm}]_{\g} = \mp B(\pm\theta, \alpha_i^{\vee}) \e_0^{\pm}$$
                            from which one gathers that:
                                $$[\hat{\alpha}_i^{\vee}, \hat{e}_0^{\pm}]_{\hat{\g}_{\kappa}} = \mp \hat{B}(\hat{\theta}, \hat{\alpha}_i^{\vee}) \hat{e}_0^{\pm} = \mp \hat{B}(\delta - \hat{\theta}, \hat{\alpha}_i^{\vee} + c) \hat{e}_0^{\pm} = \mp \hat{B}(\hat{\alpha}_0, \hat{\alpha}_i^{\vee} + c) \hat{e}_0^{\pm}$$
                            The following is also a direct consequence of the construction of $\delta$ in convention \ref{conv: root_data_of_extended_cartan_matrices}:
                                $$\hat{B}(\delta - \hat{\theta}, \del_t) = -1 - \hat{B}(\hat{\theta}, \del_t) = -1 - \hat{B}(\theta \tensor 1, \del_t) = -1 - 0 = -1$$
                            At the same time, one has:
                                $$[\del_t, \hat{e}_0^{\pm}]_{\hat{\g}_{\kappa}} = [\del_t, \e_0^{\pm} \tensor t^{\pm 1}]_{\hat{\g}_{\kappa}} = \pm \e_0^{\pm} \tensor t^{\pm 1} = \pm \hat{e}_0^{\pm}$$
                            By putting the two observations together, it becomes clear that:
                                $$[\del_t, \hat{e}_0^{\pm}]_{\hat{\g}_{\kappa}} = \mp \hat{B}(\delta - \hat{\theta}, \del_t) \hat{e}_0^{\pm} = \mp \hat{B}(\hat{\alpha}_0, \del_t) \hat{e}_0^{\pm}$$
                            It is now clear that:
                                $$[h, \hat{e}_0^{\pm}]_{\hat{\g}_{\kappa}} = \mp \hat{B}(\hat{\alpha}_0, h) \hat{e}_0^{\pm}$$
                            for all $h \in \hat{\h}$.
                                
                            Next, fix an arbitrary index $1 \leq j \leq l$ and consider the following:
                                $$[\hat{e}_0^-, \hat{e}_j^+]_{\hat{\g}_{\kappa}} = [\e_0^- \tensor t, e_j^+ \tensor 1]_{\hat{\g}_{\kappa}} = [\e_0^-, e_j^+]_{\g} \tensor t = 0 \tensor t = 0$$
                            wherein the second-to-last equality is due to $\e_0 \in \g_{\theta}$ by assumption (cf. convention \ref{conv: extra_generators_for_extended_simply_affine_lie_algebras}) and due to $\theta$ being a highest root of $\g$ (cf. convention \ref{conv: simple_finite_dimensional_lie_algebras}). The same argument also yields us $[\hat{e}_j^-, \hat{e}_0^+]_{\hat{\g}_{\kappa}} = 0$ for all $1 \leq j \leq l + 1$.
                            
                            Lastly, note that we have by convention \ref{conv: extra_generators_for_extended_simply_affine_lie_algebras} that $[\hat{e}_0^-, \hat{e}_0^+]_{\hat{\g}_{\kappa}} = - \theta^{\vee} \tensor 1 + \frac{2c}{\kappa(\theta, \theta)} =: \hat{\alpha}_0^{\vee}$.
                            \item \textbf{(Serre relations on $\hat{e}_0^{\pm}, \hat{e}_1^{\pm}, ..., \hat{e}_l^{\pm}$):} Per convention \ref{conv: simple_finite_dimensional_lie_algebras}, $\g$ is a simple finite-dimensional Lie $k$-algebra, i.e. a finite-type Kac-Moody algebra. As such, one has the Serre relations $\ad(\hat{e}_i^{\pm})^{1 - a_{ij}}(\hat{e}_j^{\pm}) = \ad(e_i^{\pm})^{1 - a_{ij}}(e_j^{\pm}) \tensor 1 = 0 \tensor 1 = 0$ by corollary \ref{coro: presentations_of_reduced_kac_moody_algebras}.
                            
                            In order to show that $\ad(\hat{e}_0^{\pm})^{1 - a_{0j}}(\hat{e}_j^{\pm}) = 0$ for all $0 \leq j \leq l + 1$. When $j = 0$, this is trivial, and for the cases wherein $1 \leq j \leq l$, simply use the Chevalley relations $[\hat{e}_0^-, \hat{e}_j^+]_{\hat{\g}_{\kappa}} = 0$ and $[\hat{e}_j^-, \hat{e}_0^+]_{\hat{\g}_{\kappa}} = 0$, which have been shown to hold in $\hat{\g}_{\kappa}$.
                        \end{itemize}
                \end{proof}
                
            \begin{convention} \label{conv: root_space_decomposition_of_extended_simply_affine_lie_algebras}
                Assume conventions \ref{conv: simple_finite_dimensional_lie_algebras}, \ref{conv: extra_generators_for_extended_simply_affine_lie_algebras}, and \ref{conv: root_data_of_extended_cartan_matrices}. 
                
                As we have constructed a minimal root datum $(\hat{\h}, \hat{\Pi}, \hat{\h}^{\vee}, \hat{\Pi}^{\vee}, \hat{B})$ of $\hat{A} \in \Mat_{l + 1}(\Z)$ and can subsequently define the root lattice $\hat{\bbX} := \span_{\Z} \hat{\Pi}$ along with following abstract root spaces for $\hat{\g}_{\kappa}$ for all abstract roots $\hat{\lambda} \in \hat{\bbX}$:
                    $$(\hat{\g}_{\kappa})_{\hat{\lambda}} := \left\{ x \in \hat{\g}_{\kappa} \mid \forall h \in \hat{\h}: \ad(h)(x) = \hat{\lambda}(h)x \right\}$$
                At the same time, let us use the root space decomposition $\g \cong \h \oplus \bigoplus_{\lambda \in \bbX} \g_{\lambda}$ in order to induce the following direct sum decomposition of the $k$-vector space $\hat{\g}_{\kappa}$ via :
                    $$\hat{\g}_{\kappa} \cong \hat{\h} \oplus \left(\bigoplus_{r \in \Z \setminus \{0\}} \h \tensor_k t^r\right) \oplus \left(\bigoplus_{\lambda \in \Phi} \g_{\lambda}[t^{\pm 1}]\right)$$
                Lastly, since we have shown that $\hat{\g}_{\kappa}$ is an affine Kac-Moody algebra with $\hat{\h}$ as a Cartan subalgebra, it is meaningful to speak of the root system $\hat{\Phi} := \Phi(\hat{\g}_{\kappa}, \hat{\h})$.
            \end{convention}
            \begin{lemma}[Root spaces of untwisted affine Kac-Moody algebras] \label{lemma: root_spaces_of_untwisted_affine_kac_moody_algebras}
                \cite[Proposition 12.2.14]{perrin_kac_moody_algebras} Assume convention \ref{conv: root_space_decomposition_of_extended_simply_affine_lie_algebras}. One has the following characterisations of the root spaces of $\hat{\g}_{\kappa}$:
                    $$(\hat{\g}_{\kappa})_{\hat{0}} \cong \hat{\h}$$
                    $$\forall \hat{\lambda} \in \Im( \hat{\Phi} ) \cong \{r \delta \mid \forall r \in \Z \setminus \{0\}\}: (\hat{\g}_{\kappa})_{\hat{\lambda}} \cong \h \tensor_k t^r$$
                    $$\forall \hat{\lambda} \in \Re( \hat{\Phi} ) \cong \Phi: (\hat{\g}_{\kappa})_{\hat{\lambda}} \cong \g_{\lambda}[t^{\pm 1}]$$
            \end{lemma}
                \begin{proof}
                    The only thing to do is to compute the action of $\hat{\h}$ on each of the $k$-vector spaces $\hat{\h}$, $\bigoplus_{r \in \Z \setminus \{0\}} \h \tensor_k t^r$, and $\bigoplus_{\lambda \in \bbX} \g_{\lambda}[t^{\pm 1}]$. This is straightforward since we know how $\h \tensor 1$, $c$, as well as $\del_t$ act on these $k$-vector spaces. As such, we leave these computations up to the reader. 
                \end{proof}  
            \begin{corollary}[Root space decomposition of untwisted affine Kac-Moody algebras] \label{coro: root_space_decomposition_of_untwisted_affine_kac_moody_algebras}
                Assume convention \ref{conv: root_space_decomposition_of_extended_simply_affine_lie_algebras}. Then, one has the following direct sum decompositions of $\rmU(\h)$-modules:
                    $$\hat{\g}_{\kappa} \cong \hat{\h} \oplus \bigoplus_{\hat{\lambda} \in \hat{\Phi}} (\hat{\g}_{\kappa})_{\hat{\lambda}} \cong \hat{\h} \oplus \left(\bigoplus_{r \in \Z \setminus \{0\}} \h \tensor_k t^r\right) \oplus \left(\bigoplus_{\lambda \in \Phi} \g_{\lambda}[t^{\pm 1}]\right)$$
                In particular, one sees that:
                    $$\forall \hat{\lambda} \in \Im( \hat{\Phi} ): \dim_k (\hat{\g}_{\kappa})_{\hat{\lambda}} = \dim_k \h = l$$
            \end{corollary}
            \begin{proposition}[Untwisted affinisation preserves root heights] \label{prop: untwisted_affinisation_preserves_root_heights}
                \cite[Theorem 12.2.15]{perrin_kac_moody_algebras} Assume convention \ref{conv: root_space_decomposition_of_extended_simply_affine_lie_algebras}. The isomorphism $\psi: \frakLie(\hat{A}) \to \hat{\g}_{\kappa}$ preserve the degrees of elements $x \in \frakLie(\hat{A})$ given by the height grading from remark \ref{remark: height_grading_on_kac_moody_algebras}. Specifically, if $H \subset \frakLie(\hat{A})$ is a Cartan subalgebra and if the set of Chevalley generators of $\frakLie(\hat{A})$ is $H \cup \{E_i^-, E_i^+\}_{0 \leq i \leq l}$ then we will have:
                    $$\psi(H) = \hat{\h}$$
                    $$\forall 0 \leq i \leq l: \psi(E_i^{\pm}) = \hat{e}_i^{\pm}$$
            \end{proposition}
                \begin{proof}
                    Straightforward from corollary \ref{coro: root_space_decomposition_of_untwisted_affine_kac_moody_algebras}. 
                \end{proof}
                
        \subsubsection{Untwisted affine Weyl groups as coextensions} \label{subsubsection: untwisted_affine_weyl_groups}
            \todo[inline]{See \cite[Chapter 6]{kac_infinite_dimensional_lie_algebras} and \cite[Subsection 12.2.3]{perrin_kac_moody_algebras}}
            
            \begin{proposition}[Untwisted affine Weyl groups as extensions] \label{prop: untwisted_affine_weyl_groups_as_coextensions}
                \cite[Theorem 12.2.19]{perrin_kac_moody_algebras} Assume convention \ref{conv: root_space_decomposition_of_extended_simply_affine_lie_algebras} and let $\hat{\rmW} := \rmW \ltimes \bbX^{\vee}$. There is thus a group isomorphism:
                    $$\Psi: \rmW_{\Dyn(\hat{A})} \to \hat{\rmW}$$
                sending generators to generators. 
            \end{proposition}