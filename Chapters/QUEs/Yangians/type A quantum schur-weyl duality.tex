\section{Hecke algebras and quantum Schur-Weyl duality for type \texorpdfstring{$\sfA_l$}{}}
    In this section, we attempt to establish the fact that there is a \say{quantum Schur-Weyl duality} between type-$\sfA_l$ Yangians and the so-called graded Hecke algebras in the sense of Lusztig. 
    
    \begin{convention}
        Throughout, we will be working over an algebraically closed field $k$ of characteristic $0$.
    \end{convention}

    \subsection{A review of classical Schur-Weyl Duality}
        \subsubsection{Specht modules}
            We recall firstly the following result from the representation theory of the symmetric group on $l$ elements $S_l$:
            \begin{lemma}[Classification of simple $S_l$-modules] \label{lemma: classification_of_simple_modules_over_symmetric_groups}
                For any $l \geq 1$, there is a bijection:
                    $$\{ \text{Isomorphism classes of finite-dimensional simple $S_l$-modules} \} \cong \{ \text{Partitions of $l$} \}$$
            \end{lemma}
            Recall also that for any finite group $W$, its category of finite-dimensional modules $W\mod^{\fd}$ is \textit{a priori} semi-simple, so the lemma above tells us that any finite-dimensional $S_l$-module is classified uniquely (up to isomorphisms) by some finite tuple of partitions of $l$. 
            \begin{definition}[Specht modules] \label{def: specht_modules}
                Fix some $l \geq 1$. To each partition $\lambda \radjoint l$, we choose a distinguished representative $\bbS^{\lambda}$ of the corresponding isomorphism class of simple $S_l$-modules. This is called the \textbf{Specht module} of weight $\lambda$. 
            \end{definition}

        \subsubsection{Schur functors and Schur-Weyl Duality}

    \subsection{Affine Schur-Weyl Duality}

    \subsection{Quantum Schur-Weyl Duality}