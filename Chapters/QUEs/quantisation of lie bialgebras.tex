\begin{convention}
    We fix once and for all a field $k$. All Hopf algebras will be assumed to have invertible antipodes. All modules will be left-modules unless mentioned explicitly to be otherwise.
\end{convention}

\section{Braidings and R-matrices}
    \subsection{Braided monoidal categories}

    \subsection{R-matrices}
        \begin{convention}
            Throughout, suppose that $(\C, \tensor, \1)$ is a \textit{braided} $k$-linear monoidal category.
        \end{convention}
    
        \begin{definition}[R-matrices] \label{def: R_matrices}
            An \textbf{R-matrix} on an object $V \in \Ob(\C)$ is an automorphism $c \in \Aut(V \tensor V)$ satisfying the so-called \textbf{Yang-Baxter equation}:
                $$(c \tensor \id_V) \circ (\id_V \tensor c) \circ (c \tensor \id_V) = (\id_V \tensor c) \circ (c \tensor \id_V) \circ (\id_V \tensor c) \in \Aut(V \tensor V \tensor V)$$
        \end{definition}
        \begin{example}
            For each object $V \in \Ob(\C)$, the twist map:
                $$\tau_{V, V}: V \tensor V \xrightarrow[]{\cong} V \tensor V$$
            will be an example of an R-matrix. For a proof, consider the following Coxeter relation in the symmetric group $S_3 \cong \Aut(\{1, 2, 3\})$:
                $$(12)(23)(12) = (23)(12)(23)$$    
        \end{example}
        \begin{remark}
            Finding all solutions to the Yang-Baxter equation is a very difficult task. To see why, specialise to the case $\C := k\mod^{\fr}$ and write the equation out in terms of a fixed basis. 
            
            Nevertheless, it is known how one might generate new solutions from a previously known one (e.g. $c = \tau_{V, V}$). In particular, if $V \in \Ob(\C)$ is an object and $c \in \Aut(V \tensor V)$ is an R-matrix then the following automorphisms on $V \tensor V$ will also be R-matrices:
                $$c^{-1}, \lambda c, \tau_{V, V} \circ c \circ \tau_{V, V}$$
            wherein $\lambda \in k$ is some scalar.
        \end{remark}
        \begin{example}[The Yang-Baxter equation for finite-dimensional simple $\rmU_q(\sl_2)$-modules]
            Let us write $\rmU_q := \rmU_q(\sl_2(k))$ and for any root lattice element $\lambda = \e q^n\in k$ (with $\e = \pm 1$), write $V_q(n) = V_q(\lambda)$ to mean the corresponding $(n + 1)$-dimensional simple $\rmU_q$-module of highest-weight $\lambda$. For details on the theory of highest-weight $\rmU_q$-modules, we refer the reader to \cite[Chapters VI and VII]{kassel_quantum_groups}. Also, for convenience, we shall also assume for this example that $k$ is algebraically closed and of characteristic $0$. 
            
            We now analyse certain R-matrices in the braided monoidal category of finite-dimensional $k$-linear $\rmU_q$-representations. We caution the reader that it is important to view these $\rmU_q$-representations as certain $\rmU_q$-modules, since R-matrices thereon are not simply certain $k$-linear automorphisms but $\rmU_q$-linear automorphisms. To that end, recall the quantum Clebsch-Gordon formula (cf. \cite[Theorem VII.7.1]{kassel_quantum_groups}), which tells us that the tensor product $V_q(n) \tensor_k V_q(m)$ is a semi-simple $\rmU_q$-module in the following manner:
                $$V_q(n) \tensor_k V_q(m) \cong \bigoplus_{0 \leq p \leq m} V_q(n + m - 2p)$$
            By specialising to the case wherein $m = n$, one gets that:
                $$V_q(n) \tensor_k V_q(n) \cong \bigoplus_{0 \leq p \leq n} V_q(2(n - p))$$
            and so $\rmU_q$-module automorphisms on $V_q(n) \tensor_k V_q(n)$ are nothing but $\rmU_q$-module automorphisms on $\bigoplus_{0 \leq p \leq m} V_q(2(n - p))$. Such automorphisms can thus be computed via examining their actions on the generators $v^{\lambda}_0 \in V_q(\lambda)$ of the simple (hence cyclic) $\rmU_q$-modules $V_q(\lambda)$ (for $\lambda \in \{\pm q^{ 2(n - p) }\}_{0 \leq p \leq n}$). In particular, one sees that any $\rmU_q$-module automorphism on $V_q(n) \tensor_k V_q(n)$ is necessarily diagonalisable. 
            
            Now, let us use the quantum Clebsch-Gordon formula again to obtain the following direct sum decomposition of $V_q(n) \tensor V_q(n) \tensor V_q(n)$:
                $$
                    \begin{aligned}
                        V_q(n) \tensor_k V_q(n) \tensor_k V_q(n) & \cong V_q(n) \tensor_k \left( \bigoplus_{0 \leq p \leq n} V_q(2(n - p)) \right)
                        \\
                        & \cong \bigoplus_{0 \leq p \leq n} ( V_q(n) \tensor_k V_q(2(n - p)) )
                        \\
                        & \cong \bigoplus_{0 \leq p \leq n} \bigoplus_{0 \leq j \leq 2(n - p)} V_q(n + 2(n - p - j))
                    \end{aligned}
                $$
        \end{example}
        
\section{Reconstructing bialgebras from their representation categories}
    \subsection{Fibre functors}
        For the sake of fixing terminologies, let us recall a few relevant definitions.
        \begin{definition}[Rigid monoidal categories] \label{def: rigid_monoidal_categories}
            A monoidal category is said to be \textbf{rigid} if and only if every of its objects has two-sided duals. 
        \end{definition}
        \begin{definition}[(Multi-)ring categories] \label{def: ring_categories}
            \cite[Definition 4.1.1]{EGNO} A \textbf{(multi-)ring category} is a locally finite $k$-linear monoidal category $(\C, \tensor, \1)$ on which the tensor product bifunctor $\tensor: \C \x \C \to \C$ is $k$-bilinear on morphisms and biexact, i.e. for all $V, V', V'' \in \Ob(\C)$, the natural map:
                $$\C(V, V') \tensor_k \C(V', V'') \to \C(V, V'')$$
            is $k$-linear. Without the local finiteness condition, we shall say \say{infinite (multi-)ring categories}.
        \end{definition}
        \begin{definition}[Tensor functors] \label{def: tensor_functors}
            Let $\C, \D$ be multi-ring categories over $k$ and $F: \C \to \D$ be a monoidal functor. One says that $F$ is a \textbf{tensor functor} if and only if there is a monoidal natural isomorphism:
                $$J_{-, -}: F(- \tensor -) \xrightarrow[]{\cong} F(-) \tensor F(-)$$
            along with an isomorphism:
                $$F(\1_{\C}) \cong \1_{\D}$$
        \end{definition}
        \begin{definition}[(Multi-)tensor categories] \label{def: tensor_categories}
            \cite[Definition 4.1.1]{EGNO} A  \textbf{multi-tensor category} over a field $k$ is a locally finite $k$-linear abelian rigid monoidal category $(\C, \tensor, \1)$ on which the bifunctor:
                $$\tensor: \C \x \C \to \C$$
            is $k$-bilinear on morphisms. If, in addition, we have that $\C(\1, \1) \cong k$ then $\C$ will be called a \textbf{tensor category}. Without the local finiteness condition, we shall say \say{infinite (multi-)tensor categories}.
        \end{definition}
        \begin{proposition}[Biexactness of tensor products] \label{prop: biexactness_of_tensor_products}
            \cite[Proposition 4.2.1]{EGNO} Let $(\C, \tensor, \1)$ be a multi-tensor category. Then the bifunctor $\tensor: \C \x \C \to \C$ will always be biexact. 
        \end{proposition}
            \begin{proof}
                This is a direct consequence of the fact that multi-tensor categories are rigid as monoidal categories. 
            \end{proof}
        \begin{corollary}
            (Infinite) (multi-)tensor categories are special cases of (infinite) (multi-)ring categories.
        \end{corollary}
        
        Now that all the etymological background materials have been put into place, let us actually begin discussing the so-called \say{Tannakian reconstruction theory}. 
        \begin{convention}
            From now on until the end of this subsection, let $(\C, \tensor, \1)$ be a ring category over $k$. 
        \end{convention}
        The following definition bears significant resemblances with the notion of fibre functors in Grothendieck's interpretation of Galois theory (cf. \cite[Expos\'e V]{SGA1}).
        \begin{definition}[Fibre functors] \label{def: fibre_functors}
            A fibre functor on $(\C, \tensor, \1)$ is an \textit{exact} tensor functor:
                $$F: \C \to k\mod^{\fr}$$
            (wherein $k\mod^{\fr}$ is a tensor category over $k$ in the usual manner).
        \end{definition}
        The following proposition, though very important, is more-or-less self-evident. 
        \begin{proposition}[Fibre functors for representation categories of associative algebras] \label{prop: fibre_functors_for_representation_categories_of_associative_algebras}
            Let $A$ be an associative $k$-algebra. Then, the forgetful functor:
                $$A\mod \xrightarrow[]{F_A := \Hom_A(A, -)} k\mod^{\fr}$$
            is a fibre functor. Furthermore, we have that:
                $$A \cong \End_{\Mon\Nat}(F_A)$$
            as associative $k$-algebras. 
        \end{proposition}
        Let us see what happens when we consider $\C$ to be the representation category of a bialgebra over $k$. Again, the proposition is somewhat self-evident. For details, see the discussion immediately preceding \cite[Section 5.2]{EGNO} as well as \cite[Section 5.3]{EGNO}.
        \begin{remark}[Tensor products of hom-sets]
            One thing that we would like to remind the reader about is that, should $(H, \mu, \eta, \Delta, \e)$ be a $k$-bialgebra and if $V, W, V', W'$ are $H$-modules then the canonical map:
                $$\Hom_H(V, W) \tensor_k \Hom_H(V', W') \to \Hom_H(V \tensor_k W, V' \tensor_k W')$$
            is generally a monomorphism of $H \tensor_k H^{\op}$-modules, and is furthermore an isomorphism whenever $V, V'$ are \textit{free} over $H$. When $H$ is furthermore a Hopf algebra with (invertible) antipode $\sigma: H \to H^{\op}$ then the above becomes an monomorphism (respectively, isomorphism) of $H$-modules via the composite algebra homomorphism:
                $$H \tensor_k H^{\op} \xrightarrow[\cong]{\id_H \tensor \sigma^{-1}} H \tensor_k H \xrightarrow[]{\mu} H$$
        \end{remark}
        \begin{proposition}[Fibre functors for representation categories of bialgebras] \label{prop: fibre_functors_for_representation_categories_of_bialgebras}
            Let $(H, \mu, \eta, \Delta, \e)$ be a $k$-bialgebra and consider the forgetful functor:
                $$H\mod \xrightarrow[]{F_H := \Hom_H(H, -)} k\mod^{\fr}$$
            which we know by proposition \ref{prop: fibre_functors_for_representation_categories_of_associative_algebras} above to be tensorial by virtue of being a fibre functor; denote this tensor structure on $F_H$ by:
                $$F_H(- \tensor_H -) \xrightarrow[\cong]{J_{-, -}} F_H(-) \tensor_k F_H(-)$$
            Then:
                $$H \cong \End_{\Mon\Nat}(F_H)$$
            not simply as $k$-algebras, but furthermore as $k$-bialgebras; the $k$-coalgebra structure on $\End_{\Mon\Nat}(F_H)$ is induced by that of $H$, via the Deligne tensor product $F_H \boxtimes F_H$, in the sense that there are the following $k$-algebra homomorphisms:
                $$\End_{\Mon\Nat}(F_H) \xrightarrow[]{\cong} H \xrightarrow[]{\Delta} H \tensor_k H \xrightarrow[]{\cong} \End_{\Mon\Nat}(F_H) \tensor_k \End_{\Mon\Nat}(F_H) \xrightarrow[]{J_{-, -}^{-1}} \End_{\Mon\Nat}(F_H \boxtimes_k F_H)$$
                $$\End_{\Mon\Nat}(F_H) \xrightarrow[]{\cong} H \xrightarrow[]{\e} k$$
        \end{proposition}
        \begin{corollary}[Fibre functors for representation categories of Hopf algebras] \label{coro: fibre_functors_for_representation_categories_of_hopf_algebras}
            Let $(H, \mu, \eta, \Delta, \e, \sigma)$ be a Hopf $k$-algebra and consider the forgetful functor:
                $$H\mod \xrightarrow[]{F_H := \Hom_H(H, -)} k\mod^{\fr}$$
            Then there will be an isomorphism of Hopf $k$-algebra isomorphism:
                $$H \cong \End_{\Mon\Nat}(F_H)$$
            Particularly, the antipode on $H$ is given by:
                $$\End_{\Mon\Nat}(F_H) \xrightarrow[]{\cong} H \xrightarrow[]{\sigma} H^{\op} \xrightarrow[]{\cong} \End_{\Mon\Nat}(F_H)^{\op}$$
        \end{corollary}
            \begin{proof}
                True because Hopf $k$-algebras form a full subcategory of that of $k$-bialgebras. 
            \end{proof}
    
    \subsection{Quasi-cocommutative Hopf algebras}
    
    \subsection{Quantum doubles of finite-dimensional Hopf algebras and Yetter-Drinfeld modules}
    
\section{Quantisation of finite-dimensional Lie bialgebras}
    \begin{convention}
        We fix once and for all a field $k$ of characteristic $0$. 
    \end{convention}
    
    \subsection{Poisson algebras and deformation quantisation}
        \begin{definition}[Flat deformations] \label{def: flat_deformations}
            Let $A$ be an associative $k$-algebra.
            
            A \textbf{$n^{th}$ order flat algebra/coalgebra/bialgebra/Hopf algebra deformation}\footnote{Algebraic geometers might call these \say{$n^{th}$ order thickenings}.} (for some $n \geq 1$) of $A$ is a \textit{flat} algebra/coalgebra/bialgebra/Hopf algebra $\tilde{A}$ over $k[\hbar]/\hbar^n$ such that there exist an isomorphism of algebra/coalgebra/bialgebra/Hopf algebra over $k$ as follows:
                $$\tilde{A} \tensor_{k[\hbar]/\hbar^n} k \xrightarrow[]{\cong} A$$
                
            A \textbf{formal flat deformation} of $A$ is a flat algebra/coalgebra/bialgebra/Hopf algebra $\tilde{A}$ over $k[\![\hbar]\!]$ such that there eixst an isomorphism of algebra/coalgebra/bialgebra/Hopf algebra over $k$ as follows:
                $$\tilde{A} \tensor_{k[\![\hbar]\!]} k \xrightarrow[]{\cong} A$$
        \end{definition}
        \begin{remark}
            Definition \ref{def: flat_deformations} actually works for $k$ being any commutative ring, provided that one requires that $A$ is flat as a $k$-module. 
            
            It is also clear that any cofiltered diagram of $n^{th}$ order flat deformations $\{A_n\}_{n \geq 1}$ of some (flat) $k$-algebra gives rise to a universal formal flat deformation:
                $$A_{\infty} := \projlim_{n \geq 1} A_n$$
            as a result of Lazard's Theorem, which tells us that cofiltered limits of flat modules are once more flat, as well as the fact that the categories of algebra/coalgebra/bialgebra/Hopf algebra over $k$ has all small cofiltered limits. 
        \end{remark}
        \begin{example}[Rees algebras] \label{example: rees_algebras}
            Let $\g$ be an arbitrary Lie algebra over $k$. THe PBW Theorem asserts that there is a canonical isomorphism of graded $k$-algebras:
                $$\gr \rmU(\g) \cong \Sym(\g)$$
            and per the fact that the functor of associated $\N$-graded algebras $\gr: k\-\Assoc\Alg^{\N} \to \gr(k\-\Assoc\Alg^{\N})$ admits the Rees algebra functor $\Rees: \gr(k\-\Assoc\Alg^{\N}) \to k\-\Assoc\Alg^{\N}$ as a right-adjoint (recall that this is given by $A \mapsto $).
        \end{example}

        Let us now introduce Poisson algebras, which in a sense are \say{dual} to Lie algebras; we will explain how this duality occurs momentarily. 
        \begin{definition}[Poisson algebras] \label{def: poisson_algebras}
            A \textbf{Poisson $k$-algebra} is a triple $(A, \cdot, \{-, -\})$ wherein the pair $(A, \cdot)$ is an associative $k$-algebra, and $\{-, -\}: A \tensor_k A \to A$ is a Lie bracket such that, for each $a \in A$, the map:
                $$\{-, a\}: A \to A$$
            is a derivation on $(A, \cdot)$, i.e. for all $f, g \in A$, one has that:
                $$\{f \cdot g, a\} = f \cdot \{g, a\} + \{f, a\} \cdot g$$
        \end{definition}
        \begin{definition}[Deformation quantisations] \label{def: deformation_quantisation}
            Let $(A, \cdot, \{-, -\})$ be a Poisson algebra over $k$. A \textbf{formal deformation quantisation} of $A$ is a formal flat deformation $\tilde{A}$ (over $k[\![\hbar]\!]$) of $A$, endowed with the Poisson structure given by the commutator $[-, -]_{\tilde{A}}$, such that:
                $$\{f, g\} = \frac{1}{\hbar}[\tilde{f}, \tilde{g}]_{\tilde{A}} \pmod{\hbar}$$
            for any lifts $\tilde{f}, \tilde{g} \in \tilde{A}$ of $f, g \in A$ (i.e. $\tilde{f} \equiv f, \tilde{g} \equiv g \pmod{\hbar}$).
        \end{definition}
        \begin{lemma}[Poisson brackets from formal flat deformations] \label{lemma: poisson_brackets_from_formal_flat_deformations}
            Let $(A, \cdot)$ be an associative $k$-algebra with a formal flat deformation $\tilde{A}$ (over $k[\![\hbar]\!]$). Then, the following - for all $f, g \in \rmZ(A)$ and all lifts $\tilde{f}, \tilde{g} \in \tilde{A}$ (i.e. $\tilde{f}, \tilde{g} \equiv f, g \pmod{\hbar}$) - makes the centre $\rmZ(A) \subseteq A$ a commutative Poisson $k$-algebra:
                $$\{f, g\} := \frac{1}{\hbar}[\tilde{f}, \tilde{g}]_{\tilde{A}} \pmod{\hbar}$$
        \end{lemma}
            \begin{proof}
                As $f, g \in \rmZ(A)$, we have that $[f, g]_A = 0$ and hence $[\tilde{f}, \tilde{g}]_{\tilde{A}} \in \hbar \tilde{A}$; we thus see firstly that the expression $\frac{1}{\hbar}[\tilde{f}, \tilde{g}]_{\tilde{A}} \in \tilde{A}$ is well-defined. It is well-known that the commutator $[-, -]_{\tilde{A}}$ is a Lie bracket on $\tilde{A}$, so $\{-, -\}$ given by $\{f, g\} := \frac{1}{\hbar}[\tilde{f}, \tilde{g}]_{\tilde{A}} \pmod{\hbar}$ for all $f, g \in \rmZ(A)$ is therefore a well-defined Lie bracket on $\rmZ(A)$. It is also easy to check that $\{f, g\}$ is depends not on the choices of lifts $\tilde{f}, \tilde{g}$. We also have that $\{f, -\}: \rmZ(A) \to \rmZ(A)$ is a derivation with respect to the multiplication $\cdot$ on $A$ for every $f \in \rmZ(A)$, owing to the fact that $[\varphi, -]_{\tilde{A}}$ is a derivation with respect to the multiplication on $\tilde{A}$, for any $\varphi \in \tilde{A}$ (one can then take $\varphi := \tilde{f}$ for any $\tilde{f} \equiv f \pmod{\hbar}$).
            \end{proof}
        \begin{corollary}[Deformation quantisations of commutative Poisson algebras induced by formal flat deformations] \label{coro: deformation_quantisation_of_poisson_algebras_from_formal_flat_deformations}
            Let $(A, \cdot)$ be a \textit{commutative} $k$-algebra with a formal flat deformation $\tilde{A}$ (over $k[\![\hbar]\!]$). Then, the following - for all $f, g \in A$ and all lifts $\tilde{f}, \tilde{g} \in \tilde{A}$ (i.e. $\tilde{f}, \tilde{g} \equiv f, g \pmod{\hbar}$) - makes $A = \rmZ(A)$ a commutative Poisson $k$-algebra:
                $$\{f, g\} := \frac{1}{\hbar}[\tilde{f}, \tilde{g}]_{\tilde{A}} \pmod{\hbar}$$
            and thus making $\tilde{A}$ a deformation quantisation of $A$ in the sense of definition \ref{def: deformation_quantisation}.
        \end{corollary}
        \begin{remark}
            We have thus seen that, formal flat deformations of 
        \end{remark}
    
    \subsection{Quasi-classical limits: from QUEs to finite-dimensional Lie bialgebras}
        \begin{definition}[Formal QUEs] \label{def: formal_QUEs}
            A \textbf{formal(ly) quantised universal enveloping algebra (QUE)} of a Lie algebra $\g$ is a formal flat Hopf algebra deformation of $\rmU(\g)$ in the sense of definition \ref{def: flat_deformations}.
        \end{definition}
    
        \begin{definition}[Finite-dimensional Lie bialgebras] \label{def: finite_dimensional_lie_bialgebras}
            A finite-dimensional \textbf{Lie bialgebra} over $k$ is a triple $(\g, [-, -], \delta)$ consisting of a finite-dimensional Lie algebra $(\g, [-, -])$ over $k$ and a Hochschild $1$-cocycle $\delta \in \HH^1(\g, \g \tensor_k \g)$ such that $\delta^*: \g^* \tensor_k \g^* \to \g^*$, which we call a \textbf{Lie cobracket} or \textbf{cocommutator}, is a Lie algebra structure on $\g^*$ (note that we are exploiting the finite-dimensionality of $\g$ to identify $(\g \tensor_k \g)^* \cong \g^* \tensor_k \g^*$). 
        \end{definition}
        \begin{remark}[Some basic properties of finite-dimensional Lie algebras]
            
        \end{remark}
    
        For $\g$ a finite-dimensional Lie algebra over $k$, the process of taking the so-called \say{classical limit} in order to obtain a finite-dimensional Lie bialgebra from a quantisation $H$ of $\rmU_{\hbar}(\g)$ amounts to constructing a \say{coproduct-like} map:
            $$\delta: \rmU(\g) \to \rmU(\g) \tensor_k \rmU(\g)$$
        whose job is to measure how non-cocommutative the comultiplication on $H$ is. It is thus natural to consider:
            $$\delta(x \pmod{\hbar}) := \frac{\Delta(x) - \Delta^{\op}(x)}{\hbar} \pmod{\hbar}$$
        for all $x \in H$; one readily checks that this map $\delta$ is well-defined up to a choice of representative $x \pmod{\hbar}$; this map is important, so we give it a name:
        \begin{definition}[Co-Poisson structures] \label{def: co_poisson_structures}
            Let $\g$ be a Lie algebra over $k$. The map $\delta: \rmU(\g) \to \rmU(\g) \tensor_k \rmU(\g)$ given by $\delta(x \pmod{\hbar}) := \frac{\Delta(x) - \Delta^{\op}(x)}{\hbar} \pmod{\hbar}$ is called the \textbf{co-Poisson structure} on $\rmU(\g)$ induced by the quantisation $H$.
        \end{definition}
        What makes the co-Poisson structure on universal enveloping algebras $\rmU(\g)$ interesting and important is that they induce a Lie bialgebra structure on $\g$ (provided that quantisations of $\rmU(\g)$ existed in the first place). 
        \begin{lemma}[Co-Poisson structures are coalternating] \label{lemma: co_poisson_structures_are_coalternating}
            Let $\g$ be a Lie algebra over $k$ and suppose that there exists a formal quantisation $H$ of $\rmU(\g)$. Then, the restriction $\delta|_{\g}: \g \to \g \tensor_k \g$ of the co-Poisson structure on $\rmU(\g)$ given by $\delta(x \pmod{\hbar}) := \frac{\Delta(x) - \Delta^{\op}(x)}{\hbar} \pmod{\hbar}$ is coalternating, in the sense that it factors through $\g \wedge \g$.
        \end{lemma}
            \begin{proof}
            
            \end{proof}
        \begin{theorem}[Lie bialgebra structures from co-Poisson structures] \label{theorem: lie_bialgebra_structures_from_co_poisson_structures}
            Let $\g$ be a Lie algebra over $k$ and suppose that there exists a formal quantisation $H$ of $\rmU(\g)$. Then, the restriction $\delta|_{\g}: \g \to \g \tensor_k \g$ of the co-Poisson structure on $\rmU(\g)$ given by $\delta(x \pmod{\hbar}) := \frac{\Delta(x) - \Delta^{\op}(x)}{\hbar} \pmod{\hbar}$ is a Lie bialgebra structure on $\g$.    
        \end{theorem}
            \begin{proof}
                
            \end{proof}
    
    \subsection{Quantisation: from finite-dimensional Lie bialgebras to QUEs}
        In the reverse direction, namely from finite-dimensional Lie bialgebras to formal quantisations of their universal enveloping algebras, we adopt a Tannakian perspective in order to construct these quantisations; as a bonus, this approach affords us a functorial description of quantisation. 
        
        \begin{convention}
            If $\calA$ is a $k$-linear category then we shall write $\calA[\hbar]$ for the category whose objects are those of $\calA$ (i.e. $\Ob(\calA[\hbar]) := \Ob(\calA)$) and whose hom-sets are given by:
                $$\Hom_{\calA[\hbar]}(V, V') := \Hom_{\calA}(V, V')[\hbar] := \Hom_{\calA}(V, V') \tensor_k k[\hbar]$$
            for all $V, V' \in \Ob(\calA)$.
        \end{convention}
        
        \begin{definition}[Drinfeld categories of bialgebras] \label{def: drinfeld_categories_of_finite_type_bialgebras}
            Suppose that $H$ is a bialgebra over $k$, and consider the localisations:
                $$H\mod[\hbar] \to H\mod[\hbar]/\hbar^n$$
            at the thick subcategories of $H\mod[\hbar]$ spanned by $\hbar^n$-torsion $H$-modules (with $n \geq 1$); in other words, the categories $H\mod[\hbar]/\hbar^n$ are those whose objects are (left-)$H$-modules and whose hom-sets are given by:
                $$\Hom_{H\mod[\hbar]/\hbar^n}(V, V') := \Hom_{H\mod}(V, V')[\hbar]/\hbar^n$$
            for all $V, V' \in \Ob(H\mod)$. We shall also be equipping each of these categories $H\mod[\hbar]/\hbar^n$ with the fibre functor:
                $$F_H[\hbar]/\hbar^n: H\mod[\hbar]/\hbar^n \to k[\hbar]/\hbar^n\mod^{\fr}$$
            that is the forgetful functor; recall that this functor is corepresentable by $H$, i.e.:
                $$F_H[\hbar]/\hbar^n \cong \Hom_{H\mod[\hbar]/\hbar^n}(H, -) := \Hom_{H\mod}(H, -)/\hbar^n$$
                
            The so-called \textbf{Drinfeld category} of $\g$, which we shall denote by $H\mod[\![\hbar]\!]$ is then the weak $2$-limit of the diagram $\{ ( H\mod[\hbar]/\hbar^n, F_H[\hbar]/\hbar^n ) \}_{n \geq 1} := \{ F_H[\hbar]/\hbar^n: H\mod[\hbar]/\hbar^n \to k[\hbar]/\hbar^n\mod^{\fr} \}_{n \geq 1}$, i.e.:
                $$( H\mod[\![\hbar]\!], \hat{F}_H ) := 2\-\projlim_{n \geq 1} ( H\mod[\hbar]/\hbar^n, F_H[\hbar]/\hbar^n )$$
        \end{definition}
        \begin{remark}[Formal properties of Drinfeld categories] \label{remark: formal_properties_of_drinfeld_categories}
            Firstly, we note that the Drinfeld category of any bialgebra $H$ is $k[\![\hbar]\!]$-linear.
            
            Furthermore (and as the notation suggests), the category $H\mod[\![\hbar]\!]$ comes equipped with a fibre functor:
                $$\hat{F}_H: H\mod[\![\hbar]\!] \to k[\![\hbar]\!]\mod^{\tfr}$$
            to the category of topologically free $k[\![\hbar]\!]$-modules (i.e. they are of the form $V[\![\hbar]\!]$ for some $k$-vector space $V$). It is also not hard to see, given the construction of the fibre functors $F_H[\hbar]/\hbar^n$, that:
                $$\hat{F}_H \cong \projlim_{n \geq 1} F_H[\hbar]/\hbar^n \cong \Hom_{H\mod}(H, -)[\![\hbar]\!] \cong F_H[\hbar]^{\wedge}$$
            wherein $F_H[\hbar]: H\mod[\hbar] \to k[\hbar]\mod^{\fr}$ is the forgetful functor to the category of free\footnote{... or equivalently, projective $k[\hbar]$-modules, since $k[\hbar]$ is a PID, owing to $k$ being a field.} $k[\hbar]$-modules, and by $(-)^{\wedge}$ we meant the object-wise $\hbar$-adic completion, i.e.:
                $$F_H[\hbar]^{\wedge}(V) := \projlim_{n \geq 1} F_H[\hbar](V)/\hbar^n$$
            
            We thus have that:
                $$\End_{\Mon\Nat}(\hat{F}_H) \cong \End_{\Mon\Nat}(F_H[\hbar]^{\wedge}(V))$$
            and hence that there is an isomorphism of associative algebras:
                $$\End_{\Mon\Nat}(\hat{F}_H) \cong \projlim_{n \geq 1} H[\hbar]/\hbar^n \cong H[\![\hbar]\!]$$
        \end{remark}
        Clearly, $H[\![\hbar]\!]$ is a formal flat deformation of $H$ (cf. definition \ref{def: flat_deformations}), so let us now attempt to construct a (braided) monoidal structure on the category $H\mod[\![\hbar]\!]$ so as to be able to use reconstruction theory to identify $H[\![\hbar]\!]$, firstly as a $k$-bialgebra (or even has a Hopf $k$-algebra, in the event that $H$ was Hopf to begin with), and secondly as a quantisation of $H$ in the sense of definition \ref{def: deformation_quantisation}. It is not always the case that $H$ admits a quantisation, seeing how $H$ might not have been a bialgebra to begin with\todo{What is a counter-example ?}\footnote{In fact, we need $H$ to be a Hopf algebra so that the tensor bifunctor on $H\mod$ would be $H$-bilinear. Otherwise, one may only consider tensor products of two-sided $H$-modules.}; nevertheless, this is precisely why finite-dimensional Manin triples are important to us: the fact that they correspond to finite-dimensional Lie bialgebras ensures that, if $(\g, \g_+, \g_-)$ is such a triple, then the universal enveloping algebra $\rmU(\g)$ of the Lie bialgebra $\g$ will admit $\rmU(\g)[\![\hbar]\!]$ as a quantisation.