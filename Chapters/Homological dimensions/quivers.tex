\section{Hereditary rings, homological dimension \texorpdfstring{$\leq 1$}{}, and quivers}
    \subsection{Quivers}
        \subsubsection{Generalities about quivers and their representations}
            We begin by introducing so-called \say{quivers}, which is more-or-less a relaxed version of directed graphs which turn out to be rather important within representation-theoretic contexts. Our approach is that quivers are certain very simple categories, and as such their representations ought to be viewed as particular instances of representations of categories. Specifically, this means we shall be studying the various algebraic properties of quiver representations via modules over their so-called \say{category algebras}, so that notions of say, finite-type-ness of quivers, could be discussed using purely ring-theoretic arguments.
        
            \begin{definition}[Quivers] \label{def: quivers}
                For any category $\C$, a \textbf{$\C$-valued quiver} $Q$ is a diagram in $\C$ of the following form:
                    $$
                        \begin{tikzcd}
                            Q_1 \arrow[r, "t"', shift right=2] \arrow[r, "s", shift left=2] \arrow["\id_{Q_1}"', loop, distance=2em, in=215, out=145] & Q_0 \arrow["\id_{Q_0}"', loop, distance=2em, in=35, out=325]
                        \end{tikzcd}
                    $$
                wherein $Q_1, Q_0 \in \Ob(\C)$ respectively are known as the objects of \textbf{arrows/vertices} and \textbf{objects/edges}, and the arrows $s, t: Q_1 \toto Q_0$ are known as the \textbf{source} and \textbf{target} morphisms. 
            \end{definition}
            \begin{remark}[The category of quivers]
                Equivalently, one might think of $\C$-valued quivers as $\C$-valued presheaves on the category:
                    $$
                        \begin{tikzcd}
                            {\1} \arrow[r, "t"', shift right=2] \arrow[r, "s", shift left=2] \arrow["\id_{\1}"', loop, distance=2em, in=215, out=145] & {\0} \arrow["\id_{\0}"', loop, distance=2em, in=35, out=325]
                        \end{tikzcd}
                    $$
                As such, for an arbitrarily fixed target category $\C$, one obtains a category $\Quiv(\C)$ of $\C$-valued quivers and natural transformations between them. In particular, when $\C \cong \Sets$, one obtains the presheaf topos $\Quiv$ of $\Sets$-valued quivers. 
            \end{remark}
            \begin{example}[Quivers internal to $\Sets$]
                A $\Sets$-valued quiver is actually nothing but a so-called \textbf{directed graph}: such a quiver is a quadruple $(Q_1, Q_0, s, t)$ wherein $Q_1$ is a set of directed edges, $Q_0$ is the set of vertices of those direct edges, and $s, t: Q_1 \toto Q_0$ are the assignments of the sources and targets vertices (i.e. beginning and endpoints) to the aforementioned directed edges. One important detail to note here is that between two given vertices, there may be many directed edges, and there might also be loops onto the same vertex, as follows:
                    $$
                        \begin{tikzcd}
                                                                                                                                           &         & \bullet \arrow[loop, distance=2em, in=125, out=55]             &                                                        & \bullet \\
                            \bullet \arrow[r, shift right=2] \arrow[r, shift left=2] \arrow[loop, distance=2em, in=215, out=145] \arrow[r] & \bullet & \bullet \arrow[r, shift right=2] \arrow[l] \arrow[u] \arrow[d] & \bullet \arrow[ru] \arrow[rd] \arrow[l, shift right=2] &         \\
                                                                                                                                           &         & \bullet \arrow[loop, distance=2em, in=305, out=235]            &                                                        & \bullet
                        \end{tikzcd}
                    $$
                These diagrams of sets and functions, however, need not be commutative: as such, one is usually interested in the free categories associated to quivers, which are nothing but those quivers with compositions and identities added on (cf. proposition \ref{prop: free_quivers}).
                
                To be very specific, Dynkin diagrams are examples of $\Sets$-valued quivers (cf. definition \ref{def: dynkin_quivers}). 
            \end{example}
            \begin{proposition}[Sub-object classifier for $\Quiv$] \label{prop: sub_object_classifier_for_topos_of_quivers}
                
            \end{proposition}
                \begin{proof}
                    
                \end{proof}
            \begin{definition}[The double negation topology]
                
            \end{definition}
            \begin{proposition}[A separatedness criterion for quivers] \label{prop: separatedness_criterion_for_quivers}
                
            \end{proposition}
                \begin{proof}
                    
                \end{proof}
                
            \begin{proposition}[Free quivers] \label{prop: free_quivers}
                The evident forgetful functor:
                    $$\oblv: 1\-\Cat_1 \to \Quiv$$
                (which forgets compositions and the associativity of said compositions) admits a left-adjoint:
                    $$[-]: \Quiv \to 1\-\Cat_1$$
                which shall be known as the \textbf{free quiver} functor. Explicitly, this functor assigns to each ($\Sets$-valued) quiver its associated free category.
            \end{proposition}
                \begin{proof}
                    
                \end{proof}
            \begin{definition}[Quiver representations] \label{def: quiver_representations}
                Given a quiver $Q \in \Ob(\Quiv)$, a commutative ring $k$, and a $k$-linear tensor category\footnote{Aside from the obvious example of $k\mod^{\heart}$ (and $k$-linear tensor subcategories thereof, like $(k\mod^{\fin})^{\heart}$), one might also consider categories such as the derived category of $k$-modules $k\mod$.} $\calV$, its category of $k$-linear representations, denoted by $\Rep_{\calV}(Q)$, is the functor category $\Func([Q], \calV)$.
            \end{definition}
            \begin{remark}[Basic properties of quiver representations]
                Given a quiver $Q \in \Ob(\Quiv)$, a commutative ring $k$, and a $k$-linear tensor category $\calV$, its category of $k$-linear representations $\Rep_{\calV}(Q)$ will also be a $k$-linear tensor category.
                
                Of course, one could consider representations with values in categories of a more general kind than linear tensor $1$-categories, such as symmetric monoidal stable $\infty$-categories. 
            \end{remark}
            
            Let us now move on to the notion of so-called \say{quiver algebras} and discuss the roles that they play in the representation theory of $\Sets$-valued quivers. 
            \begin{definition}[Category algebras] \label{def: category_algebras}
                Let $k$ be an associative ring and $\C$ be a category. Then, the \textbf{category $k$-algebra} $k\<\C\>$ of the given category $\C$ shall be the free $k$-algebra:
                    $$k\<\C_1\> := k\<\Mor(\C)\>$$
                on the set $\C_1 := \Mor(\C)$ of morphisms of $\C$: here, the multiplicative structure is given by compositions of morphisms in $\C$ (if $f\in \Mor(\C)$ can not be post-composed with $g \in \Mor(\C)$ then we set $fg = 0$).
            \end{definition}
            \begin{convention}[Path algebras ?]
                Sometimes the category algebra of the free category associated to a given $\Sets$-valued quiver is also called the path algebra of that quiver.
            \end{convention}
            \begin{remark}[Category algebras of quivers] \label{remark: category_algebras_of_quivers}
                Fix a ring $k$ along with a $\Sets$-valued quiver $Q$. The \textbf{quiver $k$-algebra} $k\<Q\>$ associated to said quiver is then the category $k$-algebra $k\<[Q]\>$ of the free quiver $[Q]$ (cf. proposition \ref{prop: free_quivers}). 
                
                Due to this definition, one obtains the following diagram of $1$-categories and functors, wherein the left-adjoints $[-]$ and $k\<-\>$ along with the horizontal arrows form a commutative square, and the unlabelled arrows are the obvious forgetful functors:
                    $$
                        \begin{tikzcd}
                        	\Quiv & {k\-\Assoc\Alg} \\
                        	\\
                        	{1\-\Cat_1} & \Sets
                        	\arrow["{(-)_1}"', from=3-1, to=3-2]
                        	\arrow[""{name=0, anchor=center, inner sep=0}, bend right, from=3-1, to=1-1]
                        	\arrow[""{name=1, anchor=center, inner sep=0}, "{[-]}"', bend right, from=1-1, to=3-1]
                        	\arrow[""{name=2, anchor=center, inner sep=0}, "{k\<-\>}", bend left, from=3-2, to=1-2]
                        	\arrow[""{name=3, anchor=center, inner sep=0}, bend left, from=1-2, to=3-2]
                        	\arrow["{k\<[-]_1\>}", dashed, from=1-1, to=1-2]
                        	\arrow["\dashv"{anchor=center}, draw=none, from=1, to=0]
                        	\arrow["\dashv"{anchor=center}, draw=none, from=2, to=3]
                        \end{tikzcd}
                    $$
                $[-]$ and $k\<-\>$ are left-adjoint and hence preserves all colimits, and it is easy to see that the functor:
                    $$(-)_1: 1\-\Cat_1 \to \Sets$$
                assigning to each $1$-category $\C$ its set of morphisms $\C_1$ preserves all coproducts. As such, the formation of quiver algebras preserves coproducts that exist in $\Quiv$ (and indeed, they all exist, since $\Quiv$ is a topos); this will become relevant once we would like to discuss representation-theoretic consequences of a quiver being \say{connected} (cf. definition \ref{def: connected_quivers}).
            \end{remark}
            \begin{example}[Path algebras of discrete quivers]
                The simplest example of a quiver is the trivial loop:
                    $$
                        \begin{tikzcd}
                            \bullet \arrow["\id"', loop, distance=2em, in=35, out=325]
                        \end{tikzcd}
                    $$
                It is easy to see that the path algebra of this quiver (over some fixed ring $k$) is:
                    $$k\<e\>/\<e^2 = e\> \cong k$$
                with one small note on the side being that the generator $e$ is not \textit{a priori} invertible, since formally, there does not exist a morphism $\id^{-1}$ in the given category, and one is only permitted to compose \say{forwardly}, i.e. to consider the compositions $\id, \id^2, \id^3, ...$.
                    
                Going off in a different direction\footnote{Pun may or may not have been intended.}, consider the following quiver $Q := (Q_1, Q_0, s, t)$ (i.e. a discrete category\footnote{Let's ignore potential set-theoretic issues for now!}):
                    $$
                        \begin{tikzcd}
                            \cdots & \bullet_i \arrow["\id_{\bullet_i}"', loop, distance=2em, in=125, out=55] & \bullet_j \arrow["\id_{\bullet_j}"', loop, distance=2em, in=125, out=55] & \bullet_k \arrow["\id_{\bullet_k}"', loop, distance=2em, in=125, out=55] & \cdots
                        \end{tikzcd}
                    $$
                Its path algebra is:
                    $$k\<\{e_i\}_{i \in Q_0}\>\/\<\forall i \in Q_0: e_i^2 = e_i, \forall i, j \in Q_0: e_i e_j = e_j e_i = \delta_{ij}\>$$
                (where $\delta_{ij}$ is the Kronecker delta), which we recognise as being isomorphic to the product:
                    $$\prod_{i \in Q_0} k$$
                taken in the category of associative $k$-algebras.
            \end{example}
            \begin{example}[Path algebras of $A_n$ quivers]
                For a simple yet non-trivial example of quiver algebras, consider the following quiver, commonly known as the \say{$A_2$ quiver}:
                    $$
                        \begin{tikzcd}
                            {\bullet_1} \arrow["\id_{\bullet_1}"', loop, distance=2em, in=215, out=145] \arrow[r] & {\bullet_2} \arrow["\id_{\bullet_2}"', loop, distance=2em, in=35, out=325]
                        \end{tikzcd}
                    $$
                which has a rather small path algebra, isomorphic to:
                    $$k\<e_1, e_2, x\>/\<e_i^2 = e_i, e_i e_j = e_j e_i = \delta_{ij}, x^2 = 0, e_2 x = x e_1 = x\>$$
                One observation to make is that there is an associative $k$-algebra isomorphism given by:
                    $$e_1 \mapsto \begin{pmatrix} 1 & 0 \\ 0 & 0 \end{pmatrix}$$
                    $$e_2 \mapsto \begin{pmatrix} 0 & 0 \\ 0 & 1 \end{pmatrix}$$
                    $$x \mapsto \begin{pmatrix} 0 & 0 \\ 1 & 0 \end{pmatrix}$$
                between this path $k$-algebra and the $k$-algebra:
                    $$\b_2^-(k) := \left\{ \begin{pmatrix} * & 0 \\ * & * \end{pmatrix} \in \Mat_2(k) \right\}$$
                of lower-triangular $2 \x 2$ matrices with entries in $k$.
                    
                Next, consider the general $A_n$ quiver\footnote{Henceforth we shall start omiting the identity paths from depictions of quivers.} (for some finite positive integer $n$):
                    $$
                        \begin{tikzcd}
                        	{\bullet_1} & {\bullet_2} & {\bullet_3} & \cdots
                        	\arrow["{x_{12}}", from=1-1, to=1-2]
                        	\arrow["{x_{23}}", from=1-2, to=1-3]
                        	\arrow["{x_{34}}", from=1-3, to=1-4]
                        \end{tikzcd}
                    $$
                Observe that the set of morphisms $[A_n]_1$ of the free category $[A_n]$ on $A_n$ is nothing but $\{x_{ij}\}_{1 \leq i \leq j \leq n}$ (also, note that $e_i = x_{ii}$ for all $1 \leq i \leq n$). Now, by letting $\1_{ij}$ denote the $n \x n$ matrix with $1$ at the $ij^{th}$ entry and $0$ everywhere else, and consider the map:
                    $$k\<A_n\> \to \Mat_n(k)$$
                    $$x_{ij} \mapsto \1_{ij}$$
                one then sees that $k\<A_n\>$ is isomorphic to the $k$-algebra of lower-triangular $n \x n$ matrices with entries in $k$.
            \end{example}
            \begin{example}[Path algebras of quivers with loops]
                If we were to add a non-trivial loop $x: \bullet \to \bullet$ to the trivial $(\{\bullet\}, \{\id\})$ to yield:
                    $$
                        \begin{tikzcd}
                            \bullet \arrow["x"', loop, distance=2em, in=35, out=325]
                        \end{tikzcd}
                    $$
                then the resulting path algebra will be:
                    $$k\<e, x\>/\<e^2 = e, ex = xe = x\> \cong k[x]$$
                    
                More generally, we can consider the quiver with one vertex and a set $\{x_i\}_{i \in I}$ of (possibly infinitely) many loops on that vertex:
                    $$
                        \begin{tikzcd}
                            \bullet \arrow["x_1"', loop, distance=2em, in=305, out=235] \arrow["x_2"', loop, distance=2em, in=35, out=325] \arrow["x_3"', loop, distance=2em, in=125, out=55] \arrow["\cdots"', loop, distance=2em, in=215, out=145]
                        \end{tikzcd}
                    $$
                then the resulting path algebra will be:
                    $$k\<\{x_i\}_{i \in I}\>$$
                (as opposed to $k[\{x_i\}_{i \in I}]$), because while one can free form compositions $x_j \circ x_i \circ ...$ of the endomorphisms/loops $x_i \in \End(\bullet)$, there is not a reason to expect that $x_j x_i = x_i x_j$. 
            \end{example}
            \begin{example}[Path algebras of non-simply-laced quivers]
                
            \end{example}
            \begin{proposition}[Quiver representations are modules over quiver algebras] \label{prop: quiver_representations_are_modules_over_quiver_algebras}
                Let $k$ be a commutative ring. Then, there is an exact and monoidal equivalence of $k$-linear categories (fibred over $k\mod$):
                    $$\Rep_{k}(Q) \to {}^lk\<Q\>\mod$$
            \end{proposition}
                \begin{proof}
                    
                \end{proof}
            \begin{remark}
                The equivalence of categories:
                    $$\Rep_{k}(Q) \to {}^lk\<Q\>\mod$$
                from proposition \ref{prop: quiver_representations_are_modules_over_quiver_algebras} has many important further properties, which all come from the very definition of linear representations of associative algebras (in this case, path algebras of quivers) themselves. For instance, this functor preserves (semi-)simplicity and (in)decomposability.
                
                As for further properties coming not from the definition of representations themselves, but rather from the basic property of this equivalence of categories, one has that the functor also preserves lengths and more generally, ascending and descending chains of objects, as a consequence of being exact. In particular, this means that should a representation of a given quiver $Q$ is finitely generated as a $k$-module (i.e. \say{finite-dimensional}), then the same would also be true for the corresponding left-$k\<Q\>$-module.
            \end{remark}
                
            Let us now move on to a more in-depth discussion of connected quivers and in particular, those of \say{finite type}, which can be shown to precisely be the so-called \say{Dynkin quivers} (cf. definition \ref{def: dynkin_quivers}), which are simply laced connected quivers of a certain kind. 
            \begin{definition}
                A $\Sets$-valued quiver is said to have a ring-theoretic property $\calP$ over some base ring $k$ if and only if its quiver $k$-algebra has property $\calP$.
            \end{definition}
            \begin{example}[Quivers of finite type]
                A quiver $Q$ is said to be of finite type over some ring $k$ if and only if its path $k$-algebra $k\<Q\>$ is of finite type as a $k$-algebra.
            \end{example}
            \begin{proposition}[Quivers of finite type are of finite presentations] \label{prop: quivers_of_finite_type_are_of_finite_presentations}
                Let $k$ be a ring and $Q$ be a $\Sets$-valued quiver. Then $Q$ is of finite type if and only if it is of finite presentation over $k$.
            \end{proposition}
                \begin{proof}
                    If $k\<Q\>$ is of finite presentation then obviously it is of fintie type. Conversely, if $k\<Q\>$ is of finite presentation, then by definition the set $[Q]_1$ of morphisms of the free category $[Q]$ on $Q$ will have to be finite. These finitely many morphisms will generate finitely many relations on the noncommutative free $k$-algebra $k\<\{v_i\}_{1 \leq i \leq |Q_1|}\>$ (which we note to be of finite type over $k$), thereby making the algebra $k\<Q\>$ finitely presented over $k$.
                \end{proof}
            \begin{definition}[Simply laced quivers] \label{def: simply_laced_quivers}
                A $\Sets$-valued quiver $Q := (Q_1, Q_0, s, t)$ is \textbf{simply laced} if and only if none of its vertices has self-loops: that is, for all $v \in Q_0$, one has:
                    $$\{f \in Q_1 \mid s(f) = t(f) = v\} = \{\id_v\}$$
            \end{definition}
            \begin{remark}
                Equivalently, one might characterise the simply laced quivers $Q := (Q_1, Q_0, s, t)$ as those wherein all non-identity paths $f \in Q_1$ generate square-free elements $f \in k\<Q\>$ (i.e. elements such that $f^2 = 0$). This is because the lack of non-identity loops implies that there is no non-identity path that can be composed with itself. 
            \end{remark}
            \begin{definition}[Connected quivers] \label{def: connected_quivers}
                A $\Sets$-valued quiver $Q: \{\1, \0, s, t\}^{\op} \to \Sets$ is said to be \textbf{connected} if and only if for all $v, w \in Q(\0)$, there exists $f \in Q(\1)$ such that $s(f) = v$ and $t(f) = w$ (i.e. it has no isolated vertices).
            \end{definition}
            \begin{remark}
                Equivalently, one can say that a $\Sets$-valued (or maybe with values in any category $\C$ with enough coproducts and sub-objects) quiver is connected if and only if it can not be written as the disjoint union of two sub-quivers. 
            \end{remark}
            
        \subsubsection{Roots and quadratic forms of quivers; Dykin quivers}
            \begin{definition}[Adjacency and Cartan matrices] \label{def: adjacency_and_cartan_matrices}
                Let $Q := (Q_1, Q_0, s, t)$ be a finite quiver\footnote{With value in any category} and suppose that $n := |Q_0|$; also, let us give the set $Q_0$ of vertices a \textit{fixed} enumeration $\{v_1, ..., v_n\}$. To such a quiver, one can associate a so-called \textbf{adjacency matrix}:
                    $$R_Q \in \Mat_n(\Z)$$
                given by:
                    $$R_Q := (r_{ij} := |\{f \in Q_1 \mid s(f) = v_i, t(f) = v_j\}|)_{1 \leq i, j \leq n}$$
                (that is, the entries $r_{ij}$ of $R_Q$ are the numbers of \textit{undirected} edges from the vertex $v_i$ to the vertex $v_j$). From this matrix, one can define the \textbf{Cartan matrix} of $Q$:
                    $$A_Q := 2I_n - R_Q$$
                along with a $\Z$-bilinear form, which we shall call the \textbf{Cartan form}:
                    $$B_Q: \Z^{\oplus n} \x \Z^{\oplus} \to \Z$$
                    $$(x, y) \mapsto x^{\top} A_Q y$$                
            \end{definition}
            \begin{definition}[Dynkin quivers] \label{def: dynkin_quivers}
                A \textbf{Dynkin quiver} is a \textit{connected} and \textit{simply laced} finite quiver whose Cartan form is positive-definite.
            \end{definition}
            \begin{proposition}[Evenness of Cartan forms of Dynkin quivers]
                Let $\Gamma := (\Gamma_1, \Gamma_0, s, t)$ be a Dynkin quiver and consider its Cartan matrix $A_{\Gamma}$. Then for all $x \in \Z^{\oplus n}$, $B_{\Gamma}(x, x)$ is even.  
            \end{proposition}
                \begin{proof}
                    Let $n := |\Gamma_0|$ and pick a basis for $\Z^{\oplus n}$ in order to write:
                        $$
                            \begin{aligned}
                                B_{\Gamma}(x, x) & = x^{\top} A_{\Gamma} x
                                \\
                                & = \sum_{1 \leq i, j \leq n} x_i a_{ij} x_j
                                \\
                                & = \sum_{1 \leq i, j \leq n} x_i (2\delta_{ij} - r_{ij}) x_j \text{(wherein $\delta_{ij}$ is the Kronecker delta)}
                                \\
                                & = 2\sum_{1 \leq i, j \leq n} x_i \delta_{ij} x_j - \sum_{1 \leq i, j \leq n} x_i (1 - \delta_{ij}) r_{ij} x_j \text{($r_{ij}$ are the entries the adjacency matrix of $\Gamma$)}
                                \\
                                & = 2\sum_{1 \leq i \leq n} x_i^2 - 2\sum_{1 \leq i < j \leq n} x_i r_{ij} x_j
                            \end{aligned}
                        $$
                    for all $x \in \Z^{\oplus n}$, wherein the last line is due to the fact that the number of undirected edges from a vertex $v_i$ to another $v_j$ is equal to the number of undirected directed edges from $v_j$ to $v_i$, which in turn is a result of the assumption that by virtue of being a Dynkin quiver, $\Gamma$ is connected and simply laced\footnote{Note how this argument fails if $\Gamma$ is either not simply laced or not connected.}. Clearly, $B_{\Gamma}(v, v)$ is even for all $x \in \Z^{\oplus}$.
                \end{proof}
            \begin{definition}[Roots] \label{def: roots_of_dynkin_quivers}
                A root of a \textit{Dynkin} quiver $\Gamma : (\Gamma_1, \Gamma_0, s, t)$ is a vector:
                    $$\alpha \in \Z^{\oplus n}$$
                (where $n := |\Gamma_0|$) such that:
                    $$B_{\Gamma}(\alpha, \alpha) = 2$$
                The set of all roots of $\Gamma$ is denoted by $\Phi(\Gamma)$, and often called the \textbf{root system} of $\Gamma$.
            \end{definition}
            \begin{example}[Simple roots] \label{example: simple_roots}
                Let $\Gamma := (\Gamma_1, \Gamma_0, s, t)$ be a Dynkin quiver and set $n := \Gamma_0$. Also, pick a basis $\{e_1, ..., e_n\}$ for $\Z^{\oplus n}$. Then the vectors of the form:
                    $$\alpha_i := \delta_{ij} e_j$$
                (for $1 \leq i, j \leq n$) are in fact roots of $\Gamma$ and furthermore, they form a basis for $\Z^{\oplus n}$.
            \end{example}
            \begin{lemma}[Roots are exclusively either negative or positive] \label{lemma: roots_are_exclusively_either_negative_or_positive}
                Let $\Gamma := (\Gamma_1, \Gamma_0, s, t)$ be a Dynkin quiver and set $n := \Gamma_0$. Also, pick a basis $\{e_1, ..., e_n\}$ for $\Z^{\oplus n}$ and choose a root $\alpha$ and write it in terms of the simple roots $\{\alpha_i\}_{1 \leq i \leq n}$ as:
                    $$\alpha := \sum_{1 \leq i \leq n} c_i \alpha_i$$
                Then either $c_i \geq 0$ or $c_i \leq 0$ for all $1 \leq i \leq n$ simultaneously.
            \end{lemma}
                \begin{proof}
                    
                \end{proof}
            \begin{definition}[Positive and negative roots] \label{def: negative_and_positive_roots}
                Let $\Gamma := (\Gamma_1, \Gamma_0, s, t)$ be a Dynkin quiver and set $n := \Gamma_0$. Also, pick a basis $\{e_1, ..., e_n\}$ for $\Z^{\oplus n}$ and choose a root $\alpha$ and write it in terms of the simple roots $\{\alpha_i\}_{1 \leq i \leq n}$ as:
                    $$\alpha := \sum_{1 \leq i \leq n} c_i \alpha_i$$
                Then we say that $\alpha$ is \textbf{positive} if and only if $c_i \geq 0$ for all $1 \leq i \leq n$ and \textbf{negative} if and only if $c_i \leq 0$ for all $1 \leq i \leq n$.
            \end{definition}
            \begin{example}
                
            \end{example}
            
        \subsubsection{Quivers associated to finite algebra}
        
        \subsubsection{Auslander-Reiten theory}
        
    \subsection{Tiltings}
        \subsubsection{Torsion in abelian categories}
            \begin{definition}[Torsion pairs] \label{def: torsion_pairs}
                Let $\calA$ be an abelian category. A \textbf{torsion pair} or \textbf{torsion theory} in $\calA$ is then a pair of full subcategories $(\calT^{\flat}, \calT^{\sharp})$ such that the $\Z$-modules $\calT^{\flat}(M, N)$ and $\calT^{\sharp}(M, N)$ are zero if and only if $M \in \Ob(\calT^{\flat})$ and $N \in \Ob(\calT^{\sharp})$.
            \end{definition}
            \begin{example}[Torsion and torsion-free abelian groups]
                It is well known that there are no non-zero homomorphism from a torsion abelian group to a torsion-free one. Furthermore, abelian group homomorphisms preserve torsion (or equivalently, torsion-free-ness). As such, an example of a torsion pair on the (abelian) category of abelian groups is the pair consisting of the full subcategory $\calT^{\flat}_{\Ob(\Tor(\Z\mod))} := \Tor(\Z\mod)$ of torsion abelian groups and that of torsion-free abelian groups (serving as $\calT^{\sharp}$). 
            \end{example}
            \begin{example}[Torsion pairs generated by classes of objects]
                Let $\calA$ be an abelian category and suppose that $\C_0 \subseteq \Ob(\calA)$ is a class of objects therein. Then, a torsion pair $(\calT^{\flat}_{\C_0}, \calT^{\sharp}_{\C_0})$ induced by $\C$ can be constructed in the following fashion:
                    $$\calT^{\sharp}_{\C_0} := \{N \in \Ob(\calA) \mid \forall M \in \C_0 : \calA(M, N) \cong 0\}$$
                    $$\calT^{\flat}_{\C_0} := \calA \setminus \calT^{\sharp}_{\C_0}$$
                where in the second line, the complement is taken at both the level of objects and morphisms. Observe that $\Ob(\T^{\flat}_{\C_0})$ is minimal among the subclasses of $\Ob(\calA)$ containing $\C_0$.
                
                To be concrete, consider once more the category of abelian groups, but this time, instead of consider the class of all torsion abelian groups, let us fix a prime $p$ and consider the class of $p$-torsion abelian groups (these are abelian groups $M$ such that for all $x \in M$, $px = 0$). The full subcategory spanned by the so-called \say{$p$-torsion-free} abelian groups can then be given by:
                    $$\T^{\sharp}_{\Ob(p\-\Tor(\Z\mod))} := \{N \in \Ob(\Z\mod) \mid \forall M \in \Ob(p\-\Tor(\Z\mod)): \Hom_{\Z}(M, N) \cong 0\}$$
                which indeed yields
                    $$p\-\Tor(\Z\mod) = \Z\mod \setminus \T^{\sharp}_{\Ob(p\-\Tor(\Z\mod))}$$
            \end{example}
            
        
        \subsubsection{Tilting objects}
        
        \subsubsection{Separating and splitting tilting objects}
        
        \subsubsection{Torsion induced by tilting objects}
        
    \subsection{Hereditary rings}    
        \subsubsection{Hereditary algebras and hereditary categories}
            \begin{definition}[Hereditary abelian categories] \label{def: hereditary_abelian_categories}
                An abelian category $\calA$ is said to be \textbf{hereditary} if and only if $\globdim \calA \leq 1$.
            \end{definition}
            \begin{convention}[Left/right-hereditary rings] \label{conv: left/right_hereditary_rings}
                When $\calA$ is the category of left/right-module over some ring $R$, one might say that the ring $R$ itself is \textbf{left/right-hereditary}
            \end{convention}
            
        \subsubsection{Reflection functors and Gabriel's Theorem}