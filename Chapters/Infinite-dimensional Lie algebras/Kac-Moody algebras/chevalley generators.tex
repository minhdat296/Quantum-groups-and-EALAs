\section{The construction and structure of general Kac-Moody algebras}
    \begin{convention}
        Throughout this section, suppose that $\Q$ is equipped with the usual absolute value $|-|$ so that $(\Q, |-|)$ would be an archimedean field, and suppose also that $k/\Q$ is an archimedean subextension of $\bbC/\Q$ so that notions such as positivity and negativity could be meaningfully discussed. Typically, $k$ is taken to either be $\Q, \R$, or $\bbC$.
    \end{convention}

    \subsection{Kac-Moody algebras via generators and relations}
        \subsubsection{Root data of square matrices}
            \begin{definition}[Root data of square matrices] \label{def: root_data_of_square_matrices}
                An $r$-dimensional \textbf{root datum} of a square matrix $A := (a_{ij})_{1 \leq i, j \leq n} \in \Mat_n(k)$ is a quintuple $(\h, \Pi, \h^{\vee}, \Pi^{\vee}, B)$ wherein:
                    \begin{itemize}
                        \item $\h$ is an $k$-vector space of dimension $r \geq n + \dim_k(\ker A)$ (when equality occurs, we say that the root datum is \textbf{minimal}) and $\h^{\vee} \cong \h^*$;
                        \item $\Pi := \{\alpha_j\}_{1 \leq j \leq n} \subset \h^{\vee}$ and $\Pi^{\vee} := \{\beta_i\}_{1 \leq i \leq n} \subset \h$ are $k$-linearly independents subsets of $\h^{\vee}$ and $\h$ respectively; elements of $\Pi$ and $\Pi^{\vee}$ are called \textbf{simple roots} and \textbf{simple coroots} respectively;
                        \item we also require that there is a bijection\footnote{Note that such a bijection can always be chosen, since the sets $\Pi, \Pi^{\vee}$ are of the same cardinality by assumption.} $(-)^{\vee}: \Pi \to \Pi^{\vee}$;
                        \item $B: \h^{\vee} \x \h \to k$ is a non-degenerate perfect\footnote{If $R$ is a commutative ring and $V, W$ are finite $R$-modules and $B: W \x V \to k$ is an $R$-bilinear map then said $R$-bilinear map is said to be \textbf{perfect} if and only if for all $w \in W$ and all $v \in V$, the canonically induced map $W \to V^*: w \mapsto B(w, -)$ and $V \to W^*: w \mapsto B(-, v)$ $R$-module isomorphisms.} symmetric $k$-bilinear pairing, sometimes called the \textbf{Cartan-Tits form}\footnote{The name originates from the theory of finite-type Dynkin quivers.}, such that $B(\alpha_i, \alpha_j^{\vee}) = a_{ij}$.
                    \end{itemize}
            \end{definition}
            \begin{remark}[Dualising (co)roots and induced simple reflections] \label{remark: dualising_(co)roots_and_induced_simple_reflections}
                Suppose that $(\h, \Pi, \h^{\vee}, \Pi^{\vee}, B)$ is a $k$-linear root datum of some square matrix $A \in \Mat_n(k)$. Thanks to the assumption that $B: \h \x \h^{\vee} \to k$ is a perfect $k$-bilinear pairing, one has for every $\beta \in \h$ and all every $\alpha \in \h^{\vee}$, a pair of $k$-vector space isomorphisms $\h \to \h^{\vee}: \beta \mapsto B(\beta, -)$ and $\h^{\vee} \to \h: \alpha \mapsto B(-, \alpha)$. From this, it is easy to see that for every $\beta \in \h$ and all every $\alpha \in \h^{\vee}$, one can construct corresponding $k$-linear automorphisms on $\h^{\vee}$ and $\h$, respectively, given by $\mu \mapsto \mu - B(\beta, \mu) \beta^{\vee}$ and $\h \to \h: \lambda \mapsto \lambda - \alpha^{\vee} B(\lambda, \alpha)$.
            \end{remark}
            \begin{example}[Root data of Dynkin quivers] \label{example: root_data_of_dynkin_quivers}
                The root system (over $\Q(\sqrt{2}), \R$, or $\bbC$) of any Dynkin quiver $\Gamma := (\Gamma_1, \Gamma_0, s, t)$ is an instance of a root datum, particular of the associated Cartan matrix $\Car(\Gamma) := (a_{ij})_{1 \leq i, j \leq n}$, whose entries we recall to be given by $a_{ij} := 2\delta_{ij} - r_{ij}$, with $r_{ij}$ being the number of undirected edges from vertex $i$ to vertex $j$. 
            \end{example}
            \begin{proposition}[Existence of minimal root data] \label{prop: existence_of_minimal_root data}
                \cite[Proposition 14.2]{carter_affine_lie_algebras} Every square matrix $A \in \Mat_n(k)$ admits a minimal root datum which is unique up to isomorphisms.
            \end{proposition}
            \begin{definition}[Morphisms of root data] \label{def: morphisms_of_root data}
                Let $A \in \Mat_n(k)$ be a fixed $n \x n$ matrix with entries in $k$. Associated to such a matrix, one can construct a category $\Root_k(A)$ wherein the objects are $k$-linear root data of $A$ and a morphism:
                    $$\varphi: (\h_1, \Pi_1, \h_1^{\vee}, \Pi_1^{\vee}, B_1) \to (\h_2, \Pi_2, \h_2^{\vee}, \Pi_2^{\vee}, B_2)$$
                between two such objects is a $k$-linear map $\varphi: \h_1 \to \h_2$ such that $\varphi(\Pi_1) = \Pi_2$ and $B_2(\varphi(\beta^{\vee})^{\vee}, \varphi(\alpha)) = B_1(\beta, \alpha)$. Often, we shall be interested in the full subcategory $\Root_k^{\red}(A)$ of minimal $k$-linear root data of $A$ and isomorphisms between them\footnote{Note that this is a groupoid.}; note that by proposition \ref{prop: existence_of_minimal_root data}, $\Ob(\Root_k^{\red}(A))$ is in fact the set of \textit{all} minimal $k$-linear root data of $A$.
            \end{definition}
    
        \subsubsection{The structure of non-reduced Kac-Moody algebras}
            \begin{definition}[Free Lie algebras] \label{def: free_lie_algebras}
                The functor of \textbf{free Lie algebras} over $k$:
                    $$k\-\lie: \Sets \to k\-\Lie\Alg$$
                arises as the composition $k\-\lie(-) := T_k^{\bullet}(-) \circ k^{\oplus (-)}$ of the functor $k^{\oplus (-)}: \Sets \to k\mod$ of free $k$-vector spaces and $T_k^{\bullet}(-): k\mod \to k\-\Lie\Alg$ which assigns to $k$-vector spaces $V$ the tensor algebra $T_k^{\bullet}(V) := \bigoplus_{n \geq 0} V^{\tensor n}$ with the usual commutator as the Lie bracket. 
            \end{definition}
            \begin{remark}
                From definition \ref{def: free_lie_algebras}, one sees that $k\-\lie: \Sets \to k\-\Lie\Alg$ is left-adjoint to the forgetful functor $\oblv: k\-\Lie\Alg \to \Sets$ as a result of both $k^{\oplus (-)}$ and $T_k^{\bullet}(-)$ being left-adjoints, and hence the functor $k\-\lie$ preserves colimits. 
            \end{remark}
            \begin{definition}[Generalised Cartan matrices] \label{def: generalised_cartan_matrices}
                An $n \x n$ \textbf{generalised Cartan matrix} (often abbreviated down to GCM) is a matrix $A := (a_{ij})_{1 \leq i, j \leq n} \in \Mat_n(\Z)$ whose entries $a_{ij}$ are such that $a_{ij} = 2$ if $i = j$, $a_{ij} \leq 0$ if $i \not = j$, and $a_{ij} \not = 0$ if and only if $a_{ji} \not = 0$.
            \end{definition}
            \begin{definition}[Non-reduced Kac-Moody algebras] \label{def: non_reduced_kac_moody_algebras}
                Suppose that we are given the datum consisting of:
                    \begin{itemize}
                        \item A GCM $A := (a_{ij})_{1 \leq i, j \leq n} \in \Mat_n(\Z)$ and
                        \item a minimal $k$-linear root datum $(\h, \Pi, \h^{\vee}, \Pi^{\vee}, B)$ of $A$, wherein $\Pi := \{\alpha_i\}_{1 \leq i \leq n} \subset \h^{\vee}$ and $\Pi^{\vee} := \{\alpha_j^{\vee}\}_{1 \leq j \leq n} \subset \h$. 
                    \end{itemize}
                To such a datum, one can associate a Lie $k$-algebra:
                    $$\bfLie(A) := k\-\lie(\h \cup \{e_i^-, e_i^+\}_{1 \leq i \leq n})/\chev(A)$$
                which we shall refer to as the \textbf{non-reduced Kac-Moody $k$-algebra} associated to the GCM $A$, which is the quotient of the free Lie $k$-algebra on the set $\h \cup \{e_i^-, e_i^+\}_{1 \leq i \leq n}$ of so-called \textbf{Chevalley generators} by the ideal $\chev(A)$ generated by the so-called \textbf{Chevalley relations}, which are given by:
                    $$\forall h, h' \in \h: [h, h'] = 0$$
                    $$\forall h \in \h: \forall 1 \leq i \leq n: [h, e_i^{\pm}] = \mp B(\alpha_i, h) e_i^{\pm}$$
                    $$\forall 1 \leq i, j \leq n: [e_i^-, e_j^+] = \delta_{ij} \alpha_i^{\vee}$$
            \end{definition}
            \begin{remark}[Non-reduced Kac-Moody algebras are uniquely determined by GCMs] \label{remark: non_reduced_kac_moody_algebras_are_uniquely_determined_by_generalised_cartan_matrices}
                The universal property of the free Lie $k$-algebra functor (cf. definition \ref{def: free_lie_algebras}) ensures that every Lie algebra is uniquely determined (up to isomorphisms, of course) by its presentation, so as a consequence, the assignment of non-reduced Kac-Moody algebras to GCMs defines a bijection:
                    $$
                        \begin{tikzcd}
                        	{\left\{ \text{GCMs over $k$} \right\}} & {\left\{ \text{Non-reduced Kac-Moody $k$-algebras} \right\}}
                        	\arrow["{\bfLie(-)}", from=1-1, to=1-2]
                        \end{tikzcd}
                    $$
            \end{remark}
            \begin{convention}[Tame and wild non-reduced Kac-Moody algebras] \label{conv: tame_and_wild_non_reduced_kac_moody_algebras}
                The non-reduced Kac-Moody algebra $\bfLie(A)$ associated a symmetric/symmetrisable GCM $A$ is said to be \textbf{symmetric/symmetrisable}.
                
                The symmetrisable non-reduced Kac-Moody algebra $\bfLie(A)$ said to be of finite type (respectively, of affine type, tame, and wild) if and only if $A$ is of finite type (respectively, of affine type, tame, and wild) in the sense of definition \ref{def: tame_and_wild_generalised_cartan_matrices}.
            \end{convention}
            
            \begin{definition}[Weight/root lattices and abstract root spaces] \label{def: weight_lattices_and_abstract_root_spaces}
                Choose a minimal root datum $(\h, \Pi, \h^{\vee}, \Pi^{\vee}, B)$ of a fixed $n \x n$ matrix $A \in \Mat_n(k)$. The \textbf{weight lattice} associated to the aforementioned choice of $\h$ is then the free $\Z$-module $\bbX(\bfLie(A), \h) := \span_{\Z} \Pi$. Elements of this free $\Z$-modules are often referred to as \textbf{abstract weights} or \textbf{abstract roots}\footnote{Traditionally, one usually excludes $0$ from being a (co)weight/root, but we shall not be following this convention since we would like to put emphasis on the fact that the Cartan subalgebra is identified with the root space of $0$ (cf. definition \ref{def: cartan_subalgebras_of_non_reduced_kac_moody_algebras}).}, and to any such element $\lambda$, associate the corresponding \textbf{abstract root space}\footnote{Later on, we shall see that these root spaces are actually specific instances of so-called \say{weight spaces} (cf. definition \ref{def: cartan_diagonalisability}). In particular, they are the weight spaces associated to the adjoint representation of a given non-reduced Kac-Moody algebra; so for instance, in the presence of other $\g$-representations, one might write $(\g, \ad)_{\lambda}$ instead of $\bfLie(A)_{\lambda}$ should we need to be specific.} to be:
                    $$\bfLie(A)_{\lambda} := \{v \in \g \mid \forall h \in \h: \ad(h)(v) = B(\lambda, h)v\}$$
                If $\bfLie(A)_{\lambda} \not = 0$ then $\bfLie(A)_{\lambda}$ shall be referred to simply as a \textbf{root space}, and the corresponding abstract root $\lambda \in \bbX(\bfLie(A), \h)$ shall likewise be referred to simply as a \textbf{root}; the set of roots - with respect to the chosen $\h$ - is nothing but:
                    $$\Phi(\g, \h) := \{\lambda \in \bbX(\bfLie(A), \h) \mid \bfLie(A)_{\lambda} \not = 0\}$$
            \end{definition}
            \begin{remark}[Root spaces and commutators] \label{remark: root_spaces_and_commutators}
                Let $\g$ be a non-reduced Kac-Moody algebra associatd to some GCM $A \in \Mat_n(\Z)$ over $k$ and let $(\h, \Pi, \h^{\vee}, \Pi^{\vee}, B)$ be a minimal root datum thereof. Also, fix two roots $\lambda, \lambda' \in \bbX := \span_{\Z} \Pi$. Then, for all $x \in \bfLie(A)_{\lambda}$ and all $y \in \g_{\lambda'}$, observe the following for any $h \in \h$:
                    $$\ad(h)([x, y]) = [\ad(h)(x), \ad(h)(y)] = [B(\lambda, h)x, B(\lambda', h)y] = B(\lambda + \lambda', h)([x, y])$$
                from which one concludes that $[x, y] \in \g_{\lambda + \lambda'}$.
            \end{remark}
            \begin{convention} \label{conv: unipotent_radicals_of_non_reduced_kac_moody_algebras}
                Suppose that we are given a GCM $A := (a_{ij})_{1 \leq i, j \leq n} \in \Mat_n(\Z)$ along with a minimal $k$-linear root datum $(\h, \Pi, \h^{\vee}, \Pi^{\vee}, B)$ of $A$, wherein $\Pi := \{\alpha_i\}_{1 \leq i \leq n} \subset \h^{\vee}$ and $\Pi^{\vee} := \{\alpha_j^{\vee}\}_{1 \leq j \leq n} \subset \h$, and denote the set of Chevalley generators of $\bfLie(A)$ by $\h \cup \{e_i^-, e_i^+\}_{1 \leq i \leq n}$. Then, we shall be writing:
                    $$\frakN^{\pm}(A) := k\-\lie(\{e_i^{\pm}\}_{1 \leq i \leq n})$$
                    $$\frakB^{\pm}(A) := k\-\lie(\h \cup \{e_i^{\pm}\}_{1 \leq i \leq n})/[\h, \h] \cong \h \oplus \frakN^{\pm}(A)$$
                    $$\frakG(A) := k\-\lie(\h \cup \{e_i^-, e_i^+\}_{1 \leq i \leq n})/[\h, \h] \cong \frakN^-(A) \oplus \h \oplus \frakN^+(A)$$
                from which we obtain isomorphisms:
                    $$\frakG(A) \cong \frakN^{\mp}(A) \oplus \frakB^{\pm}(A) \cong \frakB^{\mp}(A) \oplus \frakN^{\pm}$$
            \end{convention}
            \begin{convention}[Modules over universal enveloping algebras are representations]
                We take for granted the fact that given any Lie $k$-algebra $\g$, there is an exact, linear, and monoidal equivalence of monoidal abelian $k$-linear categories:
                    $$\Rep_k(\g) \cong {}^r\rmU(\g)\mod$$
                between the category of $k$-linear representations of $\g$ and that of right-modules over the universal enveloping algebra $\rmU(\g)$, both equipped with their usual monoidal structures. For more details, see subsection \ref{subsection: universal_enveloping_algebras_of_lie_algebras}.
            \end{convention}
            \begin{convention}[A fixed GCM] \label{conv: a_fixed_GCM_for_non_reduced_kac_moody_algebras}
                Fix a GCM $A \in \Mat_n(\Z)$ and assume convention \ref{conv: unipotent_radicals_of_non_reduced_kac_moody_algebras}. Since $A$ is fixed, let us write $\frakN^{\pm} := \frakN^{\pm}(A), \frakB^{\pm} := \frakB^{\pm}(A), \frakG := \frakG(A)$, and let us also denote $\bfg := \bfLie(A)$.
            \end{convention}
            \begin{lemma}[Cartan subalgebras are abelian] \label{lemma: cartan_subalgebras_of_non_reduced_kac_moody_algebras_are_abelian}
                Assume conventions \ref{conv: unipotent_radicals_of_non_reduced_kac_moody_algebras} and \ref{conv: a_fixed_GCM_for_non_reduced_kac_moody_algebras}. Then the Lie $k$-subalgebra of $\bfg$ generated by the set $\h$ is abelian (and hence $\rmU(\h)$ is a commutative $k$-algebra).
            \end{lemma}
                \begin{proof}
                    This is a straightforward consequence of the fact that $[h, h'] = 0$ for all $h, h' \in \h$ (cf. definition \ref{def: non_reduced_kac_moody_algebras}).
                \end{proof}
            \begin{convention} \label{conv: lower_and_upper_triangular_nilpotent_subalgebras_of_non_reduced_kac_moody_algebras}
                Assume conventions \ref{conv: unipotent_radicals_of_non_reduced_kac_moody_algebras} and \ref{conv: a_fixed_GCM_for_non_reduced_kac_moody_algebras}. Then, let us denote the Lie $k$-subalgebras of $\bfg$ generated by $\{e_i^{\pm}\}_{1 \leq i \leq n}$ by $\bfn^{\pm}$. Note that these subalgebras are \textit{not} ideals of $\bfg$, since there exist elements $x \in \bfg$ (e.g. $x = e_j^{\mp}$) such that $[e_i^{\pm}, x] \not \in \bfn^{\pm}$ (e.g. one has the Chevalley relations $[e_i^-, e_j^+] = \delta_{ij} \alpha_i^{\vee}$ for all $1 \leq i, j \leq n$).
            \end{convention}
            \begin{remark}[Non-reduced Kac-Moody algebras are non-zero] \label{remark: non_reduced_kac_modoy_algebras_are_non_zero}
                Assume conventions \ref{conv: unipotent_radicals_of_non_reduced_kac_moody_algebras} and \ref{conv: a_fixed_GCM_for_non_reduced_kac_moody_algebras}. The non-reduced Kac-Moody algebra $\bfg$ can then be constructed, following the procedure outlined in definition \ref{def: non_reduced_kac_moody_algebras}. However, before we may try to say anything substantial about such a Lie algebra, we must first of all verify that in general, it is non-zero. Using the fact that universal enveloping $k$-algebras $\rmU(\bfg)$ of Lie $k$-algebras $\bfg$ arise through a functor:
                    $$\rmU: k\-\Lie\Alg \to k\-\Assoc\Alg$$
                that is \textit{a priori} right-exact (cf. definition \ref{def: universal_enveloping_algebras}) and the fact that $0$ is the terminal object of $k\-\Comm\Alg$ (i.e. it is the limit of the empty diagram), one sees that $\bfg$ is non-zero if and only if $\rmU(\bfg)$ is non-zero. In order to achieve this, it suffices to produce a non-zero $k$-linear representation of $\rmU(\bfg)$: in particular, the representation that we shall be producing is, in some sense, precisely analogous to the Verma representation from the finite-dimensional theory. Before we attempt to construct such a representation, note that:
                    $$\rmU(\bfg) \cong k\<\h \cup \{e_i^-, e_i^+\}_{1 \leq i \leq n}\>/\frakChev(A)$$
                (cf. remark \ref{remark: canonical_grading_on_universal_enveloping_algebras_of_lie_algebras}), wherein $\frakChev(A)$ is the two-sided $k\<\h \cup \{e_i^-, e_i^+\}_{1 \leq i \leq n}\>$-ideal generated by the following relations:
                    $$\forall h, h' \in \h: h \tensor h' - h' \tensor h = 0$$
                    $$\forall h \in \h: \forall 1 \leq i \leq n: h \tensor e_i^{\pm} - e_i^{\pm} \tensor h = \mp B(\alpha_i, h) e_i^{\pm}$$
                    $$\forall 1 \leq i, j \leq n: e_i^- \tensor e_j^+ - e_i^+ \tensor e_i^- = \delta_{ij} \alpha_i^{\vee}$$
                
                In order to construct a (non-zero) right-$\rmU(\bfg)$-module, let us start by fixing a positive/negative abstract root:
                    $$\lambda \in \bbX^{\pm}(\bfg, \h)$$
                from which a (non-zero) right-$\rmU(\bfg)$-module can eventually be induced through extension of scalars. Said abstract root immediately gives rise to a $k$-linear $\frakB^{\mp}$-character:
                    $$\lambda^{\mp}: \frakB^{\mp} \to k$$
                via the canonical composition of Lie $k$-algebra homomorphisms $\frakB^{\mp} \to \h \xrightarrow[]{\lambda} k$. Such a $\frakB^{\mp}$-character gives rise to a right-$\rmU(\frakB^{\mp})$-module structure on $k$ via a $k$-linear map:
                    $$\lambda^{\mp}: k \tensor_k \rmU(\frakB^{\mp}) \to k$$
                given by:
                    $$\forall h \in \h: \lambda^{\mp}(1 \tensor h := B(\lambda, h)$$
                    $$\forall N \geq 0: \forall 1 \leq i_1, ..., i_N \leq n: \lambda^{\mp}(1 \tensor e_i^{\pm}) := 0$$
                (recall that $\frakB^{\mp} := \h \oplus \frakN^{\pm}$ so it is enough to define $\lambda^{\mp}$ on the generators $h \in \h$ and $e_i^{\pm} \in \frakN^{\pm}$; cf. convention \ref{conv: unipotent_radicals_of_non_reduced_kac_moody_algebras}). For specificity, this right-$\rmU(\frakB^{\mp})$-module shall henceforth be denoted by $k_{\lambda}$.
                
                Now, consider the right-$\rmU(\frakG)$-module\footnote{Note the similarity between this construction and the Verma module construction form the finite-dimensional theory.}:
                    $$\frakV_{\lambda}^{\pm} := k_{\lambda} \tensor_{\rmU(\frakB^{\mp})} \rmU(\frakG)$$
                (well-defined since $k_{\lambda}$ is a right-$\rmU(\frakB^{\mp})$-module and since $\rmU(\frakG) \cong \rmU(\frakB^{\mp}) \tensor_k \rmU(\frakN^{\pm})$ is naturally a left-$\rmU(\frakB^{\mp})$-module).
                
                As a preliminary observation, note that via the tensor-hom adjunction:
                    $$\Hom_{\rmU(\frakG)}(\rmU(\frakG), -): {}^r\rmU(\frakG)\mod \leftrightarrows {}^r\rmU(\frakB^{\mp})\mod : - \tensor_{\rmU(\frakB^{\mp})} \rmU(\frakG)$$
                one obtains a universal characterisation of $\frakV_{\lambda}^{\pm}$ through a natural isomorphism as follows:
                    $$\Hom_{\rmU(\frakG)}(\frakV_{\lambda}^{\pm}, -) \cong \Hom_{\rmU(\frakB^{\mp})}(k_{\lambda}, -)$$
                Using this universal characterisation of $\frakV_{\lambda}^{\pm}$, it can be inferred that the right-$\rmU(\frakG)$-module structure on $\frakV_{\lambda}^{\pm}$ can be \textit{explicitly} deduced from the right-$\rmU(\frakB^{\mp})$-module structure on $k_{\lambda}$, which is already known. To this end, notice firstly that there are isomorphisms of $k$-vector spaces as follows:
                    $$\frakV_{\lambda}^{\pm} \cong k_{\lambda} \tensor_{\rmU(\frakB^{\mp})} ( \rmU(\frakB^{\mp}) \tensor_k \rmU(\frakN^{\pm}) ) \cong k_{\lambda} \tensor_k \rmU(\frakN^{\pm})$$
                and as such, there is a natural isomorphism:
                    $$\Hom_{\rmU(\frakG)}(k_{\lambda} \tensor_k \rmU(\frakN^{\pm}), -) \cong \Hom_{\rmU(\frakB^{\mp})}(k_{\lambda}, -)$$
                Via the tensor-hom adjunction:
                    $$\Hom_{\rmU(\frakG)}(\rmU(\frakN^{\pm}), -): {}^r\rmU(\frakG)\mod \leftrightarrows k\-\Vect : - \tensor_k\rmU(\frakN^{\pm})$$
                one obtains the following natural isomorphisms:
                    $$\Hom_{\rmU(\frakG)}(\rmU(\frakN^{\pm}), -) \cong \Hom_k(k_{\lambda}, \Hom_{\rmU(\frakG)}(\rmU(\frakN^{\pm}), -)) \cong \Hom_{\rmU(\frakG)}(k_{\lambda} \tensor_k \rmU(\frakN^{\pm}), -)$$
                By putting everything together, one obtains a natural isomorphism:
                    $$\Hom_{\rmU(\frakG)}(\rmU(\frakN^{\pm}), -) \cong \Hom_{\rmU(\frakB^{\mp})}(k_{\lambda}, -)$$
                
                Using the fact that there is an isomorphism of $k$-vector spaces:
                    $$\frakV_{\lambda}^{\pm} \cong k_{\lambda} \tensor_k \rmU(\frakN^{\pm})$$
                one infers also that $\frakV_{\lambda}^{\pm}$ as constructed above is a right-$\rmU(\frakG)$-module whose underlying $k$-vector space is isomorphic to that of $\rmU(\frakN^{\pm})$, which itself is isomorphic to $\bigoplus_{N \geq 0} (E^{\pm})^{\tensor N}$ (with $E^{\pm} := \span_k \{e_i^{\pm}\}_{1 \leq i \leq n}$), since $\frakN^{\pm}$ is the free Lie $k$-algebra on the set $\{e_i^{\pm}\}_{1 \leq i \leq n}$ per convention \ref{conv: unipotent_radicals_of_non_reduced_kac_moody_algebras}. 
                
                Now, because $\rmU(\frakG) \cong k\<\h \cup \{e_i^-, e_i^+\}_{1 \leq i \leq n}\>/[\h, \h]$ and because $\frakChev(A) \supset [\h, \h]$, one deduces via the Third Isomorphism Theorem that there is a canonical short exact sequence in ${}^r\rmU(\frakG)\mod$ as follows:
                    $$0 \to \frakChev(A)\to \rmU(\frakG) \to \rmU(\bfg) \to 0$$
                Consequently, the right-$\rmU(\frakG)$-module structure on $\frakV_{\lambda}^{\pm}$ induces the existence of the existence of the following right-$\rmU(\bfg)$-module:
                    $$\bfV_{\lambda}^{\pm} := \frakV_{\lambda}^{\pm} \tensor_{\rmU(\frakG)} \rmU(\bfg) \cong k_{\lambda}  \tensor_{\rmU(\frakB^{\mp})} \rmU(\bfg) \cong k_{\lambda} \tensor_k \rmU(\bfn^{\pm})$$
                which just like $\frakV_{\lambda}^{\pm}$, has an underlying $k$-vector space that is isomorphic to $\bigoplus_{N \geq 0} (E^{\pm})^{\tensor N}$ and also, induces a natural isomorphism:
                    $$\Hom_{\rmU(\bfg)}(\rmU(\bfn^{\pm}), -) \cong \Hom_{\rmU(\frakB^{\mp})}(k_{\lambda}, -)$$
                with $\bfn^{\pm} \subset \bfg$ being the subalgebras generated by the sets $\{e_i^{\pm}\}_{1 \leq i \leq n}$ (cf. convention \ref{conv: lower_and_upper_triangular_nilpotent_subalgebras_of_non_reduced_kac_moody_algebras}). Since $k_{\lambda}$ is non-zero as a right-$\rmU(\frakB^{\mp})$-module, one thus sees that $\bfV_{\lambda}^{\pm}$ is non-zero as a right-$\rmU(\bfg)$-module. 
            \end{remark}
            \begin{definition}[Verma modules for non-reduced Kac-Moody algebras] \label{def: verma_modules_for_non_reduced_kac_moody_algebras}
                Assume conventions \ref{conv: unipotent_radicals_of_non_reduced_kac_moody_algebras} and \ref{conv: a_fixed_GCM_for_non_reduced_kac_moody_algebras}. The right-$\rmU(\bfg)$-module:
                    $$\bfV_{\lambda}^{\pm} := k_{\lambda} \tensor_{\rmU(\frakB^{\mp})} \rmU(\bfg)$$
                as constructed in remark \ref{remark: non_reduced_kac_modoy_algebras_are_non_zero} is usually referred to as the \textbf{Verma module} of weight $\lambda \in \bbX^{\pm}(\bfg, \h)$ of the non-reduced Kac-Moody algebra $\bfg$.
            \end{definition}
            \begin{remark}[Cartan action on Verma modules] \label{remark: cartan_action_on_verma_modules_of_non_reduced_kac_moody_algebras}
                Assume conventions \ref{conv: unipotent_radicals_of_non_reduced_kac_moody_algebras} and \ref{conv: a_fixed_GCM_for_non_reduced_kac_moody_algebras} and fix an abstract root $\lambda: \h \to k$. There is a natural $k$-linear $\h$-action on the (underlying $k$-vector space of) Verma module $\bfV_{\lambda}^{\pm} \in \Ob({}^r\rmU(\bfg)\mod)$ via the canonical composition:
                    $$\h \hookrightarrow \bfg \xrightarrow[]{\bfv_{\lambda}^{\pm}} \gl(\bfV_{\lambda}^{\pm})$$
                The underlying $k$-vector space of $\bfV_{\lambda}^{\pm}$ as well as that of $\frakN^{\pm}$ are both isomorphic to $\bigoplus_{N \geq 0} (E^{\pm})^{\tensor N}$ (with $E^{\mp} := \span_k \{e_i^{\mp}\}_{1 \leq i \leq n}$) so the aforementioned $k$-linear $\h$-action on $\bfv_{\lambda}^{\pm}$ induces a $k$-linear $\h$-action on the direct sum of $k$-vector spaces:
                    $$\frakN^- \oplus \h \oplus \frakN^+ \cong \left( \bigoplus_{N \geq 0} (E^-)^{\tensor N} \right) \oplus \h \oplus \left( \bigoplus_{N \geq 0} (E^+)^{\tensor N} \right)$$
            \end{remark}
            The following lemma helps us justify spending so much effort constructing the Verma modules of $\bfg$. It is clear from the discussion in remark \ref{remark: non_reduced_kac_modoy_algebras_are_non_zero}.
            \begin{lemma}[Lower/upper triangular nilpotent subagebras of non-reduced Kac-Moody algebras are free] \label{lemma: lower_and_upper_triangular_nilpotent_subalgebras_of_non_reduced_kac_moody_algebras_are_free}
                Assume conventions \ref{conv: unipotent_radicals_of_non_reduced_kac_moody_algebras}, \ref{conv: a_fixed_GCM_for_non_reduced_kac_moody_algebras}, and \ref{conv: lower_and_upper_triangular_nilpotent_subalgebras_of_non_reduced_kac_moody_algebras}. Then, for any fixed abstract root $\lambda \in \bbX^{\pm}(\bfg, \h)$, one has an isomorphism in ${}^r\rmU(\bfg)\mod$ as follows:
                    $$\bfn^{\pm} \to \bfV_{\lambda}^{\pm}$$
                The Lie $k$-subalgebras $\bfn^{\pm} \subset \bfg$ are thus free on their sets of Chevalley generators $\{e_i^{\pm}\}_{1 \leq i \leq n}$
            \end{lemma}
            \begin{proposition}[Triangular decomposition of non-reduced Kac-Moody algebras] \label{prop: triangular_decomposition_of_kac_moody_algebras}
                \cite[Theorem 1.2(a)]{kac_infinite_dimensional_lie_algebras} Assume conventions \ref{conv: unipotent_radicals_of_non_reduced_kac_moody_algebras}, \ref{conv: a_fixed_GCM_for_non_reduced_kac_moody_algebras}, and \ref{conv: lower_and_upper_triangular_nilpotent_subalgebras_of_non_reduced_kac_moody_algebras}. Then one has the following decomposition in $\rmU(\h)\mod$ of the Lie $k$-algebra $\bfg$ into Lie $k$-subalgebras, commonly referred to as the \textbf{triangular decomposition}:
                    $$\bfg \cong \frakN^- \oplus \h \oplus \frakN^+$$
            \end{proposition}
            \begin{corollary}[Cartan involutions on non-reduced Kac-Moody algebras] \label{coro: cartan_involutions_on_non_reduced_kac_moody_algebras}
                \cite[Theorem 1.2(c)]{kac_infinite_dimensional_lie_algebras} Assume conventions \ref{conv: unipotent_radicals_of_non_reduced_kac_moody_algebras} and \ref{conv: a_fixed_GCM_for_non_reduced_kac_moody_algebras}. There is an involutive automorphism $\omega \in \Aut(\h \cup \{e_i^-, e_i^+\}_{1 \leq i \leq n})$ given by:
                    $$\forall h \in \h: \omega(h) := -h$$
                    $$1 \leq i \leq n: \omega(e_i^{\pm}) = -e_i^{\mp}$$
                that extends to an involutive Lie $k$-algebra automorphism which we shall abusively denote by $\omega \in \Aut(\g)$. 
            \end{corollary}
            \begin{theorem}[Root space decomposition of non-reduced Kac-Moody algebras] \label{theorem: root_space_decomposition_of_non_reduced_kac_moody_algebras}
                \cite[Theorem 1.2(d)]{kac_infinite_dimensional_lie_algebras} Assume conventions \ref{conv: unipotent_radicals_of_non_reduced_kac_moody_algebras} and \ref{conv: a_fixed_GCM_for_non_reduced_kac_moody_algebras}. Then, one has the following so-called \textbf{root space decompositions} in $\rmU(\h)\mod$:
                    $$\h \cong \bfg_0, \frakN^{\pm}(A) \cong \bigoplus_{\lambda \in \bbX^{\pm}(\bfg, \h) \setminus \{0\}} \bfg_{\lambda}$$
                and in particular, the root spaces $\bfg_{\lambda}$ are Lie $k$-subalgebras of $\frakN^{\pm}(A)$ for all $\lambda \in \bbX^{\pm}(\bfg, \h) \setminus \{0\}$.
            \end{theorem}
                \begin{proof}
                    Recall from definition \ref{def: non_reduced_kac_moody_algebras} that one has the following Chevalley relations defining $\bfg$:
                        $$\forall h, h' \in \h: [h, h'] = 0$$
                        $$\forall h \in \h: \forall 1 \leq i \leq n: [h, e_i^{\pm}] = \mp B(\alpha_i, h) e_i^{\pm}$$
                    From these relations adn from the fact that one has a direct sum decomposition of $\bfg \in \Ob(\rmU(\h)\mod)$, one can immediately conclude that:
                        $$\h \cong \bfg_0, \frakN^{\pm}(A) \cong \bigoplus_{\lambda \in \bbX^{\pm}(\bfg, \h) \setminus \{0\}} \bfg_{\lambda}$$
                    That that the abstract root spaces $\bfg_{\lambda}$ are Lie $k$-subalgebras of $\frakN^{\pm}(A)$ for all $\lambda \in \bbX^{\pm}(\bfg, \h) \setminus \{0\}$ is then a trivial consequence of the definition \ref{def: weight_lattices_and_abstract_root_spaces}
                \end{proof}
            \begin{remark}
                Assume conventions \ref{conv: unipotent_radicals_of_non_reduced_kac_moody_algebras} and \ref{conv: a_fixed_GCM_for_non_reduced_kac_moody_algebras}. Through a combination of proposition \ref{prop: triangular_decomposition_of_kac_moody_algebras} and theorem \ref{theorem: root_space_decomposition_of_non_reduced_kac_moody_algebras}, one sees that:
                    $$\bfg \cong \left( \bigoplus_{\lambda \in \bbX^-(\bfg, \h) \setminus \{0\}} \bfg_{\lambda} \right) \oplus \h \oplus \left( \bigoplus_{\lambda \in \bbX^+(\bfg, \h) \setminus \{0\}} \bfg_{\lambda} \right) \cong \bigoplus_{\lambda \in \bbX(\bfg, \h)} \bfg_{\lambda}$$
                Since the non-concrete root spaces are identically zero, the decomposition above can be further refined down to:
                    $$\bfg \cong \bigoplus_{\lambda \in \Phi(\g, \h) \cup \{0\}} \bfg_{\lambda}$$
            \end{remark}
            \begin{definition}[Cartan subalgebras of non-reduced Kac-Moody algebras] \label{def: cartan_subalgebras_of_non_reduced_kac_moody_algebras}
                Assume conventions \ref{conv: unipotent_radicals_of_non_reduced_kac_moody_algebras} and \ref{conv: a_fixed_GCM_for_non_reduced_kac_moody_algebras}, and pick a root space decomposition $\bfg \cong \bigoplus_{\lambda \in \bbX(\bfg, \h)} \bfg_{\lambda}$. Then, any Lie subalgebra of $\bfg$ that is isomorphic to the root space $\g_0$ shall be called a \textbf{Cartan subalgebra}.
            \end{definition}
                
        \subsubsection{Reduced Kac-Moody algebras}
            \begin{lemma}[Non-reduced Kac-Moody algebras have non-trivial radicals] \label{lemma: non_reduced_kac_moody_algebras_have_non_trivial_radicals}
                \cite[Theorem 1.2(e)]{kac_infinite_dimensional_lie_algebras} Assume conventions \ref{conv: unipotent_radicals_of_non_reduced_kac_moody_algebras} and \ref{conv: a_fixed_GCM_for_non_reduced_kac_moody_algebras}, and pick a root space decomposition:
                    $$\bfg \cong \bigoplus_{\lambda \in \bbX(\bfg, \h)} \bfg_{\lambda}$$
                Then, the set of ideals $\fraku \leq \g$ such that $\fraku \cap \bfg_0 = \{0\}$ admits a \textit{unique} maximal element $\rad(\bfg)$ (which shall be called the \textbf{radical} of $\bfg$); furthermore, one has:
                    $$\rad(\bfg) = \rad^+(\bfg) \oplus \rad^-(\bfg)$$
                wherein $\rad^{\pm}(\bfg) := \rad(\bfg) \cap \frakN^-$.
            \end{lemma}
                \begin{proof}
                    Suppose that $\fraku \leq \bfg$ is an arbitrary ideal. By noting that said $\bfg$-ideal as a $\rmU(\h)$-submodule of $\bf$, one sees that there is a root space decomposition as follows $\fraku \cong (\fraku \cap \bfg_0) \oplus \bigoplus_{\lambda \in \Phi(\bfg, \h)} (\fraku \cap \bfg_{\lambda})$. If $\fraku \cap \bfg_{\lambda} = \{0\}$ then the above reduces down to  $\fraku \cong \bigoplus_{\lambda \in \Phi(\bfg, \h)} (\fraku \cap \bfg_{\lambda})$. One can thus construct the desired ideal to be $\rad(\bfg) := \sum_{\fraku \leq \bfg, \fraku \cap \bfg_0 = \{0\}} \fraku$; the second statement then becomes trivial.
                \end{proof}
            \begin{definition}[Reduced Kac-Moody algebras] \label{def: reduced_kac_moody_algebras}
                Let $A \in \Mat_n(\Z)$ be a GCM. The corresponding \textbf{reduced Kac-Moody algebra} (or simply \say{Kac-Moody algebra}) is then given by $\frakLie(A) := \bfLie(A)/\rad(\bfLie(A))$.
            \end{definition}
            \begin{remark}[Reduced Kac-Moody algebras are uniquely determined by GCMs] \label{remark: reduced_kac_moody_algebras_are_uniquely_determined_by_generalised_cartan_matrices}
                As a consequence of the uniqueness of radicals of non-reduced Kac-Moody algebras, the assignment of reduced Kac-Moody algebras to GCMs defines a bijection:
                    $$
                        \begin{tikzcd}
                        	{\left\{ \text{GCMs over $k$} \right\}} & {\left\{ \text{Reduced Kac-Moody algebras over $k$} \right\}}
                        	\arrow["{\frakLie(-)}", from=1-1, to=1-2]
                        \end{tikzcd}
                    $$
            \end{remark}
            \begin{convention} \label{conv: a_fixed_generalised_cartan_matrix}
                Until further notice, fix a GCM $A \in \Mat_n(\Z)$ of rank $l$, along with a(n) (($n - l$)-dimensional) minimal root datum $(\h, \Pi, \h^{\vee}, \Pi^{\vee}, B)$ thereof, wherein $\Pi := \{\alpha_i\}_{1 \leq i \leq n} \subset \h^{\vee}$ and $\Pi^{\vee} := \{\alpha_i^{\vee}\}_{1 \leq i \leq n} \subset \h$. Also, write $\bfg := \bfLie(A)$ and $\g := \frakLie(A)$, and denote the set of Chevalley generators thereof by $\h \cup \{e_i^-, e_i^+\}_{1 \leq i \leq n}$. 
            \end{convention}
            \begin{remark}[Root space and triangular decompositions of reduced Kac-Moody algebras] \label{remark: root_space_and_triangular_decompositions_of_reduced_kac_moody_algebras}
                Assume conventions \ref{conv: unipotent_radicals_of_non_reduced_kac_moody_algebras} and \ref{conv: a_fixed_generalised_cartan_matrix}, and pick a root space decomposition:
                    $$\bfg \cong \bigoplus_{\lambda \in \bbX(\bfg, \h)} \bfg_{\lambda}$$
                Then, as a consequence of the definition of $\rad(\bfg)$ (cf. lemma \ref{lemma: non_reduced_kac_moody_algebras_have_non_trivial_radicals}), one sees that the reduced Kac-Moody algebra $\g$ also has a root space decomposition, which is as follows:
                    $$\g \cong \left( \bigoplus_{\lambda \in \Phi^-(\g, \h)} \frac{ \bfg_{\lambda} }{ \bfg_{\lambda} \cap \rad^-(\bfg) } \right) \oplus \g_0 \oplus \left( \bigoplus_{\lambda \in \Phi^+(\g, \h)} \frac{ \bfg_{\lambda} }{ \bfg_{\lambda} \cap \rad^+(\bfg) } \right)$$
                i.e. one has $\g_{\lambda} := \bfg_{\lambda}/(\bfg_{\lambda} \cap \rad^-(\bfg))$ for all $\lambda \in \Phi^{\pm}(\bfg, \h)$. From this, one can easily deduce the following triangular decomposition:
                    $$\g \cong \n^- \oplus \h \oplus \n^+$$
                wherein $\n^{\pm} := \frakN^{\pm}/\rad^{\pm}(\bfg)$. Note that in both decompositions, the Cartan subalgebra $\g_0 \cong \h$ is left unchanged; this will become relevant later when we try to define a \say{generalised Killing form} on $\g$ (cf. definition \ref{def: generalised_killing_forms}).
            \end{remark}
            \begin{convention}[Tame and wild reduced Kac-Moody algebras] \label{conv: tame_and_wild_reduced_kac_moody_algebras}
                The reduced Kac-Moody algebra $\frakLie(A)$ associated a symmetric/symmetrisable GCM $A$ is said to be \textbf{symmetric/symmetrisable}.
                
                The symmetrisable reduced Kac-Moody algebra $\frakLie(A)$ is said to be of finite type (respectively, of affine type, tame, and wild) if and only if $A$ is of finite type (respectively, of affine type, tame, and wild) in the sense of definition \ref{def: tame_and_wild_generalised_cartan_matrices}.
            \end{convention}
            
            \begin{remark}[Gradings on (non-)reduced Kac-Moody algebras] \label{remark: height_grading_on_kac_moody_algebras}
                Assume conventions \ref{conv: unipotent_radicals_of_non_reduced_kac_moody_algebras} and \ref{conv: a_fixed_generalised_cartan_matrix}. It is then easy to see that there is a $\Z$-grading on $\g := \frakLie(A)$ coming from the partial ordering of (abstract) roots by heights (cf. proposition \ref{prop: roots_are_partially_ordered_by_heights}): specifically, one sees immediately from proposition \ref{prop: roots_are_partially_ordered_by_heights} that there is a well-define $\Z$-grading of $\g$ as follows:
                    $$\left\{ \g^{(N)} := \bigoplus_{\lambda \in \Phi(\g, \h), \height(\lambda) = N} \g_{\lambda} \right\}_{N \in \Z}$$
                since indeed, one has $[x_1, x_2] \in \g^{(N_1 + N_2)}$ for all $x_1 \in \g^{(N_1)}, x_2 \in \g^{(N_2)}$ through remark \ref{remark: root_spaces_and_commutators}. The construction of this $\Z$-grading results in $\n^{\pm}(A) \cong \bigoplus_{N \geq 1} \g^{(\pm N)}$; in particular, one has $\g^{(\pm 1)} \cong \span_k \{e_i^{\pm}\}_{1 \leq i \leq n}$, so with respect to this $\Z$-grading, one has $\deg e_i^{\pm} = \pm 1$, which is an idea that will be used in the proof of theorem \ref{theorem: untwisted_affinisation}.
            \end{remark}
            \begin{lemma}[Commutator of the (non-)reduced negative and positive unipotent radicals] \label{lemma: commutator_of_the_negative_and_positive_unipotent_radicals}
                \cite[Lemma 1.5]{kac_infinite_dimensional_lie_algebras} Assume conventions \ref{conv: unipotent_radicals_of_non_reduced_kac_moody_algebras} and \ref{conv: a_fixed_generalised_cartan_matrix}. If $x \in \frakN^{\mp}(A)$ (respectively, $x \in \n^{\pm}(A)$) is such that $\ad(e_i^{\pm})(x) \in \rad(\bfg)$ (respectively, $\ad(e_i^{\pm})(x) = 0$) for all $1 \leq i \leq n$ then $x \in \rad(\bfg)$ (respectively, then $x = 0$). 
            \end{lemma}
                \begin{proof}
                    Let us endow $\bfg$ with the $\Z$-grading by heights as in remark \ref{remark: height_grading_on_kac_moody_algebras}; in particular, recall that $\frakN^{\mp}(A) \cong \bigoplus_{N \geq 1} \bfg^{(\mp N)}$. Observe that because $[e_i^-, e_j^+] = \delta_{ij} \alpha_j^{\vee} \in \h$ (cf. definition \ref{def: non_reduced_kac_moody_algebras}), one thus has that $\ad(e_i^{\pm})(\frakN^{\mp}(A))$ is a $\bfg$-ideal such that $\ad(e_i^{\pm})(\frakN^{\mp}(A)) \cap \h = \{0\}$. $\frakN^{\mp}(A)$ is thus not annihilated by any $\ad(e_i^{\pm})$ and hence $x \in \rad(\bfg)$ necessarily. The case wherein $x \in \n^{\pm}(A)$ then follows from the construction $\g := \bfg/\rad(\bfg)$ (cf. definition \ref{def: reduced_kac_moody_algebras}).
                \end{proof}
            \begin{corollary}[Derived subalgebras of (non-)reduced Kac-Moody algebras] \label{coro: derived_subalgebras_of_kac_moody_algebras}
                Assume conventions \ref{conv: unipotent_radicals_of_non_reduced_kac_moody_algebras} and \ref{conv: a_fixed_generalised_cartan_matrix}. Then the derived subalgebra $[\bfg, \bfg] \subseteq \bfg$ shall be the Lie $k$-algebra with generating set $\{\alpha_i^{\vee}, e_i^-, e_i^+\}_{1 \leq i \leq n}$ and defining relations:
                    $$\forall 1 \leq i, j \leq n: [\alpha_i^{\vee}, \alpha_j^{\vee}] = 0$$
                    $$\forall 1 \leq i \leq n: [\alpha_i^{\vee}, e_i^{\pm}] = \mp B(\alpha_i, \alpha_j^{\vee}) e_i^{\pm} = \mp a_{ij} e_i^{\pm}$$
                    $$\forall 1 \leq i, j \leq n: [e_i^-, e_j^+] = \delta_{ij} \alpha_i^{\vee}$$
                Furthermore:
                    $$[\g, \g] \cong [\bfg, \bfg]/( \rad(\bfg) \cap [\bfg, \bfg] )$$
            \end{corollary}
            \begin{proposition}[Centres of reduced Kac-Moody algebras and of their derived subalgebras] \label{prop: centres_of_reduced_kac_moody_algebras}
                \cite[Proposition 1.6]{kac_infinite_dimensional_lie_algebras} Assume conventions \ref{conv: unipotent_radicals_of_non_reduced_kac_moody_algebras} and \ref{conv: a_fixed_generalised_cartan_matrix}. Then:
                    $$\z(\g) = \z(\g, \g) = \{h \in \h \mid \forall 1 \leq i \leq n: B(\alpha_i, h) = 0\}$$
            \end{proposition}
                \begin{proof}
                    Endow $\g$ with the height grading (cf. remark \ref{remark: height_grading_on_kac_moody_algebras}) and suppose that $z \in \g$ (respectively, $z \in [\g, \g]$). Using lemma \ref{lemma: commutator_of_the_negative_and_positive_unipotent_radicals}, one sees that $z \in \h$ necessarily, and since $z$ is central, we will then be done by applying corollary \ref{coro: derived_subalgebras_of_kac_moody_algebras}.
                \end{proof}
    
    \subsection{Root systems of tame Kac-Moody algebras}
        \subsubsection{Generalised Killing forms and associated Dynkin quivers of tame Kac-Moody algebras}
            Let us now attempt to construct an analogue of the Killing form on symmetrisable Kac-Moody algebras. Specifically, we shall be making use of such a bilinear form in order to obtain Dynkin quivers from tame Kac-Moody algebras, thereby transporting the study of root systems of tame Kac-Moody algebras to those of tame Dynkin quivers, which is a much more manageable task (cf. appendix \ref{section: tame_dynkin_quivers}).
        
            \begin{convention}[The basic setup] \label{conv: generalised_killing_forms}
                Throughout this subsubsection, fix a symmetrisable GCM $A := (a_{ij})_{1 \leq i, j \leq n} \in \Mat_n(\Z)$ along with a symmetrisation $A := DS$ thereof, wherein $D := (d_{ij})_{1 \leq i, j \leq n} \in \GL_1^n(k)$ is some invertible diagonal $n \x n$ matrix and $S := (s_{ij})_{1 \leq i, j \leq n} \in \Sym_n(k)$ is some symmetric $n \x n$ matrix. 
                
                We shall also be needing a minimal $k$-linear root datum $(\h, \Pi, \h^{\vee}, \Pi^{\vee}, B)$ of $A$, wherein $\Pi := \{\alpha_i\}_{1 \leq i \leq n} \subseteq \h^{\vee}$ and $\Pi^{\vee} := \{\alpha_i^{\vee}\}_{1 \leq i \leq n} \subseteq \h$. The $k$-vector space $\h$ shall also be decomposed as:
                    $$\h := \h' \oplus \h^c$$
                wherein $\h' := \span_k \{\alpha_i^{\vee}\}_{1 \leq i \leq n}$ and $\h^c$ is the complementary subspace. 
            \end{convention}
            
            \begin{definition}[Generalised Killing forms] \label{def: generalised_killing_forms}
                Assume conventions \ref{conv: unipotent_radicals_of_non_reduced_kac_moody_algebras}, \ref{conv: generalised_killing_forms}, along with convention \ref{conv: a_fixed_generalised_cartan_matrix}. Define a $k$-bilinear form $\hat{\kappa}_0: \h \x \h \to k$, which we shall refer to as the \textbf{generalised Killing form} on $\h$, as follows:
                    $$\forall h \in \h: \forall 1 \leq i \leq n: \hat{\kappa}_0(h, \alpha_i^{\vee}) := d_{ii} B(\alpha_i, h)$$
                    $$\forall x, y \in \h^c: \hat{\kappa}_0(x, y) := 0$$
                The generalised Killing form $\hat{\kappa}_0: \h \x \h \to k$ on the Cartan subalgebra $\g_0 \cong \h$ then extends to a $k$-bilinear form $\hat{\kappa}: \g \x \g \to k$ (also called the \textbf{generalised Killing form}) such that $\hat{\kappa}|_{\h \x \h} = \hat{\kappa}_0$ (cf. definition \ref{def: generalised_cartan_matrices}), and:
                    $$\forall 1 \leq i, j \leq n: \hat{\kappa}(e_i^{\pm}, e_j^{\mp}) = d_{ii} \delta_{ij}$$
            \end{definition}
            \begin{remark}[Genralised Killing forms are symmetric] \label{remark: generalised_killing_forms_are_symmetric}
                One readily observes that:
                    $$\forall 1 \leq i, j \leq n: \hat{\kappa}_0(\alpha_i^{\vee}, \alpha_j^{\vee}) = d_{ii} B(\alpha_i^{\vee}, \alpha_i) = d_{ii} a_{ij} = d_{jj} s_{ij}$$
                It is then clear that $\hat{\kappa}_0: \h \x \h \to k$ is symmetric; the induced bilinear form $\hat{\kappa}: \frakLie(A) \x \frakLie(A) \to k$ is therefore also symmetric by construction.
            \end{remark}
            \begin{proposition}[Generalised Killing forms are non-degnerate on Cartan subalgebras] \label{prop: generalised_killing_forms_are_non_degenerate_on_cartan_subalgebras}
                Assume conventions \ref{conv: unipotent_radicals_of_non_reduced_kac_moody_algebras}, \ref{conv: generalised_killing_forms}, along with convention \ref{conv: a_fixed_generalised_cartan_matrix}. The restriction $\hat{\kappa}|_{\h \x \h} = \hat{\kappa}_0$ as constructed in definition \ref{def: generalised_killing_forms} is then non-degenerate.
            \end{proposition}
                \begin{proof}
                    Since $\hat{\kappa}_0(x, y) = 0$ for all $x, y \in \h^c$ by construction, it suffices to only show that $\hat{\kappa}(-, \alpha_i^{\vee}): \h \to k$ has trivial kernel for all $1 \leq i \leq n$. To that end, fix some $h \in \h$ and observe firstly that since $D := (d_{ij})_{1 \leq i, j \leq n} \in \GL_1^n(k)$ is an \textit{invertible} diagonal matrix and therefore $d_{ii} \not = 0$ for all $1 \leq i \leq n$ which implies that secondly, $\hat{\kappa}(h, \alpha_i^{\vee}) := d_{ii} B(\alpha_i, h) = 0$ if and only if $B(\alpha_i, h) = 0$, but since $B(-, -)$ is non-degenerate and because the set $\{\alpha_i\}_{1 \leq i \leq n}$ is $k$-linearly independent (meaning that $\alpha_i \not = 0$ for all $1 \leq i \leq n$) by assumption (cf. definition \ref{def: root_data_of_square_matrices}), this is only the case if $h = 0$. From this, one infers that $\hat{\kappa}_0$ is indeed non-degenerate.
                \end{proof}
            \begin{proposition}[Generalised Killing forms are invariant] \label{prop: generalised_killing_forms_are_invariant}
                \cite[Theorem 2.2]{kac_infinite_dimensional_lie_algebras} Assume conventions \ref{conv: unipotent_radicals_of_non_reduced_kac_moody_algebras}, \ref{conv: generalised_killing_forms}, along with convention \ref{conv: a_fixed_generalised_cartan_matrix}.
                    \begin{enumerate}
                        \item The generalised Killing form $\hat{\kappa}: \g \x \g \to k$ is $\g$-invariant, i.e. $\hat{\kappa}([-, -], -) = \hat{\kappa}(-, [-, -])$ (cf. definition \ref{def: invariant_bilinear_forms}).
                        \item One has that for all $\lambda, \mu \in \Phi(\g, \h)$ that $\hat{\kappa}$ that $\hat{\kappa}(\g_{\lambda}, \g_{\mu}) = 0$ if $\lambda + \mu \not = 0$ and that the pairing $\hat{\kappa}|_{\g_{\lambda} \x \g_{\mu}}$ is non-degenerate if and only if $\lambda + \mu = 0$.
                    \end{enumerate}
            \end{proposition}
            \begin{proposition}[Uniqueness of generalised Killing forms] \label{prop: uniqueness_of_generalised_killing_forms}
                \cite[Proposition 16.6]{carter_affine_lie_algebras} Assume conventions \ref{conv: unipotent_radicals_of_non_reduced_kac_moody_algebras}, \ref{conv: generalised_killing_forms}, along with convention \ref{conv: a_fixed_generalised_cartan_matrix}, and suppose in addition that the given GCM $A \in \Mat_n(\Z)$ is tame (cf. definition \ref{def: tame_and_wild_generalised_cartan_matrices}). Then every $\g$-invariant symmetric $k$-bilinear form $\psi: \g \x \g \to k$ such that:
                    \begin{itemize}
                        \item The restriction $\psi|_{\h \x \h}$ is non-degenerate.
                        \item The generalised Killing form $\psi: \g \x \g \to k$ is $\g$-invariant, i.e. $\psi([-, -], -) = \psi(-, [-, -])$ (cf. definition \ref{def: invariant_bilinear_forms}).
                        \item One has that for all $\lambda, \mu \in \Phi(\g, \h)$ that $\psi$ that $\psi(\g_{\lambda}, \g_{\mu}) = 0$ if $\lambda + \mu \not = 0$ and that the pairing $\psi|_{\g_{\lambda} \x \g_{\mu}}$ is non-degenerate if and only if $\lambda + \mu = 0$.
                    \end{itemize}
                is of the form $\psi = \xi \hat{\kappa}$ for some unique $\xi \in k^{\x}$.
            \end{proposition}
                
            \begin{convention} \label{conv: kac_moody_root_systems}
                From now on, all Kac-Moody algebras shall be reduced and tame unless specifically stated to be otherwise. Also, assume conventions \ref{conv: unipotent_radicals_of_non_reduced_kac_moody_algebras}, \ref{conv: generalised_killing_forms}, along with convention \ref{conv: a_fixed_generalised_cartan_matrix}. 
            \end{convention}
            
             Now that we have managed to construct well-defined and reasonably well-behaved generalised Killing forms on Kac-Moody algebras, let us explore the roles that such bilinear forms play in the construction of root systems of Kac-Moody algebras. Particularly, a generalised Killing forms is a mean by which one can construct an undirected Dynkin graph associated to a given Kac-Moody algebra, from which information about the root system of said Kac-Moody algebra can be inferred, particularly in the tame case (cf. section \ref{section: tame_dynkin_quivers}).
            \begin{theorem}[Reconstructing GCMs from generalised Killing forms] \label{theorem: reconstructing_generalised_cartan_matrices_from_generalised_killing_forms}
                Assume convention \ref{conv: kac_moody_root_systems}. Then\footnote{Note that the pairing $\<-, -\>: \Pi \x \Pi^{\vee} \to \Z$ is symmetric if and only if the GCM $A$ from convention \ref{conv: generalised_killing_forms} is symmetric.} $a_{ij} = 2\frac{\hat{\kappa}(\alpha_i, \alpha_j^{\vee})}{\hat{\kappa}(\alpha_i, \alpha_i^{\vee})}$, for all $1 \leq i, j \leq n$.
            \end{theorem}
                \begin{proof}
                    Combine definitions \ref{def: generalised_killing_forms} and \ref{def: root_data_of_square_matrices} with proposition \ref{prop: generalised_killing_forms_are_invariant}.
                \end{proof}
            \begin{corollary}
                Assume convention \ref{conv: kac_moody_root_systems}. Then, there is a pair of mutually inverse bijections between the set of isomorphism classes of tame reduced Kac-Moody $k$-algebras and that of tame GCMs as follows:
                    $$\mathfrak{Car}: \{\text{Tame reduced Kac-Moody $k$-algebras}\} \leftrightarrows \{\text{Tame GCMs}\}: \frakLie$$
                wherein $\mathfrak{Car}(-)$ is the association of tame GCMs to tame reduced Kac-Moody $k$-algebras in the manner described above and $\frakLie(-)$ is as in remark \ref{remark: reduced_kac_moody_algebras_are_uniquely_determined_by_generalised_cartan_matrices}.
            \end{corollary}
            \begin{corollary}[Tame Dynkin quivers of tame Kac-Moody algebras] \label{coro: tame_dynkin_quivers_of_tame_kac_moody_algebras}
                Assume convention \ref{conv: kac_moody_root_systems}. Then, there is a pair of mutually inverse bijections between the set of isomorphism classes of tame reduced Kac-Moody $k$-algebras and that of tame Dynkin (undirected) graphs as follows:
                    $$\Dyn \circ \mathfrak{Car}: \{\text{Tame reduced Kac-Moody $k$-algebras}\} \leftrightarrows \{\text{Tame Dynkin graphs}\}: \frakLie \circ \Car$$
                with $\Dyn(-)$ being as in definition \ref{def: dynkkin_quivers_assocaited_to_indecopmosable_generalised_cartan_matrices}.
            \end{corollary}
                \begin{proof}
                    Apply corollary \ref{coro: dynkin_quivers_of_symmetrisable_indecomposable_generalised_cartan_matrices} to theorem \ref{theorem: reconstructing_generalised_cartan_matrices_from_generalised_killing_forms}.
                \end{proof}
                
            \begin{definition}[Kac-Moody root systems] \label{def: kac_moody_root_systems}
                Suppose that $A \in \Mat_n(\Z)$ is a tame GCM and that $\g \cong \frakLie(A)$ is a (tame and reduced) Kac-Moody algebra associated to $A$. Then, the \textbf{root system} of $\g$ is nothing but that of the associated Dynkin quiver $\Dyn(\g) := \Dyn(A)$ (cf. example \ref{example: root_systems_of_tame_dynkin_quivers}). 
            \end{definition}
        
        \subsubsection{Integrable representations; root systems and Weyl groups of Kac-Moody algebras}
            Let us now go into a brief discussion of the notion of so-called \say{integrable representations} of reduced Kac-Moody algebras. We shall see that these representations behave predictably with respect to the permutation action of Weyl groups on root systems (cf. proposition \ref{prop: weyl_group_permute_root_spaces_of_kac_moody_algebras}): in particular, we shall observe that given any reduced Kac-Moody $\g$ (implicitly with a choice of a Cartan subalgebra $\h$) and any fixed root $\lambda$ thereof, there are Lie algebra isomorphisms:
                $$w: \g_{\lambda} \to \g_{w(\lambda)}$$
            for all Weyl group elements $w \in \rmW(\g, \h)$. Later on, this shall help us establish the fact that the so-called \say{untwisted affinisation} of a finite-type Weyl group $W$ (i.e. a Weyl group associated to a finite-type reduced Kac-Moody algebra $\g$) can be obtained as the coextension (in the category $\Grp$ of groups) of $W$ by its coroot lattice (cf. subsubsection \ref{subsubsection: untwisted_affine_weyl_groups}). Furthermore, we shall establish (through theorem \ref{theorem: serre_relations_generate_radical_of_non_reduced_kac_moody_algebras}) the adjoint representation of any given reduced Kac-Moody algebra as an important instance of an integrable representation. 
        
            \begin{remark}
                It should be noted that the subject matter of this subsubsection does not necessitate us working over $k = \bbC$, as we are not concerning ourselves with any question of a topological and/or analytic nature.
            \end{remark}
            \begin{convention}
                We assume some familiarity with the theory of classical Lie groups (or better yet, linear algebraic groups over algebraically closed fields). In particular, we will be omitting any explicit exposition of notions such as (locally) finite operators, (locally) nilpotent operators, operator exponentiation, etc.
                
                For the sake fixing notations, let us note once and for all that when we speak of exponentiations, we shall always mean so in the formal sense, i.e.:
                    $$\exp(\varphi) := \sum_{r \geq 0} \frac{1}{r!} \varphi^r$$
                with $\varphi \in \End_k(V)$ being some endomorphism on a $k$-vector space $V$ (which needs not be of a finite dimension); it is not assumed that such power series would converge.
            \end{convention}
            
            \begin{definition}[Cartan-diagonalisability] \label{def: cartan_diagonalisability}
                Consider a Kac-Moody algebra $\g$ (which needs not even be symmetrisable!), for which we choose a Cartan subalgebra $\h \cong \g_0$ (thanks to theorem \ref{theorem: root_space_decomposition_of_non_reduced_kac_moody_algebras}, such a choice can always be made) along with a minimal root datum $(\h, \Pi, \h^{\vee}, \Pi^{\vee}, B)$. A $k$-linear representation $(V, \pi) \in \Ob(\Rep_k(\g))$ of $\g$ is said to be \textbf{$\h$-diagonalisable} (or more intrinsically, \textbf{Cartan-diagonalisable}) if and only if there is a direct sum decomposition in $\Rep_k(\h)$ as follows:
                    $$V \cong \bigoplus_{\lambda \in \bbX(\g, \h)} V_{\lambda}$$
                wherein each of the summands $V_{\lambda}$, defined as below, is known as the $\h$-subrepresentation of $V$ \textbf{abstractly weighted} by the abstract root $\lambda \in \bbX(\g, \h)$:
                    $$V_{\lambda} := \{v \in V \mid \forall h \in \h: \pi(h)(v) = B(\lambda, h) v\}$$
                If $V_{\lambda} \not \cong 0$ then we shall say that $V_{\lambda}$ is \textbf{concretely weighted} (or simply \textbf{weighted}) by $\lambda$.
            \end{definition}
            \begin{definition}[Integrable representations of Kac-Moody algebras] \label{def: integrable_representations_of_kac_moody_algebras}
                Consider a Kac-Moody algebra $\g$ (which needs not even be symmetrisable!), for which we choose a Cartan subalgebra $\h \cong \g_0$ (thanks to theorem \ref{theorem: root_space_decomposition_of_non_reduced_kac_moody_algebras}, such a choice can always be made), and denote the set of Chevalley generators of $\g$ by $\h \cup \{e_i^-, e_i^+\}_{1 \leq i \leq n}$. Next, let $\pi: \g \to \gl(V)$ be a $k$-linear representation of $\g$. Such a representation is said to be \textbf{$\h$-integrable} (or simply \textbf{integrable}) if and only if for all $1 \leq i \leq n$, the operators $\pi(e_i^{\pm}) \in \gl(V)$ are locally nilpotent\footnote{This is so that $\exp(\pi(e_i^{\pm})) := \sum_{r \geq 0} \frac{1}{r!} \pi(e_i^{\pm})^r$ would be finite sums.}.
            \end{definition}
        
            \begin{definition}[Weyl groups of tame Kac-Moody algebras] \label{def: weyl_groups_of_tame_kac_moody_algebras}
                Assume convention \ref{conv: kac_moody_root_systems}. By proposition \ref{prop: uniqueness_of_generalised_killing_forms}, we know that there is a well-defined $\g$-invariant symmetric $k$-bilinear form $\hat{\kappa}: \g \x \g \to k$ that is unique up to non-zero constants. As such, the triple $(\h, \hat{\kappa}_0, \Phi(\g, \h))$ can be regarded as a well-defined root system of the reduced Kac-Moody $k$-algebra $\g$. From said root system, one can construct the corresponding \textbf{Weyl group} in the manner of definition \ref{def: weyl_groups}, which shall be denoted by $\rmW(\g, \h)$.
            \end{definition}
            \begin{definition}[Standard representations of Weyl groups] \label{def: standard_representations_of_weyl_groups}
                Assume convention \ref{conv: kac_moody_root_systems} and choose a minimal root datum $(\h, \Pi, \h^{\vee}, \Pi^{\vee}, B)$ for $\g$. 
                    \begin{enumerate}
                        \item Via the canonical composition $\rmW(\g, \h) \leq \SO(\h, B) \leq \GL(\h)$ of subgroup embeddings, one obtains a canonical \textit{faithful} $k$-linear representation $\rmW(\g, \h) \hookrightarrow \GL(\h)$, which shall be called the \textbf{standard representation} of the Weyl group $\rmW(\g, \h)$.
                        \item Through any integrable representation $\pi: \g \to \gl(V)$, one can define right-$\rmU(\g)$-module automorphism $w(\pi) \in \Aut_{\rmU(\g)}(V)$ for any $w \in \rmW(\g, \h)$ in the following manner:
                            $$w(\pi) := \exp(\pi(e_i^+)) \exp(-\pi(e_i^-)) \exp(\pi(e_i^+))$$
                    \end{enumerate}
            \end{definition}
            \begin{convention}[Weight multiplicities] \label{conv: weight_multiplicites}
                Assume convention \ref{conv: kac_moody_root_systems} and suppose that $V $ is a $\h$-diagonalisable representation of $\g$ with weight space decomposition $V \cong \bigoplus_{\lambda \in \Phi(\g, \h)} V_{\lambda}$. Then, we shall typically refer to the quantity $\dim_k V_{\lambda}$ as the \textbf{multiplicity} of the weight $\lambda$, though we shall not be writing $\operatorname{mult}(\lambda) := \dim_k V_{\lambda}$ as in say, \cite[Chapter 3]{kac_infinite_dimensional_lie_algebras} or \cite[Chapter 5]{perrin_kac_moody_algebras}.
            \end{convention}
            \begin{proposition}[Weyl groups permute root spaces of Kac-Moody algebras] \label{prop: weyl_group_permute_root_spaces_of_kac_moody_algebras}
                \cite[Proposition 5.2.6]{perrin_kac_moody_algebras} Assume convention \ref{conv: kac_moody_root_systems} and suppose that $\pi: \g \to \gl(V)$ be an integrable representation, for which we choose a weight space decomposition $V \cong \bigoplus_{\lambda \in \Phi(\g, \h)} V_{\lambda}$. Then for all $\lambda \in \Phi(\g, \h)$ and all $w \in \rmW(\g, \h)$, one has a right-$\rmU(\g)$-module isomorphism:
                    $$w: V_{\lambda} \to V_{w(\lambda)}$$
                defined as in definition \ref{def: standard_representations_of_weyl_groups}.
            \end{proposition}
            \begin{corollary}[Weyl groups preserve multiplicities] \label{coro: weyl_groups_perserve_multiplicities}
                Assume convention \ref{conv: kac_moody_root_systems} and suppose that $\pi: \g \to \gl(V)$ be an integrable representation, for which we choose a weight space decomposition $V \cong \bigoplus_{\lambda \in \Phi(\g, \h)} V_{\lambda}$. For all $\lambda \in \Phi(\g, \h)$ and all $w \in \rmW(\g, \h)$, one has that:
                    $$\dim_k V_{w(\lambda)} = \dim_k V_{\lambda}$$
            \end{corollary}
                
            Before we move on, let us make a detour for establishing some technical properties of (locally) nilpotent and (locally) finite operators.
            \begin{lemma}[Actions of derivations on Lie brackets] \label{lemma: actions_of_derivations_on_lie_brackets}
                \cite[Lemma 5.1.2(i)]{perrin_kac_moody_algebras} Suppose that $k$ is a field of characteristic $0$ and that $A$ is an associative $k$-algebra\footnote{E.g. the universal enveloping algebra of some Lie algebra (as a matter of fact, recall that $\ad: \rmU(\g) \to \rmU(\g)$, given by $\ad(x) := [x, -]$, is a derivation on $\rmU(\g)$ for all Lie algebras $\g$ over a field $k$ of characteristic $0$).}, and also that $\del: A \to A$ is a derivation (cf. defintion \ref{def: derivations_on_associative_algebras}). Then for all $r \geq 0$ and all $x, y \in A$, one has:
                    $$\del^r([x, y]) \sum_{0 \leq i \leq r} \binom{r}{i} [\del^i(x), \del^{r - i}(y)]$$
            \end{lemma}
            \begin{corollary}[Derivative of conjugation is adjunction] \label{coro: derivative_of_conjugation_is_adjunction}
                \cite[Lemma 5.1.2(ii) and Corollary 5.1.3]{perrin_kac_moody_algebras} Let $k$ be a field of characterisitc $0$ and $\g$ be a Lie $k$-algebra. Then, the following equation holds in $\g$, for all $z, x, y \in \g$:
                    $$\ad(z)^r([x, y]) \sum_{0 \leq i \leq r} \binom{r}{i} [\ad(z)^i(x), \ad(z)^{r - i}(y)]$$
                In particular, this implies that $\varphi, \psi \in \End_k(V)$ are two endomorphisms on a $k$-vector space $V$ and $\varphi$ is locally nilpotent then the sequence $\{\ad^r(\varphi)(\psi)\}_{r \geq 0}$ will span a finite-dimensional $k$-vector subspace of $\End_k(V)$, and furthermore, one will have that:
                    $$\exp(\varphi) \exp(\psi) \exp(-\varphi) = \exp(\ad(\varphi))(\psi)$$
            \end{corollary}
            
            \begin{convention}[Root triples] \label{conv: root_triples}
                Assume convention \ref{conv: kac_moody_root_systems} fix a root space decomposition $\g \cong \bigoplus_{\lambda \Phi(\g, \h) \cup \{0\}}$. Then, any triple $\{e_i^-, \alpha_i^{\vee}, e_i^+\}$, with $1 \leq i \leq n$, consisting of two Chevalley generators $e_i^{\pm}$ and a \textit{simple} coroot $\alpha_i^{\vee}$ shall be called a \textbf{root triple}.
            \end{convention}
            The following proposition can be proven exactly as if $\g$ was a (semi-)simple finite-dimensional Lie $k$-algebra. The interested reader may wish to consult \cite[Theorem 4.20]{carter_affine_lie_algebras}.
            \begin{proposition}[Simple root spaces are $1$-dimensional] \label{prop: simple_root_spaces_are_1_dimensional}
                Assume convention \ref{conv: kac_moody_root_systems}. Then, for any simple root $\alpha_i \in \Phi^{\simple}(\g, \h)$ (for some $1 \leq i \leq n$), one has that:
                    $$\dim_k \g_{\alpha_i} = 1$$
            \end{proposition}
            \begin{corollary}[Multiplicities of real roots of Kac-Moody algebras] \label{coro: multiplicities_of_real_roots_of_kac_moody_algebras}
                Assume convention \ref{conv: kac_moody_root_systems}. Then, for any real root $\alpha \in \Re(\Phi(\g, \h))$ (cf. definition \ref{def: real_and_imeginary_roots_of_tame_dynkin_quivers}), one has that:
                    $$\dim_k \g_{\alpha} = 1$$
            \end{corollary}
                \begin{proof}
                    Simply combine proposition \ref{prop: simple_root_spaces_are_1_dimensional} and corollary \ref{coro: weyl_groups_perserve_multiplicities} with the fact that for every real root $\alpha \in \Re(\Phi(\g, \h))$, there exists a Weyl group element $w \in \rmW(\g, \h)$ and a simple root $\alpha_i \in \Phi^{\simple}(\g, \h)$ such that $\alpha = w(\alpha_i)$ (cf. proposition \ref{prop: real_roots_as_reflections_of_simple_roots}).
                \end{proof}
            \begin{corollary}[Root triples are $\sl_2$-triples] \label{coro: root_triples_are_sl2_triples}
                Assume convention \ref{conv: kac_moody_root_systems} and fix a minimal root datum $(\h, \Pi, \h^{\vee}, \Pi^{\vee}, B)$. Then, for all $1 \leq i \leq n$, one has an isomorphism of Lie $k$-algebras:
                    $$\g_{-\alpha_i} \oplus \span_k \{\alpha_i^{\vee}\} \oplus \g_{\alpha_i} \to \sl_2(k)$$
                which sends the generators $e_i^-, \alpha_i^{\vee}, e_i^+$ to the usual $\sl_2$-triple\footnote{We choose this notation for the sake of consistency with the remainder of the section.} $e^-, \alpha_i^{\vee}, e^+$, which we recall, is subjected to the relations $[\alpha_i^{\vee}, e^{\pm}] = \pm 2 e^{\pm}, [e^-, e^+] = \alpha_i^{\vee}$ (and indeed, the notation is justified, since the single generator of the Cartan subalgebra of $\sl_2(k)$ coincides with the unique simple coroot).
            \end{corollary}
            The observation that root triples are in fact $\sl_2$-triples begs the recollection of the following well-known result regarding the representation theory of $\sl_2(k)$, which helps us explain the etymology behind \say{integrable representations}:
            \begin{proposition}[Integrability of the adjoint representation of $\sl_2$] \label{prop: integrability_of_the_adjoint_representation_of_sl2}
                \cite[Proposition 4.2.6]{perrin_kac_moody_algebras} Denote the set of Chevalley generators of $\sl_2(k)$ by $\{e^-, \theta^{\vee}, e^+\}$ (which we recall, is subjected to the Chevalley relations $[\theta^{\vee}, e^{\pm}] = \pm 2 e^{\pm}, [e^-, e^+] = $). 
                    \begin{enumerate}
                        \item Within $\rmU(\sl_2(k))$, the following hold for all powers $r \in \N$:
                            $$[\theta^{\vee}, (e^{\pm})^{\tensor r}] = \pm 2r (e^{\pm})^{\tensor r}$$
                            $$[e^{\pm}, (E^{\pm})^{\tensor r}] = -r(r - 1) (E^{\pm})^{\tensor (r - 1)} + r (E^{\pm})^{\tensor (r - 1)} \theta^{\vee}$$
                        \item Consider a $k$-linear representation $\pi: \sl_2(k) \to \gl(V)$ of $\sl_2(k)$ along with an element $v \in V$ for which there exists $\lambda \in k$ so that:
                            $$\pi(\theta^{\vee})(v) = \lambda v$$
                        Then, the following hold for all powers $r \in \N$:
                            $$( \pi(\theta^{\vee}) \circ \pi(e^+)^r )(v) = (\lambda - 2r)\pi(e^+)^r(v)$$
                        and if one has furthermore that $\pi(e^-)(v) = 0$ then:
                            $$( \pi(e^-) \circ \pi(e^+)^r )(v) = r(\lambda - r + 1) \pi(e^+)^{r - 1}(v)$$
                        \item For every $r \in \N$, there exists a \textit{unique-up-to-isomorphisms} $(r + 1)$-dimensional irreducible representation $\pi_r: \sl_2(k) \to \gl(V_r)$ for which there exists a basis $\{v_i\}_{0 \leq s \leq r} \subset V_r$ such that:
                            $$\pi_r(\theta^{\vee})(v_s) = (r - 2s) v_s$$
                            $$\pi_r(e^+)(v_s) = v_{s + 1}$$
                            $$\pi_r(e^-)(v_s) = s(r + 1 - s) v_{s - 1}$$
                    \end{enumerate}
            \end{proposition}
            \begin{remark}
                Proposition \ref{prop: integrability_of_the_adjoint_representation_of_sl2}, incidentally, also suggests to us that in the event that one wishes to construct an infinite-dimensional Kac-Moody algebra from a finite-dimensional one (cf. theorem \ref{theorem: untwisted_affinisation}), an important step will be to adjoint a degree-preserving derivation to the Cartan subalgebra of the finite-dimensional Kac-Moody algebra that one starts out with (cf. definition \ref{def: extended_simply_affine_lie_algebras}).
            \end{remark}
            The following theorem gives an explicit description of radicals of non-reduced Kac-Moody algberas (cf. lemma \ref{lemma: non_reduced_kac_moody_algebras_have_non_trivial_radicals}) in terms of Chevalley generators. More importantly, however, it establishes the fact that the \textit{a priori} $\h$-diagonalisable (cf. remark \ref{remark: root_space_and_triangular_decompositions_of_reduced_kac_moody_algebras}) adjoint representation of any reduced Kac-Moody algebra is in fact integrable in the sense of definition \ref{def: integrable_representations_of_kac_moody_algebras}.
            \begin{theorem}[Serre relations generate radicals of non-reduced Kac-Moody algebras] \label{theorem: serre_relations_generate_radical_of_non_reduced_kac_moody_algebras}
                \cite[Proposition 4.2.7]{perrin_kac_moody_algebras} \footnote{See also \cite[Theorem 9.11]{kac_infinite_dimensional_lie_algebras}.}Assume convention \ref{conv: unipotent_radicals_of_non_reduced_kac_moody_algebras} and \ref{conv: a_fixed_generalised_cartan_matrix}, and pick a root space decomposition:
                    $$\bfg \cong \bigoplus_{\lambda \in \bbX(\bfg, \h)} \bfg_{\lambda}$$
                Then the so-called \textbf{Serre relations}:
                    $$\forall 1 \leq i, j \leq n: \ad(e_i^{\pm})^{1 - a_{ij}}(e_j^{\pm}) = 0$$
                shall generate the $\g$-ideals $\rad^{\pm}(\bfg)$ (cf. lemma \ref{lemma: non_reduced_kac_moody_algebras_have_non_trivial_radicals}).
            \end{theorem}
                \begin{proof}
                    Combine proposition \ref{prop: integrability_of_the_adjoint_representation_of_sl2} (applicable thanks to corollary \ref{coro: root_triples_are_sl2_triples}) with lemma \ref{lemma: commutator_of_the_negative_and_positive_unipotent_radicals}. 
                \end{proof}
            \begin{corollary}[Presentations of reduced Kac-Moody algebras] \label{coro: presentations_of_reduced_kac_moody_algebras}
                Assume convention \ref{conv: unipotent_radicals_of_non_reduced_kac_moody_algebras}, and pick a root space decomposition:
                    $$\g \cong \bigoplus_{\lambda \in \bbX(\g, \h)} \g_{\lambda}$$
                Then the associated reduced Kac-Moody algebra $\g$ shall be the quotient of the free Lie $k$-algebra on $h \cup \{e_i^{\pm}\}_{1 \leq i \leq n}$ by the ideal - abusively also denoted by $\chev(A)$ - generated by the following relations:
                    $$\forall h, h' \in \h: [h, h'] = 0$$
                    $$\forall h \in \h: \forall 1 \leq i \leq n: [h, e_i^{\pm}] = \mp B(\alpha_i, h) e_i^{\pm}$$
                    $$\forall 1 \leq i, j \leq n: [e_i^-, e_j^+] = \delta_{ij} \alpha_i^{\vee}$$
                    $$\forall 1 \leq i, j \leq n: \ad(e_i^{\pm})^{1 - a_{ij}}(e_j^{\pm}) = 0$$
            \end{corollary}
            \begin{remark}
                Through corollary \ref{coro: presentations_of_reduced_kac_moody_algebras} and Serre's Theorem on semi-simple finite-dimensional Lie algebras over algebraically closed fields $k$ of characteristic $0$, one sees that the semi-simple finite-dimensional Lie $k$-algebras are precisely finite direct sums\footnote{Recall that we have assumed that tame GCMs are indecomposable.} of finite-type Kac-Moody $k$-algebras. 
            \end{remark}