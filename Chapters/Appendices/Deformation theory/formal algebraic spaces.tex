\section{Formal algebraic spaces}
    \subsection{Topological rings and modules}
        \subsubsection{Linearly topologised rings and modules}
            \begin{definition}[Topological group] \label{def: topological_groups}
                A \textbf{topological group} is a group object internal to the cartesian-closed category of topological spaces. Homomrphisms of topological groups are nothing but continuous group homomorphisms.
            \end{definition}
            \begin{example}
                $(\R, +)$ with its usual metric topology is an abelian topological group. A somewhat non-trivial example is $\Gal(K^{\unr}/K)$ for any field $K$ (e.g. $\Gal(\bar{\F}_p/\F_p) \cong \hat{\Z}$): this is an instance of a profinite group, i.e. a (quasi-)compact, Hausdorff, and totally disconnected topological group; \'etale fundamental groups of connected schemes make up another natural family of examples of profinite groups. Any Lie group is also a topological group.
            \end{example}
            \begin{definition}[Topological rings] \label{def: topological_rings}
                A \textbf{topological ring} is a ring object internal to the symmetric monoidal (whose monoidal structure is given by cartesian product of spaces) category of topological spaces. Homomrphisms of topological rings are nothing but continuous ring homomorphisms.
            \end{definition}
            \begin{example}
                $(\R, +, \cdot)$ with the usual metric topology is the archetypal topological ring, and so is $\Q_p$ (and also $\Z_p$) with its usual ultrametric topology.
            \end{example}
            \begin{definition}[Topological modules] \label{def: topological_modules}
                \noindent
                \begin{enumerate}
                    \item \textbf{(Topological modules):} A \textbf{topological module} over some given topological ring $R$ is a topological abelian group $M$ endowed with a \textit{continuous} $R$-action $\rho: R \to \End_{\cont}(M)$. Homomrphisms of topological $R$-modules (for some fixed topological ring $R$) are nothing but continuous $R$-module homomorphisms.
                    \item \textbf{(Linearly topologised modules):} A topological module $M$ over some topological ring $R$ is said to be \textbf{linearly topologised} if and only if $M$ has a topological basis of open neigbourhoods of $0$ consisting of topological $R$-submodules.
                \end{enumerate}
            \end{definition}
            \begin{definition}[(Pre-)admissible and (pre-)adic rings] \label{def: pre_admissible_and_pre_adic_rings}
                A linearly topologised ring $R$ is said to have an \textbf{ideal of definition} $I \subset R$ if $I$ is open as a subset of $R$ and if every open neighbourhood $U \ni 0$ contains $I^n$, for some natural number $n \in \N$. If such an ideal of definition $I \subset R$ exists for a given topological ring $R$, we say that $R$ (or rather, the pair $(R, I)$) is \textbf{pre-admissible}; a (topologically) complete pre-admissible topological ring is said to be \textbf{admissible}. Finally, (pre-)admissible topological ring $(R, I)$ is (pre-)adic if and only if $\{I^n\}_{n \in \N}$ forms a topological basis; such rings are said to be carrying the \textbf{$I$-adic} topology (respectively, \textbf{$I$-adically complete}).
            \end{definition}
            \begin{example}
                An instance of an incomplete pre-adic ring is $\F_p$ (regarded as the quotient $\Z_p/p$) with the $p$-adic topology, whereas $\Z_p$ itself is complete with respect to the $p$-adic topology by construction. For more non-trivial instances of adic rings, consider perfectoid fields such as $\Q_p( p^{ 1/p^{\infty} } )^{\wedge} := \left( \bigcup_{n \in \N} \Q_p( p^{ 1/p^n } ) \right)^{\wedge}$, which by definition are residually semi-perfect\footnote{I.e. the Frobenius endomorphisms on their residue fields are surjective.} complete non-archimedean mixed-characteristic fields whose associated rank-$1$ valuation is non-discrete (cf. \cite{scholze2011perfectoid}).
            \end{example}
            \begin{example}
                It also bears mentioning that discrete rings are $0$-adic and as such, the category of adic rings and continuous homomorphisms between them admits the category of discrete rings as a full subcategory. 
            \end{example}
            \begin{convention}[Adic completions] \label{conv: adic_completions}
                From now on, the topological completion of a pre-adic ring $(A, I)$ shall be denoted by $(A, I)^{\wedge}$, or simply by $\hat{A}$ when the ideal of definition $I$ is clearly understood from context (e.g. $\Z_p$ with the $p$-adic topology).
            \end{convention}
            \begin{proposition}[A continuity criterion for homomorphisms between pre-adic rings] \label{prop: continuity_criterion_for_homomorphisms_between_pre_adic_rings}
                Let $(A, I)$ and $(B, J)$ be pre-adic rings and $\varphi: R \to S$ be a (not necessarily continuous) ring homomorphism. In such a situation, $\varphi$ is continuous if and only if $\varphi(I^n) \subseteq J$ for some positive integer $n \geq 1$. 
            \end{proposition}
                \begin{proof}
                    Suppose first of all that $\varphi(I^n) \subseteq J$ for some positive integer $n \geq 1$. Firstly, this implies that there exists a natural number $n \in \N$ for which $I^n \subset \varphi^{-1}(J)$. 
                    
                    Conversely, suppose that $\varphi: R \to S$ is a continuous ring homomorphism. Since $(S, J)$ is pre-adic, its ideal of definition $J$ is open, and the pre-image $\varphi^{-1}(J)$ is therefore open by continuity. Now, $(R, I)$ is also pre-adic, meaning that $\{I^n\}_{n \in \N}$ form a topological basis for $R$; this implies - via the fact that countable unions of open subsets are open - that there is a set $\calN \subseteq \N$ of natural numbers such that $\varphi^{-1}(J) = \bigcup_{n \in \calN} I^n$. Applying $\varphi$ to both sides then yields us $J = \bigcup_{n \in \calN} \varphi(I^n)$, from which we easily see that there must exist some natural number $n \in \N$ such that $\varphi(I^n) \subset J$ as claimed. 
                \end{proof}
            \begin{remark}[What does proposition \ref{prop: continuity_criterion_for_homomorphisms_between_pre_adic_rings} tell us ?]
                Let $(R, I)$ and $(S, J)$ be pre-adic rings. Since $\{I^n\}_{n \in \N}$ is actually an $\N$-filtered diagram (specifically, we have $0 \subset I \subset I^2 \subset ...$), we can think of the ideal of definition $I \subset R$ as the \say{open unit ball} inside $R$ and see that what proposition \ref{prop: continuity_criterion_for_homomorphisms_between_pre_adic_rings} tells us in one direction is that by shrinking the \say{open ball} $I \subset R$ down to an \say{open ball} $I^n \subset R$ of smaller radius to ensure that $\varphi(I^n) \subseteq J$ for some ring homomorphism $\varphi: R \to S$, we will have guaranteed that $\varphi$ is continuous; more succinctly, this direction of the proposition says that it suffices to check that a ring homomorphism between pre-adic rings is an open map on the basic open subsets to verify that it is continuous on its entire domain. In the converse direction, the proposition informs us that if $\varphi: R \to S$ is continuous then the \say{open unit ball} $J \subset R$ contains within it the image $\varphi(I^n)$ of some \say{small open ball} $I^n \subset R$; equivalently, this is saying that the pre-image $\varphi^{-1}(J)$ contains some $I^n$, but since $J$ is open by definition and $\varphi$ is continuous, and since $\{I^n\}_{n \in \N}$, as established, is an increasing family of basic open subsets of $R$, this direction of proposition \ref{prop: continuity_criterion_for_homomorphisms_between_pre_adic_rings} is entirely and purely topological.
            \end{remark}
            \begin{lemma}[The Baire Category Theorem] \label{lemma: baire_category_theorem}
                Let $E$ be a complete linearly topologised abelian group with a countable basis consisting of open neighbourhoods of $0$, and let $\{U_n\}_{n \in \N}$ is a countable set of dense open subsets of $E$. In such a situation, $\bigcap_{n \in \N} U_n$ is also dense as a subset of $E$.
            \end{lemma}
                \begin{proof}
                    
                \end{proof}
            \begin{proposition}[The Open Mapping Lemma] \label{prop: open_mapping_lemma}
                
            \end{proposition}
                \begin{proof}
                    
                \end{proof}
                
            \begin{lemma}[Sums with open submodules are open] \label{lemma: sums_with_open_submodules_are_open}
                Let $M$ be a topological module over some topological ring $R$ and let $M_0 \subseteq M$ be an open submodule. Then for any submodule $N \subseteq M$, which needs not be open, the sum $N + M_0$ is open. 
            \end{lemma}
                \begin{proof}
                    By definition, $N + M_0 := \{y + x \mid \forall y \in N, x \in M_0\}$, and since the addition operation $+: M \x M \to M$ is continuous as a consequence of the assumption that $M$ is a topological module, we have $N + M_0 = \bigcup_{y \in N} \sigma_y^{-1}(M_0)$ the pre-images of the maps $\{\sigma_y: M \to M\}_{y \in N}$, each given by $x \mapsto -y + x$. But $M_0$ is open by assumption and the maps $\sigma_y$ are all trivially continuous, so $N + M_0$ is open inside $M$ by virtue of being a union of open subsets of $M$, namely the preimages $\sigma_y^{-1}(M_0)$. 
                \end{proof}
            \begin{lemma}[Open submodules are closed] \label{lemma: open_submodules_are_closed}
                Let $M$ be a topological module. Then, any open submodule $M_0 \subseteq M$ is also automatically closed.
            \end{lemma}
                \begin{proof}
                    We shall seek to prove that $M \setminus M_0$ is open. For this, pick an arbitrary element $y \in M \setminus M_0$ and consider the coset $y + M_0$. From the proof of lemma \ref{lemma: sums_with_open_submodules_are_open}, we know that since $M_0$ is an open subset of $M$, so is the coset $y + M_0$. Now, note that $y \not = 0$ necessarily (as $0 \in M_0$ due to $M_0$ being a submodule of $M$) and therefore $(y + M_0) \cap M_0 = \varnothing$. From this, one infers that $M \setminus M_0 = \bigcup_{y \in M \setminus M_0} (y + M_0)$, i.e. $M \setminus M_0$ is a union of open subsets of $M$ and is thereby open itself. This proves that $M_0$ is closed. 
                \end{proof}
            \begin{proposition}[Closure of linearly topologised submodules] \label{prop: closure_of_linearly_topologised_submodules}
                Let $R$ be a topological ring and let $M$ be a linearly topologised $R$-module with a basis $\{M_{\lambda}\}_{\lambda \in \Lambda}$ of open neighbourhoods of $0$. The topological closure $\bar{N}$ of any submodule $N \subseteq M$ is thus given by $\bigcap_{\lambda \in \Lambda} (N + M_{\lambda})$.
            \end{proposition}
                \begin{proof}
                    By lemma \ref{lemma: sums_with_open_submodules_are_open}, each of the factor $N + M_{\lambda}$ (for every $\lambda \in \Lambda$) is an open submodule of $M$, meaning that they are also closed (cf. lemma \ref{lemma: open_submodules_are_closed}). The intersection $\bigcap_{\lambda \in \Lambda} (N + M_{\lambda})$ is therefore closed and as such is its own closure, and it thus remains to show that this intersection is the smallest closed subset of $M$ containing $N$. 
                \end{proof}
            
            \begin{remark}[(Co)limits of topological groups] \label{remark: (co)limits_of_topological_groups}
                Topological groups, by definition, are nothing but group objects internal to the category of topological spaces, which is both complete and cocomplete. As such, the category of topological groups is also complete and cocomplete. 
            \end{remark}
            \begin{remark}[(Co)limits of topological rings] \label{remark: (co)limits_of_topological_rings}
                The category of topological abelian groups, like the category of topological group, is also complete and cocomplete. It is also symmetric monoidal 
            \end{remark}
            \begin{remark}[(Co)limits of topological modules] \label{remark: (co)limits_of_topological_modules}
                
            \end{remark}
            \begin{lemma}[Adic completion as cofiltered limits] \label{lemma: adic_completion_as_cofiltered_limits}
                Let $M$ be a linearly topologised module with a basis $\{M_{\lambda}\}_{\lambda \in \Lambda}$ of open submodules, which we shall view as partially ordered set, regarded as a filtered diagram. For such a topological module, the topological completion is given by the cofiltered limit $M^{\wedge} \cong \underset{\lambda \in \Lambda}{\lim} M/M_{\lambda}$.  
            \end{lemma}
                \begin{proof}
                    
                \end{proof}
            \begin{remark}
                Let $M$ be a linearly topologised module over some topological ring $R$ and fix a topological basis $\{M_{\lambda}\}_{\lambda \in M}$ of $M$ consisting of open submodules $M_{\lambda}$ (and note that such a basis indeed exists thanks to $M$ being linearly topologised). At the same time, consider a short exact sequence of topological $R$-modules:
                    $$
                        \begin{tikzcd}
                        	0 & N & M & Q & 0
                        	\arrow[from=1-1, to=1-2]
                        	\arrow[from=1-2, to=1-3]
                        	\arrow["\pi", from=1-3, to=1-4]
                        	\arrow[from=1-4, to=1-5]
                        \end{tikzcd}
                    $$
                Such a basis induces bases $\{M_{\lambda} \cap N\}_{\lambda \in \Lambda}$ for the subspace topology on $N$ and $\{\pi(M_{\lambda})\}_{\lambda}$ for the quotient topology on $Q$ induced via the quotient map $\pi: M \to Q$. 
            \end{remark}
            \begin{proposition}[Completion is exact] \label{prop: completion_is_exact}
                Consider a short exact sequence:
                    $$
                        \begin{tikzcd}
                        	0 & N & M & Q & 0
                        	\arrow[from=1-1, to=1-2]
                        	\arrow[from=1-2, to=1-3]
                        	\arrow["\pi", from=1-3, to=1-4]
                        	\arrow[from=1-4, to=1-5]
                        \end{tikzcd}
                    $$
                of linearised topological modules over some topological ring $R$, wherein $N$ and $Q$ are endowed - respectively - with the subspace topology and the quotient topology induced from a given linear topology on $M$. Then:
                    \begin{enumerate}
                        \item the sequence $0 \to N^{\wedge} \to M^{\wedge} \to Q^{\wedge} \to 0$ of completions remains exact, 
                        \item $N^{\wedge}$ is actually the topological closure of $N$ inside $\hat{M}$, i.e. $\overline{N^{\wedge}} = N^{\wedge}$ as $R$-submodules of $M^{\wedge}$, and
                        \item if in addition, $M$ has a countable basis of open submodules then the quotient map $\pi^{\wedge}: M^{\wedge} \to Q^{\wedge}$
                    \end{enumerate}
            \end{proposition}
                \begin{proof}
                    \noindent
                    \begin{enumerate}
                        \item 
                        \item 
                        \item 
                    \end{enumerate}
                \end{proof}
        
        \subsubsection{Taut and adic ring maps}
            \begin{definition}[Weakly (pre-)admissible rings] \label{def: weakly_pre_admissible_rings}
                \noindent
                \begin{enumerate}
                    \item \textbf{(Topologically nilpotent elements):} An element $f \in A$ of a linearly topologised ring $A$ is said to be \textbf{topologically nilpotent} if $f^n \to 0$ as $n \to +\infty$. 
                    \item \textbf{(Weakly (pre-)admissible rings):} A \textbf{weakly (pre-)admissible ring} is a pair $(A, I)$ consisting of a linearly topologised ring $A$ and an open ideal $I \subset A$ consisting entirely of topologically nilpotent elements. 
                \end{enumerate}
            \end{definition}
            \begin{proposition}[Weakly admissible rings induce Henselian pairs] \label{prop: weakly_admissible_rings_induce_henselian_pairs}
                Any weakly admissible ring $(A, I)$ is a Henselian pair.
            \end{proposition}
                \begin{proof}
                    
                \end{proof}
        
            \begin{definition}[Taut ring maps] \label{def: taut_ring_maps}
                Let $\varphi: A \to B$ be a continuous homomorphism between linearly topologised rings; in addition, let $\{I_{\lambda}\}_{\lambda \in \Lambda}$ be a topological basis for $A$, consisting of open ideals $I_{\lambda}$. Such a homomorphism is said to be \textbf{taut} if for every open ideal $I \subseteq A$, the topological closure $\overline{\varphi(I)B}$ of the ideal $\varphi(I)B$ is also open, and if the open ideals $\overline{\varphi(I_{\lambda})B}$ form a topological basis for $B$.
            \end{definition}
        
    \subsection{Affine formal algebraic spaces and affine formal schemes}
    
    \subsection{The categories of formal algebraic spaces and formal schemes}