\section{Reductive group schemes and geometric reductivity}
    \subsection{Classical reductive groups}
        \begin{convention} \label{conv: classical_reductive_groups_ground_field}
            Throughout, we shall be working with smooth affine group schemes over an algebraically closed field $k$, though we do not assume that these they are Zariski-connected.
        \end{convention}
        \begin{remark}[Regarding the smoothness assumption]
            If $\chara k = 0$ (respectively, $\chara k = p$ for some prime $p$ and $k$ is perfect) then we will not have to make the extra assumption of smoothness for affine group schemes that are locally of finite type (respectively, locally of finite type and reduced), since such group schemes are already smooth \textit{a priori} (cf. \cite[\href{https://stacks.math.columbia.edu/tag/047N}{Tag 047N} and \href{https://stacks.math.columbia.edu/tag/047P}{Tag 047P}]{stacks}).
        \end{remark}
        \begin{remark}[Embeddings into $\GL_n$]
            Recall also that every so-called \say{linear algebraic group} over $k$ (i.e. group $k$-schemes which are affine and of finite type over $\Spec k$) are closed subschemes of $(\GL_n)_k$, whose functor of points is represented by the affine $k$-scheme:
                $$\Spec k[x_{11}, x_{12}, ..., x_{nn}][1/\det]$$
            wherein $\det := \det(x_{11}, x_{12}, ..., x_{n^2})$ is the polynomial that returns the determinant of the matrix $(x_{ij})_{1 \leq i, j \leq n} \in \Mat_n(k)$. This is because should $G$ be a linear algebraic group over $k$, then the fact that it is of finite type over $\Spec k$ means that it is defined by some $k[x_{11}, x_{12}, ..., x_{nn}][1/\det]$-ideal $I$ via:
                $$G \cong \Spec k[x_{11}, x_{12}, ..., x_{nn}][1/\det]/I$$
        \end{remark}
    
        \subsubsection{Solvable and reductive linear algebraic groups}
        
        \subsubsection{Roots and coroots}
        
        \subsubsection{Root data and root space decompositions}
        
        \subsubsection{Positive roots and parabolic subgroups}
        
        \subsubsection{Based root data and pinnings}
    
    \subsection{Reductive group schemes}