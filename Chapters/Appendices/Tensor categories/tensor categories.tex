\section{Tensor categories}
    \subsection{Tensor categories}
        \begin{convention}
            Fix an algebraically closed field $k$.
        \end{convention}
        
        \subsubsection{Multi-tensor categories}
            \begin{definition}[Multi-tensor categories] \label{def: mutli_tensor_categories}
                A \textbf{multi-tensor category} over $k$ is a locally finite $k$-linear abelian rigid monoidal category $(\calV, \tensor, \1)$ wherein the bifunctor $\tensor$ is $k$-bilinear on morphisms, i.e. for every triple of objects $V, V', V'' \in \Ob(\calV)$, the induced morphism $\tensor_{V, V', V''}: \calV(V, V') \x \calV(V', V'') \to \calV(V, V'')$ will be $k$-bilinear if it exists. If we have - in addition - that $\End_{\calV}(\1) \cong k$ then $(\calV, \tensor, \1)$ will be known as a \textbf{tensor category} over $k$.
            \end{definition}
            \begin{example}
                \noindent
                \begin{itemize}
                    \item The category $k\-\Vect^{\fin}$ of finite-dimensional $k$-vector spaces is a tensor category\footnote{In fact, this is a fusion category over $k$ (cf. definition \ref{def: fusion_categories}).} over $k$.
                    \item This is definitely not the case for the category $k\-\Vect$ of all $k$-vector spaces, due to the fact that given any pair of infinite-dimensional $k$-vector spaces $V, W$, the hom-space $\Hom_k(V, W)$ would be a finite-dimensional $k$-vector spaces, meaning that $k\-\Vect$ is not even locally finite.
                \end{itemize}
            \end{example}
            \begin{example}
                Let $A$ be a finite-dimensional semi-simple associative and unital algebra over $k$ and consider the category $A\bimod^{\fin}$ of finitely generated $A$-bimodules. 
            \end{example}
            \begin{example}
                \noindent
                \begin{itemize}
                    \item The category $\Rep_k^{\fin}(G)$ of finite-dimensional $k$-linear representation of a group $G$ can also be easily shown to be a tensor category\footnote{This is also a fusion category over $k$. In fact, one ought to view $k\-\Vect^{\fin}$ as the category of finite-dimensional $k$-linear representations of the trivial group $1$.} over $k$. To make a slight generalisation, one could assume that $G$ is an affine group scheme over $\Spec k$ and consider the category $\Vect^{\fin}(X/k)^G$ of $G$-equivariant finite-rank vector bundles on a $k$-scheme $X$ with a $G$-action over $\Spec k$.
                    \item The category $\Rep_k^{\fin}(\g)$ of finite-dimensional $k$-linear representations of a Lie $k$-algebra $\g$ is also a tensor category over $k$, due to being canonically isomorphic to to the category of finite-type left-$\scrU(\g)$-modules. 
                \end{itemize}
            \end{example}
            
            
    
    \subsection{Grothendieck rings and dimension theory for tensor categories}
    
    \subsection{Tensor products of tensor categories}