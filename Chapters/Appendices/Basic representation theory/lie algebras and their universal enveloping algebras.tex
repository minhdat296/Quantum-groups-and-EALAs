\section{Lie algebras and their universal enveloping algebras}
    \subsection{Generalities}
        \begin{definition}[Lie algebras] \label{def: lie_algebras}
            Let $k$ be a ring and let $(\calV, \tensor, 1, \tau)$ be a symmetric monoidal $k$-linear category with braiding isomorphisms:
                $$\tau_{x, y}: x \tensor y \cong y \tensor x$$
            A Lie algebra internal to is then an object $\g \in \calV$ equipped with a so-called Lie bracket:
                $$[-,-]: \g \tensor \g \to \g$$
            subject to two requirements:
                \begin{enumerate}
                    \item \textbf{(Skew-symmetry)}
                        $$[-,-] + [-,-] \circ \tau_{\g, \g} = 0$$
                    \item \textbf{(The Jacobi identity)}
                        $$
                            \begin{aligned}
                                & \left[-, [-,-]\right]
                                \\
                                + & \left[-, [-,-]\right] \circ \left(\id_{\g} \tensor \tau_{\g, \g}\right) \circ \left(\tau_{\g, \g} \tensor \id_{\g}\right)
                                \\
                                + & \left[-, [-,-]\right] \circ \left(\tau_{\g, \g} \tensor \id_{\g}\right) \circ \left(\id_{\g} \tensor \tau_{\g, \g}\right)
                                \\
                                = & \: 0
                            \end{aligned}
                        $$
                \end{enumerate}
            Lie algebras internal to a symmetric monoidal $k$-linear category $\calV$ form a full subcategory which we shall denote by $\Lie\Alg(\calV)$. Its objects are the Lie algebra objects of $\g$, and its morphisms are arrows in $\calV$ that intertwine with Lie brackets, i.e. they are arrows $\phi: \g \to \h$ such that:
                $$[-,-]_{\h} \circ (\phi \tensor \phi) = \phi \circ [-,-]_{\g}$$
            or equivalently, such that diagrams of the following form commute in $\calV$:
                $$
                    \begin{tikzcd}
                    	{\g \tensor \g} & {\g} \\
                    	{\h \tensor \h} & {\h}
                    	\arrow["{\phi}", from=1-2, to=2-2]
                    	\arrow["{\phi \tensor \phi}"', from=1-1, to=2-1]
                    	\arrow["{[-,-]_{\h}}", from=2-1, to=2-2]
                    	\arrow["{[-,-]_{\g}}", from=1-1, to=1-2]
                    \end{tikzcd}
                $$
        \end{definition}
        
        \begin{definition}[Enveloping algebras] \label{def: enveloping_algebras}
            Let $k$ be a ring and let $(\calV, \tensor, 1, \tau)$ be a symmetric monoidal $k$-linear category.
                \begin{enumerate}
                    \item \textbf{(The Lie functor)} The Lie functor is the one that assigns to each associative and unital algebra $A$ internal to $\calV$ a Lie algebra $\frakLie(A)$ whose underlying object is just $A$, and whose Lie bracket is given by:
                        $$[-,-]_{\frakLie(A)} := \nabla_A - \nabla_A \circ \tau_{A,A}$$
                    \item \textbf{(Enveloping algebras)} Fix a Lie algebra object $\g$ of $\calV$. Then, an enveloping algebra of a Lie algebra $\g$ internal to $\calV$ is just a Lie algebra homomorphism:
                        $$e: \g \to \frakLie(A)$$
                    for some $A \in \Assoc\Alg(\calV)$. These enveloping algebras form a category, whose objects are Lie algebra homomorphisms as described above, and whose morphisms are commutative triangles in $\Lie\Alg(\calV)$ as follows:
                        $$
                            \begin{tikzcd}
                            	& {\g} \\
                            	{\frakLie(A)} && {\frakLie(A')}
                            	\arrow[from=2-1, to=2-3]
                            	\arrow["{e}"', from=1-2, to=2-1]
                            	\arrow["{e'}", from=1-2, to=2-3]
                            \end{tikzcd}
                        $$
                    which we note to be induced by algebra homomorphisms $A \to A'$. We will denote the category of enveloping algebras of $\g$ by $\Env(\g)$. 
                    \\
                    The universal enveloping algebra of $\g$ (denoted by $\U(\g)$), if it exists, then its Lie algebra will be the initial object of $\Env(\g)$. 
                \end{enumerate}
        \end{definition}
        
        \begin{theorem}[Existence and uniqueness of universal enveloping algebras] \label{theorem: universal_enveloping_algebras_universal_property}
             Let $k$ be a ring and let $(\calV, \tensor, 1, \tau)$ be a symmetric monoidal $k$-linear category. Then:
                \begin{enumerate}
                    \item \textbf{(Existence and uniqueness)} There is the following ($\calV$-enriched) adjunction if $\calV$ has all countable coproducts:
                        $$
                            \begin{tikzcd}
                            	{(U \ladjoint \frakLie): \Assoc\Alg(\calV)} & {\Lie\Alg(\calV)}
                            	\arrow["{\frakLie}"{name=0, swap}, from=1-1, to=1-2, shift right=2]
                            	\arrow["{\U}"{name=1, swap}, from=1-2, to=1-1, shift right=2]
                            	\arrow["\dashv"{rotate=-90}, from=1, to=0, phantom]
                            \end{tikzcd}
                        $$
                    wherein $\U$ is the functor sending each Lie algebra in $\calV$ to its universal enveloping algebra.
                    \item \textbf{(Explicit construction)} If $\calV$ also has all cokernels then we can explicitly characterise the universal enveloping algebra of a Lie algebra $\left(\g, [-,-]_{\g}\right)$ as the following quotient of the (Lie algebra canonically associated to) the tensor algebra $T(\g)$:
                        $$\frakLie\left(\U(\g)\right) \cong \coker \bigg(\left([-,-]_{\frakLie T(\g)} - \left(\nabla_{T(\g)} - \nabla_{T(\g)} \circ \tau_{T(\g), T(\g)}\right)\right): T(\g) \tensor T(\g) \to T(\g)\bigg)$$
                \end{enumerate}
        \end{theorem}
            \begin{proof}
                \noindent
                \begin{enumerate}
                    \item We will be using Freyd's Adjoint Functor Theorem \cite[Theorem V.6.2]{maclane} to prove this assertion, and to that end, let us first note that the category $\Assoc\Alg(\calV)$ is complete and locally small (this can be proven in the exact same way that one might prove that algebraic categories such as $\Ab$ or $\Ring$ are complete and locally small). Next, we will try to show that the functor $\frakLie$ satisfies the \href{https://ncatlab.org/nlab/show/solution+set+condition}{\underline{solution set condition}}, which we can do by finding a small indexing set $I$ such that for all enveloping algebras $\g \to \frakLie(A)$ of $\g$, there exists a family of algebras $A_i$ indexed by $i \in I$ and factorisations:
                        $$
                            \begin{tikzcd}
                            	& {\g} \\
                            	{\frakLie(A_i)} && {\frakLie(A)}
                            	\arrow[from=2-1, to=2-3, dashed]
                            	\arrow["{}"', from=1-2, to=2-1]
                            	\arrow["{}", from=1-2, to=2-3]
                            \end{tikzcd}
                        $$
                    (we actually do not need to worry about the size of $I$, since we have already fixed a sufficiently large Grothendieck universe). To that end, note that because $\calV$ has all countable coproducts, one can always construct tensor algebras, whose universal property implies that there is the following adjunction:
                        $$
                            \begin{tikzcd}
                            	{(T \ladjoint \oblv): \Assoc\Alg(\calV)} & {\calV}
                            	\arrow["{\oblv}"{name=0, swap}, from=1-1, to=1-2, shift right=2]
                            	\arrow["{T}"{name=1, swap}, from=1-2, to=1-1, shift right=2]
                            	\arrow["\dashv"{rotate=-90}, from=1, to=0, phantom]
                            \end{tikzcd}
                        $$
                    In particular, this means that for each $A \in \Assoc\Alg(\calV)$ and each Lie algebra $\g$, there is a canonical morphism from $T(\g)$ into $A$. At the same time, note that because tensor algebras are constructed as coproducts of tensor powers, one has canonical inclusions of the tensor powers $\g^{\tensor n}$ into $T(\g)$. Thus, there are commutative diagrams in $\calV$ as follows for each $A \in \Assoc\Alg(\calV)$ and each $n \in \N$:
                        $$
                            \begin{tikzcd}
                            	{\g^{\tensor n}} \\
                            	& {T(\g)} & {A}
                            	\arrow[from=1-1, to=2-2]
                            	\arrow[from=2-2, to=2-3]
                            	\arrow[from=1-1, to=2-3]
                            \end{tikzcd}
                        $$
                    Thus, for each associative and unital algebra $A$, there is a universal morphism in $\Env(\g)$ as follows:
                        $$
                            \begin{tikzcd}
                            	& {\g} \\
                            	{\frakLie\left(T(\g)\right)} && {\frakLie(A)}
                            	\arrow[from=2-1, to=2-3]
                            	\arrow[from=1-2, to=2-1]
                            	\arrow[from=1-2, to=2-3]
                            \end{tikzcd}
                        $$
                    proving the existence of a small index set $I$ (the singleton in this instance) and a family of objects $\{A_i\}_{i \in I}$ of $\Assoc\Alg(\calV)$ such that there are factorisations as below for all $i \in I$:
                        $$
                            \begin{tikzcd}
                            	& {\g} \\
                            	{\frakLie(A_i)} && {\frakLie(A)}
                            	\arrow[from=2-1, to=2-3, dashed]
                            	\arrow["{}"', from=1-2, to=2-1]
                            	\arrow["{}", from=1-2, to=2-3]
                            \end{tikzcd}
                        $$
                    (for the sake of clarity, let us note that here, we take $A_i = T(\g)$ for all $i \in I$). Lastly, note that the category $\Assoc\Alg(\calV)$ is complete and locally small. Thus, all the conditions in Freyd's Adjoint Functor Theorem are satisfied, and therefore, $\frakLie$ is a right-adjoint. This of course means that we can construct a functor:
                        $$\U: \Lie\Alg(\calV) \to \Assoc\Alg(\calV)$$
                    to be left-adjoint to $\frakLie$. Then, as a consequence of the universal property of adjoint pairs, the category of enveloping algebras of $\g$ must have $\frakLie\left(\U(\g)\right)$ as an initial object, which by definition is the so-called universal enveloping algebra of $\g$.
                    \item This is trivial if we note that taking the quotient of $T(\g)$ by the equivalence relation generated by the image of $[-,-]_{\frakLie T(\g)} - \left(\nabla_{T(\g)} - \nabla_{T(\g)} \circ \tau_{T(\g), T(\g)}\right)$ (which incidentally, also canonically endows $T(\g)$ with a Lie bracket) is just to ensure that the canonical map:
                        $$\g \to (\frakLie \circ U \circ \frakLie \circ T)(\g)$$
                    is a Lie algebra homomorphism for every $\g$.
                \end{enumerate}
            \end{proof}
        \begin{corollary}
            $\frakLie$ is left-exact and $\U$ is right-exact. In particular, we have:
                $$\U\left(\g \oplus \g'\right) \cong \U(\g) \tensor \U(\g')$$
            for any pair $\g, \g'$ of Lie algebras. 
        \end{corollary}
        \begin{theorem}[Embedding of Lie algebras into their universal enveloping algebras] \label{theorem: embedding_lie_algebras_into_their_universal_enveloping_algebras}
            Let $k$ be a ring and let $(\calV, \tensor, 1, \tau)$ be a symmetric monoidal $k$-linear abelian category. Also, suppose that $\g$ is a (faithfully ?) flat Lie algebra object internal to $\calV$, i.e. that the functor $\g \tensor -$ is (faithfully ?) flat. Then, there is a canonical monomorphism embedding $\g$ into $\frakLie \U(\g)$ as a subobject.
        \end{theorem}
            \begin{proof}
                
            \end{proof}
        \begin{corollary}[Universal enveloping algebras are bialgebras] \label{coro: universal_enveloping_algebras_are_bialgebras}
            Let $k$ be a ring and let $\g$ be a Lie algebra internal to some $k$-linear symmetric monoidal category $(\V, \tensor, \1)$. The we can endow the universal enveloping algebra $\U(\g)$ with the structure of a Hopf algebra with comultiplication given by 
        \end{corollary}
        
        \begin{example}[The abelian case]
            If $\g$ is an abelian Lie algebra (i.e. one whose Lie bracket is just the zero morphism), then as a direct consequence of the definition of symmetric algebras and of theorem \ref{theorem: universal_enveloping_algebras_universal_property}, we have the following isomorphism of associative and unital algebras:
                $$\U(\g) \cong \Sym(\g)$$
        \end{example}
        \begin{example}[The Poincar\'e-Birkhoff-Witt Theorem]
            Let $k$ be a commutative and unital $\Q$-algebra, let $(\calV, \tensor, 1, \tau)$ be a symmetric monoidal $k$-linear category with all countable coproducts and cokernels, and let $\g$ be a Lie algebra object of $\calV$ that is (faithfully ?) flat (i.e. one such that the functor $\g \tensor -$ is left-exact). Then, there is the following isomorphism of cocommutative $\N$-graded $k$-bialgebras:
                $$\U(\g) \cong \Sym(\g)$$
            In other words, one can understand representations of such a Lie algebra $\g$ via representations of the symmetric algebra $\Sym(\g)$, which in turn are just representations of symmetric groups.
        \end{example}
        \begin{example}[A counter-example]
            Let $k$ be a field of characteristic $0$ and consider the symmetric monoidal category of finite-dimensional $k$-vector spaces. Then clearly, $\calV$ does not have all countable coproducts (for example, the vector space $k^{\oplus \aleph_0}$, which is a coproduct indexed by the countable infinite cardinal $\aleph_0$, is not an object as it is infinite-dimensional), and thus not all Lie $k$-algebras have a universal enveloping algebra.
        \end{example}
        \begin{example}[Open problem]
            It is not known if the functor:
                $$\U: \frakLie(\calV) \to \Lie\Alg(\calV)$$
            is faithful, and even if that is not the case in general, we also do not have a good understanding of which extra assumptions to impose on our ambient symmetric monoidal linear category $\calV$ so that in such a setting, $\U$ might be faithful. 
        \end{example}
        
        \begin{lemma}[Tannaka duality] \label{lemma: tannaka_duality}
            Let $\calV$ be a closed symmetric monoidal category that is locally small (such as the symmetric monoidal category of vector spaces over a field), so that it may be enriched over itself via its internal homs (which exist thanks to the monoidal closure assumption), and let $A$ be a monoid object of $\calV$. If we denote the category of $\calV$-representations on $A$ by:
                $$\Rep_{\calV}(A) := \calV\-\Cat(\rmB A^{\op}, \calV)$$
            then the algebra $\End_{\calV\-\Cat(\Rep_{\calV}(A), \calV)}(F)$ of $\calV$-natural endomorphisms on the canonical forgetful functor $F: \Rep_{\calV}(A) \to \calV$ is isomorphic to $A$.
        \end{lemma}
            \begin{proof}
                Apply the $\calV$-enriched Yoneda's lemma:
                    $$
                        \begin{aligned}
                            \End_{\calV\-\Cat(\Rep_{\calV}(A), \calV)}(F) & \cong \calV\-\Cat(\Rep_{\calV}(A), \calV)(F, F)
                            \\
                            & \cong \Psh_{\calV}\left(\Psh_{\calV}(\rmB A)\right)\bigg(\Psh_{\calV}(\rmB A)(*, -), \Psh_{\calV}(\rmB A)(*, -)\bigg)
                            \\
                            & \cong \Psh_{\calV}(\rmB A)(*, *)
                            \\
                            & \cong \Rep_{\calV}(A)(*, *)
                            \\
                            & \cong \End_{\calV}(A)
                            \\
                            & \cong A
                        \end{aligned}
                    $$
            \end{proof}
        \begin{theorem}
            If $k$ is a ring, $\calV$ is a \textit{locally small} \textit{closed} symmetric monoidal $k$-linear category with all coproducts and cokernels, and $\g$ is a (faithfully ?) flat Lie algebra object of $\calV$, then there is an equivalence of abelian monoidal $k$-linear categories as follows:
                $$\Rep_{\calV}(\g) \cong {\U(\g)}\mod$$
            wherein $\Rep_{\calV}(\g)$, the category of representations of $\g$ on $\calV$, is the category whose objects are Lie algebra homomorphisms from $\g$ to $\gl(V)$ (with $\gl(V)$ the canonical Lie algebra associated to the associative and unital endomorphism algebra $\End_{\calV}(V)$), and morphisms are commutative triangles in $\Lie\Alg(\calV)$ of the form:
                $$
                    \begin{tikzcd}
                    	& {\g} \\
                    	{\gl(V)} && {\gl(V')}
                    	\arrow[from=2-1, to=2-3]
                    	\arrow[from=1-2, to=2-1]
                    	\arrow[from=1-2, to=2-3]
                    \end{tikzcd}
                $$
            Also, note that ${\U(\g)}\mod$ is \href{https://ncatlab.org/nlab/show/module+over+a+monoid}{\underline{well-defined}} as the category of $\U(\g)$-left-equivariant objects of $\calV$, and it exhibits all properties that one might expect of a module category.
        \end{theorem}
            \begin{proof}
                Before we prove this claim, let us first note that because $\calV$ has all countable coproducts and cokernels, all Lie algebras possess universal enveloping algebras. With that out of the way, let us note that by Tannaka duality (cf. lemma \ref{lemma: tannaka_duality}), each left-$\U(\g)$-module is actually just a $\U(\g)$-representation. Thus, to show that:
                    $$\Rep_{\calV}(\g) \cong {\U(\g)}\mod$$
                it will suffice to show that:
                    $$\Rep_{\calV}(\g) \cong \Rep_{\calV}\left(\U(\g)\right)$$
                (this might seem like a round-about method, but without Tannaka duality, guaranteeing that the equivalence is actually between abelian monoidal $k$-linear categories instead of just between ordinary categories will be difficult). In turn, one can do this via showing that there is the following equivalence of categories of Lie algebra representations:
                    $$\Rep_{\calV}(\g) \cong \Rep_{\calV}\left(\frakLie \U(\g)\right)$$
                thanks to the uniqueness of the universal enveloping algebra. Then, we can simply apply lemma 1.1.1, which states that $\g$ is a subobject of $\frakLie \U(\g)$, and consider commutative diagrams as follows in $\Lie\Alg(\calV)$:
                    $$
                        \begin{tikzcd}
                        	{\g} \\
                        	& {\gl(V)} \\
                        	{\frakLie \U(\g)}
                        	\arrow[from=1-1, to=2-2]
                        	\arrow[from=3-1, to=2-2]
                        	\arrow[from=1-1, to=3-1, tail]
                        \end{tikzcd}
                    $$
                to show that for each representation of $\g$ on $V \in \calV$, there is a representation of $\frakLie \U(\g)$ on $V$ as well, and vice versa.
            \end{proof}
            
    \subsection{Structure of Lie algebras}
        \begin{convention}
            From now on, we work within a fixed $k$-linear tensor category $\calV$ (i.e. a $k$-linear abelian dualisable symmetric monoidal category). For all intents and purposes, one might think of $\calV$ as being $k\mod$.
        \end{convention}
        
        We begin by proving that the category $\Lie\Alg(\calV)$ has zero objects, kernels, and cokernels.
        \begin{proposition}[The zero Lie algebra] \label{prop: zero_lie_algebra}
            The zero object $0 \in \calV$ is also the zero object in $\Lie\Alg(\calV)$.
        \end{proposition}
            \begin{proof}
                One simply has to note that there is only one unique morphism from $0$ to any Lie algebra $\g$, and likewise, from any Lie algebra $\g$ to $0$. 
            \end{proof}
            
        \begin{proposition}[(Co)kernels of Lie algebra homomorphisms] \label{prop: (co)kernels_of_lie_algebra_homomorphisms}
            For any Lie algebra homomorphism $\phi: \g \to \h$, the following pullback and pushout exist in $\Lie\Alg(\calV)$:
                $$
                    \begin{tikzcd}
                    	{\ker \phi} & 0 \\
                    	\g & \h
                    	\arrow["\phi", from=2-1, to=2-2]
                    	\arrow[from=1-1, to=2-1]
                    	\arrow[from=1-1, to=1-2]
                    	\arrow[from=1-2, to=2-2]
                    \end{tikzcd}
                $$
                $$
                    \begin{tikzcd}
                    	\g & \h \\
                    	0 & {\coker \phi}
                    	\arrow["\phi", from=1-1, to=1-2]
                    	\arrow[from=1-1, to=2-1]
                    	\arrow[from=2-1, to=2-2]
                    	\arrow[from=1-2, to=2-2]
                    	\arrow["\lrcorner"{anchor=center, pos=0.125, rotate=180}, draw=none, from=2-2, to=1-1]
                    \end{tikzcd}
                $$
        \end{proposition}
            \begin{proof}
                We can embed $\calV$ into a category of modules over a commutative ring, \textit{\`a la} Freyd-Mitchell, so for the purposes of this proof, we might as well take $\calV \cong k\mod$. Our strategy is then to compute $\ker \phi$ and $\coker \phi$ in $k\mod$, and then verify that they have natural Lie algebra structures. 
                
                To this end, consider first of all the folloiwng:
                    $$\ker \phi := \{x \in \g \mid \phi(x) = 0\}$$
                Since $\ker \phi$ is a $k$-submodule of $\g$, we claim that the Lie bracket on $\ker \phi$ is precisely that of $\g$. To verify that this is true, simply consider:
                    $$\phi([x, y]) = [\phi(x), \phi(y)] = [0, 0] = 0$$
                wherein $x, y \in \g$ are arbitrary elements. 
                
                Now, consider:
                    $$\coker \phi := \h/\im\phi$$
            \end{proof}
        
        Now, let us define Lie subalgebras and Lie ideals. 
        \begin{definition}[Subalgebras and ideals] \label{def: lie_subalgebras_and_lie_ideals}
            Let $\g$ be a Lie algebra internal to $\calV$. 
                \begin{enumerate}
                    \item \textbf{(Subalgebras):} A \textbf{Lie subalgebra} of $\g$ is a monomorphism of Lie algebras\footnote{Note that this need not be a kernel, because only \say{$k$-linear} morphisms in the ambient tensor category $\calV$ are kernel if and only if they are monomorphisms; the same need not be true for Lie algebra homomorphisms.}.
                    \item \textbf{(Ideals):} A \textbf{Lie algebra ideal} is not just a monomorphism, but futhermore, the kernel of a Lie algebra homomorphism (i.e. it is a \textit{normal} monomorphism\footnote{Note how this is in perfect analogy with normal subgroups, which are precisely kernels of group homomorphisms.}). 
                \end{enumerate}
        \end{definition}
        \begin{definition}[Commutants] \label{def: commutants}
            The \textbf{commutant} of a Lie algebra $\g$ is the ideal $[\g, \g]$. A Lie algebra is said to be \textbf{abelian} if and only if its commutant is zero.
        \end{definition}
        \begin{remark}
            One can easily verify that not only is the quotient $\g/[\g, \g]$ abelian but furthermore, that if $I \subseteq \g$ be an ideal such that $\g/I$ is abelian then $I$ will be a Lie sub-ideal of $[\g, \g]$. 
        \end{remark}
        \begin{example}
            For every $n \geq 1$ and $k$ a field\footnote{In fact, this holds whenever $k$ is a PID and $n \geq 3$, when $k$ is PID of characteristic $\not = 2$ and $n = 2$, but not when $n = 1$ (because $\gl_1(k)$ is abelian for any commutative ring $k$).}, the commutant of $\gl_n(k)$ is $\sl_n(k)$, the Lie algebra of $n \x n$ matrices with coefficients in $k$ and trace zero. To see why this is the case, observe that since any element of $[\gl_n(k), \gl_n(k)]$ takes the form $T := AB - BA$ for some $A, B \in \gl_n(k)$, and since trace is additive and invariant under cyclic permutations, one has:
                $$\trace(T) = \trace(AB - BA) = \trace(AB) - \trace(AB) = 0$$
            for all $T \in [\gl_n(k), \gl_n(k)]$; from this, one gathers that $[\gl_n(k), \gl_n(k)]$ is a subset of $\sl_n(k)$. Conversely, in order to show that any trace-zero $n \x n$ matrix is necessary of the form $AB - BA$ for some $A, B \in \gl_n(k)$, 
        \end{example}
    
    \subsection{Semi-simplicity and reductivity}