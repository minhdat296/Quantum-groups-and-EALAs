\section{Basic representation theory}
    \subsection{(Semi-)simplicity and (in)decomposability}
        \subsubsection{(Semi)-simplicity and the Jordan-H\"older Theorem; (locally) finite (Artinian) abelian categories}
            \begin{definition}[(Semi-)simple objects] \label{def: (semi)_simple_objects}
                Let $\C$ be a category with enough monomorphisms and epimorphisms and zero objects. A \textit{non-zero} object $X \in \Ob(\C)$ is said to be \textbf{simple} if and only if any monomorphism $\iota: X' \to X$ and any epimorphism $\pi: X \to X''$ is either zero (i.e. it factors through a zero object) or the identity $\id_X$. If $\C$ additionally has enough direct sums, and if a (non-zero) object $X$ admits a direct sum decomposition:
                    $$X \cong \bigoplus_{i \in I} X_i$$
                into simple sub-objects $X_i \subseteq X$, then $X$ will be said to be \textbf{semi-simple}; a category wherein all objects are semi-simple is itself called \textbf{semi-simple}.
            \end{definition}
            \begin{remark}[(Semi-)simple objects in abelian categories]
                If $\calA$ is an abelian category, then because $f: M \to N$ is a kernel (respectively, a cokernel) if and only if it is a monomorphism (respectively, an epimorphism), one might characterise simple objects of $\calA$ as those such that the kernel of any morphism:
                    $$\iota: M' \to M$$
                and and the cokernel of any morphism:
                    $$\pi: M \to M''$$
                are identically zero. 
            \end{remark}
            \begin{lemma}[Schur's Lemma] \label{lemma: schur_lemma}
                Let $\calA$ be an abelian category and $f: M \to N$ be a non-zero morphism between two simple objects $M$ and $N$. Such a morphism must then be an isomorphism.
            \end{lemma}
                \begin{proof}
                    Since both $M$ and $N$ are simple objects of $\calA$ and since $\calA$ is abelian, the lemma comes from consideration of the following commutative diagram and from the assumption that $f: M \to N$ is a non-zero morphism:
                        $$
                            \begin{tikzcd}
                            	0 & M & 0 \\
                            	0 & N & 0
                            	\arrow["f", from=1-2, to=2-2]
                            	\arrow[from=1-1, to=2-1]
                            	\arrow[from=2-1, to=2-2]
                            	\arrow["{\ker f}", from=1-1, to=1-2]
                            	\arrow[from=1-2, to=1-3]
                            	\arrow["{\coker f}", from=2-2, to=2-3]
                            	\arrow[from=1-3, to=2-3]
                            	\arrow["\lrcorner"{anchor=center, pos=0.125, rotate=180}, draw=none, from=2-3, to=1-2]
                            	\arrow["\lrcorner"{anchor=center, pos=0.125}, draw=none, from=1-1, to=2-2]
                            \end{tikzcd}
                        $$
                \end{proof}
            \begin{corollary}[Endomorphism algebras of simple objects] \label{coro: endomorphisms_algebras_of_simple_objects}
                Let $\calA$ be an abelian category and let $M$ be a simple object. Then $\calA(M, M)$ will be a division ring (with respect to compositions of endomorphisms on $M$), and if $N \not \cong M$ is another simply object, then $\calA(M, N) \cong 0$. 
            \end{corollary}
                \begin{proof}
                    Both assertions are direct consequences of lemma \ref{lemma: schur_lemma}.
                \end{proof}
            \begin{example}[Irreducible complex representations]
                Let $k$ be a commutative ring, let $A$ be an associative $k$-algebra (e.g. one might consider the group algebra $A := k\<G\>$ of some abstract group $G$), and let $V$ and $W$ be two irreducible $k$-linear representations of $A$, viewed as simple (left-)$A$-modules. Then, any $A$-linear map $f: M \to N$ (or equivalently, any map of $k$-linear representations of $A$) will either be zero or an isomorphism; when $k$ is an algebraically closed field of characteristic $0$ (e.g. $k \cong \bbC$), such an isomorphism will also have to be given by (left-)multiplication by a scalar $a \in k$ (the scalar $1_k$ corresponds to the identity map).
            \end{example}
                
            \begin{definition}[Lengths and Jordan-H\"older series] \label{def: lengths_of_objects_and_jordan_holder_series}
                Let $\C$ be a category with enough kernels and cokernels. An object $X \in \Ob(\C)$ is then said to be of \textbf{length} $n$ (for some $n \in \N$) if and only if there exists a so-called \textbf{Jordan-H\"older filtration} of \textit{normal} monomorphisms:
                    $$0 =: X_0 \subseteq X_1 \subseteq X_2 \subseteq ... \subseteq X_n \subseteq X$$
                such that the quotients (i.e. cokernels) $X_{i + 1}/X_i$ are simple for all $0 \leq i \leq n - 1$; these quotients are commonly called \textbf{Jordan-H\"older factors}. Such a filtration is said to be of \textbf{multiplicity} $m \geq 0$ if and only if the number of isomorphic factors is $m$. 
            \end{definition}
            \begin{convention}
                Zero objects are of length $0$.
            \end{convention}
            \begin{theorem}[Uniqueness of Jordan-H\"older filtrations] \label{theorem: jordan_holder_theorem}
                Let $\C$ be a category with kernels and cokernels; assume also that $\C$ has enough cokernels. Next, consider an object $X \in \Ob(\C)$ that is of some finite length $n \geq 0$. Then, any Jordan-H\"older filtration of $X$ must be of length $n$.
            \end{theorem}
                \begin{proof}
                    Let $X_{\bullet} := \{X_i\}_{0 \leq i \leq n}$ and $Y_{\bullet} := \{Y_j\}_{0 \leq j \leq m}$ be two Jordan-H\"older filtrations of $X$ and consider some $f_{\bullet} \in \C^{\N}(X_{\bullet}, Y_{\bullet})$. Taking cokernels is functorial, so let us consider the resulting map between factors $\coker f_{\bullet} \in \C^{\N}(X_{\bullet + 1}/X_{\bullet}, Y_{\bullet + 1}/Y_{\bullet})$. By Schur's Lemma (cf. lemma \ref{lemma: schur_lemma}), this is either zero or an isomorphism term-wise, which tells us that $m = n$. 
                \end{proof}
            \begin{example}
                The Jordan-H\"older theorem holds in any semi-abelian category (such as that of groups) and more particularly, any abelian category (such as categories of modules over rings or categories of linear representations).
            \end{example}
        
        \subsubsection{Decomposability and the Krull-Schmidt Theorem}
            \begin{definition}[(In)decomposable objects] \label{def: (in)decomposable_objects}
                An object $X$ of a category $\C$ with enough products is said to be \textbf{decomposable} if there exists a family $\{X_i\}_{i \in I}$ of sub-objects $X_i \subseteq X$ such that:
                    $$X \cong \prod_{i \in I} X_i$$
                Otherwise, $X$ is said to be \textbf{indecomposable}.
            \end{definition}
            \begin{remark}
                Semi-simple objects are decomposable, while simple objects are always indecomposable. There can - however - be indecomposable objects which are not simple (e.g. any non-field commutative ring as a module over itself). 
            \end{remark}
            \begin{definition}[Local associative algebras] \label{def: local_associative_algebras}
                
            \end{definition}
            \begin{proposition}
                Let $\C$ be a category with enough products and let $Z$ be an indecomposable object. Then $\C(Z, Z)$ 
            \end{proposition}
                \begin{proof}
                    
                \end{proof}
            \begin{definition}[Krull-Schmidt categories] \label{def: krull_schmidt_categories}
                Let $k$ be a commutative ring. A \textbf{$k$-linear Krull-Schmidt category} is a category is a $k$-linear category in which every object is decomposable into \textit{finitely many} indecomposable factors.
            \end{definition}
            
            \begin{theorem}[Krull-Schmidt Decomposition Theorem] \label{theorem: krull_schmidt_theorem}
                Suppose that $\C$ is a category with enough (co)kernels and products, and that $X \in \Ob(\C)$ is an object of finite length. Then, there exists a (necessarily finite) product decomposition:
                    $$X \cong \prod_{i \in I} X_i$$
                of $X$ into indecomposable sub-objects $X_i \subseteq X$, which is unique up to isomorphisms.
            \end{theorem}
                \begin{proof}
                    Suppose to the contrary that there exists a finite-length object $X \in \Ob(\C)$ which does not admit a (finite) product decomposition into indecomposable sub-objects. Particularly, this means that $X$ is not indecomposable (because if so $X$ would just be its own unary product decomposition into indecomposable subobjects), meaning that by definition, $X$ is decomposable. Let:
                        $$X \cong \prod_{j \in J^{(1)}} Z_j^{(1)}$$
                    be a (necessarily finite) product decomposition of $X$ into sub-objects $Z_j^{(1)} \subseteq X$; the sub-objects $Z_j^{(1)}$ can not be simultaneously indecomposable, so there must exist at least one $j_1 \in J^{(1)}$ such that $Z_{j_1}^{(1)}$ is decomposable. By repeating this argument, one obtains the a (finite) product decomposition of $X$ as follows:
                        $$X \cong \left( \prod_{j \in J^{(1)} \setminus j_1} Z_j^{(1)} \right) \x Z_{j_1}^{(1)} \cong \left( \prod_{j \in J^{(1)} \setminus j_1} Z_j^{(1)} \right) \x \left( \prod_{j \in J^{(2)} \setminus j_2} Z_j^{(2)} \right) \x Z_{j_2}^{(2)} \cong ...$$
                    which yields the following \textit{infinite} filtration on $X$:
                        $$X \supseteq Z_{j_1}^{(1)} \supseteq Z_{j_2}^{(2)} \supseteq ...$$
                    This contradicts the hypothesis that $X$ is of finite length, and therefore $X$ is decomposable into finitely many indecomposable factors.
                \end{proof}
            \begin{example}
                The Krull-Schmidt Decomposition Theorem is, particularly, applicable within the category of groups and any category of modules over an arbitrarily given associative ring. More specifically (and perhaps more pertinent to our needs), the theorem is holds within categories of representations (e.g. of groups, Lie algebras, Hopf algebras, and even of monoids\footnote{Observe that all such objects admit universally defined }). 
            \end{example}
            \begin{example}[Cyclic decomposition of finitely generated modules over PIDs]
                An enhanced specialisation of the Krull-Schmidt Decomposition Theorem is the Cyclic Decomposition Theorem for finitely generated modules over PIDs $R$. Such modules are certainly of finite length (which are equal to the cardinalities of their generating sets) and as such - per the Krull-Schmidt Decomposition Theorem - admits a finite product (in fact, direct sum) decomposition into indecomposable modules. The additional lemma to prove for the Cyclic Decomposition Theorem is that over PIDs, indecomposable modules are always cyclic\footnote{Though the converse is not necessarily true (e.g. $\Z/6\Z \cong \Z/2\Z \oplus \Z/3\Z$ as a $\Z$-module).}; more particularly, one has:
                    \begin{enumerate}
                        \item a finitely generated module over a given PID is torsion-free if and only if it is free, and 
                        \item finitely generated torsion modules over PIDs $R$ can be decomposed as:
                            $$\bigoplus_{\p \in \Spec R} (R/\p)^{\oplus e_{\p}}$$
                        for some finitely supported set of powers $\{e_{\p} \in \N \mid \p \in \Spec R\}$.
                    \end{enumerate}
                
                A trivial yet interesting corollary of the Cyclic Decomposition Theorem is that over a given PID $R$, the canonical short exact sequence:
                    $$0 \to \Tor_R(M) \to M \to M/\Tor_R(M) \to 0$$
                splits for any finitely generated $R$-module $M$ (here, $\Tor_R(M)$ denotes the torsion $R$-submodules of $M$).
                
                In turn, the Cyclic Decomposition Theorem admits its own special case, that being the Chinese Remainder Theorem: given any integer $N$ and a fixed prime factorsiation $N := \prod_{i \in I} p_i^{e_i}$ thereof, one has an isomorphism of abelian groups as follows:
                    $$\Z/N\Z \cong \bigoplus_{i \in I} (\Z/p_i\Z)^{e_i}$$
            \end{example}
        
    \subsection{Finite algebras and representations of finite groups}