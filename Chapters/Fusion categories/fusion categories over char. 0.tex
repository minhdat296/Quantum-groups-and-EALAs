\section{Fusion categories over characteristic \texorpdfstring{$0$}{}}
    \begin{convention}
        Throughout the section, we work over a fixed algebraically closed field $k$ of characteristic $0$.
    \end{convention}

    \subsection{Structure and dimension theories for fusion categories}
        \subsubsection{Definition, global dimensions, and Oceanu Rigidity}
            \begin{definition}[Fusion categories] \label{def: fusion_categories}
                A \textbf{(multi-)fusion category} over $k$ is a finite semi-simple $k$-linear (multi-)tensor category $(\calV, \tensor, \1)$ with\footnote{Note that this condition is redundant for fusion categories, since we already have $\End_{\calV}(\1) \cong k$ when $(\calV, \tensor, \1)$ is a tensor category.} $\End_{\calV}(\1) \cong k$.
            \end{definition}
            \begin{remark}[Semi-simplicity of monoidal units of multi-fusion categories] \label{remark: semi_simplicity_of_monoidal_units_of_multi_fusion_categories}
                Suppose that $(\calV, \tensor, \1)$ is a multi-fusion category over $k$ and consider the monoidal unit $\1$. Then, it is easy to see that $\1$ is necessarily semi-simple. 
            \end{remark}
            
        \subsubsection{Duality}
        
        \subsubsection{Pseudo-unitary fusion categories}
        
        \subsubsection{Integral and weakly integral fusion categories}

    \subsection{Tannakian categories and Deligne's theorem}
    
    \subsection{Group-theoretic fusion categories}