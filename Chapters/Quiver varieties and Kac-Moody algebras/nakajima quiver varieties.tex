\section{Nakajima quiver varieties}
    \subsection{\textit{Pr\'elude}: \texorpdfstring{$\Quot$}{} and \texorpdfstring{$\Hilb$}{} schemes}

    \subsection{GIT quotients parametrising quiver representations}
        \begin{convention}
            Throughout this subsection, we work over a ground field $k$. All schemes, unless stated to be otherwise, shall be over $k$.
        \end{convention}
        \begin{convention}
            If $K$ is a field then by $\bar{K}$ we will usually mean the algebraic closure of $K$. When there are reasons to confuse $\bar{K}$ for the separable closure of $K$, then we shall write $K^{\alg}$ and $K^{\sep}$ instead. 
        \end{convention}
        
        \subsubsection{A moduli stack of finite-dimensional quiver representations}
            Let us begin this section by introducing so-called \say{quiver varieties}. These are algebraic varieties representing fine moduli spaces of certain algebraic stacks classifying representations of finite quivers $Q$ of a prescribed dimension vector $v \in \N^{Q_0}$. Traditionally, these varieties were usually realised as GIT quotients of certain naturally arising affine schemes, but given the fragile nature of GIT constructions, and also because we would like to put emphasis on the fact that these quiver varieties parametrise quiver representations, the stacky point of view seems more appropriate. 
            \begin{definition}[Quiver representation stacks] \label{def: quiver_representation_stacks}
                Let $Q := (Q_1, Q_0, s, t)$ be a finite quiver. Let us then define the \textbf{prestack of finite-dimensional $k$-linear representations of $Q$} as the fibration
                    $$\pi_Q^{\fin}: \scrR_Q^{\fin} \to (\Sch_{/\Spec k})_{\fppf}$$
                whose fibre over each $k$-scheme $S \in \Ob((\Sch_{/\Spec k})_{\fppf})$ is the category:
                    $$\Rep_{\calO_S}^{\fin}(Q) := \Func([Q], \Vect(S_{\fppf}))^{\circ}$$
                of representations of $Q$ valued in vector fppf-bundles on $S$ (i.e. compact projective objects of $\QCoh(S_{\fppf})$).  
            \end{definition}
            \begin{proposition}
                Let $Q := (Q_1, Q_0, s, t)$ be a finite quiver. Then, the fibration:
                    $$\pi_Q^{\fin}: \scrR_Q^{\fin} \to (\Sch_{/\Spec k})_{\fppf}$$
                as described in definition \ref{def: quiver_representation_stacks} is a stack fibred in categories on $(\Sch_{/\Spec k})_{\fppf}$ (i.e. it satisfies fppf descent).
            \end{proposition}
                \begin{proof}
                    
                \end{proof}
                
            \begin{example}[Gabriel's Theorem for $\sfA_n$ quivers]
                
            \end{example}
            \begin{example}[Gabriel's Theorem for $\sfD_n$ quivers]
                
            \end{example}
            \begin{example}[Gabriel's Theorem for $\sfE_n$ quivers]
                
            \end{example}
            \begin{example}[Gabriel's Theorem fails for the loop quiver]
                
            \end{example}
            \begin{example}[Gabriel's Theorem fails for the Kronecker quiver]
                
            \end{example}
            
        \subsubsection{(Semi-)stability of quiver representations}
    
        \subsubsection{Framings}
    
        \subsubsection{Moment maps, symplectic singularities, and Hamiltonian reductions}