\section{(Quantum) vertex algebras}
    \subsection{The classical approach to vertex operator algebras}
        \begin{convention}
            Throughout this entire subsection, we work over the complex numbers $\bbC$, not only for convenience, but also because we will have to make use of certain analytic constructions such as residues of meromorphic functions.
        \end{convention}
    
        \subsubsection{The definition of a vertex algebra}
            \begin{definition}[Formal distributions] \label{def: formal_distributions}
                By a \textbf{formal distribution} in the (complex) variables $z_1, ..., z_n$ over an associative $\bbC$-algebra $R$, we shall mean a formal power series:
                    $$A(z_1, ..., z_n) := \sum_{i_1 \in \Z} ... \sum_{i_n \in \Z} A_{i_1, ..., i_n} z_1^{i_1} ... z_n^{i_n}$$
                wherein the coefficients $A_{i_1, ..., i_n}$ are elements of $R$. It is easy to see that such formal distributions form a $\bbC$-vector space\footnote{In fact, a left/right-$R$-module.}, which we shall denote by $R[\![z_1^{\pm 1}, ..., z_n^{\pm 1}]\!]$.  
            \end{definition}
            \begin{remark}[Multiplying formal distributions ?] \label{remark: multiplying_formal_distributions}
                Let $R$ be an associative $\bbC$-algebra. Unlike for say, $R[\![z_1, ..., z_n]\!]$ or even $R(\!(z_1, ..., z_n)\!)$, the \textit{na\"ive} multiplication operation turns out to not be well-defined on $R[\![z_1^{\pm 1}, ..., z_n^{\pm 1}]\!]$. To see why, consider two arbitrary formal distributions:
                    $$A(z_1, ..., z_n), B(z_1, ..., z_n) \in R[\![z_1^{\pm 1}, ..., z_n^{\pm 1}]\!]$$
                and their \say{product}, taken in the obvious way:
                    $$A(z_1, ..., z_n) B(z_1, ..., z_n) = \sum_{i_1 \in \Z} ... \sum_{i_n \in \Z} A_{i_1, ..., i_n} \left(\sum_{j_1 \in \Z} ... \sum_{j_n \in \Z} B_{j_1, ..., j_n}\right) z_1^{i_1 + j_1} ... z_n^{i_n + j_n}$$
                will include coefficient terms which are infinite sums of elements of $R$, which are definitely not well-defined in general. 
                
                Note, however, that the \textit{na\"ive} product of a formal distribution with a formal power series or even a Laurent series is always well-defined. 
            \end{remark}
            
        
        \subsubsection{Vertex operators algebras from Lie algebras}
        
        \subsubsection{Operator product expansions (OPEs)}
        
        \subsubsection{Reperesentations of vertex operator algebras}
    
    \subsection{Vertex algebras, factorisation algebras, and chiral algebras}
        \subsubsection{Vertex operator algebras as "singular" commutative algebras}
            \begin{convention}
                Let us now fix a commutative ring $k$ along with a $k$-linear infinite\footnote{This means that we do not require that $\calA$ is locally finite as a $k$-linear category, unlike in \cite[Definition 4.1.1]{EGNO}. Perhaps one could either require that $\calA$ is locally Noetherian or Artinian as a $k$-linear category (that is to say, that the hom-spaces are Noetherian or Artinian $k$-modules, respectively), but we are not certain that this is the weakest finiteness assumption that one could impose upon $\calA$.} tensor category:
                    $$(\calA, \tensor, \1)$$
                (this means that $\E$ is a $k$-linear abelian rigid monoidal category such that firstly, the monoidal structure $\tensor: \calA \x \calA \to \calA$ is $k$-bilinear on hom-spaces and secondly, that $\calA(\1, \1) \cong k$) wherein colimits are flat. An example of such a category is the category $k\mod$ of $k$-modules. 
            \end{convention}
            \begin{remark}
                It is easy (via object-wise considerations of tensor products and hom-spaces, respectively) to see that the functor category\footnote{Here, $\Fin\Sets_{\inj}$ is understood to be equipped with the canonical monoidal structure given by disjoint unions (the monoidal unit is $\varnothing$).}:
                    $$\Mon\Func(\Fin\Sets_{\inj}^{\op}, \calA)$$
                are symmetric monoidal and $k$-linear\footnote{We shall explain later why we care about this category in particular.}. In particular, this means that it makes sense to consider coalgebras internal to either of these categories; specifically, we want to consider the monoidal functor\footnote{We shall let our dear readers try to convince themselves that $H^{\tensor (-)}$ is indeed functorial. It is certainly monoidal.}:
                    $$H^{\tensor (-)}: \Fin\Sets_{\inj}^{\op} \to \Comm\co\Comm\Alg(\calA)$$
                    $$I \mapsto H^{\tensor I}$$
                where:
                    $$H \in \Ob(\Comm\co\Comm\Alg(\calA))$$
                is a fixed commutative and cocommutative bialgebra object of $\calA$, and note that:
                    $$H^{\tensor (-)} \in \Ob(\Comm\co\Comm\Alg(\Mon\Func(\Fin\Sets_{\inj}^{\op}, \calA)))$$
                precisely because $H$ has been chosen to be a commutative and cocommutative bialgebra object of $\calA$. 
                
                If we make the further assumption that $\calA$ is monoidal-closed, then we can consider actions of $H^{\tensor (-)}$ on objects $F \in \Ob(\Mon\Func(\Fin\Sets_{\inj}^{\op}, \calA))$: specifically, these are associative algebra\footnote{In the sense of associative algebras internal to $\Mon\Func(\Fin\Sets_{\inj}^{\op}, \calA)$} homomorphisms:
                    $$\alpha: H^{\tensor (-)} \to \End_{\Mon\Func(\Fin\Sets_{\inj}^{\op}, \calA)}(F)$$
                The point here is that the category of interest to us is:
                    $$H^{\tensor (-)}\mod := \Mon\Func(\Fin\Sets_{\inj}^{\op}, \calA)^{H^{\tensor (-)}}$$
                i.e. that of $H^{\tensor (-)}$-equivariant monoidal functors from $\Fin\Sets_{\inj}^{\op}$ to $\calA$, i.e. that of $H^{\tensor (-)}$-module internal to $\Mon\Func(\Fin\Sets_{\inj}^{\op}, \calA)$. This category is also $k$-linear and symmetric monoidal, so if we were to consider commutative algebra objects:
                    $$S \in \Ob(H^{\tensor (-)}\-\Comm\Alg)$$
                internal to it (here $H^{\tensor (-)}\-\Comm\Alg := \Comm\Alg(H^{\tensor (-)}\mod)$), we will be able to subsequently consider the category of $S$-modules internal to $H^{\tensor (-)}\mod$. This category, in turn, is also $k$-linear and symmetric monoidal: its monoidal structure is given by:
                    $$\tensor_S: S\mod \x S\mod \to S\mod$$
                    $$(F, G) \mapsto \left(I \mapsto F(I) \tensor_{S(I)} G(I)\right)$$
            \end{remark}
            \begin{example}
                Fix a field $k$ and let $\calA := k\mod$; particularly, note that colimits in this category are indeed flat (every vector space is free, after all). One could consider, for instance, the commutative and cocommutative $k$-bialgebra:
                    $$k[\del] \in \Ob(k\-\Comm\co\Comm\Alg)$$
                (that this is a cocommutative coalgebra internal to $k\mod$ comes from the fact that it is the global section of the additive group $k$-scheme $(\G_a)_k \cong \A^1_k$). The resulting functor $k[\del]^{\tensor (-)}$ then sends $I \in \Ob(\Fin\Sets_{\inj}^{\op})$ to the $k$-vector space $k[\del_I] := k[\{\del_i\}_{i \in I}]$, which is indeed a commutative and cocommutative $k$-bialgebra.
                
                An example of an object $F \in \Ob(k\-\Comm\co\Comm\Alg)$ with a $k[\del]^{\tensor (-)}$-action is the functor:
                    $$k[z^{\pm 1}]^{\tensor (-)}: \Fin\Sets_{\inj}^{\op} \to k\mod$$
                    $$I \mapsto k[\{(z_i - z_j)^{\pm 1}\}_{i, j \in I, i \not = j}]$$
                An explicit example of such an action is the action of $k[\{\del_{z_i}\}_{i \in I}]$ on $k[\{(z_i - z_j)^{\pm 1}\}_{i, j \in I, i \not = j}]$ via partial differentiation. Now, $k[z^{\pm 1}]^{\tensor (-)}$ is actually a commutative $k[\del]^{\tensor (-)}$-algebra as well, so the ultimate category of interest shall be that of $k[z^{\pm 1}]^{\tensor (-)}$-modules internal to the $k$-linear symmetric monoidal category $k[\del]^{\tensor (-)}\-\Comm\Alg$. 
            \end{example}
            
            \begin{convention}[Vertex triples] \label{conv: vertex_triples}
                Let $k$ be a commutative ring, let $\calA$ be a cocomplete $k$-linear infinite tensor category wherein colimits are flat, let $H \in \Ob(\Comm\co\Comm\Alg(\calA))$ be a commutative and cocommutative bialgebra internal to $\calA$, and let $S \in \Ob(H^{\tensor (-)}\-\Comm\Alg)$ be a commutative algebra object over which we shall be considering modules. Whenever we have a triple $(\calA, H, S)$ as above, we will be calling it a \textbf{vertex triple}.
            \end{convention}
            \begin{definition}[Borcherd's singular tensor product] \label{def: singular_tensor_products}
                Let $(\calA, H, S)$ be a vertex triple. One can then define an additional monoidal structure on $S\mod$, which is:
                    $$\singtensor: S\mod \x S\mod \to S\mod$$
                    $$(V_1, V_2) \mapsto \left( I \mapsto \underset{I_1, I_2 \in (\Fin\Sets_{/I})_{\inj}}{\colim} \left(V_1(I_1) \tensor V_2(I_2)\right) \tensor_{S(I_1) \tensor S(I_2)} S(I) \right)$$
            \end{definition}
            \begin{remark}[Borcherd's singular tensor product as a Day convolution]
                
            \end{remark}
            \begin{definition}[$(\calA, H, S)$-vertex algebras] \label{def: (A, H, S)_vertex_algebras}
                Let $(\calA, H, S)$ be a vertex triple. An \textbf{$(\calA, H, S)$-vertex algebra} is then a commutative algebra object internal to the $k$-linear symmetric monoidal category $(S\mod, \singtensor, S)$.
            \end{definition}
            
    \subsection{Vertex algebra bundles and conformal blocks}
    
    \subsection{Quantum vertex operator algebras}