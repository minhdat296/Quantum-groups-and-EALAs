\section{Quivers, hereditary rings, and homological dimension \texorpdfstring{$\leq 1$}{}}
    \subsection{Quivers}
        \subsubsection{Generalities about quivers and their representations}
            We begin by introducing so-called \say{quivers}, which is more-or-less a relaxed version of directed graphs which turn out to be rather important within representation-theoretic contexts. Our approach is that quivers are certain very simple categories, and as such their representations ought to be viewed as particular instances of representations of categories. Specifically, this means we shall be studying the various algebraic properties of quiver representations via modules over their so-called \say{category algebras}, so that notions of say, finite-type-ness of quivers, could be discussed using purely ring-theoretic arguments.
        
            \begin{definition}[Quivers] \label{def: quivers}
                For any category $\C$, a \textbf{$\C$-valued quiver} $Q$ is a diagram in $\C$ of the following form:
                    $$
                        \begin{tikzcd}
                            Q_1 \arrow[r, "t"', shift right=2] \arrow[r, "s", shift left=2] \arrow["\id_{Q_1}"', loop, distance=2em, in=215, out=145] & Q_0 \arrow["\id_{Q_0}"', loop, distance=2em, in=35, out=325]
                        \end{tikzcd}
                    $$
                wherein $Q_1, Q_0 \in \Ob(\C)$ respectively are known as the objects of \textbf{arrows/vertices} and \textbf{objects/edges}, and the arrows $s, t: Q_1 \toto Q_0$ are known as the \textbf{source} and \textbf{target} morphisms. 
            \end{definition}
            \begin{remark}[The category of quivers]
                Equivalently, one might think of $\C$-valued quivers as $\C$-valued presheaves on the category:
                    $$
                        \begin{tikzcd}
                            {\1} \arrow[r, "t"', shift right=2] \arrow[r, "s", shift left=2] \arrow["\id_{\1}"', loop, distance=2em, in=215, out=145] & {\0} \arrow["\id_{\0}"', loop, distance=2em, in=35, out=325]
                        \end{tikzcd}
                    $$
                As such, for an arbitrarily fixed target category $\C$, one obtains a category $\Quiv(\C)$ of $\C$-valued quivers and natural transformations between them. In particular, when $\C \cong \Sets$, one obtains the presheaf topos $\Quiv$ of $\Sets$-valued quivers. 
            \end{remark}
            \begin{example}[Quivers internal to $\Sets$]
                A $\Sets$-valued quiver is actually nothing but a so-called \textbf{directed graph}: such a quiver is a quadruple $(Q_1, Q_0, s, t)$ wherein $Q_1$ is a set of directed edges, $Q_0$ is the set of vertices of those direct edges, and $s, t: Q_1 \toto Q_0$ are the assignments of the sources and targets vertices (i.e. beginning and endpoints) to the aforementioned directed edges. One important detail to note here is that between two given vertices, there may be many directed edges, and there might also be loops onto the same vertex, as follows:
                    $$
                        \begin{tikzcd}
                                                                                                                                           &         & \bullet \arrow[loop, distance=2em, in=125, out=55]             &                                                        & \bullet \\
                            \bullet \arrow[r, shift right=2] \arrow[r, shift left=2] \arrow[loop, distance=2em, in=215, out=145] \arrow[r] & \bullet & \bullet \arrow[r, shift right=2] \arrow[l] \arrow[u] \arrow[d] & \bullet \arrow[ru] \arrow[rd] \arrow[l, shift right=2] &         \\
                                                                                                                                           &         & \bullet \arrow[loop, distance=2em, in=305, out=235]            &                                                        & \bullet
                        \end{tikzcd}
                    $$
                These diagrams of sets and functions, however, need not be commutative: as such, one is usually interested in the free categories associated to quivers, which are nothing but those quivers with compositions and identities added on (cf. proposition \ref{prop: free_quivers}).
                
                To be very specific, Dynkin diagrams are examples of $\Sets$-valued quivers (cf. definition \ref{def: dynkin_quivers}). 
            \end{example}
            \begin{proposition}[Sub-object classifier for $\Quiv$] \label{prop: sub_object_classifier_for_topos_of_quivers}
                
            \end{proposition}
                \begin{proof}
                    
                \end{proof}
            \begin{definition}[The double negation topology]
                
            \end{definition}
            \begin{proposition}[A separatedness criterion for quivers] \label{prop: separatedness_criterion_for_quivers}
                
            \end{proposition}
                \begin{proof}
                    
                \end{proof}
                
            \begin{proposition}[Free quivers] \label{prop: free_quivers}
                The evident forgetful functor:
                    $$\oblv: 1\-\Cat_1 \to \Quiv$$
                (which forgets compositions and the associativity of said compositions) admits a left-adjoint:
                    $$[-]: \Quiv \to 1\-\Cat_1$$
                which shall be known as the \textbf{free quiver} functor. Explicitly, this functor assigns to each ($\Sets$-valued) quiver its associated free category.
            \end{proposition}
                \begin{proof}
                    
                \end{proof}
            \begin{definition}[Quiver representations] \label{def: quiver_representations}
                Given a quiver $Q \in \Ob(\Quiv)$, a commutative ring $k$, and a $k$-linear tensor category\footnote{Aside from the obvious example of $k\mod^{\heart}$ (and $k$-linear tensor subcategories thereof, like $(k\mod^{\fin})^{\heart}$), one might also consider categories such as the derived category of $k$-modules $k\mod$.} $\calV$, its category of $k$-linear representations, denoted by $\Rep_{\calV}(Q)$, is the functor category $\Func([Q], \calV)$.
            \end{definition}
            \begin{remark}[Basic properties of quiver representations]
                Given a quiver $Q \in \Ob(\Quiv)$, a commutative ring $k$, and a $k$-linear tensor category $\calV$, its category of $k$-linear representations $\Rep_{\calV}(Q)$ will also be a $k$-linear tensor category.
                
                Of course, one could consider representations with values in categories of a more general kind than linear tensor $1$-categories, such as symmetric monoidal stable $\infty$-categories. 
            \end{remark}
            \begin{convention}
                Often, we will consider the case $\calV \cong k\mod$ for some commutative ring $k$, and in which case, we shall write $\Rep_k(Q)$ for the category of $k$-linear representations of $Q$. Some important subcategories therein are:
                    $$\Rep_k^{\irr}(Q)$$
                    $$\Rep_k^{\red}(Q)$$
                which are the categories of irreducible representations and that of indecomposable representations. When $Q$ is an infinite quiver, we will also be interested in the full subcategory:
                    $$\Rep_k^{\fin}(Q)$$
                of $k$-linear representations of $Q$ which are of finite ranks over $k$.
            \end{convention}
            
            Let us now move on to the notion of so-called \say{quiver algebras} and discuss the roles that they play in the representation theory of $\Sets$-valued quivers. 
            \begin{definition}[Category algebras] \label{def: category_algebras}
                Let $k$ be an associative ring and $\C$ be a category. Then, the \textbf{category $k$-algebra} $k\<\C\>$ of the given category $\C$ shall be the free $k$-algebra:
                    $$k\<\C_1\> := k\<\Mor(\C)\>$$
                whose underlying (left-)$k$-module is the free (left-)$k$-module on the set $\C_1 := \Mor(\C)$ of morphisms of $\C$ and whose (associative and unital) multiplication is given by compositions of arrows in $\C_1$ (if two arrows are not composable then their product will be $0$).
            \end{definition}
            \begin{convention}[Path algebras ?]
                Sometimes the category $k$-algebra (for some ring $k$) of the free category associated to a given $\Sets$-valued quiver $Q$ is also called the path algebra of that quiver. It is commonly denoted simply by $k\<Q\>$ as opposed to $k\<[Q]\>$.
            \end{convention}
            \begin{remark}[A path algebra functor] \label{remark: path_algebra_functor}
                When $k$ is commutative, one has a tautological equivalence of categories:
                    $$k\bimod \cong k\mod$$
                which gives us access to the tensor $k$-algebra construction, which is \textit{a priori} a left-adjoint:
                    $$
                        \begin{tikzcd}
                        	{k\-\Assoc\Alg} & k\mod
                        	\arrow[""{name=0, anchor=center, inner sep=0}, "{(-)^{\tensor}}"', bend right, from=1-2, to=1-1]
                        	\arrow[""{name=1, anchor=center, inner sep=0}, "\oblv"', bend right, from=1-1, to=1-2]
                        	\arrow["\dashv"{anchor=center, rotate=-90}, draw=none, from=0, to=1]
                        \end{tikzcd}
                    $$
                Using this, one sees that category $k$-algebras arise functorially in the following manner:
                    $$
                        \begin{tikzcd}
                        	{k\-\Assoc\Alg} && k\mod \\
                        	\\
                        	{1\-\Cat_1} && \Sets
                        	\arrow[""{name=0, anchor=center, inner sep=0}, "{(-)^{\tensor}}"', bend right, from=1-3, to=1-1]
                        	\arrow[""{name=1, anchor=center, inner sep=0}, "\oblv"', bend right, from=1-1, to=1-3]
                        	\arrow["{(-)_1}", from=3-1, to=3-3]
                        	\arrow["{k\<-\> := ( k^{\oplus (-)_1} )^{\tensor}}", dashed, from=3-1, to=1-1]
                        	\arrow[""{name=2, anchor=center, inner sep=0}, "{k^{\oplus (-)}}", bend left, from=3-3, to=1-3]
                        	\arrow[""{name=3, anchor=center, inner sep=0}, "\oblv", bend left, from=1-3, to=3-3]
                        	\arrow["\dashv"{anchor=center, rotate=-90}, draw=none, from=0, to=1]
                        	\arrow["\dashv"{anchor=center}, draw=none, from=2, to=3]
                        \end{tikzcd}
                    $$
                Now, since the functors:
                    $$(-)^{\tensor}: k\mod \to k\-\Assoc\Alg$$
                and:
                    $$k^{\oplus (-)}: \Sets \to k\mod$$
                are left-adjoints, and because:
                    $$(-)_1: 1\-\Cat_1 \to \Sets$$
                preserves coproducts and pushouts, one sees that given any pair of categories $\C, \C'$ (which need not be disjoint), one has:
                    $$k\<\C \cup \C'\> \cong k\<\C\> \tensor_k k\<\C'\>$$
                Since the construction of free categories on (small) quivers also comes from a left-adjoint, namely:
                    $$[-]: \Quiv \to 1\-\Cat$$
                (cf. proposition \ref{prop: free_quivers}), one also has\footnote{One does not have to worry about the formal existence of $Q \cup Q'$ within the category $\Quiv$ of $\Sets$-valued quivers, since it is a topos by construction.}:
                    $$k\<Q \cup Q'\> \cong k\<Q\> \tensor_k k\<Q'\>$$
            \end{remark}
            \begin{example}[Path algebras of discrete quivers]
                The simplest example of a quiver is the trivial loop:
                    \begin{figure}[H]
                        \centering
                            $$
                                \begin{tikzcd}
                                    \bullet \arrow["\id"', loop, distance=2em, in=35, out=325]
                                \end{tikzcd}
                            $$
                        \caption{The trivial quiver}
                        \label{fig: trivial_quiver}
                    \end{figure}
                It is easy to see that the path algebra of this quiver (over some fixed ring $k$) is:
                    $$k\<e\>/\<e^2 = e\> \cong k$$
                    
                Going off in a different direction\footnote{Pun may or may not have been intended.}, consider the following quiver $Q := (Q_1, Q_0, s, t)$ (i.e. a discrete category\footnote{Let's ignore potential set-theoretic issues for now!}):
                    $$
                        \begin{tikzcd}
                            \cdots & \bullet_i \arrow["\id_{\bullet_i}"', loop, distance=2em, in=125, out=55] & \bullet_j \arrow["\id_{\bullet_j}"', loop, distance=2em, in=125, out=55] & \bullet_k \arrow["\id_{\bullet_k}"', loop, distance=2em, in=125, out=55] & \cdots
                        \end{tikzcd}
                    $$
                Its path algebra is:
                    $$k\<\{e_i\}_{i \in Q_0}\>\/\<\forall i \in Q_0: e_i^2 = e_i, \forall i, j \in Q_0: e_i e_j = e_j e_i = \delta_{ij}\>$$
                (where $\delta_{ij}$ is the Kronecker delta), which we recognise as being isomorphic to the product:
                    $$\prod_{i \in Q_0} k$$
                taken in the category of associative $k$-algebras.
            \end{example}
            \begin{example}[Path algebras of $\sfA_n$ quivers] \label{example: path_algebras_of_A_n_quivers}
                For a simple yet non-trivial example of quiver algebras, consider the following quiver, commonly known as the \say{$\sfA_2$ quiver}:
                    \begin{figure}[H]
                        \centering
                        $$
                            \begin{tikzcd}
                                {\bullet_1} \arrow["\id_{\bullet_1}"', loop, distance=2em, in=215, out=145] \arrow[r] & {\bullet_2} \arrow["\id_{\bullet_2}"', loop, distance=2em, in=35, out=325]
                            \end{tikzcd}
                        $$
                        \caption{The $\sfA_2$ quiver}
                        \label{fig: A_2_quiver}
                    \end{figure}
                which has a rather small path algebra, isomorphic to:
                    $$k\<e_1, e_2, x\>/\<e_i^2 = e_i, e_i e_j = e_j e_i = \delta_{ij}, x^2 = 0, e_2 x = x e_1 = x\>$$
                One observation to make is that there is an associative $k$-algebra isomorphism given by:
                    $$e_1 \mapsto \begin{pmatrix} 1 & 0 \\ 0 & 0 \end{pmatrix}$$
                    $$e_2 \mapsto \begin{pmatrix} 0 & 0 \\ 0 & 1 \end{pmatrix}$$
                    $$x \mapsto \begin{pmatrix} 0 & 0 \\ 1 & 0 \end{pmatrix}$$
                between this path $k$-algebra and the $k$-algebra:
                    $$\b_2^-(k) := \left\{ \begin{pmatrix} * & 0 \\ * & * \end{pmatrix} \in \Mat_2(k) \right\}$$
                of lower-triangular $2 \x 2$ matrices with entries in $k$.
                    
                Next, consider the general $\sfA_n$ quiver\footnote{Henceforth we shall start omiting the identity paths from depictions of quivers.} (for some finite positive integer $n$):
                    \begin{figure}[H]
                        \centering
                            $$
                                \begin{tikzcd}
                                	{\bullet_1} & {\bullet_2} & {\bullet_3} & \cdots
                                	\arrow["{x_{12}}", from=1-1, to=1-2]
                                	\arrow["{x_{23}}", from=1-2, to=1-3]
                                	\arrow["{x_{34}}", from=1-3, to=1-4]
                                \end{tikzcd}
                            $$
                        \caption{The $\sfA_n$ quiver}
                        \label{fig: A_n_quiver}
                    \end{figure}
                Observe that the set of morphisms $[\sfA_n]_1$ of the free category $[\sfA_n]$ on $\sfA_n$ is nothing but $\{x_{ij}\}_{1 \leq i \leq j \leq n}$ (also, note that $e_i = x_{ii}$ for all $1 \leq i \leq n$). Now, by letting $\1_{ij}$ denote the $n \x n$ matrix with $1$ at the $ij^{th}$ entry and $0$ everywhere else, and consider the map:
                    $$k\<\sfA_n\> \to \Mat_n(k)$$
                    $$x_{ij} \mapsto \1_{ij}$$
                one then sees that $k\<\sfA_n\>$ is isomorphic to the $k$-algebra:
                    $$\b_n^-(k)$$
                of lower-triangular $n \x n$ matrices with entries in $k$.
            \end{example}
            \begin{example}[Path algebras of quivers with loops]
                If we were to add a non-trivial loop $x: \bullet \to \bullet$ to the trivial quiver $(\{\bullet\}, \{\id\})$ to yield:
                    $$
                        \begin{tikzcd}
                            \bullet \arrow["x"', loop, distance=2em, in=35, out=325]
                        \end{tikzcd}
                    $$
                then the resulting path algebra will be:
                    $$k\<e, x\>/\<e^2 = e, ex = xe = x\> \cong k\<x\>$$
                Of course, when $k$ is a commutative ring, then we have a tautological isomorphism $k\<x\> \cong k[x]$, but this does not really help us flesh out the nature of the path algebra of this loop quiver.
                    
                More generally, we can consider the quiver with one vertex and a set $\{x_i\}_{i \in I}$ of (possibly infinitely) many loops on that vertex:
                    $$
                        \begin{tikzcd}
                            \bullet \arrow["x_1"', loop, distance=2em, in=305, out=235] \arrow["x_2"', loop, distance=2em, in=35, out=325] \arrow["x_3"', loop, distance=2em, in=125, out=55] \arrow["\cdots"', loop, distance=2em, in=215, out=145]
                        \end{tikzcd}
                    $$
                then the resulting path algebra will be:
                    $$k\<\{x_i\}_{i \in I}\>$$
                because while one can free form compositions $x_j \circ x_i \circ ...$ of the endomorphisms/loops $x_i \in \End(\bullet)$, there is not a reason to expect that $x_j x_i = x_i x_j$. 
            \end{example}
            \begin{example}[Path algebras of non-simply-laced quivers]
                Consider the so-called \say{Kronecker quiver}:
                    \begin{figure}[H]
                        \centering
                            $$
                                \begin{tikzcd}
                                	{\bullet_1} & {\bullet_2}
                                	\arrow["x_2"', shift right=1, from=1-1, to=1-2]
                                	\arrow["x_1", shift left=1, from=1-1, to=1-2]
                                \end{tikzcd}
                            $$
                        \caption{The Kronecker quiver}
                        \label{fig: kronecker_quiver}
                    \end{figure}
                which is easily recognisable as the (non-disjoint) union of two copies of the $\sfA_2$ quiver (cf. example \ref{example: path_algebras_of_A_n_quivers}). One sees that its path $k$-algebra is:
                    $$k\<e_1, e_2, x_1, x_2\>/\<e_i^2 = e_i, e_i e_j = e_j e_i = \delta_{ij}, x_m x_n = 0, e_2 x_m = x_m e_1 = x_m\>$$
                which can be shown to be isomorphic to the $k$-algebra\footnote{$\b_2^-(k)$ is the $k$-algebra of lower-triangular $2 \x 2$ matrices with entries in $k$; cf. example \ref{example: path_algebras_of_A_n_quivers}.}:
                    $$k\<\sfA_2 \cup \sfA_2\> \cong k\<\sfA_2\> \tensor_k k\<\sfA_2\> \cong \b_2^-(k) \tensor_k \b_2^-(k)$$
                using the fact that the path $k$-algebra functor:
                    $$k\<-\>: \Quiv \to k\-\Assoc\Alg$$
                preserves pushouts\footnote{In $\Quiv$, these are nothing but unions of quivers} (cf. remark \ref{remark: path_algebra_functor}) whenever $k$ is commutative; when $k$ is noncommutative, one can still show that the isomorphism holds using the fact that $\b_2^-(k)$ is a $k$-bimodule.
            \end{example}
            \begin{example}[Path algebras of $\sfD_n$ quivers]
                For any $n \geq 2$, the so-called \say{$\sfD_n$ quiver} is of the form:
                    \begin{figure}[H]
                        \centering
                            $$
                                \begin{tikzcd}
                                	&&&& {\bullet_n} \\
                                	{\bullet_1} & {\bullet_2} & \cdots & {\bullet_{n - 2}} \\
                                	&&&& {\bullet_{n - 1}}
                                	\arrow["{x_{12}}", from=2-1, to=2-2]
                                	\arrow["{x_{23}}", from=2-2, to=2-3]
                                	\arrow["{x_{n - 2, n - 3}}", from=2-3, to=2-4]
                                	\arrow["{x_{n, n - 2}}", from=2-4, to=1-5]
                                	\arrow["{x_{n - 1, n - 2}}"', from=2-4, to=3-5]
                                \end{tikzcd}
                            $$
                        \caption{The $\sfD_n$ quiver}
                        \label{fig: D_n_quiver}
                    \end{figure}
                (when $n = 2, 3$, note that $\sfD_n = \sfA_n$, so there is nothing new in those cases, and we shall assume hereonafter that $n \geq 3$). Observe that for all $n \geq 3$, the $\sfD_n$ quiver is the union of the $\sfA_{n - 2}$ quiver with two copies of the $\sfA_2$ quiver, both at the $(n - 2)^{th}$ vertex and in such a manner that:
                    $$x_{n - 1, n - 2} x_{n - 2, n - 3} \not = 0$$
                    $$x_{n, n - 2} x_{n - 2, n - 3} \not = 0$$
                and with this observation in mind, one sees that:
                    $$k\<\sfD_n\> \cong k\<\sfA_{n - 2}\> \tensor_k k\<\sfA_2\> \tensor_k k\<\sfA_2\> \tensor_k k\<\sfA_3\>/$$
            \end{example}
            \begin{example}[Path algebras of $\sfE_n$ quivers]
                
            \end{example}
            \begin{proposition}[Quiver representations are modules over quiver algebras] \label{prop: quiver_representations_are_modules_over_quiver_algebras}
                Let $k$ be a commutative ring. Then, there is an exact and monoidal equivalence of $k$-linear categories (fibred over $k\mod$):
                    $$\Rep_k(Q) \to {}^lk\<Q\>\mod$$
            \end{proposition}
                \begin{proof}
                    
                \end{proof}
            \begin{remark}
                The equivalence of categories:
                    $$\Rep_k(Q) \to {}^lk\<Q\>\mod$$
                from proposition \ref{prop: quiver_representations_are_modules_over_quiver_algebras} has many important further properties, which all come from the very definition of linear representations of associative algebras (in this case, path algebras of quivers) themselves. For instance, this functor preserves (semi-)simplicity and (in)decomposability.
                
                As for further properties coming not from the definition of representations themselves, but rather from the basic property of this equivalence of categories, one has that the functor also preserves lengths and more generally, ascending and descending chains of objects, as a consequence of being exact. In particular, this means that should a representation of a given quiver $Q$ is finitely generated as a $k$-module (i.e. \say{finite-dimensional}), then the same would also be true for the corresponding left-$k\<Q\>$-module.
            \end{remark}
            \begin{remark}
                One important algebraic consequence of proposition \ref{prop: quiver_representations_are_modules_over_quiver_algebras} is should $k$ be a commutative ring and $Q$ be a finite $\Sets$-valued quiver, then the path $k$-algebra $k\<Q\>$ will necessarily have finite rank as a $k$-module. $k\<Q\>$ is therefore left and right-Artinian, and as such all left/right-$k\<Q\>$-modules are necessarily finitely generated; equivalently, this means that all $k$-linear representations of $Q$ are of finite ranks over $k$.
            \end{remark}
            
        \subsubsection{Tits quadratic forms; roots; Grothendieck groups, Euler characteristics, and dimension vectors}
            \begin{definition}[Cartan quadratic forms of finite quivers] \label{def: cartan_quadratic_forms_of_finite_quivers}
                Let $Q := (Q_1, Q_0, s, t)$ be a finite quiver\footnote{With value in any category} and suppose that $n := |Q_0|$; also, let us give the set $Q_0$ of vertices a \textit{fixed} enumeration $\{v_1, ..., v_n\}$. To such a quiver, one can associate a so-called \textbf{adjacency matrix}:
                    $$R_Q \in \Mat_n(\Z)$$
                given by:
                    $$R_Q := (r_{ij} := |\{f \in Q_1 \mid s(f) = v_i, t(f) = v_j\}|)_{1 \leq i, j \leq n}$$
                (that is, the entries $r_{ij}$ of $R_Q$ are the numbers of \textit{undirected} edges from the vertex $v_i$ to the vertex $v_j$). From this matrix, one can define the \textbf{Cartan matrix} of $Q$:
                    $$A_Q := 2I_n - R_Q$$
                along with a $\Z$-bilinear form, which we shall call the \textbf{Cartan form}:
                    $$B_Q: \Z^{\oplus n} \x \Z^{\oplus n} \to \Z$$
                    $$(x, y) \mapsto x^{\top} A_Q y$$                
            \end{definition}
            \begin{definition}
                A quiver is said to have a ring-theoretic property $\calP$ over some base ring $k$ if and only if its quiver $k$-algebra has property $\calP$.
            \end{definition}
            \begin{definition}[Simply laced quivers] \label{def: simply_laced_quivers}
                A $\Sets$-valued quiver $Q := (Q_1, Q_0, s, t)$ is \textbf{simply laced} if and only if none of its vertices has self-loops: that is, for all $v \in Q_0$, one has:
                    $$\{f \in Q_1 \mid s(f) = t(f) = v\} = \{\id_v\}$$
            \end{definition}
            \begin{remark}
                Equivalently, one might characterise the simply laced quivers $Q := (Q_1, Q_0, s, t)$ as those wherein all non-identity paths $f \in Q_1$ generate square-free elements $f \in k\<Q\>$ (i.e. elements such that $f^2 = 0$). This is because the lack of non-identity loops implies that there is no non-identity path that can be composed with itself. 
            \end{remark}
            \begin{definition}[Connected quivers] \label{def: connected_quivers}
                A $\Sets$-valued quiver $Q: \{\1, \0, s, t\}^{\op} \to \Sets$ is said to be \textbf{connected} if and only if for all $v, w \in Q(\0)$, there exists $f \in Q(\1)$ such that $s(f) = v$ and $t(f) = w$ (i.e. it has no isolated vertices).
            \end{definition}
            \begin{remark}
                Equivalently, one can say that a $\Sets$-valued (or maybe with values in any category $\C$ with enough coproducts and sub-objects) quiver is connected if and only if it can not be written as the disjoint union of two sub-quivers. 
            \end{remark}
            \begin{definition}[Dynkin quivers] \label{def: dynkin_quivers}
                A \textbf{Dynkin quiver} is a \textit{connected} and \textit{simply laced} finite quiver whose Cartan form is positive-definite.
            \end{definition}
            \begin{proposition}[Evenness of Cartan forms of Dynkin quivers] \label{prop: evenness_of_cartan_forms_of_dynkin_quivers}
                Let $\Gamma := (\Gamma_1, \Gamma_0, s, t)$ be a Dynkin quiver and consider its Cartan matrix $A_{\Gamma}$. Then for all $x \in \Z^{\oplus n}$, $B_{\Gamma}(x, x)$ is even.  
            \end{proposition}
                \begin{proof}
                    Let $n := |\Gamma_0|$ and pick a basis for $\Z^{\oplus n}$ in order to write:
                        $$
                            \begin{aligned}
                                B_{\Gamma}(x, x) & = x^{\top} A_{\Gamma} x
                                \\
                                & = \sum_{1 \leq i, j \leq n} x_i a_{ij} x_j
                                \\
                                & = \sum_{1 \leq i, j \leq n} x_i (2\delta_{ij} - r_{ij}) x_j \text{(wherein $\delta_{ij}$ is the Kronecker delta)}
                                \\
                                & = 2\sum_{1 \leq i, j \leq n} x_i \delta_{ij} x_j - \sum_{1 \leq i, j \leq n} x_i (1 - \delta_{ij}) r_{ij} x_j \text{($r_{ij}$ are the entries the adjacency matrix of $\Gamma$)}
                                \\
                                & = 2\sum_{1 \leq i \leq n} x_i^2 - 2\sum_{1 \leq i < j \leq n} x_i r_{ij} x_j
                            \end{aligned}
                        $$
                    for all $x \in \Z^{\oplus n}$, wherein the last line is due to the fact that the number of undirected edges from a vertex $v_i$ to another $v_j$ is equal to the number of undirected directed edges from $v_j$ to $v_i$, which in turn is a result of the assumption that by virtue of being a Dynkin quiver, $\Gamma$ is connected and simply laced\footnote{Note how this argument fails if $\Gamma$ is either not simply laced or not connected.}. Clearly, $B_{\Gamma}(v, v)$ is even for all $x \in \Z^{\oplus n}$.
                \end{proof}
            \begin{definition}[Tits quadratic forms] \label{def: tits_quadratic_forms}
                Let $\Gamma := (\Gamma_1, \Gamma_0, s, t)$ be a Dynkin quiver and $B_{\Gamma}$ be its Cartan quadratic form. Then, inspired by the proof of proposition \ref{prop: evenness_of_cartan_forms_of_dynkin_quivers}, let us define the \textbf{Tits quadratic form} associated to $\Gamma$ by:
                    $$q_{\Gamma}: \Z^{\oplus \Gamma_0} \to \Z$$
                    $$x \mapsto \frac12 B_{\Gamma}(x, x)$$
                Since $B_{\Gamma}(x, x)$ is \textit{a priori} even (cf. proposition \ref{prop: evenness_of_cartan_forms_of_dynkin_quivers}), $q_{\Gamma}(x)$ is always well-defined. 
            \end{definition}
            \begin{definition}[Roots] \label{def: roots_of_dynkin_quivers}
                A root of a \textit{Dynkin} quiver $\Gamma := (\Gamma_1, \Gamma_0, s, t)$ is a vector:
                    $$\alpha \in \Z^{\oplus n}$$
                (where $n := |\Gamma_0|$) such that:
                    $$q_{\Gamma}(\alpha) = 1$$
                The set of all roots of $\Gamma$ is denoted by $\Phi_{\Gamma}$.
            \end{definition}
            \begin{example}[Simple roots] \label{example: simple_roots}
                Let $\Gamma := (\Gamma_1, \Gamma_0, s, t)$ be a Dynkin quiver and set $n := \Gamma_0$. Also, pick a basis $\{e_1, ..., e_n\}$ for $\Z^{\oplus n}$. Then the vectors of the form:
                    $$\alpha_i := \delta_{ij} e_j$$
                (for $1 \leq i, j \leq n$) are in fact roots of $\Gamma$ and furthermore, they form a basis for $\Z^{\oplus n}$.
            \end{example}
            \begin{lemma}[Roots are exclusively either negative or positive] \label{lemma: roots_are_exclusively_either_negative_or_positive}
                Let $\Gamma := (\Gamma_1, \Gamma_0, s, t)$ be a Dynkin quiver and set $n := \Gamma_0$. Also, pick a basis $\{e_1, ..., e_n\}$ for $\Z^{\oplus n}$ and choose a root $\alpha$ and write it in terms of the simple roots $\{\alpha_i\}_{1 \leq i \leq n}$ as:
                    $$\alpha := \sum_{1 \leq i \leq n} c_i \alpha_i$$
                Then either $c_i \geq 0$ or $c_i \leq 0$ for all $1 \leq i \leq n$ simultaneously.
            \end{lemma}
                \begin{proof}
                    
                \end{proof}
            \begin{definition}[Positive and negative roots] \label{def: negative_and_positive_roots}
                Let $\Gamma := (\Gamma_1, \Gamma_0, s, t)$ be a Dynkin quiver and set $n := \Gamma_0$. Also, pick a basis $\{e_1, ..., e_n\}$ for $\Z^{\oplus n}$ and choose a root $\alpha$ and write it in terms of the simple roots $\{\alpha_i\}_{1 \leq i \leq n}$ as:
                    $$\alpha := \sum_{1 \leq i \leq n} c_i \alpha_i$$
                Then we say that $\alpha$ is \textbf{positive} if and only if $c_i \geq 0$ for all $1 \leq i \leq n$ and \textbf{negative} if and only if $c_i \leq 0$ for all $1 \leq i \leq n$. The set of positive (respectively, negative) roots of the given Dynkin quiver $\Gamma$ is denoted by $\Phi_{\Gamma}^+$ (respectively, $\Phi_{\Gamma}^-$).
            \end{definition}
            \begin{remark}
                Obviously, one has:
                    $$\Phi_{\Gamma} = \Phi_{\Gamma}^+ \cup \Phi_{\Gamma}^-$$
                for all Dynkin quivers $\Gamma$.
            \end{remark}
            
            Now that we are familiar with Cartan quadratic forms, let us see how they arise naturally as Euler characteristics of representations of quivers (cf. theorem \ref{theorem: tits_quadratic_forms_as_euler_characteristics}). 
            \begin{definition}[Simple Grothendieck groups] \label{def: simple_grothendieck_groups}
                The \textbf{simple Grothendieck group} of an abelian (respectively, triangulated) category $\E$, denoted by $\K_0^{\simple}(\E)$, is defined to be the abelian group generated by the set of isomorphism classes of objects of $\E$, subjected to the relations:
                    $$[M] = [M'] + [M'']$$
                on all short exact sequences (respectively, exact triangles):
                    $$M' \to M \to M''$$
            \end{definition}
            \begin{remark}
                Let $\E$ be either be an abelian category or triangulated category. Then clearly:
                    $$[M \oplus N] = [M] + [N]$$
                (simply consider the canonical split short exact sequence $0 \to M \to M \oplus N \to N \to 0$).
            \end{remark}
            \begin{convention}[Jordan-H\"older multiplicities]
                Recall that by Schur's Lemma (cf. lemma \ref{lemma: schur_lemma_for_abelian_categories}), all morphisms between simple objects of any given abelian category are either zero or isomorphisms. Recall also, that by the Jordan-H\"older Theorem (cf. theorem \ref{theorem: jordan_holder_theorem}), Finite-length objects have Jordan-H\"older filtrations which are unique up to isomorphisms, which in particular implies that the simple factors are also unique up to isomorphisms. As such, one may speak of the \textbf{multiplicity} (cf. definition \ref{def: lengths_of_objects_and_jordan_holder_series}) of \textit{any} given simple object $E \in \Ob(\E)$ in any finite ($\Z$-linear) abelian category $\E$ (cf. definition \ref{def: finite_linear_categories}) within the Jordan-H\"older series of some other object $M \in \Ob(\E)$, which we shall denote by:
                    $$[M : E]$$
                Note that $M$ is necessarily of finite length per remark \ref{remark: locally_finite_linear_categories_are_jordan_holder_and_krull_schmidt} and the fact that finite abelian categories are locally finite by definition (cf. proposition \ref{prop: finite_linear_categories_are_locally_finite}).
            \end{convention}
            \begin{proposition}[Simple Grothendieck groups are free on simple objects] \label{prop: simple_grothendieck_groups_of_finite_linear_abelian_categories_are_free_on_simple_objects} 
                Let $k$ be a commutative ring and $\E$ be a finite $k$-linear abelian category with coefficient algebra $A$ (cf. definition \ref{def: finite_linear_categories}). The simple Grothendieck group $\K_0^{\simple}(\E)$ will then be isomorphic to the free abelian group on the \textit{a priori} finite (cf. proposition \ref{prop: simple_objects_in_finite_linear_categories}) set of isomorphism classes of simple objects\footnote{Hence the notation.} of $\E$, i.e.:
                    $$\K_0^{\simple}(\E) \cong \Z^{\oplus [\E^{\simple}]}$$
                In particular, given any object $M \in \Ob(\E)$, one can write:
                    $$[M] := \sum_{[E] \in [\E^{\simple}]} [M : E] [E]$$
            \end{proposition}
                \begin{proof}
                    
                \end{proof}
                
            \begin{definition}[Projective Grothendieck groups] \label{def: projective_grothendieck_groups}
                Let $k$ be a commutative ring. The \textbf{projective Grothendieck group} of a finite $k$-linear abelian category $\E$, denoted by $\K_0^{\proj}(\E)$, is defined to be the abelian group generated by the set $[\E^{\proj}]$ of isomorphism classes of projective objects of $\E$, subjected to the relations:
                    $$[M] = [M'] + [M'']$$
                on all short exact sequences:
                    $$0 \to M' \to M \to M'' \to 0$$
            \end{definition}
            \begin{proposition}[Projective Grothendieck groups are free on projective indecomposable objects] \label{prop: projective_grothendieck_groups_are_free_on_projecitve_indecomposable_objects}
                Let $k$ be a commutative ring. The projective Grothendieck group of any finite $k$-linear abelian category $\E$ is then isomorphic to the free abelian group on the set of isomorphism classes of projective indecomposable objects of $\E$, i.e.:
                    $$\K_0^{\proj}(\E) \cong \Z^{\oplus [\E^{\proj, \indecomp}]}$$
            \end{proposition}
                \begin{proof}
                    
                \end{proof}
            \begin{lemma}[Projective indecomposable modules over finite algebras are simple] \label{lemma: projective_indecomposable_modules_over_finite_algebras_are_simple}
                \footnote{One interesting consequence of this lemma (which is not too relevant to the current discussion about Dynkin quivers, as path algebras of quivers - even over fields - are generally not semi-simple, only hereditary) is that it implies via lemma \ref{lemma: projective_and_injective_modules_over_semi_simple_rings} that over a semi-simple finite-dimensional algebra over a field, a module is indecomposable if and only if it is simple.} Let $k$ be a field and $A$ be a finite-dimensional $k$-algebra. Then, there exists a surjection:
                    $$
                        \left\{\text{projective left/right-$A$-modules}\right\}
                        \to
                        \left\{\text{simple left/right-$A$-modules}\right\}
                    $$
                whose pre-images over each simple left/right-$A$-module $M$ is a choice of projective cover $\e: P \to M$ wherein $P$ is indecomposable as an $A$-module, which is unique up to isomorphisms. 
            \end{lemma}
                \begin{proof}
                    First of all, because the category of left/right-$A$-modules \textit{a priori} has enough projectives, meaning that for all left/right-$A$-modules $M$ there exists a projective cover $\e: P \to M$, we certainly have a surjection:
                        $$
                            \left\{\text{projective left/right-$A$-modules}\right\}
                            \to
                            \left\{\text{simple left/right-$A$-modules}\right\}
                        $$
                    Now, in order to show that the pre-image of the class of simple left/right-$A$-modules under this surjection is that of indecomposable projective left/right-$A$-modules, start by recalling that simple left/right-modules are cyclic\footnote{The proof is simple (pun not intended!): if we were to suppose to the contrary that there existed a simple module $M$ with $\geq 1$ generators, then said module will admit non-zero (cyclic) proper submodules generated by the generators, and therefore $M$ can not be simple, contradicting our initial assumption.}. This tells us that every simple left/right-$A$-module admits a projective cover by a free\footnote{We assume the Axiom of Choice, so that free module would be projective.} left/right-$A$-module on $1$ generator, which is of course indecomposable and unique up to isomorphisms, per the universal property of left-adjoints.
                \end{proof}
            \begin{proposition}[Projective and simple Grothendieck groups coincide] \label{prop: projecitve_and_simple_grothendieck_groups_coincide}
                Let $k$ be a commutative ring and $\E$ be any finite $k$-linear abelian category with coefficient algebra $A$ (cf. definition \ref{def: finite_linear_categories}). Then there will be a group isomorphism:
                    $$\K_0^{\proj}(\E) \cong \K_0^{\simple}(\E)$$
                given by $[M] \mapsto [M/\rad(A) M]$.
            \end{proposition}
                \begin{proof}
                    Combine lemma \ref{lemma: projective_indecomposable_modules_over_finite_algebras_are_simple} with proposition \ref{prop: simple_grothendieck_groups_of_finite_linear_abelian_categories_are_free_on_simple_objects}.
                \end{proof}
            
            Let us now discuss Euler characteristics, which is the mean through which we shall obtain roots from representations of quivers (cf. theorem \ref{theorem: tits_quadratic_forms_as_euler_characteristics}).
            \begin{definition}[Compact objects] \label{def: compact_objects}
                An object $c \in \Ob(\C)$ of a locally small category $\C$ is said to be \textbf{compact} if and only if the representable copresheaf:
                    $$\C(c, -): \C \to \Sets$$
                preserves all small filtered colimits that exist in $\C$. The (\textit{a priori} full) subcategory of $\C$ spanned by such objects is denoted by $\C^{\comp}$ (or sometimes $\C^{\omega}$) and shall be known as the \textbf{maximal compact subcategory} of $\C$. 
            \end{definition}
            \begin{remark}[Compact objects in full subcategories] \label{remark: compact_objects_in_full_subcategories}
                Suppsoe that $\C$ is a category and $\C_0 \subseteq \C$ is a full subcategory thereof. Then, should $c \in \Ob(\C_0)$ be any object of $\C_0$ that is compact as an object of the larger ambient category $\C$, then it is clear from definition \ref{def: compact_objects} that it would also be compact as an object of $\C_0$; that is:
                    $$\C_0 \cap \C^{\comp} = \C_0^{\comp}$$
            \end{remark}
            \begin{definition}[Euler characteristics] \label{def: euler_characteristics}
                Let $k$ be a commutative ring and $\E$ be a locally finite $k$-linear abelian category. Next consider some left-exact\footnote{In the triangulated sense. For instance, one might consider right-derived functors.} endofunctor:
                    $$F \in \rmD^b(\E)^{\comp} \to \rmD^b(\E)^{\comp}$$
                on the full subcategory $\rmD^b(\E)^{\comp}$ of compact complexes\footnote{This is to ensure that the cohomologies $H^i(F(M^{\bullet}))$ are of finite ranks as $k$-modules.} inside the bounded derived category\footnote{Considered as a triangulated category.} $\rmD^b(\E)$. The \textbf{Euler characteristic} of this functor is then the function:
                    $$\chi(F, -): \Ob(\rmD^b(\E)) \to \Z$$
                    $$M^{\bullet} \mapsto \sum_{i \in \Z} (-1)^i \rank_k H^i(F(M^{\bullet}))$$
            \end{definition}
            \begin{example}[Euler characteristics of topological spaces] \label{example: euler_characteristics_of_topological_spaces}
                
            \end{example}
            \begin{example}[Euler characteristics of schemes] \label{example: euler_characteristics_of_schemes}
                
            \end{example}
            \begin{example}[Euler characteristics of quiver representations] \label{example: euler_characteristics_of_quiver_representations}
                Let $Q$ be a finite quiver and $k$ be a commutative ring. The Euler characteristic of $Q$ is then defined to be the Euler characteristic of the right-derived functor:
                    $$\R\Hom_{k\<Q\>}(\calF, -): \rmD(\Rep_k(Q)) \to \rmD(\Rep_k(Q))$$
                evaluated at $\calF$ itself (or rather, some injective resolution $\calF^{\bullet}$ thereof), i.e.:
                    $$
                        \begin{aligned}
                            \chi_Q(\calF) & := \chi(\R\Hom_{k\<Q\>}(\calF, -), \calF^{\bullet})
                            \\
                            & = \sum_{i \in \Z} (-1)^i \rank_k \R\Hom_{k\<Q\>}^i(\calF, \calF^{\bullet})
                            \\
                            & = \sum_{i \in \Z} (-1)^i \rank_k \Ext^i_{k\<Q\>}(\calF, \calF)
                        \end{aligned}
                    $$
                Now, if $A$ is a hereditary $k$-algebra (e.g. $A \cong k\<Q\>$), then one has by definition that:
                    $$\Ext^i_A(M, N) \cong 0$$
                for all left/right-$A$-modules $M, N$, so in fact, one has:
                    $$\chi_Q(\calF) = \rank_k \End_{k\<Q\>}(\calF) - \rank_k \Ext^1_{k\<Q\>}(\calF, \calF)$$
                for all representations $\calF \in \Ob(\Rep_k(Q))$.
            \end{example}
            
            \begin{definition}[Dimension vectors of quiver representations] \label{def: dimension_vectors_of_quiver_representations}
                Let $k$ be a commutative ring, let $Q := (Q_1, Q_0, s, t)$ be a finite quiver, and consider the finite $k$-linear abelian category $\Rep_k(Q)$ of $k$-linear representations of $Q$. We then call the assignment:
                    $$\rank_k: \Rep_k(Q) \to \Z^{\oplus Q_0}$$
                    $$\calF \mapsto \sum_{v \in Q_0} \rank_k \calF(v) e_v$$
                the assignment of \textbf{rank vectors} (or \textbf{dimension vectors} when $k$ is a field) to $k$-linear representations of $Q$, wherein the set $\{e_v\}_{v \in Q_0}$ is some choice of basis for the free $\Z$-module $\Z^{\oplus Q_0}$.
            \end{definition}
            \begin{remark}
                Let $k$ be a field and let $Q := (Q_1, Q_0, s, t)$ be a finite quiver. It is not hard to notice that taking rank vectors of $k$-linear representations of $Q$ is additive, i.e.:
                    $$\rank_k (\calF \oplus \calF') = \rank_k \calF + \rank_k \calF'$$
                for all $\calF, \calF' \in \Ob(\Rep_k(Q))$. 
            \end{remark}
            \begin{lemma}[Representations of finite quivers are determined by dimension vectors] \label{lemma: representations_of_finite_quivers_are_determined_by_dimension_vectors}
                Let $k$ be a field and let $Q := (Q_1, Q_0, s, t)$ be a finite quiver. There is then a $\Z$-module isomorphism:
                    $$\rank_k: \K_0(Q) \to \Z^{\oplus Q_0}$$
            \end{lemma}
                \begin{proof}
                    
                \end{proof}
            \begin{corollary}[Irreducible representations of finite quivers are labelled by vertices] \label{coro: irreducible_representations_of_finite_quivers_are_labelled_by_vertices}
                Let $k$ be a commutative ring and $Q := (Q_1, Q_0, s, t)$ be a finite quiver. Then, there are choices of bijections:
                    $$[\Rep_k^{\irr}(\Gamma)] \cong Q_0$$
            \end{corollary}
                \begin{proof}
                    Since the Grothendieck group $\K_0(Q)$ is freely generated as a $\Z$-module by the set of isomorphism classes of irreducible $k$-linear representations of $Q$ (cf. proposition \ref{prop: simple_grothendieck_groups_of_finite_linear_abelian_categories_are_free_on_simple_objects}), this is an immediate consequence of lemma \ref{lemma: representations_of_finite_quivers_are_determined_by_dimension_vectors}.
                \end{proof}
            \begin{theorem}[Cartan quadratic forms as Euler characteristics] \label{theorem: tits_quadratic_forms_as_euler_characteristics}
                Let $k$ be a commutative ring and let $Q := (Q_1, Q_0, s, t)$ be a finite quiver, and let us denote the Cartan quadratic form of this quiver by $B_Q$ (cf. definition \ref{def: cartan_quadratic_forms_of_finite_quivers}). Then, there is a commutative diagram of $\Z$-linear maps as follows:
                    $$
                        \begin{tikzcd}
                        	{\K_0(Q)} & {\Z^{\oplus Q_0}} \\
                        	& \Z
                        	\arrow["{\rank_k}", from=1-1, to=1-2]
                        	\arrow["{q_Q}", from=1-2, to=2-2]
                        	\arrow["\chi_Q"', from=1-1, to=2-2]
                        \end{tikzcd}
                    $$
            \end{theorem}
                \begin{proof}
                    
                \end{proof}
            
        \subsubsection{(Co)reflection functors and Gabriel's Theorem}
            We have now come to Gabriel's Theorem. The proof thereof relies heavily on constructions known as \say{(co)reflection functors} (cf. definition \ref{def: (co)reflection_functors}) which can be thought of as gadgets that help us ensure that the representation theory of a quiver is independent of its orientation\footnote{So for instance, once we have proven Gabriel's Theorem, we can simply think of Dynkin quivers as those whose underlying undirected graph is a Dynkin diagram, and this notion is completely captured by the Cartan quadratic forms associated to these undirected graphs.}. 
            \begin{convention}
                From now on, we shall be working over a field $k$. \todo[inline]{Explain why working over a field is necessary.}
            \end{convention}
            \begin{definition}[(Co)sinks] \label{def: (co)sinks}
                
            \end{definition}
            \begin{convention}
                If $Q := (Q_1, Q_0, s, t)$ is a quiver and $\sigma \in Q_0$ is a (co)sink, then we shall implicitly write $(Q, \sigma)^{\op}$ for the quiver wherein one reverses all the arrow into/out of $\sigma$ and keep the remaining arrows unchanged.
            \end{convention}
            \begin{definition}[(Co)reflection functors] \label{def: (co)reflection_functors}
                Let $Q := (Q_1, Q_0, s, t)$ be a quiver and $\sigma \in Q_0$ be a fixed vertex. 
                    \begin{itemize}
                        \item If $v$ is a sink, then one can define the so-called \textbf{reflection functor}:
                            $$\Refl_{Q, \sigma}: \Rep_k(Q) \to \Rep_k(Q)$$
                            $$(\calF: [Q] \to k\mod) \mapsto \left( \Refl_{Q, \sigma}(\calF): [Q] \to k\mod: v \mapsto \begin{cases} \text{$\calF(v)$ if $v \in Q_0 \setminus \{\sigma\}$} \\ \text{$\underset{v \in [Q]_{/\sigma}}{\lim} \calF(v)$ if $v = \sigma$} \end{cases} \right)$$
                        \item If $v$ is a cosink, then one can instead define the \textbf{coreflection functor}:
                            $$\co\Refl_{Q, \sigma}: \Rep_k(Q) \to \Rep_k(Q)$$
                            $$(\calF: [Q] \to k\mod) \mapsto \left( \Refl_{Q, \sigma}(\calF): [Q] \to k\mod: v \mapsto \begin{cases} \text{$\calF(v)$ if $v \in Q_0 \setminus \{\sigma\}$} \\ \text{$\underset{v \in {}^{\sigma/}[Q]}{\colim} \calF(v)$ if $v = \sigma$} \end{cases} \right)$$
                    \end{itemize}
            \end{definition}
            \begin{convention}
                Let $Q := (Q_1, Q_0, s, t)$ be a quiver and $\sigma \in Q_0$ be a fixed vertex. Then, for all $\calF \in \Ob(\Rep_k(Q))$, we shall abuse notations slightly and write:
                    $$\Refl_{Q, \sigma}(\calF) := \underset{v \in [Q]_{/\sigma}}{\lim} \calF(v)$$
                    $$\co\Refl_{Q, \sigma}(\calF) := \underset{v \in {}^{\sigma/}[Q]}{\colim} \calF(v)$$
                when there is no risk of confusion. Keep in mind, over the remarks to come, that this abuse of notations of ours actually does not obscure any important categorical properties of the (co)reflection functors such as exactness.
            \end{convention}
            \begin{remark}[Categorical properties of (co)reflection functors] \label{remark: categorical_properties_of_(co)reflection_functors}
                Immediately, one sees that (co)reflection functors preserve (co)limits of representations. More specifically, if $Q := (Q_1, Q_0, s, t)$ is a quiver and $\sigma \in Q_0$ is a fixed vertex, and if:
                    $$I \to \Rep_k(Q)$$
                    $$i \mapsto \calF_i$$
                is a diagram of $k$-linear representations of $Q$, then clearly:
                    $$\Refl_{Q, \sigma}\left( \underset{i \in I^{\op}}{\lim} \calF_i \right) \cong \underset{i \in I}{\lim} \Refl_{Q, \sigma}(\calF)$$
                and:
                    $$\co\Refl_{Q, \sigma}\left( \underset{i \in I}{\colim} F_i \right) \cong \underset{i \in I}{\colim} \co\Refl_{Q, \sigma}(F)$$
                as a result of (co)limits commuting with one another. Of course, all the (co)limits at play here do indeed exist, since $\Rep_k(Q)$ is both complete and cocomplete as a result of $k\mod$ being so.
            \end{remark}
            \begin{remark}[(Co)reflection functors of finite quivers are exact] \label{remark: (co)reflection_functors_of_finite_quivers_are_exact}
                If:
                    $$Q := (Q_1, Q_0, s, t)$$
                is a \textit{finite} quiver with a fixed vertex $\sigma \in Q_0$ then the corresponding (co)reflection functors:
                    $$\Refl_{Q, \sigma}, \co\Refl_{Q, \sigma}: \Rep_k(Q) \to \Rep_k(Q)$$
                will obviously be defined via finite (co)limits. Because there is an exact equivalence of abelian categories as follows (cf. proposition \ref{prop: quiver_representations_are_modules_over_quiver_algebras}):
                    $$\Rep_k(Q) \to {}^lk\<Q\>\mod$$
                the aforementioned reflection functor $\Refl_{Q, \sigma}$ and coreflection functor $\co\Refl_{Q, \sigma}$ are, respectively, left-exact and right-exact. If, additionally, the diagram $[Q]_{/\sigma}$ is cofiltered (respectively, if ${}^{\sigma/}[Q]$ is filtered), such as when $Q$ is a Dynkin quiver\footnote{Cf. figures \ref{fig: A_n_quiver} and \ref{fig: D_n_quiver}.}, then one will be able to use the fact that (co)filtered limits are right-exact (respectively, filtered colimits are left-exact) in categories of modules over rings in order to see that in such a situation, both $\Refl_{Q, \sigma}$ and $\co\Refl_{Q, \sigma}$ are simultaneously left and right-exact (i.e. exact), and hence will \textit{preserve simple objects} (i.e. irreducible representations).
            \end{remark}
                
            \begin{theorem}[Gabriel's theorem on Dynkin quivers] \label{theorem: gabriel_theorem_on_dynkin_quivers}
                Let $k$ be a field and $\Gamma$ be a connected finite quiver. $\Gamma$ is then a Dynkin quiver if and only if $\Rep_k(\Gamma)$ has finitely many isomorphism classes of indecomposable objects.
            \end{theorem}
                \begin{proof}
                    
                \end{proof}
                
            \begin{example}[Gabriel's Theorem for $\sfA_n$ quivers]
                
            \end{example}
            \begin{example}[Gabriel's Theorem for $\sfD_n$ quivers]
                
            \end{example}
            \begin{example}[Gabriel's Theorem for $\sfE_n$ quivers]
                
            \end{example}
            \begin{example}[Gabriel's Theorem fails for the loop quiver]
                
            \end{example}
            \begin{example}[Gabriel's Theorem fails for the Kronecker quiver]
                
            \end{example}
        
    \subsection{Tilting modules and (derived) Morita equivalences}
        \subsubsection{(Partial) tilting objects}
        
        \subsubsection{Separating and splitting tilting objects}
        
        \subsubsection{Torsion induced by tilting objects}
        
    \subsection{Hereditary rings}    
        \subsubsection{Hereditary algebras and hereditary categories}
            \begin{definition}[Hereditary abelian categories] \label{def: hereditary_abelian_categories}
                An abelian category $\calA$ is said to be \textbf{hereditary}\footnote{If only the full subcategory $\calA^{\fin}$ of finite-length objects is of global dimension $\leq 1$ (i.e. if $\calA^{\fin}$ is hereditary) then one says that the larger abelian category $\calA$ is \textbf{semi-hereditary}.} if and only if $\globdim \calA \leq 1$.
            \end{definition}
            \begin{convention}[Left/right-hereditary rings] \label{conv: left/right_hereditary_rings}
                When $\calA$ is the category of left/right-module over some ring $R$, one might say that the ring $R$ itself is \textbf{left/right-hereditary}\footnote{A ring is left/right semi-hereditary if and only if all finitely generated modules over it are of projective dimension $\leq 1$.} whenever $\calA$ itself is hereditary.
            \end{convention}
            \begin{proposition}[Kernels in hereditary abelian categories are projective] \label{prop: kernels_in_hereditary_abelian_categories_are_projective}
                Let $\calA$ be a hereditary abelian category. Then, every kernel (i.e. sub-object) therein will be projective.
            \end{proposition}
                \begin{proof}
                    
                \end{proof}
            \begin{corollary}[Kaplansky's Theorem] \label{coro: kaplansky_theorem}
                If $R$ is a left/right-hereditary ring then every submodule of a free left/right-$R$-module will be a direct sum of left/right-$R$-ideals. As a direct result of this, every submodule of a projective left/right-$R$-module is also projective\footnote{Hence the terminology \say{hereditary}.}.
            \end{corollary}
            \begin{example}
                Any left/right-semi-simple category (hence any semi-simple algebra) is left/right-hereditary. Concrete examples of hereditary algebras include finite-dimensional matrix rings over division rings (cf. theorem \ref{theorem: artin_wedderburn}) and group algebras $k\<G\>$ of finite groups $G$ over semi-simple rings $k$ such that $\chara k \nmid |G|$ (cf. theorem \ref{theorem: maschke_theorem}).
                
                As for algebras which are strictly left/right-hereditary and not left/right-semi-simple, consider the algebra $\b_2^-(k)$ of lower-triangular $2 \x 2$ matrices over a field $k$.
            \end{example}
            
        \subsubsection{Admissible ideals; quivers with relations}
            \begin{convention}
                Let us henceforth assume that $k$ is field (i.e. a simple commutative ring). For the most part, this assumption is made so that given any quiver $Q := (Q_1, Q_0, s, t)$, the corresponding path $k$-algebra $k\<Q\>$ will admit $\<Q\> := \<[Q]_1\>$ as a maximal two-sided ideal, commonly referred to as the \say{arrow ideal} of $Q$.
            \end{convention}
            
            \begin{definition}[Admissible ideals of path algebras] \label{def: admissible_ideals_of_path_algebras}
                The path $k$-algebra of any given quiver is said to be \textbf{admissible} if and only if it is admissible in the sense of definition \ref{def: pre_admissible_and_pre_adic_rings}, and said to be \textbf{bounded} if and only if its ideal of definition is a finite power of the arrow ideal.
            \end{definition}
            
        \subsubsection{Quivers associated to finite-dimensional algebra}
            \begin{definition}[Basic algebras] \label{def: basic_algebras}
                An associative algebra $A$ over some commutative ring $k$ is said to be \textbf{basic} if and only if the semi-simple\footnote{Cf. proposition \ref{prop: semi_simple_iff_trivial_jacobson_radical_and_artinian}} $k$-algebra $A/\rad(A)$ admits an Artin-Wedderburn Decomposition (cf. theorem \ref{theorem: artin_wedderburn}) into finitely many copies of $k$, i.e.:
                    $$A/\rad(A) \cong \prod_{i = 1}^d k$$
                wherein the product is taken in the category of associative $k$-algebras.
            \end{definition}
            \begin{remark}
                Obviously, if $A$ is a basic algebra over some commutative ring $k$ then $A$ will be of finite rank as a $k$-module, which is equal to the number of copies of $k$ in the Artin-Wedderburn Decomposition of $A/\rad(A)$; that is to say, if:
                    $$A/\rad(A) \cong \prod_{i = 1}^d k$$
                then:
                    $$\rank_k A = d$$
            \end{remark}
            \begin{remark}[Simple modules over basic algebras are $1$-dimensional] \label{remark: simple_modules_over_basic_algebras_are_one_dimensional}
                Let $k$ be a commutative ring and $A$ be a basic $k$-algebra. By the Artin-Wedderburn Theorem (cf. theorem \ref{theorem: artin_wedderburn}), one sees that every simple (left/right-)$A$-modules are of rank $1$ as $k$-modules. Among other things, this means that in order to compute $\rad(A)$, which by definition contains all elements $a \in A$ which act as $0$ on simple (left/right-)$A$-modules, one needs to only check whether or not said elements act as $0$ on the underlying ring $k$.
            \end{remark}
            \begin{definition}[Primitive idempotents] \label{def: primitive_idempotents}
                An idempotent ring element $e \in R$ (i.e. $e^2 = e$) is said to be \textbf{primitive} if and only if the left-$R$-ideal ${}_R\<e\>$ ((or equivalently, the right-$R$-ideal $\<e\>_R$) is an indecomposable left/right-$R$-module.
            \end{definition}
            \begin{proposition}[Primitive idempotents are indecomposable] \label{prop: primitive_idempotents_are_indecomposable}
                \cite[Proposition 21.8]{lam_first_course_in_noncommutative_rings} Let $R$ be an associative ring and $e \in R$ be an idempotent element therein. Then the following are equivalent:
                    \begin{enumerate}
                        \item $e$ is primitive.
                        \item The ring $eRe$ has no non idempotent elements aside from $0$ and $1$.
                        \item There does not exist a decomposition $e := \alpha + \beta$ of $e$ into non-zero orthogonal (i.e. $\alpha \beta = \beta \alpha = 0$) idempotents $\alpha, \beta \in R$.
                    \end{enumerate}
            \end{proposition}
            \begin{example}
                Consider the matrix ring $\Mat_2(k)$ over some field $k$ of characteristic $0$. The element $\begin{pmatrix} 1 & 0 \\ 0 & 0 \end{pmatrix}$ is first of all idempotent, but moreoever, primitive, as the ring $\begin{pmatrix} 1 & 0 \\ 0 & 0 \end{pmatrix} \Mat_2(k) \begin{pmatrix} 1 & 0 \\ 0 & 0 \end{pmatrix}$ has no idempotents aside from the additive and multiplicative identities: one can easily show via some obvious computations that:
                    $$\begin{pmatrix} 1 & 0 \\ 0 & 0 \end{pmatrix} \Mat_2(k) \begin{pmatrix} 1 & 0 \\ 0 & 0 \end{pmatrix} = \left\{ \begin{pmatrix} a & 0 \\ 0 & 0 \end{pmatrix} \: \bigg| \: \forall a \in k: a^2 = a\right\}$$
                but since $k$ is a field of characteristic $0$, there are no elements $a \in k$ such that $a^2 = a$ aside from $0$ and $1$. On the other hand, the identity matrix $\begin{pmatrix} 1 & 0 \\ 0 & 1 \end{pmatrix}$ (which is also idempotent) is \textit{not} primitive, since one can easily write:
                    $$\begin{pmatrix} 1 & 0 \\ 0 & 1 \end{pmatrix} = \begin{pmatrix} 1 & 0 \\ 0 & 0 \end{pmatrix} + \begin{pmatrix} 0 & 0 \\ 0 & 1 \end{pmatrix}$$
                and it is trivial to check that the two summands are idempotent and orthogonal to one another.
            \end{example}
            \begin{remark}[Idempotents split] \label{remark: idempotents_split}
                
            \end{remark}
            \begin{definition}[Associated basic algebras] \label{def: associative_basic_algebras}
                Let $A$ be a finite algebra over a commutative ring $k$. To such a $k$-algebra, one can construct a basic one, denoted by ${}^bA$, which is given by:
                    $${}^bA := e_A A e_A$$
                wherein:
                    $$e_A := \sum_{e \in E} e$$
                for some set $E$ of pair-wise orthogonal\footnote{For all $e, e' \in E$, one has $ee' = e'e = 0$ whenever $e \not = e'$} primitive idempotents such that $|E| \leq \rank A$ and for all $e, e' \in E$, one has ${}_A\<e\> \not \cong {}_A\<e'\>$ (or $\<e\>_A \not \cong \<e'\>_A$) if $e \not = e'$. For the sake of brevity, let us call $E$ a \textbf{over-basis} of $A$ as a $k$-module.
            \end{definition}
            \begin{example}
                Let $k$ be a field of characteristic $0$ and consider the $k$-algebra $\Mat_2(k)$ of $2 \x 2$ matrices with entries in $k$; within this algebra, consider the over-basis\footnote{We will let the reader check that the elements herein are pair-wise orthogonal idempotents.}:
                    $$E := \left\{ \begin{pmatrix} 0 & 0 \\ 0 & 0 \end{pmatrix}, \begin{pmatrix} 1 & 0 \\ 0 & 0 \end{pmatrix}, \begin{pmatrix} 0 & 0 \\ 0 & 1 \end{pmatrix}, \begin{pmatrix} 1 & 0 \\ 0 & 1 \end{pmatrix} \right\}$$
                
            \end{example}
            
        \subsubsection{Revisiting Gabriel's Theorem}