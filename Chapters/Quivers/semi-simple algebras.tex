\section{Semi-simple algebras and homological dimension \texorpdfstring{$0$}{}}
    \subsection{(Semi-)simplicity and (in)decomposability}
        In this subsection, we recall and attempt to re-conceptualise the notions of (semi-)simplicity and (in)decomposability. Our approach is categorical, as our goal is to establish the fact that certain classical results on (semi-)simple and (in)decomposable modules/representations are actually just special cases of some basic categorical patterns. 
    
        \subsubsection{(Semi)-simplicity and the Jordan-H\"older Theorem}
            \begin{definition}[(Semi-)simple objects] \label{def: (semi)_simple_objects}
                Let $\C$ be a category with enough monomorphisms and epimorphisms and zero objects. A \textit{non-zero} object $X \in \Ob(\C)$ is said to be \textbf{simple} if and only if any monomorphism $\iota: X' \to X$ and any epimorphism $\pi: X \to X''$ is either zero (i.e. it factors through a zero object) or the identity $\id_X$. If $\C$ additionally has enough products, and if a (non-zero) object $X$ admits a product decomposition:
                    $$X \cong \prod_{i \in I} X_i$$
                into simple sub-objects $X_i \subseteq X$, then $X$ will be said to be \textbf{semi-simple}; a category wherein all objects are semi-simple is itself called \textbf{semi-simple}.
            \end{definition}
            \begin{example}[Irreducible representations]
                (Semi-)simplicity is not a relative notion, i.e. it does not behave well under base-change. For instance, if $A$ is an associative algebra over a field $k$, then there can be irreducible $k$-linear representations $V$ which, while being finitely generated as an $A$-module, is of infinite dimension as a $k$-vector space at the same time.
            \end{example}
            \begin{remark}[(Semi-)simple objects in abelian categories]
                If $\calA$ is an abelian category, then because $f: M \to N$ is a kernel (respectively, a cokernel) if and only if it is a monomorphism (respectively, an epimorphism), one might characterise simple objects of $\calA$ as those such that the kernel of any morphism:
                    $$\iota: M' \to M$$
                and and the cokernel of any morphism:
                    $$\pi: M \to M''$$
                are identically zero. 
            \end{remark}
            \begin{example}[Modules over semi-simple rings] \label{example: modules_over_semi_simple_rings}
                \cite[Theorem 2.5]{lam_first_course_in_noncommutative_rings} The following are equivalent are equivalent for any associative ring $R$:
                    \noindent
                    \begin{enumerate}
                        \item $R$ is left/right-semi-simple (as a left/right-module over itself).
                        \item All shorts exact sequences of left/right-$R$-modules split.
                        \item All left/right-$R$-modules are semi-simple.
                        \item All finitely generated left/right-$R$-modules are semi-simple\footnote{In fact, the category ${}^lR\mod^{\fin}$ (respectively ${}^rR\mod^{\fin}$) of finitely generated left/right-$R$-modules is - by definition (cf. definition \ref{def: finite_linear_categories}) - an instance of a finite (abelian) $\Z$-linear category, meaning that it is Jordan-H\"older (cf. definition \ref{def: jordan_holder_categories}) and Krull-Schmidt (cf. definition \ref{def: krull_schmidt_categories}).}. 
                        \item All cyclic left/right-$R$-modules are semi-simple. 
                    \end{enumerate}
                Furthermore, an associative ring is left-semi-simple if and only if it is right-semi-simple.  
            \end{example}
            \begin{lemma}[Schur's Lemma for abelian categories] \label{lemma: schur_lemma_for_abelian_categories}
                Let $\calA$ be an abelian category and $f: M \to N$ be a non-zero morphism between two simple objects $M$ and $N$. Such a morphism must then be an isomorphism.
            \end{lemma}
                \begin{proof}
                    Since both $M$ and $N$ are simple objects of $\calA$ and since $\calA$ is abelian, the lemma comes from consideration of the following commutative diagram and from the assumption that $f: M \to N$ is a non-zero morphism:
                        $$
                            \begin{tikzcd}
                            	0 & M & 0 \\
                            	0 & N & 0
                            	\arrow["f", from=1-2, to=2-2]
                            	\arrow[from=1-1, to=2-1]
                            	\arrow[from=2-1, to=2-2]
                            	\arrow["{\ker f}", from=1-1, to=1-2]
                            	\arrow[from=1-2, to=1-3]
                            	\arrow["{\coker f}", from=2-2, to=2-3]
                            	\arrow[from=1-3, to=2-3]
                            	\arrow["\lrcorner"{anchor=center, pos=0.125, rotate=180}, draw=none, from=2-3, to=1-2]
                            	\arrow["\lrcorner"{anchor=center, pos=0.125}, draw=none, from=1-1, to=2-2]
                            \end{tikzcd}
                        $$
                \end{proof}
            \begin{corollary}[Morphisms into and out of simple objects] \label{coro: morphisms_into_and_out_of_simple_objects}
                Let $\calA$ be an abelian category and let $M$ be a simple object. Then, any morphism $\pi: P \to M$ must be epic and any morphism $\iota: M \to Q$ must be monic.
            \end{corollary}
                \begin{proof}
                    Since $\calA$ is an abelian category, any monomorphism/epimorphism must be a (co)kernel, and because $M$ is simple, both $\co\ker \pi$ and $\ker \iota$ must be trivial. 
                \end{proof}
            \begin{corollary}[Endomorphism algebras of simple objects] \label{coro: endomorphism_algebras_of_simple_objects_in_abelian_categories}
                Let $\calA$ be an abelian category and let $M$ be a simple object. Then $\calA(M, M)$ will be a division ring (with respect to compositions of endomorphisms on $M$), and if $N \not \cong M$ is another simply object, then $\calA(M, N) \cong 0$. 
            \end{corollary}
                \begin{proof}
                    Both assertions are direct consequences of lemma \ref{lemma: schur_lemma_for_abelian_categories}.
                \end{proof}
            \begin{example}[Irreducible complex representations]
                Let $k$ be a commutative ring, let $A$ be an associative $k$-algebra (e.g. one might consider the group algebra $A := k\<G\>$ of some abstract group $G$), and let $V$ and $W$ be two irreducible $k$-linear representations of $A$, viewed as simple (left-)$A$-modules. Then, any $A$-linear map $f: M \to N$ (or equivalently, any map of $k$-linear representations of $A$) will either be zero or an isomorphism; when $k$ is an algebraically closed field of characteristic $0$ (e.g. $k \cong \bbC$), such an isomorphism will also have to be given by (left-)multiplication by a scalar $a \in k$ (the scalar $1_k$ corresponds to the identity map)\footnote{For more details, see remark \ref{prop: endomorphism_algebras_of_simple_objects_in_locally_finite_linear_categories}}.
            \end{example}
                
            \begin{definition}[Lengths and Jordan-H\"older series] \label{def: lengths_of_objects_and_jordan_holder_series}
                Let $\C$ be a category with enough kernels and cokernels. An object $X \in \Ob(\C)$ is then said to be of \textbf{length} $n$ (for some $n \in \N$) if and only if there exists a so-called \textbf{Jordan-H\"older filtration} of \textit{normal} monomorphisms:
                    $$0 =: X_0 \subseteq X_1 \subseteq X_2 \subseteq ... \subseteq X_n \subseteq X$$
                such that the quotients (i.e. cokernels) $X_{i + 1}/X_i$ are simple for all $0 \leq i \leq n - 1$; these quotients are commonly called \textbf{Jordan-H\"older factors}. Such a filtration is said to be of \textbf{multiplicity} $m \geq 0$ if and only if the number of isomorphic factors is $m$. 
            \end{definition}
            \begin{convention}
                Zero objects are of length $0$.
            \end{convention}
            \begin{remark}
                Obviously, semi-simple objects are of finite lengths. 
            \end{remark}
            \begin{theorem}[Uniqueness of Jordan-H\"older filtrations] \label{theorem: jordan_holder_theorem}
                Let $\C$ be a category with kernels and cokernels; assume also that $\C$ has enough cokernels. Next, consider an object $X \in \Ob(\C)$ that is of some finite length $n \geq 0$. Then, any Jordan-H\"older filtration of $X$ must be of length $n$.
            \end{theorem}
                \begin{proof}
                    Let $X_{\bullet} := \{X_i\}_{0 \leq i \leq n}$ and $Y_{\bullet} := \{Y_j\}_{0 \leq j \leq m}$ be two Jordan-H\"older filtrations of $X$ and consider some $f_{\bullet} \in \C^{\N}(X_{\bullet}, Y_{\bullet})$. Taking cokernels is functorial, so let us consider the resulting map between factors $\coker f_{\bullet} \in \C^{\N}(X_{\bullet + 1}/X_{\bullet}, Y_{\bullet + 1}/Y_{\bullet})$. By Schur's Lemma (cf. lemma \ref{lemma: schur_lemma_for_abelian_categories}), this is either zero or an isomorphism term-wise, which tells us that $m = n$. 
                \end{proof}
            \begin{example}
                The Jordan-H\"older theorem holds in any semi-abelian category (such as that of groups) and more particularly, any abelian category (such as categories of modules over rings or categories of linear representations).
            \end{example}
            \begin{definition}[Jordan-H\"older categories] \label{def: jordan_holder_categories}
                A \textbf{Jordan-H\"older category} is one wherein every object is of finite length.
            \end{definition}
        
        \subsubsection{Decomposability and the Krull-Schmidt Theorem}
            \begin{definition}[(In)decomposable objects] \label{def: (in)decomposable_objects}
                An object $X$ of a category $\C$ with enough products is said to be \textbf{decomposable} if there exists a family $\{X_i\}_{i \in I}$ of sub-objects $X_i \subseteq X$ such that:
                    $$X \cong \prod_{i \in I} X_i$$
                Otherwise, $X$ is said to be \textbf{indecomposable}.
            \end{definition}
            \begin{remark}
                Semi-simple objects are decomposable, while simple objects are always indecomposable. There can - however - be indecomposable objects which are not simple (e.g. any non-field commutative ring as a module over itself). 
            \end{remark}
            \begin{lemma}[Characterisations of local associative algebras] \label{lemma: characterisations_of_local_associative_algebras}
                \cite[Theorem 19.1]{lam_first_course_in_noncommutative_rings} Let $k$ be a commutative ring. The following are equivalent for any non-zero associative $k$-algebra $(A, +, \cdot)$:
                    \noindent
                    \begin{enumerate}
                        \item $A$ has a unique maximal left-ideal.
                        \item $A$ has a unique maximal right-ideal. 
                        \item $A/\rad(A)$ is a division $k$-algebra (with $\rad(A)$ denoting the Jacobson radical of $A$).
                        \item $A \setminus A^{\x}$ is an $A$-ideal.
                        \item $(A \setminus A^{\x}, +)$ is a group. 
                    \end{enumerate}
            \end{lemma}
            \begin{definition}[Local associative algebras] \label{def: local_associative_algebras}
                An associative algebra $A$ over a commutative ring $k$ is said to be \textbf{local} if and only if it satisfies one of the equivalent conditions of lemma \ref{lemma: characterisations_of_local_associative_algebras}. Note that this is a two-sided notion.
            \end{definition}
            \begin{proposition}[Endormophism algebras of indecomposable objects] \label{prop: endomorphism_algebras_of_indecomposable_objects}
                For some commutative ring $k$, let $\C$ be a $k$-linear category with enough products and let $Z$ be an indecomposable object. Then $\C(Z, Z)$ will be a local associative $k$-algebra. 
            \end{proposition}
                \begin{proof}
                    
                \end{proof}
            \begin{definition}[Krull-Schmidt categories] \label{def: krull_schmidt_categories}
                Let $k$ be a commutative ring. A \textbf{$k$-linear Krull-Schmidt category} is a category is a $k$-linear category in which every object is decomposable into \textit{finitely many} indecomposable factors.
            \end{definition}
            
            \begin{theorem}[Krull-Schmidt Decomposition Theorem] \label{theorem: krull_schmidt_theorem}
                Suppose that $\C$ is a category with enough (co)kernels and products, and that $X \in \Ob(\C)$ is an object of finite length. Then, there exists a (necessarily finite) product decomposition:
                    $$X \cong \prod_{i \in I} X_i$$
                of $X$ into indecomposable sub-objects $X_i \subseteq X$, which is unique up to isomorphisms.
            \end{theorem}
                \begin{proof}
                    Suppose to the contrary that there exists a finite-length object $X \in \Ob(\C)$ which does not admit a (finite) product decomposition into indecomposable sub-objects. Particularly, this means that $X$ is not indecomposable (because if so $X$ would just be its own unary product decomposition into indecomposable subobjects), meaning that by definition, $X$ is decomposable. Let:
                        $$X \cong \prod_{j \in J^{(1)}} Z_j^{(1)}$$
                    be a (necessarily finite) product decomposition of $X$ into sub-objects $Z_j^{(1)} \subseteq X$; the sub-objects $Z_j^{(1)}$ can not be simultaneously indecomposable, so there must exist at least one $j_1 \in J^{(1)}$ such that $Z_{j_1}^{(1)}$ is decomposable. By repeating this argument, one obtains the a (finite) product decomposition of $X$ as follows:
                        $$X \cong \left( \prod_{j \in J^{(1)} \setminus j_1} Z_j^{(1)} \right) \x Z_{j_1}^{(1)} \cong \left( \prod_{j \in J^{(1)} \setminus j_1} Z_j^{(1)} \right) \x \left( \prod_{j \in J^{(2)} \setminus j_2} Z_j^{(2)} \right) \x Z_{j_2}^{(2)} \cong ...$$
                    which yields the following \textit{infinite} filtration on $X$:
                        $$X \supseteq Z_{j_1}^{(1)} \supseteq Z_{j_2}^{(2)} \supseteq ...$$
                    This contradicts the hypothesis that $X$ is of finite length, and therefore $X$ is decomposable into finitely many indecomposable factors.
                \end{proof}
            \begin{example}
                The Krull-Schmidt Decomposition Theorem is, particularly, applicable within the category of groups and any category of modules over an arbitrarily given associative ring. More specifically (and perhaps more pertinent to our needs), the theorem is holds within categories of representations (e.g. of groups, Lie algebras, Hopf algebras, and even of monoids\footnote{Observe that all such objects admit universally defined }). 
            \end{example}
            \begin{example}[Cyclic decomposition of finitely generated modules over PIDs] \label{example: cyclic_decomposition_theorem_for_finitely_generated_modules_over_PIDs}
                An enhanced specialisation of the Krull-Schmidt Decomposition Theorem is the Cyclic Decomposition Theorem for finitely generated modules over PIDs $R$. Such modules are certainly of finite length (which are equal to the cardinalities of their generating sets) and as such - per the Krull-Schmidt Decomposition Theorem - admits a finite product (in fact, direct sum) decomposition into indecomposable modules. The additional lemma to prove for the Cyclic Decomposition Theorem is that over PIDs, indecomposable modules are always cyclic\footnote{Though the converse is not necessarily true (e.g. $\Z/6\Z \cong \Z/2\Z \oplus \Z/3\Z$ as a $\Z$-module).}; more particularly, one has:
                    \begin{enumerate}
                        \item a finitely generated module over a given PID is torsion-free if and only if it is free, and 
                        \item finitely generated torsion modules over PIDs $R$ can be decomposed as:
                            $$\bigoplus_{\p \in \Spec R} (R/\p)^{\oplus e_{\p}}$$
                        for some finitely supported set of powers $\{e_{\p} \in \N \mid \p \in \Spec R\}$.
                    \end{enumerate}
                
                A trivial yet interesting corollary of the Cyclic Decomposition Theorem is that over a given PID $R$, the canonical short exact sequence:
                    $$0 \to \Tor_R(M) \to M \to M/\Tor_R(M) \to 0$$
                splits for any finitely generated $R$-module $M$ (here, $\Tor_R(M)$ denotes the torsion $R$-submodules of $M$).
                
                In turn, the Cyclic Decomposition Theorem admits its own special case, that being the Chinese Remainder Theorem: given any integer $N$ and a fixed prime factorsiation $N := \prod_{i \in I} p_i^{e_i}$ thereof, one has an isomorphism of abelian groups as follows:
                    $$\Z/N\Z \cong \bigoplus_{i \in I} (\Z/p_i\Z)^{e_i}$$
            \end{example}
            
        \subsubsection{(Locally) finite abelian categories}
            \begin{definition}[Locally finite linear categories] \label{def: locally_finite_linear_categories}
                Let $k$ be a commutative ring. A $k$-linear category $\E$ is said to be \textbf{locally finite}\footnote{Also called \textbf{Artinian}.} if:
                    \begin{enumerate}
                        \item for all objects $X, Y \in \Ob(\C)$, the hom-space $\E(X, Y)$ is finitely generated as a $k$-module, and
                        \item every object of $\E$ is of finite-length.
                    \end{enumerate}
            \end{definition}
            \begin{remark}[Locally finite linear categories are Jordan-H\"older and Krull-Schmidt] \label{remark: locally_finite_linear_categories_are_jordan_holder_and_krull_schmidt}
                Clearly, every locally finite linear category is simultaneously Jordan-H\"older (this is tautological) and Krull-Schmidt (because otherwise, there would exist objects of infinite length; cf. the proof of theorem \ref{theorem: krull_schmidt_theorem}).
            \end{remark}
            \begin{proposition}[Endomorphism algebras of simple objects in locally finite linear categories] \label{prop: endomorphism_algebras_of_simple_objects_in_locally_finite_linear_categories}
                Let $k$ be a commutative and $\E$ be a locally finite $k$-linear abelian category. Then $\E(Z, Z)$ will be a division $k$-algebra for all simple objects $Z \in \Ob(\E)$ and in particular, when $k$ is an algebraically closed field, one has $\E(Z, Z) \cong k$.
            \end{proposition}
                \begin{proof}
                    The first assertion is just from corollary \ref{coro: endomorphism_algebras_of_simple_objects_in_abelian_categories}. The second comes from the fact if $D$ is a division algebra which is finite-dimensional over some algebraically closed field $k$, then in fact $D \cong k$ (to prove this, suppose that there exists an element $\alpha \in D \setminus k$ and then consider the field extension\footnote{Note that because $k$ is a $k$-subalgebra of the centre $\rmZ(D) \subseteq D$ per the definition of algebras over a ring, the element $\alpha$ must commute with all elements of $k$, and therefore $k(\alpha)$ is actually a $k$-algebra that is a field by construction.} $k(\alpha)/k$, which is \textit{a priori} algebraic\footnote{Since $k(\alpha)$ is finite-dimensional as a $k$-vector space, by virtue of being a $k$-vector subspace of $D$ (which is finite-dimensional over $k$ by assumption).}: since $k$ is algebraically closed, the minimal polynomial of $\alpha$ must be $(x - \alpha) \in k[x]$, but this in turn implies - via the hypothesis that $k$ is algebraically closed and the fact that the extension $k(\alpha)/k$ is algebraic - that $\alpha \in k$, which is clearly a contradiction, and thus $D \cong k$).
                \end{proof}
                
            \begin{definition}[Finite linear categories] \label{def: finite_linear_categories}
                For any Artinian\footnote{So that left/right-$A$-module are finitely generated if and only if it is of finite length.} commutative ring $k$, a $k$-linear (abelian) category $\E$ is said to be \textbf{finite} if and only if there exists an \textit{exact} equivalence:
                    $$F: \E \to {}^lA\mod$$
                to the category $A\mod$ of left-modules\footnote{Which are necessarily finitely generated.} over some Artinian $k$-algebra $A$, which we shall call the algebra of \textbf{coefficients} of $\E$.
            \end{definition}
            \begin{proposition}[Finite linear categories are locally finite] \label{prop: finite_linear_categories_are_locally_finite}
                Let $k$ be an Artinian commutative ring. Then, any finite $k$-linear category will also be locally finite. 
            \end{proposition}
                \begin{proof}
                    Combine definitions \ref{def: locally_finite_linear_categories} and \ref{def: finite_linear_categories}, and note that categories of finitely generated modules over finite $k$-algebras are indeed locally finite as $k$-linear categories. 
                \end{proof}
            \begin{proposition}[Finite linear categories have enough projectives] \label{prop: finite_linear_categories_have_enough_projectives}
                Let $k$ be an Artinian commutative ring. Every finite $k$-linear category has enough projectives.
            \end{proposition}
                \begin{proof}
                    This is tautologically obvious because every left-module $M$ over a ring $A$ admits a free (hence projective) resolution $\e_{\bullet}: P_{\bullet} \to M$; the existence of the two-term complex $\e_1: P_1 \to M \to 0$ then tells us that indeed, finite linear categories have enough projectives. 
                \end{proof}
            
            \begin{lemma}[Simple modules over semi-simple rings] \label{lemma: simple_modules_over_semi_simple_rings}
                Let $R$ be a semi-simple associative ring. Then there are only finitely many isomorphism classes of simple left-$R$-modules. 
            \end{lemma}
                \begin{proof}
                    For the sake of simplicity (and not at the expense of generality), suppose that the given semi-simple ring $R$ admits a product decomposition (in the category of rings) as follows:
                        $$R \cong A \x B$$
                    (wherein, of course, $A$ and $B$ are simple). It is then easy to see that every left-$R$-module $M$ decomposes as:
                        $$M \cong P \oplus N$$
                    wherein $P := (1_A, 0_B)M$ is a left-$A$-module and $N := (0_A, 1_B)M$ is a left-$B$-module. It then follows that $M$ is simple if either $P$ is simple and $N \cong 0$, or \textit{vice versa}.; this means that the number of simple left-$R$-modules is the sum of the number of simple left-modules over $A$ and those over $B$.
                    
                    From here, our task becomes to prove that there are only finitely many isomorphism classes of simple left-modules over a given simple ring $R$. For this, recall first of all that because $R$ is simple (hence semi-simple), every simple left-$R$-module must be left-cyclic (cf. example \ref{example: modules_over_semi_simple_rings}). In other words, simple left-$R$-modules are precisely the left-ideals of $R$, and since $R$ is simple, the only isomorphism class of such modules is that of ${}_R(0)$. Thus we are done. 
                \end{proof}
            \begin{example}[$1$-dimensional vector spaces]
                There is exactly one isomorphism class of simple vector spaces over a fixed field $k$ (which is indeed semi-simple, since fields do not admit any non-zero proper ideal), which is precisely the isomorphism class of $1$-dimensional $k$-vector spaces. 
            \end{example}
            \begin{example}[Irreducible representations]
                If $A$ is an associative algebra over a commutative ring $k$, then there will only be finitely many isomorphism classes of irreducible $k$-linear representations on $A$. 
            \end{example}
            \begin{remark}
                The assumption that $k$ is a field (as opposed to being any commutative ring) is crucial to validity of the following proposition. This is because we need the algebra of coefficient $A$ to be semi-simple. 
            \end{remark}
            \begin{proposition}[Simple objects in finite linear categories] \label{prop: simple_objects_in_finite_linear_categories}
                Let $k$ be a field. Every finite $k$-linear category has finitely many isomorphism classes of simple objects. 
            \end{proposition}
                \begin{proof}
                    Combine lemma \ref{lemma: simple_modules_over_semi_simple_rings} with proposition \ref{prop: semi_simple_iff_trivial_jacobson_radical_and_artinian}.
                \end{proof}
        
    \subsection{Homological dimensions}
        \subsubsection{Projective and injective dimensions; global dimensions}
            \begin{lemma}[Schaunel's Lemma] \label{lemma: schaunel_lemma}
                \cite[\href{https://stacks.math.columbia.edu/tag/00O3}{Tag 00O3}]{stacks}
            \end{lemma}
                \begin{proof}
                    
                \end{proof}
        
            \begin{definition}[Projective and injective dimensions] \label{def: projective_and_injective_dimensions}
                Let $\calA$ be an abelian category and $M \in \Ob(\calA)$ some arbitrary object therein. $M$ is said to be of \textbf{projective/injective dimension} $d \in \N \cup \{\infty\}$ if and only if any projective/injective resolution of $M$ is quasi-isomorphic to a projective/injective resolution of length $d$.
            \end{definition}
            \begin{example}
                Projective/injective modules are of projective/dimension $0$.
            \end{example}
            \begin{remark}
                Note that an object $M \in \Ob(\calA)$ of an abelian category $\calA$ is of injective dimension $d$ if and only if it is of projective dimension $d$ in $\calA^{\op}$. For this reason, the results below will only be stated in terms of projectives instead of projectives and injectives respectively.
            \end{remark}
            \begin{convention}[Resolutions]
                Let $\calA$ be an abelian category and $M \in \Ob(\calA)$ some arbitrary object therein. Then, whenever we want to consider a resolution of $M$ by projectives (supposing that such a resolution exists), we shall mean a long exact sequence of the form:
                    $$
                        M_{\bullet} :=
                        \left\{
                            \begin{tikzcd}
                            	\cdots & {M_{-2}} & {M_{-1}} & M & 0
                            	\arrow[from=1-1, to=1-2]
                            	\arrow["{\del_2}", from=1-2, to=1-3]
                            	\arrow["{\del_1}", from=1-3, to=1-4]
                            	\arrow[from=1-4, to=1-5]
                            \end{tikzcd}
                        \right\}
                    $$
                In particular, note that $M$ is placed in degree $0$, so usually, we will write:
                    $$M_{\bullet}[0] := M$$
                Also, note the \textit{cohomological} indexing.
            \end{convention}
            \begin{lemma}
                (Cf. \cite[\href{https://stacks.math.columbia.edu/tag/00O5}{Tag 00O5}]{stacks}) Let $\calA$ be an abelian category with enough projectives and let $(M_{\bullet}, \del_{\bullet}) \in \Ob(\rmD^{[-e, 0]}_{\proj}(\calA))$ be a projective resolution. Suppose also that the $M_{\bullet}[0]$ is of some finite projective dimension $d \in \N$. Then, the kernel $\ker(\del_{-e}: M_{-e} \to M_{-(e - 1)})$ will be a projective object of $\calA$.
            \end{lemma}
                \begin{proof}
                    
                \end{proof}
            \begin{proposition}
                (Cf. \cite[\href{https://stacks.math.columbia.edu/tag/0CXC}{Tag 0CXC}]{stacks}) Let $\calA$ be an abelian category with enough projectives and let $M \in \Ob(\calA)$ be an object; also let $d \in \N$ be a fixed natural number. Then, the following are equivalent:
                    \begin{enumerate}
                        \item $\projdim M \leq d$.
                        \item $M$ admits a projective resolution of length $d$.
                        \item 
                        \item 
                    \end{enumerate}
            \end{proposition}
                \begin{proof}
                    
                \end{proof}
            \begin{proposition}[Computing projective dimensions using $\Ext^*$] \label{prop: computing_projective_dimensions_using_Ext}
                Let $\calA$ be an abelian category with enough projectives and $M \in \Ob(\calA)$ be an arbitrary object thereof. Also, fix a natural number $d \in \N$. Then, the following are equivalent:
                    \begin{enumerate}
                        \item $\projdim M \leq d$.
                        \item $\Ext_{\calA}^i(M, N) \cong 0$ for all $N \in \Ob(\calA)$ and all $i \geq d + 1$.
                        \item $\Ext_{\calA}^{d + 1}(M, N) \cong 0$ for all $N \in \Ob(\calA)$.
                    \end{enumerate}
            \end{proposition}
                \begin{proof}
                    
                \end{proof}
            \begin{proposition}[Projective dimensions and short exact sequences] \label{prop: projective_dimensions_and_short_exact_sequences}
                
            \end{proposition}
                \begin{proof}
                    
                \end{proof}
                
            \begin{definition}[Global dimensions] \label{def: global_dimensions}
                The \textbf{global dimension} of an abelian category $\calA$ (often assumed to have enough projectives), denoted by $\globdim \calA$, is given by:
                    $$\globdim \calA := \sup_{M \in \Ob(\calA)} \projdim M$$
            \end{definition}
            \begin{convention}[Global dimensions of rings] \label{conv: global_dimensions_of_rings}
                When $\calA$ is the category of left/right-module over some ring $R$, one might speak instead of the \textbf{left/right-global dimension} of the ring $R$ itself instead of the global dimension of the abelian category of left/right-$R$-modules. 
            \end{convention}
            
            \begin{lemma}[Projective and injective modules over semi-simple rings] \label{lemma: projective_and_injective_modules_over_semi_simple_rings}
                \cite[Theorem 2.8]{lam_first_course_in_noncommutative_rings} The following are equivalent for any given associative ring $R$:
                    \begin{enumerate}
                        \item $R$ is left/right semi-simple.
                        \item Left/right-$R$-modules are projective (respectively, injective).
                        \item Finitely generated left/right-$R$-modules are projective (respectively, injective).
                        \item Cyclic left/right-$R$-modules are projective (respectively, injective).
                    \end{enumerate}
            \end{lemma}
            \begin{corollary}
                The category of left/right modules over a semi-simple ring has enough projectives and injectives.
            \end{corollary}
            \begin{corollary}[(Co)homology of modules over semi-simple rings] \label{coro: (co)homology_of_modules_over_semi_simple_rings}
                Let $R$ be a semi-simple associative ring and let $M$ be an arbitrary right-$R$-module. Then:
                    \begin{enumerate}
                        \item $\Ext_R^i(M, N) \cong 0$ for all right-$R$-modules $N$ and all indices $i > 0$. 
                        \item $\Tor^R_i(M, P) \cong 0$ for all left-$R$-modules $P$ and all indices $i > 0$. 
                    \end{enumerate}
            \end{corollary}
            \begin{definition}[Jacobson radicals] \label{def: jacobson_radicals}
                The \textbf{Jacobson radical} of an associative ring $R$ is the two-sided $R$-ideal, denoted by $\rad(R)$, that is maximal among two-sided $R$-ideal whose elements all act as $0$ on simple $R$-modules.
            \end{definition}
            \begin{remark}[Jacobson radicals as intersections of kernels] \label{remark: jacobson_radicals_as_intersections_of_kernels}
                Another way to view the Jacobson radical of a given associative $k$-algebra $A$ (for any commutative ring $k$) is that:
                    $$\rad(A) \cong \bigcap_{(\rho, V) \in \Ob(\Rep_k^{\irr}(A))} \ker \rho$$
                The kernels $\ker \rho$ are two-sided $A$-ideals, which ensures that $\rad(A)$ is also a two-sided $A$-ideal.
            \end{remark}
            \begin{proposition}[Jacobson radicals are nilpotent] \label{prop: jacobson_radicals_are_nilpotent} 
                The Jacobson radical is maximal among nilpotent two-sided ideals. 
            \end{proposition}
                \begin{proof}
                    
                \end{proof}
            \begin{example}[Two-sided ideals in semi-simple rings are trivial] \label{example: two_sided_ideals_in_semi_simple_rings_are_trivial}
                Let $R$ be a semi-simple associative ring and let $M$ be an arbitrary right-$R$-module. Because $\Tor^R_i(M, P) \cong 0$ for all left-$R$-modules $P$ and all indices $i > 0$, take $P \cong {}_RR/I$ for any two-sided $R$-ideal $I$ (e.g. the Jacobson radical $\rad(R)$) to see that:
                    $$\Tor^R_1(M, {}_RR/I) \cong \{x \in M \mid \forall a \in I: xa = 0\} \cong 0$$
                In particular, one has:
                    $$\Tor_1^R(R_R, {}_RR/I) \cong I \cong 0$$
                which tells us that semi-simple rings do not admit any non-trivial two-sided ideals (so for instance, $\rad(R) = 0$ when $R$ is semi-simple). 
                
                The converse statement is true by definition, so we have obtained a characterisation of semi-simple rings via their two-sided ideals (or rather, the lack thereof). This will be used in our proof of Maschke's Theorem (cf. theorem \ref{theorem: maschke_theorem}).
            \end{example}
            \begin{example}[Ideals in matrix rings and semi-simplicity of matrix rings]
                If $R$ is an associative ring and $n > 0$ is any positive integer then there are bijective functions:
                    $$\Mat_n(-): \{\text{left/right/two-sided ideals of $R$}\} \to \{\text{left/right/two-sided ideals of $\Mat_n(R)$}\}$$
                In particular, this tells us that $\Mat_n(R)$ is (semi-)simple if and only if $R$ is (semi-)simple, e.g. $R$ is a product of division rings (so in particular, fields and products thereof).
                
                We should also note that this is actually a very special case of something much more general and much deeper, called \say{Morita equivalence}: essentially, two associative rings $R$ and $S$ are Morita-equivalent if and only if there is an (adjoint and additive) equivalence\footnote{An equivalent version holds for right-modules.}:
                    $$
                        \begin{tikzcd}
                        	{{}^lR\mod} & {{}^lS\mod}
                        	\arrow[""{name=0, anchor=center, inner sep=0}, "{\Hom_S(P, -)}"', bend right, from=1-1, to=1-2]
                        	\arrow[""{name=1, anchor=center, inner sep=0}, "{- \tensor_R P}"', bend right, from=1-2, to=1-1]
                        	\arrow["\dashv"{anchor=center, rotate=-90}, draw=none, from=1, to=0]
                        \end{tikzcd}
                    $$
                wherein $P$ is an $(S, R)$-bimodule which is a finitely generated projective generator of the category of left/right-$R$-modules, such that $S \cong \End_R(P)$. For $n \x n$ matrices, one takes:
                    $$P := R^{\oplus n}$$
                so that:
                    $$S \cong \End_R(R^{\oplus n}) \cong \Mat_n(R)$$
            \end{example}
            \todo[inline]{Write something about left/right-global dimensions of semi-simple rings here}
    
        \subsubsection{Flat dimensions, weak dimensions, and the Artin-Wedderburn Theorem}
            \begin{remark}
                Some other results that could also be included here are lemma \ref{lemma: simple_modules_over_semi_simple_rings} and example \ref{example: modules_over_semi_simple_rings} (which in particular informs us of the trivial fact that left/right-semi-simple algebras are left/right Noetherian and left/right Artinian; cf. \cite[Corollary 2.6]{lam_first_course_in_noncommutative_rings}).
            \end{remark}
            
            \begin{theorem}[Artin-Wedderburn] \label{theorem: artin_wedderburn}
                Let $k$ be a commutative ring and $A$ be a left/right-semi-simple associative $k$-algebra. Then, $A$ admits a finite product decomposition as follows, indexed by the (finitely many) irreducible $k$-linear representations of $A$:
                    $$A \cong \prod_{V \in \Ob(\Rep_k^{\irr}(A))} \End_k(V)$$
            \end{theorem}
                \begin{proof}
                    
                \end{proof}
            \begin{corollary}[The Artin-Wedderburn Theorem over fields] \label{coro: artin_wedderburn_over_fields}
                
            \end{corollary}
                \begin{proof}
                    
                \end{proof}
            \begin{proposition}[Artinian rings with trivial Jacobson radicals are semi-simple] \label{prop: semi_simple_iff_trivial_jacobson_radical_and_artinian}
                An Artinian ring $R$ is semi-simple if and only if its Jacobson radical $\rad(R)$ is zero.
            \end{proposition}
                \begin{proof}
                    
                \end{proof}
            \begin{corollary}[The Sum-Of-Squares Formula for semi-simple algebras] \label{coro: sum_of_squares_formula_for_semi_simple_algebras}
                Let $A$ be an associative algebra over some field $k$. Then:
                    $$\dim_k(A) \geq \sum_{V \in \Ob(\Rep_k^{\irr}(A))} (\dim_k V)^2$$
                and equality occurs if and only if $A$ is semi-simple.
            \end{corollary}
        
    \subsection{Representations of finite groups}
        In this subsection, we turn our focus to representations of finite groups, which we view as modules over group algebras. Our goal is to tabulate a list of important properties, as well as to give a guide on how one might perform explicit computations on these representations.
        
        \subsubsection{Group cohomology; induced and restricted representations}
            \begin{definition}[Group cohomology] \label{def: group_cohomology}
                Let $\E$ be a topos wherein the category $\Ab(\E)$ of internal abelian groups has enough projectives; recall that $\Ab(\E)$ is \textit{a priori} an infinite $\Z_{\E}$-linear tensor category\footnote{This is like \cite[Definition 4.1.1]{EGNO}, but without the local finiteness condition.}, so let us denote its monoidal unit by $\Z_{\E}$. Simultaneously, fix a group object $G \in \Ob(\Grp(\E))$. Then, the functor of \textbf{$\Ab(\E)$-valued group cohomology} of $G$, denoted by:
                    $$H^*_{\Grp(\E)}(G, -)$$
                shall be nothing but:
                    $$\Ext^*_{\Z_{\E}\<G\>}(\Z_{\E}, -)$$
                wherein we view $\Z_{\E}$ as a left-$\Z_{\E}\<G\>$-module via augmentation.
            \end{definition}
            \begin{remark}[How to compute group cohomologies ?]
                The reason that we have insisted in definition \ref{def: group_cohomology} that $\Ab(\E)$ must have enough enough projectives is that in order to compute the cohomology groups:
                    $$H^i_{\Grp(\E)}(G, M) := \Ext^i_{\Z_{\E}\<G\>}(\Z_{\E}, M)$$
                (for any abelian group $M \in \Ob(\Ab(\E))$), one shall have to choose some projective resolution $(P_{\bullet}, \e_{\bullet})$ for $\Z_{\E} \in \Ob(\Ab(\E))$ (which really ought to be viewed as an injective resolution in $\Ab(\E)^{\op}$) and then calculate:
                    $$\Ext^i_{\Z_{\E}\<G\>}(\Z_{\E}, M) := H^i\left( \R\Hom_{\Z_{\E}\<G\>}(P_{\bullet}, M) \right)$$
                Since by definition, one views $\Z_{\E}$ as a left-$\Z_{\E}\<G\>$-module in this context via the augmentation map:
                    $$\e: \Z_{\E}\<G\> \to \Z_{\E}$$
                (given by $\e(f) := f(1)$ for all sections $f \in \Z_{\E}\<G\>$), and because this map is \textit{a priori} surjective (hence an epimorphism, since we are in an abelian category), a good choice of projective resolution $(P_{\bullet}, \e_{\bullet})$ for $\Z_{\E} \in \Ob(\Ab(\E))$ is:
                    $$
                        \begin{tikzcd}
                        	\cdots & {P_{-2}} & {P_{-1} := \Z_{\E}\<G\>} & {P_0 := \Z_{\E}} & 0
                        	\arrow["{\e_0}", from=1-4, to=1-5]
                        	\arrow["{\e := \e_{-1}}", from=1-3, to=1-4]
                        	\arrow["{\e_{-2}}", from=1-2, to=1-3]
                        	\arrow["{\e_{-3}}", from=1-1, to=1-2]
                        \end{tikzcd}
                    $$
                wherein $P_{-n}$ can be conveniently taken to be the free cover\footnote{Which always exists and is projective as an object of $\Ab(\E)$ should one assume the Axiom of Choice.} of $\ker \e_{-(n - 1)}$ for all $n \geq 2$.
            \end{remark}
            \begin{remark}[Group cohomology with coefficients in modules over general rings]
                
            \end{remark}
            \begin{proposition}[Low-dimensional group cohomologies] \label{prop: low_dimensional_group_cohomologies}
                Let $G$ be a group internal to a topos $\E$ wherein the infinite tensor category $(\Ab(\E), \tensor_{\Z_{\E}}, \Z_{\E})$ of internal abelian groups has enough projectives. Then, we have the following results for the low-dimensional $\Ab(\E)$-valued cohomologies of $G$:
                    \begin{enumerate}
                        \item $H^0_{\Grp(\E)}(G, -) \cong (-)^G$ is the functor of $G$-fixed points.
                        \item $H^1_{\Grp(\E)}(G, -) \cong $
                        \item $H^2_{\Grp(\E)}(G, M) \cong \{ \text{isomorphism classes of \textit{central} extensions of $G$ by $M$} \}$ for all $M \in \Ob(\Ab(\E))$.
                    \end{enumerate}
            \end{proposition}
                \begin{proof}
                    
                \end{proof}
            \begin{corollary}[Criteria for being semi-simple and hereditary] \label{coro: criteria_for_being_semi_simeple_and_hereditary}
                Let $G$ be a group internal to a topos $\E$ wherein the infinite tensor category $(\Ab(\E), \tensor_{\Z_{\E}}, \Z_{\E})$ of internal abelian groups has enough projectives. Also, let $k$ be an associative algebra internal to $\Ab(\E)$ (i.e. an associative $\Z_{\E}$-algebra).
                    \begin{enumerate}
                        \item The category ${}^lk\<G\>\mod^{\fin}$ of finitely generated left-$\Z_{\E}\<G\>$-modules\footnote{... or equivalently, the category $\Rep_k^{\fin}(G)$ of finite-rank $k$-linear $G$-representations.} is semi-simple (i.e. $\globdim {}^lk\<G\>\mod^{\fin} = 0$) if and only if the fixed-point functor $(-)^G$ is exact.
                        \item The category ${}^lk\<G\>\mod^{\fin}$ of finitely generated left-$\Z_{\E}\<G\>$-modules is hereditary (i.e. $\globdim {}^lk\<G\>\mod^{\fin} \leq 1$) if and only if there are no non-trivial central extensions of $G$ by any $M \in \Ob(k\mod)$.
                    \end{enumerate}
            \end{corollary}
                \begin{proof}
                    \noindent
                    \begin{enumerate}
                        \item Combine lemma \ref{lemma: projective_and_injective_modules_over_semi_simple_rings} with proposition \ref{prop: computing_projective_dimensions_using_Ext}.
                        \item Apply proposition \ref{prop: computing_projective_dimensions_using_Ext} directly to definition \ref{def: hereditary_abelian_categories}.
                    \end{enumerate}
                \end{proof}
        
        \subsubsection{Maschke's Theorem and characters}
            \begin{lemma}[Projectivity and restriction of scalars] \label{lemma: projectivity_and_restriction_of_scalars}
                Let $R$ be a ring, $S$ be a semi-simple associative $R$-algebra of finite presentation that is also an $R$-coalgebra\footnote{E.g. over some base ring $k$, one can take $S := k\<G\>$ for some abstract group $G$ and $R := k\<H\>$ for some subgroup $H \leq G$; for a concrete example, take $R := k\<1\> \cong k$ and suppose that the associative $S$-algebra structure thereon is given by the augmentation map $\e: k\<G\> \to k$.}, so that $R$ will carry an $(S, R)$-bimodule. Then, the restriction-of-scalars functor:
                    $$\Hom_S(R, -): {}^lS\mod \to {}^lR\mod$$
                will preserve projective objects.
            \end{lemma}
                \begin{proof}
                    
                \end{proof}
            \begin{theorem}[Maschke's Theorem] \label{theorem: maschke_theorem}
                Let $G$ be group and $k$ be a ring. Then the group $k$-algebra $k\<G\>$ will be semi-simple as a $k$-algebra (not just as a $k$-module!) if and only if:
                    \begin{enumerate}
                        \item $k$ is semi-simple ring,
                        \item $G$ is finite, and
                        \item $n$ is invertible in $k$.
                    \end{enumerate}
            \end{theorem}
                \begin{proof}
                    Assume first of all that $k\<G\>$ is semi-simple as an associative $k$-algebra; note that via the augmentation map $\e: k\<G\> \to k$ defined by $\e(f) := f(1)$ for all $f \in k\<G\>$, $k$ is also an associative $k\<G\>$-algebra. Then, by lemma \ref{lemma: projectivity_and_restriction_of_scalars}, projective left-$k\<G\>$-modules will be projective as left-$k$-modules as well. However, since $k\<G\>$ is a semi-simple ring, all left-modules over it are projective (cf. lemma \ref{lemma: simple_modules_over_semi_simple_rings}).
                    
                    Conversely, assume that the three listed conditions hold. Then, using corollary \ref{coro: criteria_for_being_semi_simeple_and_hereditary}, we shall aim to show that the fixed-point functor $(-)^G$ is exact; in fact, it shall suffice to only show that $(-)^G$ is right-exact, since it is \textit{a priori} left-exact by virtue of being a corepresentable on an abelian category, namely ${}^lk\<G\>\mod$ (according to proposition \ref{prop: low_dimensional_group_cohomologies}, $(-)^G \cong \Hom_{k\<G\>}(k, -)$).  
                \end{proof}
            \begin{example}[What happens when $|G| \mid \chara k$ ?]
                
            \end{example}
            \begin{corollary}[The Sum-Of-Squares Formula for finite groups] \label{coro: sum_of_squares_formula_for_finite_groups}
                Let $G$ be a finite group and $k$ is a field wherein $|G|$ is invertible. Then there are finitely many irreducible $k$-linear representations of $G$ (i.e. simple left-$k\<G\>$-modules), and:
                    $$|G| = \sum_{V \in \Ob(\Rep_k^{\irr}(G))} (\dim_k V)^2$$
            \end{corollary}
                \begin{proof}
                    Note that $|G| = \dim_k k\<G\>$ when $\chara k$ does not divide $|G|$, so one can simply apply the Artin-Wedderburn Theorem (cf. theorem \ref{theorem: artin_wedderburn}) to the semi-simple $k$-algebra $k\<G\>$.
                \end{proof}
                
        \subsubsection{Characters and the classification of irreducible representations of finite groups}
        
        \subsubsection{Representations of finite flat group schemes}
        
    \subsection{Nakayama rings}