\section{Quivers and homological dimensions \texorpdfstring{$\leq 1$}{}}
    \subsection{Quivers}
        \subsubsection{Generalities about quivers and their representations}
            We begin by introducing so-called \say{quivers}, which is more-or-less a relaxed version of directed graphs which turn out to be rather important within representation-theoretic contexts. Our approach is that quivers are certain very simple categories, and as such their representations ought to be viewed as particular instances of representations of categories. Specifically, this means we shall be studying the various algebraic properties of quiver representations via modules over their so-called \say{category algebras}, so that notions of say, finite-type-ness of quivers, could be discussed using purely ring-theoretic arguments.
        
            \begin{definition}[Quivers] \label{def: quivers}
                For any category $\C$, a \textbf{$\C$-valued quiver} $Q$ is a diagram in $\C$ of the following form:
                    $$
                        \begin{tikzcd}
                            Q_1 \arrow[r, "t"', shift right=2] \arrow[r, "s", shift left=2] \arrow["\id_{Q_1}"', loop, distance=2em, in=215, out=145] & Q_0 \arrow["\id_{Q_0}"', loop, distance=2em, in=35, out=325]
                        \end{tikzcd}
                    $$
                wherein $Q_1, Q_0 \in \Ob(\C)$ respectively are known as the objects of \textbf{arrows/vertices} and \textbf{objects/edges}, and the arrows $s, t: Q_1 \toto Q_0$ are known as the \textbf{source} and \textbf{target} morphisms. 
            \end{definition}
            \begin{remark}[The category of quivers]
                Equivalently, one might think of $\C$-valued quivers as $\C$-valued presheaves on the category:
                    $$
                        \begin{tikzcd}
                            {\1} \arrow[r, "t"', shift right=2] \arrow[r, "s", shift left=2] \arrow["\id_{\1}"', loop, distance=2em, in=215, out=145] & {\0} \arrow["\id_{\0}"', loop, distance=2em, in=35, out=325]
                        \end{tikzcd}
                    $$
                As such, for an arbitrarily fixed target category $\C$, one obtains a category $\Quiv(\C)$ of $\C$-valued quivers and natural transformations between them. In particular, when $\C \cong \Sets$, one obtains the presheaf topos $\Quiv$ of $\Sets$-valued quivers. 
            \end{remark}
            \begin{example}[Quivers internal to $\Sets$]
                A $\Sets$-valued quiver is actually nothing but a so-called \textbf{directed graph}: such a quiver is a quadruple $(Q_1, Q_0, s, t)$ wherein $Q_1$ is a set of directed edges, $Q_0$ is the set of vertices of those direct edges, and $s, t: Q_1 \toto Q_0$ are the assignments of the sources and targets vertices (i.e. beginning and endpoints) to the aforementioned directed edges. One important detail to note here is that between two given vertices, there may be many directed edges, and there might also be loops onto the same vertex, as follows:
                    $$
                        \begin{tikzcd}
                                                                                                                                           &         & \bullet \arrow[loop, distance=2em, in=125, out=55]             &                                                        & \bullet \\
                            \bullet \arrow[r, shift right=2] \arrow[r, shift left=2] \arrow[loop, distance=2em, in=215, out=145] \arrow[r] & \bullet & \bullet \arrow[r, shift right=2] \arrow[l] \arrow[u] \arrow[d] & \bullet \arrow[ru] \arrow[rd] \arrow[l, shift right=2] &         \\
                                                                                                                                           &         & \bullet \arrow[loop, distance=2em, in=305, out=235]            &                                                        & \bullet
                        \end{tikzcd}
                    $$
                These diagrams of sets and functions, however, need not be commutative: as such, one is usually interested in the free categories associated to quivers, which are nothing but those quivers with compositions and identities added on (cf. proposition \ref{prop: free_quivers}).
                
                To be very specific, Dynkin diagrams are examples of $\Sets$-valued quivers (cf. definition \ref{def: dynkin_quivers}). 
            \end{example}
            \begin{proposition}[Sub-object classifier for $\Quiv$] \label{prop: sub_object_classifier_for_topos_of_quivers}
                
            \end{proposition}
                \begin{proof}
                    
                \end{proof}
            \begin{definition}[The double negation topology]
                
            \end{definition}
            \begin{proposition}[A separatedness criterion for quivers] \label{prop: separatedness_criterion_for_quivers}
                
            \end{proposition}
                \begin{proof}
                    
                \end{proof}
                
            \begin{proposition}[Free quivers] \label{prop: free_quivers}
                The evident forgetful functor:
                    $$\oblv: 1\-\Cat_1 \to \Quiv$$
                (which forgets compositions and the associativity of said compositions) admits a left-adjoint:
                    $$[-]: \Quiv \to 1\-\Cat_1$$
                which shall be known as the \textbf{free quiver} functor. Explicitly, this functor assigns to each ($\Sets$-valued) quiver its associated free category.
            \end{proposition}
                \begin{proof}
                    
                \end{proof}
            \begin{definition}[Quiver representations] \label{def: quiver_representations}
                Given a quiver $Q \in \Ob(\Quiv)$, a commutative ring $k$, and a $k$-linear tensor category\footnote{Aside from the obvious example of $k\mod^{\heart}$ (and $k$-linear tensor subcategories thereof, like $(k\mod^{\fin})^{\heart}$), one might also consider categories such as the derived category of $k$-modules $k\mod$.} $\calV$, its category of $k$-linear representations, denoted by $\Rep_{\calV}(Q)$, is the functor category $\Func([Q], \calV)$.
            \end{definition}
            \begin{remark}[Basic properties of quiver representations]
                Given a quiver $Q \in \Ob(\Quiv)$, a commutative ring $k$, and a $k$-linear tensor category $\calV$, its category of $k$-linear representations $\Rep_{\calV}(Q)$ will also be a $k$-linear tensor category.
                
                Of course, one could consider representations with values in categories of a more general kind than linear tensor $1$-categories, such as symmetric monoidal stable $\infty$-categories. 
            \end{remark}
            \begin{convention}
                Often, we will consider the case $\calV \cong k\mod$ for some commutative ring $k$, and in which case, we shall write $\Rep_k(Q)$ for the category of $k$-linear representations of $Q$. Some important subcategories therein are:
                    $$\Rep_k^{\irr}(Q)$$
                    $$\Rep_k^{\red}(Q)$$
                which are the categories of irreducible representations and that of indecomposable representations. We will also be interested in the full subcategory:
                    $$\Rep_k^{\fin}(Q)$$
                of $k$-linear representations of $Q$ which are of finite ranks over $k$; note that for finite quivers, indecomposable and irreducible representations are necessarily of finite ranks.
            \end{convention}
            
            Let us now move on to the notion of so-called \say{quiver algebras} and discuss the roles that they play in the representation theory of $\Sets$-valued quivers. 
            \begin{definition}[Category algebras] \label{def: category_algebras}
                Let $k$ be an associative ring and $\C$ be a category. Then, the \textbf{category $k$-algebra} $k\<\C\>$ of the given category $\C$ shall be the free $k$-algebra:
                    $$k\<\C_1\> := k\<\Mor(\C)\>$$
                whose underlying (left-)$k$-module is the free (left-)$k$-module on the set $\C_1 := \Mor(\C)$ of morphisms of $\C$ and whose (associative and unital) multiplication is given by compositions of arrows in $\C_1$ (if two arrows are not composable then their product will be $0$).
            \end{definition}
            \begin{convention}[Path algebras ?]
                Sometimes the category $k$-algebra (for some ring $k$) of the free category associated to a given $\Sets$-valued quiver $Q$ is also called the path algebra of that quiver. It is commonly denoted simply by $k\<Q\>$ as opposed to $k\<[Q]\>$.
            \end{convention}
            \begin{remark}[A path algebra functor] \label{remark: path_algebra_functor}
                When $k$ is commutative, one has a tautological equivalence of categories:
                    $$k\bimod \cong k\mod$$
                which gives us access to the tensor $k$-algebra construction, which is \textit{a priori} a left-adjoint:
                    $$
                        \begin{tikzcd}
                        	{k\-\Assoc\Alg} & k\mod
                        	\arrow[""{name=0, anchor=center, inner sep=0}, "{(-)^{\tensor}}"', bend right, from=1-2, to=1-1]
                        	\arrow[""{name=1, anchor=center, inner sep=0}, "\oblv"', bend right, from=1-1, to=1-2]
                        	\arrow["\dashv"{anchor=center, rotate=-90}, draw=none, from=0, to=1]
                        \end{tikzcd}
                    $$
                Using this, one sees that category $k$-algebras arise functorially in the following manner:
                    $$
                        \begin{tikzcd}
                        	{k\-\Assoc\Alg} && k\mod \\
                        	\\
                        	{1\-\Cat_1} && \Sets
                        	\arrow[""{name=0, anchor=center, inner sep=0}, "{(-)^{\tensor}}"', bend right, from=1-3, to=1-1]
                        	\arrow[""{name=1, anchor=center, inner sep=0}, "\oblv"', bend right, from=1-1, to=1-3]
                        	\arrow["{(-)_1}", from=3-1, to=3-3]
                        	\arrow["{k\<-\> := ( k^{\oplus (-)_1} )^{\tensor}}", dashed, from=3-1, to=1-1]
                        	\arrow[""{name=2, anchor=center, inner sep=0}, "{k^{\oplus (-)}}", bend left, from=3-3, to=1-3]
                        	\arrow[""{name=3, anchor=center, inner sep=0}, "\oblv", bend left, from=1-3, to=3-3]
                        	\arrow["\dashv"{anchor=center, rotate=-90}, draw=none, from=0, to=1]
                        	\arrow["\dashv"{anchor=center}, draw=none, from=2, to=3]
                        \end{tikzcd}
                    $$
                Now, since the functors:
                    $$(-)^{\tensor}: k\mod \to k\-\Assoc\Alg$$
                and:
                    $$k^{\oplus (-)}: \Sets \to k\mod$$
                are left-adjoints, and because:
                    $$(-)_1: 1\-\Cat_1 \to \Sets$$
                preserves coproducts and pushouts, one sees that given any pair of categories $\C, \C'$ (which need not be disjoint), one has:
                    $$k\<\C \cup \C'\> \cong k\<\C\> \tensor_k k\<\C'\>$$
                Since the construction of free categories on (small) quivers also comes from a left-adjoint, namely:
                    $$[-]: \Quiv \to 1\-\Cat$$
                (cf. proposition \ref{prop: free_quivers}), one also has\footnote{One does not have to worry about the formal existence of $Q \cup Q'$ within the category $\Quiv$ of $\Sets$-valued quivers, since it is a topos by construction.}:
                    $$k\<Q \cup Q'\> \cong k\<Q\> \tensor_k k\<Q'\>$$
            \end{remark}
            \begin{example}[Path algebras of discrete quivers] \label{example: path_algebras_of_discrete_quivers}
                The simplest example of a quiver is the trivial loop:
                    \begin{figure}[H]
                        \centering
                            $$
                                \begin{tikzcd}
                                    \bullet \arrow["\id"', loop, distance=2em, in=35, out=325]
                                \end{tikzcd}
                            $$
                        \caption{The trivial quiver}
                        \label{fig: trivial_quiver}
                    \end{figure}
                It is easy to see that the path algebra of this quiver (over some fixed ring $k$) is:
                    $$k\<e\>/\<e^2 = e\> \cong k$$
                    
                Going off in a different direction\footnote{Pun may or may not have been intended.}, consider the following quiver $Q := (Q_1, Q_0, s, t)$ (i.e. a discrete category\footnote{Let's ignore potential set-theoretic issues for now!}):
                    $$
                        \begin{tikzcd}
                            \cdots & \bullet_i \arrow["\id_{\bullet_i}"', loop, distance=2em, in=125, out=55] & \bullet_j \arrow["\id_{\bullet_j}"', loop, distance=2em, in=125, out=55] & \bullet_k \arrow["\id_{\bullet_k}"', loop, distance=2em, in=125, out=55] & \cdots
                        \end{tikzcd}
                    $$
                Its path algebra is:
                    $$k\<\{e_i\}_{i \in Q_0}\>\/\<\forall i \in Q_0: e_i^2 = e_i, \forall i, j \in Q_0: e_i e_j = e_j e_i = \delta_{ij}\>$$
                (where $\delta_{ij}$ is the Kronecker delta), which we recognise as being isomorphic to the product:
                    $$\prod_{i \in Q_0} k$$
                taken in the category of associative $k$-algebras.
            \end{example}
            \begin{example}[Path algebras of $\sfA_n$ quivers] \label{example: path_algebras_of_A_n_quivers}
                For a simple yet non-trivial example of quiver algebras, consider the following quiver, commonly known as the \say{$\sfA_2$ quiver}:
                    \begin{figure}[H]
                        \centering
                        $$
                            \begin{tikzcd}
                                {\bullet_1} \arrow["\id_{\bullet_1}"', loop, distance=2em, in=215, out=145] \arrow[r] & {\bullet_2} \arrow["\id_{\bullet_2}"', loop, distance=2em, in=35, out=325]
                            \end{tikzcd}
                        $$
                        \caption{The quiver $\sfA_2$}
                        \label{fig: A_2_quiver}
                    \end{figure}
                which has a rather small path algebra, isomorphic to:
                    $$k\<e_1, e_2, x\>/\<e_i^2 = e_i, e_i e_j = e_j e_i = \delta_{ij}, x^2 = 0, e_2 x = x e_1 = x\>$$
                One observation to make is that there is an associative $k$-algebra isomorphism given by:
                    $$e_1 \mapsto \begin{pmatrix} 1 & 0 \\ 0 & 0 \end{pmatrix}$$
                    $$e_2 \mapsto \begin{pmatrix} 0 & 0 \\ 0 & 1 \end{pmatrix}$$
                    $$x \mapsto \begin{pmatrix} 0 & 0 \\ 1 & 0 \end{pmatrix}$$
                between this path $k$-algebra and the $k$-algebra:
                    $$\b_2^-(k) := \left\{ \begin{pmatrix} * & 0 \\ * & * \end{pmatrix} \in \Mat_2(k) \right\}$$
                of lower-triangular $2 \x 2$ matrices with entries in $k$.
                    
                Next, consider the general $\sfA_n$ quiver\footnote{Henceforth we shall start omiting the identity paths from depictions of quivers.} (for some finite positive integer $n$):
                    \begin{figure}[H]
                        \centering
                            $$
                                \begin{tikzcd}
                                	{\bullet_1} & {\bullet_2} & {\bullet_3} & \cdots
                                	\arrow["{x_{12}}", from=1-1, to=1-2]
                                	\arrow["{x_{23}}", from=1-2, to=1-3]
                                	\arrow["{x_{34}}", from=1-3, to=1-4]
                                \end{tikzcd}
                            $$
                        \caption{The quiver $\sfA_n$}
                        \label{fig: A_n_quiver}
                    \end{figure}
                Observe that the set of morphisms $[\sfA_n]_1$ of the free category $[\sfA_n]$ on $\sfA_n$ is nothing but $\{x_{ij}\}_{1 \leq i \leq j \leq n}$ (also, note that $e_i = x_{ii}$ for all $1 \leq i \leq n$). Now, by letting $\1_{ij}$ denote the $n \x n$ matrix with $1$ at the $ij^{th}$ entry and $0$ everywhere else, and consider the map:
                    $$k\<\sfA_n\> \to \Mat_n(k)$$
                    $$x_{ij} \mapsto \1_{ij}$$
                one then sees that $k\<\sfA_n\>$ is isomorphic to the $k$-algebra:
                    $$\b_n^-(k)$$
                of lower-triangular $n \x n$ matrices with entries in $k$.
            \end{example}
            \begin{example}[Path algebras of quivers with loops] \label{example: path_algebras_of_loop_quivers}
                If we were to add a non-trivial loop $x: \bullet \to \bullet$ to the trivial quiver $(\{\bullet\}, \{\id\})$ to yield:
                    $$
                        \begin{tikzcd}
                            \bullet \arrow["x"', loop, distance=2em, in=35, out=325]
                        \end{tikzcd}
                    $$
                then the resulting path algebra will be:
                    $$k\<e, x\>/\<e^2 = e, ex = xe = x\> \cong k\<x\>$$
                Of course, when $k$ is a commutative ring, then we have a tautological isomorphism $k\<x\> \cong k[x]$, but this does not really help us flesh out the nature of the path algebra of this loop quiver.
                    
                More generally, we can consider the quiver with one vertex and a set $\{x_i\}_{i \in I}$ of (possibly infinitely) many loops on that vertex:
                    $$
                        \begin{tikzcd}
                            \bullet \arrow["x_1"', loop, distance=2em, in=305, out=235] \arrow["x_2"', loop, distance=2em, in=35, out=325] \arrow["x_3"', loop, distance=2em, in=125, out=55] \arrow["\cdots"', loop, distance=2em, in=215, out=145]
                        \end{tikzcd}
                    $$
                then the resulting path algebra will be:
                    $$k\<\{x_i\}_{i \in I}\>$$
                because while one can free form compositions $x_j \circ x_i \circ ...$ of the endomorphisms/loops $x_i \in \End(\bullet)$, there is not a reason to expect that $x_j x_i = x_i x_j$. 
            \end{example}
            \begin{example}[Path algebras of non-simply-laced quivers]
                Consider the so-called \say{Kronecker quiver}:
                    \begin{figure}[H]
                        \centering
                            $$
                                \begin{tikzcd}
                                	{\bullet_1} & {\bullet_2}
                                	\arrow["x_2"', shift right=1, from=1-1, to=1-2]
                                	\arrow["x_1", shift left=1, from=1-1, to=1-2]
                                \end{tikzcd}
                            $$
                        \caption{The Kronecker quiver}
                        \label{fig: kronecker_quiver}
                    \end{figure}
                which is easily recognisable as the (non-disjoint) union of two copies of the $\sfA_2$ quiver (cf. example \ref{example: path_algebras_of_A_n_quivers}). One sees that its path $k$-algebra is:
                    $$k\<e_1, e_2, x_1, x_2\>/\<e_i^2 = e_i, e_i e_j = e_j e_i = \delta_{ij}, x_m x_n = 0, e_2 x_m = x_m e_1 = x_m\>$$
                which can be shown to be isomorphic to the $k$-algebra\footnote{$\b_2^-(k)$ is the $k$-algebra of lower-triangular $2 \x 2$ matrices with entries in $k$; cf. example \ref{example: path_algebras_of_A_n_quivers}.}:
                    $$k\<\sfA_2 \cup \sfA_2\> \cong k\<\sfA_2\> \tensor_k k\<\sfA_2\> \cong \b_2^-(k) \tensor_k \b_2^-(k)$$
                using the fact that the path $k$-algebra functor:
                    $$k\<-\>: \Quiv \to k\-\Assoc\Alg$$
                preserves pushouts\footnote{In $\Quiv$, these are nothing but unions of quivers} (cf. remark \ref{remark: path_algebra_functor}) whenever $k$ is commutative; when $k$ is noncommutative, one can still show that the isomorphism holds using the fact that $\b_2^-(k)$ is a $k$-bimodule.
            \end{example}
            \begin{example}[Path algebras of $\sfD_n$ quivers]
                For any $n \geq 2$, the so-called \say{$\sfD_n$ quiver} is of the form:
                    \begin{figure}[H]
                        \centering
                            $$
                                \begin{tikzcd}
                                	&&&& {\bullet_n} \\
                                	{\bullet_1} & {\bullet_2} & \cdots & {\bullet_{n - 2}} \\
                                	&&&& {\bullet_{n - 1}}
                                	\arrow["{x_{12}}", from=2-1, to=2-2]
                                	\arrow["{x_{23}}", from=2-2, to=2-3]
                                	\arrow["{x_{n - 2, n - 3}}", from=2-3, to=2-4]
                                	\arrow["{x_{n, n - 2}}", from=2-4, to=1-5]
                                	\arrow["{x_{n - 1, n - 2}}"', from=2-4, to=3-5]
                                \end{tikzcd}
                            $$
                        \caption{The quiver $\sfD_n$}
                        \label{fig: D_n_quiver}
                    \end{figure}
                (when $n = 2, 3$, note that $\sfD_n = \sfA_n$, so there is nothing new in those cases, and we shall assume hereonafter that $n \geq 3$). Observe that for all $n \geq 3$, the $\sfD_n$ quiver is the union of the $\sfA_{n - 2}$ quiver with two copies of the $\sfA_2$ quiver, both at the $(n - 2)^{th}$ vertex and in such a manner that:
                    $$x_{n - 1, n - 2} x_{n - 2, n - 3} \not = 0$$
                    $$x_{n, n - 2} x_{n - 2, n - 3} \not = 0$$
                and with this observation in mind, one sees that:
                    $$k\<\sfD_n\> \cong k\<\sfA_{n - 2}\> \tensor_k k\<\sfA_2\> \tensor_k k\<\sfA_2\> \tensor_k k\<\sfA_3\>/$$
            \end{example}
            \begin{example}[Path algebras of $\sfE_n$ quivers]
                
            \end{example}
            \begin{proposition}[Quiver representations are modules over quiver algebras] \label{prop: quiver_representations_are_modules_over_quiver_algebras}
                Let $k$ be a commutative ring. Then, there is an exact and monoidal equivalence of $k$-linear categories (fibred over $k\mod$):
                    $$\Rep_k(Q) \to {}^lk\<Q\>\mod$$
            \end{proposition}
                \begin{proof}
                    
                \end{proof}
            \begin{remark}
                The equivalence of categories:
                    $$\Rep_k(Q) \to {}^lk\<Q\>\mod$$
                from proposition \ref{prop: quiver_representations_are_modules_over_quiver_algebras} has many important further properties, which all come from the very definition of linear representations of associative algebras (in this case, path algebras of quivers) themselves. For instance, this functor preserves (semi-)simplicity and (in)decomposability.
                
                As for further properties coming not from the definition of representations themselves, but rather from the basic property of this equivalence of categories, one has that the functor also preserves lengths and more generally, ascending and descending chains of objects, as a consequence of being exact. In particular, this means that should a representation of a given quiver $Q$ is finitely generated as a $k$-module (i.e. \say{finite-dimensional}), then the same would also be true for the corresponding left-$k\<Q\>$-module.
            \end{remark}
            \begin{remark}
                One important algebraic consequence of proposition \ref{prop: quiver_representations_are_modules_over_quiver_algebras} is should $k$ be a commutative ring and $Q$ be a finite $\Sets$-valued quiver, then the path $k$-algebra $k\<Q\>$ will necessarily have finite rank as a $k$-module. $k\<Q\>$ is therefore left and right-Artinian, and as such all left/right-$k\<Q\>$-modules are finitely generated if and only if they are of finite lengths; equivalently, this means that all $k$-linear representations of $Q$ are of finite ranks over $k$ they are of finite lengths.
            \end{remark}
            
        \subsubsection{Tits quadratic forms and roots of Dynkin quivers; Grothendieck groups}
            \begin{definition}[Cartan quadratic forms of finite quivers] \label{def: cartan_quadratic_forms_of_finite_quivers}
                Let $Q := (Q_1, Q_0, s, t)$ be a finite quiver\footnote{With value in any category} and suppose that $n := |Q_0|$; also, let us give the set $Q_0$ of vertices a \textit{fixed} enumeration $\{v_1, ..., v_n\}$. To such a quiver, one can associate a so-called \textbf{adjacency matrix}:
                    $$R_Q \in \Mat_n(\Z)$$
                given by:
                    $$R_Q := (r_{ij} := |\{f \in Q_1 \mid s(f) = v_i, t(f) = v_j\}|)_{1 \leq i, j \leq n}$$
                (that is, the entries $r_{ij}$ of $R_Q$ are the numbers of \textit{undirected} edges from the vertex $v_i$ to the vertex $v_j$). From this matrix, one can define the \textbf{Cartan matrix} of $Q$:
                    $$A_Q := 2I_n - R_Q$$
                along with a $\Z$-bilinear form, which we shall call the \textbf{Cartan form}:
                    $$B_Q: \Z^{\oplus n} \x \Z^{\oplus n} \to \Z$$
                    $$(x, y) \mapsto x^{\top} A_Q y$$                
            \end{definition}
            \begin{definition}[Acylic quivers] \label{def: acyclic_quivers}
                A quiver $Q := (Q_1, Q_0, s, t)$ is said to be \textbf{acyclic} if and only if it for all pairs of distinct vertices $v, v' \in Q_0$, there does not exist simultaneously directed edges $x, y \in Q_1$ such that $s(x) = t(y)$ and $t(x) = s(y)$.
            \end{definition}
            \begin{definition}[Connected quivers] \label{def: connected_quivers}
                A quiver $Q: \{\1, \0, s, t\}^{\op} \to \Sets$ is said to be \textbf{connected} if and only if for all $v, w \in Q(\0)$, there exists $f \in Q(\1)$ such that $s(f) = v$ and $t(f) = w$ (i.e. it has no isolated vertices).
            \end{definition}
            \begin{remark}[Connected quivers are categorically indecomposable]
                Equivalently, one can say that a $\Sets$-valued (or maybe with values in any category $\C$ with enough coproducts and sub-objects) quiver is connected if and only if it can not be written as the disjoint union of two sub-quivers. 
            \end{remark}
            \begin{proposition}[Evenness of Cartan forms of finite connected acyclic quivers] \label{prop: evenness_of_cartan_forms_of_finite_connected_acyclic_quivers}
                Let $\Gamma := (\Gamma_1, \Gamma_0, s, t)$ be a Dynkin quiver and consider its Cartan matrix $A_{\Gamma}$. Then for all $x \in \Z^{\oplus n}$, $B_{\Gamma}(x, x)$ is even.  
            \end{proposition}
                \begin{proof}
                    Let $n := |\Gamma_0|$ and pick a basis for $\Z^{\oplus n}$ in order to write:
                        $$
                            \begin{aligned}
                                B_{\Gamma}(x, x) & = x^{\top} A_{\Gamma} x
                                \\
                                & = \sum_{1 \leq i, j \leq n} x_i a_{ij} x_j
                                \\
                                & = \sum_{1 \leq i, j \leq n} x_i (2\delta_{ij} - r_{ij}) x_j \text{(wherein $\delta_{ij}$ is the Kronecker delta)}
                                \\
                                & = 2\sum_{1 \leq i, j \leq n} x_i \delta_{ij} x_j - \sum_{1 \leq i, j \leq n} x_i (1 - \delta_{ij}) r_{ij} x_j \text{($r_{ij}$ are the entries the adjacency matrix of $\Gamma$)}
                                \\
                                & = 2\sum_{1 \leq i \leq n} x_i^2 - 2\sum_{1 \leq i < j \leq n} x_i r_{ij} x_j
                            \end{aligned}
                        $$
                    for all $x \in \Z^{\oplus n}$, wherein the last line is due to the fact that the number of undirected edges from a vertex $v_i$ to another $v_j$ is equal to the number of undirected directed edges from $v_j$ to $v_i$, which in turn is a result of the assumption that $\Gamma$ is connected and acyclic\footnote{Note how this argument fails if $\Gamma$ is either not acyclic or not connected.}. Clearly, $B_{\Gamma}(v, v)$ is even for all $x \in \Z^{\oplus n}$.
                \end{proof}
            \begin{definition}[Tits quadratic forms] \label{def: tits_quadratic_forms}
                Let $\Gamma := (\Gamma_1, \Gamma_0, s, t)$ be a finite connected acyclic quiver and $B_{\Gamma}$ be its Cartan quadratic form. Then, inspired by the proof of proposition \ref{prop: evenness_of_cartan_forms_of_finite_connected_acyclic_quivers}, let us define the \textbf{Tits quadratic form} associated to $\Gamma$ by:
                    $$q_{\Gamma}: \Z^{\oplus \Gamma_0} \to \Z$$
                    $$x \mapsto \frac12 B_{\Gamma}(x, x)$$
                Since $B_{\Gamma}(x, x)$ is \textit{a priori} even (cf. proposition \ref{prop: evenness_of_cartan_forms_of_finite_connected_acyclic_quivers}), $q_{\Gamma}(x)$ is always well-defined. 
            \end{definition}
            \begin{remark}
                Let $\Gamma := (\Gamma_1, \Gamma_0, s, t)$ be a finite connected acyclic quiver with adjacency matrix $R_{\Gamma} := (r_{ij})_{1 \leq i, j \leq |\Gamma_0|}$. Then from the proof of proposition \ref{prop: evenness_of_cartan_forms_of_finite_connected_acyclic_quivers}, we know that:
                    $$q_{\Gamma}(x) = \sum_{1 \leq i \leq |\Gamma_0|} x_i^2 - \sum_{1 \leq i < j \leq |\Gamma_0|} x_i r_{ij} x_j$$
            \end{remark}
            
            Let us now narrow our focus down to a very special class of finite connected acyclic quivers, the so-called \say{Dynkin quivers}, which can be extensively analysed via the so-called \say{roots} of their Tits quadratic forms.
            \begin{definition}[Dynkin quivers] \label{def: dynkin_quivers}
                A \textbf{Dynkin quiver} is a \textit{connected} and \textit{acyclic} finite quiver whose Cartan form is \textit{positive-definite}.
            \end{definition}
            \begin{remark}[The importance of positive-definiteness]
                
            \end{remark}
            \begin{definition}[Roots] \label{def: roots_of_dynkin_quivers}
                A root of a \textit{Dynkin} quiver $\Gamma := (\Gamma_1, \Gamma_0, s, t)$ is a vector:
                    $$\alpha \in \Z^{\oplus n}$$
                (where $n := |\Gamma_0|$) such that:
                    $$q_{\Gamma}(\alpha) = 1$$
                The set of all roots of $\Gamma$ is denoted by $\Phi_{\Gamma}$.
            \end{definition}
            \begin{example}[Simple roots] \label{example: simple_roots}
                Let $\Gamma := (\Gamma_1, \Gamma_0, s, t)$ be a Dynkin quiver and set $n := \Gamma_0$. Also, pick a basis $\{e_1, ..., e_n\}$ for $\Z^{\oplus n}$. Then the vectors of the form:
                    $$\alpha_i := \delta_{ij} e_j$$
                (for $1 \leq i, j \leq n$) are in fact roots of $\Gamma$ and furthermore, they form a basis for $\Z^{\oplus n}$.
            \end{example}
            \begin{lemma}[Roots are exclusively either negative or positive] \label{lemma: roots_are_exclusively_either_negative_or_positive}
                Let $\Gamma := (\Gamma_1, \Gamma_0, s, t)$ be a Dynkin quiver and set $n := \Gamma_0$. Also, pick a basis $\{\alpha_i\}_{1 \leq i \leq n}$ of simple roots for $\Z^{\oplus n}$ and choose a root $\alpha$ and write it as:
                    $$\alpha := \sum_{1 \leq i \leq n} c_i \alpha_i$$
                Then either $c_i \geq 0$ or $c_i \leq 0$ for all $1 \leq i \leq n$ simultaneously.
            \end{lemma}
                \begin{proof}
                    By base-changing to $\R$, which we shall equip with the norm\footnote{We shall let our dear readers verify for themselves that $|-|_{\Gamma}$ satisfies the norm axioms.} given by $|x|_{\Gamma} := \sqrt{q_{\Gamma}(x)}$ for all $x \in \R^{\oplus n}$ (note that this norm is well-defined because $q_{\Gamma}(x)$ is positive-definite as $\Gamma$ is a Dynkin quiver; cf. definition \ref{def: dynkin_quivers}), one sees that there is a bijection:
                        $$\Phi_{\Gamma} \tensor_{\Z} \R \cong \{\alpha \in \R^{\oplus n} \mid |\alpha|_{\Gamma} = 1\}$$
                    The lemma is then an immediate consequence of the defintion of simple roots (cf. example \ref{example: simple_roots}).
                \end{proof}
            \begin{definition}[Positive and negative roots] \label{def: negative_and_positive_roots}
                Let $\Gamma := (\Gamma_1, \Gamma_0, s, t)$ be a Dynkin quiver and set $n := \Gamma_0$. Also, pick a basis $\{e_1, ..., e_n\}$ for $\Z^{\oplus n}$ and choose a root $\alpha$ and write it in terms of the simple roots $\{\alpha_i\}_{1 \leq i \leq n}$ as:
                    $$\alpha := \sum_{1 \leq i \leq n} c_i \alpha_i$$
                Then we say that $\alpha$ is \textbf{positive} if and only if $c_i \geq 0$ for all $1 \leq i \leq n$ and \textbf{negative} if and only if $c_i \leq 0$ for all $1 \leq i \leq n$. The set of positive (respectively, negative) roots of the given Dynkin quiver $\Gamma$ is denoted by $\Phi_{\Gamma}^+$ (respectively, $\Phi_{\Gamma}^-$).
            \end{definition}
            \begin{remark}
                Equivalently, for any Dynkin quiver $\Gamma := (\Gamma_1, \Gamma_0, s, t)$, one might define the positive/negative roots as elements of the following sets:
                    $$\Phi_{\Gamma}^+ := \{\alpha \in \N^{\Gamma_0} \mid q_{\Gamma}(\alpha) = 1\}$$
                    $$\Phi_{\Gamma}^- := \{\alpha \in -\N^{\Gamma_0} \mid q_{\Gamma}(\alpha) = 1\}$$
            \end{remark}
            \begin{remark}[Number of positive roots of Dynkin quivers] \label{remark: number_of_positive_roots_of_dynkin_quivers}
                Obviously, one has:
                    $$\Phi_{\Gamma} = \Phi_{\Gamma}^+ \cup \Phi_{\Gamma}^- = \{\alpha \in \Z^{\oplus \Gamma_0} \mid q_{\Gamma}(\alpha) = 1\}$$
                for all Dynkin quivers $\Gamma$. Furthermore, it is easy to see that:
                    $$|\Phi_{\Gamma}^+| = |\Phi_{\Gamma}^-|$$
                and from the proof of lemma \ref{lemma: roots_are_exclusively_either_negative_or_positive}, one sees that since roots are bounded with respect to the norm $|-|_{\Gamma} := \sqrt{q_{\Gamma}(-)}$, there can only be finitely many of them.
            \end{remark}
            
            Now that we are familiar with Cartan quadratic forms, let us see how they arise naturally as Euler characteristics of representations of quivers (cf. proposition \ref{prop: tits_quadratic_forms_as_euler_characteristics}). 
            \begin{definition}[Simple Grothendieck groups] \label{def: simple_grothendieck_groups}
                The \textbf{simple Grothendieck group} of an abelian (respectively, triangulated) category $\E$, denoted by $\K_0^{\simple}(\E)$, is defined to be the abelian group generated by the set of isomorphism classes of objects of $\E$, subjected to the relations:
                    $$[M] = [M'] + [M'']$$
                on all short exact sequences (respectively, exact triangles):
                    $$M' \to M \to M''$$
            \end{definition}
            \begin{remark}
                Let $\E$ be either be an abelian category or triangulated category. Then clearly:
                    $$[M \oplus N] = [M] + [N]$$
                (simply consider the canonical split short exact sequence $0 \to M \to M \oplus N \to N \to 0$).
            \end{remark}
            \begin{convention}[Jordan-H\"older multiplicities]
                Recall that by Schur's Lemma (cf. lemma \ref{lemma: schur_lemma_for_abelian_categories}), all morphisms between simple objects of any given abelian category are either zero or isomorphisms. Recall also, that by the Jordan-H\"older Theorem (cf. theorem \ref{theorem: jordan_holder_theorem}), Finite-length objects have Jordan-H\"older filtrations which are unique up to isomorphisms, which in particular implies that the simple factors are also unique up to isomorphisms. As such, one may speak of the \textbf{multiplicity} (cf. definition \ref{def: lengths_of_objects_and_jordan_holder_series}) of \textit{any} given simple object $E \in \Ob(\E)$ in any finite ($\Z$-linear) abelian category $\E$ (cf. definition \ref{def: finite_linear_categories}) within the Jordan-H\"older series of some other object $M \in \Ob(\E)$, which we shall denote by:
                    $$[M : E]$$
                Note that $M$ is necessarily of finite length per remark \ref{remark: locally_finite_linear_categories_are_jordan_holder_and_krull_schmidt} and the fact that finite abelian categories are locally finite by definition (cf. proposition \ref{prop: finite_linear_categories_are_locally_finite}).
            \end{convention}
            \begin{proposition}[Simple Grothendieck groups are free on simple objects] \label{prop: simple_grothendieck_groups_of_finite_linear_abelian_categories_are_free_on_simple_objects} 
                Let $k$ be an Artinian commutative ring and $\E$ be a finite $k$-linear abelian category with coefficient algebra $A$ (cf. definition \ref{def: finite_linear_categories}). The simple Grothendieck group $\K_0^{\simple}(\E)$ will then be isomorphic to the free abelian group on the \textit{a priori} finite (cf. proposition \ref{prop: simple_objects_in_finite_linear_categories}) set of isomorphism classes of simple objects\footnote{Hence the notation.} of $\E$, i.e.:
                    $$\K_0^{\simple}(\E) \cong \Z^{\oplus [\E^{\simple}]}$$
                In particular, given any object $M \in \Ob(\E)$, one can write:
                    $$[M] := \sum_{[E] \in [\E^{\simple}]} [M : E] [E]$$
            \end{proposition}
                \begin{proof}
                    
                \end{proof}
                
            \begin{definition}[Projective Grothendieck groups] \label{def: projective_grothendieck_groups}
                Let $k$ be an Artinian commutative ring. The \textbf{projective Grothendieck group} of a finite $k$-linear abelian category $\E$, denoted by $\K_0^{\proj}(\E)$, is defined to be the abelian group generated by the set $[\E^{\proj}]$ of isomorphism classes of projective objects of $\E$, subjected to the relations:
                    $$[M] = [M'] + [M'']$$
                on all short exact sequences:
                    $$0 \to M' \to M \to M'' \to 0$$
            \end{definition}
            \begin{proposition}[Projective Grothendieck groups are free on projective indecomposable objects] \label{prop: projective_grothendieck_groups_are_free_on_projecitve_indecomposable_objects}
                Let $k$ be an Artinian commutative ring. The projective Grothendieck group of any finite $k$-linear abelian category $\E$ is then isomorphic to the free abelian group on the set of isomorphism classes of projective indecomposable objects of $\E$, i.e.:
                    $$\K_0^{\proj}(\E) \cong \Z^{\oplus [\E^{\proj, \indecomp}]}$$
            \end{proposition}
                \begin{proof}
                    
                \end{proof}
            \begin{lemma}[Projective indecomposable modules over Artinian algebras are simple] \label{lemma: projective_indecomposable_modules_over_artinian_algebras_are_simple}
                \footnote{One interesting consequence of this lemma (which is not too relevant to the current discussion about Dynkin quivers, as path algebras of quivers - even over fields - are generally not semi-simple, only hereditary) is that it implies via lemma \ref{lemma: projective_and_injective_modules_over_semi_simple_rings} that over an Artinian ring that is also semi-simple (i.e. an Artinian ring with trivial Jacobson radical), a module is indecomposable if and only if it is simple.} Let $A$ be a left/right-Artinian ring. Then, there exists a surjection:
                    $$
                        \left\{\text{projective left/right-$A$-modules}\right\}
                        \to
                        \left\{\text{simple left/right-$A$-modules}\right\}
                    $$
                whose pre-images over each simple left/right-$A$-module $M$ is a choice of projective cover $\e: P \to M$ wherein $P$ is indecomposable as an $A$-module, which is unique up to isomorphisms. 
            \end{lemma}
                \begin{proof}
                    First of all, because the category of left/right-$A$-modules \textit{a priori} has enough projectives, meaning that for all left/right-$A$-modules $M$ there exists a projective cover $\e: P \to M$, we certainly have a surjection:
                        $$
                            \left\{\text{projective left/right-$A$-modules}\right\}
                            \to
                            \left\{\text{simple left/right-$A$-modules}\right\}
                        $$
                    Now, in order to show that the pre-image of the class of simple left/right-$A$-modules under this surjection is that of indecomposable projective left/right-$A$-modules, start by recalling that simple left/right-modules are cyclic\footnote{The proof is simple (pun not intended!): if we were to suppose to the contrary that there existed a simple module $M$ with $\geq 1$ generators, then said module will admit non-zero (cyclic) proper submodules generated by the generators, and therefore $M$ can not be simple, contradicting our initial assumption.}. This tells us that every simple left/right-$A$-module admits a projective cover by a free\footnote{We assume the Axiom of Choice, so that free module would be projective.} left/right-$A$-module on $1$ generator, which is of course indecomposable and unique up to isomorphisms, per the universal property of left-adjoints.
                \end{proof}
            \begin{proposition}[Projective and simple Grothendieck groups coincide] \label{prop: projective_and_simple_grothendieck_groups_coincide}
                Let $k$ be an Artinian commutative ring and $\E$ be any finite $k$-linear abelian category with coefficient algebra $A$ (cf. definition \ref{def: finite_linear_categories}). Then there will be a group isomorphism:
                    $$\K_0^{\proj}(\E) \cong \K_0^{\simple}(\E)$$
                given by $[M] \mapsto [M/\rad(A) M]$.
            \end{proposition}
                \begin{proof}
                    Combine lemma \ref{lemma: projective_indecomposable_modules_over_artinian_algebras_are_simple} with proposition \ref{prop: simple_grothendieck_groups_of_finite_linear_abelian_categories_are_free_on_simple_objects}.
                \end{proof}
            \begin{convention}
                Due to proposition \ref{prop: projective_and_simple_grothendieck_groups_coincide}, we shall henceforth denote \textit{the} Grothendieck group of any given finite linear abelian category $\E$ simply by $\K_0(\E)$.
            \end{convention}
            \begin{convention}[Grothendieck groups of finite quivers]
                If $Q$ is a finite quiver and if it is understood from context that we were considering finite-rank $k$-linear representations of $Q$ (for some Artinian commutative ring $k$), then by $\K_0(Q)$ we shall mean $\K_0(\Rep_k^{\fin}(Q))$.
            \end{convention}
            
            \begin{definition}[Dimension vectors of quiver representations] \label{def: dimension_vectors_of_quiver_representations}
                Let $k$ be a commutative ring, let $Q := (Q_1, Q_0, s, t)$ be a finite quiver, and consider the finite $k$-linear abelian category $\Rep_k^{\fin}(Q)$ of finite-rank $k$-linear representations of $Q$. We then call the assignment:
                    $$\rank_k: \Rep_k^{\fin}(Q) \to \Z^{\oplus Q_0}$$
                    $$\calF \mapsto \sum_{v \in Q_0} \rank_k \calF(v) e_v$$
                the assignment of \textbf{rank vectors} (or \textbf{dimension vectors} when $k$ is a field) to $k$-linear representations of $Q$, wherein the set $\{e_v\}_{v \in Q_0}$ is some choice of basis for the free $\Z$-module $\Z^{\oplus Q_0}$.
            \end{definition}
            \begin{remark}[Taking dimension vectors is $\Z$-linear] \label{remark: taking_dimension_vectors_is_Z_linear}
                Let $k$ be a field and let $Q := (Q_1, Q_0, s, t)$ be a finite quiver. It is not hard to notice that taking rank vectors of $k$-linear representations of $Q$ is additive, i.e.:
                    $$\rank_k (\calF \oplus \calF') = \rank_k \calF + \rank_k \calF'$$
                for all $\calF, \calF' \in \Ob(\Rep_k^{\fin}(Q))$. From this, one sees that there is an induced $\Z$-module homomorphism:
                    $$\rank_k: \K_0(Q) \to \Z^{\oplus Q_0}$$
            \end{remark}
            \begin{lemma}[Irreducible representations of Dynkin quivers are labelled by vertices] \label{lemma: irreducible_representations_of_finite_quivers_are_labelled_by_vertices}
                Let $k$ be a field and let $\Gamma := (\Gamma_1, \Gamma_0, s, t)$ be a finite quiver. There is then a $\Z$-module isomorphism:
                    $$\dim_k: \K_0(\Gamma) \to \Z^{\oplus \Gamma_0}$$
            \end{lemma}
                \begin{proof}
                    The map $\dim_k: \K_0(\Gamma) \to \Z^{\oplus \Gamma_0}$ is already $\Z$-linear (cf. remark \ref{remark: taking_dimension_vectors_is_Z_linear}), so we need to only show that it is bijective. To show that it is injective, simply consider:
                        $$\ker \dim_k := \{[\calF] \in \K_0(\Gamma) \mid \dim_k \calF = 0\} = \{0\}$$
                    To see that the map is surjective, let us first enumerate the vertices of $\Gamma$ by:
                        $$\Gamma_0 := \{v_1, ..., v_n\}$$
                    Then, observe that the simple roots $\alpha_i \in \Phi_{\Gamma}^{\simple}$ (cf. example \ref{example: simple_roots}) are nothing but the dimension vectors of the representations $\Delta_i \in \Ob(\Rep_k^{\fin}(\Gamma))$ given by:
                        $$\Delta_i(v_j) := k^{\oplus \delta_{ij}}$$
                    for all $1 \leq j \leq n$ (note that the transition maps $\Delta_i(v_j) \to \Delta(v_{j'})$ do indeed exist thanks to the universal property of the zero vector space as a zero object of $k\-\Vect$). These representations are clearly irreducible, and since $\Phi_{\Gamma}^{\simple}$ and $[\Rep_k^{\irr}(\Gamma)]$ respectively form bases for the free $\Z$-modules $\Z^{\oplus \Gamma_0}$ and $\K_0(\Gamma)$, we have in fact shown that for all $\beta \in \Z^{\oplus \Gamma_0}$, there exists an isomorphism class of representations $[\calF] \in \K_0(\Gamma)$ such that $\dim_k \calF = \beta$; of course, this means that $\dim_k: \K_0(\Gamma) \to \Z^{\oplus \Gamma_0}$ is also surjective.
                \end{proof}
            \begin{corollary}[Irreducible representations of Dynkin quivers are labelled by vertices] \label{coro: irreducible_representations_of_finite_quivers_are_labelled_by_vertices}
                Let $k$ be a field and $\Gamma := (\Gamma_1, \Gamma_0, s, t)$ be a Dynkin quiver. Then, there are choices of bijections:
                    $$[\Rep_k^{\irr}(\Gamma)] \cong \Gamma_0$$
                between the set of isomorphism classes of irreducible $k$-linear representations of $\Gamma$ and that of vertices of $\Gamma_0$.
            \end{corollary}
                \begin{proof}
                    Since the Grothendieck group $\K_0(\Gamma)$ is freely generated as a $\Z$-module by the set of isomorphism classes of irreducible $k$-linear representations of $\Gamma$ (cf. proposition \ref{prop: simple_grothendieck_groups_of_finite_linear_abelian_categories_are_free_on_simple_objects}), this is an immediate consequence of lemma \ref{lemma: irreducible_representations_of_finite_quivers_are_labelled_by_vertices}.
                \end{proof}
            \begin{proposition}[Irreducible representations of Dynkin quivers are labelled by simple roots] \label{prop: irreducible_representations_of_dynkin_quivers_are_labelled_by_simple_roots}
                Let $k$ be a field and $\Gamma := (\Gamma_1, \Gamma_0, s, t)$ be a Dynkin quiver. Then, there are choices of bijections (depending on choices of basis for $\Z^{\oplus \Gamma_0}$):
                    $$\rank_k: [\Rep_k^{\irr}(\Gamma)] \to \Phi_{\Gamma}^{\simple}$$
                between the set of isomorphism classes irreducible $k$-linear representations of $\Gamma$ and the simple roots of $\Gamma$ (cf. example \ref{example: simple_roots}).
            \end{proposition}
                \begin{proof}
                    This comes from the fact that the simple roots of $\Gamma$ form a basis for $\Z^{\oplus \Gamma_0}$ (cf. example \ref{example: simple_roots}), which we know by lemma \ref{lemma: irreducible_representations_of_finite_quivers_are_labelled_by_vertices} is isomorphic to the Grothendieck group $\K_0(\Gamma)$ via:
                        $$\rank_k: \K_0(\Gamma) \to \Z^{\oplus \Gamma_0}$$
                    and the Grothendieck group $\K_0(\Gamma)$ itself is freely generated by the set $[\Rep_k^{\irr}(\Gamma)]$ of irreducible $k$-linear representations of $\Gamma$ (cf. proposition \ref{prop: simple_grothendieck_groups_of_finite_linear_abelian_categories_are_free_on_simple_objects}).
                \end{proof}
            \begin{remark}
                Let $\Gamma$ be a Dynkin quiver. Then because every root $\alpha \in \Phi_{\Gamma}$ is just a $\Z$-linear combination of simple roots, we see that:
                    $$\Z^{\oplus \Gamma_0} \cong \span_{\Z} \Phi_{\Gamma}^+ \cong \span_{\Z} \Phi_{\Gamma}^- = \span_{\Z} \Phi_{\Gamma}$$
                though neither $\Phi_{\Gamma}^+$ nor $\Phi_{\Gamma}^-$ are $\Z$-linearly independent subsets of $\Z^{\oplus \Gamma_0}$; of course, $\Phi_{\Gamma}$ is certainly not a $\Z$-linearly independent subset of $\Z^{\oplus \Gamma_0}$, since $\Phi_{\Gamma} = \Phi_{\Gamma}^+ \cup \Phi_{\Gamma}^-$.
            \end{remark}
            
        \subsubsection{(Co)reflection functors}
            The proof of Gabriel's Theorem (that a quiver is Dynkin if it has finitely many indecomposable representations over a field) relies heavily on constructions known as \say{(co)reflection functors} (cf. definition \ref{def: (co)reflection_functors}) which can be thought of as gadgets that help us ensure that the representation theory of a quiver is independent of its orientation\footnote{So for instance, once we have proven Gabriel's Theorem, we can simply think of Dynkin quivers as those whose underlying undirected graph is a Dynkin diagram, and this notion is completely captured by the Tits quadratic forms associated to these undirected graphs.}. 
            
            \begin{convention}
                If $Q := (Q_1, Q_0, s, t)$ is a quiver and $\sigma \in Q_0$ is any vertex therein, then we shall implicitly write $\sigma Q$ for the quiver wherein one reverses all the arrow into/out of $\sigma$ and keep the remaining arrows unchanged.
            \end{convention}
            \begin{convention}
                From now on, let us work over a field $k$ so that path $k$-algebras of finite quivers are hereditary. We make no further assumptions on $k$, e.g. we do not need $k$ to be algebraically closed or of any particular characteristic.
            \end{convention}
            \begin{definition}[(Co)reflection functors] \label{def: (co)reflection_functors}
                Let $Q := (Q_1, Q_0, s, t)$ be a quiver and $\sigma \in Q_0$ be a fixed vertex. One can then define the \textbf{reflection functor} based at the vertex $\sigma$ as:
                    $$\Refl_{Q, \sigma}: \Rep_k(Q) \to \Rep_k(\sigma Q)$$
                    $$(\calF: [Q] \to k\mod) \mapsto \left( \Refl_{Q, \sigma}(\calF): [Q] \to k\mod: v \mapsto \begin{cases} \text{$\calF(v)$ if $v \in Q_0 \setminus \{\sigma\}$} \\ \text{$\underset{v \to \sigma, v \not = \sigma}{\lim} \calF(v)$ if $v = \sigma$} \end{cases} \right)$$
                and the \textbf{coreflection functor} based at the vertex $\sigma$ respectively as:
                    $$\co\Refl_{Q, \sigma}: \Rep_k(Q) \to \Rep_k(\sigma Q)$$
                    $$(\calF: [Q] \to k\mod) \mapsto \left( \Refl_{Q, \sigma}(\calF): [Q] \to k\mod: v \mapsto \begin{cases} \text{$\calF(v)$ if $v \in Q_0 \setminus \{\sigma\}$} \\ \text{$\underset{v \leftarrow \sigma, v \not = \sigma}{\colim} \calF(v)$ if $v = \sigma$} \end{cases} \right)$$
            \end{definition}
            \begin{convention}
                Let $Q := (Q_1, Q_0, s, t)$ be a quiver and $\sigma \in Q_0$ be a fixed vertex. Then, for all $\calF \in \Ob(\Rep_k(Q))$, we shall abuse notations slightly and write:
                    $$\Refl_{Q, \sigma}(\calF) := \underset{v \to \sigma, v \not = \sigma}{\lim} \calF(v)$$
                    $$\co\Refl_{Q, \sigma}(\calF) := \underset{v \leftarrow \sigma, v \not = \sigma}{\colim} \calF(v)$$
                when there is no risk of confusion. Keep in mind, over the remarks to come, that this abuse of notations of ours actually does not obscure any important categorical properties of the (co)reflection functors such as exactness.
            \end{convention}
            \begin{remark}[Categorical properties of (co)reflection functors] \label{remark: categorical_properties_of_(co)reflection_functors}
                Immediately, one sees that (co)reflection functors preserve (co)limits of representations. More specifically, if $Q := (Q_1, Q_0, s, t)$ is a quiver and $\sigma \in Q_0$ is a fixed vertex, and if:
                    $$I \to \Rep_k(Q)$$
                    $$i \mapsto \calF_i$$
                is a diagram of $k$-linear representations of $Q$, then clearly:
                    $$\Refl_{Q, \sigma}\left( \underset{i \in I^{\op}}{\lim} \calF_i \right) \cong \underset{i \in I}{\lim} \Refl_{Q, \sigma}(\calF)$$
                and:
                    $$\co\Refl_{Q, \sigma}\left( \underset{i \in I}{\colim} F_i \right) \cong \underset{i \in I}{\colim} \co\Refl_{Q, \sigma}(F)$$
                as a result of (co)limits commuting with one another. Of course, all the (co)limits at play here do indeed exist, since $\Rep_k(Q)$ is both complete and cocomplete as a result of $k\mod$ being so. The aforementioned (co)limit preserving properties of the (co)reflection functors:
                    $$\Refl_{Q, \sigma}, \co\Refl_{Q, \sigma}$$
                also yields us the fact that, actually, there is an adjunction:
                    $$
                        \begin{tikzcd}
                        	{\Rep_k(\sigma Q)} & {\Rep_k(Q)}
                        	\arrow[""{name=0, anchor=center, inner sep=0}, "{\Refl_{\sigma Q, \sigma}}"', bend right, from=1-1, to=1-2]
                        	\arrow[""{name=1, anchor=center, inner sep=0}, "{\co\Refl_{Q, \sigma}}"', bend right, from=1-2, to=1-1]
                        	\arrow["\dashv"{anchor=center, rotate=-90}, draw=none, from=1, to=0]
                        \end{tikzcd}
                    $$
                For a proof, simply apply the Adjoint Functor Theorem (cf. \cite{nlab:adjoint_functor_theorem}).
                
                It should also be noted, for the sake of computational practicality, that because (co)limits are built out of (co)products and (co)equalisers, one has the following concrete characterisations of the (co)reflection functors:
                    $$\Refl_{Q, \sigma}(\calF) \cong \ker\left(\bigoplus_{v \to \sigma, v \not = \sigma} \calF(v) \to \calF(\sigma)\right)$$
                    $$\co\Refl_{Q, \sigma}(\calF) := \coker\left(\calF(\sigma) \to \bigoplus_{v \leftarrow \sigma, v \not = \sigma} \calF(v)\right)$$
            \end{remark}
            \begin{remark}[(Co)reflection functors and Grothendieck groups] \label{remark: (co)reflection_functors_and_grothendieck_groups}
                Let $Q := (Q_1, Q_0, s, t)$ be a quiver with a fixed vertex $\sigma \in Q_0$. Then, there are two $\Z$-module homomorphisms as follows, defined in the obvious manner:
                    $$
                        \begin{tikzcd}
                        	{\K_0(\sigma Q)} & {\K_0(Q)}
                        	\arrow[""{name=0, anchor=center, inner sep=0}, "{\Refl_{\sigma Q, \sigma}}"', bend right, from=1-1, to=1-2]
                        	\arrow[""{name=1, anchor=center, inner sep=0}, "{\co\Refl_{Q, \sigma}}"', bend right, from=1-2, to=1-1]
                        \end{tikzcd}
                    $$
                If $Q$ is finite then these maps will also be adjoint to one another, with respect to the Cartan form\footnote{Which is - by construction - independent of the orientation of the underlying quiver; in particular, this means that $B_{\sigma Q}(-, -) = B_Q(-, -)$} $B_Q(-, -)$ (cf. definition \ref{def: cartan_quadratic_forms_of_finite_quivers}).
            \end{remark}
            
            \begin{definition}[(Co)sinks] \label{def: (co)sinks}
                Let $Q := (Q_1, Q_0, s, t)$ be a quiver and $\sigma \in Q_0$ be a fixed vertex therein. A \textbf{(co)sink} at $\sigma$ is then a sub-quiver of $Q$ wherein all the arrows point into/out of the fixed vertex $\sigma$. 
            \end{definition}
            \begin{convention}[(Co)sinks at the level of representations]
                Since representations of quivers are functors on the free categories thereon, if $\sigma \in Q_0$ is a (co)sink in a quiver $Q := (Q_1, Q_0, s, t)$ and if $\calF \in \Ob(\Rep_k(Q))$ is a representation then we shall also say that $\calF(\sigma)$ is a (co)sink in the diagram $\calF$.  
            \end{convention}
            \begin{definition}[Admissible enumerations of (co)sinks] \label{def: admissible_enumerations_of_(co)sinks}
                Let $Q := (Q_1, Q_0, s, t)$; also, fix an arbitrary representation $\calF \in \Ob(\Rep_k(Q))$. 
                    \begin{itemize}
                        \item \textbf{(Admissible enumeration of sinks):} An \textbf{admissible enumeration of sinks} on the vertices of $Q$ (i.e. on the set $Q_0$) is then a decreasing total ordering $\geq$ on the set $Q_0$, which is defined inductively in the following manner: if some $\sigma_1 \in Q_0$ is a sink then for all $1 \leq i \leq |Q_0|$, one shall have:
                            $$\sigma_{i - 1} \geq \sigma_i$$
                        if and only if:
                            $$(\co\Refl_{\sigma_{i - 1} ... \sigma_1 Q, \sigma_i} \circ ... \circ \co\Refl_{Q, \sigma_1})(\calF)(\sigma_i)$$
                        is also a sink.
                        \item \textbf{(Admissible coenumeration of cosinks):} An \textbf{admissible coenumeration of cosinks} on the vertices of $Q$ (i.e. on the set $Q_0$) is then an increasing total ordering $\leq$ on the set $Q_0$, which is defined inductively in the following manner: if some $\sigma_1 \in Q_0$ is a sink then for all $1 \leq i \leq |Q_0|$, one shall have:
                            $$\sigma_{i - 1} \leq \sigma_i$$
                        if and only if:
                            $$(\Refl_{\sigma_{i - 1} ... \sigma_1 Q, \sigma_i} \circ ... \circ \Refl_{Q, \sigma_1})(\calF)(\sigma_i)$$
                        is also a cosink.
                    \end{itemize}
            \end{definition}
            \begin{remark}
                Obviously, if a quiver $Q := (Q_1, Q_0, s, t)$ has an admissible enumeration of sinks on its set of vertices $Q_0$, say:
                    $$Q_0 := \{\sigma_1 \geq ... \geq \sigma_n\}$$
                then the opposite quiver $Q^{\op} := (Q_1, Q_0, s^{\op} := t, t^{\op} := s)$ will have an admissible coenumeration of cosinks given by:
                    $$Q^{\op}_0 = \{\sigma_1 \leq ... \leq \sigma_n\}$$
            \end{remark}
            \begin{proposition}[Existence of admissible enumerations] \label{prop: existence_of_admissible_enumerations}
                If $Q := (Q_1, Q_0, s, t)$ is a connected acyclic quiver then there will always exist admissible (co)enumerations of (co)sinks on its set of vertices whenever $Q$ has a (co)sink. 
            \end{proposition}
                \begin{proof}
                    
                \end{proof}
                
            \begin{definition}[Coxeter functors] \label{def: coxeter_functors}
                Let $\Gamma := (\Gamma_1, \Gamma_0, s, t)$ be a connected and acyclic finite quiver with a \textit{chosen} admissible enumeration of sinks:
                    $$\Gamma_0 := \{\sigma_1 \geq ... \geq \sigma_n\}$$
                Then, for all $1 \leq i \leq n$, the $i^{th}$ \textbf{(co)Coxeter functors} associated to the aforementioned admissible (co)enumerated quiver $\Gamma$ are then the following compositions of (co)reflection functors:
                    $$\co\Cox_{\Gamma, \sigma_1 \geq ...\geq \sigma_n}^{(i)} := \co\Refl_{\sigma_i ... \sigma_1 \Gamma, \sigma_i} \circ ... \circ \co\Refl_{\Gamma, \sigma_1}$$
                    $$\Cox_{\Gamma, \sigma_1 \geq ...\geq \sigma_n}^{(i)} := \Refl_{\Gamma, \sigma_1} \circ ... \circ \Refl_{\sigma_1 ... \sigma_i \Gamma, \sigma_i}$$
            \end{definition}
            \begin{remark}
                Let $\Gamma := (\Gamma_1, \Gamma_0, s, t)$ be a connected and acyclic finite quiver with a \textit{chosen} admissible enumeration of sinks:
                    $$\Gamma_0 := \{\sigma_1 \geq ... \geq \sigma_n\}$$
                Then, the $i^{th}$ (co)Coxeter functors $\co\Cox^{(i)}_{\Gamma, \sigma_1 \geq ...\geq \sigma_n}$ and $\Cox^{(i)}_{\Gamma, \sigma_1 \geq ...\geq \sigma_n}$, respectively, shall be right and left-exact by virtue of being compositions of right and left-exact functors, namely the coreflection and reflection functors $\co\Refl_{\sigma_i ... \sigma_1 \Gamma, \sigma_i}$ and $\Refl_{\sigma_1 ... \sigma_i \Gamma, \sigma_i}$ (for all $1 \leq i \leq n$).
            \end{remark}
            \begin{proposition}[Coxeter functors are independent of admissible enumerations] \label{prop: coxeter_functors_are_independent_of_admissible_enumerations}
                Let $\Gamma := (\Gamma_1, \Gamma_0, s, t)$ be a connected and acyclic finite quiver with a chosen admissible enumeration of sinks:
                    $$\Gamma_0 := \{\sigma_1 \geq ... \geq \sigma_n\}$$
                Then, the $i^{th}$ (co)Coxeter functors $\Cox^{(i)}_{\Gamma, \sigma_1 \geq ...\geq \sigma_n}$ and $\co\Cox^{(i)}_{\Gamma, \sigma_1 \geq ...\geq \sigma_n}$ are actually independent of the choice of admissible enumeration of sinks $\sigma_1 \geq ... \geq \sigma_n$. 
            \end{proposition}
                \begin{proof}
                    
                \end{proof}
            \begin{lemma}[Coxeter functors on finite connected acyclic quivers are adjoint] \label{lemma: coxeter_functors_are_adjoints}
                Let $\Gamma := (\Gamma_1, \Gamma_0, s, t)$ be a connected and acyclic finite quiver with an admissible enumeration of sinks:
                    $$\Gamma_0 := \{\sigma_1 \geq ... \geq \sigma_n\}$$
                The $i^{th}$ (co)Coxeter functors $\Cox^{(i)}_{\Gamma, \sigma_1 \geq ...\geq \sigma_n}$ and $\co\Cox^{(i)}_{\Gamma, \sigma_1 \geq ...\geq \sigma_n}$ form a pair of adjoint functors in the following manner for all $1 \leq i \leq n$:
                    $$
                        \begin{tikzcd}
                        	{\Rep_k(\sigma_i ... \sigma_1\Gamma)} & {\Rep_k(\Gamma)}
                        	\arrow[""{name=0, anchor=center, inner sep=0}, "{\Cox^{(i)}_{\Gamma, \sigma_1 \geq ...\geq \sigma_n}}"', bend right, from=1-1, to=1-2]
                        	\arrow[""{name=1, anchor=center, inner sep=0}, "{\co\Cox^{(i)}_{\Gamma, \sigma_1 \geq ...\geq \sigma_n}}"', bend right, from=1-2, to=1-1]
                        	\arrow["\dashv"{anchor=center, rotate=-90}, draw=none, from=1, to=0]
                        \end{tikzcd}
                    $$
            \end{lemma}
                \begin{proof}
                    See remark \ref{remark: categorical_properties_of_(co)reflection_functors}.
                \end{proof}
            \begin{corollary}[Coxeter functors and Grothendieck groups] \label{coro: coxeter_functors_and_grothendieck_groups}
                Let $\Gamma := (\Gamma_1, \Gamma_0, s, t)$ be a connected and acyclic finite quiver with an admissible enumeration of sinks:
                    $$\Gamma_0 := \{\sigma_1 \geq ... \geq \sigma_n\}$$
                Then, for all $1 \leq i \leq n$, there are two $\Z$-module homomorphisms as follows, defined in the obvious manner:
                    $$
                        \begin{tikzcd}
                        	{\K_0(\sigma_n ... \sigma_1 \Gamma)} & {\K_0(\Gamma)}
                        	\arrow[""{name=0, anchor=center, inner sep=0}, "{\Cox_{\Gamma, \sigma_1 \geq ...\geq \sigma_n}^{(i)}}"', bend right, from=1-1, to=1-2]
                        	\arrow[""{name=1, anchor=center, inner sep=0}, "{\co\Cox_{\Gamma, \sigma_1 \geq ...\geq \sigma_n}^{(i)}}"', bend right, from=1-2, to=1-1]
                        \end{tikzcd}
                    $$
                which are adjoint\footnote{And hence unitary.} with respect to the Cartan form\footnote{Which is - by construction - independent of the orientation of the underlying quiver; in particular, this means that $B_{\sigma_n ... \sigma_1 \Gamma}(-, -) = B_{\Gamma}(-, -)$} $B_{\Gamma}(-, -)$ (cf. definition \ref{def: cartan_quadratic_forms_of_finite_quivers}).
            \end{corollary}
                \begin{proof}
                    Apply lemma \ref{lemma: coxeter_functors_are_adjoints} to remark \ref{remark: (co)reflection_functors_and_grothendieck_groups}. 
                \end{proof}
            \begin{remark}[Coxeter functors act as zero on irreducible representations] \label{remark: coxeter_functors_act_as_zero_on_irreducible_representations}
                Let $\Gamma := (\Gamma_1, \Gamma_0, s, t)$ be a connected and acyclic finite quiver with an admissible enumeration of sinks:
                    $$\Gamma_0 := \{\sigma_1 \geq ... \geq \sigma_n\}$$
                Then, as a consequence of definitions \ref{def: (co)reflection_functors} and \ref{def: coxeter_functors}, the $i^{th}$ Coxeter functors:
                    $$\Cox_{\Gamma, \sigma_1 \geq ...\geq \sigma_n}^{(i)}, \co\Cox_{\Gamma, \sigma_1 \geq ...\geq \sigma_n}^{(i)}$$
                act on all the Kronecker representations $\Delta_{\sigma_i ... \sigma_1, \sigma_j \Gamma, \sigma_j}, \Delta_{\Gamma, \sigma_j}$ as the identity when $i < j$ and as zero when $i \geq j$. Notably, the $n^{th}$ (co)Coxeter functors:
                    $$\Cox_{\Gamma, \sigma_1 \geq ...\geq \sigma_n}^{(n)}, \co\Cox_{\Gamma, \sigma_1 \geq ...\geq \sigma_n}^{(n)}$$
                act as zero on all objects of $\Rep_k^{\irr}(\sigma_n ... \sigma_1 \Gamma)$ and those of $\Rep_k^{\irr}(\Gamma)$, respectively (which we recall, using proposition \ref{prop: irreducible_representations_of_dynkin_quivers_are_labelled_by_simple_roots}, are in bijection with the Kronecker representations, which themselves are in bijection with $\Phi_{\Gamma}^{\simple}$ by construction). 
            \end{remark}
            
            Now, are the irreducible representations the only ones on which the (co)Coxeter functors act as zero ? The answer is no, as can be inferred from example \ref{example: indecomposable_representations_of_A2_quiver}: both (co)Coxeter functors act as zero on its only non-irreducible indecomposable representation, thereby suggesting that the \say{kernels} of these functors contain more than just the irreducible representations. This leads us to the definition of so-called preprojective and preinjective representations. 
            \begin{convention}[Kronecker representations] \label{conv: kronecker_representations}
                Let $\Gamma := (\Gamma_1, \Gamma_0, s, t)$ be a connected and acyclic finite quiver and enumerate its vertices by:
                    $$\Gamma := \{\sigma_1, ..., \sigma_n\}$$
                where $n := |\Gamma_0|$. Then for all $1 \leq i, j \leq n$, the $i^{th}$ \textbf{Kronecker representation} of $\Gamma$ is then the object $\Delta_{\Gamma, \sigma_i} \in \Ob(\Rep_k^{\irr}(\Gamma))$ given by:
                    $$\Delta_{\Gamma, \sigma_i}(\sigma_j) := k^{\oplus \delta_{ij}}$$
            \end{convention}
            \begin{definition}[Preprojective and preinjective representations] \label{def: preprojective_and_preinjective_representations}
                Let $\Gamma := (\Gamma_1, \Gamma_0, s, t)$ be a connected and acyclic finite quiver with an admissible enumeration of sinks:
                    $$\Gamma_0 := \{\sigma_1 \geq ... \geq \sigma_n\}$$
                For all $1 \leq r \leq n$, a (necessarily finite-dimensional) representation $\calF \in \Ob(\Rep_k^{\fin}(\sigma_r ... \sigma_1 \Gamma))$ is said to be \textbf{$r$-preprojective at $\sigma_i$} if and only if:
                    $$\calF \cong \co\Cox_{\Gamma, \sigma_1 \geq ...\geq \sigma_n}^{(r)}(\Delta_{\Gamma, \sigma_i})$$
                for some fixed vertex $\sigma_i \in \Gamma_0$. Dually, one says that a (necessarily finite-dimensional) representation $\calF \in \Ob(\Rep_k^{\fin}(\Gamma))$ is said to be \textbf{$r$-preinjective at $\sigma_i$} if and only if:
                    $$\calF \cong \Cox_{\Gamma, \sigma_1 \geq ...\geq \sigma_n}^{(r)}(\Delta_{\sigma_r ... \sigma_1 \Gamma, \sigma_i})$$
                for some fixed vertex $\sigma_i \in \Gamma_0$.
            \end{definition}
            \begin{definition}[Preprojective, regular, and preinjective components] \label{def: preprojective_regular_and_preinjective_components}
                Let $\Gamma := (\Gamma_1, \Gamma_0, s, t)$ be a connected and acyclic finite quiver with an admissible enumeration of sinks:
                    $$\Gamma_0 := \{\sigma_1 \geq ... \geq \sigma_n\}$$
                    \begin{itemize}
                        \item \textbf{(Preprojective components):} The \textbf{preprojective component} of $\Rep_k^{\fin}(\Gamma)$, denoted by $\Rep_k^{\preproj}(\Gamma)$, is nothing but the essential image of the $n^{th}$ coCoxeter functor:
                            $$\co\Cox_{\Gamma, \sigma_1 \geq ...\geq \sigma_n}^{(n)}: \Rep_k^{\fin}(\Gamma) \to \Rep_k^{\fin}(\sigma_n ... \sigma_1 \Gamma)$$
                        \item \textbf{(Preinjective components):} Dually, the \textbf{preprojective component} of $\Rep_k^{\fin}(\sigma_n ... \sigma_1 \Gamma)$, denoted by $\Rep_k^{\preinj}(\sigma_n ... \sigma_1 \Gamma)$, is the essential image of the $n^{th}$ coCoxeter functor:
                            $$\Cox_{\Gamma, \sigma_1 \geq ...\geq \sigma_n}^{(n)}: \Rep_k^{\fin}(\sigma_n ... \sigma_1 \Gamma) \to \Rep_k^{\fin}(\Gamma)$$
                        \item \textbf{(Regular components):} The \textbf{regular component} of $\Rep_k^{\fin}(\Gamma)$, denoted by $\Rep_k^{\reg}(\Gamma)$ (respectively of $\Rep_k^{\fin}(\sigma_n ... \sigma_1 \Gamma)$, denoted by $\Rep_k^{\reg}(\Gamma)$) is the full subcategory spanned by finite-dimensional representations of $\Gamma$ (respectively, of $\sigma_n ... \sigma_1 \Gamma$) which are neither preprojective nor preinjective. (Finite-dimensional) representations which are either preprojective or preinjective (i.e. not regular) are called \textbf{irregular}.
                    \end{itemize}
            \end{definition}
            \begin{lemma}[Fixed-points of (co)reflection adjunctions] \label{lemma: fixed_points_of_(co)reflection_adjunctions}
                Let $Q$ be a quiver with a fixed vertex $\sigma$, and let $\calF \in \Ob(\Rep_k(Q))$ be an arbitrary representation thereof.
                    \begin{enumerate}
                        \item If:
                            $$\calF(\sigma) \cong \coker \left( \ker: \Refl_{Q, \sigma}(\calF)(\sigma) \to \bigoplus_{v \to \sigma, v \not = \sigma} \calF(v) \right)$$
                        (cf. remark \ref{remark: categorical_properties_of_(co)reflection_functors}) then the adjunction counit map:
                            $$\e_{\calF}: (\co\Refl_{\sigma Q, \sigma} \circ \Refl_{Q, \sigma})(\calF) \to \calF$$
                        will be an isomorphism.
                        \item Dually, if:
                            $$\calF(\sigma) \cong \ker \left( \coker: \bigoplus_{v \leftarrow \sigma, v \not = \sigma} \calF(v) \to \co\Refl_{Q, \sigma}(\calF)(\sigma) \right)$$
                        (cf. remark \ref{remark: categorical_properties_of_(co)reflection_functors}) then the adjunction unit map:
                            $$\eta_{\calF}: \calF \to (\Refl_{\sigma Q, \sigma} \circ \co\Refl_{Q, \sigma})(\calF)$$
                        will be an isomorphism.
                    \end{enumerate}
            \end{lemma}
                \begin{proof}
                    The two cases are formally dual to one another, so we shall only prove that first.
                    
                    If:
                        $$\calF(\sigma) \cong \coker \left( \Refl_{Q, \sigma}(\calF)(\sigma) \to \bigoplus_{v \to \sigma, v \not = \sigma} \calF(v) \right)$$
                    then by remark \ref{remark: categorical_properties_of_(co)reflection_functors}, we will have:
                        $$
                            \begin{aligned}
                                (\co\Refl_{\sigma Q, \sigma} \circ \Refl_{Q, \sigma})(\calF)(\sigma) & \cong \co\Refl_{\sigma Q, \sigma}\left( \ker \left( \bigoplus_{v \to \sigma, v \not = \sigma} \calF(v) \to \calF(\sigma) \right) \right)
                                \\
                                & \cong \coker \left( \ker \left( \bigoplus_{v \to \sigma, v \not = \sigma} \calF(v) \to \calF(\sigma) \right) \to \bigoplus_{v \in \Ob({}^{\sigma/}[\sigma Q])} \calF(v) \right)
                                \\
                                & \cong \coker \left( \ker \left( \bigoplus_{v \to \sigma, v \not = \sigma} \calF(v) \to \calF(\sigma) \right) \to \bigoplus_{v \to \sigma, v \not = \sigma} \calF(v) \right)
                                \\
                                & \cong \calF(\sigma)
                            \end{aligned}
                        $$
                    This proves the claim, since the vertex $\sigma$ was picked arbitrarily. 
                \end{proof}
            \begin{lemma}[(Co)reflections preserve indecomposability] \label{lemma: (co)reflection_functors_preserve_indecomposability}
                Let $\Gamma := (\Gamma_1, \Gamma_0, s, t)$ be a connected and acyclic finite quiver and fix an arbitrary vertex $\sigma \in \Gamma_0$. Then, for all representations $\calF \in \Ob(\Rep_k(Q))$, the (co)reflections:
                    $$\Refl_{\Gamma, \sigma}(\calF), \co\Refl_{\Gamma, \sigma}(\calF)$$
                will be indecomposable\footnote{We count the zero representation as being indecomposable (cf. definition \ref{def: (in)decomposable_objects}).} if and only if $\calF$ itself is indecomposable. 
            \end{lemma}
                \begin{proof}
                    Firstly, assume that $\calF$ is indecomposable. Without any loss of generality, suppose that the fixed vertex $\sigma \in \Gamma_0$ is a sink. Observe that $\calF$ being indecomposable implies that:
                        $$\calF(\sigma) \cong \coker \left( \ker: \Refl_{\Gamma, \sigma}(\calF)(\sigma) \to \bigoplus_{v \to \sigma, v \not = \sigma} \calF(v) \right)$$
                    i.e. there is a short exact sequence of $k$-vector spaces:
                        $$0 \to \Refl_{\Gamma, \sigma}(\calF)(\sigma) \to \bigoplus_{v \to \sigma, v \not = \sigma} \calF(v) \to \calF(\sigma) \to 0$$
                    which is because if otherwise, i.e. if the canonical map $\bigoplus_{v \to \sigma, v \not = \sigma} \calF(v) \to \calF(\sigma)$ has a non-trivial cokernel, then said cokernel would have been a direct summand inside $\calF(\sigma)$ (since every $k$-vector space is free as a $k$-module), thereby violating the indecomposability assumption on $\calF$. Lemma \ref{lemma: fixed_points_of_(co)reflection_adjunctions} then tells us that the adjunction counit:
                        $$\e: \co\Refl_{\sigma \Gamma, \sigma} \circ \Refl_{\Gamma, \sigma} \to \1$$
                    is thus a natural isomorphism of functors on $\Rep_k(\Gamma)$. Suppose now, suppose for the sake of deriving a contradiction, that $\Refl_{\Gamma, \sigma}(\calF)$ is decomposable (showing that $\co\Refl_{\Gamma, \sigma}(\calF)$ is indecomposable is formally dual to this case). $\co\Refl_{\sigma \Gamma, \sigma}$ being left-exact (cf. remark \ref{remark: categorical_properties_of_(co)reflection_functors}) - and hence preserving finite direct sums - then implies $(\co\Refl_{\sigma \Gamma, \sigma} \circ \Refl_{\Gamma, \sigma})(\calF)$ is also decomposable. But this representation of $\Gamma$ is isomorphic to $\calF$, as stated above, so we have a contradiction. As such, $\Refl_{\Gamma, \sigma}(\calF)$ (and likewise, $\co\Refl_{\Gamma, \sigma}(\calF)$) must be indecomposable as a representation of $\sigma \Gamma$ whenever $\calF$ is an indecomposable representation of $\Gamma$.
                    
                    Conversely, suppose that $\Refl_{\Gamma, \sigma}(\calF)$ or $\co\Refl_{\Gamma, \sigma}(\calF)$ is indecomposable. $\calF$ must then also be indecomposable by the right/left-exactness of (co)reflection functors (cf. remark \ref{remark: categorical_properties_of_(co)reflection_functors}), which in particular tells us that finite direct sums are preserved.
                \end{proof}
            \begin{corollary}[Coxeter functors preserve indecomposability] \label{coro: coxeter_functors_preserve_indecomposability}
                Let $\Gamma := (\Gamma_1, \Gamma_0, s, t)$ be a connected and acyclic finite quiver with an admissible enumeration of sinks:
                    $$\Gamma_0 := \{\sigma_1 \geq ... \geq \sigma_n\}$$
                Suppose also that $\calF \in \Ob(\Rep_k(\Gamma))$ (respectively, $\calF \in \Ob(\Rep_k(\sigma_n ... \sigma_1 \Gamma))$) is some representation of $\Gamma$ (respectively, of $\sigma_n ... \sigma_1 \Gamma$). Then:
                    $$\Cox_{\Gamma, \sigma_1 \geq ...\geq \sigma_n}^{(n)}(\calF)$$
                    $$\co\Cox_{\Gamma, \sigma_1 \geq ...\geq \sigma_n}^{(n)}(\calF)$$
                will be indecomposable representations of $\sigma_n ... \sigma_1 \Gamma$ (respectively, of $\Gamma$) if and only if $\calF$ is an indecomposable representation of $\Gamma$ (respectively, of $\sigma_n ... \sigma_1 \Gamma$).
                
                In particular, if we suppose that $\calF \in \Ob(\Rep_k^{\red, \reg}(\Gamma))$ (respectively, $\calF \in \Ob(\Rep_k^{\red, \reg}(\sigma_n ... \sigma_1 \Gamma))$) is an regular indecomposabl representation of $\Gamma$ (respectively, $\sigma_n ... \sigma_1 \Gamma$) then both:
                    $$\e_{\calF}: (\co\Cox_{\Gamma, \sigma_1 \geq ...\geq \sigma_n}^{(n)} \circ \Cox_{\Gamma, \sigma_1 \geq ...\geq \sigma_n}^{(n)})(\calF) \to \calF$$
                and:
                    $$\eta_{\calF}: \calF \to (\Cox_{\Gamma, \sigma_1 \geq ...\geq \sigma_n}^{(n)} \circ \co\Cox_{\Gamma, \sigma_1 \geq ...\geq \sigma_n}^{(n)})(\calF)$$
                will be isomorphisms of regular indecomposable representations, of $\sigma_n ... \sigma_1 \Gamma$ and $\Gamma$ respectively, and these isomorphisms are are natural in $\calF$.
            \end{corollary}
                \begin{proof}
                    Apply lemma \ref{lemma: (co)reflection_functors_preserve_indecomposability} to remark \ref{remark: coxeter_functors_act_as_zero_on_irreducible_representations} and lemma \ref{lemma: coxeter_functors_are_adjoints}.
                \end{proof}
            \begin{corollary}[Preprojective and preinjective representations are indecomposable] \label{coro: preprojective_and_preinjective_representations_are_indecomposable}
                Any preprojective (respectively, preinjective) representation of a connected and acyclic finite quiver must be indecomposable. 
            \end{corollary}    
                \begin{proof}
                    Let $\Gamma := (\Gamma_1, \Gamma_0, s, t)$ be a connected and acyclic finite quiver with an admissible enumeration of sinks:
                        $$\Gamma_0 := \{\sigma_1 \geq ... \geq \sigma_n\}$$
                    For any $1 \leq r \leq n$, any representation $\calF \in \Ob(\Rep_k^{r\-\preproj}(\sigma_r ... \sigma_1 \Gamma))$ of $\sigma_r ... \sigma_1 \Gamma$ which is $r$-preprojective at some vertex $\sigma_i \in \Gamma_0$ is of the form:
                        $$\calF \cong \co\Cox_{\Gamma, \sigma_1 \geq ... \sigma_n}(\Delta_{\Gamma, \sigma_i})$$
                    by definition. As such, it must be indecomposable by corollary \ref{coro: coxeter_functors_preserve_indecomposability} and the fact that irreducible representations are indecomposable (cf. lemma \ref{lemma: projective_indecomposable_modules_over_artinian_algebras_are_simple}).
                \end{proof}
            \begin{remark}
                Corollary \ref{coro: coxeter_functors_preserve_indecomposability} on its own is not enough to imply theorem \ref{theorem: coxeter_functors_fix_regular_indecomposable_representations} directly. This is because we do not yet know whether or not the individual (co)Coxeters themselves preserve regularity of indecomposable representations of connected and acyclic finite quivers. This turns out to indeed be the case, evident through lemma \ref{lemma: coxeter_functors_preserve_regularity}.
            \end{remark}
            \begin{lemma}[Coxeter functors preserve regularity] \label{lemma: coxeter_functors_preserve_regularity}
                Let $\Gamma := (\Gamma_1, \Gamma_0, s, t)$ be a connected and acyclic finite quiver with an admissible enumeration of sinks:
                    $$\Gamma_0 := \{\sigma_1 \geq ... \geq \sigma_n\}$$
                and suppose that $\calF \in \Ob(\Rep_k^{\red}(\Gamma))$ (respectively, $\calF \in \Ob(\Rep_k^{\red}(\sigma_n ... \sigma_1 \Gamma))$) is an indecomposable representation of $\Gamma$ (respectively, $\sigma_n ... \sigma_1 \Gamma$). Then, both:
                    $$\Cox_{\Gamma, \sigma_1 \geq ...\geq \sigma_n}^{(n)}(\calF)$$
                and:
                    $$\co\Cox_{\Gamma, \sigma_1 \geq ...\geq \sigma_n}^{(n)}(\calF)$$
                will be regular indecomposable representations as well, of $\sigma_n ... \sigma_1 \Gamma$ and $\Gamma$ respectively, if and only if $\calF$ is regular.
            \end{lemma}
                \begin{proof}
                    By remark \ref{remark: coxeter_functors_act_as_zero_on_irreducible_representations} and definitions \ref{def: preprojective_and_preinjective_representations} and \ref{def: preprojective_regular_and_preinjective_components}, we see that if $\calF \in \Ob(\Rep_k^{\red, \not \reg}(\Gamma))$ (respectively, $\calF \in \Ob(\Rep_k^{\red, \not \reg}(\sigma_n ... \sigma_1 \Gamma))$) is an irregular indecomposable representation of $\Gamma$ (respectively, $\sigma_n ... \sigma_1 \Gamma$) then:
                        $$\co\Cox_{\Gamma, \sigma_1 \geq ...\geq \sigma_n}^{(n)}(\calF) \cong 0$$
                    and:
                        $$\Cox_{\Gamma, \sigma_1 \geq ...\geq \sigma_n}^{(n)}(\calF) \cong 0$$
                    as representations of $\sigma_n ... \sigma_1 \Gamma$ and $\Gamma$ respectively, which tells us that - by contraposition - that if:
                        $$\co\Cox_{\Gamma, \sigma_1 \geq ...\geq \sigma_n}^{(n)}(\calF) \not \cong 0$$
                    and:
                        $$\Cox_{\Gamma, \sigma_1 \geq ...\geq \sigma_n}^{(n)}(\calF) \not \cong 0$$
                    as representations of $\sigma_n ... \sigma_1 \Gamma$ and $\Gamma$ then $\calF$ will be necessarily regular as a representation of $\Gamma$ and $\sigma_n ... \sigma_1 \Gamma$ respectively; note that $\calF$ must also be indecomposable by corollary \ref{coro: coxeter_functors_preserve_indecomposability}.
                    
                    Conversely, suppose that $\calF$ is regular. Observe first of all that this implies that $\calF \not \cong 0$. Now, suppose for the sake of deriving a contradiction that either:
                        $$\co\Cox_{\Gamma, \sigma_1 \geq ...\geq \sigma_n}^{(n)}(\calF)$$
                    or:
                        $$\Cox_{\Gamma, \sigma_1 \geq ...\geq \sigma_n}^{(n)}(\calF)$$
                    are irregular as representations of $\sigma_n ... \sigma_1 \Gamma$ and $\Gamma$ respectively; the two cases are formally dual by definition, so let us work with the former, and furthermore, because one can only form the composition $\Cox_{\Gamma, \sigma_1 \geq ...\geq \sigma_n}^{(n)} \circ \co\Cox_{\Gamma, \sigma_1 \geq ...\geq \sigma_n}^{(r)}$ (for any $1 \leq r \leq n$), $\co\Cox_{\Gamma, \sigma_1 \geq ...\geq \sigma_n}^{(n)}(\calF)$ being irregular necessitates it being $r$-preprojective for some $1 \leq r \leq n$, i.e.:
                        $$\co\Cox_{\Gamma, \sigma_1 \geq ...\geq \sigma_n}^{(n)}(\calF) \cong \co\Cox_{\Gamma, \sigma_1 \geq ...\geq \sigma_n}^{(r)}(\Delta_{\Gamma, \sigma_i})$$
                    for some $1 \leq i, r \leq n$. This implies that:
                        $$\calF \cong (\Cox_{\Gamma, \sigma_1 \geq ...\geq \sigma_n}^{(n)} \circ \co\Cox_{\Gamma, \sigma_1 \geq ...\geq \sigma_n}^{(n)})(\calF) \cong (\Cox_{\Gamma, \sigma_1 \geq ...\geq \sigma_n}^{(n)} \circ \co\Cox_{\Gamma, \sigma_1 \geq ...\geq \sigma_n}^{(r)})(\Delta_{\Gamma, \sigma_i}) \cong 0$$
                    since $r \leq n$, but because we have established that $\calF \not \cong 0$, this is a contradiction. Therefore, $\co\Cox_{\Gamma, \sigma_1 \geq ...\geq \sigma_n}^{(n)}(\calF)$ must be regular. 
                \end{proof}
            \begin{theorem}[Coxeter functors fix regular indecomposable representations] \label{theorem: coxeter_functors_fix_regular_indecomposable_representations}
                Let $\Gamma := (\Gamma_1, \Gamma_0, s, t)$ be a connected and acyclic finite quiver with an admissible enumeration of sinks:
                    $$\Gamma_0 := \{\sigma_1 \geq ... \geq \sigma_n\}$$
                Then, there is an adjoint equivalence between the categories of regular indecomposable $k$-linear representations of $\sigma_n ... \sigma_1 \Gamma$ and $\Gamma$ given by the (co)Coxeter functors:
                    $$
                        \begin{tikzcd}
                        	{\Rep_k^{\red, \reg}(\sigma_n ... \sigma_1\Gamma)} & {\Rep_k^{\red, \reg}(\Gamma)}
                        	\arrow[""{name=0, anchor=center, inner sep=0}, "{\Cox_{\Gamma, \sigma_1 \geq ...\geq \sigma_n}^{(n)}}"', bend right, from=1-1, to=1-2]
                        	\arrow[""{name=1, anchor=center, inner sep=0}, "{\co\Cox_{\Gamma, \sigma_1 \geq ...\geq \sigma_n}^{(n)}}"', bend right, from=1-2, to=1-1]
                        	\arrow["\dashv"{anchor=center, rotate=-90}, draw=none, from=1, to=0]
                        \end{tikzcd}
                    $$
            \end{theorem}
                \begin{proof}
                    Combine lemma \ref{lemma: coxeter_functors_are_adjoints} with corollary \ref{coro: coxeter_functors_preserve_indecomposability} and lemma \ref{lemma: coxeter_functors_preserve_regularity}.
                \end{proof}
            
            Before we move on, let us perform several sanity checks on the quiver $\sfA_2$, the simplest possible non-trivial example of a connected and acyclic finite quiver. 
            \begin{example}[Indecomposable representations of $\sfA_2$] \label{example: indecomposable_representations_of_A2_quiver}
                Consider the quiver $\sfA_2$, which is:
                    $$
                        \begin{tikzcd}
                        	{\bullet_1} & {\bullet_2}
                        	\arrow["f", from=1-1, to=1-2]
                        \end{tikzcd}
                    $$
                (cf. figure \ref{fig: A_2_quiver}). Including the zero representation, its set of isomorphism classes of indecomposable representations can be easily seen to be:
                    $$[\Rep_k^{\red}(\sfA_2)] \cong \left\{(0 \to 0), (k \to 0), (0 \to k), (\id_k: k \to k)\right\}$$
                Consider, then, the sink $\bullet_2$: by applying the functor $\Refl_{\sfA_2, \bullet_2}$ to the indecomposable representations above, one shall get the following set of representation of the quiver:
                    $$
                        \sfA_2^{\op} :=
                        \{
                            \begin{tikzcd}
                            	{\bullet_1} & {\bullet_2}
                            	\arrow["{f^{\op}}"', from=1-2, to=1-1]
                            \end{tikzcd}
                        \}
                    $$
                and note that these representations are also indecomposable by lemma \ref{lemma: (co)reflection_functors_preserve_indecomposability}:
                    $$\left\{(0 \leftarrow 0), (k \leftarrow 0), (0 \leftarrow 0), \left(k \leftarrow \ker\left(\id_k: k \to k\right)\right)\right\} = \left\{(0 \leftarrow 0), (k \leftarrow 0)\right\}$$
                Indeed, these reflected representations are all indecomposable, so we have confirmed that lemma \ref{lemma: (co)reflection_functors_preserve_indecomposability} holds for $\sfA_2$. At the same time, note that:
                    $$\Refl_{\sfA_2, \bullet_2}(0 \to k) \cong (0 \leftarrow 0)$$
                which helps us confirm remark \ref{remark: coxeter_functors_act_as_zero_on_irreducible_representations}; we can similarly verify that this is the case for $\co\Refl_{\sfA_2^{\op}, \bullet_1}$. From the computations above, we see that the only non-irreducible representation of $\sfA_2$ is preprojective, so this quiver has no regular indecomposable representation aside from the zero representation; theorem \ref{theorem: coxeter_functors_fix_regular_indecomposable_representations} is therefore trivially true for $\sfA_2$. 
            \end{example}
            
        \subsubsection{Weyl groups; Dynkin quivers are representation-finite}
            We have now come to the first half of Gabriel's Theorem. The broad idea here is that somehow, the isomorphism classes of indecomposable representations of a Dynkin quiver are bijectively labelled - via dimension vectors - by the positive roots of said Dynkin quiver, a phenomenon which ought to be thought of as a generalisation - via lemma \ref{lemma: projective_indecomposable_modules_over_artinian_algebras_are_simple} - of proposition \ref{prop: irreducible_representations_of_dynkin_quivers_are_labelled_by_simple_roots}, which tells us that there is a bijective correspondence between the isomorphism classes of irreducible representations of a Dynkin quiver and its simple roots via taking dimension vectors. 
            
            Let us take a slight detour and discuss a more abstract notion of so-called \say{root systems} (cf. definition \ref{def: root_systems}), which encompass that of roots of Dynkin quivers as described in definition \ref{def: negative_and_positive_roots}. The point that we would like to establish here is that these entities are highly symmetric, and that in the special case of roots of Dynkin quivers, these symmetries are entirely encoded within the construction of (co)reflection functors: in fact, the decategorifications of (co)reflection functors (as well as (co)Coxeter functors) down to the level of Grothendieck groups, e.g. in the sense of corollary \ref{coro: coxeter_functors_and_grothendieck_groups}, are precisely elements of symmetry groups of the so-called \say{root systems} of Dynkin quivers (cf. example \ref{example: root_systems_of_dynkin_quivers}) after taking dimension vectors. More specifically, we shall see that the individual (co)reflection functors as in definition \ref{def: (co)reflection_functors} correspond via dimension vectors to so-called \say{simple reflections} (cf. definition \ref{def: simple_reflections}), which generate the symmetry groups of roots of Dynkin quivers (cf. definition \ref{def: weyl_groups}). 
            \begin{definition}[Euclidean spaces] \label{defL euclidean_spaces}
                A \textbf{Euclidean space} is a finite-dimensional real inner product space. Given any Euclidean space $(E, \<-, -\>)$, a(n) \textbf{(integral) Euclidean lattice} therein is then a $\Z$-submodule $E_0 \subseteq E$ such that $E \cong E_0 \tensor_{\Z} \R$ and for all $\alpha, \beta \in E_0$, one has $\<\alpha, \beta\> \in E_0$.
            \end{definition}
            \begin{definition}[Simple reflections] \label{def: simple_reflections}
                Let $(E, \<-, -\>)$ be a Euclidean space and $\alpha \in E$ be some fixed vector. The \textbf{simple reflection} about $\alpha$ is then given by:
                    $$s_{\alpha}: E \to E$$
                    $$\beta \mapsto \beta - 2\frac{\<\alpha, \beta\>}{\<\alpha, \alpha\>} \alpha$$
            \end{definition}
            \begin{remark}[Simple reflections are involutive isometries] \label{remark: simple_reflections_are_involutive_isometries}
                Let $(E, \<-, -\>)$ be a Euclidean space. Then, for all vectors $\alpha \in E$, it is not hard to check that the corresponding simple reflection $s_{\alpha}$, firstly, is an $\R$-linear map that preserves the inner product $\<-, -\>$, i.e.:
                    $$\<s_{\alpha}(-), s_{\alpha}(-)\> = \<-, -\>$$
                and secondly, that:
                    $$s_{\alpha}^2 = \id_E$$
            \end{remark}
            \begin{remark}[Integral simple reflections] \label{remark: integral_simple_reflections}
                Let $(E, \<-, -\>)$ be a Euclidean space and $E_0 \subseteq E$ be an integral lattice therein. Then, note that every simple reflection $s_{\alpha}$ corresponding to some vector $\alpha \in E_0$ is actually $E_0$-valued on $E_0$, i.e. there is an induced Euclidean lattice homomorphism:
                    $$s_{\alpha}|_{E_0}: E_0 \to E_0$$
                    $$\beta \mapsto \beta - 2\frac{\<\alpha, \beta\>}{\<\alpha, \alpha\>} \alpha$$
                It preserves the inner product $\<-, -\>|_{E_0}$, i.e.:
                    $$\<s_{\alpha}(-), s_{\alpha}(-)\> = \<-, -\>$$
                and also, similar to $s_{\alpha}$, it is involutive, i.e.:
                    $$s_{\alpha}|_{E_0}^2 = \id_{E_0}$$
            \end{remark}
            \begin{definition}[Root systems] \label{def: root_systems}
                A(n) \textbf{(integral) root system} inside a Euclidean space $(E, \<-, -\>)$ is a subset $\Phi \subset E$ satisfying the following properties:
                    \begin{itemize}
                        \item $E \cong \span_{\R} \Phi$ (though we do not require that $\Phi$ is $\R$-linearly independent\footnote{... and in fact this can not be the case!}),
                        \item $\Phi$ is closed under simple reflections, and
                        \item \footnote{This condition is usually known as \textbf{integrality}.}for all $\alpha, \beta \in \Phi$, we require that $2\frac{\<\alpha, \beta\>}{\<\alpha, \alpha\>} \in \Z$.
                    \end{itemize}
            \end{definition}
            \begin{remark}[Root lattices] \label{remark: root_lattices}
                Let $(E, \<-, -\>)$ be a Euclidean space and suppose that $\Phi \subset E$ is a root system therein. Then, given any Euclidean lattice $E_0 \subseteq E$, one sees - using integrality of root systems - that $\Phi$ can also be viewed as a so-called \textbf{(integral) root lattice} inside $E_0$, in the sense that:
                    \begin{itemize}
                        \item $E_0 \cong \span_{\Z} \Phi$,
                        \item $\Phi$ is closed under simple reflections about its elements, and
                        \item for all $\alpha, \beta \in \Phi$, we require that $2\frac{\<\alpha, \beta\>}{\<\alpha, \alpha\>} \in \Z$.
                    \end{itemize}
                Observe also, that any integral root lattice $\Phi_0 \subset E_0$ inside some Euclidean lattice $E_0 \subset E$ can be regarded as a root system $\Phi \subset E$ via:
                    $$\Phi := \Phi_0 \tensor_{\Z} \R$$
                (wherein the notation $\Phi_0 \tensor_{\Z} \R$ simply means that we are regarding the integral vectors $\alpha \in \Phi_0$ as real vector).
            \end{remark}
            \begin{example}[Root systems of Dynkin quivers] \label{example: root_systems_of_dynkin_quivers}
                Let $\Gamma := (\Gamma_1, \Gamma_0, s, t)$ be a Dynkin quiver and consider its set of roots $\Phi_{\Gamma}$ (cf. definition \ref{def: roots_of_dynkin_quivers}). By viewing $\Phi_{\Gamma} \tensor_{\Z} \R$ as a subset of $\R^{\oplus \Gamma_0} \cong \Z^{\oplus \Gamma_0} \tensor_{\Z} \R$ and by equipping the latter with the inner product $\R \tensor_{\Z} B_{\Gamma}(-, -) \tensor_{\Z} \R$ (which we shall abusively shorten down to just $B_{\Gamma}(-, -)$) induced from the Cartan quadratic form $B_{\Gamma}(-, -)$ as given in definition \ref{def: cartan_quadratic_forms_of_finite_quivers}, one can then check that $\Phi_{\Gamma} \tensor_{\Z} \R$ is in fact a root system inside the Euclidean space $(\R^{\oplus \Gamma_0}, B_{\Gamma}(-, -))$. Indeed, we know that:
                    \begin{itemize}
                        \item by remark \ref{remark: number_of_positive_roots_of_dynkin_quivers}, we know that $\R^{\oplus \Gamma_0} \cong \span_{\R} (\Phi_{\Gamma} \tensor_{\Z} \R)$;
                        \item for all $\alpha, \beta \in \Phi_{\Gamma} \tensor_{\Z} \R$, we have $B_{\Gamma}(\alpha, \alpha) = 2$ (cf. definitions \ref{def: roots_of_dynkin_quivers} and \ref{def: tits_quadratic_forms}), so:
                            $$s_{\alpha}(\beta) = \beta - 2\frac{B_{\Gamma}(\alpha, \beta)}{B_{\Gamma}(\alpha, \alpha)} = \beta - B_{\Gamma}(\alpha, \beta)$$
                        and from here, it is not hard to check that for all $\beta \in \Phi_{\Gamma} \tensor_{\Z} \R$, we have:
                            $$B_{\Gamma}(s_{\alpha}(\beta), s_{\alpha}(\beta)) = B_{\Gamma}(\beta, \beta) = 2$$
                        and hence $s_{\alpha}(\beta) \in \Phi_{\Gamma} \tensor_{\Z} \R$ according to definitions \ref{def: roots_of_dynkin_quivers} and \ref{def: tits_quadratic_forms} (i.e. $\Phi_{\Gamma} \tensor_{\Z} \R$ is closed under simple reflections);
                        \item for all $\alpha, \beta \in \Phi_{\Gamma} \tensor_{\Z} \R$, we know that $B_{\Gamma}(\alpha, \alpha) \in 2\Z$ (cf. proposition \ref{prop: evenness_of_cartan_forms_of_finite_connected_acyclic_quivers}) and that $B_{\Gamma}(\alpha, \beta) \in \Z$, so clearly $2\frac{B_{\Gamma}(\alpha, \beta)}{B_{\Gamma}(\alpha, \alpha)} \in \Z$. 
                    \end{itemize}
                From this, it is easy to see that $\Phi_{\Gamma} \subset \Z^{\oplus \Gamma_0}$ is a root lattice with respect to the integral inner product $B_{\Gamma}(-, -)$.
            \end{example}
            \begin{lemma}[Simple reflections are invertible] \label{lemma: simple_reflections_about_roots_are_invertible}
                Let $(E, \<-, -\>)$ be a Euclidean space. Then, for all $\alpha \in E$, the $\R$-linear operator $s_{\alpha}$ is invertible; in fact, it is also self-adjoint and therefore unitary.
            \end{lemma}
                \begin{proof}
                    Clear form the fact that simple reflections $s_{\alpha}$ about vectors $\alpha \in E$ are involutive isometries (cf. remark \ref{remark: simple_reflections_are_involutive_isometries}).
                \end{proof}
            \begin{definition}[Weyl groups] \label{def: weyl_groups}
                Let $(E, \<-, -\>, \Phi)$ be a Euclidean space equipped with a root system. The \textbf{Weyl group} attached to the given root system $\Phi$, denoted by $\rmW_{\Phi}$, is then the subgroup of the special orthogonal group $\SO(E, \<-, -\>)$ generated by the simple reflections $s_{\alpha}$, for all $\alpha \in \Phi$.
                
                Note that Weyl groups attached to root systems are well-defined, since simple reflections are invertible operators (cf. lemma \ref{lemma: simple_reflections_about_roots_are_invertible}).
            \end{definition}
            \begin{convention}[Weyl groups of Dynkin quivers]
                If $\Gamma$ is a Dynkin quiver then we shall write $\rmW_{\Gamma}$ for the Weyl group attached to the root lattice $\Phi_{\Gamma}$ (or equivalently, the root system $\Phi_{\Gamma} \tensor_{\Z} \R$). 
            \end{convention}
            \begin{remark}[Actions of Weyl groups on Grothendieck groups of Dynkin quivers] \label{remark: actions_of_weyl_groups_on_grothendieck_groups_of_dynkin_quivers}
                Since root systems are closed under simple reflections about its elements by definition, one also could abstractly characterise the Weyl group of a root system $\Phi$ inside some Euclidean space $(E, \<-, -\>)$ as nothing but the symmetric group on the (\textit{a priori} finite) set $\Phi$, i.e.:
                    $$\rmW_{\Phi} \cong \Aut(\Phi)$$
                    
                This suggests something interesting: the Weyl group $\rmW_{\Phi}$ naturally acts via left-multiplication (i.e. permutations, since simple reflections map roots to roots by definition) on the given root system $\Phi$. In particular, if $\Phi := \Phi_{\Gamma}$ is a root system of a Dynkin quiver $\Gamma := (\Gamma_1, \Gamma_0, s, t)$ (inside the Euclidean space $(\R^{\oplus \Gamma_0}, B_{\Gamma}(-, -))$, that is), then as a result of there being a $\Z$-module isomorphism:
                    $$\dim_k: \K_0(\Gamma) \to \span_{\Z} \Phi_{\Gamma}$$
                (combine remark \ref{remark: number_of_positive_roots_of_dynkin_quivers} and proposition \ref{prop: irreducible_representations_of_dynkin_quivers_are_labelled_by_simple_roots}) and the fact that simple reflections of roots are also roots (since simple reflections are isometries and roots are vectors of norm $\sqrt{2}$ with respect to the Cartan form $B_{\Gamma}(-, -)$), we see that the left-multiplication $\rmW_{\Gamma}$-action on $\Phi_{\Gamma}$ yields us an action of $\rmW_{\Gamma}$ via left-multiplications on the simple Grothendieck group $\K_0(\Gamma)$.
                
                The natural existence of such an $\rmW_{\Gamma}$-action on $\K_0(\Gamma)$ - which is known to admit $\Phi_{\Gamma}^{\simple}$ as a basis - implies that somehow, one could possibly obtain positive roots of $\Gamma$ from its simple roots via successive applications of (possibly different) simple reflections across hyperplanes defined by other \textit{simple} roots. That is, given an \say{initial} simple root $\alpha_0$ and a possibly \textit{non-simple} positive root $\alpha$, there might exist \textit{distinct} simple roots $\alpha_1, ..., \alpha_i$ such that:
                    $$\alpha = (s_{\alpha_i} \circ ... \circ s_{\alpha_1})(\alpha_0)$$
                What we shall see later on in lemma \ref{lemma: non_simple_positive_roots_from_simple_roots_via_simple_reflections} is that, this turns out to be true. At the same time, the simple reflections $s_{\alpha_j}$ ($1 \leq j \leq i$) turn out to correspond to the (co)reflection functors at the vertices $\sigma_j$ corresponding to the simple roots $\alpha_j$ (cf. proposition \ref{prop: (co)reflection_functors_as_simple_reflections}) via taking dimension vectors. Then, because $\alpha_0$ corresponds to some (isomorphism class of) irreducible representation via proposition \ref{prop: irreducible_representations_of_dynkin_quivers_are_labelled_by_simple_roots}, one is able to obtain non-irreducible indecomposable representations from irreducible ones, using the speculation above about the possibility of obtaining non-simple positive roots from simple ones.
            \end{remark}
            
            \begin{proposition}[(Co)reflection functors as simple reflections] \label{prop: (co)reflection_functors_as_simple_reflections}
                Let $\Gamma := (\Gamma_1, \Gamma_0, s, t)$ be a Dynkin quiver with a fixed vertex $\sigma$ and denote the simple root corresponding to that vertex (via proposition \ref{prop: irreducible_representations_of_dynkin_quivers_are_labelled_by_simple_roots}) by $\alpha_{\sigma}$. Then, for all indecomposable representations $\calF \in \Ob(\Rep_k^{\red}(\Gamma))$ such that:
                    $$\dim_k \calF \not = \alpha_{\sigma}$$
                we then have:
                    $$\dim_k \Refl_{\Gamma, \sigma}(\calF) = \dim_k \co\Refl_{\Gamma, \sigma}(\calF) = s_{\alpha_{\sigma}}(\dim_k \calF)$$
            \end{proposition}
                \begin{proof}
                    Using lemma \ref{lemma: (co)reflection_functors_preserve_indecomposability}, and relying particularly on the fact that:
                        $$\calF(\sigma) \cong \coker\left(\ker: \Refl_{\Gamma, \sigma}(\calF)(\sigma) \to \bigoplus_{v \to \sigma, v \not = \sigma} \calF(v)\right)$$
                    i.e. there is a short exact sequence of finite-dimensional $k$-vector spaces:
                        $$0 \to \Refl_{\Gamma, \sigma}(\calF)(\sigma) \to \bigoplus_{v \to \sigma, v \not = \sigma} \calF(v) \to \calF(\sigma) \to 0$$
                    whenever $\calF \in \Ob(\Rep_k^{\fin}(\sigma\Gamma))$ is indecomposable and such that $\dim_k \calF \not = \alpha_{\sigma}$, we see that:
                        $$\dim_k \Refl_{\sigma\Gamma}(\calF)(\sigma) = \sum_{v \to \sigma, v \not = \sigma} \dim_k \calF(v) - \dim_k \calF(\sigma)$$
                    One can then simply apply the definition of simple reflections (cf. definition \ref{def: simple_reflections}) directly in order to see that:
                        $$s_{\alpha_{\sigma}}(\dim_k \calF)_{\sigma} = \sum_{v \to \sigma, v \not = \sigma} \dim_k \calF(v) - \dim_k \calF(\sigma) = \dim_k \Refl_{\sigma\Gamma}(\calF)(\sigma)$$
                    wherein the subscript $\sigma$ in the expression on the left-hand side indicates the $\sigma^{th}$ entry of the vector. 
                    
                    One can show using a completely analogous method that:
                        $$s_{\alpha_{\sigma}}(\dim_k \calF)_{\sigma} = \dim_k \co\Refl_{\Gamma}(\calF)(\sigma)$$
                    for all indecomposable representations $\calF \in \Ob(\Rep_k^{\red}(\Gamma))$ such that $\dim_k \calF \not = \alpha_{\sigma}$.
                \end{proof}
                
            \begin{lemma}[Non-simple positive roots from simple roots via simple reflections] \label{lemma: non_simple_positive_roots_from_simple_roots_via_simple_reflections}
                Let $\Gamma := (\Gamma_1, \Gamma_0, s, t)$ be a Dynkin quiver with an admissible enumeration of sinks:
                    $$\Gamma_0 := \{\sigma_1 \geq ... \geq \sigma_n\}$$
                Suppose that $\alpha_i \in \Phi_{\Gamma}^{\simple}$ is an arbitrary simple root of $\Gamma$. Then the element:
                    $$\alpha := (s_{\alpha_{i - 1}}^{-1} \circ ... \circ s_{\alpha_1}^{-1})(\alpha_i) \in \Z^{\oplus \Gamma_0}$$
                shall be a non-simple positive root. 
            \end{lemma}    
                \begin{proof}
                    Compute the element $\alpha := (s_{\alpha_{i - 1}}^{-1} \circ ... \circ s_{\alpha_1}^{-1})(\alpha_i$ explicitly using definition \ref{def: simple_reflections} to show that it is an element of $\N^{\Gamma_0}$ and then check that $q_{\Gamma}(\alpha) = 1$ to verify that it is a root. 
                \end{proof}
            \begin{corollary}[Preprojective representations of Dynkin quivers are indecomposable] \label{coro: preprojective_representations_of_dynkin_quivers_are_indecomposable}
                Let $\Gamma := (\Gamma_1, \Gamma_0, s, t)$ be a Dynkin quiver with an admissible enumeration of sinks:
                    $$\Gamma_0 := \{\sigma_1 \geq ... \geq \sigma_n\}$$
                Then, for any fixed vertex $\sigma_i \in \Gamma_0$, any $(i - 1)$-preprojective representation is not only indecomposable but also non-irreducible. 
            \end{corollary}
                \begin{proof}
                    Firstly use \ref{prop: irreducible_representations_of_dynkin_quivers_are_labelled_by_simple_roots} to see that:
                        $$\dim_k \Delta_{\Gamma, \sigma_i} = \alpha_i$$
                    for all $1 \leq i \leq n$. Then, combine proposition \ref{prop: (co)reflection_functors_as_simple_reflections} with corollary \ref{coro: coxeter_functors_and_grothendieck_groups} to see that a non-simple positive root can not correspond to an irreducible representation, and hence any $(i - 1)$-preprojective representation is not only indecomposable but also non-irreducible. 
                \end{proof}
            \begin{remark}
                Weyl groups are \textit{a priori} non-abelian, so the non-simple positive root:
                    $$\alpha := (s_{\alpha_{i - 1}}^{-1} \circ ... \circ s_{\alpha_1}^{-1})(\alpha_i)$$
                that one obtains in lemma \ref{lemma: non_simple_positive_roots_from_simple_roots_via_simple_reflections} depends on the previously chosen admissible enumeration of sinks:
                    $$\Gamma_0 := \{\sigma_1 \geq ... \geq \sigma_n\}$$
                However, (co)Coxeter functors are - on the other hand - independent of our choice of admissible enumearation of sinks (cf. proposition \ref{prop: coxeter_functors_are_independent_of_admissible_enumerations}). Because we have a bijective corresondence:
                    $$\dim_k: [\Rep_k^{\irr}(\sigma_{i - 1} ... \sigma_1\Gamma)] \to \Phi_{\sigma_{i - 1} ... \sigma_1\Gamma}^{\simple}$$
                between the isomorphism classes of irreucible representations of $\sigma_{i - 1} ... \sigma_1\Gamma$ and the simple roots thereof (although, note that $\Phi_{\Gamma} = \Phi_{\sigma_{i - 1} ... \sigma_1 \Gamma}$ as the Tits quadratic form is independent of the orientation of the underlying quiver), when given any simple root $\alpha_i \in \Phi_{\sigma_{i - 1} ... \sigma_1\Gamma}^{\simple}$, one can \textit{choose} a representative:
                    $$\calF_i \in \dim_k^{-1}(\alpha_i)$$
                (which must be isomorphic to the Kronecker representation $\Delta_{\sigma_{i - 1} ... \sigma_1\Gamma, \sigma_i}$ by lemma \ref{lemma: schur_lemma_for_abelian_categories}) and obtain a preprojective representation:
                    $$\calF := \co\Cox^{(i - 1)}_{\Gamma, \sigma_1 \geq ... \geq \sigma_n}(\calF_i)$$
                which is necessarily indecomposable and non-irreducible, and independent of the choice of admissible enumeration of sinks $\Gamma_0 := \{\sigma_1 \geq ... \geq \sigma_n\}$.
            \end{remark}
            \begin{theorem}[Indecomposable representations of Dynkin quivers are labelled by positive roots] \label{theorem: indecomposable_representations_of_dynkin_quivers_are_labelled_by_positive_roots}
                Let $k$ be a field and $\Gamma$ be a Dynkin quiver. Then there will be a bijection:
                    $$\dim_k: [\Rep_k^{\red}(\Gamma)] \to \Phi_{\Gamma}^+$$
                between the set of isomorphism classes of indecomposable $k$-linear representations of $\Gamma$ and that of positive roots of $\Gamma$.
            \end{theorem}
                \begin{proof}
                    Since $\Rep_k^{\fin}(\Gamma)$ is a finite $k$-linear abelian category, all of its objects are of finite lengths (cf. remark \ref{remark: locally_finite_linear_categories_are_jordan_holder_and_krull_schmidt}). As such, fix an indecomposable representation $\calF \in \Ob(\Rep_k^{\red}(\Gamma))$ of length $n$, along with a choice of a Jordan-H\"older filtration thereon:
                        $$0 =: \calF_0 \subseteq \calF_1 \subseteq ... \subseteq \calF_n \subseteq \calF$$
                    From the definition of Grothendieck groups (cf. definition \ref{def: simple_grothendieck_groups}), we then obtain the following expression from the filtration above:
                        $$[\calF] = \sum_{1 \leq i \leq n} [\calF_i/\calF_{i - 1}]$$
                    which we note to be well-defined thanks to the fact that Jordan-H\"older filtrations are unique up to multiplicities and permutations of their factors according to the Jordan-H\"older Theorem (cf. theorem \ref{theorem: jordan_holder_theorem}). By applying the $\Z$-module isomorphism (cf. lemma \ref{lemma: irreducible_representations_of_finite_quivers_are_labelled_by_vertices}):
                        $$\dim_k: \Rep_k^{\fin}(\Gamma) \to \Z^{\oplus \Gamma_0}$$
                    to the equation above, one then obtains:
                        $$\dim_k [\calF] = \sum_{1 \leq i \leq n} \dim_k [\calF_i/\calF_{i - 1}]$$
                    Since the factors $\calF_i/\calF_{i - 1}$ (for all $1 \leq i \leq n$) are simple by definition (cf. definition \ref{def: lengths_of_objects_and_jordan_holder_series}), we then see via proposition \ref{prop: irreducible_representations_of_dynkin_quivers_are_labelled_by_simple_roots} that:
                        $$\dim_k [\calF_i/\calF_{i - 1}] \in \Phi_{\Gamma}^{\simple}$$
                    for all $1 \leq i \leq n$. As a consequence of this, one has:
                        $$\dim_k [\calF] \in \Phi_{\Gamma}^+$$
                    which tells us that $\dim_k: [\Rep_k^{\red}(\Gamma)] \to \Phi_{\Gamma}^+$ is injective.
                    
                    To show that $\dim_k: [\Rep_k^{\red}(\Gamma)] \to \Phi_{\Gamma}^+$ is surjective, firstly pick an arbitrary positive root $\alpha \in \Phi_{\Gamma}^+$. Then, using the fact that simple reflections are invertible operators (cf. lemma \ref{lemma: simple_reflections_about_roots_are_invertible}) in conjunction with lemma \ref{lemma: non_simple_positive_roots_from_simple_roots_via_simple_reflections}, followed by corollary \ref{coro: preprojective_representations_of_dynkin_quivers_are_indecomposable}, one sees that there exists a (non-irreducible) indecomposable representation $\calF \in \Ob(\Rep_k^{\red}(\Gamma))$ such that:
                        $$\dim_k \calF = \alpha$$
                \end{proof}
            \begin{corollary}[Dynkin quivers are representation-finite] \label{coro: dynkin_quivers_are_representation_finite}
                For any Dynkin quiver $\Gamma$, the set $[\Rep_k^{\red}(\Gamma)]$ of isomorphism classes of indecomposable $k$-linear representations of $\Gamma$ is finite. 
            \end{corollary}
                \begin{proof}
                    Appply theorem \ref{theorem: indecomposable_representations_of_dynkin_quivers_are_labelled_by_positive_roots} to the fact that $\Phi_{\Gamma}^+$ is a finite set (cf. remark \ref{remark: number_of_positive_roots_of_dynkin_quivers}).
                \end{proof}
            
            \begin{example}[Gabriel's Theorem for $\sfA_n$ quivers]
                We have already seen in example \ref{example: indecomposable_representations_of_A2_quiver} that not only does the $\sfA_2$ quiver has finitely many indecomposable representations (three, namely $(k \to 0), (0 \to k), (\id_k: k \to k)$, not including the zero representation), but also, said indecomposable representations are indeed labelled by the positive roots of $\sfA_2$ (namely the vectors $(1, 0), (0, 1)$, and $(1, 1)$ in $\Z^{\oplus 2}$). It is clear that these facts also hold for $\sfA_2^{\op}$. 
                
                Now, consider the quiver $\sfA_3$: in this case, there are three possibilities as follows:
                    $$
                        \begin{tikzcd}
                        	{\bullet_1} & {\bullet_2} & {\bullet_3}
                        	\arrow["f", from=1-1, to=1-2]
                        	\arrow["g", from=1-2, to=1-3]
                        \end{tikzcd}
                    $$
                    $$
                        \begin{tikzcd}
                        	{\bullet_1} & {\bullet_2} & {\bullet_3}
                        	\arrow["f"', from=1-2, to=1-1]
                        	\arrow["g", from=1-2, to=1-3]
                        \end{tikzcd}
                    $$
                    $$
                        \begin{tikzcd}
                        	{\bullet_1} & {\bullet_2} & {\bullet_3}
                        	\arrow["f", from=1-1, to=1-2]
                        	\arrow["g"', from=1-3, to=1-2]
                        \end{tikzcd}
                    $$
                Before we start writing down the indecomposble representations, let us note that because the Tits quadratic form of a Dynkin quiver is independent of its underlying orientation, all three of the quivers above ought to have the same number of indecomposable representations/positive roots. Let us start with the first quiver: its indecomposable representations, including the zero representation, along with the corresponding dimension vectors, are:
                    $$
                        \begin{tikzcd}
                        	0 & 0 & 0 & {(0, 0 ,0)} \\
                        	k & 0 & 0 & {(1, 0, 0)} \\
                        	0 & k & 0 & {(0, 1, 0)} \\
                        	0 & 0 & k & {(0, 0, 1)} \\
                        	k & k & 0 & {(1, 1, 0)} \\
                        	0 & k & k & {(0, 1, 1)} \\
                        	k & k & k & {(1, 1, 1)}
                        	\arrow[from=1-1, to=1-2]
                        	\arrow[from=1-2, to=1-3]
                        	\arrow["{\id_k}", from=7-1, to=7-2]
                        	\arrow["{\id_k}", from=7-2, to=7-3]
                        	\arrow[from=3-1, to=3-2]
                        	\arrow[from=3-2, to=3-3]
                        	\arrow[from=2-1, to=2-2]
                        	\arrow[from=2-2, to=2-3]
                        	\arrow[from=4-1, to=4-2]
                        	\arrow[from=4-2, to=4-3]
                        	\arrow["{\id_k}", from=5-1, to=5-2]
                        	\arrow[from=5-2, to=5-3]
                        	\arrow[from=6-1, to=6-2]
                        	\arrow["{\id_k}", from=6-2, to=6-3]
                        \end{tikzcd}
                    $$
                Note that $k \to 0 \to k$ is not an irreducible representation because:
                    $$(k \to 0 \to k) \cong (k \to 0 \to 0) \oplus (0 \to 0 \to k)$$
                Indeed, there are finitely many indecomposable representations, which are also all labelled by positive roots; to check that the dimension vectors $\alpha$ above are roots of $\sfA_3$, write down the Cartan matrix $A_{\sfA_3}$, which is:
                    $$
                        A_{\sfA_3} =
                        \begin{pmatrix}
                            2 & -1 & 0
                            \\
                            -1 & 2 & -1
                            \\
                            0 & -1 & 2
                        \end{pmatrix}
                    $$
                and check if:
                    $$\alpha^{\top} A_{\sfA_3} \alpha = 2$$
                The indecomposable representations of the second quiver are:
                    $$
                        \begin{tikzcd}
                        	0 & 0 & 0 & {(0, 0 ,0)} \\
                        	k & 0 & 0 & {(1, 0, 0)} \\
                        	0 & k & 0 & {(0 ,1, 0)} \\
                        	0 & 0 & k & {(0, 0, 1)} \\
                        	k & k & 0 & {(1, 1, 0)} \\
                        	0 & k & k & {(0, 1, 1)} \\
                        	k & k & k & {(1, 1, 1)}
                        	\arrow[from=1-1, to=1-2]
                        	\arrow[from=1-3, to=1-2]
                        	\arrow["{\id_k}", from=7-1, to=7-2]
                        	\arrow["{\id_k}"', from=7-3, to=7-2]
                        	\arrow["{\id_k}", from=5-1, to=5-2]
                        	\arrow[from=5-3, to=5-2]
                        	\arrow[from=6-1, to=6-2]
                        	\arrow["{\id_k}"', from=6-3, to=6-2]
                        	\arrow[from=2-3, to=2-2]
                        	\arrow[from=2-1, to=2-2]
                        	\arrow[from=4-1, to=4-2]
                        	\arrow[from=3-1, to=3-2]
                        	\arrow[from=4-3, to=4-2]
                        	\arrow[from=3-3, to=3-2]
                        \end{tikzcd}
                    $$
                and for the third, the indecomposable representations are:
                    $$
                        \begin{tikzcd}
                        	0 & 0 & 0 & {(0, 0 ,0)} \\
                        	k & 0 & 0 & {(1, 0, 0)} \\
                        	0 & k & 0 & {(0 ,1, 0)} \\
                        	0 & 0 & k & {(0, 0, 1)} \\
                        	k & k & 0 & {(1, 1, 0)} \\
                        	0 & k & k & {(0, 1, 1)} \\
                        	k & k & k & {(1, 1, 1)}
                        	\arrow[from=1-2, to=1-1]
                        	\arrow[from=1-2, to=1-3]
                        	\arrow["{\id_k}"', from=7-2, to=7-1]
                        	\arrow["{\id_k}", from=7-2, to=7-3]
                        	\arrow["{\id_k}"', from=5-2, to=5-1]
                        	\arrow[from=5-2, to=5-3]
                        	\arrow[from=6-2, to=6-1]
                        	\arrow["{\id_k}", from=6-2, to=6-3]
                        	\arrow[from=2-2, to=2-3]
                        	\arrow[from=2-2, to=2-1]
                        	\arrow[from=4-2, to=4-1]
                        	\arrow[from=3-2, to=3-1]
                        	\arrow[from=4-2, to=4-3]
                        	\arrow[from=3-2, to=3-3]
                        \end{tikzcd}
                    $$
                Like before, neither $(k \to 0 \leftarrow k)$ nor $(k \leftarrow 0 \to k)$ are indecomposable since we have the following direct sum decompositions:
                    $$(k \to 0 \leftarrow k) \cong (k \to 0 \leftarrow 0) \oplus (0 \to 0 \leftarrow k)$$
                    $$(k \leftarrow 0 \to k) \cong (k \leftarrow 0 \to 0) \oplus (0 \leftarrow 0 \to k)$$
            \end{example}
            \begin{example}[Gabriel's Theorem for $\sfD_n$ quivers]
                
            \end{example}
            \begin{example}[Gabriel's Theorem for $\sfE_n$ quivers]
                
            \end{example}
            \begin{example}[Gabriel's Theorem fails for the loop quiver]
                From example \ref{example: path_algebras_of_loop_quivers}, we know that the path $k$-algebra of the quiver with one non-trivial loop, i.e.:
                    $$
                        \begin{tikzcd}
                            \bullet \arrow["x"', loop, distance=2em, in=35, out=325]
                        \end{tikzcd}
                    $$
                is isomorphic to $k\<x\> \cong k[x]$. This $k$-algebra is not even Artinian (consider the infinite descending chain of ideals $(x) \supset (x^2) \supset ...$), so the representation category:
                    $$\Rep_k^{\fin}\left(\begin{tikzcd}\bullet \arrow["x"', loop, distance=2em, in=35, out=325]\end{tikzcd}\right)$$
                can not be expected to be \textit{a priori} finite as a $k$-linear abelian category (i.e. one would not even suspect that this quiver would have finitely many irreducible representations, let alone finitely many indecomposable representations). This is indeed the case, since the loop quiver fails to even be acyclic and hence can not be a Dynkin quiver.
            \end{example}
            \begin{example}[Gabriel's Theorem fails for the Kronecker quiver]
                
            \end{example}
        
        \subsubsection{Euler characteristics of quivers; representation-finite connected acyclic finite quivers are Dynkin}
            Let us now discuss Euler characteristics before proving the second part of Gabriel's Theorem, which states that a finite connected acyclic quiver is Dynkin if it has finitely many isomorphism classes of indecomposable representations (cf. theorem \ref{theorem: gabriel_theorem_representation_finite_implies_dynkin}). As general finite connected acyclic quivers \textit{a priori} do not have any kind of well-behaved associated set of roots (unlike the more specialised Dynkin quivers), we shall have these so-called \say{Euler characteristics} serve as a replacement for the notion of Tits quadratic form (cf. proposition \ref{prop: tits_quadratic_forms_as_euler_characteristics}).
            \begin{definition}[Compact objects] \label{def: compact_objects}
                An object $c \in \Ob(\C)$ of a locally small category $\C$ is said to be \textbf{compact} if and only if the representable copresheaf:
                    $$\C(c, -): \C \to \Sets$$
                preserves all small filtered colimits that exist in $\C$. The (\textit{a priori} full) subcategory of $\C$ spanned by such objects is denoted by $\C^{\comp}$ (or sometimes $\C^{\omega}$) and shall be known as the \textbf{maximal compact subcategory} of $\C$. 
            \end{definition}
            \begin{remark}[Compact objects in full subcategories] \label{remark: compact_objects_in_full_subcategories}
                Suppsoe that $\C$ is a category and $\C_0 \subseteq \C$ is a full subcategory thereof. Then, should $c \in \Ob(\C_0)$ be any object of $\C_0$ that is compact as an object of the larger ambient category $\C$, then it is clear from definition \ref{def: compact_objects} that it would also be compact as an object of $\C_0$; that is:
                    $$\C_0 \cap \C^{\comp} = \C_0^{\comp}$$
            \end{remark}
            \begin{example}[Finitely generated modules are compact] \label{example: finitely_generated_modules_are_compact}
                
            \end{example}
            \begin{convention}[Cohomological functors] \label{conv: cohomological_functors_dynkin_quiver_representations}
                Much more general definitions can be found throughout the literature, but for us, a so-called $k$-linear \textbf{cohomological functor} (with $k$ being some Artinian\footnote{So that $k\mod^{\fin} \cong k\mod^{\comp}$.} commutative ring) shall be a $k$-linear triangulated functor:
                    $$F: \rmD^b(\E)^{\op} \to \rmD^b(k\mod^{\fin})$$
                from the \textit{opposite} of the \textit{bounded} derived category $\rmD^b(\E)$ (equipped with some triangulation; note also that it is $k$-linear) of some finite $k$-linear abelian category $\E$ into the bounded derived category of finitely generated $k$-modules (equipped with the canonical triangulation).
            \end{convention}
            \begin{definition}[Euler characteristics] \label{def: euler_characteristics}
                Let $k$ be an Artinian commutative ring and $\E$ be a finite $k$-linear abelian category. Suppose also, that:
                    $$F: \rmD^b(\E)^{\op} \to \rmD^b(k\mod^{\fin})$$
                is a $k$-linear cohomological functor. Then, the $k$-linear \textbf{Euler characteristic}\footnote{This definition is well-defined since for any Artinian commutative ring $k$, one has:
                    $$\rmD^b(k\mod^{\fin}) \cong \Perf(k\mod^{\fin})$$} of objects $\calF^{\bullet} \in \Ob(\rmD^b(\E)^{\op})$ with respect to the given cohomological functor $F$ shall be given by:
                    $$\chi_{\E}(F(-)): \Ob(\rmD^b(\E)) \to \Z$$
                    $$\calF^{\bullet} \mapsto \sum_{i \in \Z} (-1)^i \rank_k H^i(F(\calF^{\bullet}))$$
            \end{definition}
            \begin{remark}[Euler characteristics as bilinear forms] \label{remark: euler_characteristics_as_bilinear_forms}
                Let $k$ be an Artinian commutative ring, $\E$ be a finite $k$-linear abelian category, and consider the representable cohomological functor:
                    $$\R\Hom_{\E}(-, N): \rmD^b(\E)^{\op} \to \rmD^b(k\mod^{\fin})$$
                for some object $N \in \Ob(\E)$ (note that the functor is well-defined since $\E^{\comp} = \E$). Then, observe that:
                    $$\chi_{\E}(\R\Hom_{\E}(M^{\bullet}, N)) := \sum_{i \in \Z} (-1)^i \rank_k \Ext^i_{\E}(M^{\bullet}, N)$$
                for all objects $M^{\bullet} \in \Ob(\rmD^b(\E))$. This tells us that the assignment of Euler characteristics to the \textit{representable} $k$-linear cohomological functors on $\rmD^b(\E)$ gives rise to a pairing:
                    $$\chi_{\E}(-, -): \Ob(\rmD^b(\E)^{\op}) \x \Ob(\rmD^b(\E)) \to \Z$$
                    $$(M^{\bullet}, N^{\bullet}) \mapsto \chi_{\E}(\R\Hom_{\E}(M^{\bullet}, N^{\bullet}[0]))$$
            \end{remark}
            \begin{example}[Euler characteristics of quiver representations] \label{example: euler_characteristics_of_quiver_representations}
                Let $Q$ be a finite quiver and $k$ be a field. The Euler characteristic of $Q$ shall then be defined on the representable $k$-linear cohomological functors on $\rmD^b(\Rep_k^{\fin}(Q))$, i.e. we have the following pairing:
                    $$\chi_Q(-, -): \Ob(\rmD^b(\Rep_k^{\fin}(Q))^{\op}) \x \Ob(\rmD^b(\Rep_k^{\fin}(Q))) \to \Z$$
                    $$(\calF'^{\bullet}, \calF^{\bullet}) \mapsto \chi_{\E}(\R\Hom_{\E}(\calF'^{\bullet}, \calF^{\bullet}[0])) := \sum_{i \in \Z} (-1)^i \dim_k \Ext^i_{k\<Q\>}(\calF'^{\bullet}, \calF^{\bullet}[0])$$
                
                Now, if $A$ is a hereditary $k$-algebra (e.g. $A \cong k\<Q\>$), then one has by definition that:
                    $$\forall i \in \Z \setminus \{0, 1\}: \Ext^i_A(M, N) \cong 0$$
                for all left/right-$A$-modules $M, N$, so in fact, one has:
                    $$\chi_Q(-, \calF^{\bullet}) = \dim_k \Hom_{k\<Q\>}(-, \calF^{\bullet}[0]) - \dim_k \Ext^1_{k\<Q\>}(-, \calF^{\bullet}[0])$$
                for all projective resolutions $\calF^{\bullet} \in \Ob(\rmD^b(\Rep_k^{\fin}(Q)))$. Since we will be working over a field for the remainder of this subsection, and since we are only interested in Dynkin quivers which are finite by definition, we might as well take as a \textit{definition} that the Euler characteristic of a finite quiver $Q$ is given by:
                    $$\chi_Q(-, -): \Ob(\rmD^b(\Rep_k^{\fin}(Q))^{\op}) \x \Ob(\rmD^b(\Rep_k^{\fin}(Q))) \to \Z$$
                    $$(\calF'^{\bullet}, \calF^{\bullet}) \mapsto \dim_k \Hom_{k\<Q\>}(\calF'^{\bullet}[0], \calF^{\bullet}[0]) - \dim_k \Ext^1_{k\<Q\>}(\calF'^{\bullet}, \calF^{\bullet}[0])$$
            \end{example}
            
            \begin{lemma}[Euler characteristics and Grothendieck groups] \label{lemma: euler_characteristics_and_grothendieck_groups}
                Let $k$ be an Artinian commutative ring and $\E$ be a finite $k$-linear abelian category. Suppose also, that:
                    $$F: \rmD^b(\E)^{\op} \to \rmD^b(k\mod^{\fin})$$
                is a $k$-linear cohomological functor. Then, the Euler characteristic with respect to $F$ as in definition \ref{def: euler_characteristics} gives rise to a $\Z$-linear map:
                    $$\chi_{\E}(F(-)): \K_0(\E) \to \Z$$
                    $$[\calF] \mapsto \chi_{\E}(F(\calF^{\bullet}))$$
                for some choice of projective resolution $\calF^{\bullet}$ of $\calF$.
            \end{lemma}
                \begin{proof}
                    Straightforward from the fact that the cohomological functor $F: \rmD^b(\E)^{\op} \to \rmD^b(k\mod^{\fin})$ is triangulated by definition (cf. convention \ref{conv: cohomological_functors_dynkin_quiver_representations}) and from the definition of Grothendieck groups (cf. definitions \ref{def: simple_grothendieck_groups} and \ref{def: projective_grothendieck_groups}).
                \end{proof}
            \begin{proposition}[Tits quadratic forms as Euler characteristics] \label{prop: tits_quadratic_forms_as_euler_characteristics}
                Let $k$ be a field and let $\Gamma := (\Gamma_1, \Gamma_0, s, t)$ be a finite connected acyclic quiver. Then, there is a commutative diagram of $\Z$-module and homomorphisms between them as follows:
                    $$
                        \begin{tikzcd}
                        	{\K_0(\Gamma) \x \K_0(\Gamma)} && {\Z^{\oplus \Gamma_0} \tensor_{\Z} \Z^{\oplus \Gamma_0}} \\
                        	& \Z
                        	\arrow["{\chi_{\Gamma}(-, -)}"', from=1-1, to=2-2]
                        	\arrow["{q_{\Gamma}}", from=1-3, to=2-2]
                        	\arrow["{\dim_k \tensor_{\Z} \dim_k}", from=1-1, to=1-3]
                        \end{tikzcd}
                    $$
            \end{proposition}
                \begin{proof}
                    Because the Grothendieck group $\K_0(\Gamma)$ is admits $[\Rep_k^{\irr}(\Gamma)]$ as a $\Z$-linear basis (cf. proposition \ref{prop: simple_grothendieck_groups_of_finite_linear_abelian_categories_are_free_on_simple_objects}), which itself is in bijection with the set $\Phi_{\Gamma}^{\simple}$ of simple roots of $\Gamma$ via the dimension vector $\dim_k: \K_0(\Gamma) \to \Z^{\oplus \Phi_{\Gamma}^{\simple}}$ (cf. propositions \ref{prop: irreducible_representations_of_dynkin_quivers_are_labelled_by_simple_roots}), and because $\q_{\Gamma}(\alpha) = 1$ for all $\alpha \in \Phi_{\Gamma}^{\simple}$ by definition (cf. definition \ref{def: roots_of_dynkin_quivers}), it shall suffice to show that for all \textit{irreducible} representations $\calF \in \Ob(\Rep_k^{\irr}(\Gamma))$, one has:
                        $$\chi_{\Gamma}(\calF, \calF) := \dim_k \End_{k\<\Gamma\>}(\calF) - \dim_k \Ext^1_{k\<\Gamma\>}(\calF, \calF) = 1$$
                    To that end, recall first of all that because $\calF$ is a simple module over the Artinian $k$-algebra $k\<\Gamma\>$ (indeed, $k\<\Gamma\>$ is finite-dimensional as a $k$-vector space), $\calF$ is projective (cf. lemma \ref{lemma: projective_indecomposable_modules_over_artinian_algebras_are_simple}), which in particular tells us that:
                        $$\Ext^1_{k\<\Gamma\>}(\calF, \calF) \cong 0$$
                    It now remains to show that:
                        $$\dim_k \End_{k\<\Gamma\>}(\calF) = 1$$
                    For this, we can simply make use of the fact that as a simple (left-)$k\<\Gamma\>$-module, $\calF$ is cyclic.
                \end{proof}
            
            \begin{theorem}[Representation-finite connected and acyclic finite quivers are Dynkin] \label{theorem: gabriel_theorem_representation_finite_implies_dynkin}
                Let $k$ be a field and $\Gamma$ be a connected and acyclic finite quiver. $\Gamma$ is then a Dynkin quiver if it has finitely many isomorphism classes of indecomposable $k$-linear representations.
            \end{theorem}
                \begin{proof}
                    
                \end{proof}
                
            \begin{example}[Gabriel's Theorem for $\sfA_n$ quivers]
                
            \end{example}
            \begin{example}[Gabriel's Theorem for $\sfD_n$ quivers]
                
            \end{example}
            \begin{example}[Gabriel's Theorem for $\sfE_n$ quivers]
                
            \end{example}
            \begin{example}[Gabriel's Theorem fails for the loop quiver]
                
            \end{example}
            \begin{example}[Gabriel's Theorem fails for the Kronecker quiver]
                
            \end{example}
        
    \subsection{Tilting modules and (derived) Morita equivalences}
        \begin{convention}[Minimal additive subcategories] \label{conv: minimal_additive_subcategories}
            Let $\calA$ be a Krull-Schmidt \textit{additive} category (cf. definition \ref{def: krull_schmidt_categories}) and fix an object $X \in \Ob(\calA)$, whose Krull-Schmidt Decomposition into indecomposable objects $X_i$ shall be supposed to be:
                $$X := \bigoplus_{i \in I} X_i$$
            Recall also, that such a decomposition is unique up to isomorphisms (cf. theorem \ref{theorem: krull_schmidt_theorem}). Let us then write:
                $$\span_{\calA} X$$
            for the additive full subcategory of $\calA$ freely generated via direct sums by elements of the isomorphism class of zero objects of $\calA$, along with those of isomorphism classes of direct sums of the form:
                $$\bigoplus_{i \in I'} X_i$$
            wherein $I' \subset I$ are subsets of $I$ as above. 
        \end{convention}
        \begin{remark}
            Let $\calA$ be a Krull-Schmidt additive category and $X \in \Ob(\calA)$ be an object therein. Then, notice that the triple $(\calA[X], \oplus, 0)$ for some choice of zero object $0 \in \Ob(\calA)$ is a symmetric monoidal category. 
        \end{remark}
    
        \subsubsection{(Partial) tilting objects}
            \begin{definition}[Orthogonal objects] \label{def: orthogonal_objects}
                Let $k$ be an Artinian commutative ring and $\E$ be a finite $k$-linear abelian category. Then, for every object $X^{\bullet} \in \Ob(\rmD^b(\E))$, one defines the \textbf{orthogonal complement} of $X^{\bullet}$, denoted by $\E_X^{\bot}$, to be the kernel of the cohomological functor (cf. convention \ref{conv: cohomological_functors_dynkin_quiver_representations}):
                    $$\R\Hom_{\E}(-, X^{\bullet}): \rmD^b(\E) \to \rmD^b(k\mod^{\fin})$$
                in the sense that:
                    $$\E_X^{\bot} := \{Y^{\bullet} \in \Ob(\rmD^b(\E)) \mid \R\Hom_{\E}(Y^{\bullet}, X^{\bullet}) \cong 0^{\bullet}\}$$
                If $X^{\bullet} \in \E_X^{\bot}$ then we shall say that $X^{\bullet}$ is \textbf{self-orthogonal}. One also speaks of the \textbf{$i^{th}$ orthogonal complement} of $X^{\bullet}$, denoted by $H^i(\E_X^{\bot})$, which is kernel of the functor:
                    $$\Ext^i_{\E}(-, X^{\bullet}[0]): \rmD^b(\E) \to k\mod^{\fin}$$
                i.e.:
                    $$H^i(\E_X^{\bot}) := \{Y^{\bullet} \in \Ob(\rmD^b(\E)) \mid \Ext^i_{\E}(Y^{\bullet}, X^{\bullet}[0]) \cong 0\}$$
            \end{definition}
            \begin{convention}
                Let $k$ be an Artinian commutative ring and $\E$ be a finite $k$-linear abelian category. Then, we shall write:
                    $$\E^{\bot} := \bigcap_{X \in \Ob(\E)} \E_X^{\bot}$$
                for the full subcategory of $\rmD^b(\E)$ spanned by all the self-orthogonal objects. 
            \end{convention}
            \begin{definition}[Partial tilting objects] \label{def: partial_tilting_objects}
                Let $k$ be an Artinian commutative ring and $\E$ be a finite $k$-linear abelian category. The full subcategory $\E^{\flat}$ of $\rmD^b(\E)$ spanned by the so-called \textbf{partial tilting objects} is then given by:
                    $$\E^{\flat} := \rmD^{[0, 1]}(\E) \cap H^1(\E^{\bot})$$
                The category $\E^{\flat}$ contains another full subcategory of $\rmD^b(\E)$, denoted by $\E^{\flat \flat}$, which is spanned by what are known as the \textbf{tilting objects}; it is given by:
            \end{definition}
        
        \subsubsection{Separating and splitting tilting objects}
        
        \subsubsection{Torsion induced by tilting objects}
        
    \subsection{Hereditary rings}    
        \subsubsection{Hereditary algebras and hereditary categories}
            \begin{definition}[Hereditary abelian categories] \label{def: hereditary_abelian_categories}
                An abelian category $\calA$ is said to be \textbf{hereditary}\footnote{If only the full subcategory $\calA^{\fin}$ of finite-length objects is of global dimension $\leq 1$ (i.e. if $\calA^{\fin}$ is hereditary) then one says that the larger abelian category $\calA$ is \textbf{semi-hereditary}.} if and only if $\globdim \calA \leq 1$.
            \end{definition}
            \begin{convention}[Left/right-hereditary rings] \label{conv: left/right_hereditary_rings}
                When $\calA$ is the category of left/right-module over some ring $R$, one might say that the ring $R$ itself is \textbf{left/right-hereditary}\footnote{A ring is left/right semi-hereditary if and only if all finitely generated modules over it are of projective dimension $\leq 1$.} whenever $\calA$ itself is hereditary.
            \end{convention}
            \begin{proposition}[Kernels in hereditary abelian categories are projective] \label{prop: kernels_in_hereditary_abelian_categories_are_projective}
                Let $\calA$ be a hereditary abelian category. Then, every kernel (i.e. sub-object) therein will be projective.
            \end{proposition}
                \begin{proof}
                    
                \end{proof}
            \begin{corollary}[Kaplansky's Theorem] \label{coro: kaplansky_theorem}
                If $R$ is a left/right-hereditary ring then every submodule of a free left/right-$R$-module will be a direct sum of left/right-$R$-ideals. As a direct result of this, every submodule of a projective left/right-$R$-module is also projective\footnote{Hence the terminology \say{hereditary}.}.
            \end{corollary}
            \begin{example}
                Any left/right-semi-simple category (hence any semi-simple algebra) is left/right-hereditary. Concrete examples of hereditary algebras include finite-dimensional matrix rings over division rings (cf. theorem \ref{theorem: artin_wedderburn}) and group algebras $k\<G\>$ of finite groups $G$ over semi-simple rings $k$ such that $\chara k \nmid |G|$ (cf. theorem \ref{theorem: maschke_theorem}).
                
                As for algebras which are strictly left/right-hereditary and not left/right-semi-simple, consider the algebra $\b_2^-(k)$ of lower-triangular $2 \x 2$ matrices over a field $k$.
            \end{example}
            
        \subsubsection{Admissible ideals; quivers with relations}
            \begin{convention}
                Let us henceforth assume that $k$ is field (i.e. a simple commutative ring). For the most part, this assumption is made so that given any quiver $Q := (Q_1, Q_0, s, t)$, the corresponding path $k$-algebra $k\<Q\>$ will admit $\<Q\> := \<[Q]_1\>$ as a maximal two-sided ideal, commonly referred to as the \say{arrow ideal} of $Q$.
            \end{convention}
            
            \begin{definition}[Admissible ideals of path algebras] \label{def: admissible_ideals_of_path_algebras}
                The path $k$-algebra of any given quiver is said to be \textbf{admissible} if and only if it is admissible in the sense of definition \ref{def: pre_admissible_and_pre_adic_rings}, and said to be \textbf{bounded} if and only if its ideal of definition is a finite power of the arrow ideal.
            \end{definition}
            
        \subsubsection{Quivers associated to finite-dimensional algebra}
            \begin{definition}[Basic algebras] \label{def: basic_algebras}
                An associative algebra $A$ over some commutative ring $k$ is said to be \textbf{basic} if and only if the semi-simple\footnote{Cf. proposition \ref{prop: semi_simple_iff_trivial_jacobson_radical_and_artinian}} $k$-algebra $A/\rad(A)$ admits an Artin-Wedderburn Decomposition (cf. theorem \ref{theorem: artin_wedderburn}) into finitely many copies of $k$, i.e.:
                    $$A/\rad(A) \cong \prod_{i = 1}^d k$$
                wherein the product is taken in the category of associative $k$-algebras.
            \end{definition}
            \begin{remark}
                Obviously, if $A$ is a basic algebra over some commutative ring $k$ then $A$ will be of finite rank as a $k$-module, which is equal to the number of copies of $k$ in the Artin-Wedderburn Decomposition of $A/\rad(A)$; that is to say, if:
                    $$A/\rad(A) \cong \prod_{i = 1}^d k$$
                then:
                    $$\rank_k A = d$$
            \end{remark}
            \begin{remark}[Simple modules over basic algebras are $1$-dimensional] \label{remark: simple_modules_over_basic_algebras_are_one_dimensional}
                Let $k$ be a commutative ring and $A$ be a basic $k$-algebra. By the Artin-Wedderburn Theorem (cf. theorem \ref{theorem: artin_wedderburn}), one sees that every simple (left/right-)$A$-modules are of rank $1$ as $k$-modules. Among other things, this means that in order to compute $\rad(A)$, which by definition contains all elements $a \in A$ which act as $0$ on simple (left/right-)$A$-modules, one needs to only check whether or not said elements act as $0$ on the underlying ring $k$.
            \end{remark}
            \begin{definition}[Primitive idempotents] \label{def: primitive_idempotents}
                An idempotent ring element $e \in R$ (i.e. $e^2 = e$) is said to be \textbf{primitive} if and only if the left-$R$-ideal ${}_R\<e\>$ ((or equivalently, the right-$R$-ideal $\<e\>_R$) is an indecomposable left/right-$R$-module.
            \end{definition}
            \begin{proposition}[Primitive idempotents are indecomposable] \label{prop: primitive_idempotents_are_indecomposable}
                \cite[Proposition 21.8]{lam_first_course_in_noncommutative_rings} Let $R$ be an associative ring and $e \in R$ be an idempotent element therein. Then the following are equivalent:
                    \begin{enumerate}
                        \item $e$ is primitive.
                        \item The ring $eRe$ has no non idempotent elements aside from $0$ and $1$.
                        \item There does not exist a decomposition $e := \alpha + \beta$ of $e$ into non-zero orthogonal (i.e. $\alpha \beta = \beta \alpha = 0$) idempotents $\alpha, \beta \in R$.
                    \end{enumerate}
            \end{proposition}
            \begin{example}
                Consider the matrix ring $\Mat_2(k)$ over some field $k$ of characteristic $0$. The element $\begin{pmatrix} 1 & 0 \\ 0 & 0 \end{pmatrix}$ is first of all idempotent, but moreoever, primitive, as the ring $\begin{pmatrix} 1 & 0 \\ 0 & 0 \end{pmatrix} \Mat_2(k) \begin{pmatrix} 1 & 0 \\ 0 & 0 \end{pmatrix}$ has no idempotents aside from the additive and multiplicative identities: one can easily show via some obvious computations that:
                    $$\begin{pmatrix} 1 & 0 \\ 0 & 0 \end{pmatrix} \Mat_2(k) \begin{pmatrix} 1 & 0 \\ 0 & 0 \end{pmatrix} = \left\{ \begin{pmatrix} a & 0 \\ 0 & 0 \end{pmatrix} \: \bigg| \: \forall a \in k: a^2 = a\right\}$$
                but since $k$ is a field of characteristic $0$, there are no elements $a \in k$ such that $a^2 = a$ aside from $0$ and $1$. On the other hand, the identity matrix $\begin{pmatrix} 1 & 0 \\ 0 & 1 \end{pmatrix}$ (which is also idempotent) is \textit{not} primitive, since one can easily write:
                    $$\begin{pmatrix} 1 & 0 \\ 0 & 1 \end{pmatrix} = \begin{pmatrix} 1 & 0 \\ 0 & 0 \end{pmatrix} + \begin{pmatrix} 0 & 0 \\ 0 & 1 \end{pmatrix}$$
                and it is trivial to check that the two summands are idempotent and orthogonal to one another.
            \end{example}
            \begin{remark}[Idempotents split] \label{remark: idempotents_split}
                
            \end{remark}
            \begin{definition}[Associated basic algebras] \label{def: associative_basic_algebras}
                Let $A$ be a finite algebra over a commutative ring $k$. To such a $k$-algebra, one can construct a basic one, denoted by ${}^bA$, which is given by:
                    $${}^bA := e_A A e_A$$
                wherein:
                    $$e_A := \sum_{e \in E} e$$
                for some set $E$ of pair-wise orthogonal\footnote{For all $e, e' \in E$, one has $ee' = e'e = 0$ whenever $e \not = e'$} primitive idempotents such that $|E| \leq \rank A$ and for all $e, e' \in E$, one has ${}_A\<e\> \not \cong {}_A\<e'\>$ (or $\<e\>_A \not \cong \<e'\>_A$) if $e \not = e'$. For the sake of brevity, let us call $E$ a \textbf{over-basis} of $A$ as a $k$-module.
            \end{definition}
            \begin{example}
                Let $k$ be a field of characteristic $0$ and consider the $k$-algebra $\Mat_2(k)$ of $2 \x 2$ matrices with entries in $k$; within this algebra, consider the over-basis\footnote{We will let the reader check that the elements herein are pair-wise orthogonal idempotents.}:
                    $$E := \left\{ \begin{pmatrix} 0 & 0 \\ 0 & 0 \end{pmatrix}, \begin{pmatrix} 1 & 0 \\ 0 & 0 \end{pmatrix}, \begin{pmatrix} 0 & 0 \\ 0 & 1 \end{pmatrix}, \begin{pmatrix} 1 & 0 \\ 0 & 1 \end{pmatrix} \right\}$$
                
            \end{example}
            
        \subsubsection{Revisiting Gabriel's Theorem}