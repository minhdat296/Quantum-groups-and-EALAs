\section{Quantum groups}
    \subsection{Representations of Hopf algebras}
        \subsubsection{Fibre functors and reconstruction theorems}
            \begin{definition}[(Quasi-)fibre functors] \label{def: quasi_fibre_functors}
                Let $\E$ be a ring category over an Artinian commutative ring $k$. A \textbf{(quasi-)fibre functor} for $\E$ is then a ($k$-linear) (quasi-)tensor functor:
                    $$F: \E \to k\mod^{\fin}$$
            \end{definition}
            \begin{example}
                If $G$ is an abstract group and $k$ be a field, then the forgetful functor:
                    $$\Rep_k(G) \to k\-\Vect^{\fin}$$
                will then be a fibre functor. For a concrete example, recall first of all that for any \'etale-pointed Zariski-connected scheme $(X, \bar{x})$ over a field $k$ of characteristic $p \geq 0$, along with a prime $\ell \not = p$, there is a $\bar{\Q}_{\ell}$-linear Tannakian equivalence as follows, given by taking stalks of lisse \'etale $\ell$-adic sheaves $\calF \in \Ob(\Shv_{\bar{\Q}_{\ell}}^{\lisse}(X))$ at the chosen \'etale-geometric point $\bar{x}$:
                    $$
                        \begin{tikzcd}
                        	{\Shv_{\bar{\Q}_{\ell}}^{\lisse}(X)} && {\Rep_{\bar{\Q}_{\ell}}^{\fin}(\pi_1(X_{\fet}, \bar{x}))} \\
                        	& {\bar{\Q}_{\ell}\-\Vect^{\fin}}
                        	\arrow["\cong", from=1-1, to=1-3]
                        	\arrow["{(-)_{\bar{x}}}"', from=1-1, to=2-2]
                        	\arrow[from=1-3, to=2-2]
                        \end{tikzcd}
                    $$
                In this situation, both the stalk functor:
                    $$(-)_{\bar{x}}: \Shv_{\bar{\Q}_{\ell}}^{\lisse}(X) \to \bar{\Q}_{\ell}\-\Vect^{\fin}$$
                and the forgetful functor:
                    $$\Rep_{\bar{\Q}_{\ell}}^{\fin}(\pi_1(X_{\fet}, \bar{x})) \to \bar{\Q}_{\ell}\-\Vect^{\fin}$$
                are $\bar{\Q}_{\ell}$-linear fibre functors. 
            \end{example}
            \begin{proposition}[Bialgebra structures on fibre functors] \label{prop: bialgebra_structures_on_fibre_functors}
                Let $(\E, \tensor, \1)$ be a ring category over an Artinian commutative ring $k$ and let:
                    $$F: \E \to k\mod^{\fin}$$
                be a fibre functor. $\End(F)$, the associative $k$-algebra of natural endomorphisms on $F$, can then be equipped with a coassociative $k$-coalgebra structure with the following comultiplication and counit:
                    \begin{itemize}
                        \item \textbf{(Comultiplication):} 
                        \item \textbf{(Counit):}
                    \end{itemize}
                $\End(F)$ is thus a biassociative $k$-bialgebra.
            \end{proposition}
        
        \subsubsection{Pointed tensor categories}
        
        \subsubsection{(Hopf) quasi-bialgebras}

    \subsection{The prototypical quantum group: \texorpdfstring{$\U_q(\sl_2)$}{}}
        \subsubsection{The quantised enveloping algebra \texorpdfstring{$\U_q(\sl_2)$}{}}
            \begin{definition}[$q$-deformations] \label{def: q_deformations}
                Let $k$ be a field and $q \in k^{\x}$ be non-zero element therein. A \textbf{$q$-deformation} of an associative $k$-algebra $A$ is then an associative $A[q]$-algebra.
            \end{definition}
            \begin{definition}[The quantised enveloping algebra \texorpdfstring{$\U_q(\sl_2)$}{}] \label{def: quantised_U(sl2)}
                Let $k$ be a field and $q \in k^{\x} \setminus \{\pm 1\}$, and suppose that $E, F, K$ be an $\sl_2(k)$-triple. Then, we define $\U_q(\sl_2(k))$ to be the $q$-deformation of the associative $k$-algebra $\U(\sl_2(k))[K^{-1}]$ given by the relations:
                    $$KEK^{-1} = q^2 E$$
                    $$KFK^{-1} = q^{-2} F$$
                    $$[E, F] = \frac{K - K^{-1}}{q - q^{-1}}$$
            \end{definition}
        
        \subsubsection{Representations of \texorpdfstring{$\U_q(\sl_2)$}{}}
    
    \subsection{Quantised enveloping algebras of simple finite-dimensional Lie algebras}