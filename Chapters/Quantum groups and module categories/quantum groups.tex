\section{Hopf algebras and quantum groups}
    \subsection{\texorpdfstring{$q$}{}-analogues and quantum groups arising from Lie algebras}
        \subsubsection{\texorpdfstring{$q$}{}-analogues}
            \begin{definition}[$q$-analogues] \label{def: q_analogues}
                This shall be a somewhat informal definition of the notion of $q$-analogues, namely due to the fact that the notion itself is not a mathematical one but is instead more sociological in nature. Roughly speaking, a \textbf{$q$-analogue} is an analogue of a mathematical definition or result via the introduction of a parameter $q$ (sometimes referred to as a \say{quantum parameter}, and hence the notation), such that in the \say{classical limit} (also referred to as the \say{classical case} or the \say{limiting case}) $q \to 1$, the original definition or result is recovered. 
            \end{definition}
            \begin{example}
                Suppose that $k$ is a commutative ring of characteristic $0$ and consider the polynomial ring $k[x, y]$. A natural $q$-analogue to introduce would then be the noncommutative polynomial ring $k\<x, y\>/\<xy - q yx\>$: obviously, in the limit $q \to 1$, the ideal $\<xy - q yx\>$ becomes zero, and one recovers the commutative polynomial ring $k[x, y]$.
            \end{example}
            
            \begin{remark}[Affine group schemes and Hopf algebras]
                For definition \ref{def: quantum_groups}, let us first recall that for every commutative ring $k$, there is a monoidal equivalence between the symmetric monoidal category $\Grp\Sch_{/\Spec k}^{\aff}$ of affine group schemes over $\Spec k$ and the opposite $k\-\Comm\Hopf\Alg^{\op}$ of the symmetric monoidal category of commutative Hopf algebra objects internal to the symmetric monoidal category of $k$-modules. The strategy is then to define quantum groups (at least in the sense of Drinfeld-Jimbo) as $q$-analogues (perhaps with more quantum parameters) of commutative Hopf algebras. The resulting objects will generally neither be commutative nor cocommutative, but they will still be Hopf algebras. 
            \end{remark}
            \begin{definition}[Quantum groups] \label{def: quantum_groups}
                An \textbf{affine quantum group}
            \end{definition}
            
            \begin{definition}[Quantum groups associated to simple Lie algebras] \label{}
                
            \end{definition}
        
        \subsubsection{Quantum universal enveloping algebras}
        
\section{Quasi-triangular bialgebras and modules over monoidal categories}
    