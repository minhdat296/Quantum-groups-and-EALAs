\section{The category \texorpdfstring{$\calO$}{} of a Yangian}
    \subsection{Modified Casimir operators and tensor products of representations}
        We introduce here what is usually referred to as the \say{category $\calO$} of the Yangian $\rmY(\g)$ of a symmetrisable Kac-Moody algebra $\g$ (for simplicity, let us assume that its Cartan matrix is indecomposable). Though the definition is natural from the point of view of Lie theory, in practice it is a somewhat \textit{ad hoc} device\footnote{As opposed to the classical category $\calO$ of a finite-dimensional simple Lie algebra, or even the analogue thereof for symmetrisable Kac-Moody algebras (cf. \cite[Chapter 9]{kac_infinite_dimensional_lie_algebras}), which is a natural homological category with various nice properties.} - at least as far as the structural theory of Yangians attached to symmetrisable Kac-Moody algebras is concerned - as it exists only as a mean for constructing certain tensor products of $\rmY(\g)$-modules, which themselves are in service of the construction of a coproduct on a certain topological completion of $\rmY(\g)$. 

        \begin{convention}
            Suppose for now that $\g$ is a fixed symmetrisable Kac-Moody algebra whose Cartan matrix is indecomposable. 
        \end{convention}

        \begin{definition}[The category $\calO$ of $\rmY(\g)$] \label{def: category_O_of_yangians}
            There is a full subcategory $\calO(\rmY(\g))$ of ${}^l\rmY(\g)\mod$ consisting of objects $V$ which are:
            \begin{itemize}
                \item $\h$-diagonalisable,
                \item with finite-dimensional weight spaces $V_{\mu}$ for every $\mu \in \Pi(V)$, and
                \item such that there exist $\lambda_1, ..., \lambda_k \in \h^*$ such that, if $\mu \in \Pi(V)$ and $V_{\mu} \not \cong 0$ then:
                    $$\lambda_i - \mu \in \Lambda^+$$
                for all $1 \leq i \leq k$.
            \end{itemize}
        \end{definition}
        \begin{remark}
            One easy consequence of definition \ref{def: category_O_of_yangians} (specifically, the last condition) is that, if $V$ is an object of $\calO(\rmY(\g))$ and $\mu \in \h^*$ is some abstract weight, then for any positive (real) root $\alpha \in \Phi^+$, there exists a sufficiently large natural number $N \in \N$ such that:
                $$r \geq N \implies V_{\mu + r \alpha} \cong 0$$
        \end{remark}
        \begin{definition}[Integrable $\rmY(\g)$-modules] \label{def: integrable_modules_over_yangians}
            An object $V$ of $\calO(\rmY(\g))$ is said to be \textbf{integrable} if and only if for every abstract weight $\mu \in \h^*$ and for any positive (real) root $\alpha \in \Phi^+$, there exists a sufficiently large natural number $N \in \N$ such that:
                $$r \geq N \implies V_{\mu \pm r \alpha} \cong 0$$
        \end{definition}

        \begin{theorem}[Tensor products in $\calO(\rmY(\g))$] \label{theorem: tensor_products_in_the_category_O_of_yangians}
            \cite[Theorem 4.9]{guay_nakajima_wendlandt_affine_yangian_coproduct} For any pair of objects $V_1, V_2$ of $\calO(\rmY(\g))$, the map:
                $$\Delta_{V_1, V_2}: \rmY(\g) \to \End_{\bbC}(V_1 \tensor_{\bbC} V_2)$$
            given by:
                $$\Delta_{V_1, V_2}(E_{i, 0}^{\pm}) = \bar{\Delta}(E_{i, 0}^{\pm})$$
                $$\Delta_{V_1, V_2}(H_{i, 0}) = \bar{\Delta}(H_{i, 0})$$
                $$\Delta_{V_1, V_2}(T_{i, 1}) = \bar{\Delta}(T_{i, 1}) + [H_{i, 0} \tensor 1, \sfr_{\g}^+]$$
            is a homomorphism of $\bbC$-algebras. Here, we have that:
                $$\sfr_{\g}^+ := \sfr_{\h} + \sum_{\alpha \in \Phi^+} \sum_{k = 1}^{\dim_{\bbC} \g_{\alpha}} $$
        \end{theorem}
    
    \subsection{Completions}