\section{Representations of finite-dimensional semi-simple Lie algebras}
    For this section, our main references will be \cite[Chapter VI]{humphreys_lie_algebras} as well as the lecture notes \cite{gaitsgory_semi_simple_lie_algebras_notes}, wherein the terminologies used are more modern and the perspective is more geometric/categorical.

    \begin{convention} \label{conv: a_fixed_semi_simple_lie_algebra}
        From now on until further notice, we work with the following:
            \begin{itemize}
                \item a semi-simple and finite-dimensional Lie algebra $\g$ (over an algebraically closed field $k$ of characteristic $0$) whose Chevalley-Serre generators shall be denoted by $e_i^{\pm}, h_i$ (with $1 \leq i \leq l$),
                \item a choice of Cartan subalgebra $\h \subset \g$ of rank $l$,
                \item with respect to the choice of $\h$ above, a root system $\Phi$ with a choice of positive/negative roots $\Phi^{\pm}$,
                \item the corresponding (positive/negative) root lattices $\Lambda := \span_{\N} \Phi$ (respectively, by $\Lambda^{\pm} := \span_{\N} \Phi^{\pm}$), 
                \item the half-sums of positive (co)roots $\rho := \frac12 \sum_{\alpha \in \Phi^+} \alpha$ (respectively, $\rho^{\vee} := \frac12 \sum_{\alpha \in \Phi^+} \alpha^{\vee}$),
                \item positive/negative Borel subalgebras $\b^{\pm} \subset \g$ (maximal solvable subalgebras), unipotent radicals $\n^{\pm} := [\b^{\pm}, \b^{\pm}]$, and a triangular decomposition $\g \cong \n^- \oplus \h \oplus \n^+$; recall also that $\b^{\pm}/\n^{\pm} \cong \h$ and hence that $\b^{\pm} \cong \n^{\pm} \oplus \h$. 
            \end{itemize}
    \end{convention}

    \subsection{Highest-weight modules}
        In order to discuss the theory of highest-weight spaces and its implications towards the representation category of $\g$, we firstly need the following definitions.
        \begin{definition}[Weights and weight spaces] \label{def: weights_and_weight_modules}
            A \textbf{weight} of $\g$ is nothing but a linear functional $\lambda \in \h^*$. If $V$ is a $\g$-module then its \textbf{submodule of weight $\lambda$} shall be:
                $$V_{\lambda} := \{v \in V \mid \h \cdot v = \lambda v\}$$
            This submodule may be zero; when it is zero, we might say that the corresponding weight $\lambda$ is an \textbf{abstract weight} and when it is not zero, we might call $\lambda$ a \textbf{concrete weight}. The set of concrete weights of a given $\g$-module $V$ is denoted by $\Pi(V)$ (or perhaps $\Pi(V, \pi)$ when we write the representation as $(V, \pi)$)\footnote{In this notation, we have that $\Phi = \Pi(\g, \ad)$, i.e. roots are nothing but (concrete) weights of the adjoint representation.}. Also, for a fixed $\g$-module $V$ and a weight $\lambda \in \h^*$, we call the quantity:
                $$\dim_k V_{\lambda}$$
            the \textbf{multiplicity} of the weight $\lambda$ of $V$.
        \end{definition}
        \begin{definition}[Highest weights and highest-weight spaces] \label{def: highest_weights_and_highest_weight_modules}
            If $\lambda, \mu \in \h^*$ are weights such that $\lambda - \mu \in \Lambda^+$ then we will write:
                $$\lambda \geq \mu$$
            and for a $\g$-module $V$, a weight $\lambda \in \h^*$ is said to be \textbf{highest} if and only if $V_{\lambda} \not \cong 0$ and $\lambda \geq \mu$ for all $\mu \in \h^*$ such that $V[\mu] \not \cong 0$. A $\g$-module $V$ is said to be of \textbf{highest-weight $\lambda$} if and only if $\lambda \geq \mu$ for all $\mu \in \Pi(V)$.
        \end{definition}
        \begin{remark}[Highest-weight spaces in terms of $\g$-actions]
            It is clear by definition that, if $V$ is of highest-weight $\lambda \in \h^*$ then there must exist a vector $v_{\lambda} \in V$, called the \textbf{highest-weight vector}, such that:
                $$\h \cdot v_{\lambda} = \lambda v_{\lambda}, \n^+ \cdot v_{\lambda} = 0$$
            The latter condition can equivalently characterised as follows:
                $$\forall \alpha \in \Phi^+: \g_{\alpha} \cdot v_{\lambda} = 0$$
        \end{remark}

        Now that we have some of the basic terminologies at our disposal, let us turn our attention towards the first technical ingredient in our study, namely the notion of Verma modules. These are to be thought of (and in fact, they literally are) objects of an appropriate category of $\g$-modules (namely, that of the so-called \say{type I representations}) that generate said category via finite colimits (the aforementioned category is actually abelian, so finite colimits can be built out of finite direct sums and quotients).
        \begin{definition}[Verma modules] \label{def: verma_modules}
            Let $\lambda \in \h^*$ be a weight of $\g$. The \textbf{Verma module}\footnote{Humphrey called these \say{\textbf{standard cyclic modules}}; cf. \cite[Subsection 20.2]{humphreys_lie_algebras}.} associated to the weight $\lambda$ is the left-$\rmU(\g)$-module given by:
                $$\standard_{\lambda} := \rmU(\g) \tensor_{\rmU(\b^+)} k_{\lambda}$$
            wherein by $k_{\lambda}$, we mean the $1$-dimensional left-$\rmU(\b^+)$-module given by the canonical composition $\b^+ \to \h \xrightarrow[]{\lambda} k$.
        \end{definition}
        \begin{remark}
            It is not hard to see, using the triangular decomposition of $\g$ and the fact that the universal enveloping algebra functor $\rmU(-)$ is a left-adjoint, that one can choose a vector $v_{\lambda} \in \standard_{\lambda}$ (the canonical choice being $v_{\lambda} := 1 \tensor 1$) such that:
                $$\standard_{\lambda} \cong \rmU(\n^-) \tensor_k k_{\lambda} =: \rmU(\n^-) \cdot v_{\lambda}$$
            as left-$\rmU(\g)$-modules, i.e. $\standard_{\lambda}$ is a cyclic left-$\rmU(\g)$-module. In light of this, let us provide the following (somewhat \textit{post hoc}) justification for the definition of Verma modules. Also, this identification of the Verma module $\standard_{\lambda}$ shows us that:
                $$\h \cdot v_{\lambda} = \lambda v_{\lambda}, \n^+ \cdot v_{\lambda} = 0$$
            and hence $\standard_{\lambda}$ is of highest-weight $\lambda$ in the sense of definition \ref{def: highest_weights_and_highest_weight_modules}.
        
            Verma modules are supposed to capture the fact that - at least in the finite-dimensional case (which is what we care about anyway) - all highest-weight spaces are necessarily cyclic. Intuitively, one might suspect this to be the case by noticing that, with respect to the partial ordering of weights as in definition \ref{def: highest_weights_and_highest_weight_modules}, elements $y \in \n^+$ can be thought of as \say{raising operators} in the sense that they raise the weight of vectors in $\g$-modules (this is an easy consequence of Serre's Theorem) and \textit{vice versa} for the elements $x \in \n^-$, which can be thought of as \say{lowering operators}; as such, one ought to be able to obtain every element $v \in V$ of a highest-weight $\g$-module $V$ (say, of weight $\lambda$) by simply acting on the highest-weight vector $v_{\lambda} \in V$ using the \say{lowering operators} $x \in \n^-$ via left-multiplication.
        \end{remark}
        
        An easy-to-see important property of Verma modules is that they are uniquely characterised by their highest weights. 
        \begin{proposition}[Highest weights determine Verma modules] \label{prop: highest_weights_determine_verma_modules}
            Let $\lambda, \mu \in \h^*$ be weights. Then, $\standard_{\lambda} \cong \standard_{\mu}$ if and only if $\lambda = \mu$.  
        \end{proposition}
            \begin{proof}
                Suppose to the contrary that $\lambda \not = \mu$ but $\standard_{\lambda} \cong \standard_{\mu}$. Then $\standard_{\lambda}$ will be simultaneously of highest-weights $\lambda \not = \mu$, but since Verma modules are cyclic, this can not be the case. We are therefore left with a contradiction, and hence $\lambda \not = \mu$ implies $\standard_{\lambda} \not \cong \standard_{\mu}$, i.e. $\standard_{\lambda} \cong \standard_{\mu}$ implies that $\lambda = \mu$, by contraposition.

                The converse direction is a direct consequence of the definition of Verma modules. 
            \end{proof}
        
        Let us firstly see how $\h$ acts on $\standard_{\lambda}$, particularly in order to be able to obtain highest-weight spaces from Verma modules. 
        \begin{proposition}[Cartan action on Verma modules] \label{prop: cartan_action_on_verma_modules}
            $\h$ acts locally finitely and semi-simply on the Verma module $\standard_{\lambda}$ for any weight $\lambda \in \h^*$. In particular, the eigenvalues of $\h$ on $\standard_{\lambda}$ are of the form:
                $$\lambda - \sum_{\alpha \in \Phi^+} n_{\alpha} \alpha$$
            wherein each $n_{\alpha} \in \Z_{> 0}$.
        \end{proposition}
            \begin{proof}
                Denote the PBW filtration by $\rmU(\g) := \{\rmU(\g)_n\}_{n \in \Z}$ and write $\rmU(\n^-)_n := \rmU(\n^-) \cap \rmU(\g)_n$ for each $n \in \Z$. Per the construction of the Verma module $\standard_{\lambda}$, we have that:
                    $$\standard_{\lambda} \cong \rmU(\n^-) \cdot v_{\lambda}$$
                for some vector $v_{\lambda} \in \standard_{\lambda}$ and hence:
                    $$\standard_{\lambda} \cong \bigcup_{n \in \Z} \rmU(\n^-)_n \cdot v_{\lambda}$$
                Since $\g$ is finite-dimensional, each of the filtrants $\rmU(\n^-)_n$ is finite-dimensional (by PBW) and therefore it remains to show that they are $\h$-stable in order to show that $\h$ acts locally finitely on $\standard_{\lambda}$; in doing so, we will also be able to compute the eigenvalues of $\h$ on $\standard_{\lambda}$. To this end, note that by PBW, we have that the set $\{ e_{i_1}^- \cdot ... \cdot e_{i_m}^- \}_{1 \leq m \leq n}$ is a basis of $\rmU(\n^-)_n$ as a vector space. As such, elements $h \in \h$ act on the elements $(e_{i_1}^- \cdot ... \cdot e_{i_m}^-) \cdot v_{\lambda} \in \standard_{\lambda}$ in the following manner:
                    $$(e_{i_1}^- \cdot ... \cdot e_{i_m}^-) \cdot h(v_{\lambda}) + \sum_{1 \leq m' \leq m} (e_{i_1}^- \cdot ... \cdot [h, e_{i_{m'}}] \cdot ... \cdot e_{i_m}^-) \cdot v_{\lambda} = (\lambda - \sum_{\alpha \in \Phi^+} n_{\alpha} \alpha) (e_{i_1}^- \cdot ... \cdot e_{i_m}^-) \cdot v_{\lambda}$$
                wherein $n_{\alpha} \in \Z_{> 0}$ and the equality comes from Serre's Theorem and the fact that $v_{\lambda}$ is of weight $\lambda$. It is then clear that each $\rmU(\n^-)_n$ is $\h$-stable, and thus we are done. 
            \end{proof}
        For convenience, let us now introduce the following terminology:
        \begin{definition}[$\h$-semi-simple modules] \label{def: cartan_semi_simple_modules}
            A $\g$-module $V$ is said to be \textbf{$\h$-semi-simple} or \textbf{$\h$-diagonalisable} if and only if:
                $$V \cong \bigoplus_{\lambda \in \Pi(V)} V_{\lambda}$$
            We might also say that $V$ admits a \textbf{weight space decomposition}. Note that the weight spaces $ V_{\lambda}$ are \textit{a priori} only well-defined as $k$-vector spaces since the $\g$-action thereon can change the weight $\lambda$. 
        \end{definition}
        \begin{example}[Adjoint representation]
            The representation $(\g, \ad)$ is $\h$-semi-simple, seeing how $\g$ has its root space decomposition. 
        \end{example}
        One then sees that the following corollary of proposition \ref{prop: cartan_action_on_verma_modules} holds:
        \begin{corollary}[Verma modules are $\h$-semi-simple] \label{coro: verma_modules_are_cartan_semi_simple}
            For any weight $\lambda \in \h^*$, the corresponding Verma module $\standard_{\lambda} \cong \rmU(\n^-)$ is $\h$-semi-simple. Furthermore, the weight spaces $\standard_{\lambda}[\mu]$ are all finite-dimensional\footnote{... though there are $\aleph_0$-many of such weight spaces, since $\dim_k \standard_{\lambda} = \aleph_0$ by PBW.} (i.e. the weights $\mu \in \Pi(\standard_{\lambda})$ are of finite multiplicities); in particular, we have that:
                $$\standard_{\lambda}[\lambda] \cong \span_k v_{\lambda}$$
            and hence the highest weight $\lambda$ is of multiplicity $1$. 
        \end{corollary}

        \begin{lemma} \label{lemma: noncommutative_filtered_domains}
            \cite[Exercise I.9.4, pp. 20]{kassel_quantum_groups} Let $A := \{A_n\}_{n \geq 0}$ be an $\N$-filtered algebra over some field $k$ and $\gr(A) := \bigoplus_{n \geq 0} A_n/A_{n - 1}$ the associated $\N$-graded $k$-algebra (we take $A_{-1} := 0$). If $\gr(A)$ has no non-zero zero-divisors, then neither will $A$. 
        \end{lemma}
            \begin{proof}
                Suppose for the sake deriving a contradiction that there exists $a \in A$ such that $af = 0$ (or equivalently, $fa = 0$) for some $f \in A_m \setminus A_{m - 1}$ for some $m \geq 0$, while $\gr(A)$ does not have any non-zero zero-divisor; without any loss of generality, suppose also that $a \in A_n \setminus A_{n - 1}$ for some $n \geq 0$ and write $[x] := x \pmod{A_{r - 1}}$ for all $x \in A_r$ for any $r \geq 0$. As $\gr(A)$ does not have any non-zero zero-divisor, one has that $[a] [f]\not = 0$, which is can not be the case since $[0] = [af] = af \pmod{A_{n + m - 1}}$. We thus have a contradiction, and therefore $A$ has no non-zero zero-divisors. 
            \end{proof}
        \begin{corollary}[Universal enveloping algebras are noncommutative domains] \label{coro: universal_enveloping_algebras_are_noncommutative_domains}
            The only zero-divisor in the universal enveloping algebra $\rmU(\g)$ of any Lie algebra $\g$ (over some arbitrary field $k$) is $0$.
        \end{corollary}
            \begin{proof}
                Suppose that bases of $\g$ are in bijection with some set $I$. By the PBW Theorem, we know that:
                    $$\gr \rmU(\g) \cong \Sym(\g)$$
                and we also know that:
                    $$\Sym(\g) \cong k[\{x_i\}_{i \in I}]$$
                The polynomial algebra $k[\{x_i\}_{i \in I}]$ is an integral domain since $k$ is an integral domain (fields are integral domains), so its only zero-divisor is $0$, and hence the only zero-divisor in $\gr \rmU(\g)$ is also $0$. Lemma \ref{lemma: noncommutative_filtered_domains} then tells us that the above implies that $\rmU(\g)$ itself only has $0$ as a zero-divisor.
            \end{proof}
        \begin{proposition}[Homomorphisms between Verma modules are zero or injective] \label{prop: homomorphisms_between_verma_modules_are_zero_or_injective}
            Fix two weights $\mu, \lambda \in \h^*$. Then, any left-$\rmU(\g)$-module homomorphism:
                $$\phi: \standard_{\mu} \to \standard_{\lambda}$$
            is either zero or injective, and:
                $$\dim_k \Hom_{\rmU(\g)}(\standard_{\mu}, \standard_{\lambda}) \leq 1$$
            with equality occurring if and only if the homomorphism is zero. 
        \end{proposition}
            \begin{proof}
                Fix an arbitrary left-$\rmU(\g)$-module homomorphism:
                    $$\phi: \standard_{\mu} \to \standard_{\lambda}$$
                along with a maximal vector $v^+ \in \standard_{\lambda}$. By construction, any Verma of $\g$ is isomorphic as a left-$\rmU(\n^-)$-module to $\rmU(\n^-)$ and also, is cyclic as a left-$\rmU(\g)$-module by construction (cf. \cite[Subsection 20.2]{humphreys_lie_algebras}), so there exists a unique $y \in \rmU(\n^-)$ such that:
                    $$\phi(w^+) = y \cdot v^+$$
                wherein $w^+, v^+$ are choices of generators for the left-$\rmU(\g)$-modules $\standard_{\mu}, \standard_{\lambda}$ respectively. This means that:
                    $$\phi(-) = y \cdot (-)$$
                By corollary \ref{coro: universal_enveloping_algebras_are_noncommutative_domains}, the only zero-divisor in $\rmU(\n^-)$ is $0$, which means that:
                    $$\ker(y \cdot) := \{u \in \rmU(\n^-) \mid yu = 0\} = \{0\}$$
                or in other words, $y \cdot: \rmU(\n^-) \to \rmU(\n^-)$ is injective for any $y \in U^-$. This implies that $\phi$ as above is injective, so we are done. 
            \end{proof}
            
        Interestingly, Verma modules admit unique simple quotients which also happen to be finite-dimensional when the weights they are associated to are \say{dominant} and \say{integral}. Thanks to the universal property of Verma modules, we thus see that all finite-dimensional simple $\g$-modules arise as these quotients. Together, these two statements form what is summarily known as the Theorem of the Highest Weight. 
        \begin{lemma}[$\h$-semi-simplicity of submodules of Verma modules] \label{lemma: cartan_semi_simplicity_of_submodules_of_verma_modules}
            For any weight $\lambda \in \h^*$, submodules of $\standard_{\lambda}$ are $\h$-stable.
        \end{lemma}
            \begin{proof}
                
            \end{proof}
        \begin{proposition}[Unique simple quotients of Verma modules] \label{prop: unique_simple_quotients_of_verma_modules}
            For any weight $\lambda \in \h^*$, the corresponding left-$\rmU(\g)$-module $\standard_{\lambda}$ admits a unique simple quotient, which shall be denoted\footnote{\say{W} for \say{weight}.} by $\simple_{\lambda}$.
            
            Of course, $\simple_{\lambda}$ is also cyclic (by virtue of being a quotient of a cyclic module): namely, it is generated as a left-$\rmU(\g)$-module by the image of the highest-weight vector $v_{\lambda}$ under the quotient map $\standard_{\lambda} \to \simple_{\lambda}$; as such, $\simple_{\lambda}$ is also of highest-weight $\lambda$. 

            Additionally, for any two weights $\lambda, \mu \in \h^*$, one has that $\simple_{\lambda} \cong \simple_{\mu}$ if and only if $\lambda = \mu$. 
        \end{proposition}
            \begin{proof}
                Of course simple quotients exist, since there are always maximal (proper) submodules. As such, the only thing to show is that there is but one maximal left-$\rmU(\g)$-submodule of $\standard_{\lambda}$. Suppose for the sake of deriving a contradiction that there exist two distinct maximal left-$\rmU(\g)$-submodules $M \not \cong M' \subset \standard_{\lambda}$. Submodules of $\standard_{\lambda}$ are $\h$-stable \textit{a priori} (cf. lemma \ref{lemma: cartan_semi_simplicity_of_submodules_of_verma_modules}), so $M, M'$ admit the following weight space decompositions:
                    $$M \cong \bigoplus_{\mu \in \Pi(M)} (M \cap \standard_{\lambda}[\mu]), M' \cong \bigoplus_{\mu' \in \Pi(M')} (M' \cap \standard_{\lambda}[\mu'])$$
                Now, because $M, M'$ are proper submodules of a cyclic module, they can not contain the generator of said cyclic module; in other words, since $\standard_{\lambda} \cong \rmU(\n^-) \cdot v_{\lambda}$, we have that:
                    $$v_{\lambda} \not \in M, M'$$
                The vector $v_{\lambda}$ is of weight $\lambda$ per the construction of the Verma module $\standard_{\lambda}$, so this means that:
                    $$\lambda \not \in \Pi(M), \Pi(M')$$
                At the same time, because we have assumed that $M \not \cong M'$ and hence $\bigoplus_{\mu \in \Pi(M)} (M \cap \standard_{\lambda}[\mu]) \not \cong \bigoplus_{\mu' \in \Pi(M')} (M' \cap \standard_{\lambda}[\mu'])$, we have that:
                    $$\Pi(M) \not = \Pi(M')$$
                which means that $\Pi(M) \cup \Pi(M')$ contains both $\Pi(M)$ and $\Pi(M')$ as proper subsets. As $\lambda \not \in \Pi(M) \cup \Pi(M')$, we see that there exists a proper left-$\rmU(\g)$-module:
                    $$M'' := M \cup M' \cong \bigoplus_{\mu'' \in \Pi(M) \cup \Pi(M')} ( (M \cup M') \cap \standard_{\lambda}[\mu''])$$
                containing both $M, M'$ as proper submodules of its own. This contradicts the assumption that $M, M'$ are maximal as submodules of $\standard_{\lambda}$, thereby implying that $\standard_{\lambda}$ admits only a unique maximal submodule.

                The rest of this part of the proposition is then trivial. 
            \end{proof}

        It turns out that, in a certain sense, the converse statement to proposition \ref{prop: unique_simple_quotients_of_verma_modules} also holds: namely, amongst all the simple $\g$-modules, the finite-dimensional ones always carry a highest weight and hence arise as the unique simple quotient of certain Verma modules.
        \begin{lemma}[Finite-dimensional $\g$-modules are $\h$-semi-simple] \label{lemma: finite_dimensional_g_modules_are_cartan_semi_simple}
            Any finite-dimensional $\g$-module $V$ is $\h$-semi-simple, i.e. it admits a weight space decomposition:
                $$V \cong \bigoplus_{\lambda \in \h^*} V_{\lambda}$$
        \end{lemma}
            \begin{proof}
                
            \end{proof}
        \begin{proposition}[Finite-dimensional simple $\g$-modules are the simple quotients of Verma modules] \label{prop: finite_dimensional_simple_g_modules_are_simple_quotients_of_verma_modules}
            Let $W$ be a finite-dimensional simple $\g$-module. Then, there exists some $\lambda \in \h^*$ such that:
                $$\simple_{\lambda} \cong W$$
        \end{proposition}
            \begin{proof}
                Since $W$ is simple, it is cyclic. Since $W$ is finite-dimensional, it is $\h$-semi-simple by lemma \ref{lemma: finite_dimensional_g_modules_are_cartan_semi_simple}; write its weight space decomposition as:
                    $$W \cong \bigoplus_{\mu \in \h^*} W[\mu]$$
            \end{proof}
        \begin{corollary}[Weyl's Theorem on semi-simplicity of representations] \label{corollary: weyl_semi_simplicity_theorem}
            The category $\g\mod^{\fd}$ of finite-dimensional $\g$-modules is semi-simple and in particular, generated via finite direct sums by the finite-dimensional simple $\g$-modules $\simple_{\lambda}$ (for $\lambda \in \h^*$), i.e. if $V$ is any finite-dimensional $\g$-module then:
                $$V \cong \bigoplus_{\lambda \in \Pi(V)} \simple_{\lambda}$$
        \end{corollary}
            \begin{proof}
                
            \end{proof}

        Let us briefly return to discussing homomorphisms between Verma modules for a moment (we are \textit{far} from done with this topic!). For pairs of weights that satisfy the property of being so-called \say{strongly linked}, it is possible to be much more specific about how the (injective) homomorphisms between the associated Verma modules are given, not simply that they are either injective or zero as proposition \ref{prop: homomorphisms_between_verma_modules_are_zero_or_injective} asserted. However, let us defer discussion until later, as it requires firstly some background theory. We do, however, have a rudimentary version of this result for the weights lying the Weyl orbit of a given dominant weight, which says that the decreasing partial ordering within such an orbit iduces a decreasing filtration on the corresponding Verma modules.
        
        For convenience, let us recall the following definition:
        \begin{definition}[Dominant weights] \label{def: dominant_weights}
            A weight $\lambda \in \h^*$ is \textbf{dominant} if and only if $\lambda + \rho \in \Lambda^+$, where $\rho := \frac12 \sum_{\alpha \in \Phi^+} \alpha$.
        \end{definition}
        \begin{remark}[Why the half-sum of positive roots ?]
            
        \end{remark}
        \begin{lemma}[Weyl orbits of dominant weights] \label{lemma: weyl_orbits_of_dominant_weights}
            \cite[Lemma 13.2A]{humphreys_lie_algebras}
        \end{lemma}
        \begin{theorem}[Weyl filtrations for dominant Verma modules] \label{theorem: weyl_filtrations_for_dominant_verma_modules}
            If $\lambda$ is a dominant weight, and if $w := \sigma_n ... \sigma_1 \in \rmW$ being a reduced word (where each $\sigma_i$ is the reflection about the simple root $\alpha_i$), then we will get a finite decreasing filtration:
                $$\standard_{\lambda} \supset \standard_{\sigma_1 \cdot \lambda} \supset \standard_{\sigma_2 \sigma_1 \cdot \lambda} \supset ... \supset \standard_{w \cdot \lambda}$$
            corresponding \textit{covariantly} to the decreasing partial ordering of weights, which we know to exist by lemma \ref{lemma: weyl_orbits_of_dominant_weights}:
                $$\lambda \geq \sigma_1 \cdot \lambda \geq \sigma_2 \sigma_1 \cdot \lambda \geq ... \geq w \cdot \lambda$$
        \end{theorem}
            \begin{proof}
                
            \end{proof} 

    \subsection{Characters; theorems of Chevalley and Harish-Chandra}
        \begin{convention}[Schemes associated to Lie algebras]
            Let $k$ be any commutative ring and $E$ be any $k$-module. Any such $E$ defines, uniquely, an object ${}^{\sh}E$ of $\QCoh(\Spec k)$ called the generalised vector bundle associated to $E$, and when $E$ is finite locally free, ${}^{\sh}E$ will in fact be a genuine vector bundle over $\Spec k$. In any event, one can construct a relatively affine scheme:
                $$V(E) := \Spec \Sym(({}^{\sh}E)^*)$$
            over $\Spec k$, which is to be thought of as the total space of ${}^{\sh}E$. This is the scheme associated to the original $k$-module $E$. 

            When $k$ is a field, one can carry out the procedure above to associate, to any Lie algebra $\g$ over $k$, the scheme $V(\g)$. If $\g$ is finite-dimensional, say, a Lie subalgebra of $\Mat_n(k)$ for some sufficiently large $n \geq 1$, then we will have that:
                $$V(\g) \cong \Spec k[x_{1, 1}, x_{1, 2}, ..., x_{n, n - 1}, x_{n, n}]/I_{\g}$$
            wherein $I_{\g}$ is the ideal defined by the conditions on the elements of $\Mat_n(\g)$ that define the Lie algebra $\g$. 
        \end{convention}
        \begin{remark}
            Suppose that $k$ is a field and $E$ a finite-dimensional vector space over $k$. Then, the set $V(E)(k)$ and those of elements of $E$ will be in bijection, via:
                $$V(E)(k) \cong \Hom_{k\Comm\Alg}(\Sym(E^*), k) \cong \Hom_k(E^*, k) \cong E$$
            Note that the last bijection depends crucially on the finite-dimensionality of $E$. Likewise, we have that:
                $$V(\g)(k) \cong \g$$
        \end{remark}
        \begin{example}
            The scheme associated to $\sl_n(k)$ is:
                $$V(\sl_n(k)) \cong \Spec k[x_{1, 1}, x_{1, 2}, ..., x_{n, n - 1}, x_{n, n}]/( \sum_{1 \leq i \leq n} x_{i, i} )$$
        \end{example}
        \begin{remark}
            Let $G$ be an affine algebraic group $k$-scheme with Lie algebra $\g$. Then, on the $k$-scheme $V(\g)$, there is a natural $G$-action. 
        \end{remark}
        \begin{lemma} \label{lemma: fibres_of_geometric_vector_bundles_are_varieties}
            If $E$ is a finite-dimensional vector space over $k$ then $V(E)$ will be a $k$-variety, in the sense that it is of finite type and separated over $\Spec k$, and is integral (cf. \cite[\href{https://stacks.math.columbia.edu/tag/020D}{Tag 020D}]{stacks}). 
        \end{lemma}
            \begin{proof}
                That $V(E)$ is of finite-type and separated over $\Spec k$ is clear from the assumption that $E$ is finite-dimensional over $k$, which tells us that:
                    $$V(E) \cong \Spec \Sym(E^*) \cong \A^N_k$$
                with $N = \dim_k E$. 
            
                A morphism of schemes is integral if and only if it is affine and universally closed (cf. \cite[\href{https://stacks.math.columbia.edu/tag/01WM}{Tag 01WM}]{stacks}) and clearly the structural morphism $V(E) \to \Spec k$ is affine by construction, and since $\Spec k$ is a point, it is also universally closed and hence integral. 
            \end{proof}
        \begin{proposition}[Varieties of finite-dimensional Lie algebras] \label{prop: varieties_of_finite_dimensional_lie_algebras}
            Let $\g$ be a finite-dimensional Lie algebra over $k$. Then $V(\g)$ will be a variety. 
        \end{proposition}
            \begin{proof}
                Any affine space $\A_k^N$ is separated \textit{a priori}, any closed subschemes of separated schemes are also separated \textit{a priori}, so $V(\g)$ must be separated, as it embeds into some $\A_k^N$. The rest follows directly from lemma \ref{lemma: fibres_of_geometric_vector_bundles_are_varieties}. 
            \end{proof}
        \begin{remark}
            Suppose that $\g$ is a finite-dimensional Lie algebra over $k$. 
        
            As we assumed neither reducedness nor topological irreducibility in our definition of varieties, $V(\g)$ can contain infinitesimal closed subschemes, corresponding to nilpotent sections of $\scrO_{V(\g)}$, and it is not necessarily to assume that $\g$ is simple for everything to be well-defined. 

            We do have that, if $\g$ is finite-dimensional and simple, then $V(\g)$ will be irreducible (but still not reduced, which is in fact an important feature). This implies, particularly, that any Zariski open subset of $V(\g)$ is automatically dense.  
        \end{remark}
    
        \begin{definition}[Semi-simple elements] \label{def: semi_simple_elements}
            An element $\xi \in \g$ is said to be \textbf{semi-simple} if and only if it belongs to some Cartan subalgebra of $\g$ (equivalently, if and only if $\ad(\xi)$ is a semi-simple operator). 
        \end{definition}
        \begin{definition}[Regular elements] \label{def: regular_elements}
            An element $\xi \in \g$ is said to be \textbf{regular} if and only if:
                $$\dim_k \z_{\g}(\xi) = \dim_k \h$$
        \end{definition}
        \begin{remark}[Semi-simplicity vs. regularity]
            Since all the Cartan subalgebras of $\g$ are conjugate to one another (in the sense of being isomorphic to each other via inner automorphisms of $\g$), semi-simple elements of $\g$ are automatically regular, but the converse is not true; each regular element do, however, decomposes \textit{uniquely} into a sum of a nilpotent and a semi-simple element (this is the Jordan Decomposition).
        \end{remark}

        Chevalley's Restriction Theorem tells us that there is an isomorphism of GIT quotients:
            $$V(\g)/\!/G \cong V(\h)/\!/\rmW$$
        To begin to set this up, consider firstly the canonical composition:
            $$V(\g) \to V(\h) \to V(\h)/\!/\rmW$$
        wherein the first map is given by natural restriction of global sections (which induces a natural map of structure sheaves) and the second is the canonical quotient map (given by taking $\rmW$-invariants - with respect to reflections - of global sections); this composite map is usually denoted by $\phi$ and referred by the name of the \say{Chevalley map}. What are the properties of this Chevalley map ? To answer this question, we begin with the following technical lemma.
        \begin{lemma}
            Suppose that $E$ is a finite-dimensional $k$-vector space and $W$ is a finite group acting by reflections on $E$. Then:
            \begin{enumerate}
                \item $V(E)/\!/W$ is a smooth variety, and
                \item the canonical quotient map $\varpi: V(E) \to V(E)/\!/W$ is flat. 
            \end{enumerate}
        \end{lemma}
            \begin{proof}
                
            \end{proof}
        \begin{corollary}
            The Chevalley map:
                $$\phi: V(\g) \to V(\h)/\!/\rmW$$
            is flat.
        \end{corollary}
            \begin{proof}
                Since $\rmW$ is - by construction - a finite group acting by reflections on $\h^*$, we know that $\Sym(\h^*)$ is flat over $\Sym(\h^*)^{\rmW}$. At the same time, recall that by Serre's Theorem, $\rmU(\b^{\pm})$ are flat over $\Sym(\h^*) \cong \rmU(\h) \cong \rmU(\b^{\pm}/\n^{\pm})$. Now, recall that if $M$ is a $\Z_{\geq 0}$-filtered module over a $\Z_{\geq 0}$-filtered almost commutative $k$-algebra, then $M$ will be flat over $A$ if and only if $\gr M$ it is flat over $\gr A$. If we take $M$ to be $\rmU(\b^{\pm})$ with the root height filtration and $A$ to be $\Sym(\h^*)$, then we will have that $\gr \rmU(\b^{\pm}) \cong \Sym((\b^{\pm})^*)$ is flat over $\gr \Sym(\h^*) \cong \Sym(\h^*)$. This, in turn, implies that $\Sym((\g/\n^{\pm})^*) \cong \Sym((\b^{\pm})^*)$ is flat over $\Sym(\h^*) \cong \Sym((\b^{\pm}/\n^{\pm})^*)$. 

                It now remains to show that $\Sym(\g^*)$ is flat over $\Sym((\g/\n^{\pm})^*)$. 
            \end{proof}

    \subsection{The BGG category O: construction and structure}
        \begin{definition}[Category $\calO$] \label{def: category_O}
            The \textbf{BGG category $\calO$} is the full subcategory of $\g\mod$ generated via finite colimits and finite limits by Verma modules. 
        \end{definition}
        \begin{remark}[Motivation for the category $\calO$]
            The point of introducing this category $\calO$ is that the larger category $\g\mod^{\fin}$ of finitely generated $\g$-modules is somehow too untenable while at the same time, the category $\g\mod^{\fd}$ of finite-dimensional $\g$-modules is too inflexible (e.g. it does not contain Verma modules and hence the simple objects of $\g\mod^{\fd}$ can not be classified internally within $\g\mod^{\fd}$, in light of proposition \ref{prop: unique_simple_quotients_of_verma_modules}). For instance, it is difficult to classify the simple objects of $\g\mod^{\fin}$, since infinite-dimensional simple $\g$-modules might not be cyclic and hence can not arise from Verma modules. 

            Another reason is that, as we shall see below, the category $\calO$ is abelian and Noetherian, and hence gives rise to reasonable derived categories. 
        \end{remark}
        \begin{remark}
            It should also be noted that definition \ref{def: category_O}, despite its simplicity, is not a standard one. Objects of $\calO$ are usually specified by certain representation-theoretic and finiteness properties, such as being weight modules finite-dimensionality of weight spaces\footnote{Note that weight space finite-dimensionality is equivalent to saying that, for every object $M \in \Ob(\calO)$ and every vector $v \in M$, the vector subspace $\rmU(\n^+) \cdot v \subseteq M$ is finite-dimensional.} and finite-generation over $\rmU(\g)$. However, notice that, if $M$ satisfies the aforementioned properties, then:
            \begin{enumerate}
                \item if $M$ is finite-dimensional then by Weyl's Theorem, $M$ is a finite direct sum of finite-dimenisonal modules, which are certain quotients of Verma modules (cf. proposition \ref{prop: unique_simple_quotients_of_verma_modules}), and so $M$ will indeed be given by some finite colimit of a diagram of Verma modules,
                \item if $M$ is infinite-dimensional, then one can show that $M$ will either be an infinite direct sum of finite-dimensional simple left-$\rmU(\g)$-modules, or will be indecomposable as a left-$\rmU(\g)$-module and hence must be submodule of some Verma module.
            \end{enumerate}
            
        \end{remark}

        In order to describe the structure of $\calO$, we can rely entirely on the fact that Verma modules are the generators of $\calO$ in order to do this.   
        \begin{proposition}
            The category $\calO$ is a Noetherian and Artinian Serre subcategory of $\g\mod^{\fin}$.
        \end{proposition}
            \begin{proof}
                
            \end{proof}
        \begin{corollary}
            Every object of $\calO$ has finite length.
        \end{corollary}
        \begin{remark}
            Even though we now know that every object of $\calO$ is of some finite length, identifying for these objects some explicit (finite) filtrations turns out to be a rather daunting task. Without knowing what such filtrations might look like, one has no hope of identifying concrete composition series for the objects of $\calO$, which in turn prevents any attempt at studying multiplicities of the simple factors of such composition series. Luckily, explicit (and canonical!) filtrations for the objects of $\calO$ have in fact been computed, those being the so-called \say{Jantzen filtrations}.  
        \end{remark}

        \begin{theorem}[Dual Verma modules and contragredient duality for $\calO$] \label{theorem: dual_verma_modules}
            Write $\tau$ for the Cartan involution on $\g$ as well as its extension to the bialgebra $\rmU(\g)$, on which it becomes the Hopf antipode.
            
            There is a contravariant self-equivalence:
                $$(-)^{\vee}: \calO \xrightarrow[]{\cong} \calO^{\op}$$
            called \textbf{contragredient duality}. It assigns the Verma modules $\standard_{\lambda}$ to the so-called \textbf{dual Verma modules} $\costandard_{\lambda}$, which are given as vector spaces by:
                $$\costandard_{\lambda} := \bigoplus_{\mu \in \Pi(\standard_{\lambda})} \standard_{\lambda}[\mu]^*$$
            for all $\lambda \in \h^*$, and carries a left-$\rmU(\g)^{\op}$-module structure (or equivalently, a right-$\rmU(\g)$-module structure) given by pre-composition with the antipode $\tau: \rmU(\g) \to \rmU(\g)^{\op}$, i.e.:
                $$\forall \mu \in \Pi(\standard_{\lambda}): \forall \varphi \in \standard_{\lambda}[\mu]^*: \forall x \in \rmU(\g): \varphi(-) \cdot x := \varphi( \tau(x) \cdot - )$$
        \end{theorem}
            \begin{proof}
                
            \end{proof}
        \begin{remark}[Weights of dual Verma modules] \label{remark: weights_of_dual_verma_modules}
            Some easy - but nevertheless, important - consequences of the construction of dual Verma modules are as follows:
            \begin{itemize}
                \item The Cartan involution, by construction, gives rise to isomorphisms of Lie algebras:
                    $$\n^{\pm} \xrightarrow[\cong]{\tau} \n^{\mp}$$
                Consequently, one can write:
                    $$\costandard_{\lambda} \cong \rmU(\g)^{\op} \tensor_{\rmU(\b^+)^{\op}} k_{-\lambda} \cong \rmU(\g)^{\op}/\<\n^-, \h + \lambda\>$$
                for each $\lambda \in \h^*$. 
                \item It is also easy to see that:
                    $$\Pi(\costandard_{\lambda}) = -\Pi(\standard_{\lambda})$$
                and hence $\costandard_{\lambda}$ is to be understood as having \textit{lowest} weight $-\lambda$. 
                \item Finally, because the weight spaces $\standard_{\lambda}[\mu]$ are all finite-dimensional, one sees via this observation in conjunction with the previous two, that:
                    $$\costandard_{\lambda}^{\vee} \cong \standard_{\lambda}^{\vee \vee} \cong \standard_{\lambda}$$
                i.e. the functor $(-)^{\vee}$ is anti-involutive (the prefix \say{anti-} is to account for contravariance). 
                \item Given that homomorphisms between Verma modules are either injective or zero (cf. proposition \ref{prop: homomorphisms_between_verma_modules_are_zero_or_injective}), one sees that $(-)^{\vee}$ is exact by construction. In particular, this means that: 
                    $$\cosimple_{\lambda} := \simple_{\lambda}^{\vee} \in \Ob(\calO^{\op})$$
                is the unique simple left-$\rmU(\g)^{\op}$-module of $\costandard_{\lambda}$. 
                
                If $\lambda$ is dominant integral, then:
                    $$\simple_{\lambda} \in \Ob(\calO)$$
                will be finite-dimensional \textit{a priori}, and hence:
                    $$\cosimple_{\lambda} \cong \simple_{\lambda}^*$$
                as vector spaces, but strictly speaking, they are not isomorphic as $\rmU(\g)$-modules, since $\cosimple_{\lambda}$ is a $\rmU(\g)^{\op}$-module; it just so happens that the antipode $\tau: \rmU(\g) \to \rmU(\g)^{\op}$ is an algebra isomorphism and so typically, one would see the abuse of terminologies whereby $\cosimple_{\lambda}$ is said to be \say{isomorphic as a $\g$-module to $\simple_{\lambda}$}.  
            \end{itemize}
        \end{remark}
        \begin{lemma}[Cohomology of Verma modules] \label{lemma: cohomology_of_verma_modules}
            Fix $\lambda, \mu \in \h^*$. Then:
                $$
                    \dim_k \Ext^i_{\calO}(\standard_{\lambda}, \costandard_{\mu}) =
                    \begin{cases}
                        \text{$1$ if $\lambda = \mu$ and $i = 0$,}
                        \\
                        \text{$0$ if $\lambda \not = \mu$ and $i = 0$,}
                        \\
                        \text{$0$ for all $\lambda, \mu$ if $i = 1, 2$.}
                    \end{cases}
                $$
        \end{lemma}
            \begin{proof}
                
            \end{proof}
        \begin{proposition}[$\calO$ has enough projectives] \label{prop: category_O_has_enough_projective}
            The Verma modules, when viewed as objects of $\calO$, are projective. Since Verma modules generate $\calO$ via finite colimits, this implies immediately that $\calO$ has enough projectives.
        \end{proposition}
            \begin{proof}
                Fix a weight $\lambda$ and consider the functor $\Hom_{\calO}(-, \standard_{\lambda})$, which we need to show to be exact. 
            \end{proof}
        
        \begin{theorem}[BGG Reciprocity] \label{theorem: BGG_reciprocity}
            
        \end{theorem}
            \begin{proof}
                
            \end{proof}

    \subsection{Comparison with representations of finite-dimensional algebras}
        \begin{theorem}[The block decomposition of $\calO$] \label{theorem: block_decomposition_of_category_O}
            Denote by $\calO_{\chi}$ the full subcategory of $\calO$ spanned by objects on which $\rmZ(\g)$ acts by a specified central character $\chi$ (or rather, a representative of the $\rmW$-equivalence class of weights corresponding to $\chi$ via the Harish-Chandra Isomorphism). Then, $\calO$ admits the following direct sum decomposition:
                $$\calO \cong \bigoplus_{\chi \in \Spec \rmZ(\g)} \calO_{\chi}$$
        \end{theorem}
            \begin{proof}
                
            \end{proof}
        \begin{proposition}[Morita equivalences for the blocks of $\calO$] \label{prop: morita_equivalences_for_blocks_of_category_O}
            Fix a central character $\chi \in \Spec \rmZ(\g)$ and consider the set $\Pi_{\chi}$ of weights corresponding to $\chi$ via the Harish Chandra Isomorphism. Also, let us choose, for each $\lambda \in \Pi_{\chi}$, a projective covering $(P_{\lambda} \to \simple_{\lambda}) \in \Mor(\calO_{\chi})$ (which exists thanks to the existence of enough projectives in $\calO_{\chi}$; cf. proposition \ref{prop: category_O_has_enough_projective}), and then let us denote:
                $$P_{\chi} := \bigoplus_{\lambda \in \Pi_{\chi}} P_{\lambda}, \bbE_{\chi} := \End_k( P_{\chi} )$$
            Then we will have an equivalence of categories
                $$\Hom_{\calO}(P_{\chi}, -): \calO_{\chi} \xrightarrow[]{\cong} {}^r\bbE_{\chi}\mod^{\fd}$$
            (and hence, we see that the choices of projective coverings $(P_{\lambda} \to \simple_{\lambda}) \in \Mor(\calO_{\chi})$ are immaterial). 
        \end{proposition}
            \begin{proof}
                
            \end{proof}

    \subsection{The BGG resolution and its applications to highest-weight modules}