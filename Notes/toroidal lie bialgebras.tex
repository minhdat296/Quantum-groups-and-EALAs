\documentclass[a4paper, 11pt]{article}

%\usepackage[center]{titlesec}

\usepackage{amsfonts, amssymb, amsmath, amsthm, amsxtra}

\usepackage{foekfont}

\usepackage{MnSymbol}

\usepackage{pdfrender, xcolor}
%\pdfrender{StrokeColor=black,LineWidth=.4pt,TextRenderingMode=2}

%\usepackage{minitoc}
%\setcounter{section}{-1}
%\setcounter{tocdepth}{}
%\setcounter{minitocdepth}{}
%\setcounter{secnumdepth}{}

\usepackage{graphicx}

\usepackage[english]{babel}
\usepackage[utf8]{inputenc}
%\usepackage{mathpazo}
%\usepackage{eucal}
\usepackage{eufrak}
\usepackage{bbm}
\usepackage{bm}
\usepackage{csquotes}
\usepackage[nottoc]{tocbibind}
\usepackage{appendix}
\usepackage{float}
\usepackage[T1]{fontenc}
\usepackage[
    left = \flqq{},% 
    right = \frqq{},% 
    leftsub = \flq{},% 
    rightsub = \frq{} %
]{dirtytalk}

\usepackage{imakeidx}
\makeindex

%\usepackage[dvipsnames]{xcolor}
\usepackage{hyperref}
    \hypersetup{
        colorlinks=true,
        linkcolor=teal,
        filecolor=pink,      
        urlcolor=teal,
        citecolor=magenta
    }
\usepackage{comment}

% You would set the PDF title, author, etc. with package options or
% \hypersetup.

\usepackage[backend=biber, style=alphabetic, sorting=nty]{biblatex}
    \addbibresource{bibliography.bib}
\renewbibmacro{in:}{}

\raggedbottom

\usepackage{mathrsfs}
\usepackage{mathtools} 
\mathtoolsset{showonlyrefs} 
%\usepackage{amsthm}
\renewcommand\qedsymbol{$\blacksquare$}

\usepackage{tikz-cd}
\tikzcdset{scale cd/.style={every label/.append style={scale=#1},
    cells={nodes={scale=#1}}}}
\usepackage{tikz}
\usepackage{setspace}
\usepackage[version=3]{mhchem}
\parskip=0.1in
\usepackage[margin=25mm]{geometry}

\usepackage{listings, lstautogobble}
\lstset{
	language=matlab,
	basicstyle=\scriptsize\ttfamily,
	commentstyle=\ttfamily\itshape\color{gray},
	stringstyle=\ttfamily,
	showstringspaces=false,
	breaklines=true,
	frameround=ffff,
	frame=single,
	rulecolor=\color{black},
	autogobble=true
}

\usepackage{todonotes,tocloft,xpatch,hyperref}

% This is based on classicthesis chapter definition
\let\oldsec=\section
\renewcommand*{\section}{\secdef{\Sec}{\SecS}}
\newcommand\SecS[1]{\oldsec*{#1}}%
\newcommand\Sec[2][]{\oldsec[\texorpdfstring{#1}{#1}]{#2}}%

\newcounter{istodo}[section]

% http://tex.stackexchange.com/a/61267/11984
\makeatletter
%\xapptocmd{\Sec}{\addtocontents{tdo}{\protect\todoline{\thesection}{#1}{}}}{}{}
\newcommand{\todoline}[1]{\@ifnextchar\Endoftdo{}{\@todoline{#1}}}
\newcommand{\@todoline}[3]{%
	\@ifnextchar\todoline{}
	{\contentsline{section}{\numberline{#1}#2}{#3}{}{}}%
}
\let\l@todo\l@subsection
\newcommand{\Endoftdo}{}

\AtEndDocument{\addtocontents{tdo}{\string\Endoftdo}}
\makeatother

\usepackage{lipsum}

%   Reduce the margin of the summary:
\def\changemargin#1#2{\list{}{\rightmargin#2\leftmargin#1}\item[]}
\let\endchangemargin=\endlist 

%   Generate the environment for the abstract:
%\newcommand\summaryname{Abstract}
%\newenvironment{abstract}%
    %{\small\begin{center}%
    %\bfseries{\summaryname} \end{center}}

\newtheorem{theorem}{Theorem}[section]
    \numberwithin{theorem}{subsection}
\newtheorem{proposition}{Proposition}[section]
    \numberwithin{proposition}{subsection}
\newtheorem{lemma}{Lemma}[section]
    \numberwithin{lemma}{subsection}
\newtheorem{claim}{Claim}[section]
    \numberwithin{claim}{subsection}
\newtheorem{question}{Question}[section]
    \numberwithin{question}{subsection}

\theoremstyle{definition}
    \newtheorem{definition}{Definition}[section]
        \numberwithin{definition}{subsection}

\theoremstyle{remark}
    \newtheorem{remark}{Remark}[section]
        \numberwithin{remark}{subsection}
    \newtheorem{example}{Example}[section]
        \numberwithin{example}{subsection}    
    \newtheorem{convention}{Convention}[section]
        \numberwithin{convention}{subsection}
    \newtheorem{corollary}{Corollary}[section]
        \numberwithin{corollary}{subsection}

\numberwithin{equation}{section}

\setcounter{section}{-1}

\renewcommand{\cong}{\simeq}
\newcommand{\ladjoint}{\dashv}
\newcommand{\radjoint}{\vdash}
\newcommand{\<}{\langle}
\renewcommand{\>}{\rangle}
\newcommand{\ndiv}{\hspace{-2pt}\not|\hspace{5pt}}
\newcommand{\cond}{\blacktriangle}
\newcommand{\decond}{\triangle}
\newcommand{\solid}{\blacksquare}
\newcommand{\ot}{\leftarrow}
\renewcommand{\-}{\text{-}}
\renewcommand{\mapsto}{\leadsto}
\renewcommand{\leq}{\leqslant}
\renewcommand{\geq}{\geqslant}
\renewcommand{\setminus}{\smallsetminus}
\makeatletter
\DeclareRobustCommand{\cev}[1]{%
  {\mathpalette\do@cev{#1}}%
}
\newcommand{\do@cev}[2]{%
  \vbox{\offinterlineskip
    \sbox\z@{$\m@th#1 x$}%
    \ialign{##\cr
      \hidewidth\reflectbox{$\m@th#1\vec{}\mkern4mu$}\hidewidth\cr
      \noalign{\kern-\ht\z@}
      $\m@th#1#2$\cr
    }%
  }%
}
\makeatother

\newcommand{\N}{\mathbb{N}}
\newcommand{\Z}{\mathbb{Z}}
\newcommand{\Q}{\mathbb{Q}}
\newcommand{\R}{\mathbb{R}}
\newcommand{\bbC}{\mathbb{C}}
\NewDocumentCommand{\x}{e{_^}}{%
  \mathbin{\mathop{\times}\displaylimits
    \IfValueT{#1}{_{#1}}
    \IfValueT{#2}{^{#2}}
  }%
}
\NewDocumentCommand{\pushout}{e{_^}}{%
  \mathbin{\mathop{\sqcup}\displaylimits
    \IfValueT{#1}{_{#1}}
    \IfValueT{#2}{^{#2}}
  }%
}
\newcommand{\supp}{\operatorname{supp}}
\newcommand{\im}{\operatorname{im}}
\newcommand{\coker}{\operatorname{coker}}
\newcommand{\id}{\mathrm{id}}
\newcommand{\chara}{\operatorname{char}}
\newcommand{\trdeg}{\operatorname{trdeg}}
\newcommand{\rank}{\operatorname{rank}}
\newcommand{\trace}{\operatorname{tr}}
\newcommand{\length}{\operatorname{length}}
\newcommand{\height}{\operatorname{ht}}
\renewcommand{\span}{\operatorname{span}}
\newcommand{\e}{\epsilon}
\newcommand{\p}{\mathfrak{p}}
\newcommand{\q}{\mathfrak{q}}
\newcommand{\m}{\mathfrak{m}}
\newcommand{\n}{\mathfrak{n}}
\newcommand{\calF}{\mathcal{F}}
\newcommand{\calG}{\mathcal{G}}
\newcommand{\calO}{\mathcal{O}}
\newcommand{\F}{\mathbb{F}}
\DeclareMathOperator{\lcm}{lcm}
\newcommand{\gr}{\operatorname{gr}}
\newcommand{\vol}{\mathrm{vol}}
\newcommand{\ord}{\operatorname{ord}}
\newcommand{\projdim}{\operatorname{proj.dim}}
\newcommand{\injdim}{\operatorname{inj.dim}}
\newcommand{\flatdim}{\operatorname{flat.dim}}
\newcommand{\globdim}{\operatorname{glob.dim}}
\renewcommand{\Re}{\operatorname{Re}}
\renewcommand{\Im}{\operatorname{Im}}
\newcommand{\sgn}{\operatorname{sgn}}
\newcommand{\coad}{\operatorname{coad}}

\newcommand{\Ad}{\mathrm{Ad}}
\newcommand{\GL}{\mathrm{GL}}
\newcommand{\SL}{\mathrm{SL}}
\newcommand{\PGL}{\mathrm{PGL}}
\newcommand{\PSL}{\mathrm{PSL}}
\newcommand{\Sp}{\mathrm{Sp}}
\newcommand{\GSp}{\mathrm{GSp}}
\newcommand{\GSpin}{\mathrm{GSpin}}
\newcommand{\rmO}{\mathrm{O}}
\newcommand{\SO}{\mathrm{SO}}
\newcommand{\SU}{\mathrm{SU}}
\newcommand{\rmU}{\mathrm{U}}
\newcommand{\rmu}{\mathrm{u}}
\newcommand{\rmV}{\mathrm{V}}
\newcommand{\gl}{\mathfrak{gl}}
\renewcommand{\sl}{\mathfrak{sl}}
\newcommand{\diag}{\mathfrak{diag}}
\newcommand{\pgl}{\mathfrak{pgl}}
\newcommand{\psl}{\mathfrak{psl}}
\newcommand{\fraksp}{\mathfrak{sp}}
\newcommand{\gsp}{\mathfrak{gsp}}
\newcommand{\gspin}{\mathfrak{gspin}}
\newcommand{\frako}{\mathfrak{o}}
\newcommand{\so}{\mathfrak{so}}
\newcommand{\su}{\mathfrak{su}}
%\newcommand{\fraku}{\mathfrak{u}}
\newcommand{\Spec}{\operatorname{Spec}}
\newcommand{\Spf}{\operatorname{Spf}}
\newcommand{\Spm}{\operatorname{Spm}}
\newcommand{\Spv}{\operatorname{Spv}}
\newcommand{\Spa}{\operatorname{Spa}}
\newcommand{\Spd}{\operatorname{Spd}}
\newcommand{\Proj}{\operatorname{Proj}}
\newcommand{\Gr}{\mathrm{Gr}}
\newcommand{\Hecke}{\mathrm{Hecke}}
\newcommand{\Sht}{\mathrm{Sht}}
\newcommand{\Quot}{\mathrm{Quot}}
\newcommand{\Hilb}{\mathrm{Hilb}}
\newcommand{\Pic}{\mathrm{Pic}}
\newcommand{\Div}{\mathrm{Div}}
\newcommand{\Jac}{\mathrm{Jac}}
\newcommand{\Alb}{\mathrm{Alb}} %albanese variety
\newcommand{\Bun}{\mathrm{Bun}}
\newcommand{\loopspace}{\mathbf{\Omega}}
\newcommand{\suspension}{\mathbf{\Sigma}}
\newcommand{\tangent}{\mathrm{T}} %tangent space
\newcommand{\Eig}{\mathrm{Eig}}
\newcommand{\Cox}{\mathrm{Cox}} %coxeter functors
\newcommand{\rmK}{\mathrm{K}} %Killing form
\newcommand{\km}{\mathfrak{km}} %kac-moody algebras
\newcommand{\Dyn}{\mathrm{Dyn}} %associated Dynkin quivers
\newcommand{\Car}{\mathrm{Car}} %cartan matrices of finite quivers

\newcommand{\Ring}{\mathrm{Ring}}
\newcommand{\Cring}{\mathrm{CRing}}
\newcommand{\Alg}{\mathrm{Alg}}
\newcommand{\Leib}{\mathrm{Leib}} %leibniz algebras
\newcommand{\Fld}{\mathrm{Fld}}
\newcommand{\Sets}{\mathrm{Sets}}
\newcommand{\Equiv}{\mathrm{Equiv}} %equivalence relations
\newcommand{\Cat}{\mathrm{Cat}}
\newcommand{\Grp}{\mathrm{Grp}}
\newcommand{\Ab}{\mathrm{Ab}}
\newcommand{\Sch}{\mathrm{Sch}}
\newcommand{\Coh}{\mathrm{Coh}}
\newcommand{\QCoh}{\mathrm{QCoh}}
\newcommand{\Perf}{\mathrm{Perf}} %perfect complexes
\newcommand{\Sing}{\mathrm{Sing}} %singularity categories
\newcommand{\Desc}{\mathrm{Desc}}
\newcommand{\Sh}{\mathrm{Sh}}
\newcommand{\Psh}{\mathrm{PSh}}
\newcommand{\Fib}{\mathrm{Fib}}
\renewcommand{\mod}{\-\mathrm{mod}}
\newcommand{\comod}{\-\mathrm{comod}}
\newcommand{\bimod}{\-\mathrm{bimod}}
\newcommand{\Vect}{\mathrm{Vect}}
\newcommand{\Rep}{\mathrm{Rep}}
\newcommand{\Grpd}{\mathrm{Grpd}}
\newcommand{\Arr}{\mathrm{Arr}}
\newcommand{\Esp}{\mathrm{Esp}}
\newcommand{\Ob}{\mathrm{Ob}}
\newcommand{\Mor}{\mathrm{Mor}}
\newcommand{\Mfd}{\mathrm{Mfd}}
\newcommand{\Riem}{\mathrm{Riem}}
\newcommand{\RS}{\mathrm{RS}}
\newcommand{\LRS}{\mathrm{LRS}}
\newcommand{\TRS}{\mathrm{TRS}}
\newcommand{\TLRS}{\mathrm{TLRS}}
\newcommand{\LVRS}{\mathrm{LVRS}}
\newcommand{\LBRS}{\mathrm{LBRS}}
\newcommand{\Spc}{\mathrm{Spc}}
\newcommand{\Top}{\mathrm{Top}}
\newcommand{\Topos}{\mathrm{Topos}}
\newcommand{\Nil}{\mathfrak{nil}}
\newcommand{\J}{\mathfrak{J}}
\newcommand{\Stk}{\mathrm{Stk}}
\newcommand{\Pre}{\mathrm{Pre}}
\newcommand{\simp}{\mathbf{\Delta}}
\newcommand{\Res}{\mathrm{Res}}
\newcommand{\Ind}{\mathrm{Ind}}
\newcommand{\Pro}{\mathrm{Pro}}
\newcommand{\Mon}{\mathrm{Mon}}
\newcommand{\Comm}{\mathrm{Comm}}
\newcommand{\Fin}{\mathrm{Fin}}
\newcommand{\Assoc}{\mathrm{Assoc}}
\newcommand{\Semi}{\mathrm{Semi}}
\newcommand{\Co}{\mathrm{Co}}
\newcommand{\Loc}{\mathrm{Loc}}
\newcommand{\Ringed}{\mathrm{Ringed}}
\newcommand{\Haus}{\mathrm{Haus}} %hausdorff spaces
\newcommand{\Comp}{\mathrm{Comp}} %compact hausdorff spaces
\newcommand{\Stone}{\mathrm{Stone}} %stone spaces
\newcommand{\Extr}{\mathrm{Extr}} %extremely disconnected spaces
\newcommand{\Ouv}{\mathrm{Ouv}}
\newcommand{\Str}{\mathrm{Str}}
\newcommand{\Func}{\mathrm{Func}}
\newcommand{\Crys}{\mathrm{Crys}}
\newcommand{\LocSys}{\mathrm{LocSys}}
\newcommand{\Sieves}{\mathrm{Sieves}}
\newcommand{\pt}{\mathrm{pt}}
\newcommand{\Graphs}{\mathrm{Graphs}}
\newcommand{\Lie}{\mathrm{Lie}}
\newcommand{\Env}{\mathrm{Env}}
\newcommand{\Ho}{\mathrm{Ho}}
\newcommand{\rmD}{\mathrm{D}}
\newcommand{\Cov}{\mathrm{Cov}}
\newcommand{\Frames}{\mathrm{Frames}}
\newcommand{\Locales}{\mathrm{Locales}}
\newcommand{\Span}{\mathrm{Span}}
\newcommand{\Corr}{\mathrm{Corr}}
\newcommand{\Monad}{\mathrm{Monad}}
\newcommand{\Var}{\mathrm{Var}}
\newcommand{\sfN}{\mathrm{N}} %nerve
\newcommand{\Diam}{\mathrm{Diam}} %diamonds
\newcommand{\co}{\mathrm{co}}
\newcommand{\ev}{\mathrm{ev}}
\newcommand{\bi}{\mathrm{bi}}
\newcommand{\Nat}{\mathrm{Nat}}
\newcommand{\Hopf}{\mathrm{Hopf}}
\newcommand{\Dmod}{\mathrm{D}\mod}
\newcommand{\Perv}{\mathrm{Perv}}
\newcommand{\Sph}{\mathrm{Sph}}
\newcommand{\Moduli}{\mathrm{Moduli}}
\newcommand{\Pseudo}{\mathrm{Pseudo}}
\newcommand{\Lax}{\mathrm{Lax}}
\newcommand{\Strict}{\mathrm{Strict}}
\newcommand{\Opd}{\mathrm{Opd}} %operads
\newcommand{\Shv}{\mathrm{Shv}}
\newcommand{\Char}{\mathrm{Char}} %CharShv = character sheaves
\newcommand{\Huber}{\mathrm{Huber}}
\newcommand{\Tate}{\mathrm{Tate}}
\newcommand{\Affd}{\mathrm{Affd}} %affinoid algebras
\newcommand{\Adic}{\mathrm{Adic}} %adic spaces
\newcommand{\Rig}{\mathrm{Rig}}
\newcommand{\An}{\mathrm{An}}
\newcommand{\Perfd}{\mathrm{Perfd}} %perfectoid spaces
\newcommand{\Sub}{\mathrm{Sub}} %subobjects
\newcommand{\Ideals}{\mathrm{Ideals}}
\newcommand{\Isoc}{\mathrm{Isoc}} %isocrystals
\newcommand{\Ban}{\-\mathrm{Ban}} %Banach spaces
\newcommand{\Fre}{\-\mathrm{Fr\acute{e}}} %Frechet spaces
\newcommand{\Ch}{\mathrm{Ch}} %chain complexes
\newcommand{\Pure}{\mathrm{Pure}}
\newcommand{\Mixed}{\mathrm{Mixed}}
\newcommand{\Hodge}{\mathrm{Hodge}} %Hodge structures
\newcommand{\Mot}{\mathrm{Mot}} %motives
\newcommand{\KL}{\mathrm{KL}} %category of Kazhdan-Lusztig modules
\newcommand{\Pres}{\mathrm{Pres}} %presentable categories
\newcommand{\Noohi}{\mathrm{Noohi}} %category of Noohi groups
\newcommand{\Inf}{\mathrm{Inf}}
\newcommand{\LPar}{\mathrm{LPar}} %Langlands parameters
\newcommand{\ORig}{\mathrm{ORig}} %overconvergent sites
\newcommand{\Quiv}{\mathrm{Quiv}} %quivers
\newcommand{\Def}{\mathrm{Def}} %deformation functors
\newcommand{\Root}{\mathrm{Root}}
\newcommand{\gRep}{\mathrm{gRep}}
\newcommand{\Higgs}{\mathrm{Higgs}}
\newcommand{\BGG}{\mathrm{BGG}}

\newcommand{\Aut}{\mathrm{Aut}}
\newcommand{\Inn}{\mathrm{Inn}}
\newcommand{\Out}{\mathrm{Out}}
\newcommand{\der}{\mathfrak{der}} %derivations on Lie algebras
\newcommand{\frakend}{\mathfrak{end}}
\newcommand{\aut}{\mathfrak{aut}}
\newcommand{\inn}{\mathfrak{inn}} %inner derivations
\newcommand{\out}{\mathfrak{out}} %outer derivations
\newcommand{\Stab}{\mathrm{Stab}}
\newcommand{\Cent}{\mathrm{Cent}}
\newcommand{\Norm}{\mathrm{Norm}}
\newcommand{\stab}{\mathfrak{stab}}
\newcommand{\cent}{\mathfrak{cent}}
\newcommand{\norm}{\mathfrak{norm}}
\newcommand{\Rad}{\operatorname{Rad}}
\newcommand{\Transporter}{\mathrm{Transp}} %transporter between two subsets of a group
\newcommand{\Conj}{\mathrm{Conj}}
\newcommand{\Diag}{\mathrm{Diag}}
\newcommand{\Gal}{\mathrm{Gal}}
\newcommand{\bfG}{\mathbf{G}} %absolute Galois group
\newcommand{\Frac}{\mathrm{Frac}}
\newcommand{\Ann}{\mathrm{Ann}}
\newcommand{\Val}{\mathrm{Val}}
\newcommand{\Chow}{\mathrm{Chow}}
\newcommand{\Sym}{\mathrm{Sym}}
\newcommand{\End}{\mathrm{End}}
\newcommand{\Mat}{\mathrm{Mat}}
\newcommand{\Diff}{\mathrm{Diff}}
\newcommand{\Autom}{\mathrm{Autom}}
\newcommand{\Artin}{\mathrm{Artin}} %artin maps
\newcommand{\sk}{\mathrm{sk}} %skeleton of a category
\newcommand{\eqv}{\mathrm{eqv}} %functor that maps groups $G$ to $G$-sets
\newcommand{\Inert}{\mathrm{Inert}}
\newcommand{\Fil}{\mathrm{Fil}}
\newcommand{\Prim}{\mathfrak{Prim}}
\newcommand{\Nerve}{\mathrm{N}}
\newcommand{\Hol}{\mathrm{Hol}} %holomorphic functions %holonomy groups
\newcommand{\Bi}{\mathrm{Bi}} %Bi for biholomorphic functions
\newcommand{\chev}{\mathfrak{chev}} %chevalley relations
\newcommand{\bfLie}{\mathbf{Lie}} %non-reduced lie algebra associated to generalised cartan matrices
\newcommand{\frakLie}{\mathfrak{Lie}} %reduced lie algebra associated to generalised cartan matrices
\newcommand{\frakChev}{\mathfrak{Chev}} 
\newcommand{\Rees}{\operatorname{Rees}}
\newcommand{\Dr}{\mathrm{Dr}} %Drinfeld's quantum double 

\renewcommand{\projlim}{\varprojlim}
\newcommand{\indlim}{\varinjlim}
\newcommand{\colim}{\operatorname{colim}}
\renewcommand{\lim}{\operatorname{lim}}
\newcommand{\toto}{\rightrightarrows}
%\newcommand{\tensor}{\otimes}
\NewDocumentCommand{\tensor}{e{_^}}{%
  \mathbin{\mathop{\otimes}\displaylimits
    \IfValueT{#1}{_{#1}}
    \IfValueT{#2}{^{#2}}
  }%
}
\NewDocumentCommand{\singtensor}{e{_^}}{%
  \mathbin{\mathop{\odot}\displaylimits
    \IfValueT{#1}{_{#1}}
    \IfValueT{#2}{^{#2}}
  }%
}
\NewDocumentCommand{\hattensor}{e{_^}}{%
  \mathbin{\mathop{\hat{\otimes}}\displaylimits
    \IfValueT{#1}{_{#1}}
    \IfValueT{#2}{^{#2}}
  }%
}
\NewDocumentCommand{\semidirect}{e{_^}}{%
  \mathbin{\mathop{\rtimes}\displaylimits
    \IfValueT{#1}{_{#1}}
    \IfValueT{#2}{^{#2}}
  }%
}
\newcommand{\eq}{\operatorname{eq}}
\newcommand{\coeq}{\operatorname{coeq}}
\newcommand{\Hom}{\mathrm{Hom}}
\newcommand{\Maps}{\mathrm{Maps}}
\newcommand{\Tor}{\mathrm{Tor}}
\newcommand{\Ext}{\mathrm{Ext}}
\newcommand{\Isom}{\mathrm{Isom}}
\newcommand{\stalk}{\mathrm{stalk}}
\newcommand{\RKE}{\operatorname{RKE}}
\newcommand{\LKE}{\operatorname{LKE}}
\newcommand{\oblv}{\mathrm{oblv}}
\newcommand{\const}{\mathrm{const}}
\newcommand{\free}{\mathrm{free}}
\newcommand{\adrep}{\mathrm{ad}} %adjoint representation
\newcommand{\NL}{\mathbb{NL}} %naive cotangent complex
\newcommand{\pr}{\operatorname{pr}}
\newcommand{\Der}{\mathrm{Der}}
\newcommand{\Frob}{\mathrm{Fr}} %Frobenius
\newcommand{\frob}{\mathrm{f}} %trace of Frobenius
\newcommand{\bfpt}{\mathbf{pt}}
\newcommand{\bfloc}{\mathbf{loc}}
\DeclareMathAlphabet{\mymathbb}{U}{BOONDOX-ds}{m}{n}
\newcommand{\0}{\mymathbb{0}}
\newcommand{\1}{\mathbbm{1}}
\newcommand{\2}{\mathbbm{2}}
\newcommand{\Jet}{\mathrm{Jet}}
\newcommand{\Split}{\mathrm{Split}}
\newcommand{\Sq}{\mathrm{Sq}}
\newcommand{\Zero}{\mathrm{Z}}
\newcommand{\SqZ}{\Sq\Zero}
\newcommand{\lie}{\mathfrak{lie}}
\newcommand{\y}{\mathrm{y}} %yoneda
\newcommand{\Sm}{\mathrm{Sm}}
\newcommand{\AJ}{\phi} %abel-jacobi map
\newcommand{\act}{\mathrm{act}}
\newcommand{\ram}{\mathrm{ram}} %ramification index
\newcommand{\inv}{\mathrm{inv}}
\newcommand{\Spr}{\mathrm{Spr}} %the Springer map/sheaf
\newcommand{\Refl}{\mathrm{Refl}} %reflection functor]
\newcommand{\HH}{\mathrm{HH}} %Hochschild (co)homology
\newcommand{\Poinc}{\mathrm{Poinc}}
\newcommand{\Simpson}{\mathrm{Simpson}}

\newcommand{\bbU}{\mathbb{U}}
\newcommand{\V}{\mathbb{V}}
\newcommand{\calU}{\mathcal{U}}
\newcommand{\calW}{\mathcal{W}}
\newcommand{\rmI}{\mathrm{I}} %augmentation ideal
\newcommand{\bfV}{\mathbf{V}}
\newcommand{\C}{\mathcal{C}}
\newcommand{\D}{\mathcal{D}}
\newcommand{\T}{\mathscr{T}} %Tate modules
\newcommand{\calM}{\mathcal{M}}
\newcommand{\calN}{\mathcal{N}}
\newcommand{\calP}{\mathcal{P}}
\newcommand{\calQ}{\mathcal{Q}}
\newcommand{\A}{\mathbb{A}}
\renewcommand{\P}{\mathbb{P}}
\newcommand{\calL}{\mathcal{L}}
\newcommand{\E}{\mathcal{E}}
\renewcommand{\H}{\mathbf{H}}
\newcommand{\scrS}{\mathscr{S}}
\newcommand{\calX}{\mathcal{X}}
\newcommand{\calY}{\mathcal{Y}}
\newcommand{\calZ}{\mathcal{Z}}
\newcommand{\calS}{\mathcal{S}}
\newcommand{\calR}{\mathcal{R}}
\newcommand{\scrX}{\mathscr{X}}
\newcommand{\scrY}{\mathscr{Y}}
\newcommand{\scrZ}{\mathscr{Z}}
\newcommand{\calA}{\mathcal{A}}
\newcommand{\calB}{\mathcal{B}}
\renewcommand{\S}{\mathcal{S}}
\newcommand{\B}{\mathbb{B}}
\newcommand{\bbD}{\mathbb{D}}
\newcommand{\G}{\mathbb{G}}
\newcommand{\horn}{\mathbf{\Lambda}}
\renewcommand{\L}{\mathbb{L}}
\renewcommand{\a}{\mathfrak{a}}
\renewcommand{\b}{\mathfrak{b}}
\renewcommand{\c}{\mathfrak{c}}
\renewcommand{\t}{\mathfrak{t}}
\renewcommand{\r}{\mathfrak{r}}
\newcommand{\fraku}{\mathfrak{u}}
\newcommand{\bbX}{\mathbb{X}}
\newcommand{\frakw}{\mathfrak{w}}
\newcommand{\frakG}{\mathfrak{G}}
\newcommand{\frakH}{\mathfrak{H}}
\newcommand{\frakE}{\mathfrak{E}}
\newcommand{\frakF}{\mathfrak{F}}
\newcommand{\g}{\mathfrak{g}}
\newcommand{\h}{\mathfrak{h}}
\renewcommand{\k}{\mathfrak{k}}
\newcommand{\z}{\mathfrak{z}}
\newcommand{\fraki}{\mathfrak{i}}
\newcommand{\frakj}{\mathfrak{j}}
\newcommand{\del}{\partial}
\newcommand{\bbE}{\mathbb{E}}
\newcommand{\scrO}{\mathscr{O}}
\newcommand{\bbO}{\mathbb{O}}
\newcommand{\scrA}{\mathscr{A}}
\newcommand{\scrB}{\mathscr{B}}
\newcommand{\scrF}{\mathscr{F}}
\newcommand{\scrG}{\mathscr{G}}
\newcommand{\scrM}{\mathscr{M}}
\newcommand{\scrN}{\mathscr{N}}
\newcommand{\scrP}{\mathscr{P}}
\newcommand{\frakS}{\mathfrak{S}}
\newcommand{\frakT}{\mathfrak{T}}
\newcommand{\calI}{\mathcal{I}}
\newcommand{\calJ}{\mathcal{J}}
\newcommand{\scrI}{\mathscr{I}}
\newcommand{\scrJ}{\mathscr{J}}
\newcommand{\scrK}{\mathscr{K}}
\newcommand{\calK}{\mathcal{K}}
\newcommand{\scrV}{\mathscr{V}}
\newcommand{\scrW}{\mathscr{W}}
\newcommand{\bbS}{\mathbb{S}}
\newcommand{\scrH}{\mathscr{H}}
\newcommand{\bfA}{\mathbf{A}}
\newcommand{\bfB}{\mathbf{B}}
\newcommand{\bfC}{\mathbf{C}}
\renewcommand{\O}{\mathbb{O}}
\newcommand{\calV}{\mathcal{V}}
\newcommand{\scrR}{\mathscr{R}} %radical
\newcommand{\rmZ}{\mathrm{Z}} %centre of algebra
\newcommand{\rmC}{\mathrm{C}} %centralisers in algebras
\newcommand{\bfGamma}{\mathbf{\Gamma}}
\newcommand{\scrU}{\mathscr{U}}
\newcommand{\rmW}{\mathrm{W}} %Weil group
\newcommand{\frakM}{\mathfrak{M}}
\newcommand{\frakN}{\mathfrak{N}}
\newcommand{\frakB}{\mathfrak{B}}
\newcommand{\frakX}{\mathfrak{X}}
\newcommand{\frakY}{\mathfrak{Y}}
\newcommand{\frakZ}{\mathfrak{Z}}
\newcommand{\frakU}{\mathfrak{U}}
\newcommand{\frakR}{\mathfrak{R}}
\newcommand{\frakP}{\mathfrak{P}}
\newcommand{\frakQ}{\mathfrak{Q}}
\newcommand{\sfX}{\mathsf{X}}
\newcommand{\sfY}{\mathsf{Y}}
\newcommand{\sfZ}{\mathsf{Z}}
\newcommand{\sfS}{\mathsf{S}}
\newcommand{\sfT}{\mathsf{T}}
\newcommand{\sfOmega}{\mathsf{\Omega}} %drinfeld p-adic upper-half plane
\newcommand{\rmA}{\mathrm{A}}
\newcommand{\rmB}{\mathrm{B}}
\newcommand{\calT}{\mathcal{T}}
\newcommand{\sfA}{\mathsf{A}}
\newcommand{\sfD}{\mathsf{D}}
\newcommand{\sfE}{\mathsf{E}}
\newcommand{\frakL}{\mathfrak{L}}
\newcommand{\K}{\mathrm{K}}
\newcommand{\rmT}{\mathrm{T}}
\newcommand{\bfv}{\mathbf{v}}
\newcommand{\bfg}{\mathbf{g}}
\newcommand{\frakV}{\mathfrak{V}}
\newcommand{\frakv}{\mathfrak{v}}
\newcommand{\bfn}{\mathbf{n}}
\renewcommand{\o}{\mathfrak{o}}

\newcommand{\aff}{\mathrm{aff}}
\newcommand{\ft}{\mathrm{ft}} %finite type
\newcommand{\fp}{\mathrm{fp}} %finite presentation
\newcommand{\fr}{\mathrm{fr}} %free
\newcommand{\tft}{\mathrm{tft}} %topologically finite type
\newcommand{\tfp}{\mathrm{tfp}} %topologically finite presentation
\newcommand{\tfr}{\mathrm{tfr}} %topologically free
\newcommand{\aft}{\mathrm{aft}}
\newcommand{\lft}{\mathrm{lft}}
\newcommand{\laft}{\mathrm{laft}}
\newcommand{\cpt}{\mathrm{cpt}}
\newcommand{\cproj}{\mathrm{cproj}}
\newcommand{\qc}{\mathrm{qc}}
\newcommand{\qs}{\mathrm{qs}}
\newcommand{\lcmpt}{\mathrm{lcmpt}}
\newcommand{\red}{\mathrm{red}}
\newcommand{\fin}{\mathrm{fin}}
\newcommand{\fd}{\mathrm{fd}} %finite-dimensional
\newcommand{\gen}{\mathrm{gen}}
\newcommand{\petit}{\mathrm{petit}}
\newcommand{\gros}{\mathrm{gros}}
\newcommand{\loc}{\mathrm{loc}}
\newcommand{\glob}{\mathrm{glob}}
%\newcommand{\ringed}{\mathrm{ringed}}
%\newcommand{\qcoh}{\mathrm{qcoh}}
\newcommand{\cl}{\mathrm{cl}}
\newcommand{\et}{\mathrm{\acute{e}t}}
\newcommand{\fet}{\mathrm{f\acute{e}t}}
\newcommand{\profet}{\mathrm{prof\acute{e}t}}
\newcommand{\proet}{\mathrm{pro\acute{e}t}}
\newcommand{\Zar}{\mathrm{Zar}}
\newcommand{\fppf}{\mathrm{fppf}}
\newcommand{\fpqc}{\mathrm{fpqc}}
\newcommand{\orig}{\mathrm{orig}} %overconvergent topology
\newcommand{\smooth}{\mathrm{sm}}
\newcommand{\sh}{\mathrm{sh}}
\newcommand{\op}{\mathrm{op}}
\newcommand{\cop}{\mathrm{cop}}
\newcommand{\open}{\mathrm{open}}
\newcommand{\closed}{\mathrm{closed}}
\newcommand{\geom}{\mathrm{geom}}
\newcommand{\alg}{\mathrm{alg}}
\newcommand{\sober}{\mathrm{sober}}
\newcommand{\dR}{\mathrm{dR}}
\newcommand{\rad}{\mathfrak{rad}}
\newcommand{\discrete}{\mathrm{discrete}}
%\newcommand{\add}{\mathrm{add}}
%\newcommand{\lin}{\mathrm{lin}}
\newcommand{\Krull}{\mathrm{Krull}}
\newcommand{\qis}{\mathrm{qis}} %quasi-isomorphism
\newcommand{\ho}{\mathrm{ho}} %homotopy equivalence
\newcommand{\sep}{\mathrm{sep}}
\newcommand{\unr}{\mathrm{unr}}
\newcommand{\tame}{\mathrm{tame}}
\newcommand{\wild}{\mathrm{wild}}
\newcommand{\nil}{\mathrm{nil}}
\newcommand{\defm}{\mathrm{defm}}
\newcommand{\Art}{\mathrm{Art}}
\newcommand{\Noeth}{\mathrm{Noeth}}
\newcommand{\affd}{\mathrm{affd}}
%\newcommand{\adic}{\mathrm{adic}}
\newcommand{\pre}{\mathrm{pre}}
\newcommand{\coperf}{\mathrm{coperf}}
\newcommand{\perf}{\mathrm{perf}}
\newcommand{\perfd}{\mathrm{perfd}}
\newcommand{\rat}{\mathrm{rat}}
\newcommand{\cont}{\mathrm{cont}}
\newcommand{\dg}{\mathrm{dg}}
\newcommand{\almost}{\mathrm{a}}
%\newcommand{\stab}{\mathrm{stab}}
\newcommand{\heart}{\heartsuit}
\newcommand{\proj}{\mathrm{proj}}
\newcommand{\qproj}{\mathrm{qproj}}
\newcommand{\pd}{\mathrm{pd}}
\newcommand{\crys}{\mathrm{crys}}
\newcommand{\prisma}{\mathrm{prisma}}
\newcommand{\FF}{\mathrm{FF}}
\newcommand{\sph}{\mathrm{sph}}
\newcommand{\lax}{\mathrm{lax}}
\newcommand{\weak}{\mathrm{weak}}
\newcommand{\strict}{\mathrm{strict}}
\newcommand{\mon}{\mathrm{mon}}
\newcommand{\sym}{\mathrm{sym}}
\newcommand{\lisse}{\mathrm{lisse}}
\newcommand{\an}{\mathrm{an}}
\newcommand{\ad}{\mathrm{ad}}
\newcommand{\sch}{\mathrm{sch}}
\newcommand{\rig}{\mathrm{rig}}
\newcommand{\pol}{\mathrm{pol}}
\newcommand{\plat}{\mathrm{flat}}
\newcommand{\proper}{\mathrm{proper}}
\newcommand{\compl}{\mathrm{compl}}
\newcommand{\non}{\mathrm{non}}
\newcommand{\access}{\mathrm{access}}
\newcommand{\comp}{\mathrm{comp}}
\newcommand{\tstructure}{\mathrm{t}} %t-structures
\newcommand{\pure}{\mathrm{pure}} %pure motives
\newcommand{\mixed}{\mathrm{mixed}} %mixed motives
\newcommand{\num}{\mathrm{num}} %numerical motives
\newcommand{\ess}{\mathrm{ess}}
\newcommand{\topological}{\mathrm{top}}
\newcommand{\convex}{\mathrm{cvx}}
\newcommand{\locconvex}{\mathrm{lcvx}}
\newcommand{\ab}{\mathrm{ab}} %abelian extensions
\newcommand{\inj}{\mathrm{inj}}
\newcommand{\surj}{\mathrm{surj}} %coverage on sets generated by surjections
\newcommand{\eff}{\mathrm{eff}} %effective Cartier divisors
\newcommand{\Weil}{\mathrm{Weil}} %weil divisors
\newcommand{\lex}{\mathrm{lex}}
\newcommand{\rex}{\mathrm{rex}}
\newcommand{\AR}{\mathrm{A\-R}}
\newcommand{\cons}{\mathrm{c}} %constructible sheaves
\newcommand{\tor}{\mathrm{tor}} %tor dimension
\newcommand{\semisimple}{\mathrm{ss}}
\newcommand{\connected}{\mathrm{connected}}
\newcommand{\cg}{\mathrm{cg}} %compactly generated
\newcommand{\nilp}{\mathrm{nilp}}
\newcommand{\isg}{\mathrm{isg}} %isogenous
\newcommand{\qisg}{\mathrm{qisg}} %quasi-isogenous
\newcommand{\irr}{\mathrm{irr}} %irreducible represenations
\newcommand{\simple}{\mathrm{simple}} %simple objects
\newcommand{\indecomp}{\mathrm{indecomp}}
\newcommand{\preproj}{\mathrm{preproj}}
\newcommand{\preinj}{\mathrm{preinj}}
\newcommand{\reg}{\mathrm{reg}}
\renewcommand{\ss}{\mathrm{ss}}

%prism custom command
\usepackage{relsize}
\usepackage[bbgreekl]{mathbbol}
\usepackage{amsfonts}
\DeclareSymbolFontAlphabet{\mathbb}{AMSb} %to ensure that the meaning of \mathbb does not change
\DeclareSymbolFontAlphabet{\mathbbl}{bbold}
\newcommand{\prism}{{\mathlarger{\mathbbl{\Delta}}}}

\begin{document}

    \title{Toroidal Lie bialgebras and classical limits of affine Yangians}
    
    \author{Dat Minh Ha}
    \maketitle
    
    \begin{abstract}
        
    \end{abstract}
    
    {
    \hypersetup{} 
    %\dominitoc
    \tableofcontents %sort sections alphabetically
    \listoftodos
    }

    \section{Introduction}

    \section{Constructing Lie cobrackets on (extended) toroidal Lie algebras}
        \begin{convention} \label{conv: a_fixed_finite_dimensional_simple_lie_algebra}
            Throughout this section, we fix a finite-dimensional simple Lie algebra $\g$ over $\bbC$, equipped with a symmetric and non-degenerate invariant $\bbC$-bilinear form $(-, -)_{\g}$. It is known that such a bilinear form is unique up to $\bbC^{\x}$-multiples, so for all intents and purposes, it can be assumed to be the Killing form, though this assumption is not necessary. 

            Suppose also that $\g$ is equipped with a basis $\{x_i\}_{1 \leq i \leq \dim_{\bbC} \g}$ and with respect to $(-, -)_{\g}$, we identify a dual basis $\{x_i^*\}_{1 \leq i \leq \dim_{\bbC} \g}$. Recall that the universal Casimir element/canonical element of $\g$ is:
                $$\sfr_{\g} := \sum_{1 \leq i \leq \dim_{\bbC} \g} x_i \tensor x_i^* \in \g \tensor_{\bbC} \g$$
            and recall that $\sfr_{\g}$ is independent of what we choose the basis vectors $x_i$ to be.

            Eventually, we will also be concerned with the Dynkin diagram associated to the root system of $\g$. Let us denote this by:
                $$\Gamma := (\Gamma_0, \Gamma_1)$$
            wherein $\Gamma_0$ means the (finite) set of vertices and $\Gamma_1$ means the set of undirected edges between said vertices. 

            Finally, denote the associated Cartan matrix by:
                $$C := (c_{ij})_{1 \leq i, j \leq |\Gamma_0|}$$
            Since finite-type Cartan matrices are symmetrisable, we can fix a symmetrisation:
                $$C := D A$$
            wherein $D$ is an invertible diagonal $|\Gamma_0| \x |\Gamma_0|$ matrix and $A$ is symmetric. 
        \end{convention}

        \begin{convention}
            Throughout, we shall use $(-)^{\star}$ to denote graded duals. 
        \end{convention}

        \begin{convention}
            If $k$ is a commutative ring and $A$ is a $k$-algebra, and if $L$ is a Lie algebra over $k$, then the default Lie algebra structure on the $k$-module $L \tensor_k A$ shall be the one given by extension of scalars, i.e.:
                $$[x \tensor a, y \tensor b]_{L \tensor_k A} := [x, y]_L \mu_A(a \tensor b)$$
            $L \tensor_k A$ is usually regarded as Lie algebra over $k$ instead of over $A$.  
        \end{convention}

        \begin{convention} \label{conv: multiloop_algebras}
            We fix once and for all the following notations:
            \begin{itemize}
                \item $A_{[2]} := \bbC[v^{\pm 1}, t^{\pm 1}]$, $A_{[2]}^+ := \bbC[v^{\pm 1}, t], A_{[2]}^- := t^{-1}\bbC[v^{\pm 1}, t^{-1}]$;
                \item $\g_{[2]} := \g \tensor_{\bbC} A_{[2]}$, $\g_{[2]}^{\pm} := \g \tensor_{\bbC} A_{[2]}^{\pm}, \tilde{\g}_{[2]}^{\pm} := \uce(\g_{[2]}^{\pm})$ (with $\uce$ meaning \say{universal central extension});
                \item $\z_{[2]}^{\pm} := \z(\tilde{\g}_{[2]}^{\pm})$ (cf. remark \ref{remark: centres_of_dual_toroidal_lie_bialgebras}).
            \end{itemize}
        \end{convention}
    
        \subsection{A Lie bialgebra structure on the multiloop Lie algebra \texorpdfstring{$\g_{[2]}^+$}{}}
            \begin{definition} \label{def: residue_form_on_loop_algebra}
                For $v$ a formal variable, we can extend the invariant inner product $(-, -)_{\g}$ to the following pairing on $\g[v^{\pm 1}]$ by defining:
                    $$(x f(v), y g(v))_{\g[v^{\pm 1}]} := (x, y)_{\g} \Res_{v = 0}( v^{-1} f(v) g(v) )$$
                for all $x, y \in \g$ and all $f(v), g(v) \in \bbC[v^{\pm 1}]$, and recall that:
                    $$\Res_{v = 0}\left( \sum_{n \in \Z} a_n v^n \right) := a_{-1}$$
                More algebraically\footnote{So that the definition would work still when we replace $\bbC$ with a general algebraically closed field of characteristic $0$.}, we can define this as:
                    $$(x v^m, y v^n)_{\g[v^{\pm 1}]} := (x, y)_{\g} \delta_{m + n, 0}$$
            \end{definition}
            \begin{definition} \label{def: residue_form_on_multi_loop_algebra}
                Now, let $(-, -)_{\g[v^{\pm 1}]}$ be as in definition \ref{def: residue_form_on_loop_algebra}. This can be extended furthermore to $\g_{[2]}$ in the following manner: for all $X(v), Y(v) \in \g[v^{\pm 1}]$ and all $f(t), g(t) \in \bbC[t^{\pm 1}]$, define:
                    $$(X(v) f(t), Y(v) g(t))_{\g_{[2]}} := (X(v), Y(v))_{\g[v^{\pm 1}]} \Res_{t = 0}( f(t) g(t) )$$
                More algebraically, we can define this as:
                    $$(X(v) f(t), Y(v) g(t))_{\g_{[2]}} := (X(v), Y(v))_{\g[v^{\pm 1}]} \delta_{m + n, -1}$$
            \end{definition}

            \begin{question} \label{question: multiloop_lie_bialgebras}
                \begin{enumerate}
                    \item Verify that $(-, -)_{\g_{[2]}}$ is an invariant and non-degenerate symmetric $\bbC$-bilinear form on $\g_{[2]}$.
                    \item Show that by equpping $\g_{[2]}$ with the invariant inner product $(-, -)_{\g_{[2]}}$, the following triple of Lie algebras becomes a well-defined Manin triple:
                        $$(\g_{[2]}, \g_{[2]}^+, \g_{[2]}^-)$$
                    \item Find a formula for the canonical element $\sfr_{\g[v^{\pm 1}, t} \in \g_{[2]}^+ \hattensor_{\bbC} \g_{[2]}^+$ with respect to the restriction of $(-, -)_{\g_{[2]}}$ to $\g[v^{\pm 1}, t^{-1}] \x \g_{[2]}^+$.
                    \item Find the Lie bialgebra structure on $\g_{[2]}^+$ arising from the Manin triple in 2.
                \end{enumerate}
            \end{question}
                \begin{proof}
                    \begin{enumerate}
                        \item The symmetry and bilinearity of $(-, -)_{\g_{[2]}}$ are clear from the construction of this bilinear pairing as in definition \ref{def: residue_form_on_multi_loop_algebra}. $(-, -)_{\g_{[2]}}$-invariance follows from the $\g$-invariance of $(-, -)_{\g}$, which is by hypothesis. Finally, non-degeneracy follows from the non-degeneracy of $(-, -)_{\g}$ (also by hypothesis) as well as the non-degeneracy of the residual pairings on $\bbC[v^{\pm 1}]$ (as in definition \ref{def: residue_form_on_loop_algebra}) and on $\bbC[t^{\pm 1}]$ (as in definition \ref{def: residue_form_on_multi_loop_algebra}); to see that the latter point holds, simply note that there exists no $m \in \Z$ such that $\delta_{m + n, - 1} = 0$ (respectively, such that $\delta_{m + n, 0}$) for all $n \in \Z$.
                        \item It is not hard to see that: with respect to $(-, -)_{\g_{[2]}}$ as in 1, one has that:
                            $$(\g_{[2]}^+)^{\star} \cong \bigoplus_{m \in \Z, p \in \Z_{\geq 0}} (\g v^m t^p)^* \cong \bigoplus_{m \in \Z, p \in \Z_{\geq 0}} \g v^{-m} t^{-p - 1} \cong \g_{[2]}^-$$
                        with respect to the invariant inner product $(-, -)_{\g_{[2]}}$. It is also easy to see that:
                            $$\g_{[2]} \cong \g_{[2]}^+ \oplus \g_{[2]}^-$$
                        Note also that $\g_{[2]} \supset \g_{[2]}^+, \g_{[2]}^-$ are Lie subalgebras. Finally, to prove that $(-, -)_{\g_{[2]}}$ pairs the vector subspaces $\g_{[2]}^+, \g_{[2]}^-$ isotropically, simply that there does not exist any $p, q \geq 0$ or $p, q \leq -1$ simultaneously so that:
                            $$\delta_{p + q, -1} = 0$$
                        which means that:
                            $$(\g_{[2]}^-, \g_{[2]}^-)_{\g_{[2]}} = (\g_{[2]}^+, \g_{[2]}^+)_{\g_{[2]}} = 0$$
                        \item It will be convenient for us to make the identification of topological vector spaces:
                            $$\g_{[2]}^+ \hattensor_{\bbC} \g[v^{\pm 1}, t^{-1}] \cong \g[v_2^{\pm 1}, t_1] \hattensor_{\bbC} \g[v^{\pm 1}, t_2^{-1}]$$
                        Also, let us fix the basis:
                            $$\{X_{i, m, p}\}_{1 \leq i \leq \dim_{\bbC} \g, (m, p) \in \Z^2} := \{x_i v^m t^p\}_{1 \leq i \leq \dim_{\bbC} \g, (m, p) \in \Z^2}$$
                        for $\g_{[2]}$. It is easy to see that the graded dual of this basis with respect to the invariant inner product $(-, -)_{\g_{[2]}}$ is:
                            $$\{X_{i, m, p}^{\star}\}_{1 \leq i \leq \dim_{\bbC} \g, (m, p) \in \Z^2} := \{x_i^* v^{-m} t^{-p - 1}\}_{1 \leq i \leq \dim_{\bbC} \g, (m, p) \in \Z^2}$$
                        
                        By definition, the canonical element $\sfr_{\g_{[2]}^+} \in \g[v_2^{\pm 1}, t_1] \hattensor_{\bbC} \g[v^{\pm 1}, t_2^{-1}]$ is given by:
                            $$\sfr_{\g_{[2]}^+} := \sum_{1 \leq i \leq \dim_{\bbC} \g} \sum_{(m, p) \in \Z \x \Z_{\geq 0}} X_{i, m, p} \tensor X_{i, m, p}^{\star}$$
                        As such, we have that:
                            $$
                                \begin{aligned}
                                    \sfr_{\g_{[2]}^+} & := \sum_{1 \leq i \leq \dim_{\bbC} \g} \sum_{m \in \Z} \sum_{p \in \Z_{\geq 0}} x_i v_1^m t_1^p \tensor x_i^* v_2^{-m} t_2^{-p - 1}
                                    \\
                                    & = \left( \sum_{1 \leq i \leq \dim_{\bbC} \g} x_i \tensor x_i^* \right) \left( v_2 \sum_{m \in \Z} v_1^m v_2^{-m - 1} \right) \left( t_2^{-1} \sum_{p = 0}^{+\infty} (t_1 t_2^{-1})^p \right)
                                    \\
                                    & = \sfr_{\g} v_2 \1(v_1, v_2) \1^+(t_1, t_2)
                                \end{aligned}
                            $$
                        wherein:
                            $$\1(z, w) := \sum_{m \in \Z} z^m w^{-m - 1} \in \bbC[\![z^{\pm 1}, w^{\pm 1}]\!]$$ 
                        is the $2$-variable formal Dirac distribution, and we obtained:
                            $$t_2^{-1} \sum_{p = 0}^{+\infty} (t_1 t_2^{-1})^p = \frac{1}{t_2 - t_1} =: \1^+(t_1, t_2)$$
                        simply by formally evaluating the geometric series. 
                        \item Let us keep the identification:
                            $$\g_{[2]}^+ \hattensor_{\bbC} \g[v^{\pm 1}, t^{-1}] \cong \g[v_2^{\pm 1}, t_1] \hattensor_{\bbC} \g[v^{\pm 1}, t_2^{-1}]$$
                        From \cite[pp. 5]{etingof_kazhdan_quantisation_1}\footnote{Actually, this citation is not quite right, since the result was stated for finite-dimensional Manin triples only. However, since we're dealing with graded duals with finite-dimensional graded components, I believe the analogous result still holds. Of course, I should write this down carefully at some point.}, we know that the Lie bialgebra structure (say, $\delta_{\g_{[2]}^+}$) on $\g_{[2]}$ is given at any $X(v, t) \in \g_{[2]}$ by:
                            $$\delta_{\g_{[2]}^+}( X(v, t) ) = [X(v_1, t_1) \tensor 1 + 1 \tensor X(v_2, t_2), \sfr_{\g_{[2]}^+}]$$
                        When $X(v, t) := x v^m t^p$ for some $x \in \g, m \in \Z, t \in \Z_{\geq 0}$, this can written out more explicitly as follows:
                            $$
                                \begin{aligned}
                                    \delta_{\g_{[2]}^+}( x v^m t^p ) & = \left[x v_1^m t_1^p \tensor 1 + 1 \tensor x v_2^m t_2^p, \sfr_{\g} v_2 \1(v_1, v_2) \1^+(t_1, t_2)\right]
                                    \\
                                    & = [x \tensor 1 + 1 \tensor x, \sfr_{\g}] \cdot v_1^m t_1^p \cdot v_2^m t_2^p \cdot v_2 \1(v_1, v_2) \1^+(t_1, t_2)
                                \end{aligned}
                            $$
                    \end{enumerate}
                \end{proof}

        \subsection{Failing to extend the Lie bialgebra structure to the universal central extension \texorpdfstring{$\uce( \g_{[2]}^+ )$}{}}
            \begin{question} \label{question: extending_invariant_inner_products_on_multi_loop_to_universal_central_extensions}
                \begin{enumerate}
                    \item Prove that there is a unique invariant symmetric bilinear form $(-, -)_{\tilde{\g}_{[2]}}$ on $\t$ whose restriction to $\g_{[2]}$ coincides with $(-, -)_{\g_{[2]}}$.
                    \item Find a Lie subalgebra $\tilde{\g}_{[2]}^- \subset \t$ such that:
                        $$\tilde{\g}_{[2]}\cong \tilde{\g}_{[2]}^+\oplus \tilde{\g}_{[2]}^-$$
                    and such that $\tilde{\g}_{[2]}^-$ is paired isotropically with $\s$ by $(-, -)_{\tilde{\g}_{[2]}}$. 
                    \item Why is the triple:
                        $$(\tilde{\g}_{[2]}, \tilde{\g}_{[2]}^+, \tilde{\g}_{[2]}^-)$$
                    with $\t$ being equipped with $(-, -)_{\tilde{\g}_{[2]}}$ not a Manin triple ?
                \end{enumerate}
            \end{question}
                \begin{proof}
                    \begin{enumerate}
                        \item Suppose that $B$ is any invariant inner product on $\t$ and fix an element $Z \in \z_{[2]}$. This gives us:
                            $$B([X, Y], Z) = B(X, [Y, Z]) = B(X, 0) = 0$$
                        for all $X, Y \in \t$. As such, the sought-for unique invariant inner product on $\t$ induced by $(-, -)_{\g_{[2]}}$, whose restriction to $\g_{[2]} \subset \t$ coincides with $(-, -)_{\g_{[2]}}$, must be determined by:
                            $$(X, Z)_{\tilde{\g}_{[2]}} = 0, (Z, Z)_{\tilde{\g}_{[2]}} = 0$$
                        for all $X \in \t$ and all $Z \in \z_{[2]}$.
                        \item One thing that we are able to gather from 1 is that, with respect to $(-, -)_{\tilde{\g}_{[2]}}$, the centre $\z_{[2]}$ is orthogonally complementary to $\g_{[2]}$. With this in mind, we claim that:
                            $$\tilde{\g}_{[2]}^- \cong \g_{[2]}^- \oplus \z_{[2]}^-$$
                        wherein $\z_{[2]}^-$ is such that:
                            $$\z_{[2]} \cong \z_{[2]}^+ \oplus \z_{[2]}^-$$
                        and note that $\z_{[2]}^-$ must exist due to $\s$ being a Lie subalgebra of $\t$ and hence $\z_{[2]}^+$ being a Lie subalgebra of $\z_{[2]}$ (namely, one has that $\z_{[2]}^+ = \z_{[2]} \cap \s$). To see that this is indeed that the Lie subalgebra of $\t$ that we are after, firstly note that because:
                            $$(-, -)_{\tilde{\g}_{[2]}}|_{\g_{[2]}} = (-, -)_{\g_{[2]}}$$
                        and because it has been shown that $(-, -)_{\g_{[2]}}$ pairs $\g_{[2]}^+$ and $\g_{[2]}^-$ isotropically as subspaces of $\g_{[2]}$, the only thing to demonstrate is that $(-, -)_{\tilde{\g}_{[2]}}$ pairs elements of $\z_{[2]}^+$ and $\z_{[2]}^-$ isotropically with one another in the sense that:
                            $$(\z_{[2]}^+, \z_{[2]}^+)_{\tilde{\g}_{[2]}} = (\z_{[2]}^-, \z_{[2]}^-)_{\tilde{\g}_{[2]}} = 0$$
                        This is directly due to the fact that elements of $\z_{[2]}^+$ and likewise, those of $\z_{[2]}^-$, are central as elements of $\t$. Lastly, one verifies that, one indeed has that:
                            $$\tilde{\g}_{[2]}^+\oplus \tilde{\g}_{[2]}^- \cong ( \g_{[2]}^+ \oplus \z_{[2]}^+ ) \oplus ( \g_{[2]}^- \oplus \z_{[2]}^- ) \cong \g_{[2]} \oplus \z_{[2]} \cong \t$$
                        \item $(\tilde{\g}_{[2]}, \tilde{\g}_{[2]}^+, \tilde{\g}_{[2]}^-)$ is not a Manin triple (nor a graded Manin triple, for that matter) due to the simple fact that the non-zero vector space $\z_{[2]}$ is contained entirely in $\Rad (-, -)_{\tilde{\g}_{[2]}} := \{Z \in \tilde{\g}_{[2]}\mid \forall X \in \t: (X, Z)_{\tilde{\g}_{[2]}} = 0\}$. This implies that the invariant inner product $(-, -)_{\tilde{\g}_{[2]}}$ on $\t$ is \textit{degenerate}, thereby violating the definition of Manin triples. 

                        Note that we have not even checked whether or not $\tilde{\g}_{[2]}^-$ is actually a Lie subalgebra of $\t$ or merely a vector subspace. This will turn out to be true, but we defer this discussion to question \ref{question: toroidal_dual}. 
                    \end{enumerate}
                \end{proof}
            \begin{remark}[What exactly is $\z_{[2]}^-$ ?] \label{remark: centres_of_dual_toroidal_lie_bialgebras}
                In attempting to answer question \ref{question: extending_invariant_inner_products_on_multi_loop_to_universal_central_extensions}, we relied on the existence of an abstract vector subspace $\z_{[2]}^-$ of $\z_{[2]}$ specified by the condition that:
                    $$\z_{[2]} \cong \z_{[2]}^+ \oplus \z_{[2]}^-$$
                Let us now spend a bit of time on giving an explicit description of $\z_{[2]}^-$. 

                Suppose that $k$ is an arbitrary commutative ring. Recall firstly that, should $\a$ be a perfect Lie algebra over $k$ (i.e. a Lie algebra such that $\a = [\a, \a]$) with a non-degenerate invariant inner product $(-, -)_{\a}$, then not only does $\a_A := \a \tensor_k A$ admit a universal central extension $\uce(\a_A)$ for any commutative $k$-algebra $A$ (i.e. one that is initial in the category of all central extensions of $\a$) - and recall also that any universal central extension must split - but also, that there is the following explicit description of $\uce(\a_A)$ due to Kassel\todo{Cite Kassel's paper.}:
                    $$\uce(\a_A) \cong \a_A \oplus \bar{\Omega}^1_{A/k}$$
                with $\bar{\Omega}^1_{A/k} := \coim d_{A/k} := \Omega^1_{A/k}/d_{A/k}(A)$ being the coimage of the universal K\"ahler differential map $d_{A/k}: A \to \Omega^1_{A/k}$; if we denote the composition of the universal map $d_{A/k}: A \to \Omega^1_{A/k}$ with the canonical quotient map $\Omega^1_{A/k} \to \bar{\Omega}^1_{A/k}$ by:
                    $$\bar{d}_{A/k}: A \to \bar{\Omega}^1_{A/k}$$
                then the Lie bracket on $\uce(\a_A)$ with respect to Kassel's realisation shall be given by:
                    $$
                        \begin{aligned}
                            [ x \tensor a, y \tensor b ]_{\uce(\a_A)} & = [ X \tensor a, Y \tensor b ]_{\a_A} + (x, y)_{\a} b \bar{d}_{A/k}(a)
                            \\
                            & = [X, Y]_{\a} ab - (x, y)_{\a} a \bar{d}_{A/k}(b)
                        \end{aligned}
                    $$
                for all $x, y \in \a$ and all $a, b \in A$.
                    
                If $A$ is $\Z$-graded, say:
                    $$A := \bigoplus_{n \in \Z} A_n$$
                then $\a_A$ will also be $\Z$-graded, specifically in the following manner:
                    $$\a_A := \a \tensor_k A \cong \bigoplus_{n \in \Z} \a \tensor_k A_n$$
                and for convenience, let us write $\a_{A_n} := \a \tensor_k A_n$ for each $n \in \Z$. This grading on $\a_A$ actually extends to the whole of $\uce(\a_A)$, though to be able to describe this extension in details, let us firstly how the $A$-module $\Omega^1_{A/k}$ itself is constructed. To this end, recall firstly that the $A$-module $\Omega^1_{A/k}$ is generated by the set:
                    $$\{d_{A/k}(a)\}_{a \in A}$$
                subjected to the relations:
                    $$d_{A/k}(ab) - a d_{A/k}(b) - d_{A/k}(a) b = 0$$
                defined for all $a, b \in A$. 

                We now specialise to the case wherein $k \cong \bbC$, $\a = \g$, and $(-, -)_{\a} = (-, -)_{\g}$ as in convention \ref{conv: a_fixed_finite_dimensional_simple_lie_algebra} and for the moment, let us consider:
                    $$A \in \{ A_{[n]}^+ := \bbC[v_1, ..., v_n], A_{[n]}^{\pm} := \bbC[v_1^{\pm 1}, ..., v_n^{\pm 1}] \}$$
                and also, let us abbreviate:
                    $$\Omega^+_{[n]} := \Omega^1_{A_{[n]}^+/\bbC}, \Omega^{\pm}_{[n]} := \Omega^1_{A_{[n]}^{\pm}/\bbC}$$
                    $$\bar{\Omega}^+_{[n]} := \bar{\Omega}^1_{A_{[n]}^+/\bbC}, \bar{\Omega}_{[n]}^{\pm} := \bar{\Omega}^1_{A_{[n]}^{\pm}/\bbC}$$
                    $$d := d_{A/k}, \bar{d} := \bar{d}_{A/k}$$
                Eventually, we will specialise to the case $n = 2$. \textit{A priori}, both $\Omega^+_{[n]}$ and $\Omega^{\pm}_{[n]}$ are free and of rank $n$ over $A_{[n]}^+$ and $A_{[n]}^{\pm}$ respectively, specifically generated by the basis elements:
                    $$d(v_j)$$
                In turn, this implies that the $A_{[n]}^+$-module $\bar{\Omega}^+_{[n]}$ and the $A_{[n]}^{\pm}$-module $\bar{\Omega}_{[n]}^{\pm}$ are both generated by the basis elements:
                    $$\bar{d}(v_j)$$
                that are subjected to the following relation:
                    $$0 = \bar{d}( v_1^{m_1} ... v_n^{m_n} ) = \sum_{1 \leq j \leq n} m_j v_1^{m_1} ... v_j^{m_j - 1} ... v_n^{m_n} \bar{d}(v_j)$$
                From this, one infers that the elements:
                    $$m_j^{-1} v_1^{m_1} ... v_j^{m_j - 1} ... v_n^{m_n} \bar{d}(v_j)$$
                form a basis for $\bar{\Omega}^+_{[n]}$ and $\bar{\Omega}_{[n]}^{\pm}$ as $\bbC$-vector spaces. 

                When $n = 2$, we can write things out more explicitly: $\z_{[2]} \cong \bar{\Omega}_{[2]}^{\pm}$ now decomposes as a $\bbC$-vector space in the following manner:
                    $$\z_{[2]} \cong ( \bigoplus_{(r, s) \in \Z^2} \bbC Z_{r, s}) \oplus \bbC c_v \oplus \bbC c_t$$
                and $\z_{[2]}^+ \cong \bar{\Omega}_{[2]}^+$ decomposes in the following manner:
                    $$\z_{[2]}^+ \cong ( \bigoplus_{(r, s) \in \Z \x \Z_{> 0}} \bbC Z_{r, s}) \oplus \bbC c_v$$
                wherein:
                    $$
                        Z_{r, s} :=
                        \begin{cases}
                            \text{$\frac1s v^{r - 1} t^s \bar{d}(v)$ if $(r, s) \in \Z \x (\Z \setminus \{0\})$}
                            \\
                            \text{$-\frac1r v^r t^{-1} \bar{d}(t)$ if $(r, s) \in \Z \x \{0\}$}
                            \\
                            \text{$0$ if $(r, s) = (0, 0)$}
                        \end{cases}
                    $$
                    $$c_v := v^{-1} \bar{d}(v), c_t := t^{-1} \bar{d}(t)$$
                In fact, any element of the form:
                    $$v^m t^p \bar{d}(v^n t^q) \in \z_{[2]}$$
                can be written in terms of the basis vectors $Z_{r, s}, c_v, c_t$ in the following manner:
                    $$v^m t^p \bar{d}(v^n t^q) = \delta_{(m, p) + (n, q), (0, 0)} ( n c_v + q c_t ) + (np - mq) Z_{m + n, p + q}$$

                From the above and from the requirement on $\z_{[2]}^-$ that:
                    $$\z_{[2]} \cong \z_{[2]}^+ \oplus \z_{[2]}^-$$
                one sees immediately that:
                    $$\z_{[2]}^- \cong ( \bigoplus_{(r, s) \in \Z \x \Z_{\leq 0}} \bbC Z_{r, s}) \oplus \bbC c_t$$
            \end{remark}
            \begin{question} \label{question: toroidal_dual}
                Verify that $\tilde{\g}_{[2]}^-$ is a Lie subalgebra of $\t$.
            \end{question}
                \begin{proof}
                    We now know that:
                        $$\tilde{\g}_{[2]}^- \cong \g_{[2]}^- \oplus \left( ( \bigoplus_{(r, s) \in \Z \x \Z_{\leq 0}} \bbC Z_{r, s}) \oplus \bbC c_t \right)$$
                    (with notations as in remark \ref{remark: centres_of_dual_toroidal_lie_bialgebras}), so the verification can be carried out by firstly considering the following, for any $X(v, t), Y(v, t) \in \g_{[2]}^-$ and any $Z, Z' \in \z_{[2]}^-$:
                        $$
                            \begin{aligned}
                                [ X(v, t) + Z, Y(v, t) + Z' ]_{\tilde{\g}_{[2]}} & = [ X(v, t), Y(v, t) ]_{\tilde{\g}_{[2]}} + [ Z, Y(v, t) ]_{\tilde{\g}_{[2]}} + [X(v, t) + Z, Z']_{\tilde{\g}_{[2]}}
                                \\
                                & = [ X(v, t), Y(v, t) ]_{\tilde{\g}_{[2]}}
                            \end{aligned}
                        $$
                    wherein the equalities hold thanks to the elements $Z, Z'$ being central inside $\t$, and then, without any loss of generality, we consider secondly the following for:
                        $$X(v, t) := x f(v, t), Y(v, t) := y g(v, t)$$
                    for some $x, y \in \g$ and $f(v, t), g(v, t) \in t^{-1}\bbC[v^{\pm 1}, t^{-1}]$:
                        $$
                            \begin{aligned}
                                [ X(v, t), Y(v, t) ]_{\tilde{\g}_{[2]}} & = [ x f(v, t), y g(v, t) ]_{\tilde{\g}_{[2]}}
                                \\
                                & = [x, y]_{\g} f(v, t) g(v, t) - (x, y)_{\g} f(v, t) \bar{d}( g(v, t) )
                            \end{aligned}
                        $$
                    This is clearly an element of $\t$, in light of how the Lie bracket $[-, -]_{\tilde{\g}_{[2]}}$ is given, so we are done. 
                \end{proof}

        \subsection{Extending \texorpdfstring{$ \uce(\g[v^{\pm}, t]) $}{} to fix degeneracy}
            We now attempt to fix the issue whereby any bilinear form on $\tilde{\g}_{[2]}:= \uce(\g_{[2]})$ is necessarily degenerate. We do this by formally introducing a \say{complementary} vector space $\bar{\d}_{[2]}$ whose elements shall pair non-degenerately with those of $\z_{[2]}$. 
            \begin{convention} \label{conv: orthogonal_complement_of_toroidal_centres}
                From now on, $\bar{\d}_{[2]}$ shall be the $\bbC$-vector space:
                    $$\bar{\d}_{[2]} \cong ( \bigoplus_{(r, s) \in \Z^2} \bbC D_{r, s} ) \oplus \bbC D_v \oplus \bbC D_t$$
                such that we can endow:
                    $$\hat{\g}_{[2]} := \tilde{\g}_{[2]}\oplus \bar{\d}_{[2]}$$
                with a $\bbC$-bilinear form $(-, -)_{\hat{\g}_{[2]}}$ such that:
                \begin{itemize}
                    \item the elements $D_{r, s}, D_v, D_t$ are graded-dual with respect to $(-, -)_{\hat{\g}_{[2]}}$ to the elements $Z_{r, s}, c_v, c_t$, respectively;
                    \item $(\g_{[2]}, \z_{[2]} \oplus \bar{\d}_{[2]})_{\hat{\g}_{[2]}} := 0$;
                    \item $(\z_{[2]}, \z_{[2]})_{\hat{\g}_{[2]}} = (\bar{\d}_{[2]}, \bar{\d}_{[2]})_{\hat{\g}_{[2]}} := 0$;
                    \item $(-, -)_{\hat{\g}_{[2]}}|_{\Sym^2_{\bbC}(\g_{[2]})} := (-, -)_{\g_{[2]}}$
                \end{itemize}
            \end{convention}
            \begin{convention}
                Let us assume also that, should there be a Lie algebra structure $[-, -]_{\hat{\g}_{[2]}}$ on $\hat{\g}_{[2]}$ with respect to which $\t$ becomes a Lie subalgebra of $\hat{\g}_{[2]}$, i.e.
                    $$[-, -]_{\hat{\g}_{[2]}}|_{\bigwedge^2 \tilde{\g}_{[2]}} := [-, -]_{\tilde{\g}_{[2]}}$$
                then the bilinear form $(-, -)_{\hat{\g}_{[2]}}$ will be \textit{invariant} with respect to $[-, -]_{\hat{\g}_{[2]}}$.

                Even though this will turn out to be the case, we do not assume from the beginning that:
                    $$\hat{\g}_{[2]} \cong \tilde{\g}_{[2]}\rtimes \bar{\d}_{[2]}$$
                i.e. we do not need the assumption that $\t$ is a $\bar{\d}_{[2]}$-module. This is because \textit{a priori}, we have no knowledge of the Lie algebra structure on $\bar{\d}_{[2]}$.
            \end{convention}

            \begin{remark}[How does $\bar{\d}_{[2]}$ act on $\g_{[2]}$ ?] \label{remark: derivation_action_on_multiloop_algebras}
                Let us firstly see how elements of $\bar{\d}_{[2]}$ might act on those of $\g_{[2]}$, with respect to some Lie bracket $[-, -]_{\hat{\g}_{[2]}}$. 

                To this end, fix $x, y \in \g$, $(m, p), (n, q) \in \Z^2$, along with some $D \in \bar{\d}_{[2]}$, and then consider the following:
                    $$
                        \begin{aligned}
                            ( D, [x v^m t^p, y v^n t^q]_{\tilde{\g}_{[2]}} )_{\hat{\g}_{[2]}} & = ( D, [x, y]_{\g} v^{m + n} t^{p + q} - (x, y)_{\g} v^m t^p \bar{d}( v^n t^q ) )_{\hat{\g}_{[2]}}
                            \\
                            & = -(x, y)_{\g} ( D, v^m t^p \bar{d}( v^n t^q ) )_{\hat{\g}_{[2]}}
                            \\
                            & = -(x, y)_{\g} ( D, \delta_{(m, p) + (n, q), (0, 0)} ( n c_v + q c_t ) + (np - mq) Z_{m + n, p + q} )_{\hat{\g}_{[2]}}
                        \end{aligned}
                    $$
                Now, without any loss of generality, let us suppose that $D \in \bar{\d}_{[2]}$ is some basis element, i.e.:
                    $$D \in \{ D_{r, s}, D_v, D_t \}$$
                and consider these cases separately, for the sake of clarity:
                \begin{enumerate}
                    \item \textbf{(Case 1: $D := D_{r, s}$):} Fix some $(r, s) \in \Z^2$ and consider the following:
                        $$
                            \begin{aligned}
                                ( D_{r, s}, [x v^m t^p, y v^n t^q]_{\tilde{\g}_{[2]}} )_{\hat{\g}_{[2]}} & = -(x, y)_{\g} ( D_{r, s}, \delta_{(m, p) + (n, q), (0, 0)} ( n c_v + q c_t ) + (np - mq) Z_{m + n, p + q} )_{\hat{\g}_{[2]}}
                                \\
                                & = -(x, y)_{\g} (np - mq) \delta_{(r, s), (m + n, p + q)}
                            \end{aligned}
                        $$
                    The assumption that $(-, -)_{\hat{\g}_{[2]}}$ is invariant with respect to $[-, -]_{\hat{\g}_{[2]}}$ then implies that:
                        $$( [D_{r, s}, x v^m t^p]_{\hat{\g}_{[2]}}, y v^n t^q )_{\hat{\g}_{[2]}} = -(x, y)_{\g} (np - mq) \delta_{(r, s), (m + n, p + q)}$$
                    Now, suppose that:
                        $$[D_{r, s}, x v^m t^p]_{\hat{\g}_{[2]}} := \sum_{(a, b) \in \Z^2} \lambda_{a, b}(x) v^a t^b + K_{(m, p), (r, s)}(x) + \xi_{(m, p), (r, s)}(x)$$
                    for some $\lambda_{a, b}(x) \in \g$, $K_{(m, p), (r, s)}(x) \in \z_{[2]}$, and $\xi_{(m, p), (r, s)}(x) \in \bar{\d}_{[2]}$, depending on our choices of $x \in \g$ and $(m, p) \in \Z^2$. Next, consider the following:
                        $$
                            \begin{aligned}
                                ( [D_v, x v^m t^p]_{\hat{\g}_{[2]}}, y v^n t^q )_{\hat{\g}_{[2]}} & = \left( \sum_{(a, b) \in \Z^2} \lambda_{a, b}(x) v^a t^b + K_{(m, p), (r, s)}(x) + \xi_{(m, p), (r, s)}(x), y v^n t^q \right)_{\hat{\g}_{[2]}}
                                \\
                                & = \sum_{(a, b) \in \Z^2} \left( \lambda_{a, b}(x) v^a t^b, y v^n t^q \right)_{\g_{[2]}}
                                \\
                                & = \sum_{(a, b) \in \Z^2} (\lambda_{a, b}(x), y)_{\g} \delta_{ (a, b) + (n, q), (0, -1) }
                                \\
                                & = (\lambda_{-n, -q - 1}(x), y)_{\g}
                            \end{aligned}
                        $$
                    which tells us that:
                        $$-(x, y)_{\g} (np - mq) \delta_{(r, s), (m + n, p + q)} = (\lambda_{-n, -q - 1}(x), y)_{\g}$$
                    The non-degeneracy of the inner product $(-, -)_{\g}$ as well as the arbitrariness of the choices of $y \in \g$ and $(n, q) \in \Z^2$ then together imply that:
                        $$\lambda_{-n, -q - 1}(x) = -(np - mq) \delta_{(r, s), (m + n, p + q)}$$
                    for any fixed choices of $x \in \g$ and $(m, p) \in \Z^2$. From this, we infer that:
                        $$
                            \begin{aligned}
                                [D_{r, s}, x v^m t^p]_{\hat{\g}_{[2]}} & = \sum_{(n, q) \in \Z^2} -(np - mq) \delta_{(r, s), (m + n, p + q)} v^{-n} t^{-q - 1} + K_{(m, p), (r, s)}(x) + \xi_{(m, p), (r, s)}(x)
                                \\
                                & = ( (r - m)p - m(s - p) ) x v^{m - r} t^{p - s - 1} + K_{(m, p), (r, s)}(x) + \xi_{(m, p), (r, s)}(x)
                                \\
                                & = ( rp - ms ) x v^{m - r} t^{p - s - 1} + K_{(m, p), (r, s)}(x) + \xi_{(m, p), (r, s)}(x)
                            \end{aligned}
                        $$
                        
                    We now claim that:
                        $$\xi_{(m, p), (r, s)}(x) = 0$$
                    To this end, consider firstly the following, wherein $Z \in \z_{[2]}$ is an arbitrary choice:
                        $$
                            \begin{aligned}
                                ( [D_{r, s}, x v^m t^p]_{\hat{\g}_{[2]}}, Z )_{\hat{\g}_{[2]}} & = ( D_{r, s}, [x v^m t^p, Z]_{\tilde{\g}_{[2]}} )_{\hat{\g}_{[2]}}
                                \\
                                & = (D, 0)_{\hat{\g}_{[2]}}
                                \\
                                & = 0
                            \end{aligned}
                        $$
                    Simultaneously, consider the following:
                        $$
                            \begin{aligned}
                                ( [D_{r, s}, x v^m t^p]_{\hat{\g}_{[2]}}, Z )_{\hat{\g}_{[2]}} & = \left( \sum_{(a, b) \in \Z^2} \lambda_{a, b}(x) v^a t^b + K_{(m, p), (r, s)}(x) + \xi_{(m, p), (r, s)}(x), Z \right)_{\hat{\g}_{[2]}}
                                \\
                                & = ( \xi_{(m, p), (r, s)}(x), Z )_{\hat{\g}_{[2]}}
                            \end{aligned}
                        $$
                    The previous observation along with this one imply that:
                        $$( \xi_{(m, p), (r, s)}(x), Z )_{\hat{\g}_{[2]}} = 0$$
                    for \textit{any} $Z \in \z_{[2]}$, but since $\bar{\d}_{[2]}$ is graded-dual to $\z_{[2]}$ by construction, the above implies via the non-degeneracy of the inner product $(-, -)_{\hat{\g}_{[2]}}$ that:
                        $$\xi_{(m, p), (r, s)}(x) = 0$$
                    necessarily. 

                    We can now conclude that:
                        $$[D_{r, s}, x v^m t^p]_{\hat{\g}_{[2]}} = ( rp - ms ) x v^{m - r} t^{p - s - 1} + K_{(m, p), (r, s)}(x)$$
                    \item \textbf{(Case 2: $D := D_v$):} In this case, it is easy to see that:
                        $$
                            \begin{aligned}
                                ( D_v, [x v^m t^p, y v^n t^q]_{\tilde{\g}_{[2]}} )_{\hat{\g}_{[2]}} & = -(x, y)_{\g} ( D_v, \delta_{(m, p) + (n, q), (0, 0)} ( n c_v + q c_t ) + (np - mq) Z_{m + n, p + q} )_{\hat{\g}_{[2]}}
                                \\
                                & = -(x, y)_{\g} \delta_{(m, p) + (n, q), (0, 0)} n
                            \end{aligned}
                        $$
                    Using invariance, we then see that:
                        $$( [D_v, x v^m t^p]_{\hat{\g}_{[2]}}, y v^n t^q )_{\hat{\g}_{[2]}} = -(x, y)_{\g} \delta_{(m, p) + (n, q), (0, 0)} n$$
                    Now, suppose that:
                        $$[D_v, x v^m t^p]_{\hat{\g}_{[2]}} := \sum_{(a, b) \in \Z^2} \lambda_{a, b}(x) v^a t^b + K_{m, p}(x) + \xi_{m, p}(x)$$
                    for some $\lambda_{a, b}(x) \in \g$, $K_{m, p}(x) \in \z_{[2]}$, and $\xi_{m, p}(x) \in \bar{\d}_{[2]}$, depending on our choices of $x \in \g$ and $(m, p) \in \Z^2$. Then, consider the following:
                        $$
                            \begin{aligned}
                                ( [D_v, x v^m t^p]_{\hat{\g}_{[2]}}, y v^n t^q )_{\hat{\g}_{[2]}} & = \left( \sum_{(a, b) \in \Z^2} \lambda_{a, b}(x) v^a t^b + K_{m, p}(x) + \xi_{m, p}(x), y v^n t^q \right)_{\hat{\g}_{[2]}}
                                \\
                                & = \sum_{(a, b) \in \Z^2} \left( \lambda_{a, b}(x) v^a t^b, y v^n t^q \right)_{\g_{[2]}}
                                \\
                                & = \sum_{(a, b) \in \Z^2} (\lambda_{a, b}(x), y)_{\g} \delta_{ (a, b) + (n, q), (0, -1) }
                                \\
                                & = (\lambda_{-n, -q - 1}(x), y)_{\g}
                            \end{aligned}
                        $$
                    From this, we are able to conclude that:
                        $$-(x, y)_{\g} \delta_{(m, p) + (n, q), (0, 0)} n = (\lambda_{-n, -q - 1}(x), y)_{\g}$$
                    As this holds for all $y \in \g$ and all $(n, q) \in \Z^2$, we can infer from the above and from the non-degeneracy of the inner product $(-, -)_{\g}$ that:
                        $$\lambda_{-n, -q - 1}(x) = -\delta_{(m, p) + (n, q), (0, 0)} n x$$
                    for any $x \in \g$ and any $(m, p) \in \Z^2$ (both fixed!), and hence:
                        $$
                            \begin{aligned}
                                [D_v, x v^m t^p]_{\hat{\g}_{[2]}} & = \sum_{(n, q) \in \Z^2} -\delta_{(m, p) + (n, q), (0, 0)} n x v^{-n} t^{-q - 1} + K_{m, p}(x) + \xi_{m, p}(x)
                                \\
                                & = m x v^m t^{p - 1} + K_{m, p}(x) + \xi_{m, p}(x)
                            \end{aligned}
                        $$

                    Now, by arguing as in \textbf{Case 1}, we will see that:
                        $$\xi_{m, p}(x) = 0$$
                    and afterwards we will be able to conclude that:
                        $$[D_v, x v^m t^p]_{\hat{\g}_{[2]}} = m x v^m t^{p - 1} + K_{m, p}(x)$$
                    \item \textbf{(Case 3: $D := D_t$)} Arguing as when $D = D_v$, we will obtain:
                        $$[D_t, x v^m t^p]_{\hat{\g}_{[2]}} = p x v^m t^{p - 1} + K_{m, p}(x)$$
                    for some $K_{m, p}(x) \in \z_{[2]}$.
                \end{enumerate}
            \end{remark}
            \begin{remark}
                We see now also that $\t$ is a Lie algebra ideal of $\hat{\g}_{[2]}$ with respect to $[-, -]_{\hat{\g}_{[2]}}$.
            \end{remark}

            \begin{remark}[Lie bracket(s) on $\bar{\d}_{[2]}$] \label{remark: dual_of_toroidal_centres_contains_derivations}
                There is also a way to compute - using invariance - how the brackets of the form:
                    $$[D, D']_{\hat{\g}_{[2]}}$$
                should be given, for any $D, D' \in \bar{\d}_{[2]}$; we note that even when:
                    $$\hat{\g}_{[2]} \cong \tilde{\g}_{[2]} \rtimes \bar{\d}_{[2]}$$
                the Lie brackets of pairs of the form $[D, D']_{\hat{\g}_{[2]}}$ ought not be uniquely determined, which is ultimately due to the fact that we have no reason to expect that:
                    $$H^2_{\Lie}(\der_{\bbC}(A_{[2]}), \z_{[2]}) \cong 0$$
                    
                \textit{A priori}, one can suppose that:
                    $$[D, D']_{\hat{\g}_{[2]}} := X(v, t) + K(D, D') + \xi(D, D')$$
                for some $X(v, t) \in \g_{[2]}$, some $K(D, D') \in \z_{[2]}$, and some $\xi(D, D') \in \bar{\d}_{[2]}$. Then, consider the following for any arbitrary $Y(v, t) \in \g_{[2]}$, and without any loss of generality we can take $Y(v, t) := y g(v, t)$ for some $y \in \g$ and some $g(v, t) \in A_{[2]}$:
                    $$
                        \begin{aligned}
                            ( [D, D']_{\hat{\g}_{[2]}}, Y(v, t) )_{\hat{\g}_{[2]}} & = ( X(v, t) + K(D, D') + \xi(D, D'), Y(v, t) )_{\hat{\g}_{[2]}}
                            \\
                            & = (X(v, t), Y(v, t))_{\g_{[2]}}
                        \end{aligned}
                    $$
                At the same time, we have by invariance that:
                    $$
                        \begin{aligned}
                            ( [D, D']_{\hat{\g}_{[2]}}, Y(v, t) )_{\hat{\g}_{[2]}} & = ( D, [D', Y(v, t)]_{\hat{\g}_{[2]}} )_{\hat{\g}_{[2]}}
                            \\
                            & = ( D, [D', y g(v, t)]_{\hat{\g}_{[2]}} )_{\hat{\g}_{[2]}}
                            \\
                            & = ( D, y D'(g(v, t)) )_{\hat{\g}_{[2]}}
                            \\
                            & = 0
                        \end{aligned}
                    $$
                with the second-to-last equality coming from the computations in remark \ref{remark: derivation_action_on_multiloop_algebras} and the last being due to $y D'(g(v, t)) \in \g_{[2]}$. Since $Y(v, t) := y g(v, t)$ was chosen arbitrarily, the above implies that:
                    $$(X(v, t), Y(v, t))_{\hat{\g}_{[2]}} = 0$$
                and hence:
                    $$X(v, t) = 0$$
                meaning that:
                    $$[D, D']_{\hat{\g}_{[2]}} = K(D, D') + \xi(D, D') \in \z_{[2]} \oplus \bar{\d}_{[2]}$$
                for any $D, D' \in \bar{\d}_{[2]}$. 
                
                We now claim that, given how $\bar{\d}_{[2]}$ acts on $\g_{[2]}$ (cf. remark \ref{remark: derivation_action_on_multiloop_algebras}), it can then be shown that the elements of $\bar{\d}_{[2]}$ are in fact derivations on $A_{[2]}$ (which is perhaps to be expected, since elements of $\z_{[2]}$ are certain equivalence classes of differential $1$-forms on $A_{[2]}$).
                \begin{enumerate}
                    \item Firstly, consider the map:
                        $$D_{r, s}: A_{[2]} \to A_{[2]}$$
                    given for all $(m, p) \in \Z^2$ by:
                        $$D_{r, s}(v^m t^p) := ( rp - ms ) v^{m - r} t^{p - s - 1}$$
                    Then, consider the following wherein $(m, p), (n, q) \in \Z^2$ are arbitrary:
                        $$
                            \begin{aligned}
                                D_{r, s}(v^m t^p) v^n t^q + v^m t^p D_{r, s}(v^n t^q) & = ( rp - ms ) v^{m + n - r} t^{p + q - s - 1} + ( rq - ns ) v^{m + n - r} t^{p + q - s - 1}
                                \\
                                & = ( r(p + q) - (m + n)s ) v^{m + n - r} t^{p + q - s - 1}
                                \\
                                & = D_{r, s}( v^{m + n} t^{p + q} )
                            \end{aligned}
                        $$
                    This clearly shows that $D_{r, s}$ is a derivation on $A_{[2]}$.
                    \item Next, consider the map:
                        $$D_v: A_{[2]} \to A_{[2]}$$
                    given for all $(m, p) \in \Z^2$ by:
                        $$D_v(v^m t^p) := m v^m t^{p - 1}$$
                    Then, consider the following wherein $(m, p), (n, q) \in \Z^2$ are arbitrary:
                        $$
                            \begin{aligned}
                                D_v(v^m t^p) v^n t^q + v^m t^p D_v(v^n t^q) & = (m + n) v^{m + n} t^{p + q - 1}
                                \\
                                & = D_v( v^{m + n} t^{p + q} )
                            \end{aligned}
                        $$
                    This clearly shows that $D_v$ is a derivation on $A_{[2]}$.
                    \item Likewise, we can show that the map:
                        $$D_t: A_{[2]} \to A_{[2]}$$
                    given for all $(m, p) \in \Z^2$ by:
                        $$D_t(v^m t^p) := p v^m t^{p - 1}$$
                    is a derivation on $A_{[2]}$.
                \end{enumerate}

                We can also identify the derivations $D_{r, s}, D_v, D_t$ explicitly in terms of $\del_v := \frac{\del}{\del v}, \del_t := \frac{\del}{\del t}$. For this, let us firstly equip:
                    $$\d_{[2]} := \der_{\bbC}(A_{[2]})$$
                - the $\bbC$-vector space of all $\bbC$-linear derivations on $A_{[2]}$ - with the following basis:
                    $$\{ v^m t^p \del_v, v^n t^q \del_t \}_{(m, p), (n, q) \in \Z^2}$$
                \begin{enumerate}
                    \item To compute $D_{r, s}$ in terms of $\del_v, \del_t$, suppose firstly that:
                        $$D_{r, s} := f(v, t) \del_v + g(v, t) \del_t$$
                    with $f(v, t), g(v, t) \in A_{[2]}$. Next, fix some $(m, p) \in \Z^2$ and then consider the following:
                        $$
                            \begin{aligned}
                                D_{r, s}( v^m t^p ) & = f(v, t) \del_v( v^m t^p ) + g(v, t) \del_t( v^m t^p )
                                \\
                                & = f(v, t) m v^{m - 1} t^p + g(v, t) p v^m t^{p - 1}
                            \end{aligned}
                        $$
                    At the same time, we also have that:
                        $$D_{r, s}(v^m t^p) := ( rp - ms ) v^{m - r} t^{p - s - 1}$$
                    and hence:
                        $$f(v, t) m v^{m - 1} t^p + g(v, t) p v^m t^{p - 1} = ( rp - ms ) v^{m - r} t^{p - s - 1}$$
                    From this, one infers that:
                        $$f(v, t) = -s v^{-r + 1} t^{-s - 1}, g(v, t) = r v^{-r} t^{-s}$$
                    and therefore:
                        $$D_{r, s} = -s v^{-r + 1} t^{-s - 1} \del_v + r v^{-r} t^{-s} \del_t$$
                    \item One easily checks that:
                        $$D_v = v t^{-1} \del_v$$
                    \item Likewise:
                        $$D_t = \del_t$$
                \end{enumerate}

                Now that we know that the basis elements $D_{r, s}, D_v, D_t \in \bar{\d}_{[2]}$ are actually certain derivations on $A_{[2]}$, we can also check if $\bar{\d}_{[2]}$ is in fact a Lie subalgebra of $\d_{[2]}$ (equipped with the usual commutator Lie bracket) by checking whether or not commutators of the elements $D_{r, s}, D_v, D_t$ are still elements of $\bar{\d}_{[2]}$. Doing so gives one Lie bracket on $\bar{\d}_{[2]}$, namely the one such that:
                    $$[D, D']_{\hat{\g}_{[2]}} = \xi(D, D') \in \bar{\d}_{[2]}$$
                for any $D, D' \in \bar{\d}_{[2]}$, with $\xi(D, D')$ being as above. One sees then that the freedom lies in the element:
                    $$K(D, D') \in \z_{[2]}$$
                mentioned earlier. 
            \end{remark}
            \begin{convention}
                From now on, we will be working with the usual commutator bracket on $\bar{\d}_{[2]}$. 
            \end{convention}
            
            \begin{remark}[How does $\bar{\d}_{[2]}$ act on $\z_{[2]}$ ?] \label{remark: derivation_action_on_toroidal_centres}
                We can now use what we know about how $\bar{\d}_{[2]}$ acts on $\g_{[2]}$ in conjunction with the Jacobi identity in order to compute Lie brackets of the form:
                    $$[D, Z]_{\hat{\g}_{[2]}}$$
                for any $D \in \bar{\d}_{[2]}$ and any $Z \in \z_{[2]}$. For the computations that follow, \textit{we will need to assume that}:
                    $$\hat{\g}_{[2]} \cong \tilde{\g}_{[2]}\rtimes \bar{\d}_{[2]}$$
                in which case it can be easily shown that:
                    $$[\bar{\d}_{[2]}, \g_{[2]}]_{\hat{\g}_{[2]}} \subseteq \g_{[2]}$$
                Different methods will have to be employed in the absence of this assumption.

                The idea here is to use the fact that:
                    $$[-, -]_{\tilde{\g}_{[2]}} = [-, -]_{\g_{[2]}} + \e$$
                wherein $\e: \bigwedge^2 \g_{[2]} \to \z_{[2]}$ is given by:
                    $$\e(x v^m t^p, y v^n t^q) := -v^m t^p \bar{d}(v^n t^q)$$
                for any $x, y \in \g$ and any $(m, p), (n, q) \in \Z^2$. With this in mind, consider the following for any $D \in \bar{\d}_{[2]}$, which holds thanks to the Jacobi identity:
                    $$
                        \begin{aligned}
                            [ D, [x v^m t^p, y v^n t^q]_{\tilde{\g}_{[2]}} ]_{\hat{\g}_{[2]}} & = [ [D, x v^m t^p]_{\hat{\g}_{[2]}}, y v^n t^q ]_{\tilde{\g}_{[2]}} + [ x v^m t^p, [D, y v^n t^q]_{\hat{\g}_{[2]}} ]_{\tilde{\g}_{[2]}}
                            \\
                            & = [x D(v^m t^p), y v^n t^q]_{\hat{\g}_{[2]}} + [x v^m t^p, y D(v^n t^q)]_{\hat{\g}_{[2]}}
                            \\
                            & = [x, y]_{\g} (D(v^m t^p) v^n t^q + v^m t^p D(v^n t^q)) - (x, y)_{\g} ( D(v^m t^p) \bar{d}(v^n t^q) + v^m t^p \bar{d}(D(v^n t^q)) )
                            \\
                            & = [x, y]_{\g} D(v^{m + n} t^{p + q}) - (x, y)_{\g} ( D(v^m t^p) \bar{d}(v^n t^q) + v^m t^p \bar{d}(D(v^n t^q)) )
                        \end{aligned}
                    $$
                Note that for the second equality, we implicitly invoked the fact that the basis elements of $\bar{\d}_{[2]}$ (and hence all elements thereof) act on $\g_{[2]}$ in the following manner:
                    $$[D, x v^m t^p] = x D(v^m t^p), D \in \{D_{r, s}\}_{(r, s) \in \Z^2} \cup \{D_v, D_t\}$$
                (cf. remark \ref{remark: derivation_action_on_multiloop_algebras}), and it makes sense to write this because we now know (after remark \ref{remark: dual_of_toroidal_centres_contains_derivations}) that elements of $\bar{\d}_{[2]}$ are certain derivations on $A_{[2]}$. At the same time, we have that:
                    $$
                        \begin{aligned}
                            [ D, [x v^m t^p, y v^n t^q]_{\tilde{\g}_{[2]}} ]_{\hat{\g}_{[2]}} & = [ D, [x, y]_{\g} v^{m + n} t^{p + q} - (x, y)_{\g} v^m t^p \bar{d}(v^n t^q) ]_{\hat{\g}_{[2]}}
                            \\
                            & = [x, y]_{\g} D(v^{m + n} t^{p + q}) - (x, y)_{\g} [D, v^m t^p \bar{d}(v^n t^q)]_{\hat{\g}_{[2]}}
                        \end{aligned}
                    $$
                By putting the two computations together, one yields:
                    $$[D, v^m t^p \bar{d}(v^n t^q)]_{\hat{\g}_{[2]}} = D(v^m t^p) \bar{d}(v^n t^q) + v^m t^p \bar{d}(D(v^n t^q))$$
                Since we know how the basis elements of $\bar{\d}_{[2]}$ act on $\g_{[2]}$ (see remark \ref{remark: derivation_action_on_multiloop_algebras}), the above is enough to determine how $\bar{\d}_{[2]}$ acts on $\z_{[2]}$. 

                Now, let us \textit{not} assume that:
                    $$\hat{\g}_{[2]} \cong \tilde{\g}_{[2]}\rtimes \bar{\d}_{[2]}$$
                Without any loss of generality, let us consider the following for any $h, h' \in \h$ so that\footnote{We can make this assumption because ultimately, elements of $\z_{[2]}$ do not depend on those of $\g$.}:
                    $$(h, h')_{\g} = -1$$
                any $f(v, t), g(v, t) \in A_{[2]}$, and any $D \in \bar{\d}_{[2]}$:
                    $$[ D, [h f(v, t), h' g(v, t)]_{\tilde{\g}_{[2]}} ]_{\hat{\g}_{[2]}} = [ D, f(v, t) \bar{d}( g(v, t) ) ]_{\hat{\g}_{[2]}}$$
                At the same time, we have via the Jacobi identity that:
                    $$
                        \begin{aligned}
                            [ D, [h f(v, t), h' g(v, t)]_{\tilde{\g}_{[2]}} ]_{\hat{\g}_{[2]}} & = [ h f(v, t), [D, h' g(v, t)]_{\hat{\g}_{[2]}} ]_{\tilde{\g}_{[2]}} + [ [D, h f(v, t)]_{\hat{\g}_{[2]}}, h' g(v, t) ]_{\tilde{\g}_{[2]}}
                            \\
                            & = [ h f(v, t), h' D( g(v, t) ) ]_{\tilde{\g}_{[2]}} + [ h D( f(v, t) ), h' g(v, t) ]_{\tilde{\g}_{[2]}}
                            \\
                            & = f(v, t) \bar{d}( D( g(v, t) ) ) + D( f(v, t) ) \bar{d}(g(v, t))
                        \end{aligned}
                    $$
                One thus sees that:
                    $$[ D, f(v, t) \bar{d}( g(v, t) ) ]_{\hat{\g}_{[2]}} = f(v, t) \bar{d}( D( g(v, t) ) ) + D( f(v, t) ) \bar{d}(g(v, t))$$
                and since the element $f(v, t) \bar{d}( g(v, t) )$ is central (via the map $\e$ mentioned earlier), this gives another description of:
                    $$[ \bar{\d}_{[2]}, \z_{[2]} ]_{\hat{\g}_{[2]}}$$
                With this in mind, we return quickly to remark \ref{remark: derivation_action_on_multiloop_algebras}; there, we previously demonstrated that:
                    $$[ \bar{\d}_{[2]}, \g_{[2]} ]_{\hat{\g}_{[2]}} \subseteq \g_{[2]} \oplus \z_{[2]}$$
                but we claim now that the following stronger fact holds:
                    $$[ \bar{\d}_{[2]}, \g_{[2]} ]_{\hat{\g}_{[2]}} \subseteq \g_{[2]}$$
                To see why this is the case, suppose firstly that for any $D \in \bar{\d}_{[2]}$, any $X(v, t) \in \g_{[2]}$, there is $K(X) \in \z_{[2]}$ depending on $X := X(v, t)$ (and indeed, such a $K$ exists by remark \ref{remark: derivation_action_on_multiloop_algebras}) such that:
                    $$[ D, X(v, t) ]_{\hat{\g}_{[2]}} = D( X(v, t) ) + K(X)$$
                Next, pick an arbitrary element $\xi \in \bar{\d}_{[2]}$ and then consider the following:
                    $$( [ D, X(v, t) ]_{\hat{\g}_{[2]}}, \xi )_{\hat{\g}_{[2]}} = (D( X(v, t) ) + K(X), \xi)_{\hat{\g}_{[2]}} = (K(X), \xi)_{\hat{\g}_{[2]}}$$
                At the same time, using invariance yields us:
                    $$( [ D, X(v, t) ]_{\hat{\g}_{[2]}}, \xi )_{\hat{\g}_{[2]}} = (X(v, t), [\xi, D]_{\hat{\g}_{[2]}})_{\hat{\g}_{[2]}} = 0$$
                wherein the last equality is due to the fact that:
                    $$[\xi, D]_{\hat{\g}_{[2]}} \in \bar{\d}_{[2]}$$
                (cf. remark \ref{remark: dual_of_toroidal_centres_contains_derivations}). We thus see that:
                    $$(K(X), \xi)_{\hat{\g}_{[2]}} = 0$$
                for every $\xi \in \bar{\d}_{[2]}$, which then implies via the non-degeneracy of $(-, -)_{\hat{\g}_{[2]}}$ that:
                    $$K(X) = 0$$
                As such:
                    $$[ \bar{\d}_{[2]}, \g_{[2]} ]_{\hat{\g}_{[2]}} \subseteq \g_{[2]}$$
                as claimed. 
            \end{remark}

            Putting everything together then yields us the following result:
            \begin{proposition}[Extended toroidal Lie algebras] \label{prop: extended_toroidal_lie_algebras}
                Let $\bar{\d}_{[2]}$ - viewed as a Lie subalgebra of $\d_{[2]}$ (cf. remark \ref{remark: dual_of_toroidal_centres_contains_derivations}) - act on $\g_{[2]}$ and on $\z_{[2]}$ (and hence on $\tilde{\g}_{[2]}\cong \g_{[2]} \oplus \z_{[2]}$) as in remarks \ref{remark: derivation_action_on_multiloop_algebras} and \ref{remark: derivation_action_on_toroidal_centres} respectively. These actions give rise to a Lie algebra extension:
                    $$\hat{\g}_{[2]} := \tilde{\g}_{[2]}\rtimes \bar{\d}_{[2]}$$
                Furthermore, there is an invariant non-degenerate symmetric $\bbC$-bilinear form $(-, -)_{\hat{\g}_{[2]}}$ as in convention \ref{conv: orthogonal_complement_of_toroidal_centres}. 
            \end{proposition}

            \begin{convention}
                Let us now adopt the following notations:
                \begin{itemize}
                    \item $\bar{\d}_{[2]}^+ := ( \bigoplus_{(r, s) \in \Z \x \Z_{> 0} } \bbC D_{r, s} ) \oplus \bbC D_v$ and $\bar{\d}_{[2]}^- := ( \bigoplus_{(r, s) \in \Z \x \Z_{\leq 0} } \bbC D_{r, s} ) \oplus \bbC D_t$ shall be the Lie subalgebras of $\bar{\d}_{[2]}$ which are graded-dual to $\z_{[2]}^{\pm}$ with respect to $(-, -)_{\hat{\g}_{[2]}}$;
                    \item $\hat{\g}_{[2]}^{\pm} := \tilde{\g}_{[2]}^{\pm} \rtimes \bar{\d}_{[2]}^{\pm}$.
                \end{itemize}
            \end{convention}    
            \begin{theorem} \label{theorem: extended_toroidal_manin_triples}
                There is a complete topological Manin triple:
                    $$(\hat{\g}_{[2]}, \hat{\g}_{[2]}^+, \hat{\g}_{[2]}^-)$$
                wherein $\hat{\g}_{[2]}$ is equipped with the non-degenerate invariant inner product $(-, -)_{\hat{\g}_{[2]}}$ (cf. convention \ref{conv: orthogonal_complement_of_toroidal_centres}).
            \end{theorem}
                \begin{proof}
                    The only thing to check is that:
                        $$(\hat{\g}_{[2]}^{\pm}, \hat{\g}_{[2]}^{\pm})_{\hat{\g}_{[2]}} = 0$$
                    but this is a direct consequence of the assumptions on $(-, -)_{\hat{\g}_{[2]}}$ from convention \ref{conv: orthogonal_complement_of_toroidal_centres}. 
                \end{proof}
            \begin{corollary}[Lie cobracket on $\hat{\g}_{[2]}^+$] \label{coro: extended_toroidal_lie_bialgebras}
                On the extended toroidal Lie algebra $\hat{\g}_{[2]}^+$, there is a continuous Lie cobracket\footnote{Note the completion!}, making $\hat{\g}_{[2]}^+$ a complete topological Lie bialgebra:
                    $$\delta_{\hat{\g}_{[2]}^+}: \hat{\g}_{[2]}^+ \to \hat{\g}_{[2]}^+ \hattensor_{\bbC} \hat{\g}_{[2]}^+$$
                given for any $X \in \hat{\g}_{[2]}^+$ by the following formula (cf. \cite{etingof_kazhdan_quantisation_1}):
                    $$\delta_{\hat{\g}_{[2]}^+}(X) = [ X \tensor 1 + 1 \tensor X, \sfr_{\hat{\g}_{[2]}^+} ]$$
                wherein:
                    $$\sfr_{\hat{\g}_{[2]}^+} := \sfr_{\g_{[2]}^+} + \sfr_{\z_{[2]}^+} + \sfr_{\bar{\d}_{[2]}^+} \in \hat{\g}_{[2]}^+ \hattensor_{\bbC} \hat{\g}_{[2]}^-$$
                with $\sfr_{\g_{[2]}^+} \in \g_{[2]}^+ \hattensor_{\bbC} \g_{[2]}^-$ being as in question \ref{question: multiloop_lie_bialgebras} and\footnote{Note how we are simply summing over tensor products of dual basis elements.} $\sfr_{\z_{[2]}^+} \in \z_{[2]}^+ \hattensor_{\bbC} \bar{\d}_{[2]}^+$ and $\sfr_{\bar{\d}_{[2]}^+} \in \d_{[2]}^+ \hattensor_{\bbC} \z_{[2]}^+$ being given by the following formulae:
                    $$\sfr_{\z_{[2]}^+} := \sum_{(r, s) \in \Z \x \Z_{> 0}} Z_{r, s} \tensor D_{r, s} + c_v \tensor D_v$$
                    $$\sfr_{\bar{\d}_{[2]}^+} := \sum_{(r, s) \in \Z \x \Z_{> 0}} D_{r, s} \tensor Z_{r, s} + D_v \tensor c_v$$
            \end{corollary}
            
            \begin{remark}[Total degrees of classical Yangian R-matrices] \label{remark: total_degrees_of_classical_yangian_R_matrices}
                One property of the R-matrix $\sfr_{\hat{\g}_{[2]}^+}$ from corollary \ref{coro: extended_toroidal_lie_bialgebras} that will help simplify some computations later on (see the proof of lemma \ref{lemma: toroidal_lie_bialgebras}) is that they are of total degree $-1$. 

                Recall that if $V := \bigoplus_{m \in \Z} V_m, W := \bigoplus_{n \in \Z} W_n$ are $\Z$-graded vector spaces then for any $k \in \Z$, we have that:
                    $$(V \tensor_{\bbC} W)_k \cong \bigoplus_{m + n = k} V_m \tensor_{\bbC} W_n$$
                If we now take $V = W = \rmU(\hat{\g}_{[2]})$ then the claim from above would read:
                    $$\sfr_{\hat{\g}_{[2]}^+} \in ( \rmU(\hat{\g}_{[2]}^+) \tensor_{\bbC} \rmU(\hat{\g}_{[2]}^-) )_{-1}$$
                with the $\Z$-grading on $\hat{\g}_{[2]}^{\pm}$ (and hence on $\rmU(\hat{\g}_{[2]}^{\pm})$) being the one on the second variable $t$, and actually, this is entirely due to:
                    $$\sfr_{\g_{[2]}^+} \in ( \rmU(\g_{[2]}^+) \tensor_{\bbC} \rmU(\g_{[2]}^-) )_{-1}$$
                and this can already be inferred from the computations done in question \ref{question: multiloop_lie_bialgebras}.

                What this means for us is that, should we have $X \in \hat{\g}_{[2]}^+$ such that:
                    $$\deg X \leq 0$$
                then it will automatically be the case that:
                    $$\delta_{\hat{\g}_{[2]}^+}(X) = 0$$
            \end{remark}

            \begin{remark}[Lie cobracket on $\hat{\g}_{[2]}^+$, explicitly] \label{remark: extended_toroidal_lie_bialgebras_explicit_formulae}
                Let us see how the Lie cobracket $\delta_{\hat{\g}_{[2]}^+}$ from corollary \ref{coro: extended_toroidal_lie_bialgebras} is given explicitly on basis elements of $\hat{\g}_{[2]}^+$.
                
                \begin{enumerate}
                    \item For the sake of convenience, let us firstly recall from question \ref{question: multiloop_lie_bialgebras} that the canonical element $\sfr_{\g_{[2]}^+}$ is given by:
                        $$\sfr_{\g_{[2]}^+} = \sfr_{\g} v_2\1(v_1, v_2) \1^+(t_1, t_2)$$
                    with notations as in \textit{loc. cit.}. As in convention \ref{conv: a_fixed_finite_dimensional_simple_lie_algebra}, we fix a basis $\{x_i\}_{1 \leq i \leq \dim_{\bbC} \g}$ along with a dual basis $\{x_i^*\}_{1 \leq i \leq \dim_{\bbC} \g}$ with respect to $(-, -)_{\g}$, with respect to which we can write:
                        $$\sfr_{\g} = \sum_{1 \leq i \leq \dim_{\bbC} \g} x_i \tensor x_i^*$$
                    \item We have that:
                        $$\sfr_{\z_{[2]}^+} = \sum_{(r, s) \in \Z \x \Z_{> 0}} Z_{r, s} \tensor D_{r, s} + c_{v_1} \tensor D_{v_2}$$
                    \item Likewise, we have that:
                        $$\sfr_{\bar{\d}_{[2]}^+} = \sum_{(r, s) \in \Z \x \Z_{> 0}} D_{r, s} \tensor Z_{r, s} + D_{v_1} \tensor c_{v_2}$$
                \end{enumerate}

                Now, we compute the Lie cobrackets. For convenience, let us abbreviate:
                    $$\hat{\delta}^+ := \delta_{\hat{\g}_{[2]}^+}$$
                \begin{enumerate}
                    \item Let us firstly see how the elements $\hat{\delta}^+(x v^m t^p)$, for some basis element $x \in \g$ and for some $(m, p) \in \Z \x \Z_{\geq 0}$, are given. Note that while it might seem that:
                        $$\hat{\delta}^+(x v^m t^p) = \delta^+(x v^m t^p)$$
                    with $\delta^+ := \delta_{\g_{[2]}^+}$ being given as in question \ref{question: multiloop_lie_bialgebras}, this actually should not be true, since the Lie brackets in $\tilde{\g}_{[2]}^+$ between elements of $\g_{[2]}$ actually do not lie within $\g_{[2]}$ in general, as such brackets usually would contain a central summand as well (cf. remark \ref{remark: dual_of_toroidal_centres_contains_derivations}).

                    Let us npw fix some pure tensor:
                        $$Y := y v^a t^b \in \g_{[2]}^+$$
                    (i.e. $x \in \g, (m, p) \in \Z \x \Z_{\geq 0}$) and attempt to compute:
                        $$[Y_1 \tensor 1 + 1 \tensor Y_2, \sfr], \sfr \in \{ \sfr_{\g_{[2]}^+}, \sfr_{\z_{[2]}^+}, \sfr_{\bar{\d}_{[2]}^+} \}$$
                    wherein we have formally distinguished:
                        $$Y_i := y_i v_i^a t_i^b, i \in \{1, 2\}$$
                    The sum of the terms $[Y_1 \tensor 1 + 1 \tensor Y_2, \sfr]$ will be the value of $\hat{\delta}^+(X)$. 
                    \begin{enumerate}
                        \item Firstly, we compute:
                            $$[Y_1 \tensor 1 + 1 \tensor Y_2, \sfr_{\g_{[2]}^+}]$$
                            
                        To this end, consider firstly the following:
                            $$
                                \begin{aligned}
                                    [Y_1 \tensor 1, \sfr_{\g_{[2]}^+}] & = [Y_1 \tensor 1, \sfr_{\g} v_2 \1(v_1, v_2) \1^+(t_1, t_2)]
                                    \\
                                    & = \sum_{1 \leq i \leq \dim_{\bbC} \g} \sum_{(m, p) \in \Z \x \Z_{\geq 0}} [y_1 v_1^a t_1^b \tensor 1, (x_i \tensor x_i^*) v_1^m v_2^{-m} t_1^p t_2^{-p - 1}] 
                                    \\
                                    & = \sum_{1 \leq i \leq \dim_{\bbC} \g} \sum_{(m, p) \in \Z \x \Z_{\geq 0}} [y_1 v_1^a t_1^b, x_i v_1^m t_1^p] \tensor x_i^* v_2^{-m} t_2^{-p - 1} 
                                    \\
                                    & = \sum_{1 \leq i \leq \dim_{\bbC} \g} \sum_{(m, p) \in \Z \x \Z_{\geq 0}} \left( [y_1, x_i] v_1^m t_1^p - (y_1, x_i)_{\g} \bar{d}(v_1^m t_1^p) \right) v_1^a t_1^b \tensor x_i^* v_2^{-m} t_2^{-p - 1} 
                                    \\
                                    & = 
                                    \begin{aligned}
                                        & [y_1 \tensor 1, \sfr_{\g}] v_2 \1(v_1, v_2) \1^+(t_1, t_2)
                                        \\
                                        - & \sum_{1 \leq i \leq \dim_{\bbC} \g} (y_1, x_i)_{\g} \sum_{(m, p) \in \Z \x \Z_{\geq 0}} v_1^a t_1^b \bar{d}(v_1^m t_1^p) \tensor x_i^* v_2^{-m} t_2^{-p - 1} 
                                    \end{aligned}
                                \end{aligned}
                            $$
                        We know from remark \ref{remark: centres_of_dual_toroidal_lie_bialgebras} that:
                            $$v_1^a t_1^b \bar{d}(v_1^m t_1^p) = v_1^m t_1^p \bar{d}(v_1^a t_1^b) = \delta_{(m, p) + (a, b), (0, 0)} ( a c_{v_1} + b c_{t_1} ) + (ap - mb) Z_{m + n, a + b}$$
                        and so:
                            $$[Y_1 \tensor 1, \sfr_{\g_{[2]}^+}] = [y_1 \tensor 1, \sfr_{\g}] v_2 \1(v_1, v_2) \1^+(t_1, t_2) - \sum_{1 \leq i \leq \dim_{\bbC} \g} (y_1, x_i)_{\g} ( a c_{v_1} + b c_{t_1} ) \tensor x_i^* v_2^{-a} t_2^{-b - 1}$$
                            
                        Likewise, we have that:
                            $$[1 \tensor Y_2, \sfr_{\g_{[2]}^+}] = [1 \tensor y_2, \sfr_{\g}] v_2 \1(v_1, v_2) \1^+(t_1, t_2) - \sum_{1 \leq i \leq \dim_{\bbC} \g} x_i v_1^a t_1^b \tensor (y_2, x_i^*)_{\g} ( a c_{v_2} + b c_{t_2} )$$

                        Putting the two computations together yields:
                            $$[Y_1 \tensor 1 + 1 \tensor Y_2, \sfr_{\g_{[2]}^+}] =
                            \begin{aligned}
                                & [y_1 \tensor 1 + 1 \tensor y_2, \sfr_{\g}] v_2 \1(v_1, v_2) \1^+(t_1, t_2)
                                \\
                                - & \sum_{1 \leq i \leq \dim_{\bbC} \g} \left( (y_1, x_i)_{\g} ( a c_{v_1} + b c_{t_1} ) \tensor x_i^* v_2^{-a} t_2^{-b - 1} + x_i v_1^a t_1^b \tensor (y_2, x_i^*)_{\g} ( a c_{v_2} + b c_{t_2} ) \right)
                            \end{aligned}
                            $$
                        \item Next, let us compute:
                            $$[Y_1 \tensor 1 + 1 \tensor Y_2, \sfr_{\z_{[2]}^+}]$$
                            
                        Again, we consider firstly the following:
                            $$
                                \begin{aligned}
                                    [Y_1 \tensor 1, \sfr_{\z_{[2]}^+}] & = \sum_{(r, s) \in \Z \x \Z_{> 0}} [Y_1 \tensor 1, Z_{r, s} \tensor D_{r, s} + c_{v_1} \tensor D_{v_2}]
                                    \\
                                    & = \sum_{(r, s) \in \Z \x \Z_{> 0}} [Y_1, Z_{r, s}] \tensor D_{r, s} + [Y_1, c_{v_1}] \tensor D_{v_2}
                                \end{aligned}
                            $$
                        Since $\hat{\g}_{[2]}^+ \cong \tilde{\g}_{[2]}^+ \rtimes \bar{\d}_{[2]}^+$, which in particular means that $\tilde{\g}_{[2]}^+$ is a Lie ideal of $\hat{\g}_{[2]}^+$, we have that:
                            $$[Y_1, Z_{r, s}] = [Y_1, c_{v_1}] = 0$$
                        and hence:
                            $$[Y_1 \tensor 1, \sfr_{\z_{[2]}^+}] = 0$$

                        In contrast, the term $[1 \tensor Y_2, \sfr_{\z_{[2]}^+}]$ does not necessarily vanish. To see why, consider the following:
                            $$
                                \begin{aligned}
                                    [1 \tensor Y_2, \sfr_{\z_{[2]}^+}] & = \sum_{(r, s) \in \Z \x \Z_{> 0}} [1 \tensor Y_2, Z_{r, s} \tensor D_{r, s} + c_{v_1} \tensor D_{v_2}]
                                    \\
                                    & = \sum_{(r, s) \in \Z \x \Z_{> 0}} Z_{r, s} \tensor [Y_2, D_{r, s}] + c_{v_1} \tensor [Y_2, D_{v_2}]
                                    \\
                                    & = \sum_{(r, s) \in \Z \x \Z_{> 0}} Z_{r, s} \tensor y_2 D_{r, s}(v_2^a t_2^b) + c_{v_1} \tensor y_2 D_{v_2}(v_2^a t_2^b)
                                \end{aligned}
                            $$

                        We thus gather that:
                            $$[Y_1 \tensor 1 + 1 \tensor Y_2, \sfr_{\z_{[2]}^+}] = \sum_{(r, s) \in \Z \x \Z_{> 0}} Z_{r, s} \tensor y_2 D_{r, s}(v_2^a t_2^b) + c_{v_1} \tensor y_2 D_{v_2}(v_2^a t_2^b)$$
                        For more details on how elements of $\bar{\d}_{[2]}^+$ act on those of $\g_{[2]}^+$, see remark \ref{remark: dual_of_toroidal_centres_contains_derivations}; in particular, recall that:
                            $$D_{r, s} = -s v^{-r + 1} t^{-s - 1} \del_v + r v^{-r} t^{-s} \del_t$$
                            $$D_v = v t^{-1} \del_v$$
                        \item Similarly to the previous case, we can calculate:
                            $$[Y_1 \tensor 1 + 1 \tensor Y_2, \sfr_{\bar{\d}_{[2]}^+}] = \sum_{(r, s) \in \Z \x \Z_{> 0}} y_1 D_{r, s}(v_1^a t_1^b) \tensor Z_{r, s} + y_1 D_{v_1}(v_1^a t_1^b) \tensor c_{v_2}$$
                    \end{enumerate}
                    Adding the three summands together yields:
                        $$
                            \begin{aligned}
                                \hat{\delta}^+(Y) & = [Y_1 \tensor 1 + 1 \tensor Y_2, \sfr_{ \hat{\g}_{[2]}^+ }]
                                \\
                                & = 
                                \begin{aligned}
                                    & \left( \begin{aligned}
                                        & [y_1 \tensor 1 + 1 \tensor y_2, \sfr_{\g}] v_2 \1(v_1, v_2) \1^+(t_1, t_2)
                                        \\
                                        - & \sum_{1 \leq i \leq \dim_{\bbC} \g} \left( (y_1, x_i)_{\g} ( a c_{v_1} + b c_{t_1} ) \tensor x_i^* v_2^{-a} t_2^{-b - 1} + x_i v_1^a t_1^b \tensor (y_2, x_i^*)_{\g} ( a c_{v_2} + b c_{t_2} ) \right)
                                    \end{aligned}
                                    \right)
                                    \\
                                    + & \sum_{(r, s) \in \Z \x \Z_{> 0}} \left( Z_{r, s} \tensor y_2 D_{r, s}(v_2^a t_2^b) + y_1 D_{r, s}(v_1^a t_1^b) \tensor Z_{r, s} \right)
                                    \\
                                    + & c_{v_1} \tensor y_2 D_{v_2}(v_2^a t_2^b) + y_1 D_{v_1}(v_1^a t_1^b) \tensor c_{v_2}
                                \end{aligned}
                            \end{aligned}
                        $$

                    \item Next, we compute how $\hat{\delta}^+$ acts on $\z_{[2]}^+$. 
                    \begin{enumerate}
                        \item Firstly, let us compute:
                            $$[Z_{r_1, s_1} \tensor 1 + 1 \tensor Z_{r_2}, \sfr_{\hat{\g}_{[2]}^+}]$$
                        \begin{enumerate}
                            \item Consider firstly the following:
                                $$[Z_{r_1, s_1} \tensor 1 + 1 \tensor Z_{r_2, s_2}, \sfr_{\g_{[2]}^+}] = [Z_{r_1, s_1} \tensor 1 + 1 \tensor Z_{r_2, s_2}, \sfr_{\g} \1(v_1, v_2) \1^+(t_1, t_2)] = 0$$
                            \item Secondly, consider the following:
                                $$[Z_{r_1, s_1} \tensor 1 + 1 \tensor Z_{r_2, s_2}, \sfr_{\z_{[2]}^+}] = $$
                            \item Thirdly, consider:
                                $$[Z_{r_1, s_1} \tensor 1 + 1 \tensor Z_{r_2, s_2}, \sfr_{\bar{\d}_{[2]}^+}]$$
                        \end{enumerate}
                        
                        \item Secondly, we compute:
                            $$[c_{v_1} \tensor 1 + 1 \tensor c_{v_2}, \sfr_{\hat{\g}_{[2]}^+}]$$
                        \begin{enumerate}
                            \item Consider firstly the following:
                                $$[c_{v_1} \tensor 1 + 1 \tensor c_{v_2}, \sfr_{\g_{[2]}^+}] = [c_{v_1} \tensor 1 + 1 \tensor c_{v_2}, \sfr_{\g} \1(v_1, v_2) \1^+(t_1, t_2)] = 0$$
                            \item Secondly, to compute:
                                $$[c_{v_1} \tensor 1 + 1 \tensor c_{v_2}, \sfr_{\z_{[2]}^+}]$$
                            let us firstly consider 
                                $$
                                    \begin{aligned}
                                         [c_{v_1} \tensor 1, \sfr_{\z_{[2]}^+}] & = \sum_{(r, s) \in \Z \x \Z_{> 0}} [c_{v_1} \tensor 1, Z_{r, s} \tensor D_{r, s} + c_{v_1} \tensor D_{v_2}]
                                         \\
                                         & = \sum_{(r, s) \in \Z \x \Z_{> 0}} [c_v, Z_{r, s}] \tensor D_{r, s} + [c_{v_1}, c_{v_2}] \tensor D_{v_2}
                                         \\
                                         & = \sum_{(r, s) \in \Z \x \Z_{> 0}} 0 \tensor D_{r, s} + 0 \tensor D_{v_2}
                                         \\
                                         & = 0
                                    \end{aligned}
                                $$
                            In contrast, we have that:
                                $$
                                    \begin{aligned}
                                         [1 \tensor c_{v_2}, \sfr_{\z_{[2]}^+}] & = \sum_{(r, s) \in \Z \x \Z_{> 0}} [1 \tensor c_{v_2}, Z_{r, s} \tensor D_{r, s} + c_{v_1} \tensor D_{v_2}]
                                         \\
                                         & = \sum_{(r, s) \in \Z \x \Z_{> 0}} Z_{r, s} \tensor [c_{v_1}, D_{r, s}] + c_{v_1} \tensor [c_{v_2}, D_{v_2}]
                                    \end{aligned}
                                $$
                            Now, recall from remark \ref{remark: centres_of_dual_toroidal_lie_bialgebras} and from remark \ref{remark: dual_of_toroidal_centres_contains_derivations} that:
                                $$c_v = v^{-1} \bar{d}(v)$$
                                $$D_{r, s} = -s v^{-r + 1} t^{-s - 1} \del_v + r v^{-r} t^{-s} \del_t, D_v = vt^{-1} \del_v$$
                            and then from remark \ref{remark: derivation_action_on_toroidal_centres} that $\bar{\d}_{[2]}$ acts on $\z_{[2]}$ in the following manner, given for all $D \in \bar{\d}_{[2]}$ and all $f, g \in A_{[2]}$ (recall that $\z_{[2]}$ is spanned as a $\bbC$-vector space by elements of the form $f \bar{d}(g)$; cf. remark \ref{remark: centres_of_dual_toroidal_lie_bialgebras}):
                                $$[D, f \bar{d}(g)] = D(f) \bar{d}(g) + f \bar{d}( D(g) )$$
                            Applying this formula yields:
                                $$
                                    \begin{aligned}
                                        [c_v, D_{r, s}] & = -\left( (-s v^{-r + 1} t^{-s - 1} \del_v + r v^{-r} t^{-s} \del_t)(v^{-1}) \bar{d}(v) + v^{-1} \bar{d}( (-s v^{-r + 1} t^{-s - 1} \del_v + r v^{-r} t^{-s} \del_t)(v) ) \right)
                                        \\
                                        & = -\left( -s v^{-r + 1} t^{-s - 1} \del_v v + v^{-1} \bar{d}( -s v^{-r + 1} t^{-s - 1} \del_v v ) \right)
                                        \\
                                        & = s v^{-r + 1} t^{-s - 1} - s v^{-1} \bar{d}( v^{-r + 1} t^{-s - 1} )
                                        \\
                                        & = s v^{-r + 1} t^{-s - 1} - 0
                                        \\
                                        & = s v^{-r + 1} t^{-s - 1}
                                    \end{aligned}
                                $$
                                $$
                                    \begin{aligned}
                                        [c_v, D_v] & = -\left( v t^{-1} \del_v v^{-1} \bar{d}(v) + v^{-1} \bar{d}(v t^{-1} \del_v v)\right)
                                        \\
                                        & = -\left( - v^{-1} t^{-1} \bar{d}(v) + v^{-1} t^{-1} d(v t^{-1}) ) \right)
                                        \\
                                        & = -\left( - v^{-1} t^{-1} \bar{d}(v) + 0 ) \right)
                                        \\
                                        & = v^{-1} t^{-1} \bar{d}(v)
                                    \end{aligned}
                                $$
                            From these formulas, we obtain:
                                $$$$
                            \item Thirdly, to compute:
                                $$[c_{v_1} \tensor 1 + 1 \tensor c_{v_2}, \sfr_{\bar{\d}_{[2]}^+}]$$
                            let us firstly consider the following:
                        \end{enumerate}
                        
                        \item Finally, consider:
                            $$[c_{t_1} \tensor 1 + 1 \tensor c_{t_2}, \sfr_{\hat{\g}_{[2]}^+}]$$
                    \end{enumerate}

                    \item Finally, let us compute how $\hat{\delta}^+$ acts on $\bar{\d}_{[2]}^+$. 
                \end{enumerate}

                \todo{cobracket computations link}
            \end{remark}

        \subsection{Chevalley-Serre presentations for extended toroidal Lie algebras}
            When $\g$ is of finite type as in convention \ref{conv: a_fixed_finite_dimensional_simple_lie_algebra}, we will be observing the following convention.
            \begin{convention} \label{conv: a_fixed_untwisted_affine_kac_moody_algebra}
                Let us write:
                    $$\hat{\g}_{[1]} := \tilde{\g}_{[1]} \rtimes \bbC \del_v^{(0)}$$
                to mean the untwisted affine Kac-Moody algebra attached to the finite-dimensional simple Lie algebra $\g$ from convention \ref{conv: a_fixed_finite_dimensional_simple_lie_algebra} (cf. \cite[Chapter 7]{kac_infinite_dimensional_lie_algebras}). Here, $\g_{[1]} := \g[v^{\pm 1}]$ is equipped with the invariant inner product given by:
                    $$(x v^m, y v^n)_{\g_{[1]}} := (x, y)_{\g} \delta_{m + n = 0}$$
                for all $x, y \in \g$ and all $m, n \in \Z$, and $\del_v^{(0)} \in \der_{\bbC}(\tilde{\g}_{[1]})$ is the derivation on $\tilde{\g}_{[1]} := \uce(\g_{[1]})$ acting as $\id_{\g} \tensor v \frac{d}{dv}$ on $\g_{[1]}$ and as zero on $\z_{[1]} := \z(\tilde{\g}_{[1]})$ (recall also that this centre is $1$-dimensional; we shall be writing $\z_{[1]} := \bbC c$).
    
                Fix a Cartan subalgebra $\hat{\h}_{[1]}$ of $\hat{\g}_{[1]}$.
    
                The affine Dynkin diagram associated to $\hat{\g}_{[1]}$ (cf. \cite[Chapter 4]{kac_infinite_dimensional_lie_algebras}) shall be denoted by:
                    $$\hat{\Gamma} := ( \hat{\Gamma}_0, \hat{\Gamma}_1 )$$
                with $\hat{\Gamma}_0$ denoting the set of vertices and $\hat{\Gamma}_1$ denoting the set of undirected edges.
    
                Finally, denote the associated Cartan matrix by:
                    $$\hat{C} := (\hat{c}_{ij})_{1 \leq i, j \leq |\hat{\Gamma}_0|}$$
                Since affine Cartan matrices are symmetrisable, we can fix a symmetrisation:
                    $$\hat{C} := \hat{D} \hat{A}$$
                wherein $\hat{D}$ is an invertible diagonal $|\Gamma_0| \x |\Gamma_0|$ matrix and $A$ is symmetric. 

                Also, let us denote the highest root by $\theta$. Note that we have:
                    $$\hat{\Gamma}_0 = \Gamma_0 \cup \{\theta\}$$
            \end{convention}
            \begin{lemma}[Chevalley-Serre presentation for $\tilde{\g}_{[2]}$] \label{lemma: chevalley_serre_presentation_for_central_extensions_of_multiloop_algebras}
                (Cf. \cite[Proposition 6.6]{wendlandt_formal_shift_operators_on_yangian_doubles}) The Lie algebra $\tilde{\g}_{[2]}$ is isomorphic to the Lie algebra $\t$ generated by the set:
                    $$\{ E_{i, r}^{\pm}, H_{i, r} \}_{(i, r) \in \hat{\Gamma}_0 \x \Z} \cup \{ K \}$$
                whose elements are subjected to the following relations, given for all $(i, r), (j, s) \in \hat{\Gamma}_0 \x \Z$:
                    $$[\t, K] = 0$$
                    $$[ H_{i, r}, H_{j, s} ] = 0$$
                    $$[ H_{i, r}, E_{j, s}^{\pm} ] = \pm 2 d_{ij} E_{j, r + s}^{\pm}$$
                    $$[ E_{i, r}^+, E_{j, s}^- ] = \delta_{ij} H_{i, r + s}$$
                    $$[ E_{i, r + 1}^{\pm}, E_{j, s}^{\pm} ] - [ E_{i, r}^{\pm}, E_{j, s + 1}^{\pm} ] = 0$$
                The isomorphism $\t \xrightarrow[]{\cong} \tilde{\g}_{[2]}$ in question is given as follows, for all $(i, r) \in \hat{\Gamma}_0 \x \Z$:
                    $$\forall (i, r) \in \Gamma_0 \x \Z: E_{i, r}^{\pm} \mapsto e_i^{\pm} t^r, H_{i, r} \mapsto h_i t^r$$
                    $$\forall (i, r) \in \{\theta\} \x \Z: E_{\theta, r}^{\pm} \mapsto e_{\theta}^{\mp} v^{\pm 1} t^r, H_{\theta, r} \mapsto h_{\theta} t^r + c_v t^r$$
                    $$K \mapsto c_t$$
            \end{lemma}
            \begin{corollary} \label{coro: chevalley_serre_presentation_for_central_extensions_of_multiloop_algebras}
                The Lie algebras $\tilde{\g}_{[2]}^{\pm}$ is isomorphic to the Lie algebras $\s^{\pm}$ generated, respectively, by the sets:
                    $$\{ E_{i, r}^{\pm}, H_{i, r} \}_{(i, r) \in \Gamma_0 \x \Z_{\geq 0}}$$
                    $$\{ E_{i, r}^{\pm}, H_{i, r} \}_{(i, r) \in \Gamma_0 \x \Z_{< 0}} \cup \{K\}$$
                whose elements are subjected to the same relations as in lemma \ref{lemma: chevalley_serre_presentation_for_central_extensions_of_multiloop_algebras}. The isomorphisms $\s^{\pm} \xrightarrow[]{\cong} \tilde{\g}_{[2]}^{\pm}$ in question are just codomain restrictions of the isomorphism $\t \xrightarrow[]{\cong} \tilde{\g}_{[2]}$ from lemma \ref{lemma: chevalley_serre_presentation_for_central_extensions_of_multiloop_algebras}.
            \end{corollary}
            
        \begin{convention}[Yangians associated to symmetrisable Kac-Moody algebras]
            Suppose for a moment that $\g$ is a general symmetrisable Kac-Moody algebra whose associated Cartan matrix is indecomposable. We refer the reader to \cite[Section 2]{guay_nakajima_wendlandt_affine_yangian_coproduct} for the definition of the \textbf{formal Yangian} $\rmY_{\hbar}(\g)$ and \textbf{Yangian} $\rmY(\g)$, as well as all relevant discussions about the various \say{basic} presentations of these associative algebras (living over $\bbC[\hbar]$ and $\bbC$ respectively). The only thing that we will note is that we will be denoting the Chevalley-Serre generators by:
                $$E_{i, r}^{\pm}, H_{i, r}$$
        \end{convention}
        \begin{convention}
            From now on, let us write:
                $$T_{i, 1}(\hbar) := H_{i, 1} - \frac12 \hbar H_{i, 0}^2$$
                $$T_{i, 1} := T_{i, 1}(1) = H_{i, 1} - \frac12 H_{i, 0}^2$$
        \end{convention}
         \begin{lemma}[Affine formal Yangians as flat graded deformations] \label{lemma: affine_formal_yangians_as_flat_graded_deformations}
            Suppose that $\g \not \cong \sl_2(\bbC)$. Then the $\bbC[\hbar]$-algebra $\rmY_{\hbar}(\hat{\g}_{[1]})$ will be a flat $\Z$-graded deformation of the $\Z$-graded $\bbC$-algebra $\rmU(\tilde{\g}_{[2]}^+)$. 
         \end{lemma}
            \begin{proof}
                \todo[inline]{Check if this depends on PBW for untwisted affine Yangians. See appendix. The coproduct paper only tells us that $\rmY(\hat{\g}_{[1]})$ is a PBW deformation of $\rmU(\tilde{\g}_{[2]}^+)$. We need to lift this up to the formal Yangian via $\Rees_{\hbar}$, but this construction might depend on PBW.}
            \end{proof}
        The hypotheses of the following lemma are satisfied at least when $\g$ is either of finite type or of affine type, save for the types $\sfA_1^{(1)}$ and $\sfA_1^{(2)}$.
        \begin{lemma}[Levendorskii's presentation] \label{lemma: levendorskii_presentation}
            \cite[Theorem 2.13]{guay_nakajima_wendlandt_affine_yangian_coproduct} Suppose for a moment that $\g$ is a general symmetrisable Kac-Moody algebra whose Cartan matrix is:
            \begin{itemize}
                \item indecomposable,
                \item such that, for any $i < j \in \Gamma_0$ (with respect to some choice of total ordering on $\Gamma_0$) the following $2 \x 2$ matrix is invertible:
                    $$
                        \begin{pmatrix}
                            a_{ii} & a_{ij}
                            \\
                            a_{ji} & a_{ji}
                        \end{pmatrix}
                    $$
            \end{itemize}
            The formal Yangian $\rmY_{\hbar}(\g)$ of $\g$ will then be isomorphic to the associative $\bbC$-algebra generated by the set:
                $$\{ H_{i, r}, E_{i, r}^{\pm} \}_{(i, r) \in \Gamma_0 \x \N}$$
            whose elements are subjected to the following relations\footnote{... and it is understood that the elements $H_{i, 0} = h_i, E_{i, 0}^{\pm} = e_i^{\pm}$ satisfy the Chevalley-Serre relations defining $\g$; cf. \cite[Chapter 1]{kac_infinite_dimensional_lie_algebras}.}:
                $$H_{i, 0} = h_i, E_{i, 0}^{\pm} = e_i^{\pm}$$
                $$[ H_{i, r}, H_{j, s} ] = 0$$
                $$[ H_{i, 0}, E_{j, s}^{\pm} ] = \pm c_{ij} E_{j, s}^{\pm}$$
                $$[ E_{i, r}^+, E_{j, s}^- ] = \pm \delta_{ij} H_{i, r + s}$$
                $$\left[ T_{i, 1}(\hbar), E_{j, 0}^{\pm} \right] = \pm \hbar c_{ij} E_{j, 1}^{\pm}$$
                $$[ E_{i, 1}^{\pm}, E_{j, 0}^{\pm} ] - [ E_{i, 0}^{\pm}, E_{j, 1}^{\pm} ] = \pm \frac12 \hbar c_{ij} \{E_{i, 0}^{\pm}, E_{j, 0}^{\pm}\}$$
        \end{lemma}
        \begin{proposition}[Levendorskii presentation for $\tilde{\g}_{[2]}$] \label{prop: levendorskii_presentation__for_central_extensions_of_multiloop_algebras}
            The Lie algebra $\tilde{\g}_{[2]}$ is isomorphic to the Lie algebra $\t$ generated by the set:
                $$\{ E_{i, r}^{\pm}, H_{i, r} \}_{(i, r) \in \hat{\Gamma}_0 \x \Z} \cup \{ K \}$$
            whose elements are subjected to the following relations, given for all $(i, r), (j, s) \in \hat{\Gamma}_0 \x \Z$:
                $$[\t, K] = 0$$
                $$H_{i, 0} = h_i, E_{i, 0}^{\pm} = e_i^{\pm}$$
                $$[ H_{i, r}, H_{j, s} ] = 0$$
                $$[ H_{i, 0}, E_{j, s}^{\pm} ] = \pm 2 d_{ij} E_{j, s}^{\pm}$$
                $$[ E_{i, r}^+, E_{j, s}^- ] = \delta_{ij} H_{i, r + s}$$
                $$[ E_{i, 1}^{\pm}, E_{j, 0}^{\pm} ] - [ E_{i, 0}^{\pm}, E_{j, 1}^{\pm} ] = 0$$
            The isomorphism $\t \xrightarrow[]{\cong} \tilde{\g}_{[2]}$ in question is as in lemma \ref{lemma: chevalley_serre_presentation_for_central_extensions_of_multiloop_algebras}.
        \end{proposition}
            \begin{proof}
                Combine lemma \ref{lemma: affine_formal_yangians_as_flat_graded_deformations} with lemma \ref{lemma: levendorskii_presentation}. Note that flatness (in particular, $\hbar$-torsion-freeness) is crucial for our application of lemma \ref{lemma: affine_formal_yangians_as_flat_graded_deformations}.
            \end{proof}
        \begin{corollary} \label{coro: levendorskii_presentation__for_central_extensions_of_multiloop_algebras}
            The Lie algebras $\tilde{\g}_{[2]}^{\pm}$ are, respectively, isomorphic to the Lie algebra $\s^{\pm}$ generated by the set:
                $$\{ E_{i, r}^{\pm}, H_{i, r} \}_{(i, r) \in \Gamma_0 \x \Z_{\geq 0}}$$
                $$\{ E_{i, r}^{\pm}, H_{i, r} \}_{(i, r) \in \Gamma_0 \x \Z_{< 0}} \cup \{K\}$$
            whose elements are subjected to the relations as in proposition \ref{prop: levendorskii_presentation__for_central_extensions_of_multiloop_algebras}. The isomorphism $\s^{\pm} \xrightarrow[]{\cong} \tilde{\g}_{[2]}^{\pm}$ in question is as in lemma \ref{lemma: chevalley_serre_presentation_for_central_extensions_of_multiloop_algebras}.
        \end{corollary}

    \section{Classical limits of affine Yangians}
        \begin{convention}
            We shall be using the following shorthand:
                $$\{ X_1, ..., X_n \} := \sum_{\sigma \in S_n} X_{\sigma(1)} \cdot ... \cdot X_{\sigma(n)}$$
                $$\bar{\Delta}(X) := X \tensor 1 + 1 \tensor X$$
        \end{convention}

        \begin{convention} \label{conv: general_symmetrisable_kac_moody_algebra}
            Until further notice, let us assume that $\g$ is a general symmetrisable Kac-Moody algebra whose Cartan matrix is indecomposable\footnote{The indecomposability assumption is not necessary. It is made simply so that when $\g$ is of finite type, we would be back in the situation of convention \ref{conv: a_fixed_finite_dimensional_simple_lie_algebra}}.
        \end{convention}

        \subsection{(Pseudo-)coproducts on affine Yangians}
            We begin by reviewing the constructions of a coproduct and a \say{parametrised pseudo-coproduct} giving rise, respectively, to a bialgebra and a pseudo-bialgebra structure on the Yangian $\rmY(\hat{\g}_{[1]})$, with $\hat{\g}_{[1]}$ denoting the untwisted affine Kac-Moody algebra associated to $\g$, as was done in \cite[Chapter 7]{kac_infinite_dimensional_lie_algebras} (cf. convention \ref{conv: a_fixed_untwisted_affine_kac_moody_algebra}). For more details, see \cite[Sections 4, 5, and 6]{guay_nakajima_wendlandt_affine_yangian_coproduct}.

            
        
            \begin{definition}[Parametrised pseudo-coproducts] \label{def: parametrised_pseudo_coproducts}
                Let $V$ be a $\bbC$-vector space. 

                Firstly, for $u$ a generic formal variable, fix some $\bbC$-linear map:
                    $$\Delta_u: V \to V^{\tensor 2}(\!(u)\!)$$
                A \textbf{parametrised pseudo-coproduct} on $V$ is then a sequence of $\bbC$-linear maps $\{\Delta_{u_1, ..., u_n}\}_{n \geq 1}$ defined in the following manner: for each $n \geq 1$, define a $\bbC$-linear map:
                    $$\Delta_{u_1, ..., u_n}: V \to V^{\tensor (n + 1)}(\!(u_1)\!) ... (\!(u_n)\!)$$
                given recursively by:
                    $$\Delta_{u_1, ..., u_n} := ( \id_{ V^{\tensor (n - 1)} } \tensor \Delta_{u_n} ) \circ \Delta_{u_1, ..., u_{n - 1}}$$
                The maps $\Delta_{u_1, ..., u_n}$ are to also satisfy the following \textbf{pseudo-coassociativity} property:
                    $$\Delta_{u_1, u_2} = ( \Delta_{u_1} \tensor \id_{V(\!(u_2)\!)} ) \circ \Delta_{u_2} = ( \id_{V(\!(u_1)\!)} \tensor \Delta_{u_2} ) \circ \Delta_{u_1}$$
                which extend in the obvious manner to the cases where $n > 2$. 

                If $V$ is a $\bbC$-algebra and $\Delta_u$ is a $\bbC$-algebra homomorphism, then $\Delta_u$ will define a \textbf{parametrised pseudo-bialgebra} structure on $V$.
            \end{definition}
            \begin{theorem}[The parametrised pseudo-coproduct on Yangians] \label{theorem: parametrised_pseudo_coproduct_on_yangians}
                (Cf. \cite[Theorem 6.2]{guay_nakajima_wendlandt_affine_yangian_coproduct}) If $\g$ is either of finite type but not $\sfA_1$ or of affine type but not $\sfA_1^{(1)}$ and not $\sfA_1^{(2)}$, then the following algebra homomorphism:
                    $$\Delta_u: \rmY(\g) \to \rmY(\g)^{\tensor 2}(\!(u)\!)$$
                given as follows\footnote{Note that it is given only for low-degree generators, which we know to be enough.}:
                    $$\Delta_u(E_{i, 0}^{\pm}) := E_{i, 0}^{\pm} \tensor 1 + 1 \tensor E_{i, 0}^{\pm} u^{\pm 1}$$
                    $$\Delta_u(H_{i, 0}) := \bar{\Delta}(H_{i, 0})$$
                    $$\Delta_u(T_{i, 1}) := \bar{\Delta}(T_{i, 1}) - \sum_{k \geq 1} \sum_{\alpha \in \Phi^+} (\alpha, \alpha_i) e_{\alpha}^- \tensor e_{\alpha}^+ u^k$$
                given for each $i \in \Gamma_0$ and for some choices\footnote{The expression is choice-independent, of course.} of root vectors $e_{\pm\alpha} \in \g_{\pm\alpha}$, normalised so that:
                    $$(e_{\alpha}^-, e_{\alpha}^+) = 1$$
                will define a parametrised pseudo-bialgebra structure on $\rmY(\g)$ in the sense of definition \ref{def: parametrised_pseudo_coproducts}, compatible with the algebra structure on $\rmY(\g)$.
            \end{theorem}
                \begin{proof}
                    
                \end{proof}

            \begin{theorem}[The parametrised pseudo-coproduct on formal Yangians] \label{theorem: parametrised_pseudo_coproduct_on_formal_yangians}
                If $\g$ is either of finite type but not $\sfA_1$ or of affine type but not $\sfA_1^{(1)}$ and not $\sfA_1^{(2)}$, then the following algebra homomorphism:
                    $$\Delta_{u, \hbar}: \rmY_{\hbar}(\g) \to \rmY_{\hbar}(\g)^{\tensor 2}(\!(u)\!)$$
                given as follows:
                    $$\Delta_{u, \hbar}(E_{i, 0}^{\pm}) := E_{i, 0}^{\pm} \tensor 1 + 1 \tensor E_{i, 0}^{\pm} u^{\pm 1}$$
                    $$\Delta_{u, \hbar}(H_{i, 0}) := \bar{\Delta}(H_{i, 0})$$
                    $$\Delta_{u, \hbar}(T_{i, 1}(\hbar)) := \bar{\Delta}(T_{i, 1}(\hbar)) - \hbar \sum_{k \geq 1} \sum_{\alpha \in \Phi^+} (\alpha, \alpha_i) e_{\alpha}^- \tensor e_{\alpha}^+ u^k$$
                given for each $i \in \Gamma_0$ and for some choices of root vectors $e_{\pm\alpha} \in \g_{\pm\alpha}$, normalised so that:
                    $$(e_{\alpha}^-, e_{\alpha}^+) = 1$$
                will define a parametrised pseudo-bialgebra structure on $\rmY_{\hbar}(\g)$ in the sense of definition \ref{def: parametrised_pseudo_coproducts}, compatible with the algebra structure on $\rmY_{\hbar}(\g)$.
            \end{theorem}
                \begin{proof}
                    
                \end{proof}
            \begin{corollary}[The coproduct on completed formal Yangians]
                If $\g$ is either of finite type but not $\sfA_1$ or of affine type but not $\sfA_1^{(1)}$ and not $\sfA_1^{(2)}$, then there will be a continuous coproduct:
                    $$\Delta_{\hbar}: \rmY_{\hbar}(\g) \to \rmY_{\hbar}(\g) \hattensor_{\bbC} \rmY_{\hbar}(\g)$$
                satisfying:
                    $$\Delta_{u, \hbar} = (1 \tensor s_{u, \hbar}) \circ \Delta_{\hbar}$$
                wherein:
                    $$s_{u, \hbar}: \rmY_{\hbar}(\g) \to \rmY_{\hbar}(\g)(\!(u)\!)$$
                is the $\bbC$-algebra homomorphism given as follows, for all $(i, r) \in \Gamma_0 \x \Z_{\geq 0}$ and all $h \in \h$:
                    $$s_{u, \hbar}(E_{i, r}^{\pm}) := E_{i, r}^{\pm} u^{\pm}$$
                    $$s_{u, \hbar}(H_{i, r}) := H_{i, r}, s_{u, \hbar}(h) := h$$
                In fact, $\Delta_{\hbar}$ is the lift of $\Delta$ via the Rees algebra construction (recall that $\rmY_{\hbar}(\g) \cong \Rees_{\hbar} \rmY(\g)$ with respect to the $\Z_{\geq 0}$-filtration on the second index of the Chevalley-Serre generators $E_{i, r}^{\pm}, H_{i, r}$).
            \end{corollary}
                \begin{proof}
                    
                \end{proof}

        \subsection{Toroidal Lie algebras as classical limits of affine Yangians}
            We are now finally able to put a Lie cobracket on the toroidal Lie algebra $\tilde{\g}_{[2]}^+$, compatible with the Lie bracket thereon in a manner that produces a Lie bialgebra structure. This Lie bialgebra structure is the classical limit of the coproduct on the formal Yangian $\rmY_{\h}(\hat{\g}_{[1]})$. 
            \begin{lemma}[Toroidal Lie bialgebras] \label{lemma: toroidal_lie_bialgebras}
                Assume convention \ref{conv: a_fixed_untwisted_affine_kac_moody_algebra}. Let:
                    $$\delta_{\tilde{\g}_{[2]}^+} := \delta_{\hat{\g}_{[2]}^+}|_{\tilde{\g}_{[2]}}$$
                Then $(\tilde{\g}_{[2]}^+, \delta_{\tilde{\g}_{[2]}^+})$ will be a complete topological Lie sub-bialgebra of $(\hat{\g}_{[2]}^+, \delta_{\hat{\g}_{[2]}^+})$ as given in corollary \ref{coro: extended_toroidal_lie_bialgebras}; for convenience, let us henceforth write:
                    $$\tilde{\delta}^+ := \delta_{\tilde{\g}_{[2]}^+}$$
                Additionally\footnote{We need to know how this Lie cobracket is given explicitly in order to see that it is the classical limit of the coproduct on $\rmY_{\hbar}(\hat{\g}_{[1]})$.}, the Lie cobracket:
                    $$\tilde{\delta}^+: \tilde{\g}_{[2]}^+ \to \tilde{\g}_{[2]}^+ \hattensor_{\bbC} \tilde{\g}_{[2]}^+$$
                as above is given explicitly on the set of generators:
                    $$\{ E_{i, 0}^{\pm} \}_{i \in \hat{\Gamma}_0} \cup \{ H_{i, r} \}_{i \in \hat{\Gamma}_0 \x \{0, 1\}}$$
                (which, according to corollary \ref{coro: levendorskii_presentation__for_central_extensions_of_multiloop_algebras}, are enough to specify the Lie algebra $\tilde{\g}_{[2]}^+$) as follows:
                    $$\tilde{\delta}^+(E_{i, 0}^{\pm}) = \tilde{\delta}^+(H_{i, 0}) = 0$$
                    $$
                        \tilde{\delta}^+(H_{i, 1}) =
                        \begin{cases}
                            \text{ if $i \in \Gamma_0$}
                            \\
                            \text{ if $i = \theta$}
                        \end{cases}
                    $$
            \end{lemma}
                \begin{proof}
                    It is clear from remark \ref{remark: total_degrees_of_classical_yangian_R_matrices} that:
                        $$\hat{\delta}^+(E_{i, 0}^{\pm}) = \hat{\delta}^+(H_{i, 0}) = 0$$
                    Trivially, this shows that:
                        $$\hat{\delta}^+(E_{i, 0}^{\pm}) = \hat{\delta}^+(H_{i, 0}) \in \tilde{\g}_{[2]}^+ \hattensor_{\bbC} \tilde{\g}_{[2]}^+$$
                
                    To compute $\delta(H_{i, 1})$, let us do the following. If we choose the root vector basis for $\g$, i.e.:
                        $$\{x_i\}_{1 \leq i \leq \dim_{\bbC} \g} := \{e_{\alpha}^-, e_{\alpha}^+\}_{\alpha \in \Phi^+} \cup \{h_j\}_{j \in \Gamma_0}$$
                    which are normalised to be such that:
                        $$(h_j, h_j)_{\g} = (e_{\alpha}^-, e_{\alpha}^+)_{\g} = 1$$
                    and consider $y := h_i$ being a Chevalley-Serre generator of the Cartan subalgebra $\h \subset \g$, then we can exploit the fact that $\h$ is an abelian Lie subalgebra of $\g$ to obtain the following expression for any $1 \leq i \leq l$ and any $\alpha \in \Phi^+$:
                        $$
                            \begin{aligned}
                                [y, x_i] \tensor x_i^* + x_i \tensor [y, x_i^*] & = [h_i, e_{\alpha}^{\pm}] \tensor e_{\alpha}^{\mp} + e_{\alpha}^{\pm} \tensor [h_i, e_{\alpha}^{\mp}]
                                \\
                                & = \pm (\alpha_i, \alpha) e_{\alpha}^{\pm} \tensor e_{\alpha}^{\mp} \mp (\alpha_i, \alpha) e_{\alpha}^{\pm} \tensor e_{\alpha}^{\mp}
                                \\
                                & = 0
                            \end{aligned}
                        $$
                    (note the implicit use of the fact that $(e_{\alpha}^{\pm})^* = e_{\alpha}^{\mp}$ via the non-degenerate invariant inner product $(-, -)_{\g}$). We can also exploit the fact that $(-, -)_{\g}$ differs from the trace form only by a non-zero scalar, and that in the vector representation, the matrices corresponding to $h_j$ and $e_{\alpha}^{\pm}$ are respectively diagonal and strictly upper/lower triangular to see that:
                        $$(h_i, e_{\alpha}^{\pm})_{\g} = 0$$

                    From remark \ref{remark: extended_toroidal_lie_bialgebras_explicit_formulae}, we know that:
                        $$
                            \begin{aligned}
                                \hat{\delta}^+(Y) & = [Y_1 \tensor 1 + 1 \tensor Y_2, \sfr_{ \hat{\g}_{[2]}^+ }]
                                \\
                                & = 
                                \begin{aligned}
                                    & \left( \begin{aligned}
                                        & [y_1 \tensor 1 + 1 \tensor y_2, \sfr_{\g}] v_2 \1(v_1, v_2) \1^+(t_1, t_2)
                                        \\
                                        - & \sum_{1 \leq j \leq \dim_{\bbC} \g} \left( (y_1, x_j)_{\g} ( a c_{v_1} + b c_{t_1} ) \tensor x_j^* v_2^{-a} t_2^{-b - 1} + x_j v_1^a t_1^b \tensor (y_2, x_j^*)_{\g} ( a c_{v_2} + b c_{t_2} ) \right)
                                    \end{aligned}
                                    \right)
                                    \\
                                    + & \sum_{(r, s) \in \Z \x \Z_{> 0}} \left( Z_{r, s} \tensor y_2 D_{r, s}(v_2^a t_2^b) + y_1 D_{r, s}(v_1^a t_1^b) \tensor Z_{r, s} \right)
                                    \\
                                    + & c_{v_1} \tensor y_2 D_{v_2}(v_2^a t_2^b) + y_1 D_{v_1}(v_1^a t_1^b) \tensor c_{v_2}
                                \end{aligned}
                            \end{aligned}
                        $$
                    for any $Y := y v^a t^b$, with $y \in \g$ and $(a, b) \in \Z \x \Z_{\geq 0}$. When $y := h_i$ for some \textit{finite} vertex $i \in \Gamma_0$ and $(a, b) = (0, 1)$, the above reduces to:
                        $$
                            \begin{aligned}
                                \hat{\delta}^+(H_{i, 1}) & = \hat{\delta}^+(h_i t)
                                \\
                                & =
                                \begin{aligned}
                                    & \left( [h_i \tensor 1 + 1 \tensor h_i, \sfr_{\g}] v_2 \1(v_1, v_2) \1^+(t_1, t_2) - \sum_{j \in \Gamma_0} \left( c_{t_1} \tensor h_j t_2^{-2} + h_j t_1 \tensor c_{t_2} \right)
                                    \right)
                                    \\
                                    + & \sum_{(r, s) \in \Z \x \Z_{> 0}} r \left( Z_{r, s} \tensor h_i v_2^{-r} t_2^{-s} + h_i v_1^{-r} t_1^{-s}  \tensor Z_{r, s}
                                    \right)
                                \end{aligned}
                                \\
                                & =
                                \begin{aligned}
                                    & \left( [h_i \tensor 1 + 1 \tensor h_i, \sfr_{\g}] v_2 \1(v_1, v_2) \1^+(t_1, t_2) - \sum_{j \in \Gamma_0} \left( c_{t_1} \tensor h_j t_2^{-2} + h_j t_1 \tensor c_{t_2} \right)
                                    \right)
                                    \\
                                    + & r ( Z_{r, s} \tensor h_i + h_i \tensor Z_{r, s} ) v_2 t_2 \1(v_1, v_2) \1^+(t_1, t_2)
                                \end{aligned}
                            \end{aligned}
                        $$

                    When $i = \theta$ is the affine vertex of the Dynkin diagram of $\hat{\g}_{[1]}$, recall first of all under the isomorphism:
                        $$\s \xrightarrow[]{\cong} \tilde{\g}_{[2]}^+$$
                    as in corollary \ref{coro: levendorskii_presentation__for_central_extensions_of_multiloop_algebras}, one has that:
                        $$H_{\theta, 1} \mapsto h_{\theta} t + c_v t$$
                    As such, we will break up the computation of $\hat{\delta}^+(H_{\theta, 1})$ into computations of $\hat{\delta}(h_{\theta} t)$ and of $\hat{\delta}^+(c_v t)$, and then add these summands together. For the first computation, let us recall from \cite[Chapter 7]{kac_infinite_dimensional_lie_algebras} that:
                        $$\hat{\h} \cong \h \oplus \bbC c_v \oplus \bbC \del_v^{(0)}$$
                    and that there is the following root space decomposition of the Kac-Moody algebra $\hat{\g}_{[1]}$ (see also \cite[Chapter 1]{kac_infinite_dimensional_lie_algebras}):
                        $$\hat{\g}_{[1]} \cong \hat{\h} \oplus \bigoplus_{\alpha \in \hat{\Phi}} \g_{\alpha}[v^{\pm 1}]$$
                    From this, one sees that:
                        $$h_{\theta} = c_v + \del_v^{(0)} = c_v + t D_v$$
                    (recall from remark \ref{remark: dual_of_toroidal_centres_contains_derivations} that $D_v = v t^{-1} \del_v$); we thus see that it is enough to know the values of:
                        $$\hat{\delta}^+(c_v), \hat{\delta}^+(D_v)$$
                    in order to compute $\hat{\delta}^+(H_{\theta, 1})$.
                        
                    Let us note also that the computations above, along with the fact that $\hat{\delta}^+(E_{i, 0}^{\pm}) = \hat{\delta}^+(H_{i, 0}) \in \tilde{\g}_{[2]}^+ \hattensor_{\bbC} \tilde{\g}_{[2]}^+$, shows that:
                        $$\hat{\delta}^+(H_{i, 1}) \in \tilde{\g}_{[2]}^+ \hattensor_{\bbC} \tilde{\g}_{[2]}^+$$
                    for all $i \in \hat{\Gamma}_0$, which shows that the restriction:
                        $$\tilde{\delta}^+ := \delta_{\hat{\g}_{[2]}^+}|_{\tilde{\g}_{[2]}^+}$$
                    indeed defines a Lie bialgebra structure on $\tilde{\g}_{[2]}^+$.
                    
                    \todo{Not done!}
                \end{proof}
            \begin{theorem}[Toroidal Lie algebras as classical limits of affine Yangians] \label{theorem: toroidal_lie_algebras_as_classical_limits_of_affine_yangians}
                The complete topological Lie bialgebra $(\tilde{\g}_{[2]}^+, \delta_{\tilde{\g}_{[2]}^+})$ as in lemma \ref{lemma: toroidal_lie_bialgebras} is the classical limit\footnote{\say{Quasi-classical} in the terminologies of \cite{etingof_kazhdan_quantisation_1}.} of the complete topological bialgebra $(\rmY_{\hbar}(\hat{\g}_{[1]}), \Delta_{\hbar})$.
            \end{theorem}
                \begin{proof}
                    Clear from lemma \ref{lemma: toroidal_lie_bialgebras} and the construction of the Hopf coproduct on the formal Yangian $\rmY_{\hbar}(\hat{\g}_{[1]})$. 
                \end{proof}
                
            \begin{definition}[Parametrised pseudo-quantisations] \label{def: parametrised_pseudo_quantisations}
                
            \end{definition}
            \begin{theorem}[Parametrised pseudo-classical limits of affine Yangians] \label{theorem: parametrised_pseudo_classical_limits_of_affine_yangians}
                
            \end{theorem}
                \begin{proof}
                    
                \end{proof}

    \begin{appendices}
        \section{The category \texorpdfstring{$\calO$}{} of a Yangian}
    \subsection{Modified Casimir operators and tensor products of representations}
        We introduce here what is usually referred to as the \say{category $\calO$} of the Yangian $\rmY(\g)$ of a symmetrisable Kac-Moody algebra $\g$ (for simplicity, let us assume that its Cartan matrix is indecomposable). Though the definition is natural from the point of view of Lie theory, in practice it is a somewhat \textit{ad hoc} device\footnote{As opposed to the classical category $\calO$ of a finite-dimensional simple Lie algebra, or even the analogue thereof for symmetrisable Kac-Moody algebras (cf. \cite[Chapter 9]{kac_infinite_dimensional_lie_algebras}), which is a natural homological category with various nice properties.} - at least as far as the structural theory of Yangians attached to symmetrisable Kac-Moody algebras is concerned - as it exists only as a mean for constructing certain tensor products of $\rmY(\g)$-modules, which themselves are in service of the construction of a coproduct on a certain topological completion of $\rmY(\g)$. 

        \begin{convention}
            Suppose for now that $\g$ is a fixed symmetrisable Kac-Moody algebra whose Cartan matrix is indecomposable. 
        \end{convention}

        \begin{definition}[The category $\calO$ of $\rmY(\g)$] \label{def: category_O_of_yangians}
            There is a full subcategory $\calO(\rmY(\g))$ of ${}^l\rmY(\g)\mod$ consisting of objects $V$ which are:
            \begin{itemize}
                \item $\h$-diagonalisable,
                \item with finite-dimensional weight spaces $V_{\mu}$ for every $\mu \in \Pi(V)$, and
                \item such that there exist $\lambda_1, ..., \lambda_k \in \h^*$ such that, if $\mu \in \Pi(V)$ and $V_{\mu} \not \cong 0$ then:
                    $$\lambda_i - \mu \in \Lambda^+$$
                for all $1 \leq i \leq k$.
            \end{itemize}
        \end{definition}
        \begin{remark}
            One easy consequence of definition \ref{def: category_O_of_yangians} (specifically, the last condition) is that, if $V$ is an object of $\calO(\rmY(\g))$ and $\mu \in \h^*$ is some abstract weight, then for any positive (real) root $\alpha \in \Phi^+$, there exists a sufficiently large natural number $N \in \N$ such that:
                $$r \geq N \implies V_{\mu + r \alpha} \cong 0$$
        \end{remark}
        \begin{definition}[Integrable $\rmY(\g)$-modules] \label{def: integrable_modules_over_yangians}
            An object $V$ of $\calO(\rmY(\g))$ is said to be \textbf{integrable} if and only if for every abstract weight $\mu \in \h^*$ and for any positive (real) root $\alpha \in \Phi^+$, there exists a sufficiently large natural number $N \in \N$ such that:
                $$r \geq N \implies V_{\mu \pm r \alpha} \cong 0$$
        \end{definition}

        \begin{theorem}[Tensor products in $\calO(\rmY(\g))$] \label{theorem: tensor_products_in_the_category_O_of_yangians}
            \cite[Theorem 4.9]{guay_nakajima_wendlandt_affine_yangian_coproduct} For any pair of objects $V_1, V_2$ of $\calO(\rmY(\g))$, the map:
                $$\Delta_{V_1, V_2}: \rmY(\g) \to \End_{\bbC}(V_1 \tensor_{\bbC} V_2)$$
            given by:
                $$\Delta_{V_1, V_2}(E_{i, 0}^{\pm}) = \bar{\Delta}(E_{i, 0}^{\pm})$$
                $$\Delta_{V_1, V_2}(H_{i, 0}) = \bar{\Delta}(H_{i, 0})$$
                $$\Delta_{V_1, V_2}(T_{i, 1}) = \bar{\Delta}(T_{i, 1}) + [H_{i, 0} \tensor 1, \sfr_{\g}^+]$$
            is a homomorphism of $\bbC$-algebras. Here, we have that:
                $$\sfr_{\g}^+ := \sfr_{\h} + \sum_{\alpha \in \Phi^+} \sum_{k = 1}^{\dim_{\bbC} \g_{\alpha}} $$
        \end{theorem}
    
    \subsection{Completions}
    
        \section{A PBW theorem for Yangians via vertex representations}
    \subsection{Yangian doubles and centrally extended Yangian doubles}

    \subsection{Vertex representations}

    \subsection{A PBW theorem for simply-laced affine Yangians}
    \end{appendices}
            
    \addcontentsline{toc}{section}{References}
    \printbibliography

\end{document}