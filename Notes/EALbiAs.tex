\input{article preambles}

\setcounter{section}{-1}

\renewcommand{\cong}{\simeq}
\newcommand{\ladjoint}{\dashv}
\newcommand{\radjoint}{\vdash}
\newcommand{\<}{\langle}
\renewcommand{\>}{\rangle}
\newcommand{\ndiv}{\hspace{-2pt}\not|\hspace{5pt}}
\newcommand{\cond}{\blacktriangle}
\newcommand{\solid}{\blacksquare}
\newcommand{\ot}{\leftarrow}
\renewcommand{\-}{\text{-}}
\renewcommand{\mapsto}{\leadsto}
\renewcommand{\leq}{\leqslant}
\renewcommand{\geq}{\geqslant}
\renewcommand{\setminus}{\smallsetminus}
\makeatletter
\DeclareRobustCommand{\cev}[1]{%
  {\mathpalette\do@cev{#1}}%
}
\newcommand{\do@cev}[2]{%
  \vbox{\offinterlineskip
    \sbox\z@{$\m@th#1 x$}%
    \ialign{##\cr
      \hidewidth\reflectbox{$\m@th#1\vec{}\mkern4mu$}\hidewidth\cr
      \noalign{\kern-\ht\z@}
      $\m@th#1#2$\cr
    }%
  }%
}
\makeatother

\newcommand{\N}{\mathbb{N}}
\newcommand{\Z}{\mathbb{Z}}
\newcommand{\Q}{\mathbb{Q}}
\newcommand{\R}{\mathbb{R}}
\newcommand{\bbC}{\mathbb{C}}
\NewDocumentCommand{\x}{e{_^}}{%
  \mathbin{\mathop{\times}\displaylimits
    \IfValueT{#1}{_{#1}}
    \IfValueT{#2}{^{#2}}
  }%
}
\NewDocumentCommand{\pushout}{e{_^}}{%
  \mathbin{\mathop{\sqcup}\displaylimits
    \IfValueT{#1}{_{#1}}
    \IfValueT{#2}{^{#2}}
  }%
}
\newcommand{\supp}{\operatorname{supp}}
\newcommand{\im}{\operatorname{im}}
\newcommand{\coker}{\operatorname{coker}}
\newcommand{\id}{\mathrm{id}}
\newcommand{\chara}{\operatorname{char}}
\newcommand{\trdeg}{\operatorname{trdeg}}
\newcommand{\rank}{\operatorname{rank}}
\newcommand{\trace}{\operatorname{tr}}
\newcommand{\length}{\operatorname{length}}
\newcommand{\height}{\operatorname{height}}
\renewcommand{\span}{\operatorname{span}}
\newcommand{\e}{\epsilon}
\newcommand{\p}{\mathfrak{p}}
\newcommand{\q}{\mathfrak{q}}
\newcommand{\m}{\mathfrak{m}}
\newcommand{\n}{\mathfrak{n}}
\newcommand{\calF}{\mathcal{F}}
\newcommand{\calG}{\mathcal{G}}
\newcommand{\calO}{\mathcal{O}}
\newcommand{\F}{\mathbb{F}}
\DeclareMathOperator{\lcm}{lcm}
\newcommand{\gr}{\operatorname{gr}}
\newcommand{\vol}{\mathrm{vol}}
\newcommand{\ord}{\operatorname{ord}}

\newcommand{\GL}{\operatorname{GL}}
\newcommand{\SL}{\operatorname{SL}}
\newcommand{\Sp}{\operatorname{Sp}}
\newcommand{\GSp}{\operatorname{GSp}}
\newcommand{\GSpin}{\operatorname{GSpin}}
\newcommand{\opO}{\operatorname{O}}
\newcommand{\SO}{\operatorname{SO}}
\newcommand{\SU}{\operatorname{SU}}
\newcommand{\opU}{\operatorname{U}}
\newcommand{\Spec}{\mathrm{Spec}}
\newcommand{\Spf}{\mathrm{Spf}}
\newcommand{\Spm}{\mathrm{Spm}}
\newcommand{\Spv}{\mathrm{Spv}}
\newcommand{\Spa}{\mathrm{Spa}}
\newcommand{\Spd}{\mathrm{Spd}}
\newcommand{\Proj}{\mathrm{Proj}}
\newcommand{\Gr}{\mathrm{Gr}}
\newcommand{\Hecke}{\mathrm{Hecke}}
\newcommand{\Sht}{\mathrm{Sht}}
\newcommand{\Quot}{\mathrm{Quot}}
\newcommand{\Hilb}{\mathrm{Hilb}}
\newcommand{\Pic}{\mathrm{Pic}}
\newcommand{\Div}{\mathrm{Div}}
\newcommand{\Jac}{\mathrm{Jac}}
\newcommand{\Alb}{\mathrm{Alb}} %albanese variety
\newcommand{\Bun}{\mathrm{Bun}}
\newcommand{\loopspace}{\mathbf{\Omega}}
\newcommand{\suspension}{\mathbf{\Sigma}}
\newcommand{\tangent}{\mathrm{T}} %tangent space
\newcommand{\Eig}{\mathrm{Eig}}

\newcommand{\Ring}{\mathrm{Ring}}
\newcommand{\Cring}{\mathrm{CRing}}
\newcommand{\Alg}{\mathrm{Alg}}
\newcommand{\Leib}{\mathrm{Leib}} %leibniz algebras
\newcommand{\Fld}{\mathrm{Fld}}
\newcommand{\Sets}{\mathrm{Sets}}
\newcommand{\Cat}{\mathrm{Cat}}
\newcommand{\Grp}{\mathrm{Grp}}
\newcommand{\Ab}{\mathrm{Ab}}
\newcommand{\Sch}{\mathrm{Sch}}
\newcommand{\Coh}{\mathrm{Coh}}
\newcommand{\QCoh}{\mathrm{QCoh}}
\newcommand{\Desc}{\mathrm{Desc}}
\newcommand{\Sh}{\mathrm{Sh}}
\newcommand{\Psh}{\mathrm{PSh}}
\newcommand{\Fib}{\mathrm{Fib}}
\renewcommand{\mod}{\-\mathrm{mod}}
\newcommand{\comod}{\-\mathrm{comod}}
\newcommand{\bimod}{\-\mathrm{bimod}}
\newcommand{\Vect}{\mathrm{Vect}}
\newcommand{\Rep}{\mathrm{Rep}}
\newcommand{\Grpd}{\mathrm{Grpd}}
\newcommand{\Arr}{\mathrm{Arr}}
\newcommand{\Esp}{\mathrm{Esp}}
\newcommand{\Ob}{\mathrm{Ob}}
\newcommand{\Mor}{\mathrm{Mor}}
\newcommand{\Mfd}{\mathrm{Mfd}}
%\newcommand{\LR}{\mathrm{LR}}
%\newcommand{\RSpc}{\mathrm{RSpc}}
\newcommand{\Spc}{\mathrm{Spc}}
\newcommand{\Top}{\mathrm{Top}}
\newcommand{\Topos}{\mathrm{Topos}}
\newcommand{\Nil}{\mathfrak{Nil}}
\newcommand{\J}{\mathfrak{J}}
\newcommand{\Stk}{\mathrm{Stk}}
\newcommand{\Pre}{\mathrm{Pre}}
\newcommand{\simp}{\mathbf{\Delta}}
\newcommand{\Ind}{\mathrm{Ind}}
\newcommand{\Pro}{\mathrm{Pro}}
\newcommand{\Mon}{\mathrm{Mon}}
\newcommand{\Comm}{\mathrm{Comm}}
\newcommand{\Fin}{\mathrm{Fin}}
\newcommand{\Assoc}{\mathrm{Assoc}}
\newcommand{\Co}{\mathrm{Co}}
\newcommand{\Loc}{\mathrm{Loc}}
\newcommand{\Ringed}{\mathrm{Ringed}}
\newcommand{\Comp}{\mathrm{Comp}} %compact hausdorff spaces
\newcommand{\Stone}{\mathrm{Stone}} %stone spaces
\newcommand{\sfExt}{\mathrm{Ext}} %extremely disconnected spaces
\newcommand{\Ouv}{\mathrm{Ouv}}
\newcommand{\Str}{\mathrm{Str}}
\newcommand{\Func}{\mathrm{Func}}
\newcommand{\Crys}{\mathrm{Crys}}
\newcommand{\LocSys}{\mathrm{LocSys}}
\newcommand{\Sieves}{\mathrm{Sieves}}
\newcommand{\pt}{\mathrm{pt}}
\newcommand{\Graphs}{\mathrm{Graphs}}
\newcommand{\Lie}{\mathrm{Lie}}
\newcommand{\Env}{\mathrm{Env}}
\newcommand{\Ho}{\mathrm{Ho}}
\newcommand{\rmD}{\mathrm{D}}
\newcommand{\Cov}{\mathrm{Cov}}
\newcommand{\Frames}{\mathrm{Frames}}
\newcommand{\Locales}{\mathrm{Locales}}
\newcommand{\Span}{\mathrm{Span}}
\newcommand{\Corr}{\mathrm{Corr}}
\newcommand{\Monad}{\mathrm{Monad}}
\newcommand{\Var}{\mathrm{Var}}
\newcommand{\sfN}{\mathrm{N}} %nerve
\newcommand{\Dia}{\mathrm{Dia}}
\newcommand{\co}{\mathrm{co}}
\newcommand{\ev}{\mathrm{ev}}
\newcommand{\bi}{\mathrm{bi}}
\newcommand{\Nat}{\mathrm{Nat}}
\newcommand{\Hopf}{\mathrm{Hopf}}
\newcommand{\Dmod}{\mathrm{D}\mod}
\newcommand{\Perv}{\mathrm{Perv}}
\newcommand{\Sph}{\mathrm{Sph}}
\newcommand{\Moduli}{\mathrm{Moduli}}
\newcommand{\Pseudo}{\mathrm{Pseudo}}
\newcommand{\Lax}{\mathrm{Lax}}
\newcommand{\Strict}{\mathrm{Strict}}
\newcommand{\Opd}{\mathrm{Opd}} %operads
\newcommand{\Shv}{\mathrm{Shv}}
\newcommand{\Char}{\mathrm{Char}} %CharShv = character sheaves
\newcommand{\Huber}{\mathrm{Huber}}
\newcommand{\Tate}{\mathrm{Tate}}
\newcommand{\Ad}{\mathrm{Ad}} %adic spaces
\newcommand{\Perfd}{\mathrm{Perfd}} %perfectoid spaces
\newcommand{\Sub}{\mathrm{Sub}} %subobjects
\newcommand{\Ideals}{\mathrm{Ideals}}
\newcommand{\Isoc}{\mathrm{Isoc}}
\newcommand{\Ban}{\-\mathrm{Ban}} %Banach spaces
\newcommand{\Fre}{\-\mathrm{Fre}} %Frechet spaces
\newcommand{\Ch}{\mathrm{Ch}} %chain complexes
\newcommand{\Mot}{\mathrm{Mot}} %motives
\newcommand{\KL}{\mathrm{KL}} %category of Kazhdan-Lusztig modules
\newcommand{\Pres}{\mathrm{Pres}} %presentable categories
\newcommand{\Noohi}{\mathrm{Noohi}} %category of Noohi groups
\newcommand{\Inf}{\mathrm{Inf}}

\newcommand{\Aut}{\mathrm{Aut}}
\newcommand{\Inn}{\mathrm{Inn}}
\newcommand{\Out}{\mathrm{Out}}
\newcommand{\frakgl}{\mathfrak{gl}}
\newcommand{\der}{\mathfrak{der}} %derivations on Lie algebras
\newcommand{\inn}{\mathfrak{inn}} %inner derivations
\newcommand{\out}{\mathfrak{out}} %outer derivations
\newcommand{\Stab}{\mathrm{Stab}}
\newcommand{\Cent}{\mathrm{Cent}}
\newcommand{\Norm}{\mathrm{Norm}}
\newcommand{\Rad}{\mathrm{Rad}}
\newcommand{\Transporter}{\mathrm{Transp}} %transporter between two subsets of a group
\newcommand{\Conj}{\mathrm{Conj}}
\newcommand{\Diag}{\mathrm{Diag}}
\newcommand{\Gal}{\mathrm{Gal}}
\newcommand{\bfG}{\mathbf{G}} %absolute Galois group
\newcommand{\Frac}{\mathrm{Frac}}
\newcommand{\Ann}{\mathrm{Ann}}
\newcommand{\Val}{\mathrm{Val}}
\newcommand{\Chow}{\mathrm{Chow}}
\newcommand{\Sym}{\mathrm{Sym}}
\newcommand{\End}{\mathrm{End}}
\newcommand{\Mat}{\mathrm{Mat}}
\newcommand{\Diff}{\mathrm{Diff}}
\newcommand{\Autom}{\mathrm{Autom}}
\newcommand{\Artin}{\mathrm{Artin}} %artin maps
\newcommand{\sk}{\mathrm{sk}} %skeleton of a category
\newcommand{\eqv}{\mathrm{eqv}} %functor that maps groups $G$ to $G$-sets
\newcommand{\Inert}{\mathrm{Inert}}
\newcommand{\Fil}{\mathrm{Fil}}

\newcommand{\colim}{\operatorname{colim} \:}
\renewcommand{\lim}{\operatorname{lim} \:}
\newcommand{\toto}{\rightrightarrows}
%\newcommand{\tensor}{\otimes}
\NewDocumentCommand{\tensor}{e{_^}}{%
  \mathbin{\mathop{\otimes}\displaylimits
    \IfValueT{#1}{_{#1}}
    \IfValueT{#2}{^{#2}}
  }%
}
\newcommand{\eq}{\operatorname{eq}}
\newcommand{\coeq}{\operatorname{coeq}}
\newcommand{\Hom}{\mathrm{Hom}}
\newcommand{\Maps}{\mathrm{Maps}}
\newcommand{\Tor}{\mathrm{Tor}}
\newcommand{\Ext}{\mathrm{Ext}}
\newcommand{\Isom}{\mathrm{Isom}}
\newcommand{\stalk}{\mathrm{stalk}}
\newcommand{\RKE}{\operatorname{RKE}}
\newcommand{\LKE}{\operatorname{LKE}}
\newcommand{\oblv}{\mathrm{oblv}}
\newcommand{\const}{\mathrm{const}}
%\newcommand{\forget}{\mathrm{forget}}
\newcommand{\adrep}{\mathrm{ad}} %adjoint representation
\newcommand{\NL}{\mathbb{NL}} %naive cotangent complex
\newcommand{\pr}{\operatorname{pr}}
\newcommand{\Der}{\mathrm{Der}}
\newcommand{\Frob}{\mathrm{Frob}} %Frobenius
\newcommand{\frob}{\mathrm{f}} %trace of Frobenius
\newcommand{\bfpt}{\mathbf{pt}}
\newcommand{\bfloc}{\mathbf{loc}}
\DeclareMathAlphabet{\mymathbb}{U}{BOONDOX-ds}{m}{n}
\newcommand{\0}{\mymathbb{0}}
\newcommand{\1}{\mathbbm{1}}
\newcommand{\2}{\mathbbm{2}}
\newcommand{\Jet}{\mathrm{Jet}}
\newcommand{\Split}{\mathrm{Split}}
\newcommand{\Sq}{\mathrm{Sq}}
\newcommand{\Zero}{\mathrm{Z}}
\newcommand{\SqZ}{\Sq\Zero}
\newcommand{\frakLie}{\mathfrak{Lie}}
\newcommand{\y}{\mathrm{y}} %yoneda
\newcommand{\Sm}{\mathrm{Sm}}
\newcommand{\AJ}{\phi} %abel-jacobi map
\newcommand{\act}{\mathrm{act}}
\newcommand{\ram}{\mathrm{ram}} %ramification index
\newcommand{\inv}{\mathrm{inv}}

\newcommand{\bbU}{\mathbb{U}}
\newcommand{\V}{\mathbb{V}}
\newcommand{\U}{\mathrm{U}}
\newcommand{\calU}{\mathcal{U}}
\newcommand{\calW}{\mathcal{W}}
\newcommand{\rmI}{\mathrm{I}} %augmentation ideal
\newcommand{\bfV}{\mathbf{V}}
\newcommand{\C}{\mathcal{C}}
\newcommand{\D}{\mathcal{D}}
\newcommand{\T}{\mathscr{T}} %Tate modules
\newcommand{\calM}{\mathcal{M}}
\newcommand{\calN}{\mathcal{N}}
\newcommand{\calP}{\mathcal{P}}
\newcommand{\calQ}{\mathcal{Q}}
\newcommand{\A}{\mathbb{A}}
\renewcommand{\P}{\mathbb{P}}
\newcommand{\calL}{\mathcal{L}}
\newcommand{\E}{\mathcal{E}}
\renewcommand{\H}{\mathbf{H}}
\newcommand{\scrS}{\mathscr{S}}
\newcommand{\calX}{\mathcal{X}}
\newcommand{\calY}{\mathcal{Y}}
\newcommand{\calZ}{\mathcal{Z}}
\newcommand{\calS}{\mathcal{S}}
\newcommand{\calR}{\mathcal{R}}
\newcommand{\scrX}{\mathscr{X}}
\newcommand{\scrY}{\mathscr{Y}}
\newcommand{\scrZ}{\mathscr{Z}}
\newcommand{\calA}{\mathcal{A}}
\newcommand{\calB}{\mathcal{B}}
\newcommand{\sfT}{\mathrm{T}}
\renewcommand{\S}{\mathcal{S}}
\newcommand{\B}{\mathbb{B}}
\newcommand{\bbD}{\mathbb{D}}
\newcommand{\G}{\mathbb{G}}
\newcommand{\horn}{\mathbf{\Lambda}}
\renewcommand{\L}{\mathbb{L}}
\renewcommand{\a}{\mathfrak{a}}
\renewcommand{\b}{\mathfrak{b}}
\renewcommand{\t}{\mathfrak{t}}
\renewcommand{\r}{\mathfrak{r}}
\newcommand{\bbX}{\mathbb{X}}
\newcommand{\g}{\mathfrak{g}}
\newcommand{\h}{\mathfrak{h}}
\renewcommand{\k}{\mathfrak{k}}
\newcommand{\del}{\partial}
\newcommand{\bbE}{\mathbb{E}}
\newcommand{\scrO}{\mathscr{O}}
\newcommand{\bbO}{\mathbb{O}}
\newcommand{\scrA}{\mathscr{A}}
\newcommand{\scrB}{\mathscr{B}}
\newcommand{\scrF}{\mathscr{F}}
\newcommand{\scrG}{\mathscr{G}}
\newcommand{\scrM}{\mathscr{M}}
\newcommand{\scrN}{\mathscr{N}}
\newcommand{\scrP}{\mathscr{P}}
\newcommand{\frakS}{\mathfrak{S}}
\newcommand{\calI}{\mathcal{I}}
\newcommand{\calJ}{\mathcal{J}}
\newcommand{\scrK}{\mathscr{K}}
\newcommand{\calK}{\mathcal{K}}
\newcommand{\scrV}{\mathscr{V}}
\newcommand{\bbS}{\mathbb{S}}
\newcommand{\scrH}{\mathscr{H}}
\newcommand{\bfB}{\mathbf{B}}
\newcommand{\Witt}{W}
%\newcommand{\bfA}{\mathbf{A}}
\renewcommand{\O}{\mathbb{O}}
\newcommand{\calV}{\mathcal{V}}
\newcommand{\scrR}{\mathscr{R}} %radical
\newcommand{\rmZ}{\mathrm{Z}} %centre of algebra
\newcommand{\bfGamma}{\mathbf{\Gamma}}
\newcommand{\scrU}{\mathscr{U}}
\newcommand{\rmW}{\mathrm{W}} %Weil group
\newcommand{\frakM}{\mathfrak{M}}
\newcommand{\frakN}{\mathfrak{N}}
\newcommand{\frakX}{\mathfrak{X}}
\newcommand{\frakY}{\mathfrak{Y}}
\newcommand{\frakZ}{\mathfrak{Z}}

\newcommand{\aff}{\mathrm{aff}}
\newcommand{\ft}{\mathrm{ft}} %finite type
\newcommand{\fp}{\mathrm{fp}} %finite presentation
\newcommand{\aft}{\mathrm{aft}}
\newcommand{\lft}{\mathrm{lft}}
\newcommand{\laft}{\mathrm{laft}}
\newcommand{\cmpt}{\mathrm{cmpt}}
\newcommand{\qc}{\mathrm{qc}}
\newcommand{\qs}{\mathrm{qs}}
\newcommand{\lcmpt}{\mathrm{lcmpt}}
%\newcommand{\conv}{\mathrm{conv}}
\newcommand{\red}{\mathrm{red}}
\newcommand{\fin}{\mathrm{fin}}
\newcommand{\gen}{\mathrm{gen}}
\newcommand{\petit}{\mathrm{petit}}
\newcommand{\gros}{\mathrm{gros}}
\newcommand{\loc}{\mathrm{loc}}
\newcommand{\glob}{\mathrm{glob}}
%\newcommand{\ringed}{\mathrm{ringed}}
\newcommand{\qcoh}{\mathrm{qcoh}}
\newcommand{\cl}{\mathrm{cl}}
\newcommand{\et}{\mathrm{\acute{e}t}}
\newcommand{\fet}{\mathrm{f\acute{e}t}}
\newcommand{\profet}{\mathrm{prof\acute{e}t}}
\newcommand{\proet}{\mathrm{pro\acute{e}t}}
\newcommand{\Zar}{\mathrm{Zar}}
\newcommand{\fppf}{\mathrm{fppf}}
\newcommand{\fpqc}{\mathrm{fpqc}}
\newcommand{\smooth}{\mathrm{sm}}
\newcommand{\sh}{\mathrm{sh}}
\newcommand{\op}{\mathrm{op}}
\newcommand{\open}{\mathrm{open}}
\newcommand{\closed}{\mathrm{closed}}
\newcommand{\geom}{\mathrm{geom}}
\newcommand{\alg}{\mathrm{alg}}
\newcommand{\sober}{\mathrm{sober}}
\newcommand{\dR}{\mathrm{dR}}
\newcommand{\rad}{\mathrm{rad}}
\newcommand{\discrete}{\mathrm{discrete}}
%\newcommand{\add}{\mathrm{add}}
%\newcommand{\lin}{\mathrm{lin}}
\newcommand{\Krull}{\mathrm{Krull}}
\newcommand{\qis}{\mathrm{qis}} %quasi-isomorphism
\newcommand{\ho}{\mathrm{ho}} %homotopy equivalence
\newcommand{\sep}{\mathrm{sep}}
\newcommand{\unr}{\mathrm{unr}}
\newcommand{\tame}{\mathrm{tame}}
\newcommand{\wild}{\mathrm{wild}}
\newcommand{\nil}{\mathrm{nil}}
\newcommand{\defm}{\mathrm{defm}}
\newcommand{\Art}{\mathrm{Art}}
\newcommand{\Noeth}{\mathrm{Noeth}}
\newcommand{\affd}{\mathrm{affd}}
%\newcommand{\adic}{\mathrm{adic}}
\newcommand{\pre}{\mathrm{pre}}
\newcommand{\perf}{\mathrm{perf}}
\newcommand{\perfd}{\mathrm{perfd}}
\newcommand{\rat}{\mathrm{rat}}
\newcommand{\cont}{\mathrm{cont}}
\newcommand{\dg}{\mathrm{dg}}
\newcommand{\almost}{\mathrm{a}}
%\newcommand{\stab}{\mathrm{stab}}
\newcommand{\heart}{\heartsuit}
\newcommand{\proj}{\mathrm{proj}}
\newcommand{\qproj}{\mathrm{qproj}}
\newcommand{\pd}{\mathrm{pd}}
\newcommand{\crys}{\mathrm{crys}}
\newcommand{\prisma}{\mathrm{prisma}}
\newcommand{\FF}{\mathrm{FF}}
\newcommand{\sph}{\mathrm{sph}}
\newcommand{\lax}{\mathrm{lax}}
\newcommand{\weak}{\mathrm{weak}}
\newcommand{\strict}{\mathrm{strict}}
\newcommand{\mon}{\mathrm{mon}}
\newcommand{\sym}{\mathrm{sym}}
\newcommand{\lisse}{\mathrm{lisse}}
\newcommand{\an}{\mathrm{an}}
\newcommand{\ad}{\mathrm{ad}}
\newcommand{\sch}{\mathrm{sch}}
\newcommand{\rig}{\mathrm{rig}}
\newcommand{\pol}{\mathrm{pol}}
\newcommand{\plat}{\mathrm{flat}}
\newcommand{\proper}{\mathrm{proper}}
\newcommand{\compl}{\mathrm{compl}}
\newcommand{\non}{\mathrm{non}}
\newcommand{\access}{\mathrm{access}}
\newcommand{\comp}{\mathrm{comp}}
\newcommand{\tstructure}{\mathrm{t}} %t-structures
\newcommand{\pure}{\mathrm{pure}} %pure motives
\newcommand{\mixed}{\mathrm{mixed}} %mixed motives
\newcommand{\num}{\mathrm{num}} %numerical motives
\newcommand{\ess}{\mathrm{ess}}
\newcommand{\topological}{\mathrm{top}}
\newcommand{\convex}{\mathrm{cv}}
\newcommand{\ab}{\mathrm{ab}} %abelian extensions
\newcommand{\surj}{\mathrm{surj}} %coverage on sets generated by surjections
\newcommand{\eff}{\mathrm{eff}} %effective Cartier divisors
\newcommand{\Weil}{\mathrm{Weil}} %weil divisors
\newcommand{\lex}{\mathrm{lex}}
\newcommand{\rex}{\mathrm{rex}}
\newcommand{\AR}{\mathrm{A\-R}}
\newcommand{\cons}{\mathrm{c}} %constructible sheaves
\newcommand{\tor}{\mathrm{tor}} %tor dimension
\newcommand{\semisimple}{\mathrm{ss}}

%prism custom command
\usepackage{relsize}
\usepackage[bbgreekl]{mathbbol}
\usepackage{amsfonts}
\DeclareSymbolFontAlphabet{\mathbb}{AMSb} %to ensure that the meaning of \mathbb does not change
\DeclareSymbolFontAlphabet{\mathbbl}{bbold}
\newcommand{\prism}{{\mathlarger{\mathbbl{\Delta}}}}

\begin{document}

    \title{Toroidal doubles ?}
    
    \author{Dat Minh Ha}
    \maketitle
    
    \begin{abstract}
    
    \end{abstract}
    
    {
    \hypersetup{} 
    %\dominitoc
    \tableofcontents %sort sections alphabetically
    \listoftodos
    }

    \section{Introduction}

    \section{A Lie bialgebra structures on \texorpdfstring{$\uce( \g[v^{\pm 1}, t] )$}{}}
        \begin{convention} \label{conv: a_fixed_finite_dimensional_simple_lie_algebra}
            Throughout this section, we fix a finite-dimensional simple Lie algebra $\g$ over $\bbC$, equipped with a symmetric and non-degenerate invariant $\bbC$-bilinear form $(-, -)_{\g}$. It is known that such a bilinear form is unique up to $\bbC^{\x}$-multiples, so for all intents and purposes, it can be assumed to be the Killing form, though this assumption is not necessary. 

            Suppose also that $\g$ is equipped with a basis $\{x_i\}_{1 \leq i \leq \dim_{\bbC} \g}$ and with respect to $(-, -)_{\g}$, we identify a dual basis $\{x^i\}_{1 \leq i \leq \dim_{\bbC} \g}$. Recall that the universal Casimir element/canonical element of $\g$ is:
                $$\scrR_{\g} := \sum_{1 \leq i \leq \dim_{\bbC} \g} x_i \tensor x^i \in \g \tensor_{\bbC} \g$$
            and recall that $\scrR_{\g}$ is independent of what we choose the basis vectors $x_i$ to be.
        \end{convention}

        \begin{convention}
            Throughout, we shall use $(-)^{\star}$ to denote graded duals. 
        \end{convention}

        \begin{convention}
            If $k$ is a commutative ring and $A$ is a $k$-algebra, and if $L$ is a Lie algebra over $k$, then the default Lie algebra structure on the $k$-module $L \tensor_k A$ shall be the one given by extension of scalars, i.e.:
                $$[x \tensor a, y \tensor b]_{L \tensor_k A} := [x, y]_L \mu_A(a \tensor b)$$
            $L \tensor_k A$ is usually regarded as Lie algebra over $k$ instead of over $A$.  
        \end{convention}
    
        \subsection{A Lie bialgebra structure on \texorpdfstring{$\g[v^{\pm 1}, t]$}{}}
            \begin{definition} \label{def: residue_form_on_loop_algebra}
                For $v$ a formal variable, we can extend the invariant inner product $(-, -)_{\g}$ to the following pairing on $\g[v^{\pm 1}]$ by defining:
                    $$(x f(v), y g(v))_{\g[v^{\pm 1}]} := (x, y)_{\g} \Res_{v = 0}( v^{-1} f(v) g(v) )$$
                for all $x, y \in \g$ and all $f(v), g(v) \in \bbC[v^{\pm 1}]$, and recall that:
                    $$\Res_{v = 0}\left( \sum_{n \in \Z} a_n v^n \right) := a_{-1}$$
                More algebraically\footnote{So that the definition would work still when we replace $\bbC$ with a general algebraically closed field of characteristic $0$.}, we can define this as:
                    $$(x v^m, y v^n)_{\g[v^{\pm 1}]} := (x, y)_{\g} \delta_{m + n, 0}$$
            \end{definition}
            \begin{definition} \label{def: residue_form_on_multi_loop_algebra}
                Now, let $(-, -)_{\g[v^{\pm 1}]}$ be as in definition \ref{def: residue_form_on_loop_algebra}. This can be extended furthermore to $\g[v^{\pm 1}, t^{\pm 1}]$ in the following manner: for all $X(v), Y(v) \in \g[v^{\pm 1}]$ and all $f(t), g(t) \in \bbC[t^{\pm 1}]$, define:
                    $$(X(v) f(t), Y(v) g(t))_{\g[v^{\pm 1}, t^{\pm 1}]} := (X(v), Y(v))_{\g[v^{\pm 1}]} \Res_{t = 0}( f(t) g(t) )$$
                More algebraically, we can define this as:
                    $$(X(v) f(t), Y(v) g(t))_{\g[v^{\pm 1}, t^{\pm 1}]} := (X(v), Y(v))_{\g[v^{\pm 1}]} \delta_{m + n, -1}$$
            \end{definition}

            \begin{question} 
                \begin{enumerate}
                    \item Verify that $(-, -)_{\g[v^{\pm 1}, t^{\pm 1}]}$ is an invariant and non-degenerate symmetric $\bbC$-bilinear form on $\g[v^{\pm 1}, t^{\pm 1}]$.
                    \item Show that by equpping $\g[v^{\pm 1}, t^{\pm 1}]$ with the invariant inner product $(-, -)_{\g[v^{\pm 1}, t^{\pm 1}]}$, the following triple of Lie algebras becomes a well-defined Manin triple:
                        $$(\g[v^{\pm 1}, t^{\pm 1}], \g[v^{\pm 1}, t], t^{-1}\g[v^{\pm 1}, t^{-1}])$$
                    \item Find a formula for the canonical element $\scrR_{\g[v^{\pm 1}, t} \in \g[v^{\pm 1}, t] \hattensor_{\bbC} \g[v^{\pm 1}, t]$ with respect to the restriction of $(-, -)_{\g[v^{\pm 1}, t^{\pm 1}]}$ to $\g[v^{\pm 1}, t^{-1}] \x \g[v^{\pm 1}, t]$.
                    \item Find the Lie bialgebra structure on $\g[v^{\pm 1}, t]$ arising from the Manin triple in 2.
                \end{enumerate}
            \end{question}
                \begin{proof}
                    \begin{enumerate}
                        \item The symmetry and bilinearity of $(-, -)_{\g[v^{\pm 1}, t^{\pm 1}]}$ are clear from the construction of this bilinear pairing as in definition \ref{def: residue_form_on_multi_loop_algebra}. $(-, -)_{\g[v^{\pm 1}, t^{\pm 1}]}$-invariance follows from the $\g$-invariance of $(-, -)_{\g}$, which is by hypothesis. Finally, non-degeneracy follows from the non-degeneracy of $(-, -)_{\g}$ (also by hypothesis) as well as the non-degeneracy of the residual pairings on $\bbC[v^{\pm 1}]$ (as in definition \ref{def: residue_form_on_loop_algebra}) and on $\bbC[t^{\pm 1}]$ (as in definition \ref{def: residue_form_on_multi_loop_algebra}); to see that the latter point holds, simply note that there exists no $m \in \Z$ such that $\delta_{m + n, - 1} = 0$ (respectively, such that $\delta_{m + n, 0}$) for all $n \in \Z$.
                        \item It is not hard to see that: with respect to $(-, -)_{\g[v^{\pm 1}, t^{\pm 1}]}$ as in 1, one has that:
                            $$\g[v^{\pm 1}, t]^{\star} \cong \bigoplus_{m \in \Z, p \in \Z_{\geq 0}} (\g v^m t^p)^* \cong \bigoplus_{m \in \Z, p \in \Z_{\geq 0}} \g v^{-m} t^{-p - 1} \cong t^{-1}\g[v^{\pm 1}, t^{-1}]$$
                        with respect to the invariant inner product $(-, -)_{\g[v^{\pm 1}, t^{\pm 1}]}$. It is also easy to see that:
                            $$\g[v^{\pm 1}, t^{\pm 1}] \cong \g[v^{\pm 1}, t] \oplus t^{-1}\g[v^{\pm 1}, t^{-1}]$$
                        Note also that $\g[v^{\pm 1}, t^{\pm 1}] \supset \g[v^{\pm 1}, t], t^{-1}\g[v^{\pm 1}, t^{-1}]$ are Lie subalgebras. Finally, to prove that $(-, -)_{\g[v^{\pm 1}, t^{\pm 1}]}$ pairs the vector subspaces $\g[v^{\pm 1}, t], t^{-1}\g[v^{\pm 1}, t^{-1}]$ isotropically, simply that there does not exist any $p, q \geq 0$ or $p, q \leq -1$ simultaneously so that:
                            $$\delta_{p + q, -1} = 0$$
                        which means that:
                            $$(t^{-1}\g[v^{\pm 1}, t^{-1}], t^{-1}\g[v^{\pm 1}, t^{-1}])_{\g[v^{\pm 1}, t^{\pm 1}]} = (\g[v^{\pm 1}, t], \g[v^{\pm 1}, t])_{\g[v^{\pm 1}, t^{\pm 1}]} = 0$$
                        \item It will be convenient for us to make the identification of topological vector spaces:
                            $$\g[v^{\pm 1}, t] \hattensor_{\bbC} \g[v^{\pm 1}, t^{-1}] \cong \g[v_2^{\pm 1}, t_1] \hattensor_{\bbC} \g[v^{\pm 1}, t_2^{-1}]$$
                        Also, let us fix the basis:
                            $$\{X_{m, p}\}_{(m, p) \in \Z^2} := \{x_i v^m t^p\}_{1 \leq i \leq \dim_{\bbC} \g, m \in \Z, p \in \Z}$$
                        for $\g[v^{\pm 1}, t^{\pm 1}]$. It is easy to see that the graded dual of this basis with respect to the invariant inner product $(-, -)_{\g[v^{\pm 1}, t^{\pm 1}]}$ is:
                            $$\{X_{m, p}^{\star}\}_{(m, p) \in \Z^2} := \{x^i v^{-m} t^{-p - 1}\}_{1 \leq i \leq \dim_{\bbC} \g, m \in \Z, p \in \Z}$$
                        
                        By definition, the canonical element $\scrR_{\g[v_1^{\pm 1}, t]} \in \g[v_2^{\pm 1}, t_1] \hattensor_{\bbC} \g[v^{\pm 1}, t_2^{-1}]$ is given by:
                            $$\scrR_{\g[v_1^{\pm 1}, t]} := \sum_{(m, p) \in \Z^2} X_{m, p} \hattensor X_{m, p}^{\star}$$
                        As such, we have that\todo{Probably should specify which topology we're taking the limit with respect to.}:
                            $$
                                \begin{aligned}
                                    \scrR_{\g[v^{\pm 1}, t^{\pm 1}]} & := \sum_{1 \leq i \leq \dim_{\bbC} \g} \sum_{m = -\infty}^{+\infty} \sum_{p = -\infty}^{+\infty} x_i v_1^m t_1^p \hattensor x^i v_2^{-m} t_2^{-p - 1}
                                    \\
                                    & = \left( \sum_{1 \leq i \leq \dim_{\bbC} \g} x_i \tensor x^i \right) \left( \sum_{m = 0}^{+\infty} (v_1^{-1} v_2)^m + \sum_{m = 0}^{+\infty} (v_1 v_2^{-1})^m \right) \left( t_2^{-1} \sum_{p = 0}^{+\infty} (t_1 t_2^{-1})^p \right)
                                    \\
                                    & = \scrR_{\g} \cdot \left( \frac{1}{1 - v_1 v_2^{-1}} + \frac{1}{1 - v_1^{-1} v_2} \right) \cdot \frac{t_2}{1 - t_1 t_2^{-1}}
                                \end{aligned}
                            $$
                        \item Let us keep the identification:
                            $$\g[v^{\pm 1}, t] \hattensor_{\bbC} \g[v^{\pm 1}, t^{-1}] \cong \g[v_2^{\pm 1}, t_1] \hattensor_{\bbC} \g[v^{\pm 1}, t_2^{-1}]$$
                        From \cite[pp. 5]{etingof_kazhdan_quantisation_1}\footnote{Actually, this citation is not quite right, since the result was stated for finite-dimensional Manin triples only. However, since we're dealing with graded duals with finite-dimensional graded components, I believe the analogous result still holds. Of course, I should write this down carefully at some point.}, we know that the Lie bialgebra structure (say, $\delta_{\g[v^{\pm 1}, t]}$) on $\g[v^{\pm 1}, t^{\pm 1}]$ is given at any $X(v, t) \in \g[v^{\pm 1}, t^{\pm 1}]$ by:
                            $$\delta_{\g[v^{\pm 1}, t]}( X(v, t) ) = [X(v_1, t_1) \tensor 1 + 1 \tensor X(v_2, t_2), \scrR_{\g[v^{\pm 1}, t]}]$$
                        When $X(v, t) := x v^m t^p$ for some $x \in \g, m \in \Z, t \in \Z_{\geq 0}$, this can written out more explicitly as follows:
                            $$
                                \begin{aligned}
                                    \delta_{\g[v^{\pm 1}, t]}( x v^m t^p ) & = \left[x v_1^m t_1^p \tensor 1 + 1 \tensor x v_2^m t_2^p, \scrR_{\g} \cdot \left( \frac{1}{1 - v_1 v_2^{-1}} + \frac{1}{1 - v_1^{-1} v_2} \right) \cdot \frac{t_2}{1 - t_1 t_2^{-1}}\right]
                                    \\
                                    & = [x \tensor 1 + 1 \tensor x, \scrR_{\g}] \cdot v_1^m t_1^p \cdot v_2^m t_2^p \cdot \left( \frac{1}{1 - v_1 v_2^{-1}} + \frac{1}{1 - v_1^{-1} v_2} \right) \frac{t_2}{1 - t_1 t_2^{-1}}
                                \end{aligned}
                            $$
                    \end{enumerate}
                \end{proof}

        \subsection{Failing to extend the Lie bialgebra structure to \texorpdfstring{$\uce( \g[v^{\pm 1}, t] )$}{}}
            \begin{convention}
                Set $\t := \uce( \g[v^{\pm 1}, t^{\pm 1}] )$ and $\s := \uce( \g[v^{\pm 1}, t] )$. 
            \end{convention}
        
            \begin{question} \label{question: extending_invariant_inner_products_on_multi_loop_to_universal_central_extensions}
                \begin{enumerate}
                    \item Prove that there is a unique invariant symmetric bilinear form $(-, -)_{\t}$ on $\t$ whose restriction to $\g[v^{\pm 1}, t^{\pm 1}]$ coincides with $(-, -)_{\g[v^{\pm 1}, t^{\pm 1}]}$.
                    \item Find a Lie subalgebra $\s^{\star} \subset \t$ such that:
                        $$\t \cong \s \oplus \s^{\star}$$
                    and such that $\s^{\star}$ is paired isotropically with $\s$ by $(-, -)_{\t}$. 
                    \item Why is the triple:
                        $$(\t, \s, \s^{\star})$$
                    with $\t$ being equipped with $(-, -)_{\t}$ not a Manin triple ?
                \end{enumerate}
            \end{question}
                \begin{proof}
                    \begin{enumerate}
                        \item Suppose that $B$ is any invariant inner product on $\t$ and fix an element $Z \in \z(\t)$. This gives us:
                            $$B([X, Y], Z) = B(X, [Y, Z]) = B(X, 0) = 0$$
                        for all $X, Y \in \t$. As such, the sought-for unique invariant inner product on $\t$ induced by $(-, -)_{\g[v^{\pm 1}, t^{\pm 1}]}$, whose restriction to $\g[v^{\pm 1}, t^{\pm 1}] \subset \t$ coincides with $(-, -)_{\g[v^{\pm 1}, t^{\pm 1}]}$, must be determined by:
                            $$(X, Z)_{\t} = 0, (Z, Z)_{\t} = 0$$
                        for all $X \in \t$ and all $Z \in \z(\t)$.
                        \item One thing that we are able to gather from 1 is that, with respect to $(-, -)_{\t}$, the centre $\z(\t)$ is orthogonally complementary to $\g[v^{\pm 1}, t^{\pm 1}]$. With this in mind, we claim that:
                            $$\s^{\star} \cong t^{-1}\g[v^{\pm 1}, t^{-1}] \oplus \z(\s^{\star})$$
                        wherein $\z(\s^{\star})$ is such that:
                            $$\z(\t) \cong \z(\s) \oplus \z(\s^{\star})$$
                        and note that $\z(\s^{\star})$ must exist due to $\s$ being a Lie subalgebra of $\t$ and hence $\z(\s)$ being a Lie subalgebra of $\z(\t)$ (namely, one has that $\z(\s) = \z(\t) \cap \s$). To see that this is indeed that the Lie subalgebra of $\t$ that we are after, firstly note that because:
                            $$(-, -)_{\t}|_{\g[v^{\pm 1}, t^{\pm 1}]} = (-, -)_{\g[v^{\pm 1}, t^{\pm 1}]}$$
                        and because it has been shown that $(-, -)_{\g[v^{\pm 1}, t^{\pm 1}]}$ pairs $\g[v^{\pm 1}, t]$ and $t^{-1}\g[v^{\pm 1}, t^{-1}]$ isotropically as subspaces of $\g[v^{\pm 1}, t^{\pm 1}]$, the only thing to demonstrate is that $(-, -)_{\t}$ pairs elements of $\z(\s)$ and $\z(\s^{\star})$ isotropically with one another in the sense that:
                            $$(\z(\s), \z(\s))_{\t} = (\z(\s^{\star}), \z(\s^{\star}))_{\t} = 0$$
                        This is directly due to the fact that elements of $\z(\s)$ and likewise, those of $\z(\s^{\star})$, are central as elements of $\t$. Lastly, one verifies that, one indeed has that:
                            $$\s \oplus \s^{\star} \cong ( \g[v^{\pm 1}, t] \oplus \z(\s) ) \oplus ( t^{-1}\g[v^{\pm 1}, t^{-1}] \oplus \z(\s^{\star}) ) \cong \g[v^{\pm 1}, t^{\pm 1}] \oplus \z(\t) \cong \t$$
                        \item $(\t, \s, \s^{\star})$ is not a Manin triple (nor a graded Manin triple, for that matter) due to the simple fact that the non-zero vector space $\z(\t)$ is contained entirely in $\Rad (-, -)_{\t} := \{Z \in \t \mid \forall X \in \t: (X, Z)_{\t} = 0\}$. This implies that the invariant inner product $(-, -)_{\t}$ on $\t$ is \textit{degenerate}, thereby violating the definition of Manin triples. 

                        Note that we have not even checked whether or not $\s^{\star}$ is actually a Lie subalgebra of $\t$ or merely a vector subspace. This will turn out to be true, but we defer this discussion to question \ref{question: toroidal_dual}. 
                    \end{enumerate}
                \end{proof}
            \begin{remark}[What exactly is $\z(\s^{\star})$ ?] \label{remark: centres_of_dual_toroidal_lie_bialgebras}
                In attempting to answer question \ref{question: extending_invariant_inner_products_on_multi_loop_to_universal_central_extensions}, we relied on the existence of an abstract vector subspace $\z(\s^{\star})$ of $\z(\t)$ specified by the condition that:
                    $$\z(\t) \cong \z(\s) \oplus \z(\s^{\star})$$
                Let us now spend a bit of time on giving an explicit description of $\z(\s^{\star})$. 

                Suppose that $k$ is an arbitrary commutative ring. Recall firstly that, should $\a$ be a perfect Lie algebra over $k$ (i.e. a Lie algebra such that $\a = [\a, \a]$) with a non-degenerate invariant inner product $(-, -)_{\a}$, then not only does $\a_A := \a \tensor_k A$ admit a universal central extension $\uce(\a_A)$ for any commutative $k$-algebra $A$ (i.e. one that is initial in the category of all central extensions of $\a$) - and recall also that any universal central extension must split - but also, that there is the following explicit description of $\uce(\a_A)$ due to Kassel\todo{Cite Kassel's paper.}:
                    $$\uce(\a_A) \cong \a_A \oplus \overline{\Omega^1_{A/k}}$$
                with $\overline{\Omega^1_{A/k}} := \coim d_{A/k} := \Omega^1_{A/k}/d_{A/k}(A)$ being the coimage of the universal K\"ahler differential map $d_{A/k}: A \to \Omega^1_{A/k}$; if we denote the composition of the universal map $d_{A/k}: A \to \Omega^1_{A/k}$ with the canonical quotient map $\Omega^1_{A/k} \to \overline{\Omega^1_{A/k}}$ by:
                    $$\overline{d_{A/k}}: A \to \overline{\Omega^1_{A/k}}$$
                then the Lie bracket on $\uce(\a_A)$ with respect to Kassel's realisation shall be given by:
                    $$
                        \begin{aligned}
                            [ x \tensor a, y \tensor b ]_{\uce(\a_A)} & = [ X \tensor a, Y \tensor b ]_{\a_A} + (x, y)_{\a} b \overline{d_{A/k}}(a)
                            \\
                            & = [X, Y]_{\a} ab - (x, y)_{\a} a \overline{d_{A/k}}(b)
                        \end{aligned}
                    $$
                for all $x, y \in \a$ and all $a, b \in A$.
                    
                If $A$ is $\Z$-graded, say:
                    $$A := \bigoplus_{n \in \Z} A_n$$
                then $\a_A$ will also be $\Z$-graded, specifically in the following manner:
                    $$\a_A := \a \tensor_k A \cong \bigoplus_{n \in \Z} \a \tensor_k A_n$$
                and for convenience, let us write $\a_{A_n} := \a \tensor_k A_n$ for each $n \in \Z$. This grading on $\a_A$ actually extends to the whole of $\uce(\a_A)$, though to be able to describe this extension in details, let us firstly how the $A$-module $\Omega^1_{A/k}$ itself is constructed. To this end, recall firstly that the $A$-module $\Omega^1_{A/k}$ is generated by the set:
                    $$\{d_{A/k}(a)\}_{a \in A}$$
                subjected to the relations:
                    $$d_{A/k}(ab) - a d_{A/k}(b) - d_{A/k}(a) b = 0$$
                defined for all $a, b \in A$. 

                We now specialise to the case wherein $k \cong \bbC$, $\a = \g$, and $(-, -)_{\a} = (-, -)_{\g}$ as in convention \ref{conv: a_fixed_finite_dimensional_simple_lie_algebra} and for the moment, let us consider:
                    $$A \in \{ A_{[n]}^+ := \bbC[v_1, ..., v_n], A_{[n]}^{\pm} := \bbC[v_1^{\pm 1}, ..., v_n^{\pm 1}] \}$$
                and also, let us abbreviate:
                    $$\Omega^+_{[n]} := \Omega^1_{A_{[n]}^+/\bbC}, \Omega^{\pm}_{[n]} := \Omega^1_{A_{[n]}^{\pm}/\bbC}$$
                    $$\overline{\Omega^+_{[n]}} := \overline{\Omega^1_{A_{[n]}^+/\bbC}}, \overline{\Omega^{\pm}_{[n]}} := \overline{\Omega^1_{A_{[n]}^{\pm}/\bbC}}$$
                    $$d := d_{A/k}, \overline{d} := \overline{d_{A/k}}$$
                Eventually, we will specialise to the case $n = 2$. \textit{A priori}, both $\Omega^+_{[n]}$ and $\Omega^{\pm}_{[n]}$ are free and of rank $n$ over $A_{[n]}^+$ and $A_{[n]}^{\pm}$ respectively, specifically generated by the basis elements:
                    $$d(v_j)$$
                In turn, this implies that the $A_{[n]}^+$-module $\overline{\Omega^+_{[n]}}$ and the $A_{[n]}^{\pm}$-module $\overline{\Omega^{\pm}_{[n]}}$ are both generated by the basis elements:
                    $$\overline{d}(v_j)$$
                that are subjected to the following relation:
                    $$0 = \overline{d}( v_1^{m_1} ... v_n^{m_n} ) = \sum_{1 \leq j \leq n} m_j v_1^{m_1} ... v_j^{m_j - 1} ... v_n^{m_n} \overline{d}(v_j)$$
                From this, one infers that the elements:
                    $$v_1^{m_1} ... v_j^{m_j - 1} ... v_n^{m_n}$$
                form a basis for $\overline{\Omega^+_{[n]}}$ and $\overline{\Omega^{\pm}_{[n]}}$ as $\bbC$-vector spaces. 

                When $n = 2$, we can write things out more explicitly: $\z(\t) \cong \overline{\Omega_{[2]}^{\pm}}$ now decomposes as a $\bbC$-vector space in the following manner:
                    $$\z(\t) \cong ( \bigoplus_{(r, s) \in \Z^2} \bbC Z_{r, s}) \oplus \bbC c_v \oplus \bbC c_t$$
                and $\z(\s) \cong \overline{\Omega_{[2]}^+}$ decomposes in the following manner:
                    $$\z(\s) \cong ( \bigoplus_{(r, s) \in \Z \x \Z_{> 0}} \bbC Z_{r, s}) \oplus \bbC c_v$$
                wherein:
                    $$\forall (r, s) \in \Z \x (\Z \setminus \{0\}): Z_{r, s} = v^{r - 1} t^s \overline{d}(v)$$
                    $$\forall (m, 0) \in \Z \x \{0\}: Z_{m, 0} = v^r t^{-1} \overline{d}(t)$$
                    $$c_v = v^{-1} \overline{d}(v), c_t = t^{-1} \overline{d}(t)$$

                From the above and from the requirement on $\z(\s^{\star})$ that:
                    $$\z(\t) \cong \z(\s) \oplus \z(\s^{\star})$$
                one sees immediately that:
                    $$\z(\s^{\star}) \cong ( \bigoplus_{(r, s) \in \Z \x \Z_{\leq 0}} \bbC Z_{r, s}) \oplus \bbC c_t$$
            \end{remark}
            \begin{definition}[Toroidal duals and doubles] \label{def: toroidal_duals_and_doubles}
                From now on, let us refer to $\s^{\star}$ as the \textbf{toroidal dual} of $\s$ and $\t$ as the \textbf{toroidal double} of $\s$. 
            \end{definition}
            \begin{question} \label{question: toroidal_dual}
                Verify that $\s^{\star}$ is a Lie subalgebra of $\t$.
            \end{question}
                \begin{proof}
                    We now know that:
                        $$\s^{\star} \cong t^{-1}\g[v^{\pm 1}, t^{-1}] \oplus \left( ( \bigoplus_{(r, s) \in \Z \x \Z_{\leq 0}} \bbC Z_{r, s}) \oplus \bbC c_t \right)$$
                    (with notations as in remark \ref{remark: centres_of_dual_toroidal_lie_bialgebras}), so the verification can be carried out by firstly considering the following, for any $X(v, t), Y(v, t) \in t^{-1}\g[v^{\pm 1}, t^{-1}]$ and any $Z, Z' \in \z(\s^{\star})$:
                        $$
                            \begin{aligned}
                                [ X(v, t) + Z, Y(v, t) + Z' ]_{\t} & = [ X(v, t), Y(v, t) ]_{\t} + [ Z, Y(v, t) ]_{\t} + [X(v, t) + Z, Z']_{\t}
                                \\
                                & = [ X(v, t), Y(v, t) ]_{\t}
                            \end{aligned}
                        $$
                    wherein the equalities hold thanks to the elements $Z, Z'$ being central inside $\t$, and then, without any loss of generality, we consider secondly the following for:
                        $$X(v, t) := x f(v, t), Y(v, t) := y g(v, t)$$
                    for some $x, y \in \g$ and $f(v, t), g(v, t) \in t^{-1}\bbC[v^{\pm 1}, t^{-1}]$:
                        $$
                            \begin{aligned}
                                [ X(v, t), Y(v, t) ]_{\t} & = [ x f(v, t), y g(v, t) ]_{\t}
                                \\
                                & = [x, y]_{\g} f(v, t) g(v, t) - (x, y)_{\g} f(v, t) \overline{d}( g(v, t) )
                            \end{aligned}
                        $$
                    Now, the only thing left to show is that:
                        $$f(v, t) \overline{d}( g(v, t) ) \in \t$$
                    (in fact, we shall see that $f(v, t) \overline{d}( g(v, t) ) \in \z(\t)$) and we can do this by once again assume without any loss of generality that:
                        $$f(v, t) := v^m t^p, g(v, t) := v^n t^q$$
                    for some $m, n \in \Z, p, q \in \Z_{\leq -1}$:
                        $$
                            \begin{aligned}
                                f(v, t) \overline{d}( g(v, t) ) & = v^m t^p \overline{d}( v^n t^q )
                                \\
                                & = v^m t^p ( n v^{n - 1} t^q \overline{d}(v) + q v^n t^{q - 1} \overline{d}(t) )
                            \end{aligned}
                        $$
                    Using the description of $\z(\t)$ from remark \ref{remark: centres_of_dual_toroidal_lie_bialgebras}, one thus sees that:
                        $$v^m t^p ( n v^{n - 1} t^q \overline{d}(v) + q v^n t^{q - 1} \overline{d}(t) ) \in \z(\t)$$
                    precisely as needed. 
                \end{proof}

        \subsection{Derivations on \texorpdfstring{$ \uce(\g[v^{\pm}, t]) $}{}}
            \begin{remark}[Extending $\t$ by derivations] \label{remark: extending_toroidal_lie_algebras_by_derivations}
                In order to remedy the issue of the pairing $(-, -)_{\t}$ as in question \ref{question: extending_invariant_inner_products_on_multi_loop_to_universal_central_extensions} being degenerate, we shall need to artificially adjoin elements that pair orthogonally (or for that matter, orthonormally) with the central elements of $\t$, seeing how these central elements all lie within the radical of $(-, -)_{\t}$. The most obvious thing to attempt is to simply set these new elements up as dual basis vectors to the elements of $\z(\t)$, as this forces the pairings to be non-trivial. We remark that, heuristically, these dual basis vectors ought to be derivations of some kind\footnote{... much like in the loop realisation of untwisted affine Kac-Moody algebras as well, which is to be thought of as the $1$-parameter counterpart to our situation.}, since we have the identification:
                    $$\z(\t) \cong \overline{\Omega^{\pm}_{[2]}}$$
                due to Kassel, and derivations are dual to differential forms after all; that said, this is merely a vague heuristic as this point.

                To begin, let us introduce a vector space extension (which necessarily splits):
                    $$0 \to \d^{\pm} \to \hat{\t} \to \t \to 0$$
                Since we know that:
                    $$\z(\t) \cong ( \bigoplus_{(r, s) \in \Z^2} \bbC Z_{r, s}) \oplus \bbC c_v \oplus \bbC c_t$$
                wherein:
                    $$\forall (r, s) \in \Z \x (\Z \setminus \{0\}): Z_{r, s} = v^{r - 1} t^s \overline{d}(v)$$
                    $$\forall (m, 0) \in \Z \x \{0\}: Z_{m, 0} = v^r t^{-1} \overline{d}(t)$$
                    $$c_v = v^{-1} \overline{d}(v), c_t = t^{-1} \overline{d}(t)$$
                (cf. remark \ref{remark: centres_of_dual_toroidal_lie_bialgebras}) and since we are extending $\t$ by $\d^{\pm}$ simply so we can extend the invariant inner product $(-, -)_{\t}$ to all of $\hat{\t}$ to obtain a \textit{non-degenerate} pairing $(-, -)_{\hat{\t}}$, let us declare that:
                    $$\d^{\pm} := (\bigoplus_{(r, s) \in \Z^2} \bbC D_{r, s}) \oplus \bbC D_v \oplus \bbC D_t$$
                wherein the basis vectors $D_{r, s}, D_v, D_t$ are so that:
                    $$( D_{r, s}, Z_{r', s'} )_{\hat{\t}} := \delta_{(r, s), (r', s')}$$
                    $$(D_v, c_v)_{\hat{\t}} := 1, (D_{r, s}, c_v)_{\hat{\t}} := 0$$
                    $$(D_t, c_t)_{\hat{\t}} := 1, (D_{r, s}, c_t)_{\hat{\t}} := 0$$
                for all $(r, s), (r', s') \in \Z^2$. In light of these conditions and of the idea that elements of $\d^{\pm}$ - as they pair orthonormally with the differential forms in $\z(\t)$ - ought to be derivationsm, we can let the Lie algebra $\der_{\bbC}(\t)$ of $\bbC$-linear derivations on $\t$ act on $\t$ via the following operators:
                    $$\forall (r, s) \in \Z \x (\Z \setminus \{0\}): D_{r, s} := v^{-(m - 1)} t^{-p} \del_v$$
                    $$\forall (r, 0) \in \Z \x \{0\}: D_{r, 0} := v^{-m} t \del_t$$
                    $$D_v := v \del_v, D_t := t \del_t$$
                wherein it is to be understood that:
                    $$\del_v \cdot x f(v, t) = x \del_v f(v, t), \del_t \cdot x f(v, t) = x \del_t f(v, t)$$
                    $$\del_v \cdot Z(v, t) = \del_t \cdot Z(v, t) = 0$$
                for all $x \in \g$, all $f(v, t) \in \bbC[v^{\pm 1}, t^{\pm 1}]$, and all $Z(v, t) \in \z(\t)$. $\d^{\pm}$ is therefore realisable as the Lie subalgebra of $\der_{\bbC}(\t)$ spanned by the basis $\{ D_{r, s} \}_{(r, s) \in \Z^2} \cup \{D_v, D_t\}$; note also that as a Lie algebra, $\d^{\pm}$ is abelian, since the operators $\del_v, \del_t$ commute with one another. By putting everything together, one obtains:
                    $$\hat{\t} \cong \d^{\pm} \ltimes \t$$
                This implies that the Lie algebra structure on $\hat{\t}$ is given by:
                    $$
                        \begin{aligned}
                            [ X_1(v, t) + D_1, X_2(v, t) + D_2 ]_{\hat{\t}} & = [X_1(v, t), X_2(v, t)]_{\t} + ( [D_1, D_2]_{\d^{\pm}} + ( D_2 \cdot X_1(v, t) - D_1 \cdot X_2(v, t) ) )
                            \\
                            & = [X_1(v, t), X_2(v, t)]_{\t} + ( D_2 \cdot X_1(v, t) - D_1 \cdot X_2(v, t) )
                        \end{aligned}
                    $$
                for all $X_1(v, t), X_2(v, t) \in \t$ and all $D_1, D_2 \in \d^{\pm}$.
            \end{remark}
        \begin{lemma} \label{lemma: non_degenerate_invariant_inner_product_on_extended_toroidal_algebras}
            There is an extension of $(-, -)_{\t}$ to $\hat{\t}$, denoted by $(-, -)_{\hat{\t}}$, that is unique up to non-zero multiples. It is given by:
                $$(-, -)_{\hat{\t}}|_{\t} := (-, -)_{\t}$$
                $$\forall (r, s), (r', s') \in \Z^2: (D_{r, s}, Z_{r', s'})_{\hat{\t}} := \delta_{(r, s), (r', s')}$$
                $$\forall (r, s), (r', s') \in \Z^2: (D_{r, s}, D_{r', s'})_{\hat{\t}} := 0$$
                $$\forall (r, s) \in \Z^2: (D_{r, s}, c_v)_{\hat{\t}} = (D_{r, s}, c_t)_{\hat{\t}} := 0$$ 
                $$(D_v, c_v)_{\hat{\t}} = (D_t, c_t)_{\hat{\t}} := 0, (D_v, c_t)_{\hat{\t}} = (D_t, c_v)_{\hat{\t}} := 1$$
                $$(D_v, Z_{r', s'})_{\hat{\t}} = (D_{r, s}, c_v)_{\hat{\t}} = (D_t, Z_{r', s'})_{\hat{\t}} = (D_{r, s}, c_t)_{\hat{\t}} := 0$$
                $$\forall D \in \d^{\pm}, \forall X(x, t) \in \g[v^{\pm 1}, t^{\pm 1}]: (D, X(v, t))_{\hat{\t}} := 0$$
            with notations as in remark \ref{remark: extending_toroidal_lie_algebras_by_derivations}. $(-, -)_{\hat{\t}}$ is $\bbC$-bilinear, symmetric, non-degenerate, and invariant. 
        \end{lemma}
            \begin{proof}
                The $\bbC$-bilinearity, symmetry, and non-degeneracy of $(-, -)_{\hat{\t}}$ are easy consequences of its construction, so the only thing non-trivial to show is that $(-, -)_{\hat{\t}}$ is invariant. To this end, let us firstly pick $x, y \in \g$ along with $r, r', s, s' \in \Z$ and then recall that:
                    $$[x v^r t^s, g v^{r'} t^{s'}]_{\t} = [x, y]_{\g} v^{r + r'} t^{s + s'} - (x, y)_{\g} v^r t^s \overline{d}(v^{r'} t^{s'})$$
                Using this, we see that:
                    $$
                        \begin{aligned}
                            ( D, [x v^r t^s, y v^{r'} t^{s'}]_{\t} )_{\hat{\t}} & = ( D, [x, y]_{\g} v^{r + r'} t^{s + s'} )_{\hat{\t}} - (x, y)_{\g} v^r t^s \overline{d}( v^{r'} t^{s'}) )_{\hat{\t}}
                            \\
                            & = -( D, (x, y)_{\g} v^r t^s \overline{d}( v^{r'} t^{s'}) )_{\hat{\t}}
                            \\
                            & = -(x, y)_{\g} ( D, v^r t^s ( r' v^{r' - 1} t^{s'} \overline{d}(v) + s' v^{r'} t^{s' - 1} \overline{d}(t) ) )_{\hat{\t}} 
                            \\
                            & = -(x, y)_{\g} ( D, r' v^{r + r' - 1} t^{s + s'} \overline{d}(v) + s' v^{r + r'} t^{s + s' - 1} \overline{d}(t) )_{\hat{\t}}
                        \end{aligned}
                    $$
                \todo{What central elements can we pair $D$ with ?}
                At the same time, consider the following:
                    $$
                        \begin{aligned}
                            ( [D, x v^r t^s]_{\hat{\t}}, y v^{r'} t^{s'} )_{\hat{\t}} & = ( -D \cdot x v^r t^s, y v^{r'} t^{s'} )_{\hat{\t}}
                            \\
                            & = -( x D( v^r t^s ), y v^{r'} t^{s'} )_{\hat{\t}}
                            \\
                            & = -(x, y)_{\g} ( D( v^r t^s ), v^{r'} t^{s'} )_{\hat{\t}}
                        \end{aligned}
                    $$
            \end{proof}
        \begin{proposition}
            Set:
                $$\hat{\s} := \d^+ \ltimes \s, \hat{\s}^{\star} := \d^- \ltimes \s^{\star}$$
            wherein:
                $$\d^+ := ( \bigoplus_{(r, s) \in \Z \x \Z_{> 0}} \bbC D_{r, s} ) \oplus \bbC D_v, \d^- := ( \bigoplus_{(r, s) \in \Z \x \Z_{\leq 0}} \bbC D_{r, s} ) \oplus \bbC D_t$$
            and the actions of $\d^+ < \der_{\bbC}(\s), \d^- < \der_{\bbC}(\s^{\star})$ are as described in remark \ref{remark: extending_toroidal_lie_algebras_by_derivations}. Also, let $(-, -)_{\hat{\t}}$ be as in lemma \ref{lemma: non_degenerate_invariant_inner_product_on_extended_toroidal_algebras}. Then:
                $$(\hat{\s}, \hat{\s})_{\hat{\t}} = (\hat{\s}^{\star}, \hat{\s}^{\star})_{\hat{\t}} = 0$$
        \end{proposition}
            \begin{proof}
                 It is then clear that:
                    $$\hat{\t} \cong \hat{\s} \oplus \hat{\s}^{\star}$$
                We know from question \ref{question: extending_invariant_inner_products_on_multi_loop_to_universal_central_extensions} that:
                    $$(\s, \s)_{\t} = (\s^{\star}, \s^{\star})_{\t} = 0$$
                and since $(-, -)_{\hat{\t}}|_{\t} = (-, -)_{\t}$ by construction (cf. lemma \ref{lemma: non_degenerate_invariant_inner_product_on_extended_toroidal_algebras}), we also have that:
                    $$(\s, \s)_{\hat{\t}} = (\s^{\star}, \s^{\star})_{\hat{\t}} = 0$$
                Lemma \ref{lemma: non_degenerate_invariant_inner_product_on_extended_toroidal_algebras} also tells us that:
                    $$\forall (r, s), (r', s') \in \Z^2: (D_{r, s}, D_{r', s'})_{\hat{\t}} := 0$$
                    $$\forall (r, s) \in \Z^2: (D_{r, s}, c_v)_{\hat{\t}} = (D_{r, s}, c_t)_{\hat{\t}} := 0$$ 
                and hence:
                    $$(\d^+, \d^+)_{\hat{\t}} = (\d^-, \d^-)_{\hat{\t}} = 0$$
                By combining the two facts, we see that:
                    $$(\hat{\s}, \hat{\s})_{\hat{\t}} = (\hat{\s}^{\star}, \hat{\s}^{\star})_{\hat{\t}} = 0$$
            \end{proof}
    
    \addcontentsline{toc}{section}{References}
    \printbibliography

\end{document}