Let us now demonstrate how a low-degree presentation for the Lie algebra $\toroidal$ may be obtained, as eluded to earlier. This necessitates introducing the (formal) Yangian associated to the untwisted affine Kac-Moody algebra $\hat{\g}$ because as explained above, we will be relying on the existence of such a low-degree presentation for those (formal) Yangians.
        \begin{convention}[Yangians associated to symmetrisable Kac-Moody algebras]
            If $\fraku$ a general symmetrisable Kac-Moody algebra whose associated Cartan matrix is indecomposable. We refer the reader to \cite[Section 2]{guay_nakajima_wendlandt_affine_yangian_coproduct} for the definition of the \textbf{formal Yangian} $\rmY_{\hbar}(\fraku)$ and \textbf{Yangian} $\rmY(\fraku)$, as well as all relevant discussions about the various \say{basic} presentations of these associative algebras (living over $k[\hbar]$ and $k$ respectively). The only thing that we will note is that we will be denoting the Chevalley-Serre generators by:
                $$X_{i, r}^{\pm}, H_{i, r}$$
        \end{convention}
        \begin{convention}
            From now on, let us write:
                $$T_{i, 1}(\hbar) := H_{i, 1} - \frac12 \hbar H_{i, 0}^2$$
                $$T_{i, 1} := T_{i, 1}(1) = H_{i, 1} - \frac12 H_{i, 0}^2$$
        \end{convention}

        One key property of formal (affine) Yangians that we will be relying on is the fact that 
        \begin{lemma}[Formal Yangians as Rees algebras] \label{lemma: formal_yangians_as_rees_algebras}
            \cite[Theorem 6.10]{guay_nakajima_wendlandt_affine_yangian_vertex_representations_and_PBW} If $\fraku$ is a general indecomposable symmetrisable Kac-Moody algebra. If $\fraku$ is either of finite type but not $\sfA_1$ or of untwisted affine type but not $\sfA_1^{(1)}$ and not $\sfA_1^{(2)}$ then the natural \textit{graded} $k$-algebra homomoprhism:
                $$\rmY_{\hbar}(\fraku) \to \Rees_{\hbar} \rmY(\fraku)$$
            will be an isomorphism. 
        \end{lemma}
         \begin{corollary}[Formal affine Yangians as flat graded deformations] \label{coro: formal_affine_yangians_as_flat_graded_deformations}
            Suppose that $\g \not \cong \sl_2(k)$. Then the $k[\hbar]$-algebra:
                $$\rmY_{\hbar}(\hat{\g})$$
            will be a flat $\Z$-graded deformation of the $\Z$-graded $k$-algebra:
                $$\rmU(\toroidal^+)$$
         \end{corollary}
         
        The hypotheses of the following lemma are satisfied at least when $\fraku$ is a symmetrisable Kac-Moody algebra of either finite type or affine type, save for the types $\sfA_1^{(1)}$ and $\sfA_1^{(2)}$.
        \begin{lemma}[A Levendorskii-type presentation for Yangians of symmetrisable Kac-Moody algebras] \label{lemma: levendorskii_presentation_for_yangians_of_symmetrisable_kac_moody_algebras}
            \cite[Theorem 2.13]{guay_nakajima_wendlandt_affine_yangian_coproduct} Suppose for a moment that $\fraku$ is a general symmetrisable Kac-Moody algebra whose Cartan matrix is:
            \begin{itemize}
                \item indecomposable,
                \item such that, for any $i < j \in \simpleroots$ (with respect to some choice of total ordering on $\simpleroots$) the following $2 \x 2$ matrix is invertible:
                    $$
                        \begin{pmatrix}
                            c_{ii} & c_{ij}
                            \\
                            c_{ji} & c_{ji}
                        \end{pmatrix}
                    $$
            \end{itemize}
            The formal Yangian $\rmY_{\hbar}(\fraku)$ of $\fraku$ will then be isomorphic to the associative $k$-algebra generated by the set:
                $$\{ H_{i, r}, X_{i, r}^{\pm} \}_{(i, r) \in \simpleroots \x \N}$$
            whose elements are subjected to the following relations\footnote{... and it is understood that the elements $H_{i, 0} = h_i, X_{i, 0}^{\pm} = x_i^{\pm}$ satisfy the Chevalley-Serre relations defining $\fraku$; cf. \cite[Chapter 1]{kac_infinite_dimensional_lie_algebras}.}:
                $$H_{i, 0} = h_i, X_{i, 0}^{\pm} = x_i^{\pm}$$
                $$[ H_{i, r}, H_{j, s} ] = 0$$
                $$[ H_{i, 0}, X_{j, s}^{\pm} ] = \pm c_{ij} X_{j, s}^{\pm}$$
                $$[ X_{i, r}^+, X_{j, s}^- ] = \pm \delta_{ij} H_{i, r + s}$$
                $$\left[ T_{i, 1}(\hbar), X_{j, 0}^{\pm} \right] = \pm \hbar c_{ij} X_{j, 1}^{\pm}$$
                $$[ X_{i, 1}^{\pm}, X_{j, 0}^{\pm} ] - [ X_{i, 0}^{\pm}, X_{j, 1}^{\pm} ] = \pm \frac12 \hbar c_{ij} \{X_{i, 0}^{\pm}, X_{j, 0}^{\pm}\}$$
        \end{lemma}
        \begin{proposition}[Levendorskii presentation for $\toroidal^+$] \label{prop: levendorskii_presentation__for_central_extensions_of_multiloop_algebras}
            Suppose that $\hat{\g}$ is not of type $\sfA_1^{(1)}$. 
        
            The Lie algebra $\toroidal^+$ is isomorphic to the Lie algebra  generated by the set:
                $$\{ X_{i, r}^{\pm}, H_{i, r} \}_{(i, r) \in \hat{\simpleroots} \x \Z_{\geq 0}}$$
            whose elements are subjected to the following relations, given for all $(i, r), (j, s) \in \hat{\simpleroots} \x \Z_{\geq 0}$:
                $$H_{i, 0} = h_i, X_{i, 0}^{\pm} = x_i^{\pm}$$
                $$[ H_{i, r}, H_{j, s} ] = 0$$
                $$[ H_{i, 0}, X_{j, s}^{\pm} ] = \pm (\alpha_j, \check{\alpha}_i) X_{j, s}^{\pm}$$
                $$[ X_{i, r}^+, X_{j, s}^- ] = \delta_{ij} H_{i, r + s}$$
                $$[ X_{i, 1}^{\pm}, X_{j, 0}^{\pm} ] - [ X_{i, 0}^{\pm}, X_{j, 1}^{\pm} ] = 0$$
            The isomorphism in question is as in lemma \ref{lemma: chevalley_serre_presentation_for_central_extensions_of_multiloop_algebras}.
        \end{proposition}
            \begin{proof}
                Combine corollary \ref{coro: formal_affine_yangians_as_flat_graded_deformations} with lemma \ref{lemma: levendorskii_presentation_for_yangians_of_symmetrisable_kac_moody_algebras}. Note that flatness (in particular, $\hbar$-torsion-freeness) is crucial for our application of corollary \ref{coro: formal_affine_yangians_as_flat_graded_deformations}.
            \end{proof}
        \begin{corollary}[Levendorskii presentation for $\toroidal$]
            Suppose that $\hat{\g}$ is not of type $\sfA_1^{(1)}$. 
        
            The Lie algebra $\toroidal$ is isomorphic to the Lie algebra  generated by the set:
                $$\{ X_{i, r}^{\pm}, H_{i, r} \}_{(i, r) \in \hat{\simpleroots} \x \Z} \cup \{ K \}$$
            whose elements are subjected to the following relations, given for all $(i, r), (j, s) \in \hat{\simpleroots} \x \Z$:
                $$H_{i, 0} = h_i, X_{i, 0}^{\pm} = x_i^{\pm}$$
                $$[ H_{i, r}, H_{j, s} ] = 0$$
                $$[ H_{i, 0}, X_{j, s}^{\pm} ] = \pm (\alpha_j, \check{\alpha}_i) X_{j, s}^{\pm}$$
                $$[ X_{i, r}^+, X_{j, s}^- ] = \delta_{ij} H_{i, r + s}$$
                $$[ X_{i, 1}^{\pm}, X_{j, 0}^{\pm} ] - [ X_{i, 0}^{\pm}, X_{j, 1}^{\pm} ] = 0$$
                $$[K, \toroidal] = 0$$
            The isomorphism in question is as in lemma \ref{lemma: chevalley_serre_presentation_for_central_extensions_of_multiloop_algebras}.
        \end{corollary}
            \begin{proof}
                Combine proposition \ref{prop: levendorskii_presentation__for_central_extensions_of_multiloop_algebras} with lemma \ref{lemma: chevalley_serre_presentation_for_central_extensions_of_multiloop_algebras}. 
            \end{proof}

        \begin{question}
            Is a low-degree presentation in the style of proposition \ref{prop: levendorskii_presentation__for_central_extensions_of_multiloop_algebras} available for the Lie algebras $\uce(\g_{[n]})$ for $n > 2$ ? Does this depend on the existence of a construction of \say{$(n - 1)$-affine Yangians} which are graded flat deformations of $\rmU(\g[v_1^{\pm 1}, ..., v_{n - 1}^{\pm 1}, v_n])$ ?
        \end{question}