\section{The Kassel realisation of UCEs of current Lie algebras}
    \subsection{A recollection of K\"ahler differentials}
        There are many perspectives on algebraic differential forms, but for our purposes, the following will be the easiest to use.
        \begin{definition}[Modules of K\"ahler differentials] \label{def: kahler_differentials}
            Let $k$ be a base commutative ring and let $A$ be a commutative $k$-algebra, defined by a multiplication map:
                $$\mu_{A/k}: A \tensor_k A \to A$$
            The $A$-module of K\"ahler differentials $\Omega^1_{A/k}$ relative to the ring map $k \to A$ is then given by:
                $$\Omega^1_{A/k} := I/I^2$$
            where $I := \ker \mu_{A/k}$. Elements of $\Omega^1_{A/k}$ are typically referred to as (differential) $1$-forms.
        \end{definition}
        \begin{remark}[Diffentials satify the Leibniz rule]
            Observe that the $A \tensor_k A$-ideal:
                $$I := \ker \mu_{A/k}$$
            is generated by elements of the form $1 \tensor f - f \tensor 1$, for all $f \in A$. We then see that:
                $$\Omega^1_{A/k} := I/I^2 \cong I \tensor_{A \tensor_k A} ( (A \tensor_k A)/I ) \cong I \tensor_{A \tensor_k A} A$$
            should we regard $A$ as a commutative $A \tensor_k A$-algebra; note that the last isomorphism holds thanks to the fact that the multiplication map $\mu_{A/k}: A \tensor_k A \to A$ is \textit{a priori} surjective. From this, one infers that there exists a canonical $k$-linear derivation:
                $$d: A \to \Omega^1_{A/k}$$
                $$f \mapsto 1 \tensor f - f \tensor 1$$
            Indeed, for every $f, g \in A$, the Leibniz rule is satisfied:
                $$g df + f dg = g(1 \tensor f - f \tensor 1) + f(1 \tensor g - g \tensor 1) = gf \tensor 1 - fg \tensor 1 = d(fg)$$
        \end{remark}
        The following well-known lemmas are very useful. A proof can be be found in any standard reference on general commutative algebra (cf. e.g. \cite[\href{https://stacks.math.columbia.edu/tag/00AO}{Tag 00AO}]{stacks}).
        \begin{lemma}[Universal property of modules of K\"ahler differentials]
            Let $k$ be a base commutative ring and let $A$ be a commutative $k$-algebra. Then, the $A$-module of K\"ahler differentials relative to the ring map $k \to A$ corepresents the functor of $k$-linear derivations from $A$, i.e. there is a natural isomorphism of functors $A\mod \to A\mod$ as follows:
                $$\Der_k(A, -) \cong \Hom_A(\Omega^1_{A/k}, -)$$
            Because of this, the category of $k$-linear derivations $d_M: A \to M$ admit an initial object, namely $d: A \to \Omega^1_{A/k}$. 
        \end{lemma}
        \begin{corollary}[$1$-forms are dual to derivations]
            $k$-linear derivations from $A$ to itself are dual to differential $1$-forms relative to $k \to A$ in the following manner:
                $$\Der_k(A) := \Der_k(A, A) \cong \Hom_A(\Omega^1_{A/k}, A)$$
        \end{corollary}
        \begin{lemma}[$1$-forms over polynomial algebras]
            \cite[\href{https://stacks.math.columbia.edu/tag/00RX}{Tag 00RX}]{stacks} Let $k$ be a commutative ring and fix some $n \in \Z_{\geq 0}$, and consider the canonical ring homomorphism $k \to k[v_1, ..., v_n]$. In this case, $\Omega^1_{[n]} := \Omega^1_{k[v_1, ..., v_n]/k}$ will be free and of finite rank $n$ as an $k[v_1, ..., v_n]$-module; in particular, it admits the set $\{dv_1, ..., dv_n\}$ as a $k[v_1, ..., v_n]$-linear basis.
        \end{lemma}
        \begin{corollary}[Derivations as differential operators]
            The $k$-module of $k$-linear derivations from $k[v_1, ..., v_n]$ to itself admit a particularly simple and useful description:
                $$\Der_k(k[v_1, ..., v_n]) \cong \bigoplus_{1 \leq i \leq n} k[v_1, ..., v_n] \del_{v_i}$$
        \end{corollary}

    \subsection{Centres of UCEs of current Lie algebras}
        \begin{convention} 
            From now on, we fix a finite-dimensional simple Lie algebra:
                $$\g$$
            over an algebraically closed field $k$ of characteristic $0$, equipped with a symmetric and non-degenerate invariant $k$-bilinear form $(-, -)_{\g}$. It is known that such a bilinear form is unique up to $k^{\x}$-multiples, so for all intents and purposes, it can be assumed to be the Killing form, though this assumption is not necessary. 
        \end{convention}

        \begin{convention}
            Let us fix a commutative $k$-algebra $A$.

            We shall be writing:
                $$\g_A := \g \tensor_k A$$
            and endow this $k$-vector space with the following Lie bracket:
                $$\forall x, y \in \g: \forall f, g \in A: [x f, y g]_{\g_A} := [x, y]_{\g} fg$$
        \end{convention}

        \begin{convention}
            Let $R \to S$ be a homomorphism of commutative rings. Then, let us write:
                $$\bar{\Omega}^1_{S/R} := \Omega^1_{S/R}/dS$$
            Note that this is only an $R$-module, not an $S$-module.
        \end{convention}
        
        \begin{theorem}[The Kassel realisation] \label{theorem: kassel_realisation}
            \cite[Corollary 3.5]{kassel_universal_central_extensions_of_lie_algebras} Let $k$ now an algebraically closed field of characteristic $0$ again, as in convention \ref{conv: a_fixed_finite_dimensional_simple_lie_algebra}. 

            For the perfect Lie $k$-algebra $\g_A$, we have that:
                $$\z(\uce(\g_A)) \cong \bar{\Omega}^1_{A/k}$$
        \end{theorem}
            \begin{proof}[Proof sketch]
                Kassel constructed in the proof of \cite[Theorem 3.3]{kassel_universal_central_extensions_of_lie_algebras} a $k$-linear map:
                    $$\e: \bigwedge^2 \g_A \to \bar{\Omega}^1_{A/k}$$
                by the formula:
                    $$\forall x, y \in \g, \forall f, g \in A: \e(x f, y g) := (x, y)_{\g} f \bar{d}g$$
                which can be shown - relying on the $\g$-invariance of the bilinear form $(-, -)_{\g}$ - to be an element of $H^2_{\Lie}(\g_A, k)$ and hence gives a central extension $\fraku$ of $\g_A$ by $\bar{\Omega}_{A/k}^1$, whose underlying $k$-vector space is:
                    $$\g_A \oplus \bar{\Omega}_{A/k}^1$$
                and whose Lie bracket is:
                    $$[-, -]_{\fraku} = [-, -]_{\g_A} + \e$$
                It is also not hard to see that there is a section map $\g_A \to \fraku$, implying that $\fraku$ is simply connected in the sense of definition \ref{def: simply_connected_lie_algebras}. Via proposition \ref{prop: UCEs_are_simply_connected}, we then see that:
                    $$\fraku \cong \uce(\g_A)$$
                concluding the proof.
            \end{proof}
        \begin{example}[UCEs of multiloop Lie algebras]
            Fix some $n \in \Z_{\geq 0}$.
        
            Consider the case:
                $$A := A_{[n]} := k[v_1^{\pm 1}, ..., v_n^{\pm 1}]$$
            (let $A_{[0]} := k$). Set:
                $$\g_{[n]} := \g \tensor_k A_{[n]}$$
                $$\tilde{\g}_{[n]} := \uce(\g_{[n]})$$
                $$\Omega^1_{[n]} := \Omega^1_{A_{[n]}/k}, \bar{\Omega}^1_{[n]} := \bar{\Omega}^1_{A_{[n]}/k}$$
            Much can be said about the centre $\bar{\Omega}^1_{[n]}$ of the UCE $\tilde{\g}_{[n]}$ of $\g_{[n]}$, especially in the cases where $n \leq 2$, where it is rather easy to provide an explicit basis for the $k$-vector space $\bar{\Omega}^1_{[n]}$. 

            We know that $\Omega^1_{[n]}$ is free and of rank $n$ on the set:
                $$\{dv_1, ..., dv_n\}$$
            This implies that $\bar{\Omega}^1_{[n]}$ is generated by elements:
                $$\bar{d}v_j$$
            that are subjected to the following relation:
                $$0 = \bar{d}( v_1^{m_1} ... v_n^{m_n} ) = \sum_{1 \leq j \leq n} m_j v_1^{m_1} ... v_j^{m_j - 1} ... v_n^{m_n} \bar{d}v_j$$
            From this, one infers that the elements:
                $$m_j^{-1} v_1^{m_1} ... v_j^{m_j - 1} ... v_n^{m_n} \bar{d}v_j$$
            form a basis for $\bar{\Omega}_{[n]}$ as a $k$-vector space.

            When $n = 0$ or $n = 1$, it is trivial to see that:
                $$\dim_k \bar{\Omega}^1_{[0]} \cong 0, \dim_k \bar{\Omega}^1_{[1]} = 1$$

            When $n = 2$, consider the following: the $k$-vector space:
                $$\bar{\Omega}_{[2]} := \bar{\Omega}^1_{A_{[2]}/k}$$
            now decomposes as a $k$-vector space in the following manner:
                $$\bar{\Omega}_{[2]} \cong ( \bigoplus_{(r, s) \in \Z^2} k K_{r, s}) \oplus k c_v \oplus k c_t$$
            wherein:
                $$
                    K_{r, s} :=
                    \begin{cases}
                        \text{$\frac1s v^{r - 1} t^s \bar{d}v$ if $(r, s) \in \Z \x (\Z \setminus \{0\})$}
                        \\
                        \text{$-\frac1r v^r t^{-1} \bar{d}t$ if $(r, s) \in (\Z \setminus \{0\}) \x \{0\}$}
                        \\
                        \text{$0$ if $(r, s) = (0, 0)$}
                    \end{cases}
                $$
                $$c_v := v^{-1} \bar{d}v, c_t := t^{-1} \bar{d}t$$
            In fact, any element of the form:
                $$v^m t^p \bar{d}(v^n t^q) \in \bar{\Omega}_{[2]}$$
            can be written in terms of the basis vectors $K_{r, s}, c_v, c_t$ in the following manner:
                $$v^m t^p \bar{d}(v^n t^q) = \delta_{(m, p) + (n, q), (0, 0)} ( n c_v + q c_t ) + (np - mq) K_{m + n, p + q}$$
        \end{example}
        \begin{remark}[The $\Z$-grading on $\toroidal$] \label{remark: Z_gradings_on_toroidal_lie_algebras}
            If $k$ is an arbitrary commutative ring and $A$ is a $\Z$-graded commutative $k$-algebra, say:
                $$A := \bigoplus_{n \in \Z} A_n$$
            and if $\a$ is a perfect Lie algebra over $k$, then $\a_A$ will also be $\Z$-graded, specifically in the following manner:
                $$\a_A := \a \tensor_k A \cong \bigoplus_{n \in \Z} \a \tensor_k A_n$$
            and for convenience, let us write $\a_{A_n} := \a \tensor_k A_n$ for each $n \in \Z$. This grading on $\a_A$ actually extends to the whole of $\uce(\a_A)$. Because the $A$-module $\Omega^1_{A/k}$ is generated by the set:
                $$\{da\}_{a \in A}$$
            whose elements are subjected to the relations:
                $$\forall a, b \in A: d(ab) - a d(b) - d(a) b = 0$$
           there is an induced $\Z$-grading on $\Omega^1_{A/k}$ given by:
                $$\deg d(ab) = \deg a d(b) = \deg d(a) b = \deg a + \deg b - 1$$
            for all $a, b \in A$. Inside $\Omega^1_{A/k}$, now viewed as a $k$-module, one has the $k$-submodule $\im d$, which is also $\Z$-graded: the grading is given like above, namely:
                $$\deg d(a) = \deg a - 1$$
            This $\Z$-grading induces another one on $\bar{\Omega}^1_{A/k}$, given by:
                $$\deg \bar{d}(ab) = \deg a \bar{d}(b) = \deg \bar{d}(a) b = \deg a + \deg b - 1$$
            for all $a, b \in A$.

            Now, let us focus once more on the case:
                $$A := A_{[2]}$$
            wherein the relevant $\Z$-grading is given by:
                $$\deg v := 0, \deg t := 1$$
            Since we know that the basis elements of $\bar{\Omega}_{[2]}$ are given by:
                $$
                    K_{r, s} :=
                    \begin{cases}
                        \text{$\frac1s v^{r - 1} t^s \bar{d}v$ if $(r, s) \in \Z \x (\Z \setminus \{0\})$}
                        \\
                        \text{$-\frac1r v^r t^{-1} \bar{d}t$ if $(r, s) \in (\Z \setminus \{0\}) \x \{0\}$}
                        \\
                        \text{$0$ if $(r, s) = (0, 0)$}
                    \end{cases}
                $$
                $$c_v := v^{-1} \bar{d}v, c_t := t^{-1} \bar{d}t$$
            (cf. \textit{loc. cit.}) their respective degrees with respect to the $\Z$-grading on $\bar{\Omega}_{[2]} \cong \bar{\Omega}_{[2]}$ are:
                $$
                    \deg K_{r, s} =
                    \begin{cases}
                        \text{$s - 1$ if $(r, s) \in \Z \x (\Z \setminus \{0\})$}
                        \\
                        \text{$-1$ if $(r, s) \in (\Z \setminus \{0\}) \x \{0\}$}
                        \\
                        \text{$0$ if $(r, s) = (0, 0)$}
                    \end{cases}
                $$
                $$\deg c_v = \deg c_t = -1$$
        \end{remark}

    \subsection{Chevalley-Serre presentations for UCEs of current Lie algebras}
        Finally, let us demonstrate how Kassel's realisation is useful for writing down presentations in the style of Chevalley-Serre for Lie algebras of the kind $\g_A$.

        \begin{convention}[Weight spaces]
            If $\fraku$ is a symmetrisable Kac-Moody algebra and $V$ is a $\fraku$-module, then for each weight $\lambda \in \Pi(V)$, we shall be denoting the corresponding weight space by $V[\lambda]$.
        \end{convention}

        \begin{theorem}[Chevalley-Serre presentations for current Lie algebras] \label{theorem: chevalley_serre_presentations_for_UCEs}
            (Cf. \cite[Definition 3.1 and Corollary 3.4]{kassel_universal_central_extensions_of_lie_algebras}) Again, let $k$ be an algebraically closed field of characteristic $0$ and $A$ be a commutative $k$-algebra. 

            The Lie algebra:
                $$\tilde{\g}_A := \uce(\g_A)$$
            (which we know to exist, since $\g_A$ is perfect; cf. proposition \ref{prop: perfect_lie_algebras_admit_UCEs}) is then isomorphic to the Lie algebra generated by the set:
                $$\{ E_{\alpha, f}, H_{\alpha, f} \}_{(\alpha, f) \in \Phi \x A}$$
            whose elements are subjected to the following relations:
                $$\forall \alpha \in \Phi: \forall \lambda, \mu \in k: \forall f, g \in A: E_{\alpha, \lambda f + \mu g} = \lambda E_{\alpha, f} + \mu E_{\alpha, g}$$
                $$
                    \forall (\alpha, f), (\beta, g) \in \Phi \x A: [E_{\alpha, f}, E_{\beta, g}] =
                    \begin{cases}
                        \text{$N_{\alpha, \beta} E_{\alpha + \beta, fg}$ if $\alpha + \beta \in \Phi$}
                        \\
                        \text{$H_{\alpha, fg}$ if $\alpha + \beta = 0$}
                        \\
                        \text{$0$ if $\alpha \in \beta \not \in \Phi \cup \{0\}$}
                    \end{cases}
                $$
                $$
                    \forall (\alpha, f), (\beta, g) \in \Phi \x A:
                    \begin{cases}
                        [H_{\alpha, f}, E_{\beta, g}] = (\alpha, \check{\beta})_{\g} E_{\alpha, fg}
                        \\
                        [H_{\alpha, f}, H_{\beta, g}] = 0
                    \end{cases}    
                $$
            wherein $N_{\alpha, \beta}$ are the structural constants\footnote{Note that there is only one such constant for each commutator, since the root spaces of $\g$ are equally $1$-dimensional \textit{a priori}.} of the commutators $[e_{\alpha}, e_{\beta}]_{\g}$, for some choices of root vectors $e_{\alpha} \in \g[\alpha], e_{\beta} \in \g[\beta]$, i.e.:
                $$[e_{\alpha}, e_{\beta}]_{\g} := N_{\alpha, \beta} e_{\alpha + \beta}$$
            for some root vector $e_{\alpha + \beta} \in \g[\alpha + \beta]$.
        \end{theorem}
        \begin{corollary}
            If $A$ is graded by some abelian group $Z$, then the Lie algebra $\tilde{\g}_A$ will be graded by the abelian group $Q \x Z$. 
        \end{corollary}

        A construction that we will frequent throughout the remainder of the subsection is the untwisted affine Kac-Moody algebra associated to a finite-dimensional simple Lie algebra.
        \begin{convention} \label{conv: a_fixed_untwisted_affine_kac_moody_algebra}
            Let us write:
                $$\hat{\g} \cong \g[v^{\pm 1}] \oplus k \hat{c} \rtimes k \hat{D}$$
            to denote the untwisted affine Kac-Moody algebra associated to $\g$. For details, we refer the reader to \cite[Chapter 7]{kac_infinite_dimensional_lie_algebras}.

            Fix a Cartan subalgebra $\hat{\h}$ of $\hat{\g}$. The affine Dynkin diagram associated to $\hat{\g}$ (cf. \cite[Chapter 4]{kac_infinite_dimensional_lie_algebras}) shall be denoted by:
                $$\hat{\Gamma} := ( \hat{\Gamma}_0, \hat{\Gamma}_1 )$$
            with $\hat{\Gamma}_0$ denoting the set of vertices and $\hat{\Gamma}_1$ denoting the set of undirected edges.

            Also, let us denote the highest root of in $\Phi$ by $\theta$. Note that we have a bijection (but not literally an equality; cf. \cite[Chapter 6]{kac_infinite_dimensional_lie_algebras}):
                $$\hat{\Gamma}_0 \cong \Gamma_0 \cup \{\theta\}$$
        \end{convention}

        Before we move on, let us consider a quick example illustrating theorem \ref{theorem: chevalley_serre_presentations_for_UCEs}.
        \begin{example}
            For what follows, let us recall from \cite[Chapter 7]{kac_infinite_dimensional_lie_algebras} that the root space decomposition of the untwisted affine Kac-Moody algebra $\hat{\g}$ takes the form:
                $$\hat{\g} \cong \hat{\h} \oplus \bigoplus_{\beta \in \Re(\hat{\Phi})} \hat{\g}[\beta] \oplus \bigoplus_{\beta \in \Im(\hat{\Phi})} \hat{\g}[\beta]$$
            in which the untwisted affine root system $\hat{\Phi}$ decomposes into a disjoint union of the subsets of real and imaginary roots:
                $$\hat{\Phi} \cong \Re(\hat{\Phi}) \cup \Im(\hat{\Phi})$$
            where:
                $$\Re(\hat{\Phi}) \cong \Phi + \Z\delta \cong \Phi \x \Z$$
                $$\Im(\hat{\Phi}) \cong (\Z \setminus \{0\})\delta$$
            and the corresponding root spaces are given by:
                $$\forall \alpha + m\delta \in \Re(\hat{\Phi}): \hat{\g}[\alpha + m\delta] \cong \g[\alpha] v^m$$
                $$\forall r\delta \in \Im(\hat{\Phi}): \hat{\g}[r\delta] \cong \h v^r$$
            Recall also - from \cite[Chapter 5]{kac_infinite_dimensional_lie_algebras} - that:
                $$\forall \alpha + m \delta \in \Re(\hat{\Phi}): \dim_k \hat{\g}[\alpha + m\delta] = 1$$
            entirely as a consequence of the fact that $\dim_k \g[\alpha] = 1$. 
        
            Now, consider the Lie algebra:
                $$\tilde{\g} := \uce(\g[v^{\pm 1}])$$
            which is nothing but the derived subalgebra of $\hat{\g}$. Per theorem \ref{theorem: chevalley_serre_presentations_for_UCEs}, $\tilde{\g}$ admits the following Chevalley-Serre presentation. It is isomorphic to the Lie algebra generated by the set:
                $$\{ E_{i, r}^{\pm} \}_{(i, r) \in \hat{\Gamma}_0 \x \Z}$$
            whose elements are subjected to the following relations:
                $$[E_{i, r}^{\pm}, E_{j, s}^{\pm}] = \pm N_{\alpha_i, \alpha_j} E_{i + j, r + s}^{\pm}$$
            The isomorphism in question is given by:
                $$\forall (i, r) \in \Gamma_0 \x \Z: E_{i, r}^{\pm} \mapsto e_i^{\pm} v^r$$
                $$E_{\theta, r}^{\pm} \mapsto e_{\theta}^{\mp} v^r + \hat{c}$$
            where $e_{\theta}^{\mp} \in \g[\mp \theta]$ are choices of root vectors (well-defined since $\dim_k \g[\mp \theta] = 1$).
        \end{example}

        \begin{convention}[Toroidal and positive-toroidal Lie algebras]
            From now on, we will be writing:
                $$A_{[2]} := k[v^{\pm 1}, t^{\pm 1}], A_{[2]}^+ := k[v^{\pm 1}, t]$$
                $$\g_{[2]} := \g \tensor_k A_{[2]}, \g_{[2]}^+ := \g \tensor_k A_{[2]}^+$$
                $$\bar{\Omega}_{[2]} := \bar{\Omega}^1_{A_{[2]}/k}, \bar{\Omega}^+_{[2]} := \bar{\Omega}^1_{A_{[2]^+}/k}$$

            We will be considering the Lie algebras:
                $$\toroidal := \uce(\g_{[2]})$$
                $$\toroidal^+ := \uce(\g_{[2]}^+)$$
            As the Lie algebras $\g_{[2]}$ and $\g_{[2]}^+$ are perfect, the Lie algebras $\toroidal$ and $\toroidal^+$ are well-defined and per theorem \ref{theorem: kassel_realisation}, their centres are respectively given by:
                $$\z_{[2]} := \z(\uce(\toroidal)) \cong \bar{\Omega}_{[2]}$$
                $$\z_{[2]}^+ := \z(\uce(\toroidal^+)) \cong \bar{\Omega}_{[2]}^+$$
        \end{convention}
        \begin{remark}[The centre of $\toroidal^+$]
            It is not hard to see, from the description of $\bar{\Omega}_{[2]}$ in remark \ref{remark: Z_gradings_on_toroidal_lie_algebras}, that:
                $$\z_{[2]}^+ := \bigoplus_{(r, s) \in \Z \x \Z_{> 0}} k K_{r, s} \oplus k c_v$$
        \end{remark}

        For our own convenience, let us note that when:
            $$A \cong k[v^{\pm 1}, t^{\pm 1}]$$
        the Chevalley-Serre presentation for $\toroidal$ can be described more succinctly as follows. This simplified presentation is an intermediary step towards a presentation for $\toroidal$ in terms of only elements of degrees $0$ and $1$ (in $t$), which ought to be simpler to work with than the full Chevalley-Serre presentation (cf. proposition \ref{prop: levendorskii_presentation__for_central_extensions_of_multiloop_algebras}). It should be noted that the existence of such a presentation in terms of low-degree generators is a rather curious phenomenon, as it originates from the fact that - at least when $\hat{\g}$ is made to satisfy some mild technical restrictions (cf. lemma \ref{lemma: levendorskii_presentation_for_yangians_of_symmetrisable_kac_moody_algebras}) - the universal enveloping algebra:
            $$\rmU(\toroidal^+)$$
        admits the \say{formal Yangian}:
            $$\rmY_{\hbar}(\hat{\g})$$
        associated to $\hat{\g}$ as a  flat deformation over $k[\hbar]$, and that the latter is known to possess a presentation in terms of low-degree generators (cf. lemma \ref{lemma: levendorskii_presentation_for_yangians_of_symmetrisable_kac_moody_algebras}).
        \begin{lemma}[Chevalley-Serre presentation for $\toroidal$] \label{lemma: chevalley_serre_presentation_for_central_extensions_of_multiloop_algebras}
            (Cf. \cite[Proposition 6.6]{wendlandt_formal_shift_operators_on_yangian_doubles}) The Lie algebra $\toroidal$ is isomorphic to the Lie algebra generated by the set:
                $$\{ E_{i, r}^{\pm}, H_{i, r} \}_{(i, r) \in \hat{\Gamma}_0 \x \Z} \cup \{ K \}$$
            whose elements are subjected to the following relations, given for all $(i, r), (j, s) \in \hat{\Gamma}_0 \x \Z$:
                $$[ H_{i, r}, H_{j, s} ] = 0$$
                $$[ H_{i, r}, E_{j, s}^{\pm} ] = \pm (\alpha_j, \check{\alpha}_i) E_{j, r + s}^{\pm}$$
                $$[ E_{i, r}^+, E_{j, s}^- ] = \delta_{ij} H_{i, r + s}$$
                $$[ E_{i, r + 1}^{\pm}, E_{j, s}^{\pm} ] - [ E_{i, r}^{\pm}, E_{j, s + 1}^{\pm} ] = 0$$
                $$[K, \toroidal] = 0$$
            The isomorphism in question is given as follows, for all $(i, r) \in \hat{\Gamma}_0 \x \Z$:
                $$\forall (i, r) \in \Gamma_0 \x \Z: E_{i, r}^{\pm} \mapsto e_i^{\pm} t^r, H_{i, r} \mapsto h_i t^r$$
                $$\forall (i, r) \in \{\theta\} \x \Z: E_{\theta, r}^{\pm} \mapsto e_{\theta}^{\mp} v^{\pm 1} t^r, H_{\theta, r} \mapsto h_{\theta} t^r + c_v t^r$$
                $$K \mapsto c_t$$
        \end{lemma}

        Let us now demonstrate how a low-degree presentation for the Lie algebra $\toroidal$ may be obtained, as eluded to earlier. This necessitates introducing the (formal) Yangian associated to the untwisted affine Kac-Moody algebra $\hat{\g}$ because as explained above, we will be relying on the existence of such a low-degree presentation for those (formal) Yangians.
        \begin{convention}[Yangians associated to symmetrisable Kac-Moody algebras]
            If $\fraku$ a general symmetrisable Kac-Moody algebra whose associated Cartan matrix is indecomposable. We refer the reader to \cite[Section 2]{guay_nakajima_wendlandt_affine_yangian_coproduct} for the definition of the \textbf{formal Yangian} $\rmY_{\hbar}(\fraku)$ and \textbf{Yangian} $\rmY(\fraku)$, as well as all relevant discussions about the various \say{basic} presentations of these associative algebras (living over $k[\hbar]$ and $k$ respectively). The only thing that we will note is that we will be denoting the Chevalley-Serre generators by:
                $$E_{i, r}^{\pm}, H_{i, r}$$
        \end{convention}
        \begin{convention}
            From now on, let us write:
                $$T_{i, 1}(\hbar) := H_{i, 1} - \frac12 \hbar H_{i, 0}^2$$
                $$T_{i, 1} := T_{i, 1}(1) = H_{i, 1} - \frac12 H_{i, 0}^2$$
        \end{convention}

        One key property of formal (affine) Yangians that we will be relying on is the fact that 
        \begin{lemma}[Formal Yangians as Rees algebras] \label{lemma: formal_yangians_as_rees_algebras}
            \cite[Theorem 6.10]{guay_nakajima_wendlandt_affine_yangian_vertex_representations_and_PBW} If $\fraku$ is a general indecomposable symmetrisable Kac-Moody algebra. If $\fraku$ is either of finite type but not $\sfA_1$ or of untwisted affine type but not $\sfA_1^{(1)}$ and not $\sfA_1^{(2)}$ then the natural \textit{graded} $k$-algebra homomoprhism:
                $$\rmY_{\hbar}(\fraku) \to \Rees_{\hbar} \rmY(\fraku)$$
            will be an isomorphism. 
        \end{lemma}
         \begin{corollary}[Formal affine Yangians as flat graded deformations] \label{coro: formal_affine_yangians_as_flat_graded_deformations}
            Suppose that $\g \not \cong \sl_2(k)$. Then the $k[\hbar]$-algebra:
                $$\rmY_{\hbar}(\hat{\g})$$
            will be a flat $\Z$-graded deformation of the $\Z$-graded $k$-algebra:
                $$\rmU(\toroidal^+)$$
         \end{corollary}
         
        The hypotheses of the following lemma are satisfied at least when $\fraku$ is a symmetrisable Kac-Moody algebra of either finite type or affine type, save for the types $\sfA_1^{(1)}$ and $\sfA_1^{(2)}$.
        \begin{lemma}[A Levendorskii-type presentation for Yangians of symmetrisable Kac-Moody algebras] \label{lemma: levendorskii_presentation_for_yangians_of_symmetrisable_kac_moody_algebras}
            \cite[Theorem 2.13]{guay_nakajima_wendlandt_affine_yangian_coproduct} Suppose for a moment that $\fraku$ is a general symmetrisable Kac-Moody algebra whose Cartan matrix is:
            \begin{itemize}
                \item indecomposable,
                \item such that, for any $i < j \in \Gamma_0$ (with respect to some choice of total ordering on $\Gamma_0$) the following $2 \x 2$ matrix is invertible:
                    $$
                        \begin{pmatrix}
                            c_{ii} & c_{ij}
                            \\
                            c_{ji} & c_{ji}
                        \end{pmatrix}
                    $$
            \end{itemize}
            The formal Yangian $\rmY_{\hbar}(\fraku)$ of $\fraku$ will then be isomorphic to the associative $k$-algebra generated by the set:
                $$\{ H_{i, r}, E_{i, r}^{\pm} \}_{(i, r) \in \Gamma_0 \x \N}$$
            whose elements are subjected to the following relations\footnote{... and it is understood that the elements $H_{i, 0} = h_i, E_{i, 0}^{\pm} = e_i^{\pm}$ satisfy the Chevalley-Serre relations defining $\fraku$; cf. \cite[Chapter 1]{kac_infinite_dimensional_lie_algebras}.}:
                $$H_{i, 0} = h_i, E_{i, 0}^{\pm} = e_i^{\pm}$$
                $$[ H_{i, r}, H_{j, s} ] = 0$$
                $$[ H_{i, 0}, E_{j, s}^{\pm} ] = \pm c_{ij} E_{j, s}^{\pm}$$
                $$[ E_{i, r}^+, E_{j, s}^- ] = \pm \delta_{ij} H_{i, r + s}$$
                $$\left[ T_{i, 1}(\hbar), E_{j, 0}^{\pm} \right] = \pm \hbar c_{ij} E_{j, 1}^{\pm}$$
                $$[ E_{i, 1}^{\pm}, E_{j, 0}^{\pm} ] - [ E_{i, 0}^{\pm}, E_{j, 1}^{\pm} ] = \pm \frac12 \hbar c_{ij} \{E_{i, 0}^{\pm}, E_{j, 0}^{\pm}\}$$
        \end{lemma}
        \begin{proposition}[Levendorskii presentation for $\toroidal^+$] \label{prop: levendorskii_presentation__for_central_extensions_of_multiloop_algebras}
            Suppose that $\hat{\g}$ is not of type $\sfA_1^{(1)}$. 
        
            The Lie algebra $\toroidal^+$ is isomorphic to the Lie algebra  generated by the set:
                $$\{ E_{i, r}^{\pm}, H_{i, r} \}_{(i, r) \in \hat{\Gamma}_0 \x \Z_{\geq 0}}$$
            whose elements are subjected to the following relations, given for all $(i, r), (j, s) \in \hat{\Gamma}_0 \x \Z_{\geq 0}$:
                $$H_{i, 0} = h_i, E_{i, 0}^{\pm} = e_i^{\pm}$$
                $$[ H_{i, r}, H_{j, s} ] = 0$$
                $$[ H_{i, 0}, E_{j, s}^{\pm} ] = \pm (\alpha_j, \check{\alpha}_i) E_{j, s}^{\pm}$$
                $$[ E_{i, r}^+, E_{j, s}^- ] = \delta_{ij} H_{i, r + s}$$
                $$[ E_{i, 1}^{\pm}, E_{j, 0}^{\pm} ] - [ E_{i, 0}^{\pm}, E_{j, 1}^{\pm} ] = 0$$
            The isomorphism in question is as in lemma \ref{lemma: chevalley_serre_presentation_for_central_extensions_of_multiloop_algebras}.
        \end{proposition}
            \begin{proof}
                Combine corollary \ref{coro: formal_affine_yangians_as_flat_graded_deformations} with lemma \ref{lemma: levendorskii_presentation_for_yangians_of_symmetrisable_kac_moody_algebras}. Note that flatness (in particular, $\hbar$-torsion-freeness) is crucial for our application of corollary \ref{coro: formal_affine_yangians_as_flat_graded_deformations}.
            \end{proof}
        \begin{corollary}[Levendorskii presentation for $\toroidal$]
            Suppose that $\hat{\g}$ is not of type $\sfA_1^{(1)}$. 
        
            The Lie algebra $\toroidal$ is isomorphic to the Lie algebra  generated by the set:
                $$\{ E_{i, r}^{\pm}, H_{i, r} \}_{(i, r) \in \hat{\Gamma}_0 \x \Z} \cup \{ K \}$$
            whose elements are subjected to the following relations, given for all $(i, r), (j, s) \in \hat{\Gamma}_0 \x \Z$:
                $$H_{i, 0} = h_i, E_{i, 0}^{\pm} = e_i^{\pm}$$
                $$[ H_{i, r}, H_{j, s} ] = 0$$
                $$[ H_{i, 0}, E_{j, s}^{\pm} ] = \pm (\alpha_j, \check{\alpha}_i) E_{j, s}^{\pm}$$
                $$[ E_{i, r}^+, E_{j, s}^- ] = \delta_{ij} H_{i, r + s}$$
                $$[ E_{i, 1}^{\pm}, E_{j, 0}^{\pm} ] - [ E_{i, 0}^{\pm}, E_{j, 1}^{\pm} ] = 0$$
                $$[K, \toroidal] = 0$$
            The isomorphism in question is as in lemma \ref{lemma: chevalley_serre_presentation_for_central_extensions_of_multiloop_algebras}.
        \end{corollary}
            \begin{proof}
                Combine proposition \ref{prop: levendorskii_presentation__for_central_extensions_of_multiloop_algebras} with lemma \ref{lemma: chevalley_serre_presentation_for_central_extensions_of_multiloop_algebras}. 
            \end{proof}

        \begin{question}
            Is a low-degree presentation in the style of proposition \ref{prop: levendorskii_presentation__for_central_extensions_of_multiloop_algebras} available for the Lie algebras $\uce(\g_{[n]})$ for $n > 2$ ? Does this depend on the existence of a construction of \say{$(n - 1)$-affine Yangians} which are graded flat deformations of $\rmU(\g[v_1^{\pm 1}, ..., v_{n - 1}^{\pm 1}, v_n])$ ?
        \end{question}