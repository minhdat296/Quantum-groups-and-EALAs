\section{The Kassel-Zusmanovich realisation of UCEs of current Lie algebras}
    \subsection{A recollection of K\"ahler differentials}
        There are many perspectives on algebraic differential forms, but for our purposes, the following will be the easiest to use.
        \begin{definition}[Modules of K\"ahler differentials] \label{def: kahler_differentials}
            Let $k$ be a base commutative ring and let $A$ be a commutative $k$-algebra, defined by a multiplication map:
                $$\mu_{A/k}: A \tensor_k A \to A$$
            The $A$-module of K\"ahler differentials $\Omega^1_{A/k}$ relative to the ring map $k \to A$ is then given by:
                $$\Omega^1_{A/k} := I/I^2$$
            where $I := \ker \mu_{A/k}$. Elements of $\Omega^1_{A/k}$ are typically referred to as (differential) $1$-forms.
        \end{definition}
        \begin{remark}[Diffentials satify the Leibniz rule]
            Observe that the $A \tensor_k A$-ideal:
                $$I := \ker \mu_{A/k}$$
            is generated by elements of the form $1 \tensor f - f \tensor 1$, for all $f \in A$. We then see that:
                $$\Omega^1_{A/k} := I/I^2 \cong I \tensor_{A \tensor_k A} ( (A \tensor_k A)/I ) \cong I \tensor_{A \tensor_k A} A$$
            should we regard $A$ as a commutative $A \tensor_k A$-algebra; note that the last isomorphism holds thanks to the fact that the multiplication map $\mu_{A/k}: A \tensor_k A \to A$ is \textit{a priori} surjective. From this, one infers that there exists a canonical $k$-linear derivation:
                $$d: A \to \Omega^1_{A/k}$$
                $$f \mapsto 1 \tensor f - f \tensor 1$$
            Indeed, for every $f, g \in A$, the Leibniz rule is satisfied:
                $$g df + f dg = g(1 \tensor f - f \tensor 1) + f(1 \tensor g - g \tensor 1) = gf \tensor 1 - fg \tensor 1 = d(fg)$$
        \end{remark}
        The following well-known lemmas are very useful. A proof can be be found in any standard reference on general commutative algebra (cf. e.g. \cite[\href{https://stacks.math.columbia.edu/tag/00AO}{Tag 00AO}]{stacks}).
        \begin{lemma}[Universal property of modules of K\"ahler differentials]
            Let $k$ be a base commutative ring and let $A$ be a commutative $k$-algebra. Then, the $A$-module of K\"ahler differentials relative to the ring map $k \to A$ corepresents the functor of $k$-linear derivations from $A$, i.e. there is a natural isomorphism of functors $A\mod \to A\mod$ as follows:
                $$\Der_k(A, -) \cong \Hom_A(\Omega^1_{A/k}, -)$$
            Because of this, the category of $k$-linear derivations $d_M: A \to M$ admit an initial object, namely $d: A \to \Omega^1_{A/k}$. 
        \end{lemma}
        \begin{corollary}[$1$-forms are dual to derivations]
            $k$-linear derivations from $A$ to itself are dual to differential $1$-forms relative to $k \to A$ in the following manner:
                $$\Der_k(A) := \Der_k(A, A) \cong \Hom_A(\Omega^1_{A/k}, A)$$
        \end{corollary}
        \begin{lemma}[$1$-forms over polynomial algebras]
            \cite[\href{https://stacks.math.columbia.edu/tag/00RX}{Tag 00RX}]{stacks} Let $k$ be a commutative ring and fix some $n \in \Z_{\geq 0}$, and consider the canonical ring homomorphism $k \to k[v_1, ..., v_n]$. In this case, $\Omega^1_{[n]} := \Omega^1_{k[v_1, ..., v_n]/k}$ will be free and of finite rank $n$ as an $k[v_1, ..., v_n]$-module; in particular, it admits the set $\{dv_1, ..., dv_n\}$ as a $k[v_1, ..., v_n]$-linear basis.
        \end{lemma}
        \begin{corollary}[Derivations as differential operators]
            The $k$-module of $k$-linear derivations from $k[v_1, ..., v_n]$ to itself admit a particularly simple and useful description:
                $$\Der_k(k[v_1, ..., v_n]) \cong \bigoplus_{1 \leq i \leq n} k[v_1, ..., v_n] \del_{v_i}$$
        \end{corollary}

        \begin{definition}[Cyclic $1$-forms]
            Suppose that $k$ is a commutative ring and $A$ a commutative $k$-algebra. Then, the space of \textbf{cyclic differential $1$-forms} relative to the ring map $k \to A$ is the $k$-module given by:
                $$\bar{\Omega}^1_{A/k} := \Omega^1_{A/k}/\im d$$
        \end{definition}

    \subsection{Centres of UCEs of current Lie algebras}
        \begin{convention} \label{conv: a_fixed_finite_dimensional_simple_lie_algebra}
            From now on, we fix a finite-dimensional simple Lie algebra:
                $$\g$$
            over an algebraically closed field $k$ of characteristic $0$, equipped with a symmetric and non-degenerate invariant $k$-bilinear form $(-, -)_{\g}$. It is known that such a bilinear form is unique up to $k^{\x}$-multiples, so for all intents and purposes, it can be assumed to be the Killing form, though this assumption is not necessary. 
    
            Suppose also that $\g$ is equipped with a basis $\{x_i\}_{1 \leq i \leq \dim_k \g}$ and with respect to $(-, -)_{\g}$, we identify a dual basis $\{x_i^*\}_{1 \leq i \leq \dim_k \g}$. Recall that the Casimir tensor/canonical element of $\g$ is:
                $$\sfr_{\g} := \sum_{1 \leq i \leq \dim_k \g} x_i \tensor x_i^* \in \g \tensor_k \g^*$$
            and recall that $\sfr_{\g}$ is independent of what we choose the basis vectors $x_i$ to be.
    
            Eventually, we will also be concerned with the Dynkin diagram associated to the root system of $\g$. Let us denote this by:
                $$\Gamma := (\Gamma_0, \Gamma_1)$$
            wherein $\Gamma_0$ means the (finite) set of vertices and $\Gamma_1$ means the set of undirected edges between said vertices. 
    
            The set of all roots and respectively, positive/negative roots, and simple roots of $\g$ shall be denoted by:
                $$\Phi, \Phi^{\pm}, \Phi^{\circ}$$
            The set $\Phi^{\circ}$ is $\Z$-linearly independent and its $\Z$-span:
                $$Q := \Z \Phi^{\circ}$$
            is typically referred to as the root lattice of $\g$. Recall also that there is a set of fundamental weights: if we write:
                $$\check{\Phi}$$
            for the set of coroots of $\g$, then the so-called weight lattice of $\g$ shall be given by\footnote{We avoid the usual $\Delta$ notation, as we would like to reserve this symbol for a coproduct construction on affine Yangians.}:
                $$\Pi := \Hom_{\Z}(\check{Q}, \Z), \check{Q} := \Z\check{\Phi}$$
            inside which lies the set of fundamental weights, whose elements are dual to those of $\check{\Phi}^{\circ}$ (i.e. dual to simple coroots) with respect to $(-, -)_{\g}$\footnote{Which we might as well normalise so that $(\alpha_j, \check{\alpha}_j)_{\g} = 2$ for every $j \in \Gamma_0$, and hence the fundamental weights $\lambda_i$ will be simply be subjected to the relation $\delta_{ij} = 2 \frac{(\lambda_i, \check{\alpha}_i)_{\g}}{(\alpha_j, \check{\alpha}_j)_{\g}} = (\lambda_i, \check{\alpha}_i)_{\g}$.}.
        \end{convention}

        \begin{convention}
            Let us fix a commutative $k$-algebra $A$.

            We shall be writing:
                $$\g_A := \g \tensor_k A$$
            and endow this $k$-vector space with the following Lie bracket:
                $$\forall x, y \in \g: \forall f, g \in A: [x f, y g]_{\g_A} := [x, y]_{\g} fg$$
        \end{convention}

        \begin{theorem}[The Kassel-Zusmanovich realisation] \label{theorem: kassel_realisation}
            
        \end{theorem}