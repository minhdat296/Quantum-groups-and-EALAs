\section{The Kassel realisation of UCEs of current Lie algebras}
    \subsection{A recollection of K\"ahler differentials}
        There are many perspectives on algebraic differential forms, but for our purposes, the following will be the easiest to use.
        \begin{definition}[Modules of K\"ahler differentials] \label{def: kahler_differentials}
            Let $k$ be a base commutative ring and let $A$ be a commutative $k$-algebra, defined by a multiplication map:
                $$\mu_{A/k}: A \tensor_k A \to A$$
            The $A$-module of K\"ahler differentials $\Omega^1_{A/k}$ relative to the ring map $k \to A$ is then given by:
                $$\Omega^1_{A/k} := I/I^2$$
            where $I := \ker \mu_{A/k}$. Elements of $\Omega^1_{A/k}$ are typically referred to as (differential) $1$-forms.
        \end{definition}
        \begin{remark}[Diffentials satify the Leibniz rule]
            Observe that the $A \tensor_k A$-ideal:
                $$I := \ker \mu_{A/k}$$
            is generated by elements of the form $1 \tensor f - f \tensor 1$, for all $f \in A$. We then see that:
                $$\Omega^1_{A/k} := I/I^2 \cong I \tensor_{A \tensor_k A} ( (A \tensor_k A)/I ) \cong I \tensor_{A \tensor_k A} A$$
            should we regard $A$ as a commutative $A \tensor_k A$-algebra; note that the last isomorphism holds thanks to the fact that the multiplication map $\mu_{A/k}: A \tensor_k A \to A$ is \textit{a priori} surjective. From this, one infers that there exists a canonical $k$-linear derivation:
                $$d: A \to \Omega^1_{A/k}$$
                $$f \mapsto 1 \tensor f - f \tensor 1$$
            Indeed, for every $f, g \in A$, the Leibniz rule is satisfied:
                $$g df + f dg = g(1 \tensor f - f \tensor 1) + f(1 \tensor g - g \tensor 1) = gf \tensor 1 - fg \tensor 1 = d(fg)$$
        \end{remark}
        
        The following well-known lemmas are very useful. Proofs can be be found in any standard reference on general commutative algebra (e.g. \cite[\href{https://stacks.math.columbia.edu/tag/00AO}{Tag 00AO}]{stacks}).
        \begin{lemma}[Universal property of modules of K\"ahler differentials]
            \cite[\href{https://stacks.math.columbia.edu/tag/00RO}{Tag 00RO}]{stacks} Let $k$ be a base commutative ring and let $A$ be a commutative $k$-algebra. Then, for any $A$-module $M$, there exists a natural isomorphism of $A$-modules\footnote{The LHS is the $A$-module of $k$-linear derivations from $A$ into $M$.}:
                $$\Der_k(A, M) \cong \Hom_A(\Omega^1_{A/k}, M)$$
        \end{lemma}
        \begin{corollary}
            The category of $k$-linear derivations $d_M: A \to M$ admits an initial object, namely $d: A \to \Omega^1_{A/k}$, which is to say that for every such derivation $d_M: A \to M$, there exists a \textit{unique} $A$-module homomorphism $\varphi: \Omega^1_{A/k} \to M$ such that:
                $$d_M = \varphi \circ d$$
            and the $A$-module $\Omega^1_{A/k}$ itself is unique up to unique isomorphisms satisfying the aforementioned compositional property.
        \end{corollary}
        \begin{corollary}[$1$-forms are dual to derivations] \label{coro: 1_forms_are_dual_to_vector_fields}
            $k$-linear derivations from $A$ to itself are dual to differential $1$-forms relative to $k \to A$ in the following manner:
                $$\Der_k(A) := \Der_k(A, A) \cong \Hom_A(\Omega^1_{A/k}, A)$$
            By definition, elements of $D \in \Der_k(A)$ satisfy the Leibniz rule (i.e. $D(ab) = b D(a) + a D(b)$ for every $a, b \in A$), so the above tells us also that $\Omega^1_{A/k}$ is isomorphic to the $A$-module generated by the set:
                $$\{da\}_{a \in A}$$
            whose elements are constrained by the relations:
                $$\forall a, b \in A: d(ab) = b da + a db$$
        \end{corollary}
        \begin{remark}
            If $\Omega^1_{A/k}$ is finite free of rank $n$ over $A$, e.g.:
                $$\Omega^1_{A/k} \cong \bigoplus_{1 \leq i \leq n} A dv_i$$
            then we can identify:
                $$\Der_k(A) \cong \bigoplus_{1 \leq i \leq n} A \del_{v_i}$$
            where $\del_{v_i} \in \Der_k(A)$ are the preimages under the isomorphism $\Der_k(A) \xrightarrow[]{\cong} \Hom_A(\Omega^1_{A/k}, A)$ of the $A$-linear duals of the generators $dv_i \in \Omega^1_{A/k}$. 
        \end{remark}
        \begin{lemma}[$1$-forms over polynomial algebras] \label{lemma: 1_forms_over_polynomial_algebras}
            \cite[\href{https://stacks.math.columbia.edu/tag/00RX}{Tag 00RX}]{stacks} Let $k$ be a commutative ring and fix some $n \in \Z_{\geq 0}$. In this case, $\Omega^1_{k[v_1, ..., v_n]/k}$ will be free and of finite rank $n$ as an $k[v_1, ..., v_n]$-module; in particular, it admits the set $\{dv_1, ..., dv_n\}$ as a $k[v_1, ..., v_n]$-linear basis.
        \end{lemma}
        \begin{lemma}[Localisation of $1$-forms] \label{lemma: localisation_1_forms}
            \cite[\href{https://stacks.math.columbia.edu/tag/031G}{Tag 031G}]{stacks} Let $k$ be a field\footnote{... so that the only prime ideal of $k$ would be $(0)$.} and fix some $n \in \Z_{\geq 0}$, and consider the canonical ring homomorphism $k \to k[v_1, ..., v_n]$. Then, for any $1 \leq i \leq n$, there will be a $k[v_1, ..., v_n][v_i^{-1}]$-module isomorphism:
                $$\Omega^1_{k[v_1, ..., v_n][v_i^{-1}]/k} \cong \Omega^1_{k[v_1, ..., v_n]/k}[v_i^{-1}]$$
        \end{lemma}

        Let us end this subsection with the following examples, which will be useful for what comes later on.
        \begin{example}
            Let $k$ be a field.
        
            Per lemma \ref{lemma: 1_forms_over_polynomial_algebras}, we know that:
                $$\Omega^1_{k[v, t]/k} \cong k[v, t] dv \oplus k[v, t] dv$$
            Using lemma \ref{lemma: localisation_1_forms}, we then see that:
                $$\Omega^1_{k[v^{\pm 1}, t^{\pm 1}]/k} \cong k[v^{\pm 1}, t^{\pm 1}] dv \oplus k[v^{\pm 1}, t^{\pm 1}] dt$$
                $$\Omega^1_{k[v^{\pm 1}, t]/k} \cong k[v^{\pm 1}, t] dv \oplus k[v^{\pm 1}, t] dt$$
        \end{example}

    \subsection{Centres of UCEs of current Lie algebras}
        \begin{convention} \label{conv: a_fixed_finite_dimensional_simple_lie_algebra}
            From now on, we fix a finite-dimensional simple Lie algebra:
                $$\g$$
            over an algebraically closed field $k$ of characteristic $0$, equipped with a symmetric and non-degenerate invariant $k$-bilinear form $(-, -)_{\g}$. It is known that such a bilinear form is unique up to $k^{\x}$-multiples, so for all intents and purposes, it can be assumed to be the Killing form, though this assumption is not necessary. 
    
            Suppose also that $\g$ is equipped with a basis $\{x_i\}_{1 \leq i \leq \dim_k \g}$ and with respect to $(-, -)_{\g}$, we identify a dual basis $\{x_i^*\}_{1 \leq i \leq \dim_k \g}$. Recall that the Casimir tensor/canonical element of $\g$ is:
                $$\sfr_{\g} := \sum_{1 \leq i \leq \dim_k \g} x_i \tensor x_i^* \in \g \tensor_k \g^*$$
            and recall that $\sfr_{\g}$ is independent of what we choose the basis vectors $x_i$ to be.
    
            The set of all roots and respectively, positive/negative roots, and simple roots of $\g$ shall be denoted by:
                $$\Phi, \Phi^{\pm}, \simpleroots$$
            The set $\simpleroots$ is $\Z$-linearly independent and its $\Z$-span:
                $$Q := \Z \simpleroots$$
            is typically referred to as the root lattice of $\g$. Recall also that there is a set of fundamental weights: if we write:
                $$\check{\Phi}$$
            for the set of coroots of $\g$, then the so-called weight lattice of $\g$ shall be given by\footnote{We avoid the usual $\Delta$ notation, as we would like to reserve this symbol for a coproduct construction on affine Yangians.}:
                $$\Pi := \Hom_{\Z}(\check{Q}, \Z), \check{Q} := \Z\check{\Phi}$$
            inside which lies the set of fundamental weights, whose elements are dual to those of $\check{\Phi}^{\circ}$ (i.e. dual to simple coroots) with respect to $(-, -)_{\g}$\footnote{Which we might as well normalise so that $(\alpha_j, \check{\alpha}_j)_{\g} = 2$ for every $j \in \Gamma_0$, and hence the fundamental weights $\lambda_i$ will be simply be subjected to the relation $\delta_{ij} = 2 \frac{(\lambda_i, \check{\alpha}_i)_{\g}}{(\alpha_j, \check{\alpha}_j)_{\g}} = (\lambda_i, \check{\alpha}_i)_{\g}$.}.
        \end{convention}

        \begin{convention}
            Let us fix a commutative $k$-algebra $A$.

            We shall endow the $k$-vector space:
                $$\g \tensor_k A$$
            with the following Lie bracket:
                $$\forall x, y \in \g: \forall f, g \in A: [x f, y g]_{\g \tensor_k A} := [x, y]_{\g} fg$$
        \end{convention}

        \begin{convention}
            Let $R \to S$ be a homomorphism of commutative rings. Then, let us write:
                $$\bar{\Omega}^1_{S/R} := \Omega^1_{S/R}/dS$$
            Note that this is only an $R$-module, not an $S$-module. Let us also write $\bar{d}: S \to \bar{\Omega}^1_{S/R}$ for the canonical composition:
                $$
                    \begin{tikzcd}
                	S & {\Omega^1_{S/R}} \\
                	& {\bar{\Omega}^1_{S/R}}
                	\arrow["d", from=1-1, to=1-2]
                	\arrow[two heads, from=1-2, to=2-2]
                	\arrow["{\bar{d}}"', from=1-1, to=2-2]
                    \end{tikzcd}
                $$
        \end{convention}
        
        \begin{theorem}[The Kassel realisation] \label{theorem: kassel_realisation}
            \cite[Corollary 3.5]{kassel_universal_central_extensions_of_lie_algebras} For the perfect Lie $k$-algebra $\g \tensor_k A$, we have that:
                $$\z(\uce(\g \tensor_k A)) \cong \bar{\Omega}^1_{A/k}$$
        \end{theorem}
        Kassel constructed in the proof of \cite[Theorem 3.3]{kassel_universal_central_extensions_of_lie_algebras} a $k$-linear map:
            $$\e: \bigwedge^2 (\g \tensor_k A) \to \bar{\Omega}^1_{A/k}$$
        by the formula:
            $$\forall x, y \in \g, \forall f, g \in A: \e(x f, y g) := (x, y)_{\g} f \bar{d}g$$
        which can be shown - relying on the $\g$-invariance of the bilinear form $(-, -)_{\g}$ - to be an element of $H^2_{\Lie}(\g \tensor_k A, k)$ and hence gives a central extension $\fraku$ of $\g \tensor_k A$ by $\bar{\Omega}_{A/k}^1$, whose underlying $k$-vector space is:
            $$\g \tensor_k A \oplus \bar{\Omega}_{A/k}^1$$
        and whose Lie bracket is:
            $$[-, -]_{\fraku} := [-, -]_{\g \tensor_k A} + \e$$
        We will not go into the details of why this Lie algebra extension is initial amongst central extensions of $\g \tensor_k A$, but let us at least convince ourselves that $[-, -]_{\fraku}$ is a well-defined Lie bracket. By construction, it is already bilinear and skew-symmetric, so the only thing to show is that it satisfies the Jacobi identity. To this end, pick $x, y, z \in \g$ and $f, g, h \in A$ and then consider the following:
            $$
                \begin{aligned}
                    & [xf, [yg, zh]_{\fraku}]_{\fraku} + [yg, [zh, xf]_{\fraku}]_{\fraku} + [zh, [xf, yg]_{\fraku}]_{\fraku}
                    \\
                    & = [xf, [y, z]_{\g} gh + \e(yg, zh)]_{\fraku} + [yg, [z, x]_{\g} hf + \e(zh, xf)]_{\fraku} + [zh, [x, y]_{\g} fg + \e(xf, yg)]_{\fraku}
                    \\
                    & = [xf, [y, z]_{\g} gh]_{\fraku} + [yg, [z, x]_{\g} hf]_{\fraku} + [zh, [x, y]_{\g} fg]_{\fraku}
                    \\
                    & = 
                    \begin{aligned}
                        & \left( [x, [y, z]_{\g}]_{\g} fgh + \e(xf, [y, z]_{\g} gh) \right)
                        \\
                        + & \left( [y, [z, x]_{\g}]_{\g} ghf + \e(yg, [z, x]_{\g} hf) \right)
                        \\
                        + & \left( [z, [x, y]_{\g}]_{\g} hfg + \e(zh, [x, y]_{\g} fg) \right)
                    \end{aligned}
                    \\
                    & = \e(xf, [y, z]_{\g} gh) + \e(yg, [z, x]_{\g} hf) + \e(zh, [x, y]_{\g} fg)
                    \\
                    & = (x, [y, z]_{\g})_{\g} f \bar{d}(gh) + (y, [z, x]_{\g})_{\g} g \bar{d}(hf) + (z, [x, y]_{\g})_{\g} h \bar{d}(fg)
                    \\
                    & = (x, [y, z]_{\g})_{\g} ( f \bar{d}(gh) + g \bar{d}(hf) + h \bar{d}(fg) )
                    \\
                    & = 0
                \end{aligned}
            $$
        \begin{example}[UCEs of multiloop Lie algebras]
            Fix some $n \in \Z_{\geq 0}$.
        
            Consider the case:
                $$A := k[v_1^{\pm 1}, ..., v_n^{\pm 1}]$$
            Much can be said about the centre $\bar{\Omega}^1_{A/k}$ of the UCE of $\g[v_1^{\pm 1}, ..., v_n^{\pm 1}] \cong \g \tensor_k k[v_1^{\pm 1}, ..., v_n^{\pm 1}]$, especially in the cases where $n \leq 2$, where it is rather easy to provide an explicit basis for the $k$-vector space $\bar{\Omega}^1_{A/k}$. 

            We know that $\bar{\Omega}^1_{A/k}$ is free and of rank $n$ on the set:
                $$\{dv_1, ..., dv_n\}$$
            This implies that $\bar{\Omega}^1_{[n]}$ is generated by elements:
                $$\bar{d}v_j$$
            that are subjected to the following relation:
                $$0 = \bar{d}( v_1^{m_1} ... v_n^{m_n} ) = \sum_{1 \leq j \leq n} m_j v_1^{m_1} ... v_j^{m_j - 1} ... v_n^{m_n} \bar{d}v_j$$
            From this, one infers that the elements:
                $$m_j^{-1} v_1^{m_1} ... v_j^{m_j - 1} ... v_n^{m_n} \bar{d}v_j$$
            form a basis for $\bar{\Omega}^1_{A/k}$ as a $k$-vector space.
        \end{example}
        \begin{example}
           It is trivial to see that:
                $$\dim_k \bar{\Omega}^1_{k/k} \cong 0$$
            from which one sees that:
                $$\uce(\g) \cong \g$$
            i.e. $\g$ is its own universal central extension, and hence every central extension of $\g$ is trivial.
        \end{example}
        \begin{example}[Affine Lie algebras]
            Let us compute the UCE of $\g[v^{\pm 1}]$. From this, we can construct the so-called \say{untwisted affine Kac-Moody algebra} attached to $\g$ (cf. \cite[Chapter 7]{kac_infinite_dimensional_lie_algebras}). 

            To this end, let us firstly compute the underlying vector space of the centre of $\uce(\g[v^{\pm 1}])$. Abstractly, we know that it is isomorphic to $\bar{\Omega}^1_{k[v^{\pm 1}]/k}$, and it is also known that:
                $$\Omega^1_{k[v^{\pm 1}]/k} \cong k[v^{\pm 1}] dv \cong \bigoplus_{m \in \Z} k \cdot v^m dv$$
            so the only non-trivial computation to make is that of $d k[v^{\pm 1}]$. For this, consider:
                $$d k[v^{\pm 1}] \cong d k[v] \oplus d(v^{-1} k[v^{-1}]) \cong ( \bigoplus_{m \geq 1} k \cdot v^{m - 1} dv ) \oplus ( \bigoplus_{m \geq 0} k \cdot v^{-m - 1} dv ) \cong \bigoplus_{m \in \Z \setminus \{-1\}} k \cdot v^m dv$$
            which then tells us that:
                $$\bar{\Omega}^1_{k[v^{\pm 1}]/k} \cong k \cdot v^{-1} \bar{d}v$$
            The underlying vector space of $\uce(\g[v^{\pm 1}])$ is thus isomorphic to:
                $$\g[v^{\pm 1}] \oplus k \cdot v^{-1} \bar{d}v$$

            Per theorem \ref{theorem: kassel_presentations_for_UCEs}, we know that the Lie bracket on $\uce(\g[v^{\pm 1}])$ is given by:
                $$[x f, y g]_{\g[v^{\pm 1}]} = [x, y]_{\g} fg + (x, y)_{\g} f \bar{d}g$$
            for all $x, y \in \g$ and all $f, g \in k[v^{\pm 1}]$. Since:
                $$f \bar{d}g \in k \cdot v^{-1} \bar{d}v$$
            necessarily, the bracket can be given simplier as:
                $$[x f, y g]_{\g[v^{\pm 1}]} = [x, y]_{\g} fg + (x, y)_{\g} c(f, g) v^{-1} \bar{d}v$$
            which is unique up to a choice of a scalar $c(f, g) \in k$ depending on $f, g \in k[v^{\pm 1}]$; usually, $c(f, g) := \Res_{v = 0}(fg)$.

            Note that there is an invariant symmetric $k$-bilinear form on $\uce(\g[v^{\pm 1}])$ given by:
                $$(x f, y g)_{\uce(\g[v^{\pm 1}])} := (x, y)_{\g} c(f, g)$$
            By invariance, this bilinear form is necessarily degenerate, but by adding an extra element $D$ to the Lie algebra $\uce(\g[v^{\pm 1}])$ and by requiring that:
                $$(v^{-1} \bar{d}v, D)_{\uce(\g[v^{\pm 1}]) \oplus k D} = 1$$
                $$(v^{-1} \bar{d}v, v^{-1} \bar{d}v)_{\uce(\g[v^{\pm 1}]) \oplus k D} = (D, D)_{\uce(\g[v^{\pm 1}]) \oplus k D} = 0$$
            one obtains the \textbf{untwisted affine Kac-Moody algebra} (in the sense of \cite[Chapter 7]{kac_infinite_dimensional_lie_algebras}):
                $$\hat{\g} := \uce(\g[v^{\pm 1}]) \rtimes k D$$
            If $c(f, g) := \Res_{v = 0}(fg)$, the element $D$ can be shown to be the derivation on $\g[v^{\pm 1}]$ given by $\id_{\g} \tensor v \frac{d}{dv}$. It can also be shown, in general, that:
                $$[D, v^{-1} \bar{d}v]_{\uce(\g[v^{\pm 1}]) \oplus k D} = 0$$

            Let us also note that unlike $\uce(\g[v^{\pm 1}])$, the UCE of the perfect Lie algebra $\g[v]$ is trivial, since:
                $$\forall f, g \in k[v]: \Res_{v = 0}(fg) = 0$$
        \end{example}
        \begin{example}[Toroidal Lie algebras]
            Next, let us compute the UCE of $\g[v^{\pm 1}, t^{\pm 1}]$. 
            
            Firstly, let us compute its underlying vector space, for which the only non-trivial computation to make is that of the $k$-vector space $\bar{\Omega}^1_{k[v^{\pm 1}, t^{\pm 1}]/k}$, which we know to be isomorphic to the centre of $\uce(\g[v^{\pm 1}, t^{\pm 1}])$. We know that:
                $$\Omega^1_{k[v^{\pm 1}, t^{\pm 1}]/k} \cong k[v^{\pm 1}, t^{\pm 1}] dv \oplus k[v^{\pm 1}, t^{\pm 1}] dt$$
            and since:
                $$k[v^{\pm 1}, t^{\pm 1}] \cong \bigoplus_{(m, p) \in \Z^2} k \cdot v^m t^p$$
            we consequently have that:
                $$\Omega^1_{k[v^{\pm 1}, t^{\pm 1}]/k} \cong \bigoplus_{(m, p) \in \Z^2} (k \cdot v^m t^p dv \oplus k \cdot v^m t^p dt)$$
            It can also be seen that:
                $$d k[v^{\pm 1}, t^{\pm 1}] \cong \bigoplus_{(m, p) \in \Z^2} k \cdot d(v^m t^p) \cong \bigoplus_{(m, p) \in \Z^2} ( k \cdot m v^{m - 1} t^p dv \oplus k \cdot p v^m t^{p - 1} dt )$$
            As such, the $k$-vector space:
                $$\bar{\Omega}^1_{k[v^{\pm 1}, t^{\pm 1}]/k} := \Omega^1_{k[v^{\pm 1}, t^{\pm 1}]/k}/d k[v^{\pm 1}, t^{\pm 1}]$$
            decomposes in the following manner:
                $$\bar{\Omega}^1_{k[v^{\pm 1}, t^{\pm 1}]/k} \cong ( \bigoplus_{(r, s) \in \Z^2} k K_{r, s}) \oplus k c_v \oplus k c_t$$
            wherein:
                $$
                    K_{r, s} :=
                    \begin{cases}
                        \text{$\frac1s v^{r - 1} t^s \bar{d}v$ if $(r, s) \in \Z \x (\Z \setminus \{0\})$}
                        \\
                        \text{$-\frac1r v^r t^{-1} \bar{d}t$ if $(r, s) \in (\Z \setminus \{0\}) \x \{0\}$}
                        \\
                        \text{$0$ if $(r, s) = (0, 0)$}
                    \end{cases}
                $$
                $$c_v := v^{-1} \bar{d}v, c_t := t^{-1} \bar{d}t$$
            In fact, any element of the form:
                $$v^m t^p \bar{d}(v^n t^q) \in \bar{\Omega}^1_{k[v^{\pm 1}, t^{\pm 1}]/k}$$
            can be written in terms of the basis vectors $K_{r, s}, c_v, c_t$ in the following manner:
                $$v^m t^p \bar{d}(v^n t^q) = \delta_{(m, p) + (n, q), (0, 0)} ( n c_v + q c_t ) + (np - mq) K_{m + n, p + q}$$
            (cf. \cite[pp. 35]{wendlandt_formal_shift_operators_on_yangian_doubles}).

            Finally, let us note that the Lie bracket on $\uce(\g[v^{\pm 1}, t^{\pm 1}])$ is given by:
                $$
                    \begin{aligned}
                        [x v^m t^p, y v^n t^q]_{\uce(\g[v^{\pm 1}, t^{\pm 1}])} & = [x, y]_{\g} v^{m + n} t^{p + q} + (x, y)_{\g} v^m t^p \bar{d}(v^n t^q)
                        \\
                        & = [x, y]_{\g} v^{m + n} t^{p + q} + ( \delta_{(m, p) + (n, q), (0, 0)} ( n c_v + q c_t ) + (np - mq) K_{m + n, p + q} )
                    \end{aligned}
                $$
            for all $x, y \in \g$ and all $(m, p), (n, q) \in \Z^2$. As a side note, let us note that interestingly, unlike how $\g[v]$ admits only the trivial UCE, $\g[v^{\pm 1}, t]$ admits a non-trivial UCE, on which the Lie bracket is given by:
                $$[x v^m t^p, y v^n t^q]_{\uce(\g[v^{\pm 1}, t^{\pm 1}])} = [x, y]_{\g} v^{m + n} t^{p + q} + (np - mq) K_{m + n, p + q}$$
            for all $x, y \in \g$ and all $(m, p), (n, q) \in \Z \x \Z_{\geq 0}$.
        \end{example}
        \begin{remark}[The $\Z$-grading on toroidal Lie algebras] \label{remark: Z_gradings_on_toroidal_lie_algebras}
            If $k$ is an arbitrary commutative ring and $A$ is a $\Z$-graded commutative $k$-algebra, say:
                $$A := \bigoplus_{n \in \Z} A_n$$
            and if $\a$ is a perfect Lie algebra over $k$, then $\a_A$ will also be $\Z$-graded, specifically in the following manner:
                $$\a_A := \a \tensor_k A \cong \bigoplus_{n \in \Z} \a \tensor_k A_n$$
            and for convenience, let us write $\a_{A_n} := \a \tensor_k A_n$ for each $n \in \Z$. This grading on $\a_A$ actually extends to the whole of $\uce(\a_A)$. Because the $A$-module $\Omega^1_{A/k}$ is generated by the set:
                $$\{da\}_{a \in A}$$
            whose elements are subjected to the relations:
                $$\forall a, b \in A: d(ab) - a d(b) - d(a) b = 0$$
           there is an induced $\Z$-grading on $\Omega^1_{A/k}$ given by:
                $$\deg d(ab) = \deg a d(b) = \deg d(a) b = \deg a + \deg b - 1$$
            for all $a, b \in A$. Inside $\Omega^1_{A/k}$, now viewed as a $k$-module, one has the $k$-submodule $\im d$, which is also $\Z$-graded: the grading is given like above, namely:
                $$\deg d(a) = \deg a - 1$$
            This $\Z$-grading induces another one on $\bar{\Omega}^1_{A/k}$, given by:
                $$\deg \bar{d}(ab) = \deg a \bar{d}(b) = \deg \bar{d}(a) b = \deg a + \deg b - 1$$
            for all $a, b \in A$.

            When:
                $$A := k[v^{\pm 1}, t^{\pm 1}]$$
            consider the $\Z$-grading given by:
                $$\deg v := 0, \deg t := 1$$
            Since we know that the basis elements of $\bar{\Omega}^1_{k[v^{\pm 1}, t^{\pm 1}]/k}$ are given by:
                $$
                    K_{r, s} :=
                    \begin{cases}
                        \text{$\frac1s v^{r - 1} t^s \bar{d}v$ if $(r, s) \in \Z \x (\Z \setminus \{0\})$}
                        \\
                        \text{$-\frac1r v^r t^{-1} \bar{d}t$ if $(r, s) \in (\Z \setminus \{0\}) \x \{0\}$}
                        \\
                        \text{$0$ if $(r, s) = (0, 0)$}
                    \end{cases}
                $$
                $$c_v := v^{-1} \bar{d}v, c_t := t^{-1} \bar{d}t$$
            (cf. \textit{loc. cit.}) their respective degrees with respect to the $\Z$-grading on $\bar{\Omega}^1_{k[v^{\pm 1}, t^{\pm 1}]/k}$ are:
                $$
                    \deg K_{r, s} =
                    \begin{cases}
                        \text{$s - 1$ if $(r, s) \in \Z \x (\Z \setminus \{0\})$}
                        \\
                        \text{$-1$ if $(r, s) \in (\Z \setminus \{0\}) \x \{0\}$}
                        \\
                        \text{$0$ if $(r, s) = (0, 0)$}
                    \end{cases}
                $$
                $$\deg c_v = \deg c_t = -1$$
        \end{remark}

    \subsection{Chevalley-Serre presentations for UCEs of current Lie algebras}
        Finally, let us demonstrate how Kassel's realisation is useful for writing down presentations in the style of Chevalley-Serre for Lie algebras of the kind $\g \tensor_k A$.

        \begin{convention}[Weight spaces]
            If $\fraku$ is a symmetrisable Kac-Moody algebra and $V$ is a $\fraku$-module, then for each weight $\lambda \in \Pi(V)$, we shall be denoting the corresponding weight space by $V[\lambda]$.
        \end{convention}

        \begin{theorem}[Kassel's presentations for current Lie algebras] \label{theorem: kassel_presentations_for_UCEs}
            (Cf. \cite[Definition 3.1 and Corollary 3.4]{kassel_universal_central_extensions_of_lie_algebras}) Again, let $k$ be an algebraically closed field of characteristic $0$ and $A$ be a commutative $k$-algebra. 

            The Lie algebra:
                $$\tilde{\g}_A := \uce(\g \tensor_k A)$$
            (which we know to exist, since $\g \tensor_k A$ is perfect; cf. proposition \ref{prop: perfect_lie_algebras_admit_UCEs}) is then isomorphic to the Lie algebra generated by the set:
                $$\{ X_{\alpha, f}, H_{\alpha, f} \}_{(\alpha, f) \in \Phi \x A}$$
            whose elements are subjected to the following relations:
                $$\forall \alpha \in \Phi: \forall \lambda, \mu \in k: \forall f, g \in A: X_{\alpha, \lambda f + \mu g} = \lambda X_{\alpha, f} + \mu X_{\alpha, g}$$
                $$
                    \forall (\alpha, f), (\beta, g) \in \Phi \x A: [X_{\alpha, f}, X_{\beta, g}] =
                    \begin{cases}
                        \text{$N_{\alpha, \beta} X_{\alpha + \beta, fg}$ if $\alpha + \beta \in \Phi$}
                        \\
                        \text{$H_{\alpha, fg}$ if $\alpha + \beta = 0$}
                        \\
                        \text{$0$ if $\alpha \in \beta \not \in \Phi \cup \{0\}$}
                    \end{cases}
                $$
                $$
                    \forall (\alpha, f), (\beta, g) \in \Phi \x A:
                    \begin{cases}
                        [H_{\alpha, f}, X_{\beta, g}] = (\alpha, \check{\beta})_{\g} X_{\alpha, fg}
                        \\
                        [H_{\alpha, f}, H_{\beta, g}] = 0
                    \end{cases}    
                $$
            wherein $N_{\alpha, \beta}$ are the structural constants\footnote{Note that there is only one such constant for each commutator, since the root spaces of $\g$ are equally $1$-dimensional \textit{a priori}.} of the commutators $[x_{\alpha}, x_{\beta}]_{\g}$, for some choices of root vectors $x_{\alpha} \in \g_{\alpha}, x_{\beta} \in \g_{\beta}$, i.e.:
                $$[x_{\alpha}, x_{\beta}]_{\g} := N_{\alpha, \beta} x_{\alpha + \beta}$$
            for some root vector $x_{\alpha + \beta} \in \g[\alpha + \beta]$.
        \end{theorem}
        \begin{corollary}
            If $A$ is graded by some abelian group $Z$, then the Lie algebra $\tilde{\g}_A$ will be graded by the abelian group $Q \x Z$. 
        \end{corollary}

        A construction that we will frequent throughout the remainder of the subsection is the untwisted affine Kac-Moody algebra associated to a finite-dimensional simple Lie algebra.
        \begin{convention} \label{conv: a_fixed_untwisted_affine_kac_moody_algebra}
            Let us write:
                $$\hat{\g} \cong \g[v^{\pm 1}] \oplus k \hat{c} \rtimes k \hat{D}$$
            to denote the untwisted affine Kac-Moody algebra associated to $\g$. For details, we refer the reader to \cite[Chapter 7]{kac_infinite_dimensional_lie_algebras}.
            
            We shall also fix once and for all a Cartan subalgebra $\hat{\h}$ of $\hat{\g}$, along with a non-degenerate, invariant, and symmetric $k$-bilinear form:
                $$(-, -)_{\hat{\g}}$$
            on $\hat{\g}$ (cf. \cite[Chapter 2]{kac_infinite_dimensional_lie_algebras}). The set of simple roots of $\hat{\g}$ with respect to the aforementioned choices of Cartan subalgebra and non-degenerate invariant bilinear form will be denoted by:
                $$\hat{\simpleroots}$$
        \end{convention}

        Before we move on, let us consider a quick example illustrating theorem \ref{theorem: kassel_presentations_for_UCEs}.
        \begin{example}
            For what follows, let us recall from \cite[Chapter 7]{kac_infinite_dimensional_lie_algebras} that the root space decomposition of the untwisted affine Kac-Moody algebra $\hat{\g}$ takes the form:
                $$\hat{\g} \cong \hat{\h} \oplus \bigoplus_{\beta \in \Re(\hat{\Phi})} \hat{\g}_{\beta} \oplus \bigoplus_{\beta \in \Im(\hat{\Phi})} \hat{\g}_{\beta}$$
            in which the untwisted affine root system $\hat{\Phi}$ decomposes into a disjoint union of the subsets of real and imaginary roots:
                $$\hat{\Phi} \cong \Re(\hat{\Phi}) \cup \Im(\hat{\Phi})$$
            where:
                $$\Re(\hat{\Phi}) \cong \Phi + \Z\delta \cong \Phi \x \Z$$
                $$\Im(\hat{\Phi}) \cong (\Z \setminus \{0\})\delta$$
            and the corresponding root spaces are given by:
                $$\forall \alpha + m\delta \in \Re(\hat{\Phi}): \hat{\g}[\alpha + m\delta] \cong \g_{\alpha} v^m$$
                $$\forall r\delta \in \Im(\hat{\Phi}): \hat{\g}[r\delta] \cong \h v^r$$
            Recall also - from \cite[Chapter 5]{kac_infinite_dimensional_lie_algebras} - that:
                $$\forall \alpha + m \delta \in \Re(\hat{\Phi}): \dim_k \hat{\g}[\alpha + m\delta] = 1$$
            entirely as a consequence of the fact that $\dim_k \g_{\alpha} = 1$. 
        
            Now, consider the Lie algebra:
                $$\tilde{\g} := \uce(\g[v^{\pm 1}])$$
            which is nothing but the derived subalgebra of $\hat{\g}$. Per theorem \ref{theorem: kassel_presentations_for_UCEs}, $\tilde{\g}$ admits the following presentation. It is isomorphic to the Lie algebra generated by the set:
                $$\{ X_{i, r}^{\pm} \}_{(i, r) \in \hat{\simpleroots} \x \Z}$$
            whose elements are subjected to the following relations:
                $$[X_{i, r}^{\pm}, X_{j, s}^{\pm}] = \pm N_{\alpha_i, \alpha_j} X_{i + j, r + s}^{\pm}$$
            The isomorphism in question is given by:
                $$\forall (i, r) \in \simpleroots \x \Z: X_{i, r}^{\pm} \mapsto x_i^{\pm} v^r$$
                $$X_{\theta, r}^{\pm} \mapsto x_{\theta}^{\mp} v^r + \hat{c}$$
            where $x_{\theta}^{\mp} \in \g[\mp \theta]$ are choices of root vectors (well-defined since $\dim_k \g[\mp \theta] = 1$).
        \end{example}

        \begin{convention}[Toroidal and positive-toroidal Lie algebras]
            From now on, we will be writing:
                $$A := k[v^{\pm 1}, t^{\pm 1}], A^{\positive} := k[v^{\pm 1}, t]$$
                $$\g_{[2]} := \g \tensor_k A, \g_{[2]}^{\positive} := \g \tensor_k A^{\positive}$$
                $$\bar{\Omega}_{[2]} := \bar{\Omega}^1_{A/k}, \bar{\Omega}^+_{[2]} := \bar{\Omega}^1_{A_{[2]^+}/k}$$

            We will be considering the Lie algebras:
                $$\toroidal := \uce(\g_{[2]})$$
                $$\toroidal^{\positive} := \uce(\g_{[2]}^{\positive})$$
            As the Lie algebras $\g_{[2]}$ and $\g_{[2]}^{\positive}$ are perfect, the Lie algebras $\toroidal$ and $\toroidal^{\positive}$ are well-defined and per theorem \ref{theorem: kassel_realisation}, their centres are respectively given by:
                $$\z_{[2]} := \z(\uce(\toroidal)) \cong \bar{\Omega}_{[2]}$$
                $$\z_{[2]}^{\positive} := \z(\uce(\toroidal^{\positive})) \cong \bar{\Omega}_{[2]}^{\positive}$$
        \end{convention}
        \begin{remark}[The centre of $\toroidal^{\positive}$]
            It is not hard to see, from the description of $\bar{\Omega}_{[2]}$ in remark \ref{remark: Z_gradings_on_toroidal_lie_algebras}, that:
                $$\z_{[2]}^{\positive} := \bigoplus_{(r, s) \in \Z \x \Z_{> 0}} k K_{r, s} \oplus k c_v$$
        \end{remark}
        For $\toroidal$ and $\toroidal^{\positive}$, Kassel's presentation from theorem \ref{theorem: kassel_presentations_for_UCEs} can be futhermore refined into Chevalley-Serre-style presentations from which one can obtain $\hat{Q}$-gradings for these two Lie algebras. These presentations were originally due to Moody-Rao-Yokonuma (see \cite{moody_rao_yokonuma_vertex_representations_of_toroidal_lie_algebras}). 
        \begin{lemma}[Chevalley-Serre presentation for $\toroidal$] \label{lemma: chevalley_serre_presentation_for_central_extensions_of_multiloop_algebras}
            The Lie algebra $\toroidal$ is isomorphic to the Lie algebra generated by the set:
                $$\{ X_{i, r}^{\pm}, H_{i, r} \}_{(i, r) \in \hat{\simpleroots} \x \Z} \cup \{ K \}$$
            whose elements are subjected to the following relations, given for all $(i, r), (j, s) \in \hat{\simpleroots} \x \Z$:
                $$[ H_{i, r}, H_{j, s} ] = 0$$
                $$[ H_{i, r}, X_{j, s}^{\pm} ] = \pm (\alpha_j, \check{\alpha}_i) X_{j, r + s}^{\pm}$$
                $$[ X_{i, r}^+, X_{j, s}^- ] = \delta_{ij} H_{i, r + s}$$
                $$[ X_{i, r + 1}^{\pm}, X_{j, s}^{\pm} ] - [ X_{i, r}^{\pm}, X_{j, s + 1}^{\pm} ] = 0$$
                $$\ad(X_{i, 0}^{\pm})^{1 - c_{ij}}( X_{j, s}^{\pm} ) = 0$$
                $$[K, \toroidal] = 0$$
            The isomorphism in question is given as follows, for all $(i, r) \in \hat{\simpleroots} \x \Z$:
                $$\forall (i, r) \in \simpleroots \x \Z: X_{i, r}^{\pm} \mapsto x_i^{\pm} t^r, H_{i, r} \mapsto h_i t^r$$
                $$\forall (i, r) \in \{\theta\} \x \Z: X_{\theta, r}^{\pm} \mapsto x_{\theta}^{\mp} v^{\pm 1} t^r, H_{\theta, r} \mapsto h_{\theta} t^r + c_v t^r$$
                $$K \mapsto c_t$$
        \end{lemma}
        \begin{corollary}[Chevalley-Serre presentation for $\toroidal^{\positive}$]
            The Lie algebra $\toroidal^{\positive}$ is isomorphic to the Lie algebra generated by the set:
                $$\{ X_{i, r}^{\pm}, H_{i, r} \}_{(i, r) \in \hat{\simpleroots} \x \Z_{\geq 0}}$$
            whose elements are subjected to the following relations lemma \ref{lemma: chevalley_serre_presentation_for_central_extensions_of_multiloop_algebras}. The isomorphism in question is given also as in \textit{loc. cit.}
        \end{corollary}
            \begin{proof}
                Apply remark \ref{remark: Z_gradings_on_toroidal_lie_algebras} to lemma \ref{lemma: chevalley_serre_presentation_for_central_extensions_of_multiloop_algebras}.
            \end{proof}