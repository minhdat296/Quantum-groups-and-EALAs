\section{The initial setup}
    Again, $\g$ is a finite-dimensional simple Lie algebra over an algebraically closed field of characteristic $0$. It is accompanied by all the data listed in convention \ref{conv: a_fixed_finite_dimensional_simple_lie_algebra}.

    We will be firstly concerned with the Lie algebra:
        $$\g_{[2]} := \g[v^{\pm 1}, t^{\pm 1}]$$
    whose Lie bracket is given by:
        $$\forall x, y \in \g: \forall f, g \in A: [x f, y g]_{\g_{[2]}} := [x, y]_{\g} fg$$
    A crucial detail for us is that instead of being equipped with the usual residue bilinear form of degree $(0, 0)$, this Lie algebra will be equipped with the residue bilinear form of degree $(0, -1)$:
        $$\forall x, y \in \g: \forall (m, p), (n, q) \in \Z^2: (x v^m t^p, y v^n t^q)_{\g_{[2]}} := (x, y)_{\g} \delta_{(m, p) + (n, q), (0, -1)}$$
    It is easy to see that the bilinear form:
        $$(-, -)_{\g_{[2]}}$$
    is symmetric, non-degenerate, and invariant. 

    By Kassel's Theorem (cf. theorem \ref{theorem: kassel_realisation}), we know that $\g_{[2]}$ admits a UCE, which we shall denote by:
        $$\toroidal := \uce(\g_{[2]})$$
    Clearly, any symmetric and invariant bilinear form on $\toroidal$ is necessarily degenerate, which prompts the question of what can be done to remedy this. The construction of Yangian extended affine Lie algebras is supposed to be a 

    Let us also abbreviate:
        $$\z_{[2]} := \z(\toroidal)$$
    and recall that:
        $$\z_{[2]} \cong \bar{\Omega}^1_{[2]}$$