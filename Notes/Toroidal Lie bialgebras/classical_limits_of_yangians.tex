\section{Toroidal Lie bialgebras as classical limits of formal Yangians}
    \subsection{Chevalley-Serre and Levendorskii presentations for toroidal Lie algebras}
        Let us now give a demonstration of the existence of a Chevalley-Serre-type presentation for the Lie algebras $\tilde{\g}_{[2]}$ and $\tilde{\g}_{[2]}^{\pm}$. The point of doing this is two-fold:
        \begin{enumerate}
            \item The formal Yangian $\rmY_{\hbar}(\hat{\g}_{[1]})$ (respectively, Yangian $\rmY(\hat{\g}_{[1]})$) associated to $\hat{\g}_{[1]}$ is given as a certain associative $\bbC$-algebra defined by a similar Chevalley-Serre-type presentation, and the goal is usually to somehow prove that once one reduces modulo $\hbar$ (respectively, take the associated graded algebra), one shall obtain a $\bbC$-algebra isomorphism with the universal enveloping algebra $\rmU(\tilde{\g}_{[2]}^+)$.
            \item The (formal) Yangian assoicated to $\hat{\g}_{[1]}$ enjoys a more convenient presentation by generators of degrees $0$ and $1$ (in the variable $t$) and hence so does $\tilde{\g}_{[2]}^+$ (and in fact, the Lie algebras $\tilde{\g}_{[2]}$ and $\tilde{\g}_{[2]}^-$; see proposition \ref{prop: levendorskii_presentation__for_central_extensions_of_multiloop_algebras} and corollary \ref{coro: levendorskii_presentation__for_central_extensions_of_multiloop_algebras}) which eventually, will become useful in the proof of theorem \ref{theorem: toroidal_lie_bialgebras} - wherein we establish the existence of a (topological\footnote{We will be clearer about what this means later.}) Lie bialgebra structure on $\tilde{\g}_{[2]}^+$. These presentations in terms of low-degree generators will be known as the Levendorskii-type presentations.
        \end{enumerate}

        \begin{convention}
            In this subsection, suppose that $\g$ is a general indecomposable Kac-Moody algebra. 
        \end{convention}
        When $\g$ is of finite type as in convention \ref{conv: a_fixed_finite_dimensional_simple_lie_algebra}, we will be observing the following convention.
        \begin{convention} \label{conv: a_fixed_untwisted_affine_kac_moody_algebra}
            We introduce the following notations, which deviate slightly from those of e.g. \cite[Chapter 7]{kac_infinite_dimensional_lie_algebras}, so as to be consistent with the rest of our notes.
        
            Let us write:
                $$\hat{\g}_{[1]} := \tilde{\g}_{[1]} \rtimes \d_{[1]}$$
            to mean the untwisted affine Kac-Moody algebra attached to the finite-dimensional simple Lie algebra $\g$ from convention \ref{conv: a_fixed_finite_dimensional_simple_lie_algebra} (cf. \cite[Chapter 7]{kac_infinite_dimensional_lie_algebras}). Here, $\g_{[1]} := \g[v^{\pm 1}]$ is equipped with the invariant inner product given by:
                $$(x v^m, y v^n)_{\g_{[1]}} := (x, y)_{\g} \delta_{m + n = 0}$$
            for all $x, y \in \g$ and all $m, n \in \Z$, and $\d_{[1]}$ is the $1$-dimensional Lie algebra spanned by the derivation (with notations as in remark \ref{remark: dual_of_toroidal_centres_contains_derivations}):
                $$D_{0, -1} \in \der_{\bbC}(\tilde{\g}_{[1]})$$
            on $\tilde{\g}_{[1]} := \uce(\g_{[1]})$ acting as $\id_{\g} \tensor \left( -v \frac{d}{dv} \right)$ on $\g_{[1]}$ and as zero on $\z_{[1]} := \z(\tilde{\g}_{[1]})$; recall also that this centre is $1$-dimensional, and in particular, it is identifiable as:
                $$\z_{[1]} \cong \bbC K_{0, -1}$$
            (see remark \ref{remark: centres_of_dual_toroidal_lie_algebras} for how the elements $K_{r, s}$ are given).

            Fix a Cartan subalgebra $\hat{\h}_{[1]}$ of $\hat{\g}_{[1]}$. The affine Dynkin diagram associated to $\hat{\g}_{[1]}$ (cf. \cite[Chapter 4]{kac_infinite_dimensional_lie_algebras}) shall be denoted by:
                $$\hat{\Gamma} := ( \hat{\Gamma}_0, \hat{\Gamma}_1 )$$
            with $\hat{\Gamma}_0$ denoting the set of vertices and $\hat{\Gamma}_1$ denoting the set of undirected edges.

            Finally, denote the associated Cartan matrix by:
                $$\hat{C} := (\hat{c}_{ij})_{1 \leq i, j \leq |\hat{\Gamma}_0|}$$
            Since affine Cartan matrices are symmetrisable, we can fix a symmetrisation:
                $$\hat{C} := \hat{D} \hat{A}$$
            wherein $\hat{D}$ is an invertible diagonal $|\Gamma_0| \x |\Gamma_0|$ matrix and $A$ is symmetric. 

            Also, let us denote the highest root by $\theta$. Note that we have a bijection (but not literally an equality; cf. \cite[Chapters 4, 5, 7]{kac_infinite_dimensional_lie_algebras}):
                $$\hat{\Gamma}_0 \cong \Gamma_0 \cup \{\theta\}$$
        \end{convention}
        \begin{lemma}[Chevalley-Serre presentation for $\tilde{\g}_{[2]}$] \label{lemma: chevalley_serre_presentation_for_central_extensions_of_multiloop_algebras}
            (Cf. \cite[Proposition 6.6]{wendlandt_formal_shift_operators_on_yangian_doubles}) The Lie algebra $\tilde{\g}_{[2]}$ is isomorphic to the Lie algebra $\t$ generated by the set:
                $$\{ E_{i, r}^{\pm}, H_{i, r} \}_{(i, r) \in \hat{\Gamma}_0 \x \Z} \cup \{ K \}$$
            whose elements are subjected to the following relations, given for all $(i, r), (j, s) \in \hat{\Gamma}_0 \x \Z$:
                $$[\t, K] = 0$$
                $$[ H_{i, r}, H_{j, s} ] = 0$$
                $$[ H_{i, r}, E_{j, s}^{\pm} ] = \pm 2 d_{ij} E_{j, r + s}^{\pm}$$
                $$[ E_{i, r}^+, E_{j, s}^- ] = \delta_{ij} H_{i, r + s}$$
                $$[ E_{i, r + 1}^{\pm}, E_{j, s}^{\pm} ] - [ E_{i, r}^{\pm}, E_{j, s + 1}^{\pm} ] = 0$$
            The isomorphism $\t \xrightarrow[]{\cong} \tilde{\g}_{[2]}$ in question is given as follows, for all $(i, r) \in \hat{\Gamma}_0 \x \Z$:
                $$\forall (i, r) \in \Gamma_0 \x \Z: E_{i, r}^{\pm} \mapsto e_i^{\pm} t^r, H_{i, r} \mapsto h_i t^r$$
                $$\forall (i, r) \in \{\theta\} \x \Z: E_{\theta, r}^{\pm} \mapsto e_{\theta}^{\mp} v^{\pm 1} t^r, H_{\theta, r} \mapsto h_{\theta} t^r + c_v t^r$$
                $$K \mapsto c_t$$
        \end{lemma}
        \begin{corollary} \label{coro: chevalley_serre_presentation_for_central_extensions_of_multiloop_algebras}
            The Lie algebras $\tilde{\g}_{[2]}^{\pm}$ is isomorphic to the Lie algebras $\t^{\pm}$ generated, respectively, by the sets:
                $$\{ E_{i, r}^{\pm}, H_{i, r} \}_{(i, r) \in \Gamma_0 \x \Z_{\geq 0}}$$
                $$\{ E_{i, r}^{\pm}, H_{i, r} \}_{(i, r) \in \Gamma_0 \x \Z_{< 0}} \cup \{K\}$$
            whose elements are subjected to the same relations as in lemma \ref{lemma: chevalley_serre_presentation_for_central_extensions_of_multiloop_algebras}. The isomorphisms $\t^{\pm} \xrightarrow[]{\cong} \tilde{\g}_{[2]}^{\pm}$ in question are just codomain restrictions of the isomorphism $\t \xrightarrow[]{\cong} \tilde{\g}_{[2]}$ from lemma \ref{lemma: chevalley_serre_presentation_for_central_extensions_of_multiloop_algebras}.
        \end{corollary}

        Let us now demonstrate how a low-degree presentation for the Lie algebras $\tilde{\g}_{[2]}$ and $\tilde{\g}_{[2]}^{\pm}$ may be obtained, as eluded to earlier. This necessitates introducing the (formal) Yangian associated to the untwisted affine Kac-Moody algebra $\hat{\g}_{[1]}$ because as explained above, we will be relying on the existence of such a low-degree presentation for those (formal) Yangians.
        \begin{convention}[Yangians associated to symmetrisable Kac-Moody algebras]
            Suppose for a moment that $\g$ is a general symmetrisable Kac-Moody algebra whose associated Cartan matrix is indecomposable. We refer the reader to \cite[Section 2]{guay_nakajima_wendlandt_affine_yangian_coproduct} for the definition of the \textbf{formal Yangian} $\rmY_{\hbar}(\g)$ and \textbf{Yangian} $\rmY(\g)$, as well as all relevant discussions about the various \say{basic} presentations of these associative algebras (living over $\bbC[\hbar]$ and $\bbC$ respectively). The only thing that we will note is that we will be denoting the Chevalley-Serre generators by:
                $$E_{i, r}^{\pm}, H_{i, r}$$
        \end{convention}
        \begin{convention}
            From now on, let us write:
                $$T_{i, 1}(\hbar) := H_{i, 1} - \frac12 \hbar H_{i, 0}^2$$
                $$T_{i, 1} := T_{i, 1}(1) = H_{i, 1} - \frac12 H_{i, 0}^2$$
        \end{convention}

        One key property of formal (affine) Yangians that we will be relying on is the fact that 
        \begin{lemma}[Formal Yangians as Rees algebras] \label{lemma: formal_yangians_as_rees_algebras}
            \cite[Theorem 6.10]{guay_nakajima_wendlandt_affine_yangian_vertex_representations_and_PBW} Suppose for a moment that $\g$ is a general indecomposable symmetrisable Kac-Moody algebra. If $\g$ is either of finite type but not $\sfA_1$ or of untwisted affine type but not $\sfA_1^{(1)}$ and not $\sfA_1^{(2)}$ then the natural \textit{graded} $\bbC$-algebra homomoprhism:
                $$\rmY_{\hbar}(\g) \to \Rees_{\hbar} \rmY(\g)$$
            will be an isomorphism. 
        \end{lemma}
         \begin{corollary}[Formal affine Yangians as flat graded deformations] \label{coro: formal_affine_yangians_as_flat_graded_deformations}
            Suppose that $\g \not \cong \sl_2(\bbC)$. Then the $\bbC[\hbar]$-algebra $\rmY_{\hbar}(\hat{\g}_{[1]})$ will be a flat $\Z$-graded deformation of the $\Z$-graded $\bbC$-algebra $\rmU(\tilde{\g}_{[2]}^+)$. 
         \end{corollary}
         
        The hypotheses of the following lemma are satisfied at least when $\g$ is either of finite type or of affine type, save for the types $\sfA_1^{(1)}$ and $\sfA_1^{(2)}$.
        \begin{lemma}[A Levendorskii-type presentation for Yangians of Kac-Moody algebras] \label{lemma: levendorskii_presentation}
            \cite[Theorem 2.13]{guay_nakajima_wendlandt_affine_yangian_coproduct} Suppose for a moment that $\g$ is a general symmetrisable Kac-Moody algebra whose Cartan matrix is:
            \begin{itemize}
                \item indecomposable,
                \item such that, for any $i < j \in \Gamma_0$ (with respect to some choice of total ordering on $\Gamma_0$) the following $2 \x 2$ matrix is invertible:
                    $$
                        \begin{pmatrix}
                            a_{ii} & a_{ij}
                            \\
                            a_{ji} & a_{ji}
                        \end{pmatrix}
                    $$
            \end{itemize}
            The formal Yangian $\rmY_{\hbar}(\g)$ of $\g$ will then be isomorphic to the associative $\bbC$-algebra generated by the set:
                $$\{ H_{i, r}, E_{i, r}^{\pm} \}_{(i, r) \in \Gamma_0 \x \N}$$
            whose elements are subjected to the following relations\footnote{... and it is understood that the elements $H_{i, 0} = h_i, E_{i, 0}^{\pm} = e_i^{\pm}$ satisfy the Chevalley-Serre relations defining $\g$; cf. \cite[Chapter 1]{kac_infinite_dimensional_lie_algebras}.}:
                $$H_{i, 0} = h_i, E_{i, 0}^{\pm} = e_i^{\pm}$$
                $$[ H_{i, r}, H_{j, s} ] = 0$$
                $$[ H_{i, 0}, E_{j, s}^{\pm} ] = \pm c_{ij} E_{j, s}^{\pm}$$
                $$[ E_{i, r}^+, E_{j, s}^- ] = \pm \delta_{ij} H_{i, r + s}$$
                $$\left[ T_{i, 1}(\hbar), E_{j, 0}^{\pm} \right] = \pm \hbar c_{ij} E_{j, 1}^{\pm}$$
                $$[ E_{i, 1}^{\pm}, E_{j, 0}^{\pm} ] - [ E_{i, 0}^{\pm}, E_{j, 1}^{\pm} ] = \pm \frac12 \hbar c_{ij} \{E_{i, 0}^{\pm}, E_{j, 0}^{\pm}\}$$
        \end{lemma}
        \begin{proposition}[Levendorskii presentation for $\tilde{\g}_{[2]}$] \label{prop: levendorskii_presentation__for_central_extensions_of_multiloop_algebras}
            The Lie algebra $\tilde{\g}_{[2]}$ is isomorphic to the Lie algebra $\t$ generated by the set:
                $$\{ E_{i, r}^{\pm}, H_{i, r} \}_{(i, r) \in \hat{\Gamma}_0 \x \Z} \cup \{ K \}$$
            whose elements are subjected to the following relations, given for all $(i, r), (j, s) \in \hat{\Gamma}_0 \x \Z$:
                $$[\t, K] = 0$$
                $$H_{i, 0} = h_i, E_{i, 0}^{\pm} = e_i^{\pm}$$
                $$[ H_{i, r}, H_{j, s} ] = 0$$
                $$[ H_{i, 0}, E_{j, s}^{\pm} ] = \pm 2 d_{ij} E_{j, s}^{\pm}$$
                $$[ E_{i, r}^+, E_{j, s}^- ] = \delta_{ij} H_{i, r + s}$$
                $$[ E_{i, 1}^{\pm}, E_{j, 0}^{\pm} ] - [ E_{i, 0}^{\pm}, E_{j, 1}^{\pm} ] = 0$$
            The isomorphism $\t \xrightarrow[]{\cong} \tilde{\g}_{[2]}$ in question is as in lemma \ref{lemma: chevalley_serre_presentation_for_central_extensions_of_multiloop_algebras}.
        \end{proposition}
            \begin{proof}
                Combine corollary \ref{coro: formal_affine_yangians_as_flat_graded_deformations} with lemma \ref{lemma: levendorskii_presentation}. Note that flatness (in particular, $\hbar$-torsion-freeness) is crucial for our application of corollary \ref{coro: formal_affine_yangians_as_flat_graded_deformations}.
            \end{proof}
        \begin{corollary} \label{coro: levendorskii_presentation__for_central_extensions_of_multiloop_algebras}
            The Lie algebras $\tilde{\g}_{[2]}^{\pm}$ are, respectively, isomorphic to the Lie algebra $\t^{\pm}$ generated by the set:
                $$\{ E_{i, r}^{\pm}, H_{i, r} \}_{(i, r) \in \Gamma_0 \x \Z_{\geq 0}}$$
                $$\{ E_{i, r}^{\pm}, H_{i, r} \}_{(i, r) \in \Gamma_0 \x \Z_{< 0}} \cup \{K\}$$
            whose elements are subjected to the relations as in proposition \ref{prop: levendorskii_presentation__for_central_extensions_of_multiloop_algebras}. The isomorphism $\t^{\pm} \xrightarrow[]{\cong} \tilde{\g}_{[2]}^{\pm}$ in question is as in lemma \ref{lemma: chevalley_serre_presentation_for_central_extensions_of_multiloop_algebras}.
        \end{corollary}

    \subsection{Toroidal Lie bialgebras}
        \begin{convention} \label{conv: positive_and_negative_parts_of_derivation_subalgebras_of_extended_toroidal_algebras}
            Let us now adopt the following notations:
            \begin{itemize}
                \item
                    $$\d_{[2]}^+ := ( \bigoplus_{(r, s) \in \Z \x \Z_{\leq 0} } \bbC D_{r, s} ) \oplus \bbC D_t$$
                    $$\d_{[2]}^- := ( \bigoplus_{(r, s) \in \Z \x \Z_{> 0} } \bbC D_{r, s} ) \oplus \bbC D_v$$
                shall respectively be the Lie subalgebras of $\d_{[2]}$ which are graded-dual to $\z_{[2]}^{\pm}$ with respect to $(-, -)_{\hat{\g}_{[2]}}$;
                \item $\hat{\g}_{[2]}^{\pm} := \tilde{\g}_{[2]}^{\pm} \rtimes \d_{[2]}^{\pm}$.
            \end{itemize}
        \end{convention}    
        \begin{theorem} \label{theorem: extended_toroidal_manin_triples}
            There is a complete topological Manin triple:
                $$(\hat{\g}_{[2]}, \hat{\g}_{[2]}^+, \hat{\g}_{[2]}^-)$$
            wherein $\hat{\g}_{[2]}$ is equipped with the non-degenerate invariant inner product $(-, -)_{\hat{\g}_{[2]}}$ (cf. convention \ref{conv: orthogonal_complement_of_toroidal_centres}).
        \end{theorem}
            \begin{proof}
                Firstly, let us check that $\hat{\g}_{[2]}^{\pm}$ are Lie subalgebras of $\hat{\g}_{[2]}$ and note that for this, it suffices to check that $\d_{[2]}^{\pm}$ are Lie subalgebras of $\d_{[2]}$, since we have that:
                    $$\hat{\g}_{[2]} \cong \tilde{\g}_{[2]} \rtimes \d_{[2]}$$
                (cf. theorem \ref{theorem: extended_toroidal_lie_algebras}) and that:
                    $$\hat{\g}_{[2]}^{\pm} := \tilde{\g}_{[2]}^{\pm} \rtimes \d_{[2]}^{\pm}$$
                by construction. 
            \end{proof}
        \begin{corollary}[Lie cobracket on $\hat{\g}_{[2]}^+$] \label{coro: extended_toroidal_lie_bialgebras}
            On the extended toroidal Lie algebra $\hat{\g}_{[2]}^+$, there is a continuous Lie cobracket\footnote{Note the completion!}, making $\hat{\g}_{[2]}^+$ a complete topological Lie bialgebra:
                $$\delta_{\hat{\g}_{[2]}^+}: \hat{\g}_{[2]}^+ \to \hat{\g}_{[2]}^+ \hattensor_{\bbC} \hat{\g}_{[2]}^+$$
            given for any $X \in \hat{\g}_{[2]}^+$ by the following formula (cf. \cite{etingof_kazhdan_quantisation_1}):
                $$\delta_{\hat{\g}_{[2]}^+}(X) = [ X \tensor 1 + 1 \tensor X, \sfr_{\hat{\g}_{[2]}^+} ]$$
            wherein:
                $$\sfr_{\hat{\g}_{[2]}^+} := \sfr_{\g_{[2]}^+} + \sfr_{\z_{[2]}^+} + \sfr_{\d_{[2]}^+} \in \hat{\g}_{[2]}^+ \hattensor_{\bbC} \hat{\g}_{[2]}^-$$
            with $\sfr_{\g_{[2]}^+} \in \g_{[2]}^+ \hattensor_{\bbC} \g_{[2]}^-$ being as in question \ref{question: multiloop_lie_bialgebras} and\footnote{Note how we are simply summing over tensor products of dual basis elements.} $\sfr_{\z_{[2]}^+} \in \z_{[2]}^+ \hattensor_{\bbC} \d_{[2]}^+$ and $\sfr_{\d_{[2]}^+} \in \d_{[2]}^+ \hattensor_{\bbC} \z_{[2]}^+$ being given by the following formulae:
                $$\sfr_{\z_{[2]}^+} := \sum_{(r, s) \in \Z \x \Z_{> 0}} K_{r, s} \tensor D_{r, s} + c_v \tensor D_v$$
                
                $$\sfr_{\d_{[2]}^+} := \sum_{(r, s) \in \Z \x \Z_{\leq 0}} D_{r, s} \tensor K_{r, s} + D_t \tensor c_t$$
        \end{corollary}
    
        \begin{convention}[Dirac distributions]
            We will be using the following shorthands:
                $$\1(z, w) = \sum_{m \in \Z} z^m w^{-m - 1}$$
                $$\1^+(z, w) = \sum_{m \in \Z_{\geq 0}} z^m w^{-m - 1}$$
            as well as:
                $$\{ X_1, ..., X_n \} := \sum_{\sigma \in S_n} X_{\sigma(1)} \cdot ... \cdot X_{\sigma(n)}$$
                $$\bar{\Delta}(X) := X \tensor 1 + 1 \tensor X$$
        \end{convention}

        \begin{remark}[Total degrees of \say{Yangian} canonical elements] \label{remark: total_degrees_of_classical_yangian_R_matrices}
            One property of the R-matrix $\sfr_{\hat{\g}_{[2]}^+}$ from corollary \ref{coro: extended_toroidal_lie_bialgebras} that will help simplify some computations later on (see the proof of theorem \ref{theorem: toroidal_lie_bialgebras}) is that they are of total degree $-1$. 

            Recall that if $V := \bigoplus_{m \in \Z} V_m, W := \bigoplus_{n \in \Z} W_n$ are $\Z$-graded vector spaces then for any $k \in \Z$, we have that:
                $$(V \tensor_{\bbC} W)_k \cong \bigoplus_{m + n = k} V_m \tensor_{\bbC} W_n$$
                
            If we now take $V = W = \rmU(\tilde{\g}_{[2]})$ then the claim from above would read:
                $$\sfr_{\tilde{\g}_{[2]}^+} \in ( \rmU(\tilde{\g}_{[2]}^+) \tensor_{\bbC} \rmU(\tilde{\g}_{[2]}^-) )_{-1}$$
            with the $\Z$-grading on $\tilde{\g}_{[2]}^{\pm}$ (and hence on $\rmU(\tilde{\g}_{[2]}^{\pm})$) being the one on the second variable $t$ (cf. remark \ref{remark: Z_gradings_on_toroidal_lie_algebras}), and actually, this is entirely due to:
                $$\sfr_{\g_{[2]}^+} \in ( \rmU(\g_{[2]}^+) \tensor_{\bbC} \rmU(\g_{[2]}^-) )_{-1}$$
            and this can already be inferred from the computations done in question \ref{question: multiloop_lie_bialgebras}, where we showed that:
                $$\sfr_{\g_{[2]}^+} = \sfr_{\g} v_2 \1(v_1, v_2) \1^+(t_1, t_2)$$
            (the crucial detail to notice here is that $\deg \1^+(t_1, t_2) = -1$ since $\1^+(t_1, t_2) := \sum_{p \in \Z_{\geq 0}} t_1^p t_2^{-p - 1}$).

            What this means for us is that, should we have $X \in \tilde{\g}_{[2]}^+$ such that:
                $$\deg X \leq 0$$
            then it will automatically be the case that:
                $$\delta_{\tilde{\g}_{[2]}^+}(X) = 0$$
        \end{remark}
        
        We are now finally able to put a Lie cobracket on the toroidal Lie algebra $\tilde{\g}_{[2]}^+$, compatible with the Lie bracket thereon in a manner that produces a Lie bialgebra structure. This Lie bialgebra structure is the classical limit of the coproduct on the formal Yangian $\rmY_{\hbar}(\hat{\g}_{[1]})$. 
        \begin{theorem}[Toroidal Lie bialgebras] \label{theorem: toroidal_lie_bialgebras}
            Assume convention \ref{conv: a_fixed_untwisted_affine_kac_moody_algebra} and let us abbreviate:
                $$\hat{\delta}^+ := \delta_{\hat{\g}_{[2]}^+}$$
            with $\delta_{\hat{\g}_{[2]}^+}$ as in corollary \ref{coro: extended_toroidal_lie_bialgebras}. Let:
                $$\tilde{\delta}^+ := \hat{\delta}^+|_{\tilde{\g}_{[2]}}$$
            Then $(\tilde{\g}_{[2]}^+, \tilde{\delta}^+)$ will be a complete topological Lie sub-bialgebra of $(\hat{\g}_{[2]}^+, \hat{\delta}^+)$ as given in corollary \ref{coro: extended_toroidal_lie_bialgebras}. Thanks to corollary \ref{coro: levendorskii_presentation__for_central_extensions_of_multiloop_algebras}, we know that it is enough to specify how $\tilde{\delta}^+$ is given on the set of generators:
                $$\{E_{i, 0}^{\pm}\}_{i \in \hat{\Gamma}_0} \cup \{H_{i, r}\}_{ (i, r) \in \hat{\Gamma}_0 \x \{0, 1\} }$$
            and since we know that under the isomorphism $\t^+ \xrightarrow[]{\cong} \tilde{\g}_{[2]}^+$ in \textit{loc. cit.}, we have the following assignments:
                $$\forall i \in \hat{\Gamma}_0: E_{i, 0}^{\pm} \mapsto e_i^{\pm}, H_{i, 0} \mapsto h_i$$
                $$\forall i \in \Gamma_0: H_{i, 1} \mapsto h_i t$$
                $$H_{\theta, 1} \mapsto h_{\theta} t + t c_v$$
            it is enough to specify the following, wherein $h \in \h$ is arbitrary:
                $$\tilde{\delta}^+(h) = 0$$
                $$\tilde{\delta}^+(ht) = [h_1 \tensor 1, \sfr_{\g} v_2 \1(v_1, v_2)]$$
                $$\tilde{\delta}^+(t c_v) = 0$$
        \end{theorem}
            \begin{proof}
                \begin{enumerate}
                    \item Since $\deg x = 0$ for all $x \in \g$, we get via remark \ref{remark: total_degrees_of_classical_yangian_R_matrices} that:
                        $$\hat{\delta}^+(x) = 0$$
                    and in particular, we have that:
                        $$\hat{\delta}^+(h) = 0$$

                    \item Let us now compute $\hat{\delta}^+(ht)$ for an arbitrary $h \in \h$. 
                    \begin{enumerate}
                        \item \textbf{($\g_{[2]}^+$-component):} Firstly, to compute:
                            $$[\bar{\Delta}(ht), \sfr_{\g_{[2]}^+}]$$
                        let us firstly recall from question \ref{question: multiloop_lie_bialgebras} that:
                            $$\sfr_{\g_{[2]}^+} = \sfr_{\g} v_2\1(v_1, v_2) \1^+(t_1, t_2)$$
                        (with notations as in \textit{loc. cit.}); let us also choose a root basis for $\g$ for writing out $\sfr_{\g}$ explicitly: this is to say that for each positive root $\alpha \in \Phi^+$, we choose corresponding basis vectors $e_{\alpha}^{\pm} \in \g_{\pm \alpha}$ normalised so that:
                            $$(e_{\alpha}^-, e_{\alpha}^+)_{\g} = 1$$
                        to get the following basis for $\g$:
                            $$\{h_i\}_{i \in \Gamma_0} \cup \{e_{\alpha}^-, e_{\alpha}^+\}_{\alpha \in \Phi^+}$$
                        From this, we see that:
                            $$
                                \begin{aligned}
                                    & [\bar{\Delta}(ht), \sfr_{\g_{[2]}^+}]
                                    \\
                                    = & -\sum_{i \in \Gamma_0} [\bar{\Delta}(ht), h_i \tensor h_i v_2\1(v_1, v_2) \1^+(t_1, t_2)] - \sum_{\alpha \in \Phi^+} [\bar{\Delta}(ht), (e_{\alpha}^- \tensor e_{\alpha}^+ + e_{\alpha}^+ \tensor e_{\alpha}^-) v_2\1(v_1, v_2) \1^+(t_1, t_2)]
                                \end{aligned}
                            $$

                        Now, for each $i \in \Gamma_0$, observe that:
                            $$
                                \begin{aligned}
                                    & [h t_1 \tensor 1, h_i \tensor h_i v_2\1(v_1, v_2) \1^+(t_1, t_2)]
                                    \\
                                    = & \sum_{(m, p) \in \Z \x \Z_{\geq 0}} [ht_1 \tensor 1, h_i v_1^m t_1^p \tensor h_i v_2^{-m} t_2^{-p - 1}]
                                    \\
                                    = & \sum_{(m, p) \in \Z \x \Z_{\geq 0}} [ht_1, h_i v_1^m t_1^p]_{\tilde{\g}_{[2]}^+} \tensor h_i v_2^{-m} t_2^{-p - 1}
                                    \\
                                    = & \sum_{(m, p) \in \Z \x \Z_{\geq 0}} (h, h_i)_{\g} v_1^m t_1^p \bar{d}(t_1) \tensor h_i v_2^{-m} t_2^{-p - 1}
                                \end{aligned}
                            $$
                        and likewise, that:
                            $$[1 \tensor h t_2, h_i \tensor h_i v_2\1(v_1, v_2) \1^+(t_1, t_2)] = \sum_{(m, p) \in \Z \x \Z_{\geq 0}} h_i v_1^m t_1^p \tensor (h, h_i)_{\g} v_2^{-m} t_2^{-p - 1} \bar{d}(t_2)$$
                        Adding the two summands together then yields:
                            $$
                                \begin{aligned}
                                    & [\bar{\Delta}(ht), h_i \tensor h_i v_2\1(v_1, v_2) \1^+(t_1, t_2)]
                                    \\
                                    = & (h, h_i)_{\g} \sum_{(m, p) \in \Z \x \Z_{\geq 0}} \left( v_1^m t_1^p \bar{d}(t_1) \tensor h_i v_2^{-m} t_2^{-p - 1} + h_i v_1^m t_1^p \tensor v_2^{-m} t_2^{-p - 1} \bar{d}(t_2) \right)
                                    \\
                                    = & (h, h_i)_{\g} ( \bar{d}(t_1) \tensor h_i + h_i \tensor \bar{d}(t_2) ) v_2\1(v_1, v_2) \1^+(t_1, t_2)
                                \end{aligned}
                            $$
                        
                        Next, consider the following:
                            $$
                                \begin{aligned}
                                    & [ht_1 \tensor 1, e_{\alpha}^- \tensor e_{\alpha}^+ v_2\1(v_1, v_2) \1^+(t_1, t_2)]
                                    \\
                                    = & \sum_{(m, p) \in \Z \x \Z_{\geq 0}} [ht_1 \tensor 1, e_{\alpha}^- v_1^m t_1^p \tensor e_{\alpha}^+ v_2^{-m} t_2^{-p - 1}]
                                    \\
                                    = & \sum_{(m, p) \in \Z \x \Z_{\geq 0}} [ht_1, e_{\alpha}^- v_1^m t_1^p]_{\tilde{\g}_{[2]}^+} \tensor e_{\alpha}^+ v_2^{-m} t_2^{-p - 1}
                                    \\
                                    = & \sum_{(m, p) \in \Z \x \Z_{\geq 0}} \left( -\alpha(h) e_{\alpha}^- v_1^m t_1^{p + 1} + (h, e_{\alpha}^-)_{\g} t_1 \bar{d}(v_1^m t_1^p) \right) \tensor e_{\alpha}^+ v_2^{-m} t_2^{-p - 1}
                                    \\
                                    = & \sum_{(m, p) \in \Z \x \Z_{\geq 0}} -\alpha(h) e_{\alpha}^- v_1^m t_1^{p + 1} \tensor e_{\alpha}^+ v_2^{-m} t_2^{-p - 1}
                                    \\
                                    & = -\alpha(h) ( e_{\alpha}^- \tensor e_{\alpha}^+ ) v_2 \1(v_1, v_2) t_1 \1^+(t_1, t_2)
                                \end{aligned}    
                            $$
                        wherein the second-to-last identity comes from the fact that\footnote{This can be proven easily by passing to the vector representation of $\g$, wherein $h$ is represented by a diagonal matrix while $e^{\pm}$ is represented by an upper/lower triangular matrix, and then using the fact that $(-, -)_{\g}$ differs from the trace form only by a non-zero constant.}:
                            $$(h, e^{\pm})_{\g} = 0$$
                        for every $h \in \h$ and every $e^{\pm} \in \n^{\pm}$. Similarly, we find that:
                            $$[ht_1 \tensor 1, e_{\alpha}^+ \tensor e_{\alpha}^- v_2\1(v_1, v_2) \1^+(t_1, t_2)] = \alpha(h) ( e_{\alpha}^+ \tensor e_{\alpha}^- ) v_2 \1(v_1, v_2) t_1 \1^+(t_1, t_2)$$
                        By putting the two together, one obtains:
                            $$[h t_1 \tensor 1, (e_{\alpha}^- \tensor e_{\alpha}^+ + e_{\alpha}^+ \tensor e_{\alpha}^-) v_2\1(v_1, v_2) \1^+(t_1, t_2)] = -\alpha(h) ( e_{\alpha}^- \tensor e_{\alpha}^+ - e_{\alpha}^+ \tensor e_{\alpha}^- ) v_2 \1(v_1, v_2) t_1 \1^+(t_1, t_2)$$
                        Likewise, we find that:
                            $$[1 \tensor h t_2, (e_{\alpha}^- \tensor e_{\alpha}^+ + e_{\alpha}^+ \tensor e_{\alpha}^-) v_2\1(v_1, v_2) \1^+(t_1, t_2)] = \alpha(h) ( e_{\alpha}^- \tensor e_{\alpha}^+ - e_{\alpha}^+ \tensor e_{\alpha}^- ) v_2 \1(v_1, v_2) t_2 \1^+(t_1, t_2)$$
                        and hence:
                            $$
                                \begin{aligned}
                                    & [\bar{\Delta}(ht), \sfr_{\g_{[2]}^+}]
                                    \\
                                    = & -\left( \sum_{i \in \Gamma_0} (h, h_i)_{\g} ( \bar{d}(t_1) \tensor h_i + h_i \tensor \bar{d}(t_2) ) + \sum_{\alpha \in \Phi^+} \alpha(h) ( e_{\alpha}^- \tensor e_{\alpha}^+ - e_{\alpha}^+ \tensor e_{\alpha}^- )(t_2 - t_1) \right) v_2 \1(v_1, v_2) \1^+(t_1, t_2)
                                    \\
                                    & = -\left( \bar{d}(t_1) \tensor h + h \tensor \bar{d}(t_2) + [h_1 \tensor 1, \sfr_{\g}] (t_2 - t_1) \right) v_2 \1(v_1, v_2) \1^+(t_1, t_2)
                                    \\
                                    & = -\left( \bar{d}(t_1) \tensor h + h \tensor \bar{d}(t_2) \right) v_2 \1(v_1, v_2) \1^+(t_1, t_2) + [h_1 \tensor 1, \sfr_{\g}] v_2 \1(v_1, v_2)
                                \end{aligned}
                            $$
                        We note that the last equality holds thanks to the fact that:
                            $$(t_2 - t_1) \1^+(t_1, t_2) = (t_2 - t_1) \sum_{p \in \Z_{\geq 0}} t_1^p t_2^{-p - 1} = (t_2 - t_1) \frac{1}{t_2 - t_1} = 1$$
                            
                        \item \textbf{($\z_{[2]}^+$-component):} Recall from corollary \ref{coro: extended_toroidal_lie_bialgebras} that:
                            $$\sfr_{\z_{[2]}^+} := \sum_{(r, s) \in \Z \x \Z_{> 0}} K_{r, s} \tensor D_{r, s} + c_{v_1} \tensor D_{v_2}$$
                        and so:
                            $$
                                \begin{aligned}
                                    & [\bar{\Delta}(ht), \sfr_{\z_{[2]}^+}]
                                    \\
                                    = & \sum_{(r, s) \in \Z \x \Z_{> 0}} [\bar{\Delta}(ht), K_{r, s} \tensor D_{r, s}] + [\bar{\Delta}(ht), c_{v_1} \tensor D_{v_2}]
                                    \\
                                    = & -\sum_{(r, s) \in \Z \x \Z_{> 0}} K_{r, s} \tensor h D_{r, s}(t) - c_{v_1} \tensor h D_{v_2}(t_2)
                                    \\
                                    = & -\sum_{(r, s) \in \Z \x \Z_{> 0}} K_{r, s} \tensor r h v_2^{-r} t_2^{-s}
                                \end{aligned}
                            $$
                        where the minus sign in the third equation appeared because:
                            $$[ht, D_{r, s}] = -[D_{r, s}, ht] = -h D_{r, s}(t)$$
                            $$[ht, D_v] = -[D_v, ht] = -h D_v(t)$$
                        (cf. remark \ref{remark: derivation_action_on_multiloop_algebras}) and the last equality is due to the fact that:
                            $$D_{r, s} = -s v^{-r + 1} t^{-s - 1} \del_v + r v^{-r} t^{-s} \del_t$$
                            $$D_v = -v t^{-1} \del_v$$
                        (cf. remark \ref{remark: dual_of_toroidal_centres_contains_derivations}). 
                        
                        \item \textbf{($\d_{[2]}^+$-component):} Recall from corollary \ref{coro: extended_toroidal_lie_bialgebras} that:
                            $$\sfr_{\z_{[2]}^+} := \sum_{(r, s) \in \Z \x \Z_{\leq 0}} D_{r, s} \tensor K_{r, s} + D_{t_1} \tensor c_{t_2}$$
                        and so:
                            $$
                                \begin{aligned}
                                    & [\bar{\Delta}(ht), \sfr_{\z_{[2]}^+}]
                                    \\
                                    = & \sum_{(r, s) \in \Z \x \Z_{\leq 0}} [\bar{\Delta}(ht), D_{r, s} \tensor K_{r, s}] + [\bar{\Delta}(ht), D_{t_1} \tensor c_{t_2}]
                                    \\
                                    = & -\sum_{(r, s) \in \Z \x \Z_{\leq 0}} h D_{r, s}(t_1) \tensor K_{r, s} - h D_{t_1}(t_1) \tensor c_{t_2}
                                    \\
                                    = & -\sum_{(r, s) \in \Z \x \Z_{\leq 0}} r h v_1^{-r} t_1^{-s} \tensor K_{r, s} + h \tensor c_{t_2}
                                \end{aligned}
                            $$
                        where the minus sign in the third equation appeared because:
                            $$[ht, D_{r, s}] = -[D_{r, s}, ht] = -h D_{r, s}(t)$$
                            $$[ht, D_t] = -[D_t, ht] = -h D_t(t)$$
                        (cf. remark \ref{remark: derivation_action_on_multiloop_algebras}) and the the last equality is due to the fact that:
                            $$D_{r, s} = -s v^{-r + 1} t^{-s - 1} \del_v + r v^{-r} t^{-s} \del_t$$
                            $$D_t = -\del_t$$
                        (cf. remark \ref{remark: dual_of_toroidal_centres_contains_derivations}). 
                    \end{enumerate}

                    Since we know that:
                        $$[\bar{\Delta}(ht), \sfr_{\g_{[2]}^+}] = -\left( \bar{d}(t_1) \tensor h + h \tensor \bar{d}(t_2) \right) v_2 \1(v_1, v_2) \1^+(t_1, t_2) + [h_1 \tensor 1, \sfr_{\g}] v_2 \1(v_1, v_2)$$
                    we now claim that:
                        $$[\bar{\Delta}(ht), \sfr_{\z_{[2]}^+} + \sfr_{\d_{[2]}^+}] = \left( \bar{d}(t_1) \tensor h + h \tensor \bar{d}(t_2) \right) v_2 \1(v_1, v_2) \1^+(t_1, t_2)$$
                    (since ultimately, we would like to show that $\hat{\delta}^+(ht) = \sfr_{\g} v_2 \1(v_1, v_2)$), and to prove that this is the case, let us first note that we now have that:
                        $$
                            \begin{aligned}
                                & [\bar{\Delta}(ht), \sfr_{\z_{[2]}^+} + \sfr_{\d_{[2]}^+}]
                                \\
                                = & -\sum_{(r, s) \in \Z \x \Z_{> 0}} \left( K_{r, s} \tensor r h v_2^{-r} t_2^{-s} + r h v_1^{-r} t_1^s \tensor K_{r, -s} \right) - \sum_{r \in \Z} r h v_1^{-r} \tensor K_{r, 0} + h \tensor c_{t_2}
                            \end{aligned}
                        $$
                    wherein the first summand corresponds to the indices $(r, 0) \in \Z \x \Z_{\leq 0}$. From remark \ref{remark: centres_of_dual_toroidal_lie_algebras}, we know that:
                        $$
                            K_{r, s} :=
                            \begin{cases}
                                \text{$\frac1s v^{r - 1} t^s \bar{d}(v)$ if $(r, s) \in \Z \x \Z$}
                                \\
                                \text{$-\frac1r v^r t^{-1} \bar{d}(t)$ if $(r, s) \in \Z \x \{0\}$}
                                \\
                                \text{$0$ if $(r, s) = (0, 0)$}
                            \end{cases}
                        $$
                    from which one infers that:
                        $$
                            \begin{aligned}
                                & -\sum_{r \in \Z} r h v_1^{-r} \tensor K_{r, 0}
                                \\
                                = & -\sum_{r \in \Z} r h v_1^{-r} \tensor \left( -\frac1r v_2^r t_2^{-1} \bar{d}(t_2) \right)
                                \\
                                = & \sum_{r \in \Z} h v_1^{-r} \tensor v_2^r t_2^{-1} \bar{d}(t_2)
                                \\
                                = & \sum_{r \in \Z} h v_1^{-r} \tensor v_2^r t_2^{-1} \bar{d}(t_2)
                            \end{aligned}
                        $$
                        
                    Next, recall again from remark \ref{remark: centres_of_dual_toroidal_lie_algebras} that:
                        $$(r, s) \in \Z^2 \implies K_{r, s} = \frac1s v^{r - 1} t^s \bar{d}(v) = -\frac1r v^r t^{s - 1} \bar{d}(t) \in \bar{\Omega}_{[2]}$$
                    and then consider the following:
                        $$
                            \begin{aligned}
                                & -\sum_{(r, s) \in \Z \x \Z_{> 0}} \left( K_{r, s} \tensor r h v_2^{-r} t_2^{-s} + r h v_1^{-r} t_1^s \tensor K_{r, -s} \right)
                                \\
                                = & \sum_{(r, s) \in \Z \x \Z_{> 0}} \left( v_1^r t_1^{s - 1} \bar{d}(t_1) \tensor h v_2^{-r} t_2^{-s} - h v_1^{-r} t_1^s \tensor v_2^r t_2^{-s - 1} \bar{d}(t_2) \right)
                            \end{aligned}
                        $$
                    wherein we note that for all $s \in \Z_{> 0}$, the summands corresponding to the indices $(0, s)$ vanish.

                    We now have that:
                        $$
                            \begin{aligned}
                                & [\bar{\Delta}(ht), \sfr_{\z_{[2]}^+} + \sfr_{\d_{[2]}^+}]
                                \\
                                = & \sum_{(r, s) \in \Z \x \Z_{> 0}} \left( K_{r, s} \tensor r h v_2^{-r} t_2^{-s} + r h v_1^{-r} t_1^s \tensor K_{r, -s} \right) - \sum_{r \in \Z} r h v_1^{-r} \tensor K_{r, 0} + h \tensor c_{t_2}
                                \\
                                = & \sum_{(r, s) \in \Z \x \Z_{> 0}} \left( v_1^r t_1^{s - 1} \bar{d}(t_1) \tensor h v_2^{-r} t_2^{-s} - h v_1^{-r} t_1^s \tensor v_2^r t_2^{-s - 1} \bar{d}(t_2) \right) + \sum_{r \in \Z} h v_1^{-r} \tensor v_2^r t_2^{-1} \bar{d}(t_2) + h \tensor t_2^{-1} \bar{d}(t_2)
                                \\
                                = & \sum_{(r, s) \in \Z \x \Z_{> 0}} \left( v_1^r t_1^{s - 1} \bar{d}(t_1) \tensor h v_2^{-r} t_2^{-s} + h v_1^r t_1^s \tensor v_2^{-r} t_2^{-s - 1} \bar{d}(t_2) \right) + \sum_{r \in \Z} h v_1^{-r} \tensor v_2^r t_2^{-1} \bar{d}(t_2)
                                \\
                                = & \sum_{(r, s) \in \Z \x \Z_{> 0}} \left( v_1^r t_1^{s - 1} \bar{d}(t_1) \tensor h v_2^{-r} t_2^{-s} + h v_1^{-r} t_1^s \tensor v_2^r t_2^{-s - 1} \bar{d}(t_2) \right) + \sum_{r \in \Z} h v_1^r \tensor v_2^{-r} t_2^{-1} \bar{d}(t_2)
                                \\
                                = & ( \bar{d}(t_1) \tensor h ) \sum_{(r, s) \in \Z \x \Z_{> 0}} v_1^r t_1^{s - 1} \tensor v_2^{-r} t_2^{-s} + ( h \tensor \bar{d}(t_2) ) \left( \sum_{(r, s) \in \Z \x \Z_{> 0}} v_1^r t_1^s \tensor v_2^{-r} t_2^{-s - 1} + \sum_{r \in \Z} v_1^r \tensor v_2^{-r} t_2^{-1} \right)
                                \\
                                = & ( \bar{d}(t_1) \tensor h ) \sum_{(r, s) \in \Z \x \Z_{\geq 0}} v_1^r t_1^s \tensor v_2^{-r} t_2^{-s - 1} + ( h \tensor \bar{d}(t_2) ) \left( \sum_{(r, s) \in \Z \x \Z_{> 0}} v_1^r t_1^s \tensor v_2^{-r} t_2^{-s - 1} + \sum_{r \in \Z} v_1^r \tensor v_2^{-r} t_2^{-1} \right)
                                \\
                                = & ( \bar{d}(t_1) \tensor h + h \tensor \bar{d}(t_2) ) \sum_{(r, s) \in \Z \x \Z_{\geq 0}} v_1^r t_1^s \tensor v_2^{-r} t_2^{-s - 1}
                                \\
                                = & ( \bar{d}(t_1) \tensor h + h \tensor \bar{d}(t_2) ) v_2 \1(v_1, v_2) \1^+(t_1, t_2)
                            \end{aligned}
                        $$

                    We can now add the three components together to yield:
                        $$[\bar{\Delta}(ht), \sfr_{\hat{\g}_{[2]}^+}] = [ \bar{\Delta}(ht), \sfr_{\g_{[2]}^+} + (\sfr_{\z_{[2]}^+} + \sfr_{\d_{[2]}^+}) ] =  [h_1 \tensor 1] v_2 \1(v_1, v_2)$$
                    precisely as claimed. 
                    
                    \item Finally, in order to compute $\hat{\delta}^+(t c_v)$, let us simply note that because:
                        $$\deg t = 1, \deg c_v = -1$$
                    in $\z_{[2]}$ (cf. remark \ref{remark: Z_gradings_on_toroidal_lie_algebras}), we have that:
                        $$\deg t c_v = 0$$
                    and hence:
                        $$\hat{\delta}^+(t c_v) = 0$$
                    per remark \ref{remark: total_degrees_of_classical_yangian_R_matrices}.
                \end{enumerate}
            \end{proof}

        \begin{remark}[Topological issues surrounding $\tilde{\delta}^+$] \label{remark: topological_toroidal_lie_bialgebras}
            One issue that we have neglected to discuss so far is the fact that $\tilde{\delta}^+$ is somehow inherently topological: e.g. we now know that:
                $$\tilde{\delta}^+(ht) = [ht \tensor 1, \sfr_{\g} v_2 \1(v_1, v_2)]$$
            which means that \textit{a priori}, $\tilde{\delta}^+(ht)$ is not an element of $\tilde{\g}_{[2]}^+ \tensor_{\bbC} \tilde{\g}_{[2]}^+$ but rather of some topological completion thereof. To make sense of this, we can rely on the natural $\Z$-grading on $\tilde{\g}_{[2]}^+$.

            Recall once again that:
                $$\tilde{\g}_{[2]}^+ \cong \g_{[2]}^+ \oplus \bar{\Omega}_{[2]}^+$$
            (with notations as in remarks \ref{remark: centres_of_dual_toroidal_lie_algebras} and \ref{remark: Z_gradings_on_toroidal_lie_algebras}) and then recall that on the $\g_{[2]}^+$, there is a $\Z_{\geq 0}$-grading given by:
                $$\forall x \in \g: \deg x = 0$$
                $$\deg v = 0, \deg t = 1$$
            which then induces a $(-1 + \Z)$-grading on the centre $\z_{[2]}^+ \cong \bar{\Omega}_{[2]}^+$ given by:
                $$\deg dv = -1, \deg dt = 0$$
            Equivalent, and more explicitly, this grading is given on the basis elements of $\bar{\Omega}_{[2]}^+$ by:
                $$
                    \deg K_{r, s} =
                    \begin{cases}
                        \text{$s - 1$ if $(r, s) \in \Z \x (\Z \setminus \{0\})$}
                        \\
                        \text{$-1$ if $(r, s) \in \Z \x \{0\}$}
                        \\
                        \text{$0$ if $(r, s) = (0, 0)$}
                    \end{cases}
                $$
                $$\deg c_v = \deg c_t = -1$$
            (cf. remark \ref{remark: Z_gradings_on_toroidal_lie_algebras}). 

            \todo[inline]{How do we topologically complete graded Lie algebras ?}
        \end{remark}
    
    \subsection{Hopf coproducts and classical limits of completed affine Yangians}
        \begin{convention}
            In this subsection, we assume that $\g$ is simply laced, excluding the case where $\g$ is of type $\sfA_1$. 
        \end{convention}

        We begin this subsection by reviewing the construction of what we shall call the \say{Hopf coproduct} $\Delta$ on the Yangian $\rmY_{\hbar}(\hat{\g}_{[1]})$, as was done in \cite[Sections 4 and 5]{guay_nakajima_wendlandt_affine_yangian_coproduct}. We will then lift this map to the formal Yangian $\rmY_{\hbar}(\hat{\g}_{[1]})$ to get another \say{Hopf coproduct} $\Delta_{\hbar}$ thereon. The point of doing this is so that ultimately, we would obtain:
            $$\frac{1}{\hbar}(\Delta_{\hbar} - \Delta_{\hbar}^{\cop}) \pmod{\hbar} \equiv \tilde{\delta}^+$$
        and hence be able to realise the topological Lie bialgebra $(\tilde{\g}_{[2]}^+, \tilde{\delta}^+)$ from theorem \ref{theorem: toroidal_lie_bialgebras} as the classical limit of the formal Yangian $\rmY_{\hbar}(\hat{\g}_{[1]})$ in some sense (which, let us caution, is not exactly the same as in \cite{etingof_kazhdan_quantisation_1}).
        
        The fundamental inspiration behind this construction of \cite{guay_nakajima_wendlandt_affine_yangian_coproduct} is - to our understanding - the fact that, should $(H, \mu, \eta, \Delta, \e, \sigma)$ be a Hopf algebra, then given any two left-$H$-modules:
            $$\rho_V: H \to \End(V)$$
            $$\rho_W: H \to \End(W)$$
        (and likewise for right-modules), one can always form the $H$-modules tensor product using the following series of compositions:
            $$H \xrightarrow[]{\Delta} H \tensor H \xrightarrow[]{\id_H \tensor \sigma} H \tensor H^{\op, \cop} \xrightarrow[]{\rho_V \tensor \rho_W^{\op}} \End(V) \tensor \End(W) \to \End(V \tensor W)$$
        wherein the last arrow is the natural one.
        
        In \textit{loc. cit.}, the authors first established that the so-called \say{category $\calO$} of $\rmY(\hat{\g}_{[1]})$ (the definition is similar to the usual BGG category $\calO$ of $\g$; cf. lemma \ref{lemma: category_O_affine_yangian}) is closed under tensor products. One as such is able to write down a $\bbC$-algebra homomorphism:
            $$\Delta_{V_1, V_2}: \rmY(\hat{\g}_{[1]}) \to \End_{\bbC}(V_1 \tensor_{\bbC} V_2)$$
        corresponding to any pair of objects $V_1, V_2$ in the aforementioned category $\calO$ of $\rmY(\hat{\g}_{[1]})$; furthermore, such algebra homomorphisms satisfy coassociativity in the sense that any $\bbC$-vector space isomorphism:
            $$(V_1 \tensor_{\bbC} V_2) \tensor_{\bbC} V_3 \xrightarrow[]{\cong} V_1 \tensor_{\bbC} (V_2 \tensor_{\bbC} V_3)$$
        between objects $V_1, V_2, V_3$ of the category $\calO$ of $\rmY(\hat{\g}_{[1]})$ upgrades to an isomorphism of left-$\rmY(\hat{\g}_{[1]})$-modules. As such, one obtains a kind of \say{\textit{faux} coproduct} on $\rmY(\hat{\g}_{[1]})$. 

        Now, how may we \say{package} the maps $\Delta_{V_1, V_2}$ together (say, by varying the representations $V_1, V_2 \in \Ob(\calO)$ somehow) to obtain a bialgebra coproduct on the Yangian $\rmY(\hat{\g}_{[1]})$ ? To this end, note firstly that, due to the Serre relations in the definition of Yangians (cf. e.g. \cite[Equation 2.7]{guay_nakajima_wendlandt_affine_yangian_coproduct}), the adjoint representation of $\rmY(\hat{\g}_{[1]})$ is \say{integrable} \todo{Not done}
        
        \begin{lemma}[The category $\calO$ for the affine Yangian $\rmY(\hat{\g}_{[1]})$] \label{lemma: category_O_affine_yangian}
            (Cf. \cite[Theorem 4.9]{guay_nakajima_wendlandt_affine_yangian_coproduct}).
        
            There is a full subcategory of ${}^l\rmY(\hat{\g}_{[1]})\mod$, called the \textbf{category $\calO$}. This category satisfies the following properties:
            \begin{itemize}
                \item Every object $V \in \Ob(\calO)$ is $\hat{\h}_{[1]}$-diagonalisable and with finite-dimensional ($\h$-)weight spaces, and
                \item For every object $V \in \Ob(\calO)$, there exist \textbf{maximal weights} $\lambda_1, ..., \lambda_k \in \hat{\h}_{[1]}^*$ such that, for any $\mu \in \Pi(V)$, one has that:
                    $$\forall 1 \leq i \leq k: \lambda_i - \mu \in \hat{\Lambda}^+$$
            \end{itemize}

            The aforementioned category $\calO$ of $\rmY(\hat{\g}_{[1]})$ is closed under tensor products over $\bbC$, i.e. if $V_1, V_2$ are any two objects of the category $\calO$, then there will be a $\bbC$-algebra homomorphism:
                $$\Delta_{V_1, V_2}: \rmY(\hat{\g}_{[1]}) \to \End_{\bbC}(V_1 \tensor_{\bbC} V_2)$$
            Furthermore, these tensor products are coassociative in the sense that any $\bbC$-vector space isomorphism:
                $$(V_1 \tensor_{\bbC} V_2) \tensor_{\bbC} V_3 \xrightarrow[]{\cong} V_1 \tensor_{\bbC} (V_2 \tensor_{\bbC} V_3)$$
            between objects $V_1, V_2, V_3 \in \Ob(\calO)$ upgrades to an isomorphism of left-$\rmY(\hat{\g}_{[1]})$-modules.

            Explicitly, for each $V_1, V_2 \in \Ob(\calO)$, the map $\Delta_{V_1, V_2}$ is given on the generating set\footnote{Using the Levendorskii presentation for $\rmY(\hat{\g}_{[1]})$, one sees that this generating set suffices.} $\hat{\h}_{[1]} \cup \{T_{i, 1}, E_{i, 0}^{\pm}\}_{i \in \hat{\Gamma}_0}$ by:
                $$\forall h \in \hat{\h}_{[1]}: \Delta_{V_1, V_2}(h) := \bar{\Delta}(h)$$
                $$\forall i \in \hat{\Gamma}_0: \Delta_{V_1, V_2}(E_{i, 0}^{\pm}) := \bar{\Delta}(E_{i, 0}^{\pm})$$
                $$\forall i \in \hat{\Gamma}_0: \Delta_{V_1, V_2}(T_{i, 0}) = \bar{\Delta}(T_{i, 0}) + [H_{i, 0} \tensor 1, \sfr_{ \hat{\g}_{[1]} }^-]$$
            with $\sfr_{ \hat{\g}_{[1]} }^-$ being the Casimir tensor\footnote{This is denoted by $\Omega_+$ in \cite{guay_nakajima_wendlandt_affine_yangian_coproduct} and \cite{guay_nakajima_wendlandt_affine_yangian_vertex_representations_and_PBW}. We opted to designate this the \say{negative} half of the Casimir tensor of $\hat{\g}_{[1]}$ in accordance with the root-degree of the first tensor factor. Also, in \textit{loc. cit.}, the authors considered the Casimir tensor associated to the Kac-Moody pairing on $\hat{\h}_{[1]} \tensor_{\bbC} \hat{\h}_{[1]} \oplus \hat{\n}_{[1]}^- \hattensor_{\bbC} \hat{\n}_{[1]}^+$, but we need only the \say{triangular} component since the Cartan component will be killed by $[H_{i, 0} \tensor 1, -]$ anyway.} associated to the non-degenerate Kac-Moody pairing on $\hat{\n}_{[1]}^- \hattensor_{\bbC} \hat{\n}_{[1]}^+$.
        \end{lemma}
        \begin{remark}
            The category $\calO$ as in lemma \ref{lemma: category_O_affine_yangian} is \textit{not} monoidal, since it lacks a monoidal unit. 
        \end{remark}
        \begin{remark}[Why involve the category $\calO$ ?]
            For a moment, let us pick the root bases $\{ e_{\alpha, k}^{\pm} \}_{(\alpha, k) \in \hat{\Phi}^+ \x \{1, ..., \dim_{\bbC} (\hat{\g}_{[1]})_{\alpha} \}}$ for $\hat{\n}_{[1]}^{\pm}$ in such a way that they are dual to one another with respect to the Kac-Moody pairing on $\hat{\g}_{[1]}$. In terms of these bases, one can write:
                $$\sfr_{\hat{\g}_{[1]}}^+ = \sum_{\alpha \in \hat{\Phi}^+} \sum_{k = 1}^{ \dim_{\bbC} (\hat{\g}_{[1]})_{\alpha} } e_{\alpha, k}^- \tensor e_{\alpha, k}^+$$
        
            One notable detail is the fact that the sum\footnote{The completed tensor product $\rmY(\hat{\g}_{[1]}) \hattensor_{\bbC} \rmY(\hat{\g}_{[1]})$ is only to be understood in the vague sense that it denotes some completion of the algebraic tensor product $\rmY(\hat{\g}_{[1]}) \tensor_{\bbC} \rmY(\hat{\g}_{[1]})$ wherein the sum in question converges.}:
                $$\sum_{\alpha \in \hat{\Phi}^+} \sum_{k = 1}^{ \dim_{\bbC} (\hat{\g}_{[1]})_{\alpha} } e_{\alpha, k}^- \tensor e_{\alpha, k}^+ \in \rmY(\hat{\g}_{[1]}) \hattensor_{\bbC} \rmY(\hat{\g}_{[1]})$$
            is infinite \textit{a priori}, since the affine Kac-Moody algebra $\hat{\g}_{[1]}$ has infinitely many positive roots. However, this is precisely why we have restricted our attention down to the category $\calO$: notice that for any $V \in \Ob(\calO)$ and any $\mu \in \Pi(V)$, there exists a natural number $N \in \N$ such that:
                $$\forall \alpha \in \hat{\Phi}^+: r \geq N \implies V_{\mu + r \alpha} \cong 0$$
            From this, one sees that even though it is given by an infinite sum, the operator:
                $$\sum_{\alpha \in \hat{\Phi}^+} \sum_{k = 1}^{ \dim_{\bbC} (\hat{\g}_{[1]})_{\alpha} } e_{\alpha, k}^- \tensor e_{\alpha, k}^+ \in \End_{\bbC}(V_1 \tensor_{\bbC} V_2)$$
            is ultimately locally nilpotent on the vector spaces of the kind $V_1 \tensor_{\bbC} V_2$, wherein $V_1, V_2 \in \Ob(\calO)$; as such, one sees that the infinite sum above actually becomes finite (and hence converges) after evaluation on elements of the $\rmY(\hat{\g}_{[1]})$-modules in the category $\calO$, and the maps $\Delta_{V_1, V_2}$ as in lemma \ref{lemma: category_O_affine_yangian} are therefore well-defined. 
        \end{remark}
        \begin{convention}
            If $L$ is a Kac-Moody algebra of some simply laced untwisted affine type and then we will denote by $\breve{\rmY}(L)$ the completion of $\rmY(L)$ with respect to its root grading, with \say{completion} being understood to be in the sense of \cite[Appendix A]{wendlandt_formal_shift_operators_on_yangian_doubles}.
        \end{convention}
        \begin{lemma}[$\rmY(\hat{\g}_{[1]})$-modules are $\breve{\rmY}(\hat{\g}_{[1]})$-modules] \label{lemma: lifting_representations_of_affine_yangians_to_root_grading_completions}
            (Cf. \cite[Proposition 5.14]{guay_nakajima_wendlandt_affine_yangian_coproduct}) Any left-$\rmY(\hat{\g}_{[1]})$-module $V$ in the category $\calO$, given by a $\bbC$-algebra homomorphism:
                $$\rho: \rmY(\hat{\g}_{[1]}) \to \End_{\bbC}(V)$$
            gives rise to a unique left-$\breve{\rmY}(\hat{\g}_{[1]})$-module structure on $V$, which is the same as a $\bbC$-algebra homomorphism:
                $$\breve{\rho}: \breve{\rmY}(\hat{\g}_{[1]}) \to \End_{\bbC}(V)$$
            fitting into the following commutative diagram of $\bbC$-algebras and homomorphisms between them, where the vertical arrow is the canonical one as in \cite[Section 5, Lemma 5.3]{guay_nakajima_wendlandt_affine_yangian_coproduct}:
                $$
                    \begin{tikzcd}
                	{\breve{\rmY}(\hat{\g}_{[1]})} & {\End_{\bbC}(V)} \\
                	{\rmY(\hat{\g}_{[1]})}
                	\arrow[from=2-1, to=1-1]
                	\arrow["{\breve{\rho}}", dashed, from=1-1, to=1-2]
                	\arrow["\rho"', from=2-1, to=1-2]
                    \end{tikzcd}
                $$
        \end{lemma}
        \begin{proposition}[Hopf coproduct on affine Yangians] \label{prop: hopf_coproduct_on_yangians}
            (Cf. \cite[Proposition 5.18]{guay_nakajima_wendlandt_affine_yangian_coproduct}) There exists a $\bbC$-algebra homomorphism:
                $$\Delta: \rmY(\hat{\g}_{[1]}) \to \breve{\rmY}(\hat{\g}_{[1]} \oplus \hat{\g}_{[1]})$$
            satisfying:
                $$\Delta_{V_1, V_2} = (\breve{\rho}_1 \tensor \breve{\rho}_2) \circ \Delta$$
            for any objects $(V_1, \rho_1), (V_2, \rho_2)$ of the category $\calO$ of $\rmY(\hat{\g}_{[1]})$.
        \end{proposition}
        
        \begin{lemma}[The category $\calO_{\hbar}$ of the formal affine Yangian $\rmY_{\hbar}(\hat{\g}_{[1]})$] \label{lemma: category_O_formal_affine_yangian}
            For the formal affine Yangian $\rmY_{\hbar}(\hat{\g}_{[1]})$, one can define a category $\calO_{\hbar}$ in the exact same way\footnote{Ultimately, this is because we have that $[h, E_{i, r}^{\pm}] = \pm \alpha_i(h) E_{i, r}^{\pm} \in \rmY_{\hbar}(L) \setminus \hbar\rmY_{\hbar}(L) \cong \rmY^0(L)$ for all Cartan elements $h \in \hat{\h}_{[1]}$.} as how the category $\calO$ was defined for $\rmY(\hat{\g}_{[1]})$ in lemma \ref{lemma: category_O_affine_yangian}. 

            The category $\calO_{\hbar}$ is closed under $\tensor_{\bbC}$ (cf. lemma \ref{lemma: category_O_affine_yangian}): for every $V_1, V_2 \in \Ob(\calO_{\hbar})$, there is a corresponding $\bbC$-algebra homomorphism:
                $$\Delta_{V_1, V_2, \hbar}: \rmY_{\hbar}(\hat{\g}_{[1]}) \to \End_{\bbC}(V_1 \tensor_{\bbC} V_2)$$
            given by:
                $$\forall h \in \hat{\h}_{[1]}: \Delta_{V_1, V_2, \hbar}(h) := \bar{\Delta}(h)$$
                $$\forall i \in \hat{\Gamma}_0: \Delta_{V_1, V_2, \hbar}(E_{i, 0}^{\pm}) := \bar{\Delta}(E_{i, 0}^{\pm})$$
                $$\forall i \in \hat{\Gamma}_0: \Delta_{V_1, V_2, \hbar}(T_{i, 0}) = \bar{\Delta}(T_{i, 0}) + \hbar [H_{i, 0} \tensor 1, \sfr_{ \hat{\g}_{[1]} }^- ]$$
            Furthermore, the tensor products in $\calO_{\hbar}$ are coassociative in the same sense as in lemma \ref{lemma: category_O_affine_yangian}.
        \end{lemma}
            \begin{proof}
                This is a consequence of lemma \ref{lemma: category_O_affine_yangian} and the fact that we have a graded $\bbC$-algebra isomorphism:
                    $$\rmY_{\hbar}(\hat{\g}_{[1]}) \xrightarrow[]{\cong} \Rees_{\hbar} \rmY(\hat{\g}_{[1]})$$
                (cf. lemma \ref{lemma: formal_yangians_as_rees_algebras}).
            \end{proof}
        \begin{convention}
            If $L$ is a Kac-Moody algebra of some simply laced untwisted affine type and then we will denote by $\breve{\rmY}_{\hbar}(L)$ the completion of $\rmY_{\hbar}(L)$ with respect to its root grading, with \say{completion} being understood to be in the sense of \cite[Appendix A]{wendlandt_formal_shift_operators_on_yangian_doubles}. Note that this root grading is the same as the one on $\rmY(L)$ due to the fact that:
                $$\forall h \in \hat{\h}_{[1]}: [h, E_{i, r}^{\pm}] = \pm \alpha_i(h) E_{i, r}^{\pm} \in \rmY_{\hbar}(L) \setminus \hbar\rmY_{\hbar}(L) \cong \rmY^0(L)$$
            so the construction of $\breve{\rmY}_{\hbar}(L)$ from $\rmY_{\hbar}(L)$ is the same as that of $\breve{\rmY}(L)$ from $\rmY(L)$.
        \end{convention}
        \begin{lemma}[$\rmY_{\hbar}(\hat{\g}_{[1]})$-modules are $\breve{\rmY}_{\hbar}(\hat{\g}_{[1]})$-modules] \label{lemma: lifting_representations_of_formal_affine_yangians_to_root_grading_completions}
            Any left-$\rmY_{\hbar}(\hat{\g}_{[1]})$-module $V$ in the category $\calO$, given by a $\bbC$-algebra homomorphism:
                $$\rho: \rmY_{\hbar}(\hat{\g}_{[1]}) \to \End_{\bbC}(V)$$
            gives rise to a unique left-$\breve{\rmY}_{\hbar}(\hat{\g}_{[1]})$-module structure on $V$, which is the same as a $\bbC$-algebra homomorphism:
                $$\breve{\rho}: \breve{\rmY}_{\hbar}(\hat{\g}_{[1]}) \to \End_{\bbC}(V)$$
            fitting into the following commutative diagram of $\bbC$-algebras and homomorphisms between them, where the vertical arrow is the canonical inclusion (cf. \cite[Section 5, Lemma 5.3]{guay_nakajima_wendlandt_affine_yangian_coproduct}):
                $$
                    \begin{tikzcd}
                	{\breve{\rmY}_{\hbar}(\hat{\g}_{[1]})} & {\End_{\bbC}(V)} \\
                	{\rmY_{\hbar}(\hat{\g}_{[1]})}
                	\arrow[from=2-1, to=1-1]
                	\arrow["{\breve{\rho}}", dashed, from=1-1, to=1-2]
                	\arrow["\rho"', from=2-1, to=1-2]
                    \end{tikzcd}
                $$
        \end{lemma}
            \begin{proof}
                
            \end{proof}
        \begin{theorem}[Hopf coproduct on formal affine Yangians] \label{theorem: hopf_coproduct_on_formal_yangians}
            The $\bbC$-algebra homomorphism $\Delta: \rmY(\hat{\g}_{[1]}) \to \breve{\rmY}(\hat{\g}_{[1]} \oplus \hat{\g}_{[1]})$ from proposition \ref{prop: hopf_coproduct_on_yangians} lifts\footnote{... in the sense that $\Delta_{\hbar} \pmod{(\hbar - \hbar_0)} \equiv \Delta$ for any $\hbar_0 \in \bbC^{\x}$.} to a $\bbC$-algebra homomorphism:
                $$\Delta_{\hbar}: \rmY_{\hbar}(\hat{\g}_{[1]}) \to \breve{\rmY}_{\hbar}(\hat{\g}_{[1]} \oplus \hat{\g}_{[1]})$$
            satisfying:
                $$\Delta_{V_1, V_2, \hbar} = (\breve{\rho}_1 \tensor \breve{\rho}_2) \circ \Delta_{\hbar}$$
            for any $(V_1, \rho_1), (V_2, \rho_2) \in \Ob(\calO_{\hbar})$.
        \end{theorem}
            \begin{proof}
                
            \end{proof}
        
        \begin{theorem}[Toroidal Lie algebras as classical limits of formal affine Yangians] \label{theorem: toroidal_lie_algebras_as_classical_limits_of_formal_affine_yangians}
           The Lie bialgebra $(\tilde{\g}_{[2]}^+, \tilde{\delta}^+)$ from theorem \ref{theorem: toroidal_lie_bialgebras} is the classical limit of the formal affine Yangian $\rmY_{\hbar}(\hat{\g}_{[1]})$ with the \say{coproduct} $\Delta_{\hbar}$ (as in theorem \ref{theorem: hopf_coproduct_on_formal_yangians}), in the sense that:
                $$\frac{1}{\hbar}( \Delta_{\hbar} - \Delta_{\hbar}^{\cop} ) \equiv \tilde{\delta}^+ \pmod{\hbar}$$
        \end{theorem}
            \begin{proof}
                Before we begin computing, let us make the preliminary observation that $T_{i, 1}(\hbar) \in \rmY_{\hbar}(\hat{\g}_{[1]})$ is a lift modulo $\hbar$ of $H_{i, 1} \in \t^+ \cong \tilde{\g}_{[2]}^+$ (cf. corollary \ref{coro: chevalley_serre_presentation_for_central_extensions_of_multiloop_algebras}):
                    $$T_{i, 1}(\hbar) := H_{i, 1} - \frac12 \hbar H_{i, 0}^2 \equiv H_{i, 1} \pmod{\hbar}$$
                Also, let us note that, we know from lemma \ref{lemma: levendorskii_presentation} and corollary \ref{coro: levendorskii_presentation__for_central_extensions_of_multiloop_algebras} that it is enough to only check the value of $\Delta_{\hbar}^{\cop}$ on the generators $H_{i, 0}, E_{i, 0}^{\pm}$, and $T_{i, 1} \equiv H_{i, 1} \pmod{\hbar}$, for all $i \in \hat{\Gamma}_0$.
            
                Firstly, from theorem \ref{theorem: hopf_coproduct_on_formal_yangians}, we know that:
                    $$\forall h \in \hat{\h}_{[1]}: \Delta_{\hbar}(h) := \bar{\Delta}(h)$$
                    $$\forall i \in \hat{\Gamma}_0: \Delta_{\hbar}(E_{i, 0}^{\pm}) := \bar{\Delta}(E_{i, 0}^{\pm})$$
                    $$\forall i \in \hat{\Gamma}_0: \Delta_{\hbar}(T_{i, 0}) = \bar{\Delta}(T_{i, 0}) + [H_{i, 0} \tensor 1, \sfr_{ \hat{\g}_{[1]} }^-]$$
                This tells us that:
                    $$\forall h \in \hat{\h}_{[1]}: \Delta_{\hbar}^{\cop}(h) := \bar{\Delta}(h)$$
                    $$\forall i \in \hat{\Gamma}_0: \Delta_{\hbar}^{\cop}(E_{i, 0}^{\pm}) := \bar{\Delta}(E_{i, 0}^{\pm})$$
                    $$\forall i \in \hat{\Gamma}_0: \Delta_{\hbar}^{\cop}(T_{i, 0}) = \bar{\Delta}(T_{i, 0}) + [1 \tensor H_{i, 0}, \sfr_{ \hat{\g}_{[1]} }^+]$$
                with $\sfr_{\hat{\g}_{[1]}}^+$ being the Casimir tensor associated to the non-degenerate Kac-Moody pairing on $\hat{\n}_{[1]}^+ \hattensor_{\bbC} \hat{\n}_{[1]}^-$.

                It is then trivial that:
                    $$\frac{1}{\hbar}( \Delta_{\hbar} - \Delta_{\hbar}^{\cop} )(X) = 0$$
                for:
                    $$X \in \hat{\h}_{[1]} \cup \{E_{i, 0}^{\pm}\}_{i \in \hat{\Gamma}_0}$$
                which implies that:
                    $$\frac{1}{\hbar}( \Delta_{\hbar} - \Delta_{\hbar}^{\cop} )(X) \equiv \tilde{\delta}^+(X) \pmod{\hbar}$$
                which is because it is known from theorem \ref{theorem: toroidal_lie_bialgebras} that:
                    $$\tilde{\delta}^+(X) = 0$$
                whenever $\deg X = 0$, which is the case here.
                
                Now, let us verify that:
                    $$\frac{1}{\hbar}(\Delta_{\hbar} - \Delta_{\hbar}^{\cop})(T_{i, 1}) \equiv \tilde{\delta}^+(H_{i, 1})$$
                It is not hard to see that\footnote{One can prove this by e.g. picking the root bases foor $\hat{\n}_{[1]}^{\pm}$.}:
                    $$[1 \tensor H_{i, 0}, \sfr_{ \hat{\g}_{[1]} }^+] = -[H_{i, 0} \tensor 1, \sfr_{ \hat{\g}_{[1]} }^+]$$
                which tells us that:
                    $$\frac{1}{\hbar}( \Delta_{\hbar} - \Delta_{\hbar}^{\cop} )(T_{i, 1}) = [H_{i, 0} \tensor 1, \sfr_{ \hat{\g}_{[1]} }^- + \sfr_{ \hat{\g}_{[1]} }^+]$$
                Since $H_{i, 0}$ commutes with every element of $\hat{\h}_{[1]}$, we can equivalently rewrite the above into:
                    $$\frac{1}{\hbar}( \Delta_{\hbar} - \Delta_{\hbar}^{\cop} )(T_{i, 1}) = [H_{i, 0} \tensor 1, \sfr_{\hat{\h}_{[1]}} + \sfr_{ \hat{\g}_{[1]} }^- + \sfr_{ \hat{\g}_{[1]} }^+] = [H_{i, 0} \tensor 1, \sfr_{ \hat{\g}_{[1]} }]$$
                wherein $\sfr_{\hat{\h}_{[1]}}$ is the Casimir element associated to the Kac-Moody pairing on $\hat{\h}_{[1]} \tensor_{\bbC} \hat{\h}_{[1]}$. 

                We know from theorem \ref{theorem: toroidal_lie_bialgebras} that:
                    $$\tilde{\delta}^+(H_{i, 1}) = [ H_{i, 0} \tensor 1, \sfr_{\g} v_2 \1(v_1, v_2) ]$$
                so we will be done if we can show that:
                    $$[H_{i, 0} \tensor 1, \sfr_{ \hat{\g}_{[1]} }] = [ H_{i, 0} \tensor 1, \sfr_{\g} v_2 \1(v_1, v_2) ]$$
                From the fact that:
                    $$\hat{\g}_{[1]} \cong \g_{[1]} \oplus \z_{[1]} \oplus \d_{[1]} \cong \g_{[1]} \oplus \bbC c_v \oplus \bbC D_{0, -1}$$
                (with notations as in conventions \ref{conv: a_fixed_untwisted_affine_kac_moody_algebra} and \ref{conv: orthogonal_complement_of_toroidal_centres}) and implicitly from the solution to question \ref{question: multiloop_lie_bialgebras} (notice that the bilinear form on $\g_{[2]}$ in \textit{loc. cit.} is nothing but the Kac-Moody form; cf. \cite[Chapter 7]{kac_infinite_dimensional_lie_algebras}), which is given by:
                    $$\forall x, y \in \g: \forall m, n \in \Z: (x v^m, y v^n)_{\hat{\g}_{[1]}} = (x, y)_{\g} \delta_{m + n, 0}$$
                    $$(K_{0, -1}, D_{0, -1})_{\hat{\g}_{[1]}} = 1$$
                we infer that:
                    $$\sfr_{ \hat{\g}_{[1]} } = \sfr_{\g} v_2 \1(v_1, v_2) + \sfr_{\z_{[1]}} + \sfr_{\d_{[1]}} = \sfr_{\g} v_2 \1(v_1, v_2) + K_{0, -1} \tensor D_{0, -1} + D_{0, -1} \tensor K_{0, -1}$$
                wherein $\sfr_{\z_{[1]}}, \sfr_{\d_{[1]}}$ respectively denote the Casimir elements corresponding to the Kac-Moody form on $\z_{[1]} \tensor_{\bbC} \d_{[1]}$ and on $\d_{[1]} \tensor_{\bbC} \z_{[1]}$ respectively. The element $K_{0, -1} \in \tilde{\g}_{[1]} := \g_{[1]} \oplus \z_{[1]}$ is central and therefore commutes with $H_{i, 0}$, which implies that:
                    $$[H_{i, 0} \tensor 1, \sfr_{\z_{[1]}}] = [H_{i, 0} \tensor 1, K_{0, -1} \tensor D_{0, -1}] = 0$$
                At the same time, we also know that $D_{0, -1}$ acts as $\id_{\g} \tensor \left(-v \frac{d}{dv}\right)$ on $\g_{[1]}$ and hence as zero on the elements of $\g$ (i.e. degree-$0$ elements of $\g_{[1]}$), and so:
                    $$[H_{i, 0} \tensor 1, \sfr_{\d_{[1]}}] = [H_{i, 0} \tensor 1, D_{0, -1} \tensor K_{0, -1}] = 0$$
                as well. As such, we have demonstrated that:
                    $$[H_{i, 0} \tensor 1, \sfr_{ \hat{\g}_{[1]} }] = [ H_{i, 0} \tensor 1, \sfr_{\g} v_2 \1(v_1, v_2) ]$$
                as we sought to. As mentioned above, this allows us to conclude that:
                    $$\frac{1}{\hbar}( \Delta_{\hbar} - \Delta_{\hbar}^{\cop} )(T_{i, 1}) \equiv \tilde{\delta}^+(H_{i, 1}) \pmod{\hbar}$$
            \end{proof}