\section{R-matrices of toroidal Lie bialgebras and of affine Yangians}
    \subsection{Classical R-matrices of toroidal Lie bialgebras}
        \begin{convention}
            We assume familiarity with the fact that, for any homomorphism of commutative rings $R \to S$, one has that:
                $$\Hom_S(\Omega_{S/R}^1, N) \cong \Der_R(S, N)$$
            for all $S$-modules $N$ (cf. \cite[\href{https://stacks.math.columbia.edu/tag/00RO}{Tag 00RO}]{stacks}).
        \end{convention}
        
        \begin{remark}[Pairing of $1$-forms and vector fields] \label{remark: pairing_1_forms_and_vector_fields} 
            Suppose for a moment that $A$ is a commutative algebra over a field $k$ generated by some set:
                $$\{v_i\}_{i \in I}$$
            and suppose furthermore that $A$ is equipped with a non-degenerate and symmetric $k$-bilinear form:
                $$(-, -)_A: A \x A \to k$$
            Next, consider the following natural pairing between $\Omega^1_{A/k}$ and $\der_k(A)$, given as the interior product/contraction of differential forms by vector fields, i.e. in the following manner:
                $$\iota_A(f dv_i, g \del_{v_j}) := (f, g)_A \delta_{i, j}$$
            for all $f, g \in A$. Clearly, the pairing:
                $$\iota_A: \Omega^1_{A/k} \x \der_k(A) \to k$$
            is a non-degenerate and symmetric $k$-bilinear form.

            Now, in the particular case of $A := A_{[n]}$ (for some $n \geq 1$), note firstly that the algebra $A_{[n]}$ is naturally equipped with the non-degenerate and symmetric $k$-bilinear form:
                $$(v_1^{m_1} ... v_n^{m_n}, v_1^{m_1'} ... v_n^{m_n'})_{A_{[n]}} := \delta_{(m_1 + m_1', ..., m_{n - 1} + m_{n - 1}', m_n + m_n'), (0, ..., 0, -1)}$$
            (cf. question \ref{question: multiloop_lie_bialgebras}) and with respect to this, the pairing:
                $$\iota_{A_{[n]}}: \Omega_{[n]} \x \der_k(A_{[n]}) \to k$$
            is then given by:
                $$\iota_{A_{[n]}}(f dv_i, g \del_{v_j}) := (f, g)_{A_{[n]}} \delta_{i, j}$$
            for all $f, g \in A_{[n]}$ and all $1 \leq i, j \leq n$. 
        \end{remark}

        \begin{remark}[Pairing of cyclic $1$-forms and divergence-zero vector fields] \label{remark: pairing_cyclic_1_forms_and_div_zero_vector_fields} 
            Again, suppose for a moment that $A$ is an arbitrary \textit{smooth}\footnote{... so that $\Omega^1_{A/k}$ would be finite free as an $A$-module.} commutative algebra over a field $k$ of characteristic $0$ and that $A$ is generated by some set:
                $$\{v_i\}_{i \in I}$$
            and that $A$ carries a non-degenerate and symmetric $k$-bilinear form $(-, -)_A$. Via the canonical quotient map of $k$-vector spaces:
                $$\Omega^1_{A/k} \to \bar{\Omega}^1_{A/k}$$
            one obtains an induced bilinear form that we will denote by:
                $$\bar{\iota}_A: \bar{\Omega}^1_{A/k} \x \der_k(A) \to k$$

            When $A = A_{[n]}$, recall from remark \ref{remark: centres_of_dual_toroidal_lie_algebras} that there is a canonical basis for $\bar{\Omega}^1_{A/k}$ consisting of the elements:
                $$m_i^{-1} v_1^{m_1} ... v_i^{m_i - 1} ... v_n^{m_n} \bar{d}(v_i)$$
            wherein $1 \leq i \leq n$. The bilinear form $\bar{\iota}_{[n]} := \bar{\iota}_{A_{[n]}}$ is then given by:
                $$\bar{\iota}_{[n]}( m_i^{-1} v_1^{m_1} ... v_i^{m_i - 1} ... v_n^{m_n} \bar{d}(v_i), v_1^{m_1'} ... v_n^{m_n'} \del_{v_j} ) = m_i^{-1} \delta_{(m_1 + m_1', ..., m_i + m_i' - 1, ..., m_n + m_n'), (0, ..., 0, -1)} \delta_{i, j}$$
            for all $1 \leq i, j \leq n$. While it is true that $\bar{\iota}_{[n]}$ is non-degenerate, we caution that - based on the calculation above - $\bar{\iota}_{[n]}$ is not symmetric due to the appearance of the $m_i^{-1}$ factor: switching the two inputs would yield a $(-m'_i + 1)^{-1}$ factor instead. The upshot is that, once we restrict the second input from $\der_k(A_{[n]})$ to the \say{Yangian divergence-zero} subspace $\d_{[n]}$, the bilinear form $\bar{\iota}_{[n]}$ ought to get symmetrised, and symmetry is needed for us to be able to write down a canonical element corresponding to the resulting symmetric pairing.

            Let us now focus on the case where $k = \bbC$ and $n = 2$, partly because this is the only case wherein we have a concrete basis for $\d_{[2]}$ to work with, but also because ultimately, this is the only case of particular interest to us. Recall from remark \ref{remark: dual_of_toroidal_centres_contains_derivations} and from the construction of $\d_{[2]}$ (cf. convention \ref{conv: orthogonal_complement_of_toroidal_centres}) that:
                $$\d_{[2]} := (\bigoplus_{(r, s) \in \Z^2} \bbC D_{r, s}) \oplus \bbC D_v \oplus \bbC D_t$$
            wherein:
                $$\forall (r, s) \in \Z^2: D_{r, s} = s v^{-r + 1} t^{-s - 1} \del_v - r v^{-r} t^{-s} \del_t$$
                $$D_v = -v t^{-1} \del_v, D_t = -\del_t$$
            Let us also recall from remark \ref{remark: centres_of_dual_toroidal_lie_algebras} that the canonical basis elements of $\bar{\Omega}_{[2]}$ can be alternatively written as:
                $$
                    K_{r, s} :=
                    \begin{cases}
                        \text{$\frac1s v^{r - 1} t^s \bar{d}(v)$ if $(r, s) \in \Z \x (\Z \setminus \{0\})$}
                        \\
                        \text{$-\frac1r v^r t^{-1} \bar{d}(t)$ if $(r, s) \in (\Z \setminus \{0\}) \x \{0\}$}
                        \\
                        \text{$0$ if $(r, s) = (0, 0)$}
                    \end{cases}
                $$
                $$c_v := v^{-1} \bar{d}(v), c_t := t^{-1} \bar{d}(t)$$
            It is clear that:
                $$
                    \bar{\iota}_{[2]}( K_{r, s}, D ) =
                    \begin{cases}
                        \text{$1$ if $D = D_{r, s}$}
                        \\
                        \text{$0$ if $D \not \in \bbC D_{r, s}$}
                    \end{cases}
                $$
                $$
                    \bar{\iota}_{[2]}( c_v, D ) =
                    \begin{cases}
                        \text{$1$ if $D = D_v$}
                        \\
                        \text{$0$ if $D \not \in \bbC D_v$}
                    \end{cases}
                $$
                $$
                    \bar{\iota}_{[2]}( c_t, D ) =
                    \begin{cases}
                        \text{$1$ if $D = D_t$}
                        \\
                        \text{$0$ if $D \not \in \bbC D_t$}
                    \end{cases}
                $$
            so we have managed to show that not only is there a non-degenerate and symmetric $\bbC$-bilinear form:
                $$\bar{\iota}_{[2]}: \bar{\Omega}_{[2]} \x \d_{[2]} \to \bbC$$
            induced naturally by the canonical pairing of differential $1$-forms and derivations on $A_{[2]}$, but also that this induced bilinear form coincides with the restriction of the bilinear form $(-, -)_{\hat{\g}_{[2]}}$ (as constructed in convention \ref{conv: orthogonal_complement_of_toroidal_centres}) to $\z_{[2]} \oplus \d_{[2]} \cong \bar{\Omega}_{[2]} \oplus \d_{[2]}$, since we have per its construction that the elements $D_{r, s}, D_v, D_t$ are dual with respect to $(-, -)_{\hat{\g}_{[2]}}$ to the elements $K_{r, s}, c_v, c_t$, respectively. 
        \end{remark}

        We can package the previous two remarks into the following result:
        \begin{proposition}[Pairing of cyclic $1$-forms and divergence-zero vector fields] \label{prop: pairing_cyclic_1_forms_and_div_zero_vector_fields} 
            Denote the usual non-degenerate and symmetric bilinear pairing of differential $1$-forms and derivations on $A_{[2]}$ by:
                $$\iota_{[2]}: \Omega_{[2]} \x \der_{\bbC}(A_{[2]}) \to \bbC$$
            Via the canonical quotient map of $\bbC$-vector spaces:
                $$\Omega_{[2]} \to \bar{\Omega}_{[2]}$$
            one obtains an induced non-degenerate and symmetric bilinear pairing:
                $$\bar{\iota}_{[2]}: \bar{\Omega}_{[2]} \x \d_{[2]} \to \bbC$$
            coninciding with the restriction of the bilinear form $(-, -)_{\hat{\g}_{[2]}}$ (as constructed in convention \ref{conv: orthogonal_complement_of_toroidal_centres}) to $\z_{[2]} \oplus \d_{[2]} \cong \bar{\Omega}_{[2]} \oplus \d_{[2]}$.
            
            In other words, there is a commutative diagram of $\bbC$-vector spaces and $\bbC$-linear maps between them as follows:
                $$
                    \begin{tikzcd}
                	{\Omega_{[2]} \tensor_{\bbC} \der_{\bbC}(A_{[2]})} \\
                	& \bbC \\
                	{\bar{\Omega}_{[2]} \tensor_{\bbC} \d_{[2]}}
                	\arrow[dashed, from=1-1, to=3-1]
                	\arrow["{\iota_{[2]}}", from=1-1, to=2-2]
                	\arrow["{(-, -)_{\hat{\g}_{[2]}}}"', from=3-1, to=2-2]
                    \end{tikzcd}
                $$
        \end{proposition}
        \begin{lemma}[Canonical element for the canonical pairing of $1$-forms and vector fields] \label{lemma: canonical_elements_for_the_canonical_pairing_of_1_forms_and_vector_fields}
            Let $k$ be a field of characteristic $0$. Denote by:
                $$\iota_{[n]}^+: \Omega_{[n]}^+ \x \der_k(A_{[n]}^-) \to k$$
            the restriction of $\iota_{[n]}$ to be between differential $1$-forms and derivations on $A_{[n]}^+ \subset A_{[n]}$ and $A_{[n]}^- \subset A_{[n]}$ respectively. The canonical element attached to $\iota_{[n]}^+$ shall then be given by:
                $$\sfr_{[n]}^+ := v_1' \1(v_1, v_1') ... v_{n - 1}' \1(v_{n - 1}, v_{n - 1}') \1^+(v_n, v_n') \sum_{1 \leq i \leq n} dv_i \tensor \del_{v_i}$$
        \end{lemma}
            \begin{proof}
                A basis for $\Omega_{[n]}^+$ is given by:
                    $$\{ v_1^{m_1} ... v_{n - 1}^{m_{n - 1}} v_n^{m_n} dv_i \}_{1 \leq i \leq n}$$
                and its dual (which is a basis for $\der_k(A_{[n]}^-)$) with respect to $\iota_{[n]}^+$ is given by:
                    $$\{ v_1^{-m_1} ... v_{n - 1}^{m_{n - 1}} v_n^{-m_n - 1} \del_{v_i} \}_{1 \leq i \leq n}$$
                We then have - more-or-less tautologically - that:
                    $$
                        \begin{aligned}
                            \sfr_{[n]}^+ & := \sum_{1 \leq i \leq n} \sum_{(m_1, ..., m_{n - 1}, m_n) \in \Z^{n - 1} \x \Z_{\geq 0}} v_1^{m_1} ... v_{n - 1}^{m_{n - 1}} v_n^{m_n} dv_i \tensor v_1^{-m_1} ... v_{n - 1}^{m_{n - 1}} v_n^{-m_n - 1} \del_{v_i}
                            \\
                            & = v_1' \1(v_1, v_1') ... v_{n - 1}' \1(v_{n - 1}, v_{n - 1}') \1^+(v_n, v_n') \sum_{1 \leq i \leq n} dv_i \tensor \del_{v_i}
                        \end{aligned}
                    $$
            \end{proof}
        The lemma above suggests to us that it is possible to rewrite the canonical elements $\sfr_{\z_{[2]}^+}, \sfr_{\d_{[2]}^+}$ in terms of formal Dirac distributions. Furthermore, it implies that the problem now reduces to computing the complements of the bases of $\bar{\Omega}_{[2]}^+$ (i.e. a basis for $d(A_{[2]}^+)$) and of $\d_{[2]}^+$ inside the bases of $\Omega_{[2]}^+$ and of $\der_{\bbC}(A_{[2]}^+)$ respectively; due to the duality of $\Omega_{[2]}^+$ and $\der_{\bbC}(A_{[2]})$ via the bilinear form $\iota_{[2]}$, these complemenets ought to be in bijection with one another. This is because, the canonical element arising from those complementary basis elements, say $\sfr_{[2]}^{+ \varnothing}$, satisfies:
            $$\sfr_{\z_{[2]}^+} = \sfr_{[2]}^+ - \sfr_{[2]}^{+ \varnothing}$$
        (and likewise for $\sfr_{\d_{[2]}^+}$ after a flip map). 
        \begin{remark}[A basis for $A_{[n]}$] \label{remark: basis_for_global_functions_on_split_tori}
            Let $k$ be a field of characteristic $0$ and fix some $n \geq 1$. Set:
                $$v_0 := 1$$
            so that:
                $$A_{[0]} \cong k$$

            The $k$-algebra:
                $$A_{[n]} := k[v_1^{\pm 1}, ..., v_n^{\pm 1}]$$
            has a natural $\Z$-grading in the last variable $v_n$ given by:
                $$A_{[n]} \cong \bigoplus_{m_n \in \Z} A_{[n - 1]} v_n^{m_n}$$
            Applying the universal K\"ahler differential map $d: A_{[n]} \to \Omega_{[n]}$ then yields:
                $$d( A_{[n]} ) \cong \bigoplus_{m_n \in \Z} d( A_{[n - 1]} v_n^{m_n} ) \cong \bigoplus_{p \in \Z} ( d( A_{[n - 1]} ) v_n^{m_n} \oplus A_{[n - 1]} v_n^{m_n - 1} dv_n )$$

            When $n = 1$, the equation above specialises to:
                $$d(A_{[1]}) \cong \bigoplus_{m \in \Z} ( d( A_{[0]} ) v^m \oplus A_{[0]} v^{m - 1} dv ) \cong \bigoplus_{m \in \Z} k v^{m - 1} dv$$
            When $n = 2$, we subsequently have that:
                $$d(A_{[2]}) \cong \bigoplus_{p \in \Z} ( d(A_{[1]}) t^p \oplus A_{[1]} t^{p - 1} dt ) \cong \bigoplus_{(m, p) \in \Z^2} ( k v^{m - 1} t^p dv \oplus k v^m t^{p - 1} dt )$$
            from which it is inferable that the set:
                $$\{v^{m - 1} t^p dv\}_{(m, p) \in \Z^2} \cup \{v^m t^{p - 1} dt\}_{(m, p) \in \Z^2}$$
            forms a basis for $d(A_{[2]})$; to obtain a basis for $d(A_{[2]}^{\pm})$, simply change the indexing set from $\Z^2$ to $\Z \x \Z_{\geq 0}$ and $\Z \x \Z_{< 0}$ respectively.

            The subset of the standard basis for $\der_k(A_{[2]})$ whose elements are dual with respect to the bilinear form $\iota_{[2]}$ to the elements of the basis for $d(A_{[2]})$ (i.e. a basis for $\der_k(A_{[2]})/\d_{[2]}$) as above is thus:
                $$\{v^{-m + 1} t^{-p - 1} \del_v\}_{(m, p) \in \Z^2} \cup \{v^{-m} t^{-p} \del_t\}_{(m, p) \in \Z^2}$$
            and likewise, to obtain the corresponding subsets of the standard basis for $\der_k(A_{[2]}^{\pm})$, simply change the indexing set from $\Z^2$ to $\Z \x \Z_{\geq 0}$ and $\Z \x \Z_{< 0}$ respectively.

            Given the above, the canonical element for the restricted pairing:
                $$\iota_{[2]}^+: d(A_{[2]}^+) \x \der_k( A_{[2]}^- )/\d_{[2]}^- \to k$$
            (which is still non-degenerate and symmetric) then takes the form:
                $$
                    \begin{aligned}
                        \sfr_{[2]}^{+ \varnothing} & = \sum_{(m, p) \in \Z \x \Z_{\geq 0}} \left( v_1^{m - 1} t_1^p dv_1 \tensor v_2^{-m + 1} t_2^{-p - 1} \del_{v_2} + v_1^m t_1^{p - 1} dt_1 \tensor v_2^{-m} t_2^{-p} \del_{t_2} \right)
                        \\
                        & = v_1^{-1} v_2^2 \1(v_1, v_2) \1^+(t_1, t_2) dv_1 \tensor \del_{v_2} + v_2 \1(v_1, v_2) t_1^{-1} t_2 \1^+(t_1, t_2) dt_1 \tensor \del_{t_2}
                    \end{aligned}
                $$
            By applying the flip map, one gets $\sfr_{[2]}^{- \varnothing}$, the canonical element for the restricted pairing:
                $$\iota_{[2]}^-: \der_k( A_{[2]}^- )/\d_{[2]}^- \x d(A_{[2]}^+) \to k$$
        \end{remark}
        We have therefore arrived at the following result:
        \begin{proposition}[Toroidal classical R-matrices in terms of formal distributions] \label{prop: toroidal_classical_R_matrices_in_terms_of_formal_distributions}
            The canonical elements $\sfr_{\z_{[2]}^+}$ and $\sfr_{\d_{[2]}^+}$ (cf. corollary \ref{coro: extended_toroidal_lie_bialgebras}) can be alternatively written in the following manner, using formal Dirac distributions (cf. convention \ref{conv: formal_dirac_distributions})\footnote{Note that the total degrees of $\sfr_{\z_{[2]}^+}$ and $\sfr_{\d_{[2]}^+}$ are both $0$ in $v$ and $-1$ in $t$ (ultimately because $\deg \1(w, z) = \deg \1^+(z, w) = -1$ by construction), agreeing with remark \ref{remark: total_degrees_of_classical_yangian_R_matrices}.}:
                $$\sfr_{\z_{[2]}^+} = \sfr_{[2]}^+ - \sfr_{[2]}^{+ \varnothing}$$
                $$\sfr_{\d_{[2]}^+} = \sfr_{[2]}^- - \sfr_{[2]}^{- \varnothing}$$
            with $\sfr_{[2]}^{\pm}$ and $\sfr_{[2]}^{\pm \varnothing}$ as in lemma \ref{lemma: canonical_elements_for_the_canonical_pairing_of_1_forms_and_vector_fields} and remark \ref{remark: basis_for_global_functions_on_split_tori}. 
        \end{proposition}
        
        \begin{remark}[Toroidal classical R-matrices as meromorphic functions] \label{remark: toroidal_classical_R_matrices_as_meromorphic_functions}
            The appearance of formal Dirac distributions in the expressions for $\sfr_{\z_{[2]}^+}$ and $\sfr_{\d_{[2]}^+}$ (as well as for $\sfr_{\g_{[2]}^+}$; cf. question \ref{question: multiloop_lie_bialgebras}), and hence for $\sfr_{\tilde{\g}_{[2]}^+} = \sfr_{\hat{\g}_{[2]}^+}$ suggests to us that when regarded complex-analytically, these are meromorphic $\bbC$-valued functions in the variables $v, t$. It is thus natural to inquire into where the singularities of these functions lie. 
        \end{remark}
        \begin{question}
            Classify the singularities of $\sfr_{\tilde{\g}_{[2]}^+} = \sfr_{\hat{\g}_{[2]}^+}$ and compare them to those of the universal quantum R-matrix of $\rmY_{\hbar}(\hat{\g}_{[1]})$. 
        \end{question}

    \subsection{\textit{Interlude}: A root space decomposition for Yangian extended toroidal Lie algebras}
        As is now standard practice in infinite-dimensional Lie theory, infinite-dimensional Lie algebra induced from finite-dimensional simple Lie algebras ought to carry a grading by some kind of induced \say{higher root lattice} (e.g. affine Kac-Moody algebras are graded by the affinisations of the root lattices of the underlying finite-dimensional simple Lie algebras; cf. \cite[Chapter 6]{kac_infinite_dimensional_lie_algebras}). There are many reasons as to why one might seek to endow Lie algebras with such gradings, but one rather important reason is that without a root grading of some sort - which in turn would give rise to some kind of triangular decomposition - one would have no hope of setting up a theory of highest-weight modules which, from practical experiences with cases such as $\g$ and $\hat{\g}_{[1]}$, we know to be an extremely powerful method for attacking the problem of classifying say, simple modules over Lie algebras. Therfore, it is natural to ask the question of whether or not our Yangian extended toroidal Lie algebra $\hat{\g}_{[2]}$ can also be endowed with such an induced grading, primarily because $\hat{\g}_{[2]}$ carries a non-degenerate invariant symmetric bilinear form.

        \newpage
        
        \begin{convention}
            If $\fraku$ is a symmetrisable Kac-Moody algebra and $V$ is a $\fraku$-module, then for each weight $\lambda \in \Pi(V)$, we shall be denoting the corresponding weight space by $V[\lambda]$.
        \end{convention}

        Let us firstly recall two equivalent natural gradings on the affine Kac-Moody algebra $\hat{\g}_{[1]}$. 
        \begin{remark}[$\hat{Q}$-grading on $\hat{\g}_{[1]}$]
            The $Q$-grading on $\g$ and the natural $\Z$-grading on $A_{[1]} := \bbC[v^{\pm 1}]$ induce, together, a $Q \x \Z$-grading on $\g_{[1]}$. Explicitly, for each $\lambda \in \Phi$, each $x \in \g[\alpha]$, and each $m \in \Z$, one has that:
                $$\deg x v^m = (\alpha, m)$$
            Following \cite[Chapter 6]{kac_infinite_dimensional_lie_algebras}, we know that there is an isomorphism of $\Z$-modules:
                $$\hat{Q} \xrightarrow[]{\cong} Q \x \Z$$
                $$\alpha + m \delta \mapsto (\alpha, m)$$
            (given for all $\alpha \in \Phi$ and all $m \in \Z$), with $\delta$ denoting the lowest positive imaginary root. As such, $\g_{[1]}$ can be equivalently viewed as being $\hat{Q}$-graded in the sense that for each $\alpha \in Q$, each $x \in \g[\alpha]$, and each $m \in \Z$, one has that:
                $$\deg x v^m = \alpha + m \delta$$
        \end{remark}
        \begin{proposition}[Induced $Q \x \Z$-grading on $\hat{\g}_{[2]}$] \label{prop: root_grading_on_extended_toroidal_lie_algebras}
            For what follows, we will need to make the choice\footnote{See proposition \ref{prop: lie_bracket_on_orthogonal_complement_of_toroidal_centre} for an elaboration.} that the restriction of $[-, -]_{\hat{\g}_{[2]}}$ to the vector subspace $\d_{[2]}$ is just the ordinary commutator of derivations.
        
            Define the following grading on $\tilde{\g}_{[2]}$\footnote{Note that we can not simply define a grading on $\g_{[2]}$ alone, since $[\g_{[2]}, \g_{[2]}]_{\tilde{\g}_{[2]}} \not \subset \g_{[2]}$.}, naturally induced by the natural $Q \x \Z$-grading on $\g_{[1]}$. Firstly, let us declare that:
                $$\forall \alpha \in \Phi: \forall x \in \g[\alpha]: \deg x := (\alpha, 0)$$
                $$\deg v := (0, 1)$$
                $$\deg t := (0, 0)$$
            If we are to extend the $Q \x \Z$-grading on $\tilde{\g}_{[2]}$ as above to $\hat{\g}_{[2]}$ then the Lie bracket $[-, -]_{\hat{\g}_{[2]}}$ ought to be $Q \x \Z$-graded in a compatible manner. Given the adjoint actions of the derivations $D_{r, s}, D_v, D_t$ on the monomials $x v^m t^p \in \g_{[2]}$ (in particular, how said actions affect the $Q \x \Z$-degrees of said monomials; cf. remarks \ref{remark: derivation_action_on_multiloop_algebras} and \ref{remark: dual_of_toroidal_centres_contains_derivations}), let us declare that:
                $$\forall (r, s) \in \Z^2: \deg D_{r, s} := (0, -r)$$
                $$\deg D_v = \deg D_t := (0, 0)$$
            We would also like the bilinear form $(-, -)_{\hat{\g}_{[2]}}$ to be of total degree $(0, 0)$, which forces:
                $$\forall (r, s) \in \Z^2: \deg K_{r, s} := (0, r)$$
                $$\deg c_v = \deg c_t := (0, 0)$$
        \end{proposition}
            \begin{proof}
                Let us check whether the constructed $Q \x \Z$-grading on $\hat{\g}_{[2]}$ is well-defined.

                Firstly, let us check that the grading is well-define on $\tilde{\g}_{[2]} := \g_{[2]} \oplus \z_{[2]}$. To this end, pick $x, y \in \g$ and that $x \in \g[\alpha], y \in \g[\beta]$ for some $\alpha, \beta \in \Phi \cup \{0\}$; also, choose some arbitrary $(m, p), (n, q) \in \Z^2$. Next, consider:
                    $$
                        \begin{aligned}
                            [x v^m t^p, y v^n t^q]_{\tilde{\g}_{[2]}} & = [x, y]_{\g} v^{m + n} t^{p + q} + (x, y)_{\g} v^m t^p \bar{d}(v^n t^p)
                            \\
                            & = [x, y]_{\g} v^{m + n} t^{p + q} + (x, y)_{\g} \delta_{(m, p) + (n, q), (0, 0)} ( n c_v + q c_t ) + (np - mq) K_{m + n, p + q}
                        \end{aligned}
                    $$
                Now, note that if either:
                    $$\alpha + \beta = 0, \alpha \not = 0$$
                or:
                    $$\alpha = \beta = 0$$
                (i.e. $x, y \in \h$) then:
                    $$[x, y] \in \h$$
                and hence:
                    $$\deg [x, y]_{\g} v^{m + n} t^{p + q} = \deg K_{m + n, p + q} = (0, m + n)$$
                On the other hand, if:
                    $$\alpha + \beta \not = 0$$
                then:
                    $$[x, y] \in \n^- \oplus \n^+$$
                which means in particular that at leeast either $x$ or $y$ is nilpotent under the vector representation of $\g$, and hence:
                    $$(x, y)_{\g} = 0$$
                as $(-, -)_{\g}$ is some non-zero multiple of the trace form, and traces of nilpotent matrices are equally $0$. Hence, in this case, we have that:
                    $$\deg [x v^m t^p, y v^n t^q]_{\tilde{\g}_{[2]}} = \deg [x, y]_{\g} v^{m + n} t^{p + q} = (\alpha + \beta, m + n)$$
                Both cases together show that the constructed $Q \x \Z$-grading on $\tilde{\g}_{[2]}$ is well-defined. 
                
                Secondly, note that from how commutators of elements of $\d_{[2]} := \bigoplus_{(r, s) \in \Z^2} \bbC D_{r, s} \oplus \bbC D_v \oplus \bbC D_t$ are given (cf. lemma \ref{lemma: explicit_commutators_between_basis_elements_of_toroidal_central_orthogonal_complement}), one sees that:
                    $$\deg [D_v, D_t] = (0, 0) = \deg D_v + \deg D_t$$
                    $$\deg [D_v, D_{r, s}] = (0, -r) = \deg D_v + \deg D_{r, s}$$
                    $$\deg [D_t, D_{r, s}] = (0, -r) = \deg D_t + \deg D_{r, s}$$
                    $$\deg [D_{a, b}, D_{r, s}] = \deg D_{a + r, b + s + 1} = (0, -(a + r)) = \deg D_{a, b} + \deg D_{r, s}$$
                Thus, the constructed grading is well-defined on $\d_{[2]}$. Recall also from proposition \ref{prop: lie_bracket_on_orthogonal_complement_of_toroidal_centre} that:
                    $$[\d_{[2]}, \d_{[2]}]_{\hat{\g}_{[2]}} \subseteq \z_{[2]} \oplus \d_{[2]}$$
                with the $\d_{[2]}$-summand being the usual commutator of derivations $[-, -]$ inherited from $\der_{\bbC}(A_{[2]})$, while the $\z_{[2]}$-summand is undetermined, but can be viewed as twist of $[-, -]$ by a cocycle $\sigma \in H^2_{\Lie}(\d_{[2]}, \z_{[2]})$ (cf. theorem \ref{theorem: non_uniqueness_of_yangian_extended_lie_algebras}). For this reason, we can and must choose the restriction of $[-, -]_{\hat{\g}_{[2]}}$ down to $\d_{[2]}$ to be the usual commutator $[-, -]$ for the construction of our $Q \x \Z$-grading. 
            \end{proof}
        
        \begin{remark}
            For what follows, let us recall from \cite[Chapter 7]{kac_infinite_dimensional_lie_algebras} that the root space decomposition of the untwisted affine Kac-Moody algebra $\hat{\g}_{[1]}$ takes the form:
                $$\hat{\g}_{[1]} \cong \hat{\h}_{[1]} \oplus \bigoplus_{\beta \in \Re(\hat{\Phi})} \hat{\g}_{[1]}[\beta] \oplus \bigoplus_{\beta \in \Im(\hat{\Phi})} \hat{\g}_{[1]}[\beta]$$
            in which the untwisted affine root system $\hat{\Phi}$ decomposes into a disjoint union of the subsets of real and imaginary roots:
                $$\hat{\Phi} \cong \Re(\hat{\Phi}) \cup \Im(\hat{\Phi})$$
            where:
                $$\Re(\hat{\Phi}) \cong \Phi + \Z\delta \cong \Phi \x \Z$$
                $$\Im(\hat{\Phi}) \cong (\Z \setminus \{0\})\delta$$
            and the corresponding root spaces are given by:
                $$\forall \alpha + m\delta \in \Re(\hat{\Phi}): \hat{\g}_{[1]}[\alpha + m\delta] \cong \g[\alpha] v^m$$
                $$\forall r\delta \in \Im(\hat{\Phi}): \hat{\g}_{[1]}[r\delta] \cong \h v^r$$
            The Cartan subalgebra $\hat{\h}_{[1]}$ is as in convention \ref{conv: a_fixed_untwisted_affine_kac_moody_algebra}.
        \end{remark}    
        The following is a corollary to proposition \ref{prop: root_grading_on_extended_toroidal_lie_algebras}. One can see it to be true simply by looking at the degrees of elements of $\hat{\g}_{[2]}$. 
        \begin{theorem}[Root space decomposition for extended toroidal Lie algebras] \label{theorem: root_space_decomposition_for_extended_toroidal_lie_algebras}
            The weight spaces of the adjoint action of $\hat{\g}_{[1]}$ on $\hat{\g}_{[2]}$ can be given explicitly in terms of the basis elements of the latter in the following manner:
                $$\forall (\alpha, m) \in \Phi \x \Z: \hat{\g}_{[2]}[\alpha + m\delta] \cong \hat{\g}_{[1]}[\alpha + m\delta] \tensor_{\bbC} \bbC[t^{\pm 1}]$$
                $$
                    \forall r \in \Z \setminus \{0\}: \hat{\g}_{[2]}[r \delta] \cong \hat{\g}_{[1]}[r \delta] \tensor_{\bbC} \bbC[t^{\pm 1}] \oplus \bigoplus_{s \in \Z} \bbC (K_{r, s} \oplus D_{-r, s})
                $$
                $$\hat{\g}_{[2]}[0] \cong \h \oplus (\bbC c_v \oplus \bbC c_t) \oplus (\bbC D_v \oplus \bbC D_t)$$
            Furthermore, $\hat{\g}_{[2]}$ is a weight module of $\hat{\g}_{[1]}$, i.e.:
                $$\hat{\g}_{[2]} \cong \bigoplus_{\beta \in \hat{\Phi} \cup \{0\}} \hat{\g}_{[2]}[\beta]$$
        \end{theorem}
        \begin{corollary}
            Recall the Manin triple:
                $$(\hat{\g}_{[2]}, \hat{\g}_{[2]}^+, \hat{\g}_{[2]}^-)$$
            from theorem \ref{theorem: extended_toroidal_manin_triples}. Precisely because this is a Manin triple, one has the following root space decompositions for the Lie subalgebras $\hat{\g}_{[2]}^{\pm}$, induced by the one on $\hat{\g}_{[2]}$:
                $$
                    \forall (\alpha, m) \in \Phi \x \Z:
                    \begin{cases}
                        \hat{\g}_{[2]}^+[\alpha + m\delta] \cong \hat{\g}_{[1]}[\alpha + m\delta] \tensor_{\bbC} \bbC[t]
                        \\
                        \hat{\g}_{[2]}^-[\alpha + m\delta] \cong \hat{\g}_{[1]}[\alpha + m\delta] \tensor_{\bbC} t^{-1}\bbC[t^{-1}]
                    \end{cases}
                $$
                $$
                    \forall r \in \Z \setminus \{0\}:
                    \begin{cases}
                        \text{$
                        \begin{cases}
                            \hat{\g}_{[2]}^+[r \delta] \cong \hat{\g}_{[1]}[r \delta] \oplus \bigoplus_{s \in \Z_{> 0}} \bbC K_{r, s}
                            \\
                            \hat{\g}_{[2]}^-[r \delta] \cong \hat{\g}_{[1]}[r \delta] \oplus \bigoplus_{s \in \Z_{\leq 0}} \bbC K_{r, s}
                        \end{cases}
                        $ if $r > 0$}
                        \\
                        \text{$
                        \begin{cases}
                            \hat{\g}_{[2]}^+[r \delta] \cong \hat{\g}_{[1]}[r \delta] \oplus \bigoplus_{s \in \Z_{\leq 0}} \bbC D_{r, s}
                            \\
                            \hat{\g}_{[2]}^-[r \delta] \cong \hat{\g}_{[1]}[r \delta] \oplus \bigoplus_{s \in \Z_{> 0}} \bbC D_{r, s}
                        \end{cases}
                        $ if $r < 0$}
                    \end{cases}
                $$
                $$\hat{\g}_{[2]}^+[0] \cong \h \oplus \bbC c_v \oplus \bbC D_t, \hat{\g}_{[2]}^-[0] \cong \h \oplus \bbC c_t \bbC D_v$$
            Of course, one has also from these constructions that:
                $$\hat{\g}_{[2]}^{\pm} \cong \bigoplus_{\beta \in \hat{\Phi} \cup \{0\}} \hat{\g}_{[2]}^{\pm}[\beta]$$
        \end{corollary}
        
        \begin{remark}[Toroidal root systems] \label{remark: toroidal_root_systems}
            The root space decomposition of $\hat{\g}_{[2]}$ as in theorem \ref{theorem: root_space_decomposition_for_extended_toroidal_lie_algebras} suggests to us that it is possible to construct a \say{toroidal root system}:
                $$\hat{\Phi}_{[2]}$$
            for $\hat{\g}_{[2]}$ with the following features.
            \begin{enumerate}
                \item \textbf{(Anisotropic roots):} Firstly, there is a set of anisotropic roots - which are in bijection with the real roots of $\hat{\g}_{[1]}$ or equivalently, the roots of $\g$. Interestingly, the corresponding root spaces:
                    $$\hat{\g}_{[2]}[\alpha + m\delta], (\alpha, m) \in \Phi \x \Z$$
                are free and of rank $1$ over $\bbC[t^{\pm 1}]$, in good analogy with how real roots of an affine Kac-Moody algebras are equally of multiplicity $1$.

                Such roots are \textbf{anisotropic} in the following sense. If we fix:
                    $$(\alpha, m, p), (\beta, n, q) \in \Phi \x \Z^2$$
                along with root vectors:
                    $$x_{\alpha} \in \g[\alpha], x_{\beta} \in \g[\beta]$$
                then:
                    $$( x_{\alpha} v^m t^p, x_{\beta} v^n t^q )_{\hat{\g}_{[2]}} = \delta_{(\alpha, m, p) + (\beta, n, q), (0, 0, -1)}$$
                This suggest to us that for each positive real root:
                    $$\alpha + m\delta \in \hat{\Phi}^+ \cong \Phi^+ \x \Z_{\geq 0}$$
                one has the following non-trivial pairing of subspaces:
                    $$\left( \hat{\g}_{[2]}^-[\pm (\alpha + m\delta)], \hat{\g}_{[2]}^+[\pm (\alpha + m\delta)] \right)_{\hat{\g}_{[2]}} \not = 0$$
                thus justifying our use of the term \say{anisotropic}.
                \item \textbf{(Isotropic roots):} Observe that the \say{Cartan subalgebra}:
                    $$\hat{\g}_{[2]}[0]$$
                is finite-dimensional, namely of dimension $\dim_{\bbC} \h + 2 + 2$, with each summand of $2$ corresponding to one of the direct summands $\bbC c_v \oplus \bbC D_v$ and $\bbC c_t \oplus \bbC D_t$ of $\hat{\g}_{[2]}[0]$, similar to how:
                    $$\dim_{\bbC} \hat{\h}_{[1]} = \dim_{\bbC} \h + 2$$
                in the affine Kac-Moody case, where the summand of $2$ corressponds to the direct summand of the $1$-dimensional centre and the subspace spanned by the canonical degree derivation. From this, we infer that $\hat{\g}_{[2]}$ ought to admit two distinct weights with respect to the adjoint action of $\hat{\g}_{[1]}$ that can be reasonably called \textbf{isotropic roots}. In particular, these anisotropic roots are to be the image of the derivations:
                    $$D_v, D_t \in \hat{\g}_{[2]}[0]$$
                under the dualising map $\hat{\g}_{[2]}[0] \xrightarrow[]{\cong} \hat{\g}_{[2]}[0]^*$. Let us denote these roots, respectively, by:
                    $$\delta_v, \delta_t \in \hat{\g}_{[2]}[0]^*$$
                and note that because:
                    $$(\d_{[2]}, \d_{[2]})_{\hat{\g}_{[2]}} = 0$$
                per the construction of the bilinear form $(-, -)_{\hat{\g}_{[2]}}$ (cf. convention \ref{conv: orthogonal_complement_of_toroidal_centres}), we have that:
                    $$(\delta_v, \delta_v)_{\hat{\g}_{[2]}} = (\delta_t, \delta_t)_{\hat{\g}_{[2]}} = 0$$
                thus justifying our use of the term \say{isotropic}.

                Note also that, once again because $(\d_{[2]}, \d_{[2]})_{\hat{\g}_{[2]}} = 0$, we also have that:
                    $$(\delta_v, \delta_t)_{\hat{\g}_{[2]}} = 0$$
                i.e. the two isotropic roots of $\hat{\g}_{[2]}$ are perpendicular to one another. 
            \end{enumerate}

            In conclusion, we have that:
                $$\hat{\Phi}_{[2]} \cong \Phi \cup (\Z \delta_v \setminus \{0\}) \cup (\Z \delta_t \setminus \{0\}) \cong \hat{\Phi}_{[1]} \cup \Z \delta_t \setminus \{0\}$$
            where $\hat{\Phi}_{[1]} := \hat{\Phi}$ is the root system of the affine Kac-Moody algebra $\hat{\g}_{[1]}$. This allows us to rewrite the root space decomposition of $\hat{\g}_{[2]}$, as discussed in theorem \ref{theorem: root_space_decomposition_for_extended_toroidal_lie_algebras}, in the following more meaningful manner:
                $$\hat{\g}_{[2]} \cong \bigoplus_{\beta \in \hat{\Phi}_{[2]} \cup \{0\}} \hat{\g}_{[2]}[\beta]$$

            Note also that the set $\hat{\Phi}_{[2]}$ satisfies also some (but not all) the properties expected of an abstract root system. 

            \todo[inline]{What are the simple roots, positive/negative roots ?}
        \end{remark}
        The following result is nothing but a formal consequence of the discussion above.
        \begin{proposition}[Triangular decomposition of extended toroidal Lie algebras] \label{prop: triangular_decomposition_of_extended_toroidal_lie_algebras}
            Set:
                $$\hat{\g}_{[2]}[\hat{\Phi}_{[2]}^{\pm}] := \bigoplus_{\beta \in \hat{\Phi}_{[2]}^{\pm}} \hat{\g}_{[2]}[\beta]$$
        
            The root space decomposition of the Lie algebra $\hat{\g}_{[2]}$ (respectively, of $\hat{\g}_{[2]}^{\pm}$) induces a triangular decomposition thereof as follows:
                $$\hat{\g}_{[2]} \cong \hat{\g}_{[2]}[\hat{\Phi}_{[2]}^-] \oplus \hat{\g}_{[2]}[0] \oplus \hat{\g}_{[2]}[\hat{\Phi}_{[2]}^+]$$
        \end{proposition}
        Remark \ref{remark: toroidal_root_systems} also prompts the following result.
        \begin{lemma}[Chevalley-Serre presentation for extended toroidal Lie algebras] \label{lemma: chevalley_serre_presentation_for_extended_toroidal_lie_algebras}
            The extended toroidal Lie algebra $\hat{\g}_{[2]}$ is isomorphic to the Lie algebra $\hat{\t}$ generated by the set:
                $$\{ H_{i, r}, E_{i, r}^{\pm} \}_{(i, r) \in \hat{\Gamma}_0 \x \Z} \cup \{K, D\}$$
            whose elements are subjected to the following relations, given for all $(i, r), (j, s) \in \hat{\Gamma}_0 \x \Z$:
                $$[ H_{i, r}, H_{j, s} ] = 0$$
                $$[ H_{i, r}, E_{j, s}^{\pm} ] = \pm (\alpha_j, \check{\alpha}_i) E_{j, r + s}^{\pm}$$
                $$[ E_{i, r}^+, E_{j, s}^- ] = \delta_{ij} H_{i, r + s}$$
                $$[ E_{i, r + 1}^{\pm}, E_{j, s}^{\pm} ] - [ E_{i, r}^{\pm}, E_{j, s + 1}^{\pm} ] = 0$$
                $$[K, \hat{\t}] = 0$$
                $$[D, H_{i, r}] = -r H_{i, r}, [D, E_{i, r}^{\pm}] = -r E_{i, r}^{\pm}$$
            The isomorphism $\hat{\t} \xrightarrow[]{\cong} \hat{\g}_{[2]}$ in question is given as follows, for all $(i, r) \in \hat{\Gamma}_0 \x \Z$:
                $$\forall (i, r) \in \Gamma_0 \x \Z: E_{i, r}^{\pm} \mapsto e_i^{\pm} t^r, H_{i, r} \mapsto h_i t^r$$
                $$\forall (i, r) \in \{\theta\} \x \Z: E_{\theta, r}^{\pm} \mapsto e_{\theta}^{\mp} v^{\pm 1} t^r, H_{\theta, r} \mapsto h_{\theta} t^r + c_v t^r$$
                $$K \mapsto c_t$$
                $$D \mapsto D_t$$
        \end{lemma}
            \begin{proof}
                Clear from a combination of lemma \ref{lemma: chevalley_serre_presentation_for_central_extensions_of_multiloop_algebras} and remark \ref{remark: toroidal_root_systems}.
            \end{proof}
        \begin{corollary} \label{coro: chevalley_serre_presentation_for_extended_toroidal_lie_algebras}
            The Lie algebras $\hat{\g}_{[2]}^{\pm}$ is isomorphic to the Lie algebras $\hat{\t}^{\pm}$ generated, respectively, by the sets:
                $$\{ E_{i, r}^{\pm}, H_{i, r} \}_{(i, r) \in \Gamma_0 \x \Z_{\geq 0}} \cup \{D\}$$
                $$\{ E_{i, r}^{\pm}, H_{i, r} \}_{(i, r) \in \Gamma_0 \x \Z_{< 0}} \cup \{K\}$$
            whose elements are subjected to the same relations as in lemma \ref{lemma: chevalley_serre_presentation_for_extended_toroidal_lie_algebras}. The isomorphisms $\hat{\t}^{\pm} \xrightarrow[]{\cong} \hat{\g}_{[2]}^{\pm}$ in question are just codomain restrictions of the isomorphism $\hat{\t} \xrightarrow[]{\cong} \hat{\g}_{[2]}$ from \textit{loc. cit.}
        \end{corollary}
        \begin{remark}[Comparison to the Chevalley-Serre presentation for affine Yangians]
            
        \end{remark}
            
        \begin{convention}
            To streamline notations, let us from now on write:
                $$\hat{\h}_{[2]} := \hat{\g}_{[2]}[0]$$
            
            Also, let us use the subscript \say{$[1]$} to denote objects related to the affine Kac-Moody algebra $\hat{\g}_{[1]}$ (e.g. its root system is to be $\hat{\Phi}_{[1]}$, its root lattice is to be $\hat{Q}_{[1]}$, etc.). The unique lowest positive imaginary root of $\hat{\g}_{[1]}$ will be denoted by $\delta_{[1]}$ from now on (note that $\delta_{[1]} \not = \delta_v$).
        \end{convention}

        \newpage

    \subsection{Classical limits of universal quantum R-matrices of affine Yangians}
        In preparation for comparing the classical R-matrix $\sfr_{ \hat{\g}_{[2]}^+ }$ and the quantum R-matrix of $\rmY_{\hbar}(\hat{\g}_{[1]})$, let us firstly study a Gaussian/LU decomposition of $\sfr_{ \hat{\g}_{[2]}^+ }$:
            $$\sfr_{ \hat{\g}_{[2]}^+ } := \sfr_{ \hat{\g}_{[2]}^+ }^- + \sfr_{ \hat{\g}_{[2]}^+ }^0 + \sfr_{ \hat{\g}_{[2]}^+ }^+$$
        This is in fact one of the reasons for studying root space decompositions for $\hat{\g}_{[2]}$ and for $\hat{\g}_{[2]}^{\pm}$, as it affords us a description of the summands:
            $$\sfr_{ \hat{\g}_{[2]}^+ }^{\pm}$$
        as canonical tensors of the pairings:
            $$\bigoplus_{(\alpha, m) \in \Phi^+ \x \Z} \hat{\g}_{[2]}^{\mp}[-\alpha + m\delta] \hattensor_{\bbC} \hat{\g}_{[2]}^{\pm}[\alpha + m\delta]$$
        \begin{convention}
            Suppose that $\fraku$ is a symmetrisable Kac-Moody algebra whose Cartan matrix is indecomposable. Fix for it a Cartan subalgebra $\fraku^0$ and denote the corresponding triangular decomposition by:
                $$\fraku := \fraku^- \oplus \fraku^0 \oplus \fraku^+$$
        
            Denote by:
                $$\sfr_{\fraku}^{\pm}$$
            the Casimir tensor associated to the non-degenerate Kac-Moody pairing on $\fraku^{\mp} \hattensor_{\bbC} \fraku^{\pm}$ (where $\hattensor_{\bbC}$ denotes an appropriate completion with respect to the root grading on $\fraku^{\pm}$), and by:
                $$\sfr_{\fraku}^0$$
            the Casimir tensor associated to the non-degenerate Kac-Moody pairing on $\fraku^0 \tensor_{\bbC} (\fraku^0)^*$.

            Note that the classical R-matrix of the usual Lie bialgebra structure on $\fraku$ is nothing but:
                $$\sfr_{\fraku} := \sfr_{\fraku}^- + \sfr_{\fraku}^0 + \sfr_{\fraku}^+$$
        \end{convention}
        \begin{theorem}[A Gaussian decomposition for $\sfr_{ \hat{\g}_{[2]}^+ }$]
            The classical R-matrix $\sfr_{ \hat{\g}_{[2]}^+ }$ from corollary \ref{coro: extended_toroidal_lie_bialgebras} admits an additive decomposition:
                $$\sfr_{ \hat{\g}_{[2]}^+ } := \sfr_{ \hat{\g}_{[2]}^+ }^- + \sfr_{ \hat{\g}_{[2]}^+ }^0 + \sfr_{ \hat{\g}_{[2]}^+ }^+$$
            corresponding to the triangular decomposition of $\hat{\g}_{[2]}^+$ (cf. proposition \ref{prop: triangular_decomposition_of_extended_toroidal_lie_algebras}) in which:
                $$\sfr_{ \hat{\g}_{[2]}^+ }^{\pm} := \sfr_{\hat{\g}_{[1]}}^{\pm} v_2 \1(v_1, v_2) \1^+(t_1, t_2)$$
                $$\sfr_{ \hat{\g}_{[2]}^+ }^0 := \sfr_{\hat{\g}_{[1]}}^0 + D_t \tensor c_t$$
            In other words, $\sfr_{\hat{\g}_{[1]}}^{\pm}$ are - respectively, the Casimir tensor associated to the non-degenerate pairing\footnote{See remark \ref{remark: toroidal_root_systems} for more details.} on $\hat{\g}_{[2]}^{\pm}[\hat{\Phi}_{[1]}^{\pm}] \hattensor_{\bbC} \hat{\g}_{[2]}^{\mp}[\hat{\Phi}_{[1]}^{\mp}]$ via $(-, -)_{\hat{\g}_{[2]}}$ and likewise, $\sfr_{ \hat{\g}_{[2]}^+ }^0$ is the Casimir tensor associated to the non-degenerate pairing on $\hat{\h}_{[2]}^+ \tensor_{\bbC} \hat{\h}_{[2]}^-$ via $(-, -)_{\hat{\g}_{[2]}}$. 
        \end{theorem}
        \begin{corollary}
            The classical R-matrix $\sfr_{ \hat{\g}_{[2]}^+ }$ from corollary \ref{coro: extended_toroidal_lie_bialgebras} can be rewritten as:
                $$\sfr_{ \hat{\g}_{[2]}^+ } = \sfr_{\hat{\g}_{[1]}} v_2 \1(v_1, v_2) \1^+(t_1, t_2) + D_t \tensor c_t$$
        \end{corollary}