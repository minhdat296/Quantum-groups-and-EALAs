\section{R-matrices of toroidal Lie bialgebras and of affine Yangians}
    \subsection{Classical R-matrices of toroidal Lie bialgebras}
        \begin{convention}
            We assume familiarity with the fact that, for any homomorphism of commutative rings $R \to S$, one has that:
                $$\Hom_S(\Omega_{S/R}^1, N) \cong \Der_R(S, N)$$
            for all $S$-modules $N$ (cf. \cite[\href{https://stacks.math.columbia.edu/tag/00RO}{Tag 00RO}]{stacks}).
        \end{convention}
        \begin{remark}[Pairing of $1$-forms and vector fields] \label{remark: pairing_1_forms_and_vector_fields} 
            Suppose for a moment that $A$ is a commutative algebra over a field $k$ generated by some set:
                $$\{v_i\}_{i \in I}$$
            and suppose furthermore that $A$ is equipped with a non-degenerate and symmetric $k$-bilinear form:
                $$(-, -)_A: A \x A \to k$$
            Next, consider the following natural pairing between $\Omega^1_{A/k}$ and $\der_k(A)$, given as the interior product/contraction of differential forms by vector fields, i.e. in the following manner:
                $$\iota_A(f dv_i, g \del_{v_j}) := (f, g)_A \delta_{i, j}$$
            for all $f, g \in A$. Clearly, the pairing:
                $$\iota_A: \Omega^1_{A/k} \x \der_k(A) \to k$$
            is a non-degenerate and symmetric $k$-bilinear form.

            Now, in the particular case of $A := A_{[n]}$ (for some $n \geq 1$), note firstly that the algebra $A_{[n]}$ is naturally equipped with the non-degenerate and symmetric $k$-bilinear form:
                $$(v_1^{m_1} ... v_n^{m_n}, v_1^{m_1'} ... v_n^{m_n'})_{A_{[n]}} := \delta_{(m_1 + m_1', ..., m_{n - 1} + m_{n - 1}', m_n + m_n'), (0, ..., 0, -1)}$$
            (cf. question \ref{question: multiloop_lie_bialgebras}) and with respect to this, the pairing:
                $$\iota_{A_{[n]}}: \Omega_{[n]} \x \der_k(A_{[n]}) \to k$$
            is then given by:
                $$\iota_{A_{[n]}}(f dv_i, g \del_{v_j}) := (f, g)_{A_{[n]}} \delta_{i, j}$$
            for all $f, g \in A_{[n]}$ and all $1 \leq i, j \leq n$. 
        \end{remark}

        \begin{remark}[Pairing of cyclic $1$-forms and divergence-zero vector fields] \label{remark: pairing_cyclic_1_forms_and_div_zero_vector_fields} 
            Again, suppose for a moment that $A$ is an arbitrary commutative algebra over a field $k$ of characteristic $0$ and that $A$ is generated by some set:
                $$\{v_i\}_{i \in I}$$
            and that $A$ carries a non-degenerate and symmetric $k$-bilinear form $(-, -)_A$. Via the canonical quotient map of $k$-vector spaces:
                $$\Omega^1_{A/k} \to \bar{\Omega}^1_{A/k}$$
            one obtains an induced bilinear form that we will denote by:
                $$\bar{\iota}_A: \bar{\Omega}^1_{A/k} \x \der_k(A) \to k$$

            When $A = A_{[n]}$, recall from remark \ref{remark: centres_of_dual_toroidal_lie_algebras} that there is a canonical basis for $\bar{\Omega}^1_{A/k}$ consisting of the elements:
                $$m_i^{-1} v_1^{m_1} ... v_i^{m_i - 1} ... v_n^{m_n} \bar{d}(v_i)$$
            wherein $1 \leq i \leq n$. The bilinear form $\bar{\iota}_{[n]} := \bar{\iota}_{A_{[n]}}$ is then given by:
                $$\bar{\iota}_{[n]}( m_i^{-1} v_1^{m_1} ... v_i^{m_i - 1} ... v_n^{m_n} \bar{d}(v_i), v_1^{m_1'} ... v_n^{m_n'} \del_{v_j} ) = m_i^{-1} \delta_{(m_1 + m_1', ..., m_i + m_i' - 1, ..., m_n + m_n'), (0, ..., 0, -1)} \delta_{i, j}$$
            for all $1 \leq i, j \leq n$. While it is true that $\bar{\iota}_{[n]}$ is non-degenerate, we caution that - based on the calculation above - $\bar{\iota}_{[n]}$ is not symmetric due to the appearance of the $m_i^{-1}$ factor: switching the two inputs would yield a $(-m'_i + 1)^{-1}$ factor instead. The upshot is that, once we restrict the second input from $\der_k(A_{[n]})$ to the \say{Yangian divergence-zero} subspace $\d_{[n]}$, the bilinear form $\bar{\iota}_{[n]}$ ought to get symmetrised, and symmetry is needed for us to be able to write down a canonical element corresponding to the resulting symmetric pairing.

            Let us now focus on the case where $k = \bbC$ and $n = 2$, partly because this is the only case wherein we have a concrete basis for $\d_{[2]}$ to work with, but also because ultimately, this is the only case of particular interest to us. Recall from remark \ref{remark: dual_of_toroidal_centres_contains_derivations} and from the construction of $\d_{[2]}$ (cf. convention \ref{conv: orthogonal_complement_of_toroidal_centres}) that:
                $$\d_{[2]} := (\bigoplus_{(r, s) \in \Z^2} \bbC D_{r, s}) \oplus \bbC D_v \oplus \bbC D_t$$
            wherein:
                $$\forall (r, s) \in \Z^2: D_{r, s} = s v^{-r + 1} t^{-s - 1} \del_v - r v^{-r} t^{-s} \del_t$$
                $$D_v = -v t^{-1} \del_v, D_t = -\del_t$$
            Let us also recall from remark \ref{remark: centres_of_dual_toroidal_lie_algebras} that the canonical basis elements of $\bar{\Omega}_{[2]}$ can be alternatively written as:
                $$
                    K_{r, s} :=
                    \begin{cases}
                        \text{$\frac1s v^{r - 1} t^s \bar{d}(v)$ if $(r, s) \in \Z \x (\Z \setminus \{0\})$}
                        \\
                        \text{$-\frac1r v^r t^{-1} \bar{d}(t)$ if $(r, s) \in (\Z \setminus \{0\}) \x \{0\}$}
                        \\
                        \text{$0$ if $(r, s) = (0, 0)$}
                    \end{cases}
                $$
                $$c_v := v^{-1} \bar{d}(v), c_t := t^{-1} \bar{d}(t)$$
            It is clear that:
                $$
                    \bar{\iota}_{[2]}( K_{r, s}, D ) =
                    \begin{cases}
                        \text{$1$ if $D = D_{r, s}$}
                        \\
                        \text{$0$ if $D \not \in \bbC D_{r, s}$}
                    \end{cases}
                $$
                $$
                    \bar{\iota}_{[2]}( c_v, D ) =
                    \begin{cases}
                        \text{$1$ if $D = D_v$}
                        \\
                        \text{$0$ if $D \not \in \bbC D_v$}
                    \end{cases}
                $$
                $$
                    \bar{\iota}_{[2]}( c_t, D ) =
                    \begin{cases}
                        \text{$1$ if $D = D_t$}
                        \\
                        \text{$0$ if $D \not \in \bbC D_t$}
                    \end{cases}
                $$
            so we have managed to show that not only is there a non-degenerate and symmetric $\bbC$-bilinear form:
                $$\bar{\iota}_{[2]}: \bar{\Omega}_{[2]} \x \d_{[2]} \to \bbC$$
            induced naturally by the canonical pairing of differential $1$-forms and derivations on $A_{[2]}$, but also that this induced bilinear form coincides with the restriction of the bilinear form $(-, -)_{\hat{\g}_{[2]}}$ (as constructed in convention \ref{conv: orthogonal_complement_of_toroidal_centres}) to $\z_{[2]} \oplus \d_{[2]} \cong \bar{\Omega}_{[2]} \oplus \d_{[2]}$, since we have per its construction that the elements $D_{r, s}, D_v, D_t$ are dual with respect to $(-, -)_{\hat{\g}_{[2]}}$ to the elements $K_{r, s}, c_v, c_t$, respectively. 
        \end{remark}

        We can package the previous two remarks into the following result:
        \begin{proposition}[Pairing of cyclic $1$-forms and divergence-zero vector fields] \label{prop: pairing_cyclic_1_forms_and_div_zero_vector_fields} 
            Denote the usual non-degenerate and symmetric bilinear pairing of differential $1$-forms and derivations on $A_{[2]}$ by:
                $$\iota_{[2]}: \Omega_{[2]} \x \der_{\bbC}(A_{[2]}) \to \bbC$$
            Via the canonical quotient map of $\bbC$-vector spaces:
                $$\Omega_{[2]} \to \bar{\Omega}_{[2]}$$
            one obtains an induced non-degenerate and symmetric bilinear pairing:
                $$\bar{\iota}_{[2]}: \bar{\Omega}_{[2]} \x \d_{[2]} \to \bbC$$
            coninciding with the restriction of the bilinear form $(-, -)_{\hat{\g}_{[2]}}$ (as constructed in convention \ref{conv: orthogonal_complement_of_toroidal_centres}) to $\z_{[2]} \oplus \d_{[2]} \cong \bar{\Omega}_{[2]} \oplus \d_{[2]}$.
            
            In other words, there is a commutative diagram of $\bbC$-vector spaces and $\bbC$-linear maps between them as follows:
                $$
                    \begin{tikzcd}
                	{\Omega_{[2]} \tensor_{\bbC} \der_{\bbC}(A_{[2]})} \\
                	& \bbC \\
                	{\bar{\Omega}_{[2]} \tensor_{\bbC} \d_{[2]}}
                	\arrow[dashed, from=1-1, to=3-1]
                	\arrow["{\iota_{[2]}}", from=1-1, to=2-2]
                	\arrow["{(-, -)_{\hat{\g}_{[2]}}}"', from=3-1, to=2-2]
                    \end{tikzcd}
                $$
        \end{proposition}
        \begin{lemma}[Canonical element for the canonical pairing of $1$-forms and vector fields] \label{lemma: canonical_elements_for_the_canonical_pairing_of_1_forms_and_vector_fields}
            Let $k$ be a field of characteristic $0$. Denote by:
                $$\iota_{[n]}^+: \Omega_{[n]}^+ \x \der_k(A_{[n]}^-) \to k$$
            the restriction of $\iota_{[n]}$ to be between differential $1$-forms and derivations on $A_{[n]}^+ \subset A_{[n]}$ and $A_{[n]}^- \subset A_{[n]}$ respectively. The canonical element attached to $\iota_{[n]}^+$ shall then be given by:
                $$\sfr_{[n]}^+ := v_1' \1(v_1, v_1') ... v_{n - 1}' \1(v_{n - 1}, v_{n - 1}') \1^+(v_n, v_n') \sum_{1 \leq i \leq n} dv_i \tensor \del_{v_i}$$
        \end{lemma}
            \begin{proof}
                A basis for $\Omega_{[n]}^+$ is given by:
                    $$\{ v_1^{m_1} ... v_{n - 1}^{m_{n - 1}} v_n^{m_n} dv_i \}_{1 \leq i \leq n}$$
                and its dual (which is a basis for $\der_k(A_{[n]}^-)$) with respect to $\iota_{[n]}^+$ is given by:
                    $$\{ v_1^{-m_1} ... v_{n - 1}^{m_{n - 1}} v_n^{-m_n - 1} \del_{v_i} \}_{1 \leq i \leq n}$$
                We then have - more-or-less tautologically - that:
                    $$
                        \begin{aligned}
                            \sfr_{[n]}^+ & := \sum_{1 \leq i \leq n} \sum_{(m_1, ..., m_{n - 1}, m_n) \in \Z^{n - 1} \x \Z_{\geq 0}} v_1^{m_1} ... v_{n - 1}^{m_{n - 1}} v_n^{m_n} dv_i \tensor v_1^{-m_1} ... v_{n - 1}^{m_{n - 1}} v_n^{-m_n - 1} \del_{v_i}
                            \\
                            & = v_1' \1(v_1, v_1') ... v_{n - 1}' \1(v_{n - 1}, v_{n - 1}') \1^+(v_n, v_n') \sum_{1 \leq i \leq n} dv_i \tensor \del_{v_i}
                        \end{aligned}
                    $$
            \end{proof}
        The lemma above suggests to us that it is possible to rewrite the canonical elements $\sfr_{\z_{[2]}^+}, \sfr_{\d_{[2]}^+}$ in terms of formal Dirac distributions. Furthermore, it implies that the problem now reduces to computing the complements of the bases of $\bar{\Omega}_{[2]}^+$ (i.e. a basis for $d(A_{[2]}^+)$) and of $\d_{[2]}^+$ inside the bases of $\Omega_{[2]}^+$ and of $\der_{\bbC}(A_{[2]}^+)$ respectively; due to the duality of $\Omega_{[2]}^+$ and $\der_{\bbC}(A_{[2]})$ via the bilinear form $\iota_{[2]}$, these complemenets ought to be in bijection with one another. This is because, the canonical element arising from those complementary basis elements, say $\sfr_{[2]}^{+ \varnothing}$, satisfies:
            $$\sfr_{\z_{[2]}^+} = \sfr_{[2]}^+ - \sfr_{[2]}^{+ \varnothing}$$
        (and likewise for $\sfr_{\d_{[2]}^+}$ after a flip map). 
        \begin{remark}[A basis for $A_{[n]}$] \label{remark: basis_for_global_functions_on_split_tori}
            Let $k$ be a field of characteristic $0$ and fix some $n \geq 1$. Set:
                $$v_0 := 1$$
            so that:
                $$A_{[0]} \cong k$$

            The $k$-algebra:
                $$A_{[n]} := k[v_1^{\pm 1}, ..., v_n^{\pm 1}]$$
            has a natural $\Z$-grading in the last variable $v_n$ given by:
                $$A_{[n]} \cong \bigoplus_{m_n \in \Z} A_{[n - 1]} v_n^{m_n}$$
            Applying the universal K\"ahler differential map $d: A_{[n]} \to \Omega_{[n]}$ then yields:
                $$d( A_{[n]} ) \cong \bigoplus_{m_n \in \Z} d( A_{[n - 1]} v_n^{m_n} ) \cong \bigoplus_{p \in \Z} ( d( A_{[n - 1]} ) v_n^{m_n} \oplus A_{[n - 1]} v_n^{m_n - 1} dv_n )$$

            When $n = 1$, the equation above specialises to:
                $$d(A_{[1]}) \cong \bigoplus_{m \in \Z} ( d( A_{[0]} ) v^m \oplus A_{[0]} v^{m - 1} dv ) \cong \bigoplus_{m \in \Z} k v^{m - 1} dv$$
            When $n = 2$, we subsequently have that:
                $$d(A_{[2]}) \cong \bigoplus_{p \in \Z} ( d(A_{[1]}) t^p \oplus A_{[1]} t^{p - 1} dt ) \cong \bigoplus_{(m, p) \in \Z^2} ( k v^{m - 1} t^p dv \oplus k v^m t^{p - 1} dt )$$
            from which it is inferable that the set:
                $$\{v^{m - 1} t^p dv\}_{(m, p) \in \Z^2} \cup \{v^m t^{p - 1} dt\}_{(m, p) \in \Z^2}$$
            forms a basis for $d(A_{[2]})$; to obtain a basis for $d(A_{[2]}^{\pm})$, simply change the indexing set from $\Z^2$ to $\Z \x \Z_{\geq 0}$ and $\Z \x \Z_{< 0}$ respectively.

            The subset of the standard basis for $\der_k(A_{[2]})$ whose elements are dual with respect to the bilinear form $\iota_{[2]}$ to the elements of the basis for $d(A_{[2]})$ (i.e. a basis for $\der_k(A_{[2]})/\d_{[2]}$) as above is thus:
                $$\{v^{-m + 1} t^{-p - 1} \del_v\}_{(m, p) \in \Z^2} \cup \{v^{-m} t^{-p} \del_t\}_{(m, p) \in \Z^2}$$
            and likewise, to obtain the corresponding subsets of the standard basis for $\der_k(A_{[2]}^{\pm})$, simply change the indexing set from $\Z^2$ to $\Z \x \Z_{\geq 0}$ and $\Z \x \Z_{< 0}$ respectively.

            Given the above, the canonical element for the restricted pairing:
                $$\iota_{[2]}: d(A_{[2]}^+) \x \der_k( A_{[2]}^- )/\d_{[2]}^- \to k$$
            (which is still non-degenerate and symmetric) then takes the form:
                $$
                    \begin{aligned}
                        \sfr_{[2]}^{+ \varnothing} & = \sum_{(m, p) \in \Z \x \Z_{\geq 0}} \left( v_1^{m - 1} t_1^p dv_1 \tensor v_2^{-m + 1} t_2^{-p - 1} \del_{v_2} + v_1^m t_1^{p - 1} dt_1 \tensor v_2^{-m} t_2^{-p} \del_{t_2} \right)
                        \\
                        & = v_1^{-1} v_2^2 \1(v_1, v_2) \1^+(t_1, t_2) dv_1 \tensor \del_{v_2} + v_2 \1(v_1, v_2) 
                    \end{aligned}
                $$
        \end{remark}
        \begin{proposition}[Toroidal classical R-matrices in terms of formal distributions] \label{prop: toroidal_classical_R_matrices_in_terms_of_formal_distributions}
            The canonical elements $\sfr_{\z_{[2]}^+}$ and $\sfr_{\d_{[2]}^+}$ (cf. corollary \ref{coro: extended_toroidal_lie_bialgebras}) can be alternatively written in the following manner, using formal Dirac distributions (cf. convention \ref{conv: formal_dirac_distributions})\footnote{Note that the total degrees of $\sfr_{\z_{[2]}^+}$ and $\sfr_{\d_{[2]}^+}$ are both $0$ in $v$ and $-1$ in $t$ (ultimately because $\deg \1(w, z) = \deg \1^+(z, w) = -1$ by construction), agreeing with remark \ref{remark: total_degrees_of_classical_yangian_R_matrices}.}:
                $$\sfr_{\z_{[2]}^+} = $$
                $$\sfr_{\d_{[2]}^+} = $$
        \end{proposition}
            \begin{proof}
                
            \end{proof}
        \begin{remark}[Toroidal classical R-matrices as meromorphic functions] \label{remark: toroidal_classical_R_matrices_as_meromorphic_functions}
            The appearance of formal Dirac distributions in the expressions for $\sfr_{\z_{[2]}^+}$ and $\sfr_{\d_{[2]}^+}$ (as well as for $\sfr_{\g_{[2]}^+}$; cf. question \ref{question: multiloop_lie_bialgebras}), and hence for $\sfr_{\tilde{\g}_{[2]}^+} = \sfr_{\hat{\g}_{[2]}^+}$ suggests to us that when regarded complex-analytically, these are meromorphic $\bbC$-valued functions in the variables $v, t$. It is thus natural to inquire into where the singularities of these functions lie. 
        \end{remark}
        \begin{question}
            Classify the singularities of $\sfr_{\tilde{\g}_{[2]}^+} = \sfr_{\hat{\g}_{[2]}^+}$ and compare them to those of the universal quantum R-matrix of $\rmY_{\hbar}(\hat{\g}_{[1]})$. 
        \end{question}

    \subsection{Classical limits of universal quantum R-matrices of affine Yangians}