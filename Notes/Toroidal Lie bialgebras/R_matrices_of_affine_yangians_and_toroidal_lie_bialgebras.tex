\section{R-matrices of toroidal Lie bialgebras and of affine Yangians}
    \subsection{Classical R-matrices of toroidal Lie bialgebras}
        \begin{remark}[Pairing of $1$-forms and vector fields] \label{remark: pairing_1_forms_and_vector_fields} 
            Suppose for a moment that $A$ is a commutative algebra over a field $k$ generated by some set:
                $$\{v_i\}_{i \in I}$$
            and suppose furthermore that $A$ is equipped with a non-degenerate and symmetric $k$-bilinear form:
                $$(-, -)_A: A \x A \to k$$
            Next, consider the following natural pairing between $\Omega^1_{A/k}$ and $\der_k(A)$, given as the interior product/contraction of differential forms by vector fields, i.e. in the following manner:
                $$\iota_A(f dv_i, g \del_{v_j}) := (f, g)_A \delta_{i, j}$$
            for all $f, g \in A$. Clearly, the pairing:
                $$\iota_A: \Omega^1_{A/k} \x \der_k(A) \to k$$
            is a non-degenerate and symmetric $k$-bilinear form.

            Now, in the particular case of $A := A_{[n]}$ (for some $n \geq 1$), note firstly that the algebra $A_{[n]}$ is naturally equipped with the non-degenerate and symmetric $k$-bilinear form:
                $$(v_1^{m_1} ... v_n^{m_n}, v_1^{m_1'} ... v_n^{m_n'})_{A_{[n]}} := \delta_{(m_1 + m_1', ..., m_{n - 1} + m_{n - 1}', m_n + m_n'), (0, ..., 0, -1)}$$
            (cf. question \ref{question: multiloop_lie_bialgebras}) and with respect to this, the pairing:
                $$\iota_{A_{[n]}}: \Omega_{[n]} \x \der_k(A_{[n]}) \to k$$
            is then given by:
                $$\iota_{A_{[n]}}(f dv_i, g \del_{v_j}) := (f, g)_{A_{[n]}} \delta_{i, j}$$
            for all $f, g \in A_{[n]}$ and all $1 \leq i, j \leq n$. 
        \end{remark}

        \begin{remark}[Pairing of cyclic $1$-forms and divergence-zero vector fields] \label{remark: pairing_cyclic_1_forms_and_div_zero_vector_fields} 
            Again, suppose for a moment that $A$ is an arbitrary commutative algebra over a field $k$ of characteristic $0$ and that $A$ is generated by some set:
                $$\{v_i\}_{i \in I}$$
            and that $A$ carries a non-degenerate and symmetric $k$-bilinear form $(-, -)_A$. Via the canonical quotient map of $k$-vector spaces:
                $$\Omega^1_{A/k} \to \bar{\Omega}^1_{A/k}$$
            one obtains an induced bilinear form that we will denote by:
                $$\bar{\iota}_A: \bar{\Omega}^1_{A/k} \x \der_k(A) \to k$$

            When $A = A_{[n]}$, recall from remark \ref{remark: centres_of_dual_toroidal_lie_algebras} that there is a canonical basis for $\bar{\Omega}^1_{A/k}$ consisting of the elements:
                $$m_i^{-1} v_1^{m_1} ... v_i^{m_i - 1} ... v_n^{m_n} \bar{d}(v_i)$$
            wherein $1 \leq i \leq n$. The bilinear form $\bar{\iota}_{[n]} := \bar{\iota}_{A_{[n]}}$ is then given by:
                $$\bar{\iota}_{[n]}( m_i^{-1} v_1^{m_1} ... v_i^{m_i - 1} ... v_n^{m_n} \bar{d}(v_i), v_1^{m_1'} ... v_n^{m_n'} \del_{v_j} ) = m_i^{-1} \delta_{(m_1 + m_1', ..., m_i + m_i' - 1, ..., m_n + m_n'), (0, ..., 0, -1)} \delta_{i, j}$$
            for all $1 \leq i, j \leq n$. While it is true that $\bar{\iota}_{[n]}$ is non-degenerate, we caution that - based on the calculation above - $\bar{\iota}_{[n]}$ is not symmetric due to the appearance of the $m_i^{-1}$ factor: switching the two inputs would yield a $(-m'_i + 1)^{-1}$ factor instead. The upshot is that, once we restrict the second input from $\der_k(A_{[n]})$ to the \say{Yangian divergence-zero} subspace $\d_{[n]}$, the bilinear form $\bar{\iota}_{[n]}$ ought to get symmetrised, and symmetry is needed for us to be able to write down a canonical element corresponding to the resulting symmetric pairing.

            Let us now focus on the case where $k = \bbC$ and $n = 2$, partly because this is the only case wherein we have a concrete basis for $\d_{[2]}$ to work with, but also because ultimately, this is the only case of particular interest to us. Recall from remark \ref{remark: dual_of_toroidal_centres_contains_derivations} and from the construction of $\d_{[2]}$ (cf. convention \ref{conv: orthogonal_complement_of_toroidal_centres}) that:
                $$\d_{[2]} := (\bigoplus_{(r, s) \in \Z^2} \bbC D_{r, s}) \oplus \bbC D_v \oplus \bbC D_t$$
            wherein:
                $$\forall (r, s) \in \Z^2: D_{r, s} = s v^{-r + 1} t^{-s - 1} \del_v - r v^{-r} t^{-s} \del_t$$
                $$D_v = -v t^{-1} \del_v, D_t = -\del_t$$
            Let us also recall from remark \ref{remark: centres_of_dual_toroidal_lie_algebras} that the canonical basis elements of $\bar{\Omega}_{[2]}$ can be alternatively written as:
                $$
                    K_{r, s} :=
                    \begin{cases}
                        \text{$\frac1s v^{r - 1} t^s \bar{d}(v)$ if $(r, s) \in \Z \x \Z$}
                        \\
                        \text{$-\frac1r v^r t^{-1} \bar{d}(t)$ if $(r, s) \in \Z \x \{0\}$}
                        \\
                        \text{$0$ if $(r, s) = (0, 0)$}
                    \end{cases}
                $$
                $$c_v := v^{-1} \bar{d}(v), c_t := t^{-1} \bar{d}(t)$$
            It is clear that:
                $$
                    \bar{\iota}_{[2]}( K_{r, s}, D ) =
                    \begin{cases}
                        \text{$1$ if $D = D_{r, s}$}
                        \\
                        \text{$0$ if $D \not \in \bbC D_{r, s}$}
                    \end{cases}
                $$
                $$
                    \bar{\iota}_{[2]}( c_v, D ) =
                    \begin{cases}
                        \text{$1$ if $D = D_v$}
                        \\
                        \text{$0$ if $D \not \in \bbC D_v$}
                    \end{cases}
                $$
                $$
                    \bar{\iota}_{[2]}( c_t, D ) =
                    \begin{cases}
                        \text{$1$ if $D = D_t$}
                        \\
                        \text{$0$ if $D \not \in \bbC D_t$}
                    \end{cases}
                $$
            so we have managed to show that not only is there a non-degenerate and symmetric $\bbC$-bilinear form:
                $$\bar{\iota}_{[2]}: \bar{\Omega}_{[2]} \x \d_{[2]} \to \bbC$$
            induced naturally by the canonical pairing of differential $1$-forms and derivations on $A_{[2]}$, but also that this induced bilinear form coincides with the restriction of the bilinear form $(-, -)_{\hat{\g}_{[2]}}$ (as constructed in convention \ref{conv: orthogonal_complement_of_toroidal_centres}) to $\z_{[2]} \oplus \d_{[2]} \cong \bar{\Omega}_{[2]} \oplus \d_{[2]}$, since we have per its construction that the elements $D_{r, s}, D_v, D_t$ are dual with respect to $(-, -)_{\hat{\g}_{[2]}}$ to the elements $K_{r, s}, c_v, c_t$, respectively. 
        \end{remark}

        We can package the previous two remarks into the following result:
        \begin{proposition}[Pairing of cyclic $1$-forms and divergence-zero vector fields] \label{prop: pairing_cyclic_1_forms_and_div_zero_vector_fields} 
            Denote the usual non-degenerate and symmetric bilinear pairing of differential $1$-forms and derivations on $A_{[2]}$ by:
                $$\iota_{[2]}: \Omega_{[2]} \x \der_{\bbC}(A_{[2]}) \to \bbC$$
            Via the canonical quotient map of $\bbC$-vector spaces:
                $$\Omega_{[2]} \to \bar{\Omega}_{[2]}$$
            one obtains an induced non-degenerate and symmetric bilinear pairing:
                $$\bar{\iota}_{[2]}: \bar{\Omega}_{[2]} \x \d_{[2]} \to \bbC$$
            coninciding with the restriction of the bilinear form $(-, -)_{\hat{\g}_{[2]}}$ (as constructed in convention \ref{conv: orthogonal_complement_of_toroidal_centres}) to $\z_{[2]} \oplus \d_{[2]} \cong \bar{\Omega}_{[2]} \oplus \d_{[2]}$.
            
            In other words, there is a commutative diagram of $\bbC$-vector spaces and $\bbC$-linear maps between them as follows:
                $$
                    \begin{tikzcd}
                	{\Omega_{[2]} \tensor_{\bbC} \der_{\bbC}(A_{[2]})} \\
                	& \bbC \\
                	{\bar{\Omega}_{[2]} \tensor_{\bbC} \d_{[2]}}
                	\arrow[dashed, from=1-1, to=3-1]
                	\arrow["{\iota_{[2]}}", from=1-1, to=2-2]
                	\arrow["{(-, -)_{\hat{\g}_{[2]}}}"', from=3-1, to=2-2]
                    \end{tikzcd}
                $$
        \end{proposition}
        \begin{corollary} \label{coro: pairing_cyclic_1_forms_and_div_zero_vector_fields}
            There are non-degenerate and symmetric $\bbC$-bilinear pairings:
                $$\bar{\iota}_{[2]}^{\pm}: \bar{\Omega}_{[2]}^{\pm} \x \d_{[2]}^{\pm} \to \bbC$$
            coinciding with the restriction to $\z_{[2]}^{\pm} \oplus \d_{[2]}^{\pm} \cong \bar{\Omega}_{[2]}^{\pm} \oplus \d_{[2]}^{\pm}$ of the bilinear form $(-, -)_{\hat{\g}_{[2]}}$. 
        \end{corollary}
        Let us note now that the components:
            $$\sfr_{\z_{[2]}^+} := \sum_{(r, s) \in \Z \x \Z_{> 0}} K_{r, s} \tensor D_{r, s} + c_v \tensor D_v$$
            $$\sfr_{\d_{[2]}^+} := \sum_{(r, s) \in \Z \x \Z_{\leq 0}} D_{r, s} \tensor K_{r, s} + D_t \tensor c_t$$
        of the classical R-matrix associated to the Lie bialgebra structure on $\hat{\g}_{[2]}^+$ as in corollary \ref{coro: extended_toroidal_lie_bialgebras} (or equivalently, on $\tilde{\g}_{[2]}^+$; cf. theorem \ref{theorem: toroidal_lie_bialgebras}) are - in light of corollary \ref{coro: pairing_cyclic_1_forms_and_div_zero_vector_fields} - nothing but the canonical elements associated to the (non-degenerate and symmetric) pairings:
            $$\bar{\iota}_{[2]}$$
        and:
            $$\bar{\iota}_{[2]} \circ (1 \: 2)$$
        respectively. 
        
        \begin{corollary}[Another way to write classical R-matrices of toroidal Lie bialgebras]
            The canonical elements $\sfr_{\z_{[2]}^+}$ and $\sfr_{\d_{[2]}^+}$ (cf. corollary \ref{coro: extended_toroidal_lie_bialgebras}) can be alternatively written in the following manner, using formal Dirac distributions:
                $$\sfr_{\z_{[2]}^+} = $$
                $$\sfr_{\d_{[2]}^+} = $$
        \end{corollary}
            \begin{proof}
                
            \end{proof}

    \subsection{Classical limits of universal quantum R-matrices of affine Yangians}