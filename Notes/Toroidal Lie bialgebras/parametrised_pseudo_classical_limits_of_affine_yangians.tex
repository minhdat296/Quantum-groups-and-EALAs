\section{Parametrised pseudo-classical limits of affine Yangians}
    \begin{convention}
        Throughout this section, we assume that $\g$ is simply laced, excluding the case where $\g$ is of type $\sfA_1$. 
    \end{convention}

    \subsection{Parametrised pseudo-coproducts}
        \begin{definition}[Parametrised pseudo-coproducts] \label{def: parametrised_pseudo_coproducts}
            Let $V$ be a $\bbC$-vector space. 

            Firstly, for $u$ a generic formal variable, fix some $\bbC$-linear map:
                $$\Delta_u: V \to V^{\tensor 2}(\!(u)\!)$$
            A \textbf{parametrised pseudo-coproduct} on $V$ is then a sequence of $\bbC$-linear maps $\{\Delta_{u_1, ..., u_n}\}_{n \geq 1}$ defined in the following manner: for each $n \geq 1$, define a $\bbC$-linear map:
                $$\Delta_{u_1, ..., u_n}: V \to V^{\tensor (n + 1)}(\!(u_1)\!) ... (\!(u_n)\!)$$
            given recursively by:
                $$\Delta_{u_1, ..., u_n} := ( \id_{ V^{\tensor (n - 1)} } \tensor \Delta_{u_n} ) \circ \Delta_{u_1, ..., u_{n - 1}}$$
            The maps $\Delta_{u_1, ..., u_n}$ are to also satisfy the following \textbf{pseudo-coassociativity} property:
                $$\Delta_{u_1, u_2} = ( \Delta_{u_1} \tensor \id_{V(\!(u_2)\!)} ) \circ \Delta_{u_2} = ( \id_{V(\!(u_1)\!)} \tensor \Delta_{u_2} ) \circ \Delta_{u_1}$$
            which extend in the obvious manner to the cases where $n > 2$. 

            If $V$ is a $\bbC$-algebra and $\Delta_u$ is a $\bbC$-algebra homomorphism, then $\Delta_u$ will define a \textbf{parametrised pseudo-bialgebra} structure on $V$.
        \end{definition}
        For us, the point of introducing the notion of parametrised pseudo-coproducts is that, as it turns out, such a structure exists on the Yangian $\rmY(\g)$ whenever $\g$ is of finite-type (except when $\g$ is of type $\sfA_1$) or $\g$ is of untwisted affine type (except when $\g$ is either of type $\sfA_1^{(1)}$ or $\sfA_1^{(2)}$). This parametrised pseudo-coproduct was constructed by Guay, Nakajima, and Wendlandt in \cite[Section 6]{guay_nakajima_wendlandt_affine_yangian_coproduct}.
        \begin{theorem}[The parametrised pseudo-coproduct on Yangians] \label{theorem: parametrised_pseudo_coproduct_on_yangians}
            (Cf. \cite[Theorem 6.2]{guay_nakajima_wendlandt_affine_yangian_coproduct}) The following algebra homomorphism:
                $$\Delta_u: \rmY(\g) \to \rmY(\g)^{\tensor 2}(\!(u)\!)$$
            given as follows\footnote{Note that it is given only for low-degree generators, which we know to be enough.}:
                $$\Delta_u(E_{i, 0}^{\pm}) := E_{i, 0}^{\pm} \tensor 1 + 1 \tensor E_{i, 0}^{\pm} u^{\pm 1}$$
                $$\Delta_u(H_{i, 0}) := \bar{\Delta}(H_{i, 0})$$
                $$\Delta_u(T_{i, 1}) := \bar{\Delta}(T_{i, 1}) - \sum_{\alpha \in \hat{\Phi}^+} (\alpha, \alpha_i) e_{\alpha}^- \tensor e_{\alpha}^+ u^{\height(\alpha)}$$
            given for each $i \in \Gamma_0$ and for some choices\footnote{The expression is choice-independent, of course.} of root vectors $e_{\pm\alpha} \in \g_{\pm\alpha}$, normalised so that:
                $$(e_{\alpha}^-, e_{\alpha}^+) = 1$$
            will define a parametrised pseudo-bialgebra structure on $\rmY(\g)$ in the sense of definition \ref{def: parametrised_pseudo_coproducts}, compatible with the algebra structure on $\rmY(\g)$.
        \end{theorem}
        
        \begin{theorem}[The parametrised pseudo-coproduct on formal Yangians] \label{theorem: parametrised_pseudo_coproduct_on_formal_yangians}
            The following algebra homomorphism:
                $$\Delta_{u, \hbar}: \rmY_{\hbar}(\g) \to \rmY_{\hbar}(\g)^{\tensor 2}(\!(u)\!)$$
            given as follows:
                $$\Delta_{u, \hbar}(E_{i, 0}^{\pm}) := E_{i, 0}^{\pm} \tensor 1 + 1 \tensor E_{i, 0}^{\pm} u^{\pm 1}$$
                $$\Delta_{u, \hbar}(H_{i, 0}) := \bar{\Delta}(H_{i, 0})$$
                $$\Delta_{u, \hbar}(T_{i, 1}(\hbar)) := \bar{\Delta}(T_{i, 1}) - \hbar \sum_{\alpha \in \hat{\Phi}^+} (\alpha, \alpha_i) e_{\alpha}^- \tensor e_{\alpha}^+ u^{\height(\alpha)}$$
            given for each $i \in \Gamma_0$ and for some choices of root vectors $e_{\pm\alpha} \in \g_{\pm\alpha}$, normalised so that:
                $$(e_{\alpha}^-, e_{\alpha}^+) = 1$$
            will define a parametrised pseudo-bialgebra structure on $\rmY_{\hbar}(\g)$ in the sense of definition \ref{def: parametrised_pseudo_coproducts}, compatible with the algebra structure on $\rmY_{\hbar}(\g)$.
        \end{theorem}
            \begin{proof}
                
            \end{proof}

    \subsection{Parametrised pseudo-classical limits of affine Yangians; toroidal parametrised Lie pseudo-bialgebras}
        \begin{definition}[Parametrised pseudo-quantisations] \label{def: parametrised_pseudo_quantisations}
            
        \end{definition}
        \begin{theorem}[Parametrised pseudo-classical limits of affine Yangians] \label{theorem: parametrised_pseudo_classical_limits_of_affine_yangians}
            
        \end{theorem}
            \begin{proof}
                
            \end{proof}