\section{Classical limits of affine Yangians}
    \begin{convention}
        We shall be using the following shorthand:
            $$\{ X_1, ..., X_n \} := \sum_{\sigma \in S_n} X_{\sigma(1)} \cdot ... \cdot X_{\sigma(n)}$$
            $$\bar{\Delta}(X) := X \tensor 1 + 1 \tensor X$$
    \end{convention}

    \begin{convention} \label{conv: general_symmetrisable_kac_moody_algebra}
        Until further notice, let us assume that $\g$ is a general symmetrisable Kac-Moody algebra whose Cartan matrix is indecomposable\footnote{The indecomposability assumption is not necessary. It is made simply so that when $\g$ is of finite type, we would be back in the situation of convention \ref{conv: a_fixed_finite_dimensional_simple_lie_algebra}}.
    \end{convention}

    \subsection{(Pseudo-)coproducts on affine Yangians}
        We begin by reviewing the constructions of a coproduct and a \say{parametrised pseudo-coproduct} giving rise, respectively, to a bialgebra and a pseudo-bialgebra structure on the Yangian $\rmY(\hat{\g}_{[1]})$, with $\hat{\g}_{[1]}$ denoting the untwisted affine Kac-Moody algebra associated to $\g$, as was done in \cite[Chapter 7]{kac_infinite_dimensional_lie_algebras} (cf. convention \ref{conv: a_fixed_untwisted_affine_kac_moody_algebra}). For more details, see \cite[Sections 4, 5, and 6]{guay_nakajima_wendlandt_affine_yangian_coproduct}.
    
        \begin{definition}[Parametrised pseudo-coproducts] \label{def: parametrised_pseudo_coproducts}
            Let $V$ be a $\bbC$-vector space. 

            Firstly, for $u$ a generic formal variable, fix some $\bbC$-linear map:
                $$\Delta_u: V \to V^{\tensor 2}(\!(u)\!)$$
            A \textbf{parametrised pseudo-coproduct} on $V$ is then a sequence of $\bbC$-linear maps $\{\Delta_{u_1, ..., u_n}\}_{n \geq 1}$ defined in the following manner: for each $n \geq 1$, define a $\bbC$-linear map:
                $$\Delta_{u_1, ..., u_n}: V \to V^{\tensor (n + 1)}(\!(u_1)\!) ... (\!(u_n)\!)$$
            given recursively by:
                $$\Delta_{u_1, ..., u_n} := ( \id_{ V^{\tensor (n - 1)} } \tensor \Delta_{u_n} ) \circ \Delta_{u_1, ..., u_{n - 1}}$$
            The maps $\Delta_{u_1, ..., u_n}$ are to also satisfy the following \textbf{pseudo-coassociativity} property:
                $$\Delta_{u_1, u_2} = ( \Delta_{u_1} \tensor \id_{V(\!(u_2)\!)} ) \circ \Delta_{u_2} = ( \id_{V(\!(u_1)\!)} \tensor \Delta_{u_2} ) \circ \Delta_{u_1}$$
            which extend in the obvious manner to the cases where $n > 2$. 

            If $V$ is a $\bbC$-algebra and $\Delta_u$ is a $\bbC$-algebra homomorphism, then $\Delta_u$ will define a \textbf{parametrised pseudo-bialgebra} structure on $V$.
        \end{definition}
        \begin{theorem}[The parametrised pseudo-coproduct on Yangians] \label{theorem: parametrised_pseudo_coproduct_on_yangians}
            (Cf. \cite[Theorem 6.2]{guay_nakajima_wendlandt_affine_yangian_coproduct}) If $\g$ is either of finite type but not $\sfA_1$ or of affine type but not $\sfA_1^{(1)}$ and not $\sfA_1^{(2)}$, then the following algebra homomorphism:
                $$\Delta_u: \rmY(\g) \to \rmY(\g)^{\tensor 2}(\!(u)\!)$$
            given as follows\footnote{Note that it is given only for low-degree generators, which we know to be enough.}:
                $$\Delta_u(E_{i, 0}^{\pm}) := E_{i, 0}^{\pm} \tensor 1 + 1 \tensor E_{i, 0}^{\pm} u^{\pm 1}$$
                $$\Delta_u(H_{i, 0}) := \bar{\Delta}(H_{i, 0})$$
                $$\Delta_u(T_{i, 1}) := \bar{\Delta}(T_{i, 1}) - \sum_{k \geq 1} \sum_{\alpha \in \Phi^+} (\alpha, \alpha_i) e_{\alpha}^- \tensor e_{\alpha}^+ u^k$$
            given for each $i \in \Gamma_0$ and for some choices\footnote{The expression is choice-independent, of course.} of root vectors $e_{\pm\alpha} \in \g_{\pm\alpha}$, normalised so that:
                $$(e_{\alpha}^-, e_{\alpha}^+) = 1$$
            will define a parametrised pseudo-bialgebra structure on $\rmY(\g)$ in the sense of definition \ref{def: parametrised_pseudo_coproducts}, compatible with the algebra structure on $\rmY(\g)$.
        \end{theorem}
            \begin{proof}
                
            \end{proof}

        \begin{theorem}[The parametrised pseudo-coproduct on formal Yangians] \label{theorem: parametrised_pseudo_coproduct_on_formal_yangians}
            If $\g$ is either of finite type but not $\sfA_1$ or of affine type but not $\sfA_1^{(1)}$ and not $\sfA_1^{(2)}$, then the following algebra homomorphism:
                $$\Delta_{u, \hbar}: \rmY_{\hbar}(\g) \to \rmY_{\hbar}(\g)^{\tensor 2}(\!(u)\!)$$
            given as follows:
                $$\Delta_{u, \hbar}(E_{i, 0}^{\pm}) := E_{i, 0}^{\pm} \tensor 1 + 1 \tensor E_{i, 0}^{\pm} u^{\pm 1}$$
                $$\Delta_{u, \hbar}(H_{i, 0}) := \bar{\Delta}(H_{i, 0})$$
                $$\Delta_{u, \hbar}(T_{i, 1}(\hbar)) := \bar{\Delta}(T_{i, 1}(\hbar)) - \hbar \sum_{k \geq 1} \sum_{\alpha \in \Phi^+} (\alpha, \alpha_i) e_{\alpha}^- \tensor e_{\alpha}^+ u^k$$
            given for each $i \in \Gamma_0$ and for some choices of root vectors $e_{\pm\alpha} \in \g_{\pm\alpha}$, normalised so that:
                $$(e_{\alpha}^-, e_{\alpha}^+) = 1$$
            will define a parametrised pseudo-bialgebra structure on $\rmY_{\hbar}(\g)$ in the sense of definition \ref{def: parametrised_pseudo_coproducts}, compatible with the algebra structure on $\rmY_{\hbar}(\g)$.
        \end{theorem}
            \begin{proof}
                
            \end{proof}
        \begin{corollary}[The coproduct on completed formal Yangians]
            If $\g$ is either of finite type but not $\sfA_1$ or of affine type but not $\sfA_1^{(1)}$ and not $\sfA_1^{(2)}$, then there will be a continuous coproduct:
                $$\Delta_{\hbar}: \rmY_{\hbar}(\g) \to \rmY_{\hbar}(\g) \hattensor_{\bbC} \rmY_{\hbar}(\g)$$
            satisfying:
                $$\Delta_{u, \hbar} = (1 \tensor s_{u, \hbar}) \circ \Delta_{\hbar}$$
            wherein:
                $$s_{u, \hbar}: \rmY_{\hbar}(\g) \to \rmY_{\hbar}(\g)(\!(u)\!)$$
            is the $\bbC$-algebra homomorphism given as follows, for all $(i, r) \in \Gamma_0 \x \Z_{\geq 0}$ and all $h \in \h$:
                $$s_{u, \hbar}(E_{i, r}^{\pm}) := E_{i, r}^{\pm} u^{\pm}$$
                $$s_{u, \hbar}(H_{i, r}) := H_{i, r}, s_{u, \hbar}(h) := h$$
            In fact, $\Delta_{\hbar}$ is the lift of $\Delta$ via the Rees algebra construction (recall that $\rmY_{\hbar}(\g) \cong \Rees_{\hbar} \rmY(\g)$ with respect to the $\Z_{\geq 0}$-filtration on the second index of the Chevalley-Serre generators $E_{i, r}^{\pm}, H_{i, r}$).
        \end{corollary}
            \begin{proof}
                
            \end{proof}

    \subsection{Toroidal Lie algebras as classical limits of affine Yangians}
        \begin{convention}[Dirac distributions]
            We will be using the following shorthands:
                $$\1(z, w) = \sum_{m \in \Z} z^m w^{-m - 1}$$
                $$\1^+(z, w) = \sum_{m \in \Z_{\geq 0}} z^m w^{-m - 1}$$
        \end{convention}

        \begin{remark}[Total degrees of \say{Yangian} canonical elements] \label{remark: total_degrees_of_classical_yangian_R_matrices}
            One property of the R-matrix $\sfr_{\hat{\g}_{[2]}^+}$ from corollary \ref{coro: extended_toroidal_lie_bialgebras} that will help simplify some computations later on (see the proof of lemma \ref{lemma: toroidal_lie_bialgebras}) is that they are of total degree $-1$. 

            Recall that if $V := \bigoplus_{m \in \Z} V_m, W := \bigoplus_{n \in \Z} W_n$ are $\Z$-graded vector spaces then for any $k \in \Z$, we have that:
                $$(V \tensor_{\bbC} W)_k \cong \bigoplus_{m + n = k} V_m \tensor_{\bbC} W_n$$
                
            If we now take $V = W = \rmU(\tilde{\g}_{[2]})$ then the claim from above would read:
                $$\sfr_{\tilde{\g}_{[2]}^+} \in ( \rmU(\tilde{\g}_{[2]}^+) \tensor_{\bbC} \rmU(\tilde{\g}_{[2]}^-) )_{-1}$$
            with the $\Z$-grading on $\tilde{\g}_{[2]}^{\pm}$ (and hence on $\rmU(\tilde{\g}_{[2]}^{\pm})$) being the one on the second variable $t$ (cf. remark \ref{remark: Z_gradings_on_toroidal_lie_algebras}), and actually, this is entirely due to:
                $$\sfr_{\g_{[2]}^+} \in ( \rmU(\g_{[2]}^+) \tensor_{\bbC} \rmU(\g_{[2]}^-) )_{-1}$$
            and this can already be inferred from the computations done in question \ref{question: multiloop_lie_bialgebras}, where we showed that:
                $$\sfr_{\g_{[2]}^+} = \sfr_{\g} v_2 \1(v_1, v_2) \1^+(t_1, t_2)$$
            (the crucial detail to notice here is that $\deg \1^+(t_1, t_2) = -1$ since $\1^+(t_1, t_2) := \sum_{p \in \Z_{\geq 0}} t_1^p t_2^{-p - 1}$).

            What this means for us is that, should we have $X \in \tilde{\g}_{[2]}^+$ such that:
                $$\deg X \leq 0$$
            then it will automatically be the case that:
                $$\delta_{\tilde{\g}_{[2]}^+}(X) = 0$$
        \end{remark}
        
        We are now finally able to put a Lie cobracket on the toroidal Lie algebra $\tilde{\g}_{[2]}^+$, compatible with the Lie bracket thereon in a manner that produces a Lie bialgebra structure. This Lie bialgebra structure is the classical limit of the coproduct on the formal Yangian $\rmY_{\hbar}(\hat{\g}_{[1]})$. 
        \begin{lemma}[Toroidal Lie bialgebras] \label{lemma: toroidal_lie_bialgebras}
            Assume convention \ref{conv: a_fixed_untwisted_affine_kac_moody_algebra} and let us abbreviate:
                $$\hat{\delta}^+ := \delta_{\hat{\g}_{[2]}^+}$$
            with $\delta_{\hat{\g}_{[2]}^+}$ as in corollary \ref{coro: extended_toroidal_lie_bialgebras}. Let:
                $$\tilde{\delta}^+ := \hat{\delta}^+|_{\tilde{\g}_{[2]}}$$
            Then $(\tilde{\g}_{[2]}^+, \tilde{\delta}^+)$ will be a complete topological Lie sub-bialgebra of $(\hat{\g}_{[2]}^+, \hat{\delta}^+)$ as given in corollary \ref{coro: extended_toroidal_lie_bialgebras}. Thanks to corollary \ref{coro: levendorskii_presentation__for_central_extensions_of_multiloop_algebras}, we know that it is enough to specify how $\tilde{\delta}^+$ is given on the set of generators:
                $$\{E_{i, 0}^{\pm}\}_{i \in \hat{\Gamma}_0} \cup \{H_{i, r}\}_{ (i, r) \in \hat{\Gamma}_0 \x \{0, 1\} }$$
            and since we know that under the isomorphism $\s^+ \xrightarrow[]{\cong} \tilde{\g}_{[2]}^+$ in \textit{loc. cit.}, we have the following assignments:
                $$\forall i \in \hat{\Gamma}_0: E_{i, 0}^{\pm} \mapsto e_i^{\pm}, H_{i, 0} \mapsto h_i$$
                $$\forall i \in \Gamma_0: H_{i, 1} \mapsto h_i t$$
                $$H_{\theta, 1} \mapsto h_{\theta} t + t c_v$$
            it is enough to specify the following, wherein $h \in \h$ is arbitrary:
                $$\tilde{\delta}^+(h) = 0$$
                $$\tilde{\delta}^+(ht) = [h_1 \tensor 1, \sfr_{\g} v_2 \1(v_1, v_2)]$$
                $$\tilde{\delta}^+(t c_v) = 0$$
        \end{lemma}
            \begin{proof}
                \begin{enumerate}
                    \item Since $\deg x = 0$ for all $x \in \g$, we get via remark \ref{remark: total_degrees_of_classical_yangian_R_matrices} that:
                        $$\hat{\delta}^+(x) = 0$$
                    and in particular, we have that:
                        $$\hat{\delta}^+(h) = 0$$

                    \item Let us now compute $\hat{\delta}^+(ht)$ for an arbitrary $h \in \h$. 
                    \begin{enumerate}
                        \item \textbf{($\g_{[2]}^+$-component):} Firstly, to compute:
                            $$[\bar{\Delta}(ht), \sfr_{\g_{[2]}^+}]$$
                        let us firstly recall from question \ref{question: multiloop_lie_bialgebras} that:
                            $$\sfr_{\g_{[2]}^+} = \sfr_{\g} v_2\1(v_1, v_2) \1^+(t_1, t_2)$$
                        (with notations as in \textit{loc. cit.}); let us also choose a root basis for $\g$ for writing out $\sfr_{\g}$ explicitly: this is to say that for each positive root $\alpha \in \Phi^+$, we choose corresponding basis vectors $e_{\alpha}^{\pm} \in \g_{\pm \alpha}$ normalised so that:
                            $$(e_{\alpha}^-, e_{\alpha}^+)_{\g} = 1$$
                        to get the following basis for $\g$:
                            $$\{h_i\}_{i \in \Gamma_0} \cup \{e_{\alpha}^-, e_{\alpha}^+\}_{\alpha \in \Phi^+}$$
                        From this, we see that:
                            $$
                                \begin{aligned}
                                    & [\bar{\Delta}(ht), \sfr_{\g_{[2]}^+}]
                                    \\
                                    = & -\sum_{i \in \Gamma_0} [\bar{\Delta}(ht), h_i \tensor h_i v_2\1(v_1, v_2) \1^+(t_1, t_2)] - \sum_{\alpha \in \Phi^+} [\bar{\Delta}(ht), (e_{\alpha}^- \tensor e_{\alpha}^+ + e_{\alpha}^+ \tensor e_{\alpha}^-) v_2\1(v_1, v_2) \1^+(t_1, t_2)]
                                \end{aligned}
                            $$

                        Now, for each $i \in \Gamma_0$, observe that:
                            $$
                                \begin{aligned}
                                    & [h t_1 \tensor 1, h_i \tensor h_i v_2\1(v_1, v_2) \1^+(t_1, t_2)]
                                    \\
                                    = & \sum_{(m, p) \in \Z \x \Z_{\geq 0}} [ht_1 \tensor 1, h_i v_1^m t_1^p \tensor h_i v_2^{-m} t_2^{-p - 1}]
                                    \\
                                    = & \sum_{(m, p) \in \Z \x \Z_{\geq 0}} [ht_1, h_i v_1^m t_1^p]_{\tilde{\g}_{[2]}^+} \tensor h_i v_2^{-m} t_2^{-p - 1}
                                    \\
                                    = & \sum_{(m, p) \in \Z \x \Z_{\geq 0}} (h, h_i)_{\g} v_1^m t_1^p \bar{d}(t_1) \tensor h_i v_2^{-m} t_2^{-p - 1}
                                \end{aligned}
                            $$
                        and likewise, that:
                            $$[1 \tensor h t_2, h_i \tensor h_i v_2\1(v_1, v_2) \1^+(t_1, t_2)] = \sum_{(m, p) \in \Z \x \Z_{\geq 0}} h_i v_1^m t_1^p \tensor (h, h_i)_{\g} v_2^{-m} t_2^{-p - 1} \bar{d}(t_2)$$
                        Adding the two summands together then yields:
                            $$
                                \begin{aligned}
                                    & [\bar{\Delta}(ht), h_i \tensor h_i v_2\1(v_1, v_2) \1^+(t_1, t_2)]
                                    \\
                                    = & (h, h_i)_{\g} \sum_{(m, p) \in \Z \x \Z_{\geq 0}} \left( v_1^m t_1^p \bar{d}(t_1) \tensor h_i v_2^{-m} t_2^{-p - 1} + h_i v_1^m t_1^p \tensor v_2^{-m} t_2^{-p - 1} \bar{d}(t_2) \right)
                                    \\
                                    = & (h, h_i)_{\g} ( \bar{d}(t_1) \tensor h_i + h_i \tensor \bar{d}(t_2) ) v_2\1(v_1, v_2) \1^+(t_1, t_2)
                                \end{aligned}
                            $$
                        
                        Next, consider the following:
                            $$
                                \begin{aligned}
                                    & [ht_1 \tensor 1, e_{\alpha}^- \tensor e_{\alpha}^+ v_2\1(v_1, v_2) \1^+(t_1, t_2)]
                                    \\
                                    = & \sum_{(m, p) \in \Z \x \Z_{\geq 0}} [ht_1 \tensor 1, e_{\alpha}^- v_1^m t_1^p \tensor e_{\alpha}^+ v_2^{-m} t_2^{-p - 1}]
                                    \\
                                    = & \sum_{(m, p) \in \Z \x \Z_{\geq 0}} [ht_1, e_{\alpha}^- v_1^m t_1^p]_{\tilde{\g}_{[2]}^+} \tensor e_{\alpha}^+ v_2^{-m} t_2^{-p - 1}
                                    \\
                                    = & \sum_{(m, p) \in \Z \x \Z_{\geq 0}} \left( -\alpha(h) e_{\alpha}^- v_1^m t_1^{p + 1} + (h, e_{\alpha}^-)_{\g} t_1 \bar{d}(v_1^m t_1^p) \right) \tensor e_{\alpha}^+ v_2^{-m} t_2^{-p - 1}
                                    \\
                                    = & \sum_{(m, p) \in \Z \x \Z_{\geq 0}} -\alpha(h) e_{\alpha}^- v_1^m t_1^{p + 1} \tensor e_{\alpha}^+ v_2^{-m} t_2^{-p - 1}
                                    \\
                                    & = -\alpha(h) ( e_{\alpha}^- \tensor e_{\alpha}^+ ) v_2 \1(v_1, v_2) t_1 \1^+(t_1, t_2)
                                \end{aligned}    
                            $$
                        wherein the second-to-last identity comes from the fact that\footnote{This can be proven easily by passing to the vector representation of $\g$, wherein $h$ is represented by a diagonal matrix while $e^{\pm}$ is represented by an upper/lower triangular matrix, and then using the fact that $(-, -)_{\g}$ differs from the trace form only by a non-zero constant.}:
                            $$(h, e^{\pm})_{\g} = 0$$
                        for every $h \in \h$ and every $e^{\pm} \in \n^{\pm}$. Similarly, we find that:
                            $$[ht_1 \tensor 1, e_{\alpha}^+ \tensor e_{\alpha}^- v_2\1(v_1, v_2) \1^+(t_1, t_2)] = \alpha(h) ( e_{\alpha}^+ \tensor e_{\alpha}^- ) v_2 \1(v_1, v_2) t_1 \1^+(t_1, t_2)$$
                        By putting the two together, one obtains:
                            $$[h t_1 \tensor 1, (e_{\alpha}^- \tensor e_{\alpha}^+ + e_{\alpha}^+ \tensor e_{\alpha}^-) v_2\1(v_1, v_2) \1^+(t_1, t_2)] = -\alpha(h) ( e_{\alpha}^- \tensor e_{\alpha}^+ - e_{\alpha}^+ \tensor e_{\alpha}^- ) v_2 \1(v_1, v_2) t_1 \1^+(t_1, t_2)$$
                        Likewise, we find that:
                            $$[1 \tensor h t_2, (e_{\alpha}^- \tensor e_{\alpha}^+ + e_{\alpha}^+ \tensor e_{\alpha}^-) v_2\1(v_1, v_2) \1^+(t_1, t_2)] = \alpha(h) ( e_{\alpha}^- \tensor e_{\alpha}^+ - e_{\alpha}^+ \tensor e_{\alpha}^- ) v_2 \1(v_1, v_2) t_2 \1^+(t_1, t_2)$$
                        and hence:
                            $$
                                \begin{aligned}
                                    & [\bar{\Delta}(ht), \sfr_{\g_{[2]}^+}]
                                    \\
                                    = & -\left( \sum_{i \in \Gamma_0} (h, h_i)_{\g} ( \bar{d}(t_1) \tensor h_i + h_i \tensor \bar{d}(t_2) ) + \sum_{\alpha \in \Phi^+} \alpha(h) ( e_{\alpha}^- \tensor e_{\alpha}^+ - e_{\alpha}^+ \tensor e_{\alpha}^- )(t_2 - t_1) \right) v_2 \1(v_1, v_2) \1^+(t_1, t_2)
                                    \\
                                    & = -\left( \bar{d}(t_1) \tensor h + h \tensor \bar{d}(t_2) + [h_1 \tensor 1, \sfr_{\g}] (t_2 - t_1) \right) v_2 \1(v_1, v_2) \1^+(t_1, t_2)
                                    \\
                                    & = -\left( \bar{d}(t_1) \tensor h + h \tensor \bar{d}(t_2) \right) v_2 \1(v_1, v_2) \1^+(t_1, t_2) + [h_1 \tensor 1, \sfr_{\g}] v_2 \1(v_1, v_2)
                                \end{aligned}
                            $$
                        We note that the last equality holds thanks to the fact that:
                            $$(t_2 - t_1) \1^+(t_1, t_2) = (t_2 - t_1) \sum_{p \in \Z_{\geq 0}} t_1^p t_2^{-p - 1} = (t_2 - t_1) \frac{1}{t_2 - t_1} = 1$$
                            
                        \item \textbf{($\z_{[2]}^+$-component):} Recall from corollary \ref{coro: extended_toroidal_lie_bialgebras} that:
                            $$\sfr_{\z_{[2]}^+} := \sum_{(r, s) \in \Z \x \Z_{> 0}} K_{r, s} \tensor D_{r, s} + c_{v_1} \tensor D_{v_2}$$
                        and so:
                            $$
                                \begin{aligned}
                                    & [\bar{\Delta}(ht), \sfr_{\z_{[2]}^+}]
                                    \\
                                    = & \sum_{(r, s) \in \Z \x \Z_{> 0}} [\bar{\Delta}(ht), K_{r, s} \tensor D_{r, s}] + [\bar{\Delta}(ht), c_{v_1} \tensor D_{v_2}]
                                    \\
                                    = & -\sum_{(r, s) \in \Z \x \Z_{> 0}} K_{r, s} \tensor h D_{r, s}(t) - c_{v_1} \tensor h D_{v_2}(t_2)
                                    \\
                                    = & -\sum_{(r, s) \in \Z \x \Z_{> 0}} K_{r, s} \tensor r h v_2^{-r} t_2^{-s}
                                \end{aligned}
                            $$
                        where the minus sign in the third equation appeared because:
                            $$[ht, D_{r, s}] = -[D_{r, s}, ht] = -h D_{r, s}(t)$$
                            $$[ht, D_v] = -[D_v, ht] = -h D_v(t)$$
                        (cf. remark \ref{remark: derivation_action_on_multiloop_algebras}) and the last equality is due to the fact that:
                            $$D_{r, s} = -s v^{-r + 1} t^{-s - 1} \del_v + r v^{-r} t^{-s} \del_t$$
                            $$D_v = -v t^{-1} \del_v$$
                        (cf. remark \ref{remark: dual_of_toroidal_centres_contains_derivations}). 
                        
                        \item \textbf{($\d_{[2]}^+$-component):} Recall from corollary \ref{coro: extended_toroidal_lie_bialgebras} that:
                            $$\sfr_{\z_{[2]}^+} := \sum_{(r, s) \in \Z \x \Z_{\leq 0}} D_{r, s} \tensor K_{r, s} + D_{t_1} \tensor c_{t_2}$$
                        and so:
                            $$
                                \begin{aligned}
                                    & [\bar{\Delta}(ht), \sfr_{\z_{[2]}^+}]
                                    \\
                                    = & \sum_{(r, s) \in \Z \x \Z_{\leq 0}} [\bar{\Delta}(ht), D_{r, s} \tensor K_{r, s}] + [\bar{\Delta}(ht), D_{t_1} \tensor c_{t_2}]
                                    \\
                                    = & -\sum_{(r, s) \in \Z \x \Z_{\leq 0}} h D_{r, s}(t_1) \tensor K_{r, s} - h D_{t_1}(t_1) \tensor c_{t_2}
                                    \\
                                    = & -\sum_{(r, s) \in \Z \x \Z_{\leq 0}} r h v_1^{-r} t_1^{-s} \tensor K_{r, s} + h \tensor c_{t_2}
                                \end{aligned}
                            $$
                        where the minus sign in the third equation appeared because:
                            $$[ht, D_{r, s}] = -[D_{r, s}, ht] = -h D_{r, s}(t)$$
                            $$[ht, D_t] = -[D_t, ht] = -h D_t(t)$$
                        (cf. remark \ref{remark: derivation_action_on_multiloop_algebras}) and the the last equality is due to the fact that:
                            $$D_{r, s} = -s v^{-r + 1} t^{-s - 1} \del_v + r v^{-r} t^{-s} \del_t$$
                            $$D_t = -\del_t$$
                        (cf. remark \ref{remark: dual_of_toroidal_centres_contains_derivations}). 
                    \end{enumerate}

                    Since we know that:
                        $$[\bar{\Delta}(ht), \sfr_{\g_{[2]}^+}] = -\left( \bar{d}(t_1) \tensor h + h \tensor \bar{d}(t_2) \right) v_2 \1(v_1, v_2) \1^+(t_1, t_2) + [h_1 \tensor 1, \sfr_{\g}] v_2 \1(v_1, v_2)$$
                    we now claim that:
                        $$[\bar{\Delta}(ht), \sfr_{\z_{[2]}^+} + \sfr_{\d_{[2]}^+}] = \left( \bar{d}(t_1) \tensor h + h \tensor \bar{d}(t_2) \right) v_2 \1(v_1, v_2) \1^+(t_1, t_2)$$
                    (since ultimately, we would like to show that $\hat{\delta}^+(ht) = \sfr_{\g} v_2 \1(v_1, v_2)$), and to prove that this is the case, let us first note that we now have that:
                        $$
                            \begin{aligned}
                                & [\bar{\Delta}(ht), \sfr_{\z_{[2]}^+} + \sfr_{\d_{[2]}^+}]
                                \\
                                = & -\sum_{(r, s) \in (\Z \setminus \{0\}) \x \Z_{> 0}} \left( K_{r, s} \tensor r h v_2^{-r} t_2^{-s} + r h v_1^{-r} t_1^s \tensor K_{r, -s} \right) - \sum_{r \in \Z \setminus \{0\}} r h v_1^{-r} \tensor K_{r, 0} + h \tensor c_{t_2}
                            \end{aligned}
                        $$
                    wherein the first summand corresponds to the indices $(r, 0) \in \Z \x \Z_{\leq 0}$. From remark \ref{remark: centres_of_dual_toroidal_lie_algebras}, we know that:
                        $$
                            K_{r, s} :=
                            \begin{cases}
                                \text{$\frac1s v^{r - 1} t^s \bar{d}(v)$ if $(r, s) \in \Z \x (\Z \setminus \{0\})$}
                                \\
                                \text{$-\frac1r v^r t^{-1} \bar{d}(t)$ if $(r, s) \in \Z \x \{0\}$}
                                \\
                                \text{$0$ if $(r, s) = (0, 0)$}
                            \end{cases}
                        $$
                    from which one infers that:
                        $$
                            \begin{aligned}
                                & -\sum_{r \in \Z \setminus \{0\}} r h v_1^{-r} \tensor K_{r, 0}
                                \\
                                = & -\sum_{r \in \Z \setminus \{0\}} r h v_1^{-r} \tensor \left( -\frac1r v_2^r t_2^{-1} \bar{d}(t_2) \right)
                                \\
                                = & \sum_{r \in \Z \setminus \{0\}} h v_1^{-r} \tensor v_2^r t_2^{-1} \bar{d}(t_2)
                                \\
                                = & \sum_{r \in \Z \setminus \{0\}} h v_1^{-r} \tensor v_2^r t_2^{-1} \bar{d}(t_2)
                            \end{aligned}
                        $$
                        
                    Next, recall again from remark \ref{remark: centres_of_dual_toroidal_lie_algebras} that:
                        $$(r, s) \in (\Z \setminus \{0\})^2 \implies K_{r, s} = \frac1s v^{r - 1} t^s \bar{d}(v) = -\frac1r v^r t^{s - 1} \bar{d}(t) \in \bar{\Omega}_{[2]}$$
                    and then consider the following:
                        $$
                            \begin{aligned}
                                & -\sum_{(r, s) \in (\Z \setminus \{0\}) \x \Z_{> 0}} \left( K_{r, s} \tensor r h v_2^{-r} t_2^{-s} + r h v_1^{-r} t_1^s \tensor K_{r, -s} \right)
                                \\
                                = & -\sum_{(r, s) \in (\Z \setminus \{0\}) \x \Z_{> 0}} \left( -\frac1r v_1^r t_1^{s - 1} \bar{d}(t_1) \tensor r h v_2^{-r} t_2^{-s} + r h v_1^{-r} t_1^s \tensor \left( -\frac1r v_2^r t_2^{-s - 1} \bar{d}(t_2) \right) \right)
                                \\
                                = & \sum_{(r, s) \in (\Z \setminus \{0\}) \x \Z_{> 0}} \left( v_1^r t_1^{s - 1} \bar{d}(t_1) \tensor h v_2^{-r} t_2^{-s} + h v_1^{-r} t_1^s \tensor v_2^r t_2^{-s - 1} \bar{d}(t_2) \right)
                            \end{aligned}
                        $$
                    wherein we note that for all $s \in \Z_{> 0}$, the summands corresponding to the indices $(0, s)$ vanish.

                    We now have that:
                        $$
                            \begin{aligned}
                                & [\bar{\Delta}(ht), \sfr_{\z_{[2]}^+} + \sfr_{\d_{[2]}^+}]
                                \\
                                = & \sum_{(r, s) \in (\Z \setminus \{0\}) \x \Z_{> 0}} \left( K_{r, s} \tensor r h v_2^{-r} t_2^{-s} + r h v_1^{-r} t_1^s \tensor K_{r, -s} \right) - \sum_{r \in \Z \setminus \{0\}} r h v_1^{-r} \tensor K_{r, 0} + h \tensor c_{t_2}
                                \\
                                = & \sum_{(r, s) \in (\Z \setminus \{0\}) \x \Z_{> 0}} \left( v_1^r t_1^{s - 1} \bar{d}(t_1) \tensor h v_2^{-r} t_2^{-s} + h v_1^{-r} t_1^s \tensor v_2^r t_2^{-s - 1} \bar{d}(t_2) \right) + \sum_{r \in \Z \setminus \{0\}} h v_1^{-r} \tensor v_2^r t_2^{-1} \bar{d}(t_2) + h \tensor t_2^{-1} \bar{d}(t_2)
                                \\
                                = & \sum_{(r, s) \in (\Z \setminus \{0\}) \x \Z_{> 0}} \left( v_1^r t_1^{s - 1} \bar{d}(t_1) \tensor h v_2^{-r} t_2^{-s} + h v_1^{-r} t_1^s \tensor v_2^r t_2^{-s - 1} \bar{d}(t_2) \right) + \sum_{r \in \Z} h v_1^{-r} \tensor v_2^r t_2^{-1} \bar{d}(t_2)
                            \end{aligned}
                        $$
                    where we got the last equality by adding the second and third summands. Before we proceed, let us note that:
                        $$( (\Z \setminus \{0\}) \x \Z_{> 0} ) \cup (\Z \x \{0\}) \cong (\Z \x \Z_{\geq 0}) \setminus (\{0\} \x \Z_{> 0})$$
                    We then get that:
                        $$
                            \begin{aligned}
                                & \sum_{(r, s) \in (\Z \setminus \{0\}) \x \Z_{> 0}} \left( v_1^r t_1^{s - 1} \bar{d}(t_1) \tensor h v_2^{-r} t_2^{-s} + h v_1^{-r} t_1^s \tensor v_2^r t_2^{-s - 1} \bar{d}(t_2) \right) + \sum_{r \in \Z} h v_1^{-r} \tensor v_2^r t_2^{-1} \bar{d}(t_2)
                                \\
                                = & \sum_{(r, s) \in (\Z \x \Z_{\geq 0}) \setminus (\{0\} \x \Z_{> 0})} \left( v_1^r t_1^{s - 1} \bar{d}(t_1) \tensor h v_2^{-r} t_2^{-s} + h v_1^{-r} t_1^s \tensor v_2^r t_2^{-s - 1} \bar{d}(t_2) \right)
                                \\
                                = & \sum_{(r, s) \in \Z \x \Z_{\geq 0}} \left( v_1^r t_1^{s - 1} \bar{d}(t_1) \tensor h v_2^{-r} t_2^{-s} + h v_1^{-r} t_1^s \tensor v_2^r t_2^{-s - 1} \bar{d}(t_2) \right) - \sum_{s \in \Z_{> 0}} \left( t_1^{s - 1} \bar{d}(t_1) \tensor h t_2^{-s} + h t_1^s \tensor t_2^{-s - 1} \bar{d}(t_2) \right)
                                \\
                                = & ( \bar{d}(t_1) \tensor h + h \tensor \bar{d}(t_2) )
                                \left( \sum_{(r, s) \in \Z \x \Z_{\geq 0}} \left( v_1^r t_1^{s - 1} \tensor v_2^{-r} t_2^{-s} + v_1^{-r} t_1^s \tensor v_2^r t_2^{-s - 1} \right) - \sum_{s \in \Z_{> 0}} \left( t_1^{s - 1} \tensor t_2^{-s} + t_1^s \tensor t_2^{-s - 1} \right) \right)
                                \\
                                = & ( \bar{d}(t_1) \tensor h + h \tensor \bar{d}(t_2) ) v_2 \1(v_1, v_2) \1^+(t_1, t_2)
                            \end{aligned}
                        $$

                    We can now add the three components together to yield:
                        $$[\bar{\Delta}(ht), \sfr_{\hat{\g}_{[2]}^+}] = [ \bar{\Delta}(ht), \sfr_{\g_{[2]}^+} + (\sfr_{\z_{[2]}^+} + \sfr_{\d_{[2]}^+}) ] =  [h_1 \tensor 1] v_2 \1(v_1, v_2)$$
                    precisely as claimed. 
                    
                    \item Finally, in order to compute $\hat{\delta}^+(t c_v)$, let us simply note that because:
                        $$\deg t = 1, \deg c_v = -1$$
                    in $\z_{[2]}$ (cf. remark \ref{remark: Z_gradings_on_toroidal_lie_algebras}), we have that:
                        $$\deg t c_v = 0$$
                    and hence:
                        $$\hat{\delta}^+(t c_v) = 0$$
                    per remark \ref{remark: total_degrees_of_classical_yangian_R_matrices}.
                \end{enumerate}
            \end{proof}
        \begin{theorem}[Toroidal Lie algebras as classical limits of affine Yangians] \label{theorem: toroidal_lie_algebras_as_classical_limits_of_affine_yangians}
            The complete topological Lie bialgebra $(\tilde{\g}_{[2]}^+, \tilde{\delta}^+)$ as in lemma \ref{lemma: toroidal_lie_bialgebras} is the classical limit\footnote{\say{Quasi-classical} in the terminologies of \cite{etingof_kazhdan_quantisation_1}.} of the complete topological bialgebra $(\rmY_{\hbar}(\hat{\g}_{[1]}), \Delta_{\hbar})$.
        \end{theorem}
            \begin{proof}
                Clear from lemma \ref{lemma: toroidal_lie_bialgebras} and the construction of the Hopf coproduct on the formal Yangian $\rmY_{\hbar}(\hat{\g}_{[1]})$. 
            \end{proof}
            
        \begin{definition}[Parametrised pseudo-quantisations] \label{def: parametrised_pseudo_quantisations}
            
        \end{definition}
        \begin{theorem}[Parametrised pseudo-classical limits of affine Yangians] \label{theorem: parametrised_pseudo_classical_limits_of_affine_yangians}
            
        \end{theorem}
            \begin{proof}
                
            \end{proof}