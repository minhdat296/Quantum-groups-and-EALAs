\section{The category \texorpdfstring{$\calO$}{} of a Yangian}
    \subsection{Modified Casimir operators and tensor products of representations}
        We introduce here what is usually referred to as the \say{category $\calO$} of the Yangian $\rmY(\g)$ of a symmetrisable Kac-Moody algebra $\g$ (for simplicity, let us assume that its Cartan matrix is indecomposable). Though the definition is natural from the point of view of Lie theory, in practice it is a somewhat \textit{ad hoc} device\footnote{As opposed to the classical category $\calO$ of a finite-dimensional simple Lie algebra, or even the analogue thereof for symmetrisable Kac-Moody algebras (cf. \cite[Chapter 9]{kac_infinite_dimensional_lie_algebras}), which is a natural homological category with various nice properties and is appropriately used for homological purposes.} - at least as far as the structural theory of Yangians attached to symmetrisable Kac-Moody algebras is concerned - as it exists only as a mean for constructing certain tensor products of $\rmY(\g)$-modules, which themselves are in service of the construction of a coproduct on a certain topological completion of $\rmY(\g)$. 

        \begin{convention}
            Suppose for now that $\g$ is a fixed symmetrisable Kac-Moody algebra whose Cartan matrix is indecomposable. 
        \end{convention}

        \begin{definition}[The category $\calO$ of $\rmY(\g)$] \label{def: category_O_of_yangians}
            There is a full subcategory $\calO(\rmY(\g))$ of ${}^l\rmY(\g)\mod$ consisting of objects $V$ which are:
            \begin{itemize}
                \item $\h$-diagonalisable,
                \item with finite-dimensional weight spaces $V_{\mu}$ for every $\mu \in \Pi(V)$, and
                \item such that there exist $\lambda_1, ..., \lambda_k \in \h^*$ such that, if $\mu \in \Pi(V)$ and $V_{\mu} \not \cong 0$ then:
                    $$\lambda_i - \mu \in \Lambda^+$$
                for all $1 \leq i \leq k$.
            \end{itemize}
        \end{definition}
        \begin{remark}
            One easy consequence of definition \ref{def: category_O_of_yangians} (specifically, the last condition) is that, if $V$ is an object of $\calO(\rmY(\g))$ and $\mu \in \h^*$ is some abstract weight, then for any positive (real) root $\alpha \in \Phi^+$, there exists a sufficiently large natural number $N \in \N$ such that:
                $$r \geq N \implies V_{\mu + r \alpha} \cong 0$$
        \end{remark}
        \begin{definition}[Integrable $\rmY(\g)$-modules] \label{def: integrable_modules_over_yangians}
            An object $V$ of $\calO(\rmY(\g))$ is said to be \textbf{integrable} if and only if for every abstract weight $\mu \in \h^*$ and for any positive (real) root $\alpha \in \Phi^+$, there exists a sufficiently large natural number $N \in \N$ such that:
                $$r \geq N \implies V_{\mu \pm r \alpha} \cong 0$$
        \end{definition}

        The following theorem captures the idea that, if $(H, \mu, \eta, \Delta, \e, S)$ is an arbitrary Hopf algebra\footnote{... whose antipode $S$ is always assumed to be invertible.} (say, over $\bbC$) and $V_1, V_2$ are left-$H$-modules via algebra homomorphisms:
            $$\rho_1: H \to \End_{\bbC}(V_1)$$
            $$\rho_2: H \to \End_{\bbC}(V_2)$$
        then via the following composition of algebra homomorphisms, wherein the last map is the natural one:
            $$H \xrightarrow[]{\Delta} H \tensor_{\bbC} H \xrightarrow[\cong]{\id_H \tensor S} H \tensor_{\bbC} H^{\op} \xrightarrow[]{\rho_1 \tensor \rho_2^{\op}} \End_{\bbC}(V_1) \tensor_{\bbC} \End_{\bbC}(V_2) \to \End_{\bbC}(V_1 \tensor_{\bbC} V_2)$$
        one obtains a $H$-module structure on $V_1 \tensor_{\bbC} V_2$ from those on $V_1, V_2$. In fact, for any $\bbC$-algebra $H$, the category ${}^lH\mod$ is monoidal if and only if $H$ is a Hopf algebra, so there being a monoidal structure on the full subcategory $\calO(\rmY(\g)) \subset {}^l\rmY(\g)\mod$ suggests to us that a Hopf structure on $\rmY(\g)$ ought to exist. We shall see that this is only true after a performing a certain topological completion on $\rmY(\g)$, but the original idea merits some discussion nevertheless. 
        \begin{theorem}[Tensor products in $\calO(\rmY(\g))$] \label{theorem: tensor_products_in_the_category_O_of_yangians}
            \begin{enumerate}
                \item \cite[Theorem 4.9]{guay_nakajima_wendlandt_affine_yangian_coproduct} For any pair of objects $V_1, V_2$ of $\calO(\rmY(\g))$, the map:
                    $$\Delta_{V_1, V_2}: \rmY(\g) \to \End_{\bbC}(V_1 \tensor_{\bbC} V_2)$$
                given by:
                    $$\Delta_{V_1, V_2}(E_{i, 0}^{\pm}) = \bar{\Delta}(E_{i, 0}^{\pm})$$
                    $$\Delta_{V_1, V_2}(H_{i, 0}) = \bar{\Delta}(H_{i, 0})$$
                    $$\Delta_{V_1, V_2}(T_{i, 1}) = \bar{\Delta}(T_{i, 1}) + [H_{i, 0} \tensor 1, \sfr_{\g}^+]$$
                is a homomorphism of $\bbC$-algebras. Here, we have that:
                    $$\sfr_{\g}^+ := \sfr_{\h} + \sum_{\alpha \in \Phi^+} \sum_{k = 1}^{\dim_{\bbC} \g_{\alpha}} $$
                \item \cite[Proposition 4.24]{guay_nakajima_wendlandt_affine_yangian_coproduct} Furthermore, tensor products of objects in $\calO(\rmY(\g))$ enjoy associativity, which is in the sense that for every triple of objects $V_1, V_2, V_3$ of $\calO(\rmY(\g))$, then one will have an isomorphism of $\rmY(\g)$-bimodules:
                    $$(V_1 \tensor_{\bbC} V_2) \tensor_{\bbC} V_3 \xrightarrow[]{a_{V_1, V_2, V_3}} V_1 \tensor_{\bbC} (V_2 \tensor_{\bbC} V_3)$$
                that is natural in $V_1, V_2, V_3$.
            \end{enumerate}
            We therefore have a well-defined $\bbC$-bilinear monoidal structure on the category $\calO(\rmY(\g))$. 
        \end{theorem}
    
    \subsection{Completions}
        \begin{convention} \label{conv: canonical_grading_on_yangians}
            Recall also that there is a $\Lambda \x \Z_{\geq 0}$-grading (respectively, filtration) on $\rmY_{\hbar}(\g)$ (respectively, on $\rmY(\g)$) given by:
                $$\deg E_{i, r}^{\pm} := (\pm\height(\alpha_i), r) = (\pm 1, r)$$
                $$\deg H_{i, r} := (0, r)$$
                $$\deg \hbar := (0, 1)$$
            Let us denote the graded components (respectively, filtrants) of $\rmY_{\hbar}(\g)$ (respectively, of $\rmY(\g)$) by:
                $$\rmY_{\hbar}^{i, r}(\g), \rmY^{i, r}(\g)$$

            Now, even though it is not known whether this holds true for general $\g$, let us assume that for every $r \in \Z_{\geq 0}$, every $\bbC$-vector space:
                $$\rmY_{\hbar}(\g)_{\bullet, r}$$
            (respectively, every $\rmY(\g)_{\bullet, r}$) has a \textbf{triangular decomposition}:
                $$\rmY_{\hbar}^{\bullet, r}(\g) \cong \rmY_{\hbar}^{-, r}(\g) \tensor_{\bbC} \rmY_{\hbar}^{0, r}(\g) \tensor_{\bbC} \rmY_{\hbar}^{+, r}(\g)$$
            wherein:
                $$\rmY_{\hbar}(\g)_{\pm, r}, \rmY_{\hbar}(\g)_{0, r}$$
            are respectively the $\bbC$-vector subspaces generated by the elements $E_{i, r}^{\pm}$ and the elements $H_{i, r}$ (and similarly for $\rmY(\g)_{\bullet, r}$). By taking the direct sum $\bigoplus_{r \in \Z_{\geq 0}}$ (respectively, the union $\bigcup_{r \in \Z_{\geq 0}}$), one obtains a triangular decomposition of $\rmY_{\hbar}(\g)$ itself (respectively, of $\rmY(\g)$) into the subalgebras generated by the elements $E_{i, r}^{\pm}, H_{i, r}$, now for \textit{all} $r \in \Z_{\geq 0}$:
                $$\rmY_{\hbar}(\g) \cong \rmY_{\hbar}^-(\g) \tensor_{\bbC} \rmY_{\hbar}^0(\g) \tensor_{\bbC} \rmY_{\hbar}^+(\g)$$
            wherein:
                $$\rmY_{\hbar}^{\pm}(\g) := \bigoplus_{r \in \Z_{\geq 0}} \rmY_{\hbar}^{\pm, r}(\g), \rmY_{\hbar}^0(\g) := \bigoplus_{r \in \Z_{\geq 0}} \rmY_{\hbar}^{0, r}(\g)$$
            (and likewise for $\rmY(\g)$).
        \end{convention}