\section{Extended toroidal Lie (bi)algebras}
    \begin{convention} \label{conv: a_fixed_finite_dimensional_simple_lie_algebra}
        Throughout this section, we fix a finite-dimensional simple Lie algebra $\g$ over $\bbC$, equipped with a symmetric and non-degenerate invariant $\bbC$-bilinear form $(-, -)_{\g}$. It is known that such a bilinear form is unique up to $\bbC^{\x}$-multiples, so for all intents and purposes, it can be assumed to be the Killing form, though this assumption is not necessary. 

        Suppose also that $\g$ is equipped with a basis $\{x_i\}_{1 \leq i \leq \dim_{\bbC} \g}$ and with respect to $(-, -)_{\g}$, we identify a dual basis $\{x_i^*\}_{1 \leq i \leq \dim_{\bbC} \g}$. Recall that the universal Casimir element/canonical element of $\g$ is:
            $$\sfr_{\g} := \sum_{1 \leq i \leq \dim_{\bbC} \g} x_i \tensor x_i^* \in \g \tensor_{\bbC} \g$$
        and recall that $\sfr_{\g}$ is independent of what we choose the basis vectors $x_i$ to be.

        Eventually, we will also be concerned with the Dynkin diagram associated to the root system of $\g$. Let us denote this by:
            $$\Gamma := (\Gamma_0, \Gamma_1)$$
        wherein $\Gamma_0$ means the (finite) set of vertices and $\Gamma_1$ means the set of undirected edges between said vertices. 
    \end{convention}

    \begin{convention}
        Throughout, we shall use $(-)^{\star}$ to denote graded duals. 
    \end{convention}

    \begin{convention}
        If $k$ is a commutative ring and $A$ is a $k$-algebra, and if $L$ is a Lie algebra over $k$, then the default Lie algebra structure on the $k$-module $L \tensor_k A$ shall be the one given by extension of scalars, i.e.:
            $$[x \tensor a, y \tensor b]_{L \tensor_k A} := [x, y]_L \mu_A(a \tensor b)$$
        $L \tensor_k A$ is usually regarded as Lie algebra over $k$ instead of over $A$.  
    \end{convention}

    \begin{convention} \label{conv: multiloop_algebras}
        We fix once and for all the following notations:
        \begin{itemize}
            \item $A_{[n]} := \bbC[v_1^{\pm 1}, ..., v_n^{\pm 1}]$, $A_{[2]}^+ := \bbC[v_1^{\pm 1}, ..., v_n], A_{[n]}^- := v_n^{-1}\bbC[v_1^{\pm 1}, ..., v_n^{-1}]$, and in particular, when $n \leq 2$, let us set $v_1 := v$ and $v_2 := t$ per the usual Yangian conventions (cf. e.g. \cite{wendlandt_formal_shift_operators_on_yangian_doubles});
            \item $\g_{[n]} := \g \tensor_{\bbC} A_{[n]}$, $\g_{[n]}^{\pm} := \g \tensor_{\bbC} A_{[n]}^{\pm}, \tilde{\g}_{[n]}^{\pm} := \uce(\g_{[n]}^{\pm})$ (with $\uce$ meaning \say{universal central extension}\footnote{These are indeed universal \textit{a priori}, thanks to the fact that the Lie algebras $\g_{[n]}, \g_{[n]}^{\pm}$ are perfect.});
            \item $\z_{[n]}^{\pm} := \z(\tilde{\g}_{[n]}^{\pm})$ (cf. remark \ref{remark: centres_of_dual_toroidal_lie_algebras}).
        \end{itemize}
    \end{convention}

    \subsection{A Lie bialgebra structure on the \texorpdfstring{$2$}{}-loop Lie algebra \texorpdfstring{$\g_{[2]}$}{}}
        \begin{definition} \label{def: residue_form_on_loop_algebra}
            For $v$ a formal variable, we can extend the invariant inner product $(-, -)_{\g}$ to the following pairing on $\g_{[1]}$ by defining:
                $$(x f(v), y g(v))_{\g_{[1]}} := (x, y)_{\g} \Res_{v = 0}( v^{-1} f(v) g(v) )$$
            for all $x, y \in \g$ and all $f(v), g(v) \in A_{[1]}$, and recall that:
                $$\Res_{v = 0}\left( \sum_{n \in \Z} a_n v^n \right) := a_{-1}$$
            More algebraically\footnote{So that the definition would work still when we replace $\bbC$ with a general algebraically closed field of characteristic $0$.}, we can define this as:
                $$(x v^m, y v^n)_{\g_{[1]}} := (x, y)_{\g} \delta_{m + n, 0}$$
        \end{definition}
        \begin{definition} \label{def: residue_form_on_multiloop_algebra}
            Now, let $(-, -)_{\g_{[1]}}$ be as in definition \ref{def: residue_form_on_loop_algebra}. This can be extended furthermore to $\g_{[2]}$ in the following manner: for all $X(v), Y(v) \in \g_{[1]}$ and all $f(t), g(t) \in \bbC[t^{\pm 1}]$, define:
                $$(X(v) f(t), Y(v) g(t))_{\g_{[2]}} := -(X(v), Y(v))_{\g_{[1]}} \Res_{t = 0}( f(t) g(t) )$$
            More algebraically, we can define this as:
                $$(X(v) f(t), Y(v) g(t))_{\g_{[2]}} := -(X(v), Y(v))_{\g_{[1]}} \delta_{m + n, -1}$$
            In both cases, the appearance of the minus sign is a crucial choice for our purposes. 
        \end{definition}

        \begin{question} \label{question: multiloop_lie_bialgebras}
            \begin{enumerate}
                \item Verify that $(-, -)_{\g_{[2]}}$ is an invariant and non-degenerate symmetric $\bbC$-bilinear form on $\g_{[2]}$.
                \item Show that by equpping $\g_{[2]}$ with the invariant inner product $(-, -)_{\g_{[2]}}$, the following triple of Lie algebras becomes a well-defined Manin triple:
                    $$(\g_{[2]}, \g_{[2]}^+, \g_{[2]}^-)$$
                \item Find a formula for the canonical element $\sfr_{\g[v^{\pm 1}, t} \in \g_{[2]}^+ \hattensor_{\bbC} \g_{[2]}^+$ with respect to the restriction of $(-, -)_{\g_{[2]}}$ to $\g[v^{\pm 1}, t^{-1}] \x \g_{[2]}^+$.
                \item Find the Lie bialgebra structure on $\g_{[2]}^+$ arising from the Manin triple in 2.
            \end{enumerate}
        \end{question}
            \begin{proof}
                \begin{enumerate}
                    \item The symmetry and bilinearity of $(-, -)_{\g_{[2]}}$ are clear from the construction of this bilinear pairing as in definition \ref{def: residue_form_on_multiloop_algebra}. $(-, -)_{\g_{[2]}}$-invariance follows from the $\g$-invariance of $(-, -)_{\g}$, which is by hypothesis. Finally, non-degeneracy follows from the non-degeneracy of $(-, -)_{\g}$ (also by hypothesis) as well as the non-degeneracy of the residual pairings on $A_{[1]}$ (as in definition \ref{def: residue_form_on_loop_algebra}) and on $\bbC[t^{\pm 1}]$ (as in definition \ref{def: residue_form_on_multiloop_algebra}); to see that the latter point holds, simply note that there exists no $m \in \Z$ such that $\delta_{m + n, - 1} = 0$ (respectively, such that $\delta_{m + n, 0}$) for all $n \in \Z$.
                    \item It is not hard to see that: with respect to $(-, -)_{\g_{[2]}}$ as in 1, one has that:
                        $$(\g_{[2]}^+)^{\star} \cong \bigoplus_{m \in \Z, p \in \Z_{\geq 0}} (\g v^m t^p)^* \cong \bigoplus_{m \in \Z, p \in \Z_{\geq 0}} \g v^{-m} t^{-p - 1} \cong \g_{[2]}^-$$
                    with respect to the invariant inner product $(-, -)_{\g_{[2]}}$. It is also easy to see that:
                        $$\g_{[2]} \cong \g_{[2]}^+ \oplus \g_{[2]}^-$$
                    Note also that $\g_{[2]} \supset \g_{[2]}^+, \g_{[2]}^-$ are Lie subalgebras. Finally, to prove that $(-, -)_{\g_{[2]}}$ pairs the vector subspaces $\g_{[2]}^+, \g_{[2]}^-$ isotropically, simply that there does not exist any $p, q \geq 0$ or $p, q \leq -1$ simultaneously so that:
                        $$\delta_{p + q, -1} = 0$$
                    which means that:
                        $$(\g_{[2]}^-, \g_{[2]}^-)_{\g_{[2]}} = (\g_{[2]}^+, \g_{[2]}^+)_{\g_{[2]}} = 0$$
                    \item It will be convenient for us to make the identification of topological vector spaces:
                        $$\g_{[2]}^+ \hattensor_{\bbC} \g[v^{\pm 1}, t^{-1}] \cong \g[v_2^{\pm 1}, t_1] \hattensor_{\bbC} \g[v^{\pm 1}, t_2^{-1}]$$
                    Also, let us fix the basis:
                        $$\{X_{i, m, p}\}_{1 \leq i \leq \dim_{\bbC} \g, (m, p) \in \Z^2} := \{x_i v^m t^p\}_{1 \leq i \leq \dim_{\bbC} \g, (m, p) \in \Z^2}$$
                    for $\g_{[2]}$. It is easy to see that the graded dual of this basis with respect to the invariant inner product $(-, -)_{\g_{[2]}}$ is:
                        $$\{X_{i, m, p}^{\star}\}_{1 \leq i \leq \dim_{\bbC} \g, (m, p) \in \Z^2} := \{x_i^* v^{-m} t^{-p - 1}\}_{1 \leq i \leq \dim_{\bbC} \g, (m, p) \in \Z^2}$$
                    
                    By definition, the canonical element $\sfr_{\g_{[2]}^+} \in \g[v_2^{\pm 1}, t_1] \hattensor_{\bbC} \g[v^{\pm 1}, t_2^{-1}]$ is given by:
                        $$\sfr_{\g_{[2]}^+} := \sum_{1 \leq i \leq \dim_{\bbC} \g} \sum_{(m, p) \in \Z \x \Z_{\geq 0}} X_{i, m, p} \tensor X_{i, m, p}^{\star}$$
                    As such, we have that:
                        $$
                            \begin{aligned}
                                \sfr_{\g_{[2]}^+} & := \sum_{1 \leq i \leq \dim_{\bbC} \g} \sum_{m \in \Z} \sum_{p \in \Z_{\geq 0}} x_i v_1^m t_1^p \tensor x_i^* v_2^{-m} t_2^{-p - 1}
                                \\
                                & = -\left( \sum_{1 \leq i \leq \dim_{\bbC} \g} x_i \tensor x_i^* \right) \left( v_2 \sum_{m \in \Z} v_1^m v_2^{-m - 1} \right) \left( t_2^{-1} \sum_{p \in \Z_{\geq 0}} (t_1 t_2^{-1})^p \right)
                                \\
                                & = -\sfr_{\g} v_2 \1(v_1, v_2) \1^+(t_1, t_2)
                            \end{aligned}
                        $$
                    wherein:
                        $$\1(z, w) := \sum_{m \in \Z} z^m w^{-m - 1} \in \bbC[\![z^{\pm 1}, w^{\pm 1}]\!]$$ 
                    is the $2$-variable formal Dirac distribution, and we obtained:
                        $$t_2^{-1} \sum_{p \in \Z_{\geq 0}} (t_1 t_2^{-1})^p = \frac{1}{t_2 - t_1} =: \1^+(t_1, t_2)$$
                    simply by formally evaluating the geometric series. 
                    \item Let us keep the identification:
                        $$\g_{[2]}^+ \hattensor_{\bbC} \g[v^{\pm 1}, t^{-1}] \cong \g[v_2^{\pm 1}, t_1] \hattensor_{\bbC} \g[v^{\pm 1}, t_2^{-1}]$$
                    From \cite[pp. 5]{etingof_kazhdan_quantisation_1}\footnote{Actually, this citation is not quite right, since the result was stated for finite-dimensional Manin triples only. However, since we're dealing with graded duals with finite-dimensional graded components, I believe the analogous result still holds. Of course, I should write this down carefully at some point.}, we know that the Lie bialgebra structure (say, $\delta_{\g_{[2]}^+}$) on $\g_{[2]}$ is given at any $X(v, t) \in \g_{[2]}$ by:
                        $$\delta_{\g_{[2]}^+}( X(v, t) ) = [X(v_1, t_1) \tensor 1 + 1 \tensor X(v_2, t_2), \sfr_{\g_{[2]}^+}]$$
                    When $X(v, t) := x v^m t^p$ for some $x \in \g, m \in \Z, t \in \Z_{\geq 0}$, this can written out more explicitly as follows:
                        $$
                            \begin{aligned}
                                \delta_{\g_{[2]}^+}( x v^m t^p ) & = -\left[x v_1^m t_1^p \tensor 1 + 1 \tensor x v_2^m t_2^p, \sfr_{\g} v_2 \1(v_1, v_2) \1^+(t_1, t_2)\right]
                                \\
                                & = -[x \tensor 1 + 1 \tensor x, \sfr_{\g}] \cdot v_1^m t_1^p \cdot v_2^m t_2^p \cdot v_2 \1(v_1, v_2) \1^+(t_1, t_2)
                            \end{aligned}
                        $$
                \end{enumerate}
            \end{proof}

    \subsection{Failing to extend the Lie bialgebra structure to the universal central extension \texorpdfstring{$\tilde{\g}_{[2]}$}{}}
        \begin{question} \label{question: extending_invariant_inner_products_on_multi_loop_to_universal_central_extensions}
            \begin{enumerate}
                \item Prove that there is a unique invariant symmetric bilinear form $(-, -)_{\tilde{\g}_{[2]}}$ on $\t$ whose restriction to $\g_{[2]}$ coincides with $(-, -)_{\g_{[2]}}$.
                \item Find a Lie subalgebra $\tilde{\g}_{[2]}^- \subset \t$ such that:
                    $$\tilde{\g}_{[2]}\cong \tilde{\g}_{[2]}^+\oplus \tilde{\g}_{[2]}^-$$
                and such that $\tilde{\g}_{[2]}^-$ is paired isotropically with $\s$ by $(-, -)_{\tilde{\g}_{[2]}}$. 
                \item Why is the triple:
                    $$(\tilde{\g}_{[2]}, \tilde{\g}_{[2]}^+, \tilde{\g}_{[2]}^-)$$
                with $\tilde{\g}_{[2]}$ being equipped with $(-, -)_{\tilde{\g}_{[2]}}$ not a Manin triple ?
            \end{enumerate}
        \end{question}
            \begin{proof}
                \begin{enumerate}
                    \item Suppose that $B$ is any invariant inner product on $\t$ and fix an element $Z \in \z_{[2]}$. This gives us:
                        $$B([X, Y], Z) = B(X, [Y, Z]) = B(X, 0) = 0$$
                    for all $X, Y \in \t$. As such, the sought-for unique invariant inner product on $\t$ induced by $(-, -)_{\g_{[2]}}$, whose restriction to $\g_{[2]} \subset \t$ coincides with $(-, -)_{\g_{[2]}}$, must be determined by:
                        $$(X, Z)_{\tilde{\g}_{[2]}} = 0, (Z, Z)_{\tilde{\g}_{[2]}} = 0$$
                    for all $X \in \t$ and all $Z \in \z_{[2]}$.
                    \item One thing that we are able to gather from 1 is that, with respect to $(-, -)_{\tilde{\g}_{[2]}}$, the centre $\z_{[2]}$ is orthogonally complementary to $\g_{[2]}$. With this in mind, we claim that:
                        $$\tilde{\g}_{[2]}^- \cong \g_{[2]}^- \oplus \z_{[2]}^-$$
                    wherein $\z_{[2]}^-$ is such that:
                        $$\z_{[2]} \cong \z_{[2]}^+ \oplus \z_{[2]}^-$$
                    and note that $\z_{[2]}^-$ must exist due to $\s$ being a Lie subalgebra of $\t$ and hence $\z_{[2]}^+$ being a Lie subalgebra of $\z_{[2]}$ (namely, one has that $\z_{[2]}^+ = \z_{[2]} \cap \s$). To see that this is indeed that the Lie subalgebra of $\t$ that we are after, firstly note that because:
                        $$(-, -)_{\tilde{\g}_{[2]}}|_{\g_{[2]}} = (-, -)_{\g_{[2]}}$$
                    and because it has been shown that $(-, -)_{\g_{[2]}}$ pairs $\g_{[2]}^+$ and $\g_{[2]}^-$ isotropically as subspaces of $\g_{[2]}$, the only thing to demonstrate is that $(-, -)_{\tilde{\g}_{[2]}}$ pairs elements of $\z_{[2]}^+$ and $\z_{[2]}^-$ isotropically with one another in the sense that:
                        $$(\z_{[2]}^+, \z_{[2]}^+)_{\tilde{\g}_{[2]}} = (\z_{[2]}^-, \z_{[2]}^-)_{\tilde{\g}_{[2]}} = 0$$
                    This is directly due to the fact that elements of $\z_{[2]}^+$ and likewise, those of $\z_{[2]}^-$, are central as elements of $\t$. Lastly, one verifies that, one indeed has that:
                        $$\tilde{\g}_{[2]}^+\oplus \tilde{\g}_{[2]}^- \cong ( \g_{[2]}^+ \oplus \z_{[2]}^+ ) \oplus ( \g_{[2]}^- \oplus \z_{[2]}^- ) \cong \g_{[2]} \oplus \z_{[2]} \cong \t$$
                    \item $(\tilde{\g}_{[2]}, \tilde{\g}_{[2]}^+, \tilde{\g}_{[2]}^-)$ is not a Manin triple (nor a graded Manin triple, for that matter) due to the simple fact that the non-zero vector space $\z_{[2]}$ is contained entirely in $\Rad (-, -)_{\tilde{\g}_{[2]}} := \{Z \in \tilde{\g}_{[2]}\mid \forall X \in \t: (X, Z)_{\tilde{\g}_{[2]}} = 0\}$. This implies that the invariant inner product $(-, -)_{\tilde{\g}_{[2]}}$ on $\t$ is \textit{degenerate}, thereby violating the definition of Manin triples. 

                    Note that we have not even checked whether or not $\tilde{\g}_{[2]}^-$ is actually a Lie subalgebra of $\t$ or merely a vector subspace. This will turn out to be true, but we defer this discussion to question \ref{question: toroidal_dual}. 
                \end{enumerate}
            \end{proof}
        \begin{remark}[What exactly is $\z_{[2]}^-$ ?] \label{remark: centres_of_dual_toroidal_lie_algebras}
            In attempting to answer question \ref{question: extending_invariant_inner_products_on_multi_loop_to_universal_central_extensions}, we relied on the existence of an abstract vector subspace $\z_{[2]}^-$ of $\z_{[2]}$ specified by the condition that:
                $$\z_{[2]} \cong \z_{[2]}^+ \oplus \z_{[2]}^-$$
            Let us now spend a bit of time on giving an explicit description of $\z_{[2]}^-$. 

            Suppose that $k$ is an arbitrary commutative ring. Recall firstly that, should $\a$ be a perfect Lie algebra over $k$ (i.e. a Lie algebra such that $\a = [\a, \a]$) with a non-degenerate invariant inner product $(-, -)_{\a}$, then not only does $\a_A := \a \tensor_k A$ admit a universal central extension $\uce(\a_A)$ for any commutative $k$-algebra $A$ (i.e. one that is initial in the category of all central extensions of $\a$) - and recall also that any universal central extension must split - but also, that there is the following explicit description of $\uce(\a_A)$ due to Kassel\todo{Cite Kassel's paper.}:
                $$\uce(\a_A) \cong \a_A \oplus \bar{\Omega}^1_{A/k}$$
            with $\bar{\Omega}^1_{A/k} := \coim d_{A/k} := \Omega^1_{A/k}/d_{A/k}(A)$ being the coimage of the universal K\"ahler differential map $d_{A/k}: A \to \Omega^1_{A/k}$; if we denote:
                $$
                    \begin{tikzcd}
                    A & {\Omega^1_{A/k}} \\
                    & {\bar{\Omega}^1_{A/k}}
                    \arrow[two heads, from=1-2, to=2-2]
                    \arrow["{d_{A/k}}", from=1-1, to=1-2]
                    \arrow["{\bar{d}_{A/k}}"', from=1-1, to=2-2]
                    \end{tikzcd}
                $$
            then the Lie bracket on $\uce(\a_A)$ with respect to Kassel's realisation shall be given by:
                $$
                    \begin{aligned}
                        [ x \tensor a, y \tensor b ]_{\uce(\a_A)} & = [ X \tensor a, Y \tensor b ]_{\a_A} + (x, y)_{\a} b \bar{d}_{A/k}(a)
                        \\
                        & = [X, Y]_{\a} ab + (x, y)_{\a} a \bar{d}_{A/k}(b)
                    \end{aligned}
                $$
            for all $x, y \in \a$ and all $a, b \in A$.

            We now specialise to the case wherein $k \cong \bbC$, $\a = \g$, and $(-, -)_{\a} = (-, -)_{\g}$ as in convention \ref{conv: a_fixed_finite_dimensional_simple_lie_algebra} and for the moment, let us consider:
                $$A \in \{ A_{[n]}^+ := \bbC[v_1, ..., v_n], A_{[n]}^{\pm} := \bbC[v_1^{\pm 1}, ..., v_n^{\pm 1}] \}$$
            and also, let us abbreviate:
                $$\Omega_{[n]} := \Omega^1_{A_{[n]}/\bbC}, \Omega^{\pm}_{[n]} := \Omega^1_{A_{[n]}^{\pm}/\bbC}$$
                $$\bar{\Omega}_{[n]} := \bar{\Omega}^1_{A_{[n]}/\bbC}, \bar{\Omega}_{[n]}^{\pm} := \bar{\Omega}^1_{A_{[n]}^{\pm}/\bbC}$$
                $$d := d_{A/k}, \bar{d} := \bar{d}_{A/k}$$
            Eventually, we will specialise to the case $n = 2$. \textit{A priori}, both $\Omega_{[n]}$ and $\Omega^{\pm}_{[n]}$ are free and of rank $n$ over $A_{[n]}^+$ and $A_{[n]}^{\pm}$ respectively, specifically generated by the basis elements:
                $$d(v_j)$$
            In turn, this implies that the $A_{[n]}^+$-module $\bar{\Omega}_{[n]}$ and the $A_{[n]}^{\pm}$-module $\bar{\Omega}_{[n]}^{\pm}$ are both generated by the basis elements:
                $$\bar{d}(v_j)$$
            that are subjected to the following relation:
                $$0 = \bar{d}( v_1^{m_1} ... v_n^{m_n} ) = \sum_{1 \leq j \leq n} m_j v_1^{m_1} ... v_j^{m_j - 1} ... v_n^{m_n} \bar{d}(v_j)$$
            From this, one infers that the elements:
                $$m_j^{-1} v_1^{m_1} ... v_j^{m_j - 1} ... v_n^{m_n} \bar{d}(v_j)$$
            form a basis for $\bar{\Omega}^+_{[n]}$ and $\bar{\Omega}_{[n]}^{\pm}$ as $\bbC$-vector spaces. 

            When $n = 2$, we can write things out more explicitly: $\z_{[2]} \cong \bar{\Omega}_{[2]}$ now decomposes as a $\bbC$-vector space in the following manner:
                $$\z_{[2]} \cong ( \bigoplus_{(r, s) \in \Z^2} \bbC K_{r, s}) \oplus \bbC c_v \oplus \bbC c_t$$
            and $\z_{[2]}^+ \cong \bar{\Omega}_{[2]}^+$ decomposes in the following manner:
                $$\z_{[2]}^+ \cong ( \bigoplus_{(r, s) \in \Z \x \Z_{> 0}} \bbC K_{r, s}) \oplus \bbC c_v$$
            wherein:
                $$
                    K_{r, s} :=
                    \begin{cases}
                        \text{$\frac1s v^{r - 1} t^s \bar{d}(v)$ if $(r, s) \in \Z \x (\Z \setminus \{0\})$}
                        \\
                        \text{$-\frac1r v^r t^{-1} \bar{d}(t)$ if $(r, s) \in (\Z \setminus \{0\}) \x \{0\}$}
                        \\
                        \text{$0$ if $(r, s) = (0, 0)$}
                    \end{cases}
                $$
                $$c_v := v^{-1} \bar{d}(v), c_t := t^{-1} \bar{d}(t)$$
            In fact, any element of the form:
                $$v^m t^p \bar{d}(v^n t^q) \in \z_{[2]}$$
            can be written in terms of the basis vectors $K_{r, s}, c_v, c_t$ in the following manner:
                $$v^m t^p \bar{d}(v^n t^q) = \delta_{(m, p) + (n, q), (0, 0)} ( n c_v + q c_t ) + (np - mq) K_{m + n, p + q}$$

            From the above and from the requirement on $\z_{[2]}^-$ that:
                $$\z_{[2]} \cong \z_{[2]}^+ \oplus \z_{[2]}^-$$
            one sees immediately that:
                $$\z_{[2]}^- \cong ( \bigoplus_{(r, s) \in \Z \x \Z_{\leq 0}} \bbC K_{r, s}) \oplus \bbC c_t$$
        \end{remark}
        \begin{question} \label{question: toroidal_dual}
            Verify that $\tilde{\g}_{[2]}^-$ is a well-defined Lie subalgebra of $\t$.
        \end{question}
            \begin{proof}
                We now know that:
                    $$\tilde{\g}_{[2]}^- \cong \g_{[2]}^- \oplus \left( ( \bigoplus_{(r, s) \in \Z \x \Z_{\leq 0}} \bbC K_{r, s}) \oplus \bbC c_t \right)$$
                (with notations as in remark \ref{remark: centres_of_dual_toroidal_lie_algebras}), so the verification can be carried out by firstly considering the following, for any $X(v, t), Y(v, t) \in \g_{[2]}^-$ and any $Z, Z' \in \z_{[2]}^-$:
                    $$
                        \begin{aligned}
                            [ X(v, t) + Z, Y(v, t) + Z' ]_{\tilde{\g}_{[2]}} & = [ X(v, t), Y(v, t) ]_{\tilde{\g}_{[2]}} + [ Z, Y(v, t) ]_{\tilde{\g}_{[2]}} + [X(v, t) + Z, Z']_{\tilde{\g}_{[2]}}
                            \\
                            & = [ X(v, t), Y(v, t) ]_{\tilde{\g}_{[2]}}
                        \end{aligned}
                    $$
                wherein the equalities hold thanks to the elements $Z, Z'$ being central inside $\t$, and then, without any loss of generality, we consider secondly the following for:
                    $$X(v, t) := x f(v, t), Y(v, t) := y g(v, t)$$
                for some $x, y \in \g$ and $f(v, t), g(v, t) \in t^{-1}\bbC[v^{\pm 1}, t^{-1}]$:
                    $$
                        \begin{aligned}
                            [ X(v, t), Y(v, t) ]_{\tilde{\g}_{[2]}} & = [ x f(v, t), y g(v, t) ]_{\tilde{\g}_{[2]}}
                            \\
                            & = [x, y]_{\g} f(v, t) g(v, t) + (x, y)_{\g} f(v, t) \bar{d}( g(v, t) )
                        \end{aligned}
                    $$
                This is clearly an element of $\t$, in light of how the Lie bracket $[-, -]_{\tilde{\g}_{[2]}}$ is given, so we are done. 
            \end{proof}

        \begin{remark}[The $\Z$-grading on $\tilde{\g}_{[2]}$] \label{remark: Z_gradings_on_toroidal_lie_algebras}
            Throughout, the $\Z$-grading on $\tilde{\g}_{[2]}$ as well as those on its Lie subalgebras $\tilde{\g}_{[2]}^{\pm}$ will play crucial roles in many computations that we will end up performing (cf. remark \ref{remark: total_degrees_of_classical_yangian_R_matrices} and theorem \ref{theorem: toroidal_lie_bialgebras} in particular), so let us spend some time describing it in details before moving on.

            If $k$ is an arbitrary commutative ring and $A$ is a $\Z$-graded commutative $k$-algebra, say:
                $$A := \bigoplus_{n \in \Z} A_n$$
            and if $\a$ is a perfect Lie algebra over $k$, then $\a_A$ will also be $\Z$-graded, specifically in the following manner:
                $$\a_A := \a \tensor_k A \cong \bigoplus_{n \in \Z} \a \tensor_k A_n$$
            and for convenience, let us write $\a_{A_n} := \a \tensor_k A_n$ for each $n \in \Z$. This grading on $\a_A$ actually extends to the whole of $\uce(\a_A)$, though to be able to describe this extension in details, let us firstly how the $A$-module $\Omega^1_{A/k}$ itself is constructed. To this end, recall that the $A$-module $\Omega^1_{A/k}$ is generated by the set:
                $$\{d_{A/k}(a)\}_{a \in A}$$
            subjected to the relations:
                $$d_{A/k}(ab) - a d_{A/k}(b) - d_{A/k}(a) b = 0$$
            defined for all $a, b \in A$. From this, one infers that there is an induced $\Z$-grading on $\Omega^1_{A/k}$ given by:
                $$\deg d_{A/k}(ab) = \deg a d_{A/k}(b) = \deg d_{A/k}(a) b = \deg a + \deg b - 1$$
            for all $a, b \in A$. Inside $\Omega^1_{A/k}$, now viewed as a $k$-module, one has the $k$-submodule $\im d_{A/k}$, which is also $\Z$-graded: the grading is given like above, namely:
                $$\deg d(a) = \deg a - 1$$
            This $\Z$-grading induces another one on $\bar{\Omega}^1_{A/k}$, given by:
                $$\deg \bar{d}_{A/k}(ab) = \deg a \bar{d}_{A/k}(b) = \deg \bar{d}_{A/k}(a) b = \deg a + \deg b - 1$$
            for all $a, b \in A$.

            Now, let us focus once more on the case:
                $$A := A_{[2]}$$
            (cf. remark \ref{remark: centres_of_dual_toroidal_lie_algebras}) wherein the relevant $\Z$-grading is given by:
                $$\deg v := 0, \deg t := 1$$
            Since we know that the basis elements of $\z_{[2]}$ are given by:
                $$
                    K_{r, s} :=
                    \begin{cases}
                        \text{$\frac1s v^{r - 1} t^s \bar{d}(v)$ if $(r, s) \in \Z \x (\Z \setminus \{0\})$}
                        \\
                        \text{$-\frac1r v^r t^{-1} \bar{d}(t)$ if $(r, s) \in (\Z \setminus \{0\}) \x \{0\}$}
                        \\
                        \text{$0$ if $(r, s) = (0, 0)$}
                    \end{cases}
                $$
                $$c_v := v^{-1} \bar{d}(v), c_t := t^{-1} \bar{d}(t)$$
            (cf. \textit{loc. cit.}) their respective degrees with respect to the $\Z$-grading on $\bar{\Omega}_{[2]} \cong \z_{[2]}$ are:
                $$
                    \deg K_{r, s} =
                    \begin{cases}
                        \text{$s - 1$ if $(r, s) \in \Z \x (\Z \setminus \{0\})$}
                        \\
                        \text{$-1$ if $(r, s) \in (\Z \setminus \{0\}) \x \{0\}$}
                        \\
                        \text{$0$ if $(r, s) = (0, 0)$}
                    \end{cases}
                $$
                $$\deg c_v = \deg c_t = -1$$
        \end{remark}

    \subsection{Extending \texorpdfstring{$\tilde{\g}_{[2]}$}{} to fix degeneracy} \label{subsection: extended_toroidal_lie_algebras}
        We now attempt to fix the issue whereby any bilinear form on $\tilde{\g}_{[2]}:= \uce(\g_{[2]})$ is necessarily degenerate. We do this by formally introducing a \say{complementary} vector space $\d_{[2]}$ whose elements shall pair non-degenerately with those of $\z_{[2]}$. 
        \begin{convention} \label{conv: orthogonal_complement_of_toroidal_centres}
            From now on, $\d_{[2]}$ shall be the $\bbC$-vector space:
                $$\d_{[2]} \cong ( \bigoplus_{(r, s) \in \Z^2} \bbC D_{r, s} ) \oplus \bbC D_v \oplus \bbC D_t$$
            such that we can endow:
                $$\hat{\g}_{[2]} := \tilde{\g}_{[2]}\oplus \d_{[2]}$$
            with a $\bbC$-bilinear form $(-, -)_{\hat{\g}_{[2]}}$ such that:
            \begin{itemize}
                \item the elements $D_{r, s}, D_v, D_t$ are graded-dual with respect to $(-, -)_{\hat{\g}_{[2]}}$ to the elements $K_{r, s}, c_v, c_t$, respectively;
                \item $(\g_{[2]}, \z_{[2]} \oplus \d_{[2]})_{\hat{\g}_{[2]}} := 0$;
                \item $(\z_{[2]}, \z_{[2]})_{\hat{\g}_{[2]}} = (\d_{[2]}, \d_{[2]})_{\hat{\g}_{[2]}} := 0$;
                \item $(-, -)_{\hat{\g}_{[2]}}|_{\Sym^2_{\bbC}(\g_{[2]})} := (-, -)_{\g_{[2]}}$
            \end{itemize}
        \end{convention}
        \begin{convention}
            Let us assume also that, should there be a Lie algebra structure $[-, -]_{\hat{\g}_{[2]}}$ on $\hat{\g}_{[2]}$ with respect to which $\t$ becomes a Lie subalgebra of $\hat{\g}_{[2]}$, i.e.
                $$[-, -]_{\hat{\g}_{[2]}}|_{\bigwedge^2 \tilde{\g}_{[2]}} := [-, -]_{\tilde{\g}_{[2]}}$$
            then the bilinear form $(-, -)_{\hat{\g}_{[2]}}$ will be \textit{invariant} with respect to $[-, -]_{\hat{\g}_{[2]}}$.

            Even though this will turn out to be the case, we do not assume from the beginning that:
                $$\hat{\g}_{[2]} \cong \tilde{\g}_{[2]}\rtimes \d_{[2]}$$
            i.e. we do not need the assumption that $\t$ is a $\d_{[2]}$-module. This is because \textit{a priori}, we have no knowledge of the Lie algebra structure on $\d_{[2]}$.
        \end{convention}

        \begin{remark}[How does $\d_{[2]}$ act on $\g_{[2]}$ ?] \label{remark: derivation_action_on_multiloop_algebras}
            Let us firstly see how elements of $\d_{[2]}$ might act on those of $\g_{[2]}$, with respect to some Lie bracket $[-, -]_{\hat{\g}_{[2]}}$. 

            To this end, fix $x, y \in \g$, $(m, p), (n, q) \in \Z^2$, along with some $D \in \d_{[2]}$, and then consider the following:
                $$
                    \begin{aligned}
                        ( D, [x v^m t^p, y v^n t^q]_{\tilde{\g}_{[2]}} )_{\hat{\g}_{[2]}} & = ( D, [x, y]_{\g} v^{m + n} t^{p + q} + (x, y)_{\g} v^m t^p \bar{d}( v^n t^q ) )_{\hat{\g}_{[2]}}
                        \\
                        & = (x, y)_{\g} ( D, v^m t^p \bar{d}( v^n t^q ) )_{\hat{\g}_{[2]}}
                        \\
                        & = (x, y)_{\g} ( D, \delta_{(m, p) + (n, q), (0, 0)} ( n c_v + q c_t ) + (np - mq) K_{m + n, p + q} )_{\hat{\g}_{[2]}}
                    \end{aligned}
                $$
            Now, without any loss of generality, let us suppose that $D \in \d_{[2]}$ is some basis element, i.e.:
                $$D \in \{ D_{r, s}, D_v, D_t \}$$
            and consider these cases separately, for the sake of clarity:
            \begin{enumerate}
                \item \textbf{(Case 1: $D := D_{r, s}$):} Fix some $(r, s) \in \Z^2$ and consider the following: 
                    $$
                        \begin{aligned}
                            ( D_{r, s}, [x v^m t^p, y v^n t^q]_{\tilde{\g}_{[2]}} )_{\hat{\g}_{[2]}} & = (x, y)_{\g} ( D_{r, s}, \delta_{(m, p) + (n, q), (0, 0)} ( n c_v + q c_t ) + (np - mq) K_{m + n, p + q} )_{\hat{\g}_{[2]}}
                            \\
                            & = (x, y)_{\g} (np - mq) \delta_{(r, s), (m + n, p + q)}
                        \end{aligned}
                    $$
                The assumption that $(-, -)_{\hat{\g}_{[2]}}$ is invariant with respect to $[-, -]_{\hat{\g}_{[2]}}$ then implies that:
                    $$( [D_{r, s}, x v^m t^p]_{\hat{\g}_{[2]}}, y v^n t^q )_{\hat{\g}_{[2]}} = (x, y)_{\g} (np - mq) \delta_{(r, s), (m + n, p + q)}$$
                Now, suppose that:
                    $$[D_{r, s}, x v^m t^p]_{\hat{\g}_{[2]}} := \sum_{(a, b) \in \Z^2} \lambda_{a, b}(x) v^a t^b + K_{(m, p), (r, s)}(x) + \xi_{(m, p), (r, s)}(x)$$
                for some $\lambda_{a, b}(x) \in \g$, $K_{(m, p), (r, s)}(x) \in \z_{[2]}$, and $\xi_{(m, p), (r, s)}(x) \in \d_{[2]}$, depending on our choices of $x \in \g$ and $(m, p) \in \Z^2$. Next, consider the following:
                    $$
                        \begin{aligned}
                            ( [D_v, x v^m t^p]_{\hat{\g}_{[2]}}, y v^n t^q )_{\hat{\g}_{[2]}} & = \left( \sum_{(a, b) \in \Z^2} \lambda_{a, b}(x) v^a t^b + K_{(m, p), (r, s)}(x) + \xi_{(m, p), (r, s)}(x), y v^n t^q \right)_{\hat{\g}_{[2]}}
                            \\
                            & = \sum_{(a, b) \in \Z^2} \left( \lambda_{a, b}(x) v^a t^b, y v^n t^q \right)_{\g_{[2]}}
                            \\
                            & = -\sum_{(a, b) \in \Z^2} (\lambda_{a, b}(x), y)_{\g} \delta_{ (a, b) + (n, q), (0, -1) }
                            \\
                            & = -(\lambda_{-n, -q - 1}(x), y)_{\g}
                        \end{aligned}
                    $$
                which tells us that:
                    $$(x, y)_{\g} (np - mq) \delta_{(r, s), (m + n, p + q)} = -(\lambda_{-n, -q - 1}(x), y)_{\g}$$
                The non-degeneracy of the inner product $(-, -)_{\g}$ as well as the arbitrariness of the choices of $y \in \g$ and $(n, q) \in \Z^2$ then together imply that:
                    $$\lambda_{-n, -q - 1}(x) = -(np - mq) \delta_{(r, s), (m + n, p + q)} = (mq - np) \delta_{(r, s), (m + n, p + q)}$$
                for any fixed choices of $x \in \g$ and $(m, p) \in \Z^2$. From this, we infer that:
                    $$
                        \begin{aligned}
                            [D_{r, s}, x v^m t^p]_{\hat{\g}_{[2]}} & = \sum_{(n, q) \in \Z^2} -(np - mq) \delta_{(r, s), (m + n, p + q)} v^{-n} t^{-q - 1} + K_{(m, p), (r, s)}(x) + \xi_{(m, p), (r, s)}(x)
                            \\
                            & = ( m(s - p) - (r - m)p ) x v^{m - r} t^{p - s - 1} + K_{(m, p), (r, s)}(x) + \xi_{(m, p), (r, s)}(x)
                            \\
                            & = ( ms - rp ) x v^{m - r} t^{p - s - 1} + K_{(m, p), (r, s)}(x) + \xi_{(m, p), (r, s)}(x)
                        \end{aligned}
                    $$
                    
                We now claim that:
                    $$\xi_{(m, p), (r, s)}(x) = 0$$
                To this end, consider firstly the following, wherein $Z \in \z_{[2]}$ is an arbitrary choice:
                    $$
                        \begin{aligned}
                            ( [D_{r, s}, x v^m t^p]_{\hat{\g}_{[2]}}, Z )_{\hat{\g}_{[2]}} & = ( D_{r, s}, [x v^m t^p, Z]_{\tilde{\g}_{[2]}} )_{\hat{\g}_{[2]}}
                            \\
                            & = (D, 0)_{\hat{\g}_{[2]}}
                            \\
                            & = 0
                        \end{aligned}
                    $$
                Simultaneously, consider the following:
                    $$
                        \begin{aligned}
                            ( [D_{r, s}, x v^m t^p]_{\hat{\g}_{[2]}}, Z )_{\hat{\g}_{[2]}} & = \left( \sum_{(a, b) \in \Z^2} \lambda_{a, b}(x) v^a t^b + K_{(m, p), (r, s)}(x) + \xi_{(m, p), (r, s)}(x), Z \right)_{\hat{\g}_{[2]}}
                            \\
                            & = ( \xi_{(m, p), (r, s)}(x), Z )_{\hat{\g}_{[2]}}
                        \end{aligned}
                    $$
                The previous observation along with this one imply that:
                    $$( \xi_{(m, p), (r, s)}(x), Z )_{\hat{\g}_{[2]}} = 0$$
                for \textit{any} $Z \in \z_{[2]}$, but since $\d_{[2]}$ is graded-dual to $\z_{[2]}$ by construction, the above implies via the non-degeneracy of the inner product $(-, -)_{\hat{\g}_{[2]}}$ that:
                    $$\xi_{(m, p), (r, s)}(x) = 0$$
                necessarily. 

                We can now conclude that:
                    $$[D_{r, s}, x v^m t^p]_{\hat{\g}_{[2]}} = ( rp - ms ) x v^{m - r} t^{p - s - 1} + K_{(m, p), (r, s)}(x)$$
                \item \textbf{(Case 2: $D := D_v$):} In this case, it is easy to see that:
                    $$
                        \begin{aligned}
                            ( D_v, [x v^m t^p, y v^n t^q]_{\tilde{\g}_{[2]}} )_{\hat{\g}_{[2]}} & = (x, y)_{\g} ( D_v, \delta_{(m, p) + (n, q), (0, 0)} ( n c_v + q c_t ) + (np - mq) K_{m + n, p + q} )_{\hat{\g}_{[2]}}
                            \\
                            & = (x, y)_{\g} \delta_{(m, p) + (n, q), (0, 0)} n
                        \end{aligned}
                    $$
                Using invariance, we then see that:
                    $$( [D_v, x v^m t^p]_{\hat{\g}_{[2]}}, y v^n t^q )_{\hat{\g}_{[2]}} = (x, y)_{\g} \delta_{(m, p) + (n, q), (0, 0)} n$$
                Now, suppose that:
                    $$[D_v, x v^m t^p]_{\hat{\g}_{[2]}} := \sum_{(a, b) \in \Z^2} \lambda_{a, b}(x) v^a t^b + K_{m, p}(x) + \xi_{m, p}(x)$$
                for some $\lambda_{a, b}(x) \in \g$, $K_{m, p}(x) \in \z_{[2]}$, and $\xi_{m, p}(x) \in \d_{[2]}$, depending on our choices of $x \in \g$ and $(m, p) \in \Z^2$. Then, consider the following:
                    $$
                        \begin{aligned}
                            ( [D_v, x v^m t^p]_{\hat{\g}_{[2]}}, y v^n t^q )_{\hat{\g}_{[2]}} & = \left( \sum_{(a, b) \in \Z^2} \lambda_{a, b}(x) v^a t^b + K_{m, p}(x) + \xi_{m, p}(x), y v^n t^q \right)_{\hat{\g}_{[2]}}
                            \\
                            & = \sum_{(a, b) \in \Z^2} \left( \lambda_{a, b}(x) v^a t^b, y v^n t^q \right)_{\g_{[2]}}
                            \\
                            & = -\sum_{(a, b) \in \Z^2} (\lambda_{a, b}(x), y)_{\g} \delta_{ (a, b) + (n, q), (0, -1) }
                            \\
                            & = -(\lambda_{-n, -q - 1}(x), y)_{\g}
                        \end{aligned}
                    $$
                From this, we are able to conclude that:
                    $$(x, y)_{\g} \delta_{(m, p) + (n, q), (0, 0)} n = -(\lambda_{-n, -q - 1}(x), y)_{\g}$$
                As this holds for all $y \in \g$ and all $(n, q) \in \Z^2$, we can infer from the above and from the non-degeneracy of the inner product $(-, -)_{\g}$ that:
                    $$\lambda_{-n, -q - 1}(x) = \delta_{(m, p) + (n, q), (0, 0)} n x$$
                for any $x \in \g$ and any $(m, p) \in \Z^2$ (both fixed!), and hence:
                    $$
                        \begin{aligned}
                            [D_v, x v^m t^p]_{\hat{\g}_{[2]}} & = \sum_{(n, q) \in \Z^2} \delta_{(m, p) + (n, q), (0, 0)} n x v^{-n} t^{-q - 1} + K_{m, p}(x) + \xi_{m, p}(x)
                            \\
                            & = -m x v^m t^{p - 1} + K_{m, p}(x) + \xi_{m, p}(x)
                        \end{aligned}
                    $$

                Now, by arguing as in \textbf{Case 1}, we will see that:
                    $$\xi_{m, p}(x) = 0$$
                and afterwards we will be able to conclude that:
                    $$[D_v, x v^m t^p]_{\hat{\g}_{[2]}} = -m x v^m t^{p - 1} + K_{m, p}(x)$$
                \item \textbf{(Case 3: $D := D_t$)} Arguing as when $D = D_v$, we will obtain:
                    $$[D_t, x v^m t^p]_{\hat{\g}_{[2]}} = -p x v^m t^{p - 1} + K_{m, p}(x)$$
                for some $K_{m, p}(x) \in \z_{[2]}$.
            \end{enumerate}
        \end{remark}
        \begin{remark}
            We see now also that $\tilde{\g}_{[2]}$ is a Lie algebra ideal of $\hat{\g}_{[2]}$ with respect to $[-, -]_{\hat{\g}_{[2]}}$.
        \end{remark}

        \begin{remark}[$\d_{[2]}$ acts by derivations] \label{remark: dual_of_toroidal_centres_contains_derivations}
            \begin{enumerate}
                \item We can identify the derivations $D_{r, s}, D_v, D_t$ explicitly in terms of $\del_v := \frac{\del}{\del v}, \del_t := \frac{\del}{\del t}$. For this, let us firstly equip $\der_{\bbC}(A_{[2]})$ - the $\bbC$-vector space of all $\bbC$-linear derivations on $A_{[2]}$ - with the following basis:
                    $$\{ v^m t^p \del_v, v^n t^q \del_t \}_{(m, p), (n, q) \in \Z^2}$$
                \begin{enumerate}
                    \item To compute $D_{r, s}$ in terms of $\del_v, \del_t$, suppose firstly that:
                        $$D_{r, s} := f(v, t) \del_v + g(v, t) \del_t$$
                    with $f(v, t), g(v, t) \in A_{[2]}$. Next, fix some $(m, p) \in \Z^2$ and then consider the following:
                        $$
                            \begin{aligned}
                                D_{r, s}( v^m t^p ) & = f(v, t) \del_v( v^m t^p ) + g(v, t) \del_t( v^m t^p )
                                \\
                                & = f(v, t) m v^{m - 1} t^p + g(v, t) p v^m t^{p - 1}
                            \end{aligned}
                        $$
                    At the same time, we also have that:
                        $$D_{r, s}(v^m t^p) := ( ms - rp ) v^{m - r} t^{p - s - 1}$$
                    and hence:
                        $$f(v, t) m v^{m - 1} t^p + g(v, t) p v^m t^{p - 1} = ( ms - rp ) v^{m - r} t^{p - s - 1}$$
                    From this, one infers that:
                        $$f(v, t) = s v^{-r + 1} t^{-s - 1}, g(v, t) = -r v^{-r} t^{-s}$$
                    and therefore:
                        $$D_{r, s} = s v^{-r + 1} t^{-s - 1} \del_v - r v^{-r} t^{-s} \del_t$$
                    \item One easily checks that:
                        $$D_v = -v t^{-1} \del_v$$
                    \item Likewise:
                        $$D_t = -\del_t$$
                \end{enumerate}
                Consequently, we see that elements of $\d_{[2]}$ are derivations on $A_{[2]}$.
    
                \item Now that we know that the basis elements $D_{r, s}, D_v, D_t \in \d_{[2]}$ are actually certain derivations on $A_{[2]}$, we can also check that the commutators of the elements $D_{r, s}, D_v, D_t$ are still elements of $\d_{[2]}$. This ensures us that we can \textit{choose} to endow $\d_{[2]}$ with the structure of a Lie subalgebra of $\der_{\bbC}(A_{[2]})$, i.e. the Lie algebra structure such that:
                    $$[D, D']_{\hat{\g}_{[2]}} \in \d_{[2]}$$
                for any $D, D' \in \d_{[2]}$. In general, however, we are only guaranteed that:
                    $$[D, D']_{\hat{\g}_{[2]}} = \z_{[2]} \oplus \d_{[2]}$$
                We note also that it is not even guaranteed \textit{a priori} that the $\d_{[2]}$-summand of the commutators of the form $[D, D']_{\hat{\g}_{[2]}}$ has to be the usual commutator inherited from $\der_{\bbC}(A_{[2]})$; this turns out to be true, but is a somewhat non-trivial fact (cf. proposition \ref{prop: lie_bracket_on_orthogonal_complement_of_toroidal_centre}). 
            \end{enumerate}
        \end{remark}
        
        \begin{remark}[How does $\d_{[2]}$ act on $\z_{[2]}$ ?] \label{remark: derivation_action_on_toroidal_centres}
            We can now use what we know about how $\d_{[2]}$ acts on $\g_{[2]}$ in conjunction with the Jacobi identity in order to compute Lie brackets of the form:
                $$[D, Z]_{\hat{\g}_{[2]}}$$
            for any $D \in \d_{[2]}$ and any $Z \in \z_{[2]}$. 
        
            \begin{enumerate}
                \item For the computations that follow, \textit{we will need to assume that}:
                    $$\hat{\g}_{[2]} \cong \tilde{\g}_{[2]}\rtimes \d_{[2]}$$
                in which case it can be easily shown that:
                    $$[\d_{[2]}, \g_{[2]}]_{\hat{\g}_{[2]}} \subseteq \g_{[2]}$$
                Different methods will have to be employed in the absence of this assumption.
    
                The idea here is to use the fact that:
                    $$[-, -]_{\tilde{\g}_{[2]}} = [-, -]_{\g_{[2]}} + \e$$
                wherein $\e: \bigwedge^2 \g_{[2]} \to \z_{[2]}$ is given by:
                    $$\e(x v^m t^p, y v^n t^q) := (x, y)_{\g} v^m t^p \bar{d}(v^n t^q)$$
                for any $x, y \in \g$ and any $(m, p), (n, q) \in \Z^2$. With this in mind, consider the following for any $D \in \d_{[2]}$, which holds thanks to the Jacobi identity:
                    $$
                        \begin{aligned}
                            & [ D, [x v^m t^p, y v^n t^q]_{\tilde{\g}_{[2]}} ]_{\hat{\g}_{[2]}}
                            \\
                            = & [ [D, x v^m t^p]_{\hat{\g}_{[2]}}, y v^n t^q ]_{\tilde{\g}_{[2]}} + [ x v^m t^p, [D, y v^n t^q]_{\hat{\g}_{[2]}} ]_{\tilde{\g}_{[2]}}
                            \\
                            = & [x D(v^m t^p), y v^n t^q]_{\hat{\g}_{[2]}} + [x v^m t^p, y D(v^n t^q)]_{\hat{\g}_{[2]}}
                            \\
                            = & [x, y]_{\g} (D(v^m t^p) v^n t^q + v^m t^p D(v^n t^q)) + (x, y)_{\g} ( D(v^m t^p) \bar{d}(v^n t^q) + v^m t^p \bar{d}(D(v^n t^q)) )
                            \\
                            = & [x, y]_{\g} D(v^{m + n} t^{p + q}) + (x, y)_{\g} ( D(v^m t^p) \bar{d}(v^n t^q) + v^m t^p \bar{d}(D(v^n t^q)) )
                        \end{aligned}
                    $$
                Note that for the second equality, we implicitly invoked the fact that the basis elements of $\d_{[2]}$ (and hence all elements thereof) act on $\g_{[2]}$ in the following manner:
                    $$[D, x v^m t^p] = x D(v^m t^p), D \in \{D_{r, s}\}_{(r, s) \in \Z^2} \cup \{D_v, D_t\}$$
                (cf. remark \ref{remark: derivation_action_on_multiloop_algebras}), and it makes sense to write this because we now know (after remark \ref{remark: dual_of_toroidal_centres_contains_derivations}) that elements of $\d_{[2]}$ are certain derivations on $A_{[2]}$. At the same time, we have that:
                    $$
                        \begin{aligned}
                            [ D, [x v^m t^p, y v^n t^q]_{\tilde{\g}_{[2]}} ]_{\hat{\g}_{[2]}} & = [ D, [x, y]_{\g} v^{m + n} t^{p + q} + (x, y)_{\g} v^m t^p \bar{d}(v^n t^q) ]_{\hat{\g}_{[2]}}
                            \\
                            & = [x, y]_{\g} D(v^{m + n} t^{p + q}) + (x, y)_{\g} [D, v^m t^p \bar{d}(v^n t^q)]_{\hat{\g}_{[2]}}
                        \end{aligned}
                    $$
                By putting the two computations together, one yields:
                    $$[D, v^m t^p \bar{d}(v^n t^q)]_{\hat{\g}_{[2]}} = D(v^m t^p) \bar{d}(v^n t^q) + v^m t^p \bar{d}(D(v^n t^q))$$
                Since we know how the basis elements of $\d_{[2]}$ act on $A_{[2]}$ (cf. remark \ref{remark: dual_of_toroidal_centres_contains_derivations}), the above is enough to determine how $\d_{[2]}$ acts on $\z_{[2]}$. 

                Let us also note the similarity between the formulae:
                    $$[D, f \bar{d}(g)]_{\hat{\g}_{[2]}} = D(f) \bar{d}(g) + f \bar{d}(D(g))$$
                and those for Lie derivatives.
                
                \item Now, let us \textit{not} assume that:
                    $$\hat{\g}_{[2]} \cong \tilde{\g}_{[2]}\rtimes \d_{[2]}$$
                i.e. that $\tilde{\g}_{[2]}$ might not be a $\d_{[2]}$-module from the start. Without any loss of generality, let us consider the following for any $h, h' \in \h$ so that\footnote{We can make this assumption because ultimately, elements of $\z_{[2]}$ do not depend on those of $\g$.}:
                    $$(h, h')_{\g} = 1$$
                any $f(v, t), g(v, t) \in A_{[2]}$, and any $D \in \d_{[2]}$:
                    $$[ D, [h f(v, t), h' g(v, t)]_{\tilde{\g}_{[2]}} ]_{\hat{\g}_{[2]}} = [ D, f(v, t) \bar{d}( g(v, t) ) ]_{\hat{\g}_{[2]}}$$
                At the same time, we have via the Jacobi identity that:
                    $$
                        \begin{aligned}
                            [ D, [h f(v, t), h' g(v, t)]_{\tilde{\g}_{[2]}} ]_{\hat{\g}_{[2]}} & = [ h f(v, t), [D, h' g(v, t)]_{\hat{\g}_{[2]}} ]_{\tilde{\g}_{[2]}} + [ [D, h f(v, t)]_{\hat{\g}_{[2]}}, h' g(v, t) ]_{\tilde{\g}_{[2]}}
                            \\
                            & = [ h f(v, t), h' D( g(v, t) ) ]_{\tilde{\g}_{[2]}} + [ h D( f(v, t) ), h' g(v, t) ]_{\tilde{\g}_{[2]}}
                            \\
                            & = f(v, t) \bar{d}( D( g(v, t) ) ) + D( f(v, t) ) \bar{d}(g(v, t))
                        \end{aligned}
                    $$
                One thus sees that:
                    $$[ D, f(v, t) \bar{d}( g(v, t) ) ]_{\hat{\g}_{[2]}} = f(v, t) \bar{d}( D( g(v, t) ) ) + D( f(v, t) ) \bar{d}(g(v, t))$$
                and since the element $f(v, t) \bar{d}( g(v, t) )$ is central (via the map $\e$ mentioned earlier), this gives another description of:
                    $$[ \d_{[2]}, \z_{[2]} ]_{\hat{\g}_{[2]}}$$
                With this in mind, we return quickly to remark \ref{remark: derivation_action_on_multiloop_algebras}; there, we previously demonstrated that:
                    $$[ \d_{[2]}, \g_{[2]} ]_{\hat{\g}_{[2]}} \subseteq \g_{[2]} \oplus \z_{[2]}$$
                but we claim now that the following stronger fact holds:
                    $$[ \d_{[2]}, \g_{[2]} ]_{\hat{\g}_{[2]}} \subseteq \g_{[2]}$$
                To see why this is the case, suppose firstly that for any $D \in \d_{[2]}$, any $X := x f(v, t) \in \g_{[2]}$ (for some $f(v, t) \in A_{[2]}$), there is $K(X) \in \z_{[2]}$ depending on $X$ (and indeed, such a $K(X)$ exists by remark \ref{remark: derivation_action_on_multiloop_algebras}) such that:
                    $$[ D, X ]_{\hat{\g}_{[2]}} = x D( f(v, t) ) + K(X)$$
                Next, pick an arbitrary element $\xi \in \d_{[2]}$ and then consider the following:
                    $$( [ D, X ]_{\hat{\g}_{[2]}}, \xi )_{\hat{\g}_{[2]}} = (D(X) + K(X), \xi)_{\hat{\g}_{[2]}} = (K(X), \xi)_{\hat{\g}_{[2]}}$$
                wherein the last equality holds as a consequence of the fact that:
                    $$( \g_{[2]}, \d_{[2]} )_{\hat{\g}_{[2]}} = 0$$
                per the construction of the bilnear form $(-, -)_{\hat{\g}_{[2]}}$ as in convention \ref{conv: orthogonal_complement_of_toroidal_centres}. At the same time, using invariance yields us:
                    $$( [ D, X ]_{\hat{\g}_{[2]}}, \xi )_{\hat{\g}_{[2]}} = (X, [\xi, D]_{\hat{\g}_{[2]}})_{\hat{\g}_{[2]}} = 0$$
                wherein the last equality is due to the fact that:
                    $$[\xi, D]_{\hat{\g}_{[2]}} \in \z_{[2]} \oplus \d_{[2]}$$
                (cf. remark \ref{remark: dual_of_toroidal_centres_contains_derivations}) and the fact that:
                    $$( \g_{[2]}, \z_{[2]} \oplus \d_{[2]} )_{\hat{\g}_{[2]}} = 0$$
                per the construction of the bilnear form $(-, -)_{\hat{\g}_{[2]}}$ as in convention \ref{conv: orthogonal_complement_of_toroidal_centres}. We thus have that:
                    $$(K(X), \xi)_{\hat{\g}_{[2]}} = (X, [\xi, D]_{\hat{\g}_{[2]}})_{\hat{\g}_{[2]}} = 0$$
                for every $\xi \in \d_{[2]}$. The non-degeneracy of $(-, -)_{\hat{\g}_{[2]}}$ then implies through this fact that:
                    $$K(X) = 0$$
                As such, we have that:
                    $$[ \d_{[2]}, \g_{[2]} ]_{\hat{\g}_{[2]}} \subseteq \g_{[2]}$$
                as claimed. 
            \end{enumerate}

        \end{remark}

        Remarks \ref{remark: derivation_action_on_multiloop_algebras}, \ref{remark: dual_of_toroidal_centres_contains_derivations}, and \ref{remark: derivation_action_on_toroidal_centres} can now be packaged together into the following result.
        \begin{proposition}[$\tilde{\g}_{[2]}$ as a $\der_{\bbC}(A_{[2]})$-module] \label{prop: toroidal_lie_algebras_as_modules_over_vector_field_lie_algebras}
            $\tilde{\g}_{[2]}$ is a $\der_{\bbC}(A_{[2]})$-module, decomposing into a direct sum of the submodules $\g_{[2]}$ and $\z_{[2]}$. 
        \end{proposition}

        Now, let us see if and how $\d_{[2]}$ might be endowed with a Lie algebra structure of its own, so that we might have that:
            $$\hat{\g}_{[2]} \cong \tilde{\g}_{[2]} \rtimes \d_{[2]}$$
        The upshot is that in general, the vector space $\d_{[2]}$ fails to be a Lie algebra, namely due to the commutators $[D, D']_{\hat{\g}_{[2]}}$ (for any two $D, D' \in \d_{[2]}$) having a non-vanishing $\z_{[2]}$-summand in general, and their $\d_{[2]}$-summand is actually nothing but the usual commutator of derivations inherited from $\der_{\bbC}(A_{[2]})$, and $\d_{[2]}$ will only be a Lie algebra if we choose to endow it with said commutator bracket (so that the $\z_{[2]}$ would be made to vanish). 
        
        \begin{proposition}[How does $\d_{[2]}$ act on itself] \label{prop: lie_bracket_on_orthogonal_complement_of_toroidal_centre}
            Let $\d_{[2]}$ be given as in convention \ref{conv: orthogonal_complement_of_toroidal_centres}. Then:
                $$[ \d_{[2]}, \d_{[2]} ]_{\hat{\g}_{[2]}} \subset \z_{[2]} \oplus \d_{[2]}$$
            i.e. the $\g_{[2]}$-summand of any commutator of the kind $[D, D']_{\hat{\g}_{[2]}}$ (for any two $D, D' \in \d_{[2]}$) actually vanishes. Furthermore, neither the $\z_{[2]}$- nor the $\d_{[2]}$-summand of those commutators $[D, D']_{\hat{\g}_{[2]}}$ necessarily vanish in general. 
        \end{proposition}
            \begin{proof}
                Pick arbitrary elements $D, D' \in \d_{[2]}$ and set:
                    $$[D, D']_{\hat{\g}_{[2]}} := X(D, D') + K(D, D') + \xi(D, D')$$
                for some $X(D, D') \in \g_{[2]}, K(D, D') \in \z_{[2]}$, and $\xi(D, D') \in \d_{[2]}$ depending on $D, D'$. Pick also a test element $y g(v, t) \in \g_{[2]}$, for some arbitrary $y \in \g$ and $g(v, t) \in A_{[2]}$ and set:
                    $$[D, y g(v, t)]_{\hat{\g}_{[2]}} := y D( g(v, t) ) + K_{D, Y}$$
                    $$[D', y g(v, t)]_{\hat{\g}_{[2]}} := y D'( g(v, t) ) + K_{D', Y}$$
                for some $K_{D, Y} \in \z_{[2]}$ depending on $Y$ (cf. remark \ref{remark: derivation_action_on_multiloop_algebras}).
                
                Via the Jacobi identity, we get that:
                    $$
                        \begin{aligned}
                            & [ [D, D']_{\hat{\g}_{[2]}}, y g(v, t) ]_{\hat{\g}_{[2]}}
                            \\
                            = & [ D, [ D', y g(v, t) ]_{\hat{\g}_{[2]}} ]_{\hat{\g}_{[2]}} + [ D', [ y g(v, t), D ]_{\hat{\g}_{[2]}} ]_{\hat{\g}_{[2]}}
                            \\
                            = & [ D, y D'( g(v, t) ) + K_{D', Y} ]_{\hat{\g}_{[2]}} - [ D', y D( g(v, t) ) + K_{D, Y} ]_{\hat{\g}_{[2]}}
                            \\
                            = & \left( y D( D'(g(v, t)) ) + K_{DD', Y} + [ D, K_{D', Y} ]_{\hat{\g}_{[2]}} \right) - \left( y D'( D(g(v, t)) ) + K_{D'D, Y} + [ D', K_{D, Y} ]_{\hat{\g}_{[2]}} \right)
                            \\
                            = & y (DD' - D'D)( g(v, t) ) + ( K_{DD', Y} - K_{D'D, Y} ) + ( [ D, K_{D', Y} ]_{\hat{\g}_{[2]}} - [ D', K_{D, Y} ]_{\hat{\g}_{[2]}} )
                        \end{aligned}
                    $$
                for some $K_{DD', Y}, K_{D'D, Y} \in \z_{[2]}$ such that:
                    $$[ D, y D'( g(v, t) ) ]_{\hat{\g}_{[2]}} := y D( D'( g(v, t) ) ) + K_{DD', Y}$$
                    $$[ D', y D( g(v, t) ) ]_{\hat{\g}_{[2]}} := y D( D'( g(v, t) ) ) + K_{D'D, Y}$$
                At the same time, we have that:
                    $$
                        \begin{aligned}
                            & [ [D, D']_{\hat{\g}_{[2]}}, y g(v, t) ]_{\hat{\g}_{[2]}}
                            \\
                            = & [ X(D, D') + K(D, D') + \xi(D, D') , y g(v, t) ]_{\hat{\g}_{[2]}}
                            \\
                            = & [ X(D, D') + \xi(D, D') , y g(v, t) ]_{\hat{\g}_{[2]}}
                            \\
                            = & [ X(D, D') , y g(v, t) ]_{\hat{\g}_{[2]}} + \left( y \xi(D, D')(g(v, t)) + K_{\xi(D, D'), Y} \right)
                        \end{aligned}
                    $$
                wherein the second equality holds thanks to the fact that $[\z_{[2]}, \g_{[2]}]_{\hat{\g}_{[2]}} = 0$, and $K_{\xi(D, D'), Y} \in \z_{[2]}$ is some element (cf. remark \ref{remark: derivation_action_on_multiloop_algebras}). Combining the two observations together then yields:
                    $$
                        \begin{aligned}
                            & [ X(D, D') , y g(v, t) ]_{\hat{\g}_{[2]}} + \left( y \xi(D, D')(g(v, t)) + K_{\xi(D, D'), Y} \right)
                            \\
                            = & y (DD' - D'D)( g(v, t) ) + ( K_{DD', Y} - K_{D'D, Y} ) + ( [ D, K_{D', Y} ]_{\hat{\g}_{[2]}} - [ D', K_{D, Y} ]_{\hat{\g}_{[2]}} )
                        \end{aligned}
                    $$
                From remark \ref{remark: centres_of_dual_toroidal_lie_algebras}, we know that there exists $K_{X(D, D'), Y} \in \z_{[2]}$ such that:
                    $$[ X(D, D') , y g(v, t) ]_{\hat{\g}_{[2]}} = [ X(D, D') , Y ]_{\hat{\g}_{[2]}} = [X(D, D'), Y]_{\g_{[2]}} + K_{X(D, D'), Y}$$
                using which we can write:
                    $$
                        \begin{aligned}
                            & [X(D, D'), Y]_{\g_{[2]}} - y \left( ( DD' - D'D) - \xi(D, D') \right)( g(v, t) )
                            \\
                            = & \left( [ D, K_{D', Y} ]_{\hat{\g}_{[2]}} - [ D', K_{D, Y} ]_{\hat{\g}_{[2]}} \right) - \left( K_{X(D, D'), Y} + K_{\xi(D, D'), Y} \right)
                        \end{aligned}
                    $$
                    
                We note right away that the LHS lies entirely in $\g_{[2]}$, whereas the RHS is an element of $\z_{[2]}$ due to the fact that $[\d_{[2]}, \z_{[2]}]_{\hat{\g}_{[2]}} \subseteq \z_{[2]}$ (cf. remark \ref{remark: derivation_action_on_toroidal_centres}), which tells us that $[ D, K_{D', Y} ]_{\hat{\g}_{[2]}}, [ D', K_{D, Y} ]_{\hat{\g}_{[2]}} \in \z_{[2]}$ in particular. Because $\g_{[2]}$ is centreless (as $\g$ is simple and the Lie bracket on $\g_{[2]}$ is given by extension of scalars), this observation subsequently implies that the LHS must vanish, i.e.:
                    $$[X(D, D'), Y]_{\g_{[2]}} - y \left( ( DD' - D'D) - \xi(D, D') \right)( g(v, t) ) = 0$$
                Because we have by construction that:
                    $$DD' - D'D - \xi(D, D') \in \d_{[2]}$$
                we now make the following claim: \textit{if we fix some arbitrary $E \in \g_{[2]}$ and some $P \in \d_{[2]}$ then:}
                    $$\forall H := h \varphi \in \g_{[2]}: [E, H]_{\g_{[2]}} = h P( \varphi ) \implies E = 0$$

                Using the root space decomposition for $\g$, we see that if $h \in \h$ then we then will have that $[E, H]_{\g_{[2]}} \in \n^{\pm}_{[2]}$, but at the same time, that $h P(\varphi) \in \h_{[2]}$. The only way for this to be true is that $[E, H]_{\g_{[2]}} = 0$, which is the case if and only if $E = 0$. If $h \in \n^{\pm}$, then $[E, H]_{\g_{[2]}} \in \n^{\pm}_{[2]} \oplus \h_{[2]}$ and the $\h_{[2]}$-summand will be non-zero in general; at the same time, $h P(\varphi) \in \n^{\pm}_{[2]}$ in this case, and again, the only way for these to facts to be true simultaneously is that $E = 0$ necessarily. 

                Apply the claim to the fact that:
                    $$[X(D, D'), Y]_{\g_{[2]}} = y \left( ( DD' - D'D) - \xi(D, D') \right)( g(v, t) )$$
                - and again, note that $( DD' - D'D) - \xi(D, D') \in \d_{[2]}$ - then yields:
                    $$X(D, D') = 0$$
                precisely as desired. 
            \end{proof}
        \begin{corollary}
            For any $D, D' \in \d_{[2]}$, the $\d_{[2]}$-summand of $[D, D']_{\hat{\g}_{[2]}}$ is nothing but the commutator $DD' - D'D$.
        \end{corollary} 
        
        \begin{question}[A different Lie bracket on $\der_{\bbC}(A_{[2]})$ ?] \label{question: alternative_derivation_lie_bracket}
            Is there a Lie algebra structure $\der_{\bbC}(A_{[2]})$ (different from the standard one given by commutators $[D, D'] := DD' - D'D$) with respect to which $\d_{[2]}$ is always a Lie subalgebra ? 
        \end{question}
        
        \begin{proposition}[$\tilde{\g}_{[2]}$ as a $\d_{[2]}$-module] \label{prop: toroidal_lie_algebras_as_modules_over_div_0_vector_field_lie_algebras}
            When we assume that:
                $$\hat{\g}_{[2]} \cong \tilde{\g}_{[2]} \rtimes \d_{[2]}$$
            i.e. when $\d_{[2]}$ is a Lie subalgebra of $\der_{\bbC}(A_{[2]})$ with the usual commutator bracket, we will have that $\tilde{\g}_{[2]}$ is a $\d_{[2]}$-module, decomposing into a direct sum of the submodules $\g_{[2]}$ and $\z_{[2]}$. 
        \end{proposition}
        
        In summary, we have yielded the following result:
        \begin{theorem} \label{theorem: extended_toroidal_lie_algebras}
            There is a Lie bracket on the vector space:
                $$\hat{\g}_{[2]} := \tilde{\g}_{[2]} \oplus \d_{[2]}$$
            given as in remarks \ref{remark: derivation_action_on_multiloop_algebras} and \ref{remark: derivation_action_on_toroidal_centres}, uniquely determined by the choice of an invariant non-degenerate symmetric $\bbC$-bilinear form $(-, -)_{\hat{\g}_{[2]}}$ as in convention \ref{conv: orthogonal_complement_of_toroidal_centres}, up to choices of central summands for the elements of $[\d_{[2]}, \d_{[2]}]_{\hat{\g}_{[2]}}$ (cf. remark \ref{remark: dual_of_toroidal_centres_contains_derivations}).

            When said central summands are chosen to vanish, i.e. when $\d_{[2]}$ is a Lie subalgebra of $\der_{\bbC}(A_{[2]})$ with the usual commutator bracket, one shall obtain:
                $$\hat{\g}_{[2]} := \tilde{\g}_{[2]} \rtimes \d_{[2]}$$
        \end{theorem}
        \begin{definition}[Yangian extended toroidal Lie algebras] \label{def: extended_toroidal_lie_algebras}
            We refer to $\hat{\g}_{[2]}$ as in theorem \ref{theorem: extended_toroidal_lie_algebras} as a \textbf{Yangian extended ($2$-)toroidal Lie algebra}. 
        \end{definition}
        \begin{remark}
            Our notion of Yangian extended $2$-toroidal Lie algebras does not quite coincide with the very similar notion of an \say{extended affine Lie algebra of nullity $2$}. Ultimately, this is because the bilinear form that we have endowed $\g_{[2]}$ with is of degree $-1$ (as opposed to $0$) in the second variable. For the latter, $\d_{[2]}$ would be the Lie algebra of divergence-free vector fields on the (smooth) affine $\bbC$-scheme $\Spec A_{[2]}$.
        \end{remark}
        \begin{convention}
            Since we are not concerned here with nullity-$2$ extended affine Lie algebras, but rather only the Yangian analogue thereof, we will drop the descriptor \say{Yangian}. 
        \end{convention}

        Even though we are now perfectly ready to endow $\hat{\g}_{[2]}^+$ with a Lie bialgebra structure (which eventually will be shown to descend down to a Lie bialgebra structure on $\tilde{\g}_{[2]}^+$; cf. theorem \ref{theorem: toroidal_lie_bialgebras}) arising from some Manin triple of the kind:
            $$(\hat{\g}_{[2]}, \hat{\g}_{[2]}^+, \hat{\g}_{[2]}^-)$$
        wherein $\hat{\g}_{[2]}$ is equipped with the non-degenerate invariant bilinear form $(-, -)_{\hat{\g}_{[2]}}$, let us defer this construction to theorem \ref{theorem: extended_toroidal_manin_triples} for organisational purposes.

        One question that is not too relevant to our eventual study of affine Yangians, but is nevertheless natural and interesting in its own right, is the following:
        \begin{question}[Uniqueness of extended toroidal Lie algebras] \label{question: uniqueness_of_extended_toroidal_lie_algebras}
            How freely does the $\z_{[2]}$-summand of the commutators of the kind:
                $$[D, D']_{ \hat{\g}_{[2]} } \in \z_{[2]} \oplus \d_{[2]}$$
            vary ? Alternatively (and perhaps not equivalently), what is the dimension of the $\bbC$-vector space:
                $$H^2_{\Lie}( \der_{\bbC}(A_{[2]}), \bar{\Omega}_{[2]} )$$
            wherein the $\der_{\bbC}(A_{[2]})$-action on $\bar{\Omega}_{[2]}$ is determined by how $\d_{[2]}$ acts on $\z_{[2]}$ (cf. remark \ref{remark: centres_of_dual_toroidal_lie_algebras}).

            This question ought to also be equivalent to question \ref{question: alternative_derivation_lie_bracket}. 
        \end{question}
        \begin{question}
            More generally, one can pose the following problem about the cyclic cohomology of smooth varieties. Suppose that $X$ is a smooth variety over an algebraically closed field $k$ of characteristic $0$ and write:
                $$\bar{\Omega}_{X/k}^n := \Omega_{X/k}^n/d( \Omega_{X/k}^{n - 1} )$$
            as well as:
                $$\scrD_{X/k}^1$$
            for the tangent sheaf of $X$ (which is naturally a sheaf of Lie algebras via local commutators of vector fields). $\scrD_{X/k}^1$ acts naturally on $\bar{\Omega}_{X/k}^n$ via Lie derivatives, so it is natural to inquire about the dimension of the $k$-vector spaces:
                $$H^{\bullet}_{\Lie}( \scrD_{X/k}^1, \bar{\Omega}_{X/k}^n )$$
            and furthermore, if there is any dependence thereof on $n \geq 1$.  
        \end{question}

        Let us conclude this subsection with the following question, which is natural now that we have a solid handle on how the Lie bracket on $\hat{\g}_{[2]}$ is given:
        \begin{question}
            What is the centre $\hat{\z}_{[2]} := \z( \hat{\g}_{[2]} )$ ? This ought to be smaller than $\z_{[2]}$ somehow, since elements of $\z_{[2]}$ need not be central in $\hat{\g}_{[2]}$. 
        \end{question}
        \begin{remark}[Computing the centre without computing all the brackets ...]
            Since $\g_{[2]}$ is centreless, we have that:
                $$\hat{\z}_{[2]} = \z( \z_{[2]} \oplus \d_{[2]} )$$
            As $\z_{[2]}$ is an abelian Lie algebra, this implies that in order to compute $\hat{\z}_{[2]}$, it suffices to explicitly compute the commutators of the form:
                $$[D, K]_{\hat{\g}_{[2]}}, [D, D']_{\hat{\g}_{[2]}}$$
            for $D, D' \in \d_{[2]}$ and $K \in \z_{[2]}$, to see which ones vanish. However, this is rather tedious and not very insightful.
            
            An alternative method is as follows: exploiting the fact that the symmetric bilinear form $(-, -)_{\hat{\g}_{[2]}}$ is both invariant and non-degenerate (by construction; cf. convention \ref{conv: orthogonal_complement_of_toroidal_centres}), we can characterise the centre $\hat{\z}_{[2]}$ as the Lie ideal of $\hat{\g}_{[2]}$ containing elements $Z$ such that:
                $$0 = ([Z, X]_{\hat{\g}_{[2]}}, Y)_{\hat{\g}_{[2]}} = (Z, [X, Y]_{\hat{\g}_{[2]}})_{\hat{\g}_{[2]}}$$
            for any $X, Y \in \hat{\g}_{[2]}$, with the first equality holding thanks to the fact that $Z$ is supposed to commute with every other element of $\hat{\g}_{[2]}$ by assumption of being central. We are thus left with the task of finding elements:
                $$Z \in \hat{\g}_{[2]}$$
            such that:
                $$(Z, [\hat{\g}_{[2]}, \hat{\g}_{[2]}]_{\hat{\g}_{[2]}})_{\hat{\g}_{[2]}} = 0$$
            Since $\g_{[2]}$ and $\d_{[2]}$ is are both non-abelian Lie subalgebras of $\hat{\g}_{[2]}$, their elements can not be central in $\hat{\g}_{[2]}$. As such, we have narrowed the scope of our search down to:
                $$\hat{\z}_{[2]} \subset \z_{[2]}$$
        \end{remark}
        \begin{proposition}[Centres of extended toroidal Lie algebras] \label{prop: centres_of_extended_toroidal_lie_algebras}
            The centre $\hat{\z}_{[2]}$ is a two-dimensional Lie subalgebra of $\z_{[2]}$, namely spanned by $c_v$ and $c_t$. 
        \end{proposition}
            \begin{proof}
                Since we know that $(\z_{[2]}, \g_{[2]} \oplus \z_{[2]})_{\hat{\g}_{[2]}} = 0$, in order to see that $\hat{\z}_{[2]} = \bbC c_v \oplus \bbC c_t$, it suffices to consider $(\hat{\z}_{[2]}, [\d_{[2]}, \d_{[2]}]_{\hat{\g}_{[2]}})$, as we know that $[\d_{[2]}, \d_{[2]}]_{\hat{\g}_{[2]}} \subset \z_{[2]} \oplus \d_{[2]}$, (cf. proposition \ref{prop: lie_bracket_on_orthogonal_complement_of_toroidal_centre}). Towards this end, we can assume without any loss of generality to only consider the case:
                    $$[\d_{[2]}, \d_{[2]}]_{\hat{\g}_{[2]}} \subset \d_{[2]}$$
                (according to proposition \ref{prop: lie_bracket_on_orthogonal_complement_of_toroidal_centre}, we have the freedom to choose the $\z_{[2]}$-component of the brackets of the elements of $\d_{[2]}$). Some straightforward (but tedious) computations shall show that:
                    $$[\d_{[2]}, \d_{[2]}]_{\hat{\g}_{[2]}} \subseteq \bigoplus_{(r, s) \in \Z^2} \bbC D_{r, s}$$
                thus implying that:
                    $$(\bbC c_v \oplus \bbC c_v, [\d_{[2]}, \d_{[2]}]_{\hat{\g}_{[2]}})_{\hat{\g}_{[2]}} = 0$$
                and hence:
                    $$\hat{\z}_{[2]} = \bbC c_v \oplus \bbC c_v$$
            \end{proof}