\section{Recollections about (twisted) Yangians of classical types}
    \subsection{(Extended) untwisted Yangians}
        Let us quickly recall some general features of the RTT formalism, mostly so that we can set up and justify some key notations. For a moment, let $\g$ be a general finite-dimensional simple Lie algebra.
            
        Begin with the \textbf{formal loop algebra}:
            $$\g(\!(t^{-1})\!) := \g \tensor \bbC(\!(t^{-1})\!)$$
        which shall always be equipped with the canonical $\Z$-grading given by $\deg xt^m := m$ for all $x \in \g$ and all $m \in \Z$. This Lie algebra can be endowed with the non-degenerate, invariant, and symmetric bilinear form given for all $x, y \in \g$ and $m, n \in \Z$ by:
            $$(xt^m, yt^n) := (x, y)_{\g} \delta_{m + n, -1}$$
        wherein $(-, -)_{\g}$ is a non-degenerate, invariant, and symmetric bilinear form on $\g$ (\textit{a priori}, a $\bbC^{\x}$-multiple of the Killing form). Using this bilinear form, one can then construct the following $\Z$-graded Manin triple:
            $$\left( \g(\!(t^{-1})\!), \g[t], \g[\![t^{-1}]\!] \right)$$
        This then defines a Lie cobracket $\delta: \g[t] \to \g[t_1] \tensor \g[t_2]$ given by:
            $$\delta( X ) := [\Box(X), \calr(t_1 - t_2)]$$
        for all $X \in \g[t]$ and for:
            $$\calr(u) := \frac{\calr_{\g}}{u} + O(1)$$
        (with $u$ a formal variable), which we call the Yangian \textbf{classical r-matrix}. Here, $\Box(X) := X \tensor 1 + 1 \tensor X$ is the \say{standard} classical coproduct, and $\calr_{\g} \in \g \tensor \g$ is the Casimir tensor\footnote{To be more precise, we have $\calr_{\g} \in \Sym^2(\g)^{\g}$. By viewing $(-, -)_{\g}$ as an element of $\Hom( \Sym^2(\g)^{\g}, \bbC )$, we can then regard $\calr_{\g}$ as the pre-image of $(-, -)_{\g}$ under the canonical map $\g \tensor \g \to \Hom(\g \tensor \g, \bbC)$. Since $(-, -)_{\g}$ is non-degenerate, symmetric, and invariant, the pre-image is unique and lies in the subspace $\Sym^2(\g)^{\g} \subset \g \tensor \g$.}.
        
        \textit{A priori}, the r-matrix $\calr$ is a solution to the \textbf{classical Yang-Baxter equation (CYBE)}:
            \begin{equation} \label{equation: CYBE}
                [\calr_{1, 2}(u), \calr_{1, 3}(u + v)] + [\calr_{1, 2}(u), \calr_{2, 3}(v)] + [\calr_{1, 3}(u + v), \calr_{2, 3}(v)] = 0
            \end{equation}
        Note also that:
            \begin{equation} \label{equation: classical_unitarity}
                \calr(u) = -\calr(-u)
            \end{equation}
        and because of this, we characterise the Yangian classical r-matrix as being \textbf{unitary}.
        
        A general result of Etingof-Kazhdan tells us that the graded Lie bialgebra $(\g[t], \delta)$ admits a \say{graded quantisation}:
            $$\calY_{\hbar}(\g)$$
        of the aforementioned Lie bialgebra structure on $\g[t]$; in particular, this means that there is isomorphism of $\Z$-graded \textit{Hopf algebras}:
            $$\calU(\g[t]) \cong \calY_{\hbar}(\g)/\hbar$$
        with the comultiplication on $\calU(\g[t])$ being given by $\Box(X)$ for all $X \in \g[t]$. This graded quantisation is we call the \textbf{formal Yangian}, and by specialising $\hbar$ to any $\hbar_0 \in \bbC \setminus \{0\}$, one obtains the \textbf{Yangian}:
            $$\calY(\g)$$
        It turns out that, up to isomorphisms, the Yangian does not depend on the choice of $\hbar_0$ (hence why we do not specify this choice in the notation), and this is because the Yangian $\calY(\g)$ carries a natural $\Z$-filtration such that $\calY_{\hbar}(\g) \cong \bigoplus_{n \geq 0} \calY(\g) \hbar^n$ and $\gr \calY(\g) \cong \calU(\g[t]) \cong \calY_{\hbar}(\g)/\hbar$ (cf. \cite{drinfeld_original_yangian_paper}); as such, we refer to the Yangian as a \say{filtered quantisation} of $\calU(\g[t])$.

        Let:
            $$\calY_{\hbar}(\g)'$$
        be the $\bbC[\![\hbar]\!]$-submodule of $\Hom_{\bbC[\![\hbar]\!]}( \calY_{\hbar}(\g), \bbC[\![\hbar]\!] )$ spanned by so-called \say{tempered} $\bbC[\![\hbar]\!]$-linear functionals, which are elements $f := \sum_{m \geq 0} f_m$ such that $\lim_{m \to +\infty} \deg f_m = +\infty$.
            
        Such an element satisfies the \textbf{quantum Yang-Baxter equation (QYBE)}:
            \begin{equation} \label{equation: QYBE}
                \calR_{1, 2}(u) \calR_{1, 3}(u + v) \calR_{2, 3}(v) = \calR_{2, 3}(v) \calR_{1, 3}(u + v) \calR_{1, 2}(u)
            \end{equation}
        wherein the subscript pairs indicate the pair of tensor copies that $\calR(u)$ is acting on.

        Finally, by letting 
            \begin{equation} \label{equation: RTT_relation}
                \calR(u - v) T_1(u) T_2(v) = T_2(v) T_1(u) \calR(u - v)
            \end{equation}
        obtaining the so-called \textbf{transfer matrix}.
            $$T(u) \in \Mat_N( \calY(\g) )[\![u^{-1}]\!]$$
    
        Following \cite[Definition 2.1]{guay_regelskis_twisted_yangians_for_symmetric_pairs_of_types_BCD}, we work with the following definition of extended untwisted Yangians associated to a rational solution $\calR(u)$ to the QYBE on $U^{\tensor 3}$.
        \begin{definition}[Extended untwisted Yangians] \label{def: extended_untwisted_yangians}
            The \textbf{extended untwisted Yangian} associated to the classical Lie algebra $\g_N$ and a rational solution $\calR(u)$ to the QYBE \eqref{equation: QYBE} is the associative algebra:
                $$\calX(\g_N, \calR(u))$$
            generated by the coefficients of the matrix entries of the elements:
                $$T(u) \in \Mat_N( \calX(\g_N)[\![u^{-1}]\!] )$$
            subjected to the RTT relation \eqref{equation: RTT_relation}.
        \end{definition}

        We will be fixing once and for all a solution:
            $$\calR(u) = 1 + \frac{P}{u} + \frac{Q}{u - \kappa}$$
        for $\kappa := \sgn(\g_N) + \frac{N}{2}$, with $Q = -2P$ for type $\sfA$, and:
            \begin{equation} \label{equation: BCD_signature}
                \sgn(\g_N) =
                \begin{cases}
                    \text{$0$ if $\g_N$ is of type $\sfA$}
                    \\
                    \text{$-1$ if $\g_N$ is of either type $\sfB$ or $\sfD$}
                    \\
                    \text{$1$ if $\g_N$ is of type $\sfC$}
                \end{cases}
            \end{equation}
        As such, we can abbreviate:
            $$\calX(\g_N) := \calX(\g_N, \calR)$$

        \begin{lemma}[Automorphisms of extended untwisted Yangians] \label{lemma: automorphisms_of_extended_untwisted_yangians}
            \begin{enumerate}
                \item Let $f(u) \in \bbC[\![u^{-1}]\!]^{\x}$ be an invertible formal power series in $u^{-1}$; recall that, because $\bbC[\![u^{-1}]\!]$ is a local commutative ring with (unique) maximal ideal $u^{-1}\bbC[\![u^{-1}]\!]$, $f(u)$ must be of the form $1 + \sum_{r \geq 0} f^{(r)} u^{-r - 1}$. Then, the map:
                    $$\mu_f: \calX(\g_N)[\![u^{-1}]\!] \to \Mat_N(\calX(\g_N)[\![u^{-1}]\!])$$
                given by:
                    $$\mu_f( T(u) ) := f(u) T(u)$$
                defines an algebra automorphism of $\calX(\g_N)$.
                \item Let $a \in \bbC$. Then, the map:
                    $$\tau_a: \Mat_N(\calX(\g_N)[\![u^{-1}]\!]) \to \Mat_N(\calX(\g_N)[\![u^{-1}]\!])$$
                given by:
                    $$\tau_a( T(u) ) := T(u - a)$$
                defines an algebra automorphism of $\calX(\g_N)$.
            \end{enumerate}
        \end{lemma}
            \begin{proof}
                See \cite[Section 2]{guay_regelskis_twisted_yangians_for_symmetric_pairs_of_types_BCD}.
            \end{proof}

        \begin{lemma}[Quantum contractions] \label{lemma: quantum_contractions}
            The transfer matrix:
                $$T(u) \in \Mat_N( \calX(\g_N) )[\![u^{-1}]\!]$$
            satisfies the \textbf{quantum contraction} formula:
                \begin{equation} \label{equation: quantum_contraction}
                    T(u + \kappa)^t T(u) = T(u) T(u + \kappa)^t = z(u)
                \end{equation}
            wherein $z(u) \in \calX(\g_N)[\![u^{-1}]\!]^{\x}$ is some invertible scalar series.
        \end{lemma}
            \begin{proof}
                See \cite[Section 2]{guay_regelskis_twisted_yangians_for_symmetric_pairs_of_types_BCD}, Equation 2.11 in particular.
            \end{proof}
            
        This leads us to the following definition.
        \begin{definition}[Untwisted Yangians] \label{def: untwisted_yangians}
            The \textbf{untwisted Yangian} is given by:
                $$\calY(\g_N) := \calX(\g_N)/\<z(u) - 1\>$$
        \end{definition}
        \begin{remark}[Unitarity]
            Through equation \eqref{equation: quantum_contraction}, we see that $\calY(\g_N)$ is equivalently the quotient of $\calX(\g_N)$ by the two-sided ideal generated by the relation $T(u + \kappa)^t T(u) - 1$ (or equivalently $T(u) T(u + \kappa)^t - 1$). Often, we will refer to these relations collectively as the \textbf{unitarity relation} or the \textbf{unitary condition}.
        \end{remark}
        \begin{lemma}[Centres of extended untwisted Yangians] \label{lemma: centres_of_extended_untwisted_yangians}
            \begin{enumerate}
                \item The coefficients of the quantum contraction $z(u)$ from equation \eqref{equation: quantum_contraction} generated the centre $\calZ(\g_N) := \rmZ( \calX(\g_N) )$.
                \item Moreover, the coefficients of $z(u)$ are algebraically independent from one another, and thus:
                    $$\calX(\g_N) \cong \calY(\g_N) \tensor \calZ(\g_N)$$
            \end{enumerate}
        \end{lemma}
            \begin{proof}
                \begin{enumerate}
                    \item 
                    \item 
                \end{enumerate}
            \end{proof}

    \subsection{(Extended) twisted Yangians of classical types}
        All throughout:
            $$\g_N \subset \gl_N$$
        will be used to denote a finite-dimensional simple Lie algebra over $\bbC$ that is of one of the classical types in the Cartan-Killing Classification; equivalently, suppose that:
            $$
                \g_N \in
                \begin{cases}
                    \text{$\sl_N$}
                    \\
                    \text{$\{ \o_{2n + 1} \}_{n \geq 0}$ and $N = 2n + 1$}
                    \\
                    \text{$\{ \sp_{2n} \}_{n \geq 0}$ and $N = 2n$}
                    \\
                    \text{$\{ \o_{2n} \}_{n \geq 0}$ and $N = 2n$}
                \end{cases}
            $$
        and respectively, we say that $\g_N$ is linear, odd-orthogonal, symplectic, or even-orthogonal. This Lie algebra has a natural action on $U := \bbC^{\oplus N}$ (the so-called \say{vector representation}), which shall always be equipped with the standard basis:
            $$\{e_i\}_{1 \leq i \leq N}$$
        consisting of vectors with $1$ in the $i^{th}$ entry and $0$ elsewhere. By finite-dimensionality, we identify:
            $$U^{\tensor 2} \xrightarrow[]{\cong} \End(U)$$
        and then write:
            $$E_{i, j}$$
        for the image of $e_i \tensor e_j$ under the identification above; explicitly, $E_{i, j}$ is the $N \x N$ matrix with $1$ in the $(i, j)^{th}$ entry and $0$ elsewhere. Moreover, recall that $U$ comes equipped with a natural $\g_N$-invariant and non-degenerate bilinear form that we shall denote by:
            $$\eta$$
        This form is symmetric for the linear type $\sfA$ and orthogonal types $\sfB, \sfD$, while skew-symmetric for the symplectic type $\sfC$.
    
        \begin{definition}[Rational K-matrices] \label{def: boundary_quantum_yang_baxter_equations}
            Given a rational solution $\calR(u)$ to the QYBE, the resulting boundary quantum Yang-Baxter equation (bQYBE) with (meromorphic) solution $\calK(u)$ reads:
                \begin{equation} \label{equation: boundary_quantum_yang_baxter_equations}
                    \calR_{1, 2}(u - v) \calK_1(u) \calR_{1, 2}(u + v) \calK_2(v) = \calK_2(v) \calR_{1, 2}(u + v) \calK_1(u) \calR_{1, 2}(u - v) 
                \end{equation}
        \end{definition}
    
        \begin{definition}[Boundary transfer matrices] \label{def: boundary_transfer_matrices}
            
        \end{definition}
        \begin{lemma}[Reflection equations] \label{lemma: reflection_equations}
            The boundary transfer matrix $S(u)$ as in definition \ref{def: boundary_transfer_matrices} satisfy the following so-called \textbf{reflection equation}\footnote{Also called the quaternary relation, especially in the context of twisted Yangians. See e.g. \cite[Proposition 2.2.1]{molev_yangians_and_classical_lie_algebras}.}:
                \begin{equation} \label{equation: reflection_equations}
                    \calR_{1, 2}(u - v) S_1(u) \calR_{1, 2}(u + v) S_2(v) = S_2(v) \calR_{1, 2}(u + v) S_1(u) \calR_{1, 2}(u - v) 
                \end{equation}
        \end{lemma}
            \begin{proof}
                
            \end{proof}
    
        \begin{definition}[(Extended) twisted Yangians associated to symmetric pairs] \label{def: (extended)_twisted_yangians}
            
        \end{definition}

        \begin{lemma}[Twisted quantum contractions] \label{lemma: twisted_quantum_contractions}
            The boundary transfer matrix:
                $$S(u) \in \Mat_N( \calX^{\tw}(\g_N, \vartheta) )[\![u^{-1}]\!]$$
            satisfies the following so-called \textbf{twisted quantum contraction} formula:
                \begin{equation} \label{equation: twisted_quantum_contraction}
                    S(u) S(-u) = z^{\tw}(u)
                \end{equation}
            for some invertible \textit{even} scalar series $z^{\tw}(u) \in \calX^{\tw}(\g_N, \vartheta)$.
        \end{lemma}
            \begin{proof}
                
            \end{proof}
        \begin{lemma}[Centres of extended twisted Yangians] \label{lemma: centres_of_extended_twisted_yangians}
            \begin{enumerate}
                \item The coefficients of the twisted quantum contraction $z^{\tw}(u)$ from equation \eqref{equation: twisted_quantum_contraction} generated the centre $\calZ^{\tw}(\g_N, \vartheta) := \rmZ( \calX^{\tw}(\g_N, \vartheta) )$.
                \item Moreover, the even coefficients of $z^{\tw}(u)$ are algebraically independent from one another, and thus:
                    $$\calX^{\tw}(\g_N, \vartheta) \cong \calY^{\tw}(\g_N, \vartheta) \tensor \calZ^{\tw}(\g_N, \vartheta)$$
            \end{enumerate}
        \end{lemma}
            \begin{proof}
                \begin{enumerate}
                    \item See \cite[Corollary 3.5]{guay_regelskis_twisted_yangians_for_symmetric_pairs_of_types_BCD}.
                    \item See \cite[Corollary 3.6]{guay_regelskis_twisted_yangians_for_symmetric_pairs_of_types_BCD}.
                \end{enumerate}
            \end{proof}