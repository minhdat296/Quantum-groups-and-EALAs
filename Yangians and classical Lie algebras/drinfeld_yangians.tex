\section{Untwisted Yangians}
    \begin{convention} \label{conv: a_fixed_semi_simple_lie_algebra}
        We fix once and for all a finite-dimensional simple Lie algebra $\g$ over $\bbC$ (or for that matter, any algebraically closed field of characteristic $0$) along with a \textit{non-degenerate} invariant symmetric $\bbC$-bilinear form $(-, -)_{\g}$. Set:
            $$n := \dim \g$$
            
        Denote the undirected Dynkin graph associated to $\g$ by $\Gamma$ and suppose that it has $l$ vertices. Denote its Cartan matrix by $C$ and for this matrix, a symmetrisation:
            $$C := DA$$
        with $D$ invertible and diagonal, and $A$ symmetric. Denote the root system of $\Gamma$ by $\Phi$ and choose a subset of simple roots $\simpleroots := \{\alpha_i\}_{1 \leq i \leq l}$ therein. 
        
        Fix once and for all the following set of Chevalley-Serre generators for $\g$:
            $$\{h_i, x_i^{\pm}\}_{i \in \simpleroots}$$
        (here, we are identifying the set of vertices of $\Gamma$ with $\simpleroots$) whose elements are normalised so that:
            $$(x_i^-, x_i^+)_{\g} = 1$$
    \end{convention}

    There is an $\Z_{\geq 0}$-graded Hopf algebra over $\bbC[\hbar]$, often denoted by $\calY_{\hbar}(\g)$, is a quantisation of a certain Lie bialgebra structure on the so-called \textbf{current algebra}:
        $$\g[t] := \g \tensor_{\bbC} \bbC[t]$$
    We shall refer to $\calY_{\hbar}(\g)$ as the \textbf{formal untwisted Yangian} in order to distinguish it from its \say{special fibre} at any $\hbar_0 \in \bbC^{\x}$ (and we might as well set $\hbar_0 := 1$):
        $$\calY(\g) := \calY_{\hbar}(\g)/(\hbar - \hbar_0)\calY_{\hbar}(\g)$$
    which shall be known to us as \say{the} \textbf{untwisted Yangian}. The former shall turn out to be the Rees algebra of the latter, with respect to a natural $\Z_{\geq 0}$-grading thereon. Furthermore, it will turn out to be the case that:
        $$\calU( \g[t] ) \cong \gr \calY(\g) \cong \calY_{\hbar}(\g)/\hbar\calY_{\hbar}(\g)$$
    The last isomorphism is just a general property of the Rees algebra construction, though it is worth keeping in mind as it implies to us - in conjunction with other evidences - that not only is $\calY_{\hbar}(\g)$ a deformation of $\calU( \g[t] )$ but also, via the first isomorphism, that $\calY(\g)$ is actually a PBW deformation of $\calU( \g[t] )$, while $\calY_{\hbar}(\g)$ is a graded deformation, both in the sense of definition \ref{def: graded_and_PBW_deformations}. We caution the reader that, at this point in our discussion, both $\calY_{\hbar}(\g)$ and $\calY(\g)$ are merely deformations of associative algebras, not yet quantisations. 

    Now, there are two equivalent presentations for $\calY_{\hbar}(\g)$ as an associative, each serving different practical purposes. The first, commonly referred to as \say{Drinfeld's first presentation}, makes it easy (or indeed, possible!) to explicitly write down what the Hopf structure on $\calY_{\hbar}(\g)$ that quantises the aforementioned Lie bialgebra structure on $\g[t]$ is; $\g[t]$ is then realisable as the classical limit as $\hbar \to 0$ of this quantisation by virtue of being the bi-ideal of primitive elements of this Hopf algebra. At the same time, there is a second presentation in terms of generators $X_{i, r}^{\pm}, H_{i, r}$ that very much resemble the Chevalley-Serre generators of $\g$. Naturally, this second presentation is very useful for studying $\calY_{\hbar}(\g)$-modules and in particular, it was by using a slight modification of this presentation that Levendorskii was able to construct a PBW basis for $\calY_{\hbar}(\g)$, which is essential for defining notions such as highest-weight modules and Verma modules. At the same time, one major drawback with working with it is that as of right now, it is still not known how one might explicitly compute comultiplication formulae in terms of the generators $X_{i, r}^{\pm}, H_{i, r}$. 

    There is another presentation, the so-called \say{RTT presentation}, that is rather explicit and easy to work with. Originally, this presentation was only known to hold for $\g \cong \sl_{l + 1}$, but has now been shown to hold for all finite-dimensional simple Lie algebras $\g$. This presentation is almost exclusively for the purpose of representation theory, so we will defer discussing it until subsection \ref{subsection: RTT_presentation_for_untwisted_yangians}. This presentation also historically preceded the other ones.

    \subsection{untwisted Yangians as PBW deformations}
        Contrary to our description of the formal untwisted Yangian $\calY_{\hbar}(\g)$ in the introductory segment, where we were thinking of it as a quantisation of a Lie bialgebra structure on $\g[t]$, let us begin by taking the following to be our working definition of $\calY_{\hbar}(\g)$, merely as an associative $\bbC[\hbar]$-algebra for now.

        \begin{convention}
            We shall be using the following shorthand:
                $$\{ X_1, ..., X_n \} := \sum_{\sigma \in S_n} X_{\sigma(1)} \cdot ... \cdot X_{\sigma(n)}$$
        \end{convention}
        \begin{definition}[Formal untwisted Yangians] \label{def: formal_untwisted_yangians}
            (Cf. \cite[Definition 2.1]{guay_nakajima_wendlandt_affine_yangian_coproduct}) The \textbf{formal untwisted Yangian} of $\g$, denoted by $\calY_{\hbar}(\g)$, is the associative $\bbC[\hbar]$ generated by the set:
                $$\{ H_{i, r}, X_{i, r}^{\pm} \}_{(i, r) \in \simpleroots \x \Z_{\geq 0}}$$
            whose elements are subjected to the following relations\footnote{Note the similarities with the Chevalley-Serre presentation of $\g$ in terms of the generating set $\{ h_i, x_i^{\pm} \}_{i \in \simpleroots}$.}, given for all $(i, r), (j, s) \in \simpleroots \x \Z_{\geq 0}$:
                $$[ H_{i, r}, H_{j, s} ] = 0$$
                $$[ X_{i, r}^+, X_{j, s}^- ] = \pm \delta_{ij} H_{i, r + s}$$
                $$[ H_{i, 0}, X_{j, s}^{\pm} ] = \pm C_{ij} X_{j, s}^{\pm}$$
                $$[ H_{i, r + 1}, X_{j, s}^{\pm} ] - [ H_{i, r}, X_{j, s + 1}^{\pm} ] = \pm \frac12 \hbar C_{ij} \{H_{i, r}, X_{j, s}^{\pm}\}$$
                $$[ X_{i, r + 1}, X_{j, s}^{\pm} ] - [ X_{i, r}, X_{j, s + 1}^{\pm} ] = \pm \frac12 \hbar C_{ij} \{X_{i, r}, X_{j, s}^{\pm}\}$$
                $$\sum_{\sigma \in S_{N_{ij}}} \ad(X_{i, r_{\sigma(1)}}) \cdot ... \cdot \ad(X_{i, r_{\sigma(N_{ij})}}) \cdot X_{j, s}^{\pm} = 0, N_{ij} := 1 - A_{ij}$$
                
            We will be referring to this presentation as the \textbf{Chevalley-Serre presentation} due to its similarity to the presentation of the same name for $\g$ (cf. \cite[Section 8]{humphreys_lie_algebras}). 
        \end{definition}
        \begin{definition}[untwisted Yangians] \label{def: untwisted_yangians}
            The \textbf{untwisted Yangian} of $\g$, denoted by $\calY(\g)$, is given by:
                $$\calY(\g) := \calY_{\hbar}(\g)/\calY_{\hbar}(\g) \cdot (\hbar - 1)$$
        \end{definition}
        \begin{remark}[The degree grading/filtration on $\calY_{\hbar}(\g)$ and $\calY(\g)$] \label{remark: the_degree_grading_on_untwisted_yangians}
            There is a natural $\Z_{\geq 0}$-grading on $\calY_{\hbar}(\g)$ (and likewise, on $\calY(\g)$) given by:
                $$\deg \hbar := 1$$
                $$\deg X_{i, r}^{\pm} = \deg H_{i, r} := r$$
            The vector subspace of $\calY_{\hbar}(\g)$ (respectively, of $\calY(\g)$) consisting of degree-$r$ elements, i.e. the degree-$r$ graded component, shall be:
                $$\calY_{\hbar}(\g)_r := \span_{\bbC} \{ X \in \calY_{\hbar}(\g) \mid \deg X = r \}$$
            (and respectively, filtrants $\calY(\g)_r := \span_{\bbC} \{ X \in \calY(\g) \mid \deg X = r \}$) and one checks that, indeed:
                $$\calY_{\hbar}(\g)_r \calY_{\hbar}(\g)_s \subseteq \calY_{\hbar}(\g)_{r + s}$$
            (and likewise, $\calY(\g)_r \calY(\g)_s \subseteq \calY(\g)_{r + s}$). It is easy to see that $\calU(\g)$ is isomorphic to the subalgebra $\calY_{\hbar}(\g)_0$ of $\calY(\g)$ generated by the set:
                $$\{ H_{i, 0}, X_{i, 0}^{\pm} \}_{i \in \simpleroots}$$
        \end{remark}

        It turns out that one can give a more minimalistic presentation for $\calY_{\hbar}(\g)$ mostly in terms of the low-degree generators $H_{i, 0}, X_{i, 0}^{\pm}$. This was originally due to Levendorskii and is useful for explicitly writing down untwisted Yangians of specific instances of $\g$ (e.g. $\g \cong \sl_2$; see example \ref{example: Y(sl_2)}).
        \begin{theorem}[Levendorskii's presentation] \label{theorem: levendorskii_presentation}
            \cite[Theorem 1.2]{levendorskii_finite_type_yangians_presentation} The formal untwisted Yangian $\calY_{\hbar}(\g)$ of $\g$ will be isomorphic to the associative $\bbC$-algebra generated by the set:
                $$\{ H_{i, r}, X_{i, r}^{\pm} \}_{(i, r) \in \simpleroots \x \Z_{\geq 0}}$$
            whose elements are subjected to the following relations:
                $$H_{i, 0} = h_i, X_{i, 0}^{\pm} = x_i^{\pm}$$
                $$[ H_{i, r}, H_{j, s} ] = 0$$
                $$[ X_{i, r}^+, X_{j, s}^- ] = \pm \delta_{ij} H_{i, r + s}$$
                $$[ H_{i, 0}, X_{j, s}^{\pm} ] = \pm C_{ij} X_{j, s}^{\pm}$$
                $$\left[ H_{i, 1} - \frac12 H_{i, 0}^2, X_{j, 0}^{\pm} \right] = \pm \hbar C_{ij} X_{j, 1}^{\pm}$$
                $$[ X_{i, 1}^{\pm}, X_{j, 0}^{\pm} ] - [ X_{i, 0}^{\pm}, X_{j, 1}^{\pm} ] = \pm \frac12 \hbar C_{ij} \{X_{i, 0}^{\pm}, X_{j, 0}^{\pm}\}$$
        \end{theorem}
        \begin{remark}
            An interesting merit of Levendorskii's presentation is that, in contrast with the presentation given in definition \ref{def: formal_untwisted_yangians}, the final three relations are given only for $r, s = 0$. Also, thanks to the relations:
                $$H_{i, 0} = h_i, X_{i, 0}^{\pm} = x_i^{\pm}$$
            (given for all $i \in \simpleroots$), it is no longer necessary to impose Serre relations (e.g. the last relation in the definition of $\calY_{\hbar}(\g')$ as in \ref{def: formal_untwisted_yangians}), since this has already been guaranteed by the Serre relations of $\g$, namely:
                $$\ad(x_i^{\pm})^{1 - C_{ij}}( x_j^{\pm} ) = 0$$
            (cf. \cite{humphreys_lie_algebras}).
        \end{remark}
        \begin{example}[$\calY_{\hbar}(\sl_2)$] \label{example: Y(sl_2)}
            Consider, for example, the case:
                $$\g \cong \sl_2$$
            and let us write $h, x^{\pm}$ for the $\sl_2$-triple. In this case, $\calY_{\hbar}(\sl_2)$ shall be isomorphic to the associative $\bbC$-algebra generated by the set:
                $$\{ H_r, X_r^{\pm} \}_{r \in \Z_{\geq 0}}$$
            whose elements satisfy the following relations\footnote{In this case, there isn't even any Serre relation for $\sl_2$ in the background!}, given for all $r, s \in \Z_{\geq 0}$:
                $$H_0 = h, X_0^{\pm} = x^{\pm}$$
                $$[ H_r, H_s ] = 0$$
                $$[ X_r^+, X_s^- ] = \pm H_{r + s}$$
                $$[ H_0, X_s^{\pm} ] = \pm 2 X_s^{\pm}$$
                $$\left[ H_1 - \frac12 H_0^2, X_0^{\pm} \right] = \pm 2 \hbar X_1^{\pm}$$
                $$[ X_1^{\pm}, X_0^{\pm} ] - [ X_0^{\pm}, X_1^{\pm} ] = \pm 2\hbar (X_0^{\pm})^2$$
        \end{example}

        \begin{definition}[Graded and PBW deformations] \label{def: graded_and_PBW_deformations}
            Fix an $\Z_{\geq 0}$-graded associative algebra $U_0 := \bigoplus_{r \in \Z_{\geq 0}} U_r$ over a field $k$. An \textbf{$\Z_{\geq 0}$-graded deformation} of such an algebra $U_0$ is then an $\Z_{\geq 0}$-graded associative $k[\hbar]$-algebra $U_{\hbar}$, free as a $k[\hbar]$-module, and such that:
                $$U_{\hbar}/\hbar U_{\hbar} \cong U_0$$
            Now, fix some $\hbar_0 \in k^{\x}$. The algebra:
                $$U_{\hbar_0} := U_{\hbar}/(\hbar - \hbar_0)U_{\hbar}$$
            is then called the \textbf{PBW deformation} of $U_0$ at $\hbar_0$.  
        \end{definition}
        \begin{convention}
            Suppose that $\alpha \in \Phi^+$ is a positive root of $\g$ that can be written as a sum of simple roots in the following manner:
                $$\alpha := \sum_m \alpha_{i_m}, 1 \leq i_m \leq l$$
            Fix also a natural number $r \in \Z_{\geq 0}$ along with a partition:
                $$r := \sum_m r_m, r_m \in \Z_{\geq 0}$$
            From these data, let us define:
                $$X_{\alpha, r}^{\pm} := c_{\alpha, r} \ad( X_{i_m, r_m}^{\pm} ) \cdot ... \cdot \ad( X_{i_{m - 1}, r_{m - 1}}^{\pm} ) \cdot X_{i_m, r_m}^{\pm}$$
            for some $c_{\alpha, r} \in \bbC$.
        \end{convention}
        \begin{theorem}[PBW bases for formal untwisted Yangians] \label{theorem: PBW_bases_for_formal_yangians}
            Fix a total ordering on the set:
                $$\Sigma := \{X_{\alpha, r}^{\pm}\}_{(\alpha, r) \in \Phi^+ \x \Z_{\geq 0}} \cup \{H_{i, r}\}_{(i, r) \in \simpleroots \x \Z_{\geq 0}}$$
            Then, the set of all \textit{ordered} monomials in elements of $\Sigma$ forms a basis for $\calY_{\hbar}(\g)$ as a $\bbC[\hbar]$-module; we refer to such a basis as a \textbf{PBW basis} of $\calY_{\hbar}(\g)$. 
        \end{theorem}
        \begin{corollary}[Formal untwisted Yangians as graded deformations] \label{coro: formal_yangians_as_graded_deformations}
            $\calY_{\hbar}(\g)$ is a free $\bbC[\hbar]$-module on the set of ordered monomials in elements of $\Sigma$. Also, there is an isomorphism of $\Z_{\geq 0}$-graded $\bbC[\hbar]$-algebras:
                $$\Rees_{\hbar} \calY(\g) \cong \calY_{\hbar}(\g)$$
        \end{corollary}

        What follows can hold independently of theorem \ref{theorem: PBW_bases_for_formal_yangians}. 
        \begin{lemma}[Finite-type untwisted Yangians as PBW deformations] \label{lemma: untwisted_yangians_as_PBW_deformations}
            (Cf. \cite[Proposition 12.1.6]{chari_pressley_quantum_groups}) Define an $\Z_{\geq 0}$-grading on $\calU(\g[t])$ by setting:
                $$\deg x t^r := r$$
            for any $x \in \g$ and any $r \in \Z_{\geq 0}$. With respect to the $\Z_{\geq 0}$-grading on $\calY(\g)$ from remark \ref{remark: the_degree_grading_on_untwisted_yangians}, one has an isomorphism of associative $\bbC$-algebras:
                $$\calU( \g[t] ) \to \gr \calY(\g)$$
            As such, $\calY(\g)$ is a PBW deformation of $\calU(\g[t])$ in the sense of definition \ref{def: graded_and_PBW_deformations}.
        \end{lemma}
            \begin{proof}
                Using definition \ref{def: formal_untwisted_yangians}, we see that the associative $\bbC$-algebra $\gr \calY(\g)$ is generated by the set:
                    $$\{ H_{i, r}, X_{i, r}^{\pm} \}_{(i ,r) \in \simpleroots \x \Z_{\geq 0}}$$
                whose elements are subjected to the following relations, given for all $(i, r), (j, s) \in \simpleroots \x \Z_{\geq 0}$:
                    $$[ H_{i, r}, H_{j, s} ] = 0$$
                    $$[ X_{i, r}^+, X_{j, s}^- ] = \pm \delta_{ij} H_{i, r + s}$$
                    $$[ H_{i, 0}, X_{j, s}^{\pm} ] = \pm C_{ij} X_{j, s}^{\pm}$$
                    $$[ H_{i, r + 1}, X_{j, s}^{\pm} ] - [ H_{i, r}, X_{j, s + 1}^{\pm} ] = 0$$
                    $$[ X_{i, r + 1}, X_{j, s}^{\pm} ] - [ X_{i, r}, X_{j, s + 1}^{\pm} ] = 0$$
                    $$\sum_{\sigma \in S_{N_{ij}}} \ad(X_{i, r_{\sigma(1)}}^{\pm}) \cdot ... \cdot \ad(X_{i, r_{\sigma(N_{ij})}}^{\pm}) \cdot X_{j, s}^{\pm} = 0, N_{ij} := 1 - A_{ij}$$
                These are nothing but the relations defining $\calU( \g[t] )$ (if one identifies $H_{i, r} \mapsto h_i t^r$ and $X_{i, r}^{\pm} \mapsto x_i^{\pm} t^r$), and so we have an isomorphism of associative $\bbC$-algebras:
                    $$\calU( \g[t] ) \cong \gr \calY(\g)$$ 
            \end{proof}
        \begin{corollary}[Centres of untwisted Yangians] \label{coro: centres_of_untwisted_yangians}
            The centre of the untwisted Yangian $\calY(\g)$ consists of scalar multiples of the multiplicative identity $1$, i.e.:
                $$\rmZ( \calY(\g) ) \cong \bbC 1$$
        \end{corollary}
            \begin{proof}
                Set $\calY^{-1}(\g) := 0$ and consider an arbitrary element $z \in \calY^r(\g) \setminus \calY^{r - 1}(\g)$ for any $r \in \Z_{\geq 0}$. Per lemma \ref{lemma: untwisted_yangians_as_PBW_deformations}, one sees that the image of $z$ under the canonical morphism $\calY(\g) \to \gr \calY(\g)$ is a central element of $\calU(\g[t])$. It now remains to show that in fact:
                    $$\rmZ( \calU(\g[t]) ) \cong \bbC 1$$
                \todo[inline]{Give a citation for this. Perhaps \cite{chari_pressley_quantum_groups} ?}
            \end{proof}

        Finally, let us discuss an analogue of the PBW theorem for untwisted Yangians.
        \begin{theorem}[PBW bases for untwisted Yangians] \label{theorem: PBW_bases_for_untwisted_yangians}
            \cite[Proposition 12.1.8]{chari_pressley_quantum_groups} Fix a total ordering on the set:
                $$\Sigma := \{X_{\alpha, r}^{\pm}\}_{(\alpha, r) \in \Phi^+ \x \Z_{\geq 0}} \cup \{H_{i, r}\}_{(i, r) \in \simpleroots \x \Z_{\geq 0}}$$
            Then, the set of all \textit{ordered} monomials in elements of $\Sigma$ forms a basis for $\calY(\g)$ as a $\bbC$-vector space; we refer to such a basis as a \textbf{PBW basis} of $\calY(\g)$. 
        \end{theorem}
        \begin{corollary}[Triangular decompositions of untwisted Yangians] \label{coro: triangular_decompositions_of_untwisted_yangians}
            Fix a total ordering on the set:
                $$\Sigma := \{X_{\alpha, r}^{\pm}\}_{(\alpha, r) \in \Phi^+ \x \Z_{\geq 0}} \cup \{H_{i, r}\}_{(i, r) \in \simpleroots \x \Z_{\geq 0}}$$
            The sets of totally ordered monomials in:
                $$\Sigma^{\pm} := \{X_{\alpha, r}^{\pm}\}_{(\alpha, r) \in \Phi^+ \x \Z_{\geq 0}}$$
            and in:
                $$\Sigma^0 := \{H_{i, r}\}_{(i, r) \in \simpleroots \x \Z_{\geq 0}}$$
            respectively, form bases for $\calY(\g)^{\pm}$ and for $\calY(\g)^0$ as $\bbC$-vector spaces. As such, one obtains a triangular decomposition:
                $$\calY(\g) \cong \calY(\g)^- \tensor_{\bbC} \calY(\g)^0 \tensor_{\bbC} \calY(\g)^+$$
            The associative $\bbC$-subalgebras $\calY(\g)^{\pm}, \calY(\g)^0$ of $\calY(\g)$ are therefore PBW deformations of the universal enveloping algebras $\calU(\n^{\pm}[t]), \calU(\h[t])$ respectively. 
        \end{corollary}
        \begin{remark}
            Since $\h$ is abelian, we actually have that:
                $$\calY(\g)^0 \cong \calU(\h[t]) \cong \Sym(\h[t])$$
            See the discussion at the beginning of subsection \ref{subsection: yangians_of_reductive_lie_algebras} for more details on this phenomenon. 
        \end{remark}

    \subsection{untwisted Yangians as quantisations}
        Let us now move on towards the discussion surrounding Drinfeld's original presentation, given for both the formal untwisted Yangian $\calY_{\hbar}(\g)$ and for $\calY(\g)$. To begin, let us give a description of the Lie bialgebra structure on $\g[t]$ of which the formal untwisted Yangian $\calY_{\hbar}(\g)$ is a quantisation. We assume that the reader is familiar with the relationship between Manin triples and Lie bialgebras.

        \begin{convention}
            Fix an orthonormal basis $\{x_{\lambda}\}_{1 \leq \lambda \leq n}$ for $\g$, with respect to the inner product $(-, -)_{\g}$. Additionally, we shall be needing the Casimir element:
                $$\sfr_{\g} := \sum_{1 \leq \lambda \leq n} x_{\lambda} \tensor x_{\lambda}^*$$
            with $x_{\lambda}^*$ denoting the dual basis vectors with respect to $(-, -)_{\g}$ (though recall that the value of $\sfr_{\g}$ is actually basis-independent).
        \end{convention}
        \begin{convention}
            From now on, we will write $(-)^*$ to mean linear duals, while $(-)^{\star}$ to mean graded duals. 
        \end{convention}
        \begin{convention}
            If $k$ is a commutative ring and $A$ is a $k$-algebra, and if $L$ is a Lie algebra over $k$, then the default Lie algebra structure on the $k$-module $L \tensor_k A$ shall be the one given by extension of scalars, i.e.:
                $$[x \tensor a, y \tensor b]_{L \tensor_k A} := [x, y]_L \mu_{A/k}(a \tensor b)$$
            (here, $\mu_{A/k}: A \tensor_k A \to A$ is the $k$-linear multiplication map). $L \tensor_k A$ is usually regarded as Lie algebra over $k$ instead of over $A$.  
        \end{convention}
        
        \begin{proposition}[The untwisted Yangian Manin triple] \label{prop: the_yangian_manin_triple}
            There is a $\Z$-graded Manin triple:
                $$( \g[t^{\pm 1}], \g[t], t^{-1}\g[t^{-1}] )$$
            wherein $\g[t^{\pm 1}]$ is equipped with the following \textit{a priori} invariant inner product, given for all $x, y \in \g$ and all $r, s \in \Z$:
                $$(x t^r, y t^s)_{\g[t^{\pm 1}]} := (x, y)_{\g} \delta_{r + s, -1}$$
            Corresponding to this Manin triple is a topological Lie bialgebra structure on $\g[t]$, wherein the Lie cobracket:
                $$\delta: \g[t] \to \g[t] \hattensor_{\bbC} \g[t']$$
            is given by:
                $$\delta(X(t)) = [ X(t) \tensor 1 + 1 \tensor X(t'), \scrR_{\g[t]} ]$$
            for all $X(t) \in \g[t]$, wherein $\scrR_{\g[t]} := \sfr_{\g} \frac{1}{t' - t} \in \g[t] \hattensor_{\bbC} \g[t']$, with $\frac{1}{t' - t}$ being understood as a shorthand for a geometric series.
        \end{proposition}
        \begin{theorem}[Drinfeld's first presentation] \label{theorem: drinfeld_current_presentation}
            (Cf. \cite[Theorem 12.1.3]{chari_pressley_quantum_groups}) The formal untwisted Yangian $\calY_{\hbar}(\g)$ is isomorphic, as an $\Z_{\geq 0}$-graded associative $\bbC[\hbar]$-algebra to the $\bbC[\hbar]$-algebra $Y$ generated by the set:
                $$\{ x_{\lambda}, y_{\lambda} \}_{1 \leq \lambda \leq n}$$
            whose elements subjected to the following relations:
                $$[ x_{\lambda}, x_{\mu} ] = \sum_{1 \leq \lambda \leq n} c_{\lambda \mu \nu} x_{\nu}, [ x_{\lambda}, y_{\mu} ] = \sum_{1 \leq \lambda \leq n} c_{\lambda \mu \nu} y_{\nu}$$
                $$[ y_{\lambda}, [y_{\mu}, x_{\nu}] ] - [ x_{\lambda}, [y_{\mu}, y_{\nu}] ] = \hbar^2 \sum_{1 \leq \alpha, \beta, \gamma \leq n} a_{\lambda \mu \nu \alpha \beta \gamma} \{ x_{\alpha}, x_{\beta}, x_{\gamma} \}$$
                $$[ [y_{\lambda}, y_{\mu}], [x_r, x_s] ] + [ [y_r, y_s], [x_{\lambda}, x_{\mu}] ] = \hbar^2 \sum_{1 \leq \alpha, \beta, \gamma, \nu \leq n} ( a_{\lambda \mu \nu \alpha \beta \gamma} c_{r s \nu} + a_{r s \nu \alpha \beta \gamma} c_{\lambda \mu \nu} ) \{ x_{\alpha}, x_{\beta}, x_{\gamma} \}$$
            wherein $c_{\cdot \cdot \cdot}$ are the structural constants of $\g$, and:
                $$a_{\lambda \mu \nu \alpha \beta \gamma} = \frac{1}{24} \sum_{1 \leq i, j, k \leq n} c_{\lambda \alpha i} c_{\mu \beta j} c_{\nu \gamma k} c_{i j k}$$
            and we set:
                $$\deg x_{\lambda} := 0, \deg y_{\lambda} := 1$$
            for all $1 \leq \lambda \leq n$.
            
            Denote the isomorphism in question by $\varphi: Y \to \calY_{\hbar}(\g)$. It is given by:
                $$\varphi(h_i) = D_{ii}^{-1} H_{i, 0}, \varphi(h_i t) = D_{ii}^{-1} H_{i, 0} + \varphi(v_i)$$
                $$\varphi(x_i^{\pm}) = D_{ii}^{-1} X_{i, 0}^{\pm}, \varphi(h_i t) = D_{ii}^{-1} H_{i, 0} + \varphi(w_i^{\pm})$$
            wherein:
                $$v_i := -\frac12 D_{ii} h_i^2 + \frac14 \sum_{\alpha \in \Phi^+} \length(\alpha)^2 D_{ii}^{-1} \alpha(h_i) \{e_{\alpha}^+, e_{\alpha}^-\}$$
                $$w_i^{\pm} := -\frac12 D_{ii} \{x_i^{\pm}, h_i\} + \frac14 \sum_{\alpha \in \Phi^+} \length(\alpha)^2 D_{ii}^{-1} \alpha(h_i) \{[x_i^{\pm}, e_{\alpha}^{\pm}], e_{\alpha}^{\mp}\}$$
            with choices of roots vectors $e_{\alpha}^{\pm} \in \g_{\pm \alpha}$ such that $(e_{\alpha}^-, e_{\alpha}^+)_{\g} = 1$. One notes also that $\varphi$ respects the $\Z_{\geq 0}$-grading on both algebras.
        \end{theorem}

        \begin{theorem}[untwisted Yangian Hopf structure] \label{theorem: yangian_hopf_structure}
            There is an $\Z_{\geq 0}$-graded Hopf $\bbC[\hbar]$-algebra structure $(\Delta_{\hbar}, S_{\hbar}, \e_{\hbar})$ on $\calY_{\hbar}(\g)$ given by:
                $$\Delta_{\hbar}(x_{\lambda}) := x_{\lambda} \tensor 1 + 1 \tensor x_{\lambda}$$
                $$\Delta_{\hbar}(y_{\lambda}) := y_{\lambda} \tensor 1 + 1 \tensor y_{\lambda} + \frac12 \hbar [x_{\lambda} \tensor 1, \sfr_{\g}]$$
                $$S_{\hbar}(x_{\lambda}) = -x_{\lambda}, S_{\hbar}(y_{\lambda}) = -y_{\lambda} + \frac14 c_{} x_{\lambda}$$
                $$\e_{\hbar}(x_{\lambda}) = \e_{\hbar}(y_{\lambda}) = 0$$
            with $c$ being the eigenvalue of $\ad(\sfr_{\g})$. This induces an $\Z_{\geq 0}$-graded Hopf $\bbC$-algebra structure $(\Delta_1, S_1, \e_1)$ on $\calY(\g)$.
        \end{theorem}
        \begin{corollary}[untwisted Yangians as quantisations] \label{coro: yangians_as_quantisations}
            $\calY_{\hbar}(\g)$ is a quantisation of the Lie bialgebra structure $\delta$ on $\g[t]$ as in proposition \ref{prop: the_yangian_manin_triple}.
        \end{corollary}
            \begin{proof}
                From lemma \ref{coro: formal_yangians_as_graded_deformations}, we know that:
                    $$\calY_{\hbar}/\hbar\calY_{\hbar}(\g) \cong \calU(\g[t])$$
                so it only remains to show that:
                    $$\frac{1}{\hbar}( \Delta_{\hbar} - \Delta_{\hbar}^{\cop} ) \equiv \delta \pmod{\hbar}$$
                This is easy to check on the generators $x_{\lambda}$, so we focus on the generators $y_{\lambda}$. Since $\deg x_{\lambda} = 0$ while $\deg y_{\lambda} = 1$, we can take:
                    $$y_{\lambda} \equiv x_{\lambda} t \pmod{\hbar} \in \g[t]$$
                Now, consider:
                    $$\delta(x_{\lambda} t) = \left[ x_{\lambda} t + x_{\lambda} t', \frac{\sfr_{\g}}{t' - t} \right] = [x_{\lambda} \tensor 1, \sfr_{\g}] \frac{t}{t' - t} + [1 \tensor x_{\lambda}, \sfr_{\g}] \frac{t'}{t' - t}$$
                Since $\sfr_{\g}$ is an invariant element, we have that:
                    $$[1 \tensor x_{\lambda}, \sfr_{\g}] = -[x_{\lambda} \tensor 1, \sfr_{\g}]$$
                and hence:
                    $$\delta(x_{\lambda} t) = -[x_{\lambda} \tensor 1, \sfr_{\g}] \frac{t - t'}{t' - t} = [x_{\lambda} \tensor 1, \sfr_{\g}]$$
                and also that:
                    $$\frac12 \hbar ( [x_{\lambda} \tensor 1, \sfr_{\g}] - [1 \tensor x_{\lambda}, \sfr_{\g}] ) = \hbar [x_{\lambda} \tensor 1, \sfr_{\g}]$$
                We thus see clearly that:
                    $$\frac{1}{\hbar}( \Delta_{\hbar} - \Delta_{\hbar}^{\cop} )(y_{\lambda}) \equiv \delta(x_{\lambda} t) \pmod{\hbar}$$
                which is as desired. 
            \end{proof}
        \begin{remark}
            In the terminologies of \cite{wendlandt_restricted_quantum_doubles_of_yangians}, $\calY_{\hbar}(\g)$ is therefore a \textbf{homogeneous quantisation} (over $\bbC[\hbar]$) of the $\Z_{\geq 0}$-graded Lie bialgebra structure on $\g[t]$ (cf. proposition \ref{prop: the_yangian_manin_triple}). This is a consequence of a combination of lemma \ref{coro: formal_yangians_as_graded_deformations}, which states that $\calY_{\hbar}(\g)$ is an $\Z_{\geq 0}$-graded deformation of the $\bbC$-algebra $\calU(\g[t])$ and theorem \ref{theorem: yangian_hopf_structure}, which tells us that the Hopf structure on $\calY_{\hbar}(\g)$ respects the $\Z_{\geq 0}$-grading thereon. 
        \end{remark}
        \begin{question}
            How does on demonstrate \textit{explicitly} that $(\Delta_{\hbar}, S_{\hbar}, \e_{\hbar}) \pmod{\hbar}$ as in theorem \ref{theorem: yangian_hopf_structure} coincides with the usual Hopf structure on the universal enveloping algebra $\calU(\g[t])$ ?
        \end{question}
        Interestingly, the Hopf structure $(\Delta_1, \e_1, S_1)$ of $\calY(\g)$ can also be described \textit{explicitly} in terms of Levendorskii's presentation. This is yet another merit of this presentation. 
        \begin{proposition}[untwisted Yangian Hopf structure in terms of Levendorskii's presentation] \label{prop: yangian_hopf_structure_via_levendorskii_presentation}
            (Cf. \cite[Definition 4.6, Theorem 4.9, and Proposition 5.18]{guay_nakajima_wendlandt_affine_yangian_coproduct}) Fix two $\calY(\g)$-modules $(V, \rho_V), (W, \rho_W)$ in the category $\calO(\calY(\g))$. Also, let us write:
                $$\Box: \calY(\g) \to \calY(\g) \tensor_{\bbC} \calY(\g)$$
            for the map given by:
                $$\Box(X): X \tensor 1 + 1 \tensor X$$
            and note while it fails to be a $\bbC$-algebra homomorphism, it does satisfy:
                $$[ \Box(X), \Box(Y) ] = \Box( [X, Y] )$$
            There is a $\bbC$-algebra homomorphism:
                $$\Delta_{V, W}: \calY(\g) \to \End_{\bbC}(V \tensor_{\bbC} W)$$
            given on the Levendorskii generators $H_{i, 0}, X_{i, 0}^{\pm}, H_{i, 1}$ as follows
                $$\Delta_{V, V'}(H_{i, 0}) := \Box(H_{i, 0}), \Delta_{V, V'}(X_{i, 0}^{\pm}) := \Box(X_{i, 0}^{\pm})$$
                $$\Delta_{V, V'}(H_{i, 1}) := \Box(H_{i, 1}) + H_{i, 0} \tensor H_{i, 0} + [H_{i, 0} \tensor 1, \sfr_{\g}^+]$$
            wherein:
                $$\sfr_{\g}^+ := \sum_{1 \leq i \leq l} h_i \tensor h_i^* + \sum_{\alpha \in \Phi^+} e_{\alpha}^+ \tensor ( e_{\alpha}^+ )^*$$
            Moreover, $\Delta_{V, W}$ factorises into the following composition:
                $$\calY(\g) \xrightarrow[]{\Delta_1} \calY(\g) \tensor_{\bbC} \calY(\g) \xrightarrow[]{\rho_V \tensor \rho_W} \End_{\bbC}(V) \tensor_{\bbC} \End_{\bbC}(W) \to \End_{\bbC}(V \tensor_{\bbC} W)$$
            wherein the last map is the canonical one; because of this, we know how the untwisted Yangian coproduct $\Delta_1$ is given on the Levendorskii generators $H_{i, 0}, X_{i, 0}^{\pm}, H_{i, 1}$, at least as operators on $V \tensor_{\bbC} W$. 
        \end{proposition}
        \begin{remark}
            Even though the result above holds for all $\g$, the proof methods from \cite[Theorem 4.9]{guay_nakajima_wendlandt_affine_yangian_coproduct} can not handle the case $\g \cong \sl_2$. That said, modifications of these methods that can accommodate the case $\g \cong \sl_2$ are known. 
        \end{remark}

        We conclude this subsection with an example of how the different presentations for $\calY(\g)$ interplay with one another. The so-called \say{shift automorphisms} that we are about to present are to be thought of as quantum analogues of the translation automorphisms $t \mapsto t - a$.
        \begin{proposition}[The shift automorphisms] \label{prop: untwisted_yangians_shift_automorphisms}
            \cite[Proposition 12.1.5]{chari_pressley_quantum_groups} The additive abelian group $\G_a$ acts on $\calY(\g)$ via the so-called \textbf{shift automorphisms} $\tau_a$ ($a \in \G_a$), which are given by:
                $$\tau_a(H_{i, r}) := \sum_{s = 0}^r \binom{r}{s} a^{r - s} H_{i, s}$$
                $$\tau_a(X_{i, r}) := \sum_{s = 0}^r \binom{r}{s} a^{r - s} X_{i, s}$$
            Furthermore, one can verify that:
                $$S_1^2 = \tau_{c/2}$$
            with $c$ being the eigenvalue of $\ad(\sfr_{\g})$.
        \end{proposition}
            \begin{proof}[Proof sketch]
                Using Drinfeld's first presentation from theorem \ref{theorem: drinfeld_current_presentation}, one sees that:
                    $$\tau_a(x_{\lambda}) = x_{\lambda}, \tau_a(y_{\lambda}) = y_{\lambda} + a x_{\lambda}$$
                From this, it is trivial to check that for any $a \in \G_a$, the map $\tau_a$ is a Hopf $\bbC$-algebra automorphism on $\calY(\g)$. One can also check that the formula:
                    $$S_1^2 = \tau_{c/2}$$
                holds by checking on the generators $x_{\lambda}, y_{\lambda}$. 

                Now, to prove that the formulae:
                    $$\tau_a(H_{i, r}) := \sum_{s = 0}^r \binom{r}{s} a^{r - s} H_{i, s}$$
                    $$\tau_a(X_{i, r}) := \sum_{s = 0}^r \binom{r}{s} a^{r - s} X_{i, s}$$
                hold, we will have to perform induction on $r$. The base case $r = 0$ is trivial, and the case $r = 1$ follows from the fact that $\tau_a(x_{\lambda}) = x_{\lambda}, \tau_a(y_{\lambda}) = y_{\lambda} + a x_{\lambda}$ and from the isomorphism relating Drinfeld's presentation and the Chevalley-Serre presentation (cf. definition \ref{def: formal_untwisted_yangians}) as in theorem \ref{theorem: drinfeld_current_presentation}. For the inductive step, use the fourth and fifth relations from definition \ref{def: formal_untwisted_yangians} after specialising $\hbar \to 1$. 
            \end{proof}

        One can also package the untwisted Yangian generators $X_{i, r}^{\pm}, H_{i, r}$ from definition \ref{def: formal_untwisted_yangians} into \textbf{generating  series}:
            $$X_i^{\pm}(u) := 1 + \hbar \sum_{r \in \Z_{\geq 0}} X_{i, r}^{\pm} u^{-r - 1} \in 1 + u^{-1} \bbC[\![u^{-1}]\!]$$
            $$H_i(u) := 1 + \hbar \sum_{r \in \Z_{\geq 0}} H_{i, r} u^{-r - 1} \in 1 + u^{-1} \bbC[\![u^{-1}]\!]$$
        for yet another presentation of $\calY_{\hbar}(\g)$. Doing this yields a nicer description of the shift automorphisms, which makes it clearer what these maps \say{shift}. 
        \begin{proposition}[Generating series for formal untwisted Yangians] \label{prop: generating_series_for_finite_untwisted_yangians}
            (Cf. \cite[Proposition 2.3]{gautam_and_toledano_laredo_yangians_quantum_loop_algebras_and_abelian_difference_equations}) The formal untwisted Yangian $\calY_{\hbar}(\g)$ from definition \ref{def: formal_untwisted_yangians} can be recovered from the following presentation for $\calY_{\hbar}(\g)(\!(u^{-1}, v^{-1})\!)$ as the $\bbC[\![u^{-1}, v^{-1}]\!]$-algebra generated by the set:
                $$\{X_i^{\pm}(u), X_i^{\pm}(v), H_i(u), H_i(v)\}_{i \in \simpleroots}$$
            whose elements are constrained by the following relations, given for all $i, j \in \simpleroots$ and all $h, h' \in \h$:
                $$[H_i(u), H_j(v)] = 0, [h, H_i(u)] = 0$$
                $$[h, X_i^{\pm}(u)] = \pm \alpha_i(h) X_i^{\pm}(u)$$
                $$(u - v) [X_i^+(u), X_j^-(u)] = -\delta_{ij} \hbar ( H_i(u) - H_i(v) )$$
                $$\left(u - v \pm \frac12 \hbar C_{ij}\right) X_j^{\pm}(v) H_i(u) - \left(u - v \mp  \frac12 \hbar C_{ij}\right) H_i(u) X_j^{\pm}(v) = 2 \hbar C_{ij} X_j^{\pm}(u) \left(u \mp \frac12 \hbar C_{ij}\right) H_i(u)$$
                $$\left(u - v \pm \frac12 \hbar C_{ij}\right) X_j^{\pm}(v) X_i^{\pm}(u) - \left(u - v \mp \frac12 \hbar C_{ij}\right) X_i^{\pm}(u) X_j^{\pm}(v) = -\hbar( [X_{i, 0}^{\pm}, X_j^{\pm}(u)] - [X_i^{\pm}(u), X_{j, 0}^{\pm}] )$$
                $$\sum_{\sigma \in S_{N_{ij}}} \ad( X_i^{\pm}( u_{\sigma(i)} ) ) \cdot ... \cdot \ad( X_i^{\pm}( u_{\sigma(N_{ij})} ) ) \cdot X_j^{\pm}(v) = 0, N_{ij} := 1 - C_{ij}$$
        \end{proposition}
        \begin{proposition}[Shift automorphisms in terms of generating series] \label{prop: shift_automorphisms_via_generating_series}
            (Cf. \cite[Subsection 2.8]{gautam_and_toledano_laredo_yangians_quantum_loop_algebras_and_abelian_difference_equations} and \cite[Remark 2.4]{wendlandt_formal_shift_operators_on_yangian_doubles}) In terms of the generating series:
                $$X_i^{\pm}(u) := 1 + \hbar \sum_{r \in \Z_{\geq 0}} X_{i, r}^{\pm} u^{-r - 1} \in 1 + u^{-1} \bbC[\![u^{-1}]\!]$$
                $$H_i(u) := 1 + \hbar \sum_{r \in \Z_{\geq 0}} H_{i, r} u^{-r - 1} \in 1 + u^{-1} \bbC[\![u^{-1}]\!]$$
            (defined for all $i \in \simpleroots$) we have the following expressions for the shift automorphisms introduced in proposition \ref{prop: untwisted_yangians_shift_automorphisms}, given for all $a \in \bbC$ and all $i \in \simpleroots$:
                $$\tau_a(H_i(u)) = H_i(u - a)$$
                $$\tau_a(X_i^{\pm}(u)) = X_i^{\pm}(u - a)$$
        \end{proposition}
        \begin{remark}
            Proposition \ref{prop: shift_automorphisms_via_generating_series} (in conjunction with theorem \ref{theorem: yangian_hopf_structure}, of course) also rather clearly implies that:
                $$S_1^2 = \tau_{c/2}$$
            with $c$ being the eigenvalue of $\ad(\sfr_{\g})$ (cf. proposition \ref{prop: untwisted_yangians_shift_automorphisms}).
        \end{remark}

    \subsection{untwisted Yangians of reductive Lie algebras and quantum determinants} \label{subsection: yangians_of_reductive_lie_algebras}
        Now is rather convenient time to bring the reader's attention to the existence of the slightly more general construction of untwisted Yangians for reductive Lie algebra.

        The notion of reductive Lie algebras is more or less the same as that of finite-dimensional semi-simple Lie algebras, with the only notable difference being that the former have possibly non-trivial centres. In fact, it is known that any reductive Lie algebra $\s$ can be written as:
            $$\s := \g \oplus \z$$
        wherein $\g$ is a finite-dimensional semi-simple Lie algebras and $\z := \z(\s)$; for instance, we have:
            $$\gl_n \cong \sl_n \oplus \gl_1$$

        Let us observe also that the untwisted Yangian construction (as in definitions \ref{def: formal_untwisted_yangians} and \ref{def: untwisted_yangians}) works also for abelian Lie algebras, say $\z$. As there are no non-trivial relations between the generators of such Lie algebras, their current algebras $\z[t]$ are also abelian, and hence any non-trivial relation between generators of $\calY_{\hbar}(\z)$ (and hence of $\calY(\z) := \calY_{\hbar}(\z)/\calY_{\hbar}(\z) \cdot (\hbar - 1)$) can only come from the deformation of the currents. More specifically, if $\z$ is generated by:
            $$\{z_i\}_{i \in I}$$
        wherein $I$ is some set, then by mimicking definition \ref{def: formal_untwisted_yangians}, one sees that the associated formal untwisted Yangian $\calY_{\hbar}(\z)$ shall be the associative algebra generated by the set:
            $$\{Z_{i, r}\}_{(i, r) \in I \x \Z_{\geq 0}}$$
        whose elements satisfy the following relations:
            $$[ Z_{i, r}, Z_{j, s} ] = 0$$
        given for all $(i, r), (j, s) \in I \x \Z_{\geq 0}$. We see then that the left-ideal generated by $\hbar - 1$ is actually just zero, and hence:
            $$\calY_{\hbar}(\z) \cong \calY(\z) \cong \calU(\z[t]) \cong \Sym(\z[t])$$

        Now, another observation to make is that if $\g_1, \g_2$ are two Lie algebras which are either finite-dimensional simple or abelian, then:
            $$\calY_{\hbar}(\g_1 \oplus \g_2) \cong \calY_{\hbar}(\g_1) \tensor_{\bbC} \calY_{\hbar}(\g_2)$$
        and this can be proven simply by noticing that the set of generators of $\g_1 \oplus \g_2$ is nothing but the union of those of $\g_1$ and $\g_2$. From this, we see that the untwisted Yangian of a reductive Lie algebra $\s := \g \oplus \z$ can be given by:
            $$\calY_{\hbar}(\s) := \calY_{\hbar}(\g) \tensor_{\bbC} \Sym(\z[t])$$

        In corollary \ref{coro: centres_of_untwisted_yangians}, we saw that the centre of $\calY(\g)$ is actually only $1$-dimensional, but in light of the construction of untwisted Yangians for reductive Lie algebras $\s$, it is possible for such algebras to have larger centres when $\s$ is not semi-simple.

        \begin{lemma}[Centres of tensor products of associative algebras] \label{lemma: centres_of_tensor_products_of_algebras}
            Over a field $k$, if $A, B$ are associative algebras, then:
                $$\rmZ(A \tensor_k B) \cong \rmZ(A) \tensor_k \rmZ(B)$$
            as commutative algebras.
        \end{lemma}
            \begin{proof}
                Elements of $A \tensor_k B$ are finite sums of the form $\sum_{i \in I} a_i \tensor b_i$ (where $a_i \in A, b_i \in B$ for all $i \in I$), and the multiplication is given on pure tensors by:
                    $$\left( \sum_{i \in I} a_i \tensor b_i \right) \cdot \left( \sum_{j \in J} a'_j \tensor b'_j \right) := \sum_{(i, j) \in I \x J} a_i a'_j \tensor b_i b'_j$$
                From this, we see that commutators in $A \tensor_k B$ are given by:
                    $$\left[ \sum_{i \in I} a_i \tensor b_i, \sum_{j \in J} a'_j \tensor b'_j \right] = \sum_{(i, j) \in I \x J} [a_i, a'_j] \tensor [b_i, b'_j]$$
                Suppose now that the element $\sum_{i \in I} a_i \tensor b_i \in A \tensor_k B$ is arbitrary but fixed. Then, such a commutator vanishes if and only if $a'_j \in \rmZ(A)$ and $b'_j \in \rmZ(B)$ simultaneously for all $j \in J$. From this, we see easily that:
                    $$\rmZ(A \tensor_k B) \subseteq \rmZ(A) \tensor_k \rmZ(B)$$
                The other inclusion is trivial to verify.

                Also, let us remark that it is absolutely necessary that we work over a field so that the lemma would hold true in general. Consider, for instance, the following counter-example, where the issue ultimately is due to torsion. Let $p$ be a prime number, and let $A := \frac{\Q\<x, \del\>}{\<[x, \del] - p\>}$ and $B := \F_p$ be considered as $\Z$-algebras; note that the latter is not flat over $\Z$. Then:
                    $$\rmZ(A \tensor_{\Z} B) \cong \rmZ\left( \frac{\Q\<x, \del\>}{\<[x, \del] - p\>} \tensor_{\Z} \F_p \right) \cong \rmZ\left( \frac{\Q\<x, \del\>}{\<[x, \del]\>} \right) \cong \rmZ(\Q[x, \del]) = \Q[x, \del]$$
                    $$\rmZ(A) \tensor_{\Z} \rmZ(B) \cong \rmZ\left( \frac{\Q\<x, \del\>}{\<[x, \del] - p\>} \right) \tensor_{\Z} \rmZ(\F_p) \cong \Q \tensor_{\Z} \F_p \cong 0$$
                where we got that $\rmZ\left( \frac{\Q\<x, \del\>}{\<[x, \del] - p\>} \right) \cong \Q$ from the fact that as a $\Q$-vector space, $\frac{\Q\<x, \del\>}{\<[x, \del] - p\>}$ has an ordered basis consisting of monomials of the form $x^m \del^n$ for $m, n \geq 0$.
            \end{proof}
        \begin{corollary}
            For any reductive Lie algebra $\s := \g \oplus \z$, we have that:
                $$\rmZ( \calY(\s) ) \cong \Sym(\z[t])$$
        \end{corollary}
        \begin{convention}
            For notational convenience, let us henceforth write:
                $$\calZ(\s)$$
            to mean the centre of the untwisted Yangian of a reductive Lie algebra $\s$.
        \end{convention}
            
        \begin{definition}[Quantum determinants] \label{def: quantum_determinants}
            
        \end{definition}

    \subsection{The RTT presentation for untwisted Yangians} \label{subsection: RTT_presentation_for_untwisted_yangians}
        A \say{formal distribution} in $n$ variables $z_1, ..., z_n$ with coefficients in a vector space $U$ is an element of the vector space $U[\![z_1^{\pm}, ..., z_n^{\pm}]\!]$. It is very important to note that even when $U$ is an associative algebra, $U[\![z_1^{\pm}, ..., z_n^{\pm}]\!]$ is \textit{not} an algebra in general. This is because the coefficients of monomials in a product of the form:
            $$\left( \sum_{(i_1, ..., i_n) \in \Z^n} a^{(i_1), ..., (i_n)} z_1^{i_1} ... z_n^{i_n} \right) \left( \sum_{(j_1, ..., j_n) \in \Z^n} b^{(j_1), ..., (j_n)} z_1^{j_1} ... z_n^{j_n} \right)$$
        (wherein $a^{(i_1), ..., (i_n)}, b^{(j_1), ..., (j_n)} \in U$) are of the form:
            $$\sum_{(i_1, ..., i_n) \in \Z^n} \sum_{(j_1, ..., j_n) \in \Z^n} a^{(i_1), ..., (i_n)} b^{(j_1), ..., (j_n)}$$
        which, generally speaking, shall be an infinite sum.
    
        In the single-variable case, we have the following additional notations. Given any formal distribution:
            $$a(z) := \sum_{m \in \Z} a^{(m)} z^{-m - 1}$$
        let us write:
            $$a(z)^- := \sum_{m \in \Z_{\geq 0}} a^{(m)} z^{-m - 1}, a(z)^+ := \sum_{m \in \Z_{< 0}} a^{(m)} z^{-m - 1}$$
        This is mostly to conform with notational conventions from mathematical physics, but as a side bonus, we also get:
            $$(\del a(z))^{\pm} = \del( a(z)^{\pm} )$$
    
        All throughout:
            $$\g_N \subset \gl_N$$
        will be used to denote a finite-dimensional simple Lie algebra over $\bbC$. This Lie algebra has a natural action on $U := \bbC^{\oplus N}$ (the so-called \say{vector representation}), which shall always be equipped with the standard basis:
            $$\{e_i\}_{1 \leq i \leq N}$$
        consisting of vectors with $1$ in the $i^{th}$ entry and $0$ elsewhere. By finite-dimensionality, we identify:
            $$U^{\tensor 2} \xrightarrow[]{\cong} \End(U)$$
        and then write:
            $$E_{i, j}$$
        for the image of $e_i \tensor e_j$ under the identification above; explicitly, $E_{i, j}$ is the $N \x N$ matrix with $1$ in the $(i, j)^{th}$ entry and $0$ elsewhere.

    \subsection{Uniqueness of untwisted Yangians}