\input{article preambles}

\setcounter{section}{-1}

\renewcommand{\cong}{\simeq}
\newcommand{\ladjoint}{\dashv}
\newcommand{\radjoint}{\vdash}
\newcommand{\<}{\langle}
\renewcommand{\>}{\rangle}
\newcommand{\ndiv}{\hspace{-2pt}\not|\hspace{5pt}}
\newcommand{\cond}{\blacktriangle}
\newcommand{\solid}{\blacksquare}
\newcommand{\ot}{\leftarrow}
\renewcommand{\-}{\text{-}}
\renewcommand{\mapsto}{\leadsto}
\renewcommand{\leq}{\leqslant}
\renewcommand{\geq}{\geqslant}
\renewcommand{\setminus}{\smallsetminus}
\makeatletter
\DeclareRobustCommand{\cev}[1]{%
  {\mathpalette\do@cev{#1}}%
}
\newcommand{\do@cev}[2]{%
  \vbox{\offinterlineskip
    \sbox\z@{$\m@th#1 x$}%
    \ialign{##\cr
      \hidewidth\reflectbox{$\m@th#1\vec{}\mkern4mu$}\hidewidth\cr
      \noalign{\kern-\ht\z@}
      $\m@th#1#2$\cr
    }%
  }%
}
\makeatother

\newcommand{\N}{\mathbb{N}}
\newcommand{\Z}{\mathbb{Z}}
\newcommand{\Q}{\mathbb{Q}}
\newcommand{\R}{\mathbb{R}}
\newcommand{\bbC}{\mathbb{C}}
\NewDocumentCommand{\x}{e{_^}}{%
  \mathbin{\mathop{\times}\displaylimits
    \IfValueT{#1}{_{#1}}
    \IfValueT{#2}{^{#2}}
  }%
}
\NewDocumentCommand{\pushout}{e{_^}}{%
  \mathbin{\mathop{\sqcup}\displaylimits
    \IfValueT{#1}{_{#1}}
    \IfValueT{#2}{^{#2}}
  }%
}
\newcommand{\supp}{\operatorname{supp}}
\newcommand{\im}{\operatorname{im}}
\newcommand{\coker}{\operatorname{coker}}
\newcommand{\id}{\mathrm{id}}
\newcommand{\chara}{\operatorname{char}}
\newcommand{\trdeg}{\operatorname{trdeg}}
\newcommand{\rank}{\operatorname{rank}}
\newcommand{\trace}{\operatorname{tr}}
\newcommand{\length}{\operatorname{length}}
\newcommand{\height}{\operatorname{height}}
\renewcommand{\span}{\operatorname{span}}
\newcommand{\e}{\epsilon}
\newcommand{\p}{\mathfrak{p}}
\newcommand{\q}{\mathfrak{q}}
\newcommand{\m}{\mathfrak{m}}
\newcommand{\n}{\mathfrak{n}}
\newcommand{\calF}{\mathcal{F}}
\newcommand{\calG}{\mathcal{G}}
\newcommand{\calO}{\mathcal{O}}
\newcommand{\F}{\mathbb{F}}
\DeclareMathOperator{\lcm}{lcm}
\newcommand{\gr}{\operatorname{gr}}
\newcommand{\vol}{\mathrm{vol}}
\newcommand{\ord}{\operatorname{ord}}

\newcommand{\GL}{\operatorname{GL}}
\newcommand{\SL}{\operatorname{SL}}
\newcommand{\Sp}{\operatorname{Sp}}
\newcommand{\GSp}{\operatorname{GSp}}
\newcommand{\GSpin}{\operatorname{GSpin}}
\newcommand{\opO}{\operatorname{O}}
\newcommand{\SO}{\operatorname{SO}}
\newcommand{\SU}{\operatorname{SU}}
\newcommand{\opU}{\operatorname{U}}
\newcommand{\Spec}{\mathrm{Spec}}
\newcommand{\Spf}{\mathrm{Spf}}
\newcommand{\Spm}{\mathrm{Spm}}
\newcommand{\Spv}{\mathrm{Spv}}
\newcommand{\Spa}{\mathrm{Spa}}
\newcommand{\Spd}{\mathrm{Spd}}
\newcommand{\Proj}{\mathrm{Proj}}
\newcommand{\Gr}{\mathrm{Gr}}
\newcommand{\Hecke}{\mathrm{Hecke}}
\newcommand{\Sht}{\mathrm{Sht}}
\newcommand{\Quot}{\mathrm{Quot}}
\newcommand{\Hilb}{\mathrm{Hilb}}
\newcommand{\Pic}{\mathrm{Pic}}
\newcommand{\Div}{\mathrm{Div}}
\newcommand{\Jac}{\mathrm{Jac}}
\newcommand{\Alb}{\mathrm{Alb}} %albanese variety
\newcommand{\Bun}{\mathrm{Bun}}
\newcommand{\loopspace}{\mathbf{\Omega}}
\newcommand{\suspension}{\mathbf{\Sigma}}
\newcommand{\tangent}{\mathrm{T}} %tangent space
\newcommand{\Eig}{\mathrm{Eig}}

\newcommand{\Ring}{\mathrm{Ring}}
\newcommand{\Cring}{\mathrm{CRing}}
\newcommand{\Alg}{\mathrm{Alg}}
\newcommand{\Leib}{\mathrm{Leib}} %leibniz algebras
\newcommand{\Fld}{\mathrm{Fld}}
\newcommand{\Sets}{\mathrm{Sets}}
\newcommand{\Cat}{\mathrm{Cat}}
\newcommand{\Grp}{\mathrm{Grp}}
\newcommand{\Ab}{\mathrm{Ab}}
\newcommand{\Sch}{\mathrm{Sch}}
\newcommand{\Coh}{\mathrm{Coh}}
\newcommand{\QCoh}{\mathrm{QCoh}}
\newcommand{\Desc}{\mathrm{Desc}}
\newcommand{\Sh}{\mathrm{Sh}}
\newcommand{\Psh}{\mathrm{PSh}}
\newcommand{\Fib}{\mathrm{Fib}}
\renewcommand{\mod}{\-\mathrm{mod}}
\newcommand{\comod}{\-\mathrm{comod}}
\newcommand{\bimod}{\-\mathrm{bimod}}
\newcommand{\Vect}{\mathrm{Vect}}
\newcommand{\Rep}{\mathrm{Rep}}
\newcommand{\Grpd}{\mathrm{Grpd}}
\newcommand{\Arr}{\mathrm{Arr}}
\newcommand{\Esp}{\mathrm{Esp}}
\newcommand{\Ob}{\mathrm{Ob}}
\newcommand{\Mor}{\mathrm{Mor}}
\newcommand{\Mfd}{\mathrm{Mfd}}
%\newcommand{\LR}{\mathrm{LR}}
%\newcommand{\RSpc}{\mathrm{RSpc}}
\newcommand{\Spc}{\mathrm{Spc}}
\newcommand{\Top}{\mathrm{Top}}
\newcommand{\Topos}{\mathrm{Topos}}
\newcommand{\Nil}{\mathfrak{Nil}}
\newcommand{\J}{\mathfrak{J}}
\newcommand{\Stk}{\mathrm{Stk}}
\newcommand{\Pre}{\mathrm{Pre}}
\newcommand{\simp}{\mathbf{\Delta}}
\newcommand{\Ind}{\mathrm{Ind}}
\newcommand{\Pro}{\mathrm{Pro}}
\newcommand{\Mon}{\mathrm{Mon}}
\newcommand{\Comm}{\mathrm{Comm}}
\newcommand{\Fin}{\mathrm{Fin}}
\newcommand{\Assoc}{\mathrm{Assoc}}
\newcommand{\Co}{\mathrm{Co}}
\newcommand{\Loc}{\mathrm{Loc}}
\newcommand{\Ringed}{\mathrm{Ringed}}
\newcommand{\Comp}{\mathrm{Comp}} %compact hausdorff spaces
\newcommand{\Stone}{\mathrm{Stone}} %stone spaces
\newcommand{\sfExt}{\mathrm{Ext}} %extremely disconnected spaces
\newcommand{\Ouv}{\mathrm{Ouv}}
\newcommand{\Str}{\mathrm{Str}}
\newcommand{\Func}{\mathrm{Func}}
\newcommand{\Crys}{\mathrm{Crys}}
\newcommand{\LocSys}{\mathrm{LocSys}}
\newcommand{\Sieves}{\mathrm{Sieves}}
\newcommand{\pt}{\mathrm{pt}}
\newcommand{\Graphs}{\mathrm{Graphs}}
\newcommand{\Lie}{\mathrm{Lie}}
\newcommand{\Env}{\mathrm{Env}}
\newcommand{\Ho}{\mathrm{Ho}}
\newcommand{\rmD}{\mathrm{D}}
\newcommand{\Cov}{\mathrm{Cov}}
\newcommand{\Frames}{\mathrm{Frames}}
\newcommand{\Locales}{\mathrm{Locales}}
\newcommand{\Span}{\mathrm{Span}}
\newcommand{\Corr}{\mathrm{Corr}}
\newcommand{\Monad}{\mathrm{Monad}}
\newcommand{\Var}{\mathrm{Var}}
\newcommand{\sfN}{\mathrm{N}} %nerve
\newcommand{\Dia}{\mathrm{Dia}}
\newcommand{\co}{\mathrm{co}}
\newcommand{\ev}{\mathrm{ev}}
\newcommand{\bi}{\mathrm{bi}}
\newcommand{\Nat}{\mathrm{Nat}}
\newcommand{\Hopf}{\mathrm{Hopf}}
\newcommand{\Dmod}{\mathrm{D}\mod}
\newcommand{\Perv}{\mathrm{Perv}}
\newcommand{\Sph}{\mathrm{Sph}}
\newcommand{\Moduli}{\mathrm{Moduli}}
\newcommand{\Pseudo}{\mathrm{Pseudo}}
\newcommand{\Lax}{\mathrm{Lax}}
\newcommand{\Strict}{\mathrm{Strict}}
\newcommand{\Opd}{\mathrm{Opd}} %operads
\newcommand{\Shv}{\mathrm{Shv}}
\newcommand{\Char}{\mathrm{Char}} %CharShv = character sheaves
\newcommand{\Huber}{\mathrm{Huber}}
\newcommand{\Tate}{\mathrm{Tate}}
\newcommand{\Ad}{\mathrm{Ad}} %adic spaces
\newcommand{\Perfd}{\mathrm{Perfd}} %perfectoid spaces
\newcommand{\Sub}{\mathrm{Sub}} %subobjects
\newcommand{\Ideals}{\mathrm{Ideals}}
\newcommand{\Isoc}{\mathrm{Isoc}}
\newcommand{\Ban}{\-\mathrm{Ban}} %Banach spaces
\newcommand{\Fre}{\-\mathrm{Fre}} %Frechet spaces
\newcommand{\Ch}{\mathrm{Ch}} %chain complexes
\newcommand{\Mot}{\mathrm{Mot}} %motives
\newcommand{\KL}{\mathrm{KL}} %category of Kazhdan-Lusztig modules
\newcommand{\Pres}{\mathrm{Pres}} %presentable categories
\newcommand{\Noohi}{\mathrm{Noohi}} %category of Noohi groups
\newcommand{\Inf}{\mathrm{Inf}}

\newcommand{\Aut}{\mathrm{Aut}}
\newcommand{\Inn}{\mathrm{Inn}}
\newcommand{\Out}{\mathrm{Out}}
\newcommand{\frakgl}{\mathfrak{gl}}
\newcommand{\der}{\mathfrak{der}} %derivations on Lie algebras
\newcommand{\inn}{\mathfrak{inn}} %inner derivations
\newcommand{\out}{\mathfrak{out}} %outer derivations
\newcommand{\Stab}{\mathrm{Stab}}
\newcommand{\Cent}{\mathrm{Cent}}
\newcommand{\Norm}{\mathrm{Norm}}
\newcommand{\Rad}{\mathrm{Rad}}
\newcommand{\Transporter}{\mathrm{Transp}} %transporter between two subsets of a group
\newcommand{\Conj}{\mathrm{Conj}}
\newcommand{\Diag}{\mathrm{Diag}}
\newcommand{\Gal}{\mathrm{Gal}}
\newcommand{\bfG}{\mathbf{G}} %absolute Galois group
\newcommand{\Frac}{\mathrm{Frac}}
\newcommand{\Ann}{\mathrm{Ann}}
\newcommand{\Val}{\mathrm{Val}}
\newcommand{\Chow}{\mathrm{Chow}}
\newcommand{\Sym}{\mathrm{Sym}}
\newcommand{\End}{\mathrm{End}}
\newcommand{\Mat}{\mathrm{Mat}}
\newcommand{\Diff}{\mathrm{Diff}}
\newcommand{\Autom}{\mathrm{Autom}}
\newcommand{\Artin}{\mathrm{Artin}} %artin maps
\newcommand{\sk}{\mathrm{sk}} %skeleton of a category
\newcommand{\eqv}{\mathrm{eqv}} %functor that maps groups $G$ to $G$-sets
\newcommand{\Inert}{\mathrm{Inert}}
\newcommand{\Fil}{\mathrm{Fil}}

\newcommand{\colim}{\operatorname{colim} \:}
\renewcommand{\lim}{\operatorname{lim} \:}
\newcommand{\toto}{\rightrightarrows}
%\newcommand{\tensor}{\otimes}
\NewDocumentCommand{\tensor}{e{_^}}{%
  \mathbin{\mathop{\otimes}\displaylimits
    \IfValueT{#1}{_{#1}}
    \IfValueT{#2}{^{#2}}
  }%
}
\newcommand{\eq}{\operatorname{eq}}
\newcommand{\coeq}{\operatorname{coeq}}
\newcommand{\Hom}{\mathrm{Hom}}
\newcommand{\Maps}{\mathrm{Maps}}
\newcommand{\Tor}{\mathrm{Tor}}
\newcommand{\Ext}{\mathrm{Ext}}
\newcommand{\Isom}{\mathrm{Isom}}
\newcommand{\stalk}{\mathrm{stalk}}
\newcommand{\RKE}{\operatorname{RKE}}
\newcommand{\LKE}{\operatorname{LKE}}
\newcommand{\oblv}{\mathrm{oblv}}
\newcommand{\const}{\mathrm{const}}
%\newcommand{\forget}{\mathrm{forget}}
\newcommand{\adrep}{\mathrm{ad}} %adjoint representation
\newcommand{\NL}{\mathbb{NL}} %naive cotangent complex
\newcommand{\pr}{\operatorname{pr}}
\newcommand{\Der}{\mathrm{Der}}
\newcommand{\Frob}{\mathrm{Frob}} %Frobenius
\newcommand{\frob}{\mathrm{f}} %trace of Frobenius
\newcommand{\bfpt}{\mathbf{pt}}
\newcommand{\bfloc}{\mathbf{loc}}
\DeclareMathAlphabet{\mymathbb}{U}{BOONDOX-ds}{m}{n}
\newcommand{\0}{\mymathbb{0}}
\newcommand{\1}{\mathbbm{1}}
\newcommand{\2}{\mathbbm{2}}
\newcommand{\Jet}{\mathrm{Jet}}
\newcommand{\Split}{\mathrm{Split}}
\newcommand{\Sq}{\mathrm{Sq}}
\newcommand{\Zero}{\mathrm{Z}}
\newcommand{\SqZ}{\Sq\Zero}
\newcommand{\frakLie}{\mathfrak{Lie}}
\newcommand{\y}{\mathrm{y}} %yoneda
\newcommand{\Sm}{\mathrm{Sm}}
\newcommand{\AJ}{\phi} %abel-jacobi map
\newcommand{\act}{\mathrm{act}}
\newcommand{\ram}{\mathrm{ram}} %ramification index
\newcommand{\inv}{\mathrm{inv}}

\newcommand{\bbU}{\mathbb{U}}
\newcommand{\V}{\mathbb{V}}
\newcommand{\U}{\mathrm{U}}
\newcommand{\calU}{\mathcal{U}}
\newcommand{\calW}{\mathcal{W}}
\newcommand{\rmI}{\mathrm{I}} %augmentation ideal
\newcommand{\bfV}{\mathbf{V}}
\newcommand{\C}{\mathcal{C}}
\newcommand{\D}{\mathcal{D}}
\newcommand{\T}{\mathscr{T}} %Tate modules
\newcommand{\calM}{\mathcal{M}}
\newcommand{\calN}{\mathcal{N}}
\newcommand{\calP}{\mathcal{P}}
\newcommand{\calQ}{\mathcal{Q}}
\newcommand{\A}{\mathbb{A}}
\renewcommand{\P}{\mathbb{P}}
\newcommand{\calL}{\mathcal{L}}
\newcommand{\E}{\mathcal{E}}
\renewcommand{\H}{\mathbf{H}}
\newcommand{\scrS}{\mathscr{S}}
\newcommand{\calX}{\mathcal{X}}
\newcommand{\calY}{\mathcal{Y}}
\newcommand{\calZ}{\mathcal{Z}}
\newcommand{\calS}{\mathcal{S}}
\newcommand{\calR}{\mathcal{R}}
\newcommand{\scrX}{\mathscr{X}}
\newcommand{\scrY}{\mathscr{Y}}
\newcommand{\scrZ}{\mathscr{Z}}
\newcommand{\calA}{\mathcal{A}}
\newcommand{\calB}{\mathcal{B}}
\newcommand{\sfT}{\mathrm{T}}
\renewcommand{\S}{\mathcal{S}}
\newcommand{\B}{\mathbb{B}}
\newcommand{\bbD}{\mathbb{D}}
\newcommand{\G}{\mathbb{G}}
\newcommand{\horn}{\mathbf{\Lambda}}
\renewcommand{\L}{\mathbb{L}}
\renewcommand{\a}{\mathfrak{a}}
\renewcommand{\b}{\mathfrak{b}}
\renewcommand{\t}{\mathfrak{t}}
\renewcommand{\r}{\mathfrak{r}}
\newcommand{\bbX}{\mathbb{X}}
\newcommand{\g}{\mathfrak{g}}
\newcommand{\h}{\mathfrak{h}}
\renewcommand{\k}{\mathfrak{k}}
\newcommand{\del}{\partial}
\newcommand{\bbE}{\mathbb{E}}
\newcommand{\scrO}{\mathscr{O}}
\newcommand{\bbO}{\mathbb{O}}
\newcommand{\scrA}{\mathscr{A}}
\newcommand{\scrB}{\mathscr{B}}
\newcommand{\scrF}{\mathscr{F}}
\newcommand{\scrG}{\mathscr{G}}
\newcommand{\scrM}{\mathscr{M}}
\newcommand{\scrN}{\mathscr{N}}
\newcommand{\scrP}{\mathscr{P}}
\newcommand{\frakS}{\mathfrak{S}}
\newcommand{\calI}{\mathcal{I}}
\newcommand{\calJ}{\mathcal{J}}
\newcommand{\scrK}{\mathscr{K}}
\newcommand{\calK}{\mathcal{K}}
\newcommand{\scrV}{\mathscr{V}}
\newcommand{\bbS}{\mathbb{S}}
\newcommand{\scrH}{\mathscr{H}}
\newcommand{\bfB}{\mathbf{B}}
\newcommand{\Witt}{W}
%\newcommand{\bfA}{\mathbf{A}}
\renewcommand{\O}{\mathbb{O}}
\newcommand{\calV}{\mathcal{V}}
\newcommand{\scrR}{\mathscr{R}} %radical
\newcommand{\rmZ}{\mathrm{Z}} %centre of algebra
\newcommand{\bfGamma}{\mathbf{\Gamma}}
\newcommand{\scrU}{\mathscr{U}}
\newcommand{\rmW}{\mathrm{W}} %Weil group
\newcommand{\frakM}{\mathfrak{M}}
\newcommand{\frakN}{\mathfrak{N}}
\newcommand{\frakX}{\mathfrak{X}}
\newcommand{\frakY}{\mathfrak{Y}}
\newcommand{\frakZ}{\mathfrak{Z}}

\newcommand{\aff}{\mathrm{aff}}
\newcommand{\ft}{\mathrm{ft}} %finite type
\newcommand{\fp}{\mathrm{fp}} %finite presentation
\newcommand{\aft}{\mathrm{aft}}
\newcommand{\lft}{\mathrm{lft}}
\newcommand{\laft}{\mathrm{laft}}
\newcommand{\cmpt}{\mathrm{cmpt}}
\newcommand{\qc}{\mathrm{qc}}
\newcommand{\qs}{\mathrm{qs}}
\newcommand{\lcmpt}{\mathrm{lcmpt}}
%\newcommand{\conv}{\mathrm{conv}}
\newcommand{\red}{\mathrm{red}}
\newcommand{\fin}{\mathrm{fin}}
\newcommand{\gen}{\mathrm{gen}}
\newcommand{\petit}{\mathrm{petit}}
\newcommand{\gros}{\mathrm{gros}}
\newcommand{\loc}{\mathrm{loc}}
\newcommand{\glob}{\mathrm{glob}}
%\newcommand{\ringed}{\mathrm{ringed}}
\newcommand{\qcoh}{\mathrm{qcoh}}
\newcommand{\cl}{\mathrm{cl}}
\newcommand{\et}{\mathrm{\acute{e}t}}
\newcommand{\fet}{\mathrm{f\acute{e}t}}
\newcommand{\profet}{\mathrm{prof\acute{e}t}}
\newcommand{\proet}{\mathrm{pro\acute{e}t}}
\newcommand{\Zar}{\mathrm{Zar}}
\newcommand{\fppf}{\mathrm{fppf}}
\newcommand{\fpqc}{\mathrm{fpqc}}
\newcommand{\smooth}{\mathrm{sm}}
\newcommand{\sh}{\mathrm{sh}}
\newcommand{\op}{\mathrm{op}}
\newcommand{\open}{\mathrm{open}}
\newcommand{\closed}{\mathrm{closed}}
\newcommand{\geom}{\mathrm{geom}}
\newcommand{\alg}{\mathrm{alg}}
\newcommand{\sober}{\mathrm{sober}}
\newcommand{\dR}{\mathrm{dR}}
\newcommand{\rad}{\mathrm{rad}}
\newcommand{\discrete}{\mathrm{discrete}}
%\newcommand{\add}{\mathrm{add}}
%\newcommand{\lin}{\mathrm{lin}}
\newcommand{\Krull}{\mathrm{Krull}}
\newcommand{\qis}{\mathrm{qis}} %quasi-isomorphism
\newcommand{\ho}{\mathrm{ho}} %homotopy equivalence
\newcommand{\sep}{\mathrm{sep}}
\newcommand{\unr}{\mathrm{unr}}
\newcommand{\tame}{\mathrm{tame}}
\newcommand{\wild}{\mathrm{wild}}
\newcommand{\nil}{\mathrm{nil}}
\newcommand{\defm}{\mathrm{defm}}
\newcommand{\Art}{\mathrm{Art}}
\newcommand{\Noeth}{\mathrm{Noeth}}
\newcommand{\affd}{\mathrm{affd}}
%\newcommand{\adic}{\mathrm{adic}}
\newcommand{\pre}{\mathrm{pre}}
\newcommand{\perf}{\mathrm{perf}}
\newcommand{\perfd}{\mathrm{perfd}}
\newcommand{\rat}{\mathrm{rat}}
\newcommand{\cont}{\mathrm{cont}}
\newcommand{\dg}{\mathrm{dg}}
\newcommand{\almost}{\mathrm{a}}
%\newcommand{\stab}{\mathrm{stab}}
\newcommand{\heart}{\heartsuit}
\newcommand{\proj}{\mathrm{proj}}
\newcommand{\qproj}{\mathrm{qproj}}
\newcommand{\pd}{\mathrm{pd}}
\newcommand{\crys}{\mathrm{crys}}
\newcommand{\prisma}{\mathrm{prisma}}
\newcommand{\FF}{\mathrm{FF}}
\newcommand{\sph}{\mathrm{sph}}
\newcommand{\lax}{\mathrm{lax}}
\newcommand{\weak}{\mathrm{weak}}
\newcommand{\strict}{\mathrm{strict}}
\newcommand{\mon}{\mathrm{mon}}
\newcommand{\sym}{\mathrm{sym}}
\newcommand{\lisse}{\mathrm{lisse}}
\newcommand{\an}{\mathrm{an}}
\newcommand{\ad}{\mathrm{ad}}
\newcommand{\sch}{\mathrm{sch}}
\newcommand{\rig}{\mathrm{rig}}
\newcommand{\pol}{\mathrm{pol}}
\newcommand{\plat}{\mathrm{flat}}
\newcommand{\proper}{\mathrm{proper}}
\newcommand{\compl}{\mathrm{compl}}
\newcommand{\non}{\mathrm{non}}
\newcommand{\access}{\mathrm{access}}
\newcommand{\comp}{\mathrm{comp}}
\newcommand{\tstructure}{\mathrm{t}} %t-structures
\newcommand{\pure}{\mathrm{pure}} %pure motives
\newcommand{\mixed}{\mathrm{mixed}} %mixed motives
\newcommand{\num}{\mathrm{num}} %numerical motives
\newcommand{\ess}{\mathrm{ess}}
\newcommand{\topological}{\mathrm{top}}
\newcommand{\convex}{\mathrm{cv}}
\newcommand{\ab}{\mathrm{ab}} %abelian extensions
\newcommand{\surj}{\mathrm{surj}} %coverage on sets generated by surjections
\newcommand{\eff}{\mathrm{eff}} %effective Cartier divisors
\newcommand{\Weil}{\mathrm{Weil}} %weil divisors
\newcommand{\lex}{\mathrm{lex}}
\newcommand{\rex}{\mathrm{rex}}
\newcommand{\AR}{\mathrm{A\-R}}
\newcommand{\cons}{\mathrm{c}} %constructible sheaves
\newcommand{\tor}{\mathrm{tor}} %tor dimension
\newcommand{\semisimple}{\mathrm{ss}}

%prism custom command
\usepackage{relsize}
\usepackage[bbgreekl]{mathbbol}
\usepackage{amsfonts}
\DeclareSymbolFontAlphabet{\mathbb}{AMSb} %to ensure that the meaning of \mathbb does not change
\DeclareSymbolFontAlphabet{\mathbbl}{bbold}
\newcommand{\prism}{{\mathlarger{\mathbbl{\Delta}}}}
\renewcommand{\simpleroots}{\mathbb{I}}

\begin{document}

    \title{Isaev-Molev-Ogievetsky reflection algebras vs. $\BCDzero$ twisted Yangians}
    
    \author{Dat Minh Ha}
    \maketitle
    
    \begin{abstract}
        We compare the constructions of the reflection algebras of Isaev-Molev-Ogievetsky (IMO) with the twisted Yangian - in the sense of Guay-Regelskis - of type $\BCDzero$.
    \end{abstract}
    
    {
    \hypersetup{} 
    %\dominitoc
    \tableofcontents %sort sections alphabetically
    }

    \section{Notations}
        \subsection{Generating series}

        \subsection{Symmetric pairs of classical Lie algebras}
            All throughout:
                $$\g_N \subset \gl_N$$
            will be used to denote a finite-dimensional simple Lie algebra over $\bbC$ that is of one of the types $\sfB, \sfC, \sfD$ in the Cartan-Killing Classification; equivalently, suppose that:
                $$
                    \g_N \in
                    \begin{cases}
                        \text{$\{ \o_{2n + 1} \}_{n \geq 0}$ and $N = 2n + 1$}
                        \\
                        \text{$\{ \sp_{2n} \}_{n \geq 0}$ and $N = 2n$}
                        \\
                        \text{$\{ \o_{2n} \}_{n \geq 0}$ and $N = 2n$}
                    \end{cases}
                $$
            and respectively, we say that $\g_N$ is odd-orthogonal, symplectic, or even-orthogonal. This Lie algebra has a natural action on $U := \bbC^{\oplus N}$ (the so-called \say{vector representation}), which shall always be equipped with the standard basis:
                $$\{e_i\}_{1 \leq i \leq N}$$
            consisting of vectors with $1$ in the $i^{th}$ entry and $0$ elsewhere. By finite-dimensionality, we identify:
                $$U^{\tensor 2} \xrightarrow[]{\cong} \End(U)$$
            and then write:
                $$E_{i, j}$$
            for the image of $e_i \tensor e_j$ under the identification above; explicitly, $E_{i, j}$ is the $N \x N$ matrix with $1$ in the $(i, j)^{th}$ entry and $0$ elsewhere. Moreover, recall that $U$ comes equipped with a natural $\g_N$-invariant and non-degenerate bilinear form that we shall denote by:
                $$\eta$$
            This form is symmetric for the orthogonal types $\sfB, \sfD$ and skew-symmetric for the symplectic type $\sfC$.

        \subsection{Opposite (co)algebras}
            If $A$ is an algebra (not even necessarily associative or unital) with multiplication $\mu: A \tensor A \to A$, then its \textbf{opposite algebra} shall be denoted by $A^{\op}$. This object has the same underlying vector space, but the opposite multiplication is now given by $\mu^{\op} := \mu \circ \tau_{1, 2}$, wherein $\tau_{1, 2}: A \tensor A \xrightarrow[]{\cong} A \tensor A$ is the isomorphism given by $\tau_{1, 2}(x \tensor y) := y \tensor x$. Likewise, if $C$ is a coalgebra with comultiplication $\Delta: C \to C \tensor C$, then its \textbf{(co-)opposite} shall be denoted by $C^{\cop}$; the co-opposite multiplication is given by $\Delta^{\cop} := \tau_{1, 2} \circ \Delta$. These notions generalise in a straightforward manner to the setting of topological (co)algebras.

    \section{Recollections about (extended) Yangians}
        \subsection{Transfer matrices}
            Let us quickly recall some general features of the RTT formalism, mostly so that we can set up and justify some key notations. For a moment, let $\g$ be a general finite-dimensional simple Lie algebra.
            
            Begin with the \textbf{formal loop algebra}:
                $$\g(\!(t^{-1})\!) := \g \tensor \bbC(\!(t^{-1})\!)$$
            which shall always be equipped with the canonical $\Z$-grading given by $\deg xt^m := m$ for all $x \in \g$ and all $m \in \Z$. This Lie algebra can be endowed with the non-degenerate, invariant, and symmetric bilinear form given for all $x, y \in \g$ and $m, n \in \Z$ by:
                $$(xt^m, yt^n) := (x, y)_{\g} \delta_{m + n, -1}$$
            wherein $(-, -)_{\g}$ is a non-degenerate, invariant, and symmetric bilinear form on $\g$ (\textit{a priori}, a $\bbC^{\x}$-multiple of the Killing form). Using this bilinear form, one can then construct the following $\Z$-graded Manin triple:
                $$\left( \g(\!(t^{-1})\!), \g[t], \g[\![t^{-1}]\!] \right)$$
            This then defines a Lie cobracket $\delta: \g[t] \to \g[t_1] \tensor \g[t_2]$ given by:
                $$\delta( X ) := [\Box(X), \calr(t_1 - t_2)]$$
            for all $X \in \g[t]$ and for:
                $$\calr(u) := \frac{\calr_{\g}}{u} + O(1)$$
            (with $u$ a formal variable), which we call the Yangian \textbf{classical r-matrix}. Here, $\Box(X) := X \tensor 1 + 1 \tensor X$ is the \say{standard} classical coproduct, and $\calr_{\g} \in \g \tensor \g$ is the Casimir tensor\footnote{To be more precise, we have $\calr_{\g} \in \Sym^2(\g)^{\g}$. By viewing $(-, -)_{\g}$ as an element of $\Hom( \Sym^2(\g)^{\g}, \bbC )$, we can then regard $\calr_{\g}$ as the pre-image of $(-, -)_{\g}$ under the canonical map $\g \tensor \g \to \Hom(\g \tensor \g, \bbC)$. Since $(-, -)_{\g}$ is non-degenerate, symmetric, and invariant, the pre-image is unique and lies in the subspace $\Sym^2(\g)^{\g} \subset \g \tensor \g$.}.
            
            \textit{A priori}, the r-matrix $\calr$ is a solution to the \textbf{classical Yang-Baxter equation (CYBE)}:
                \begin{equation} \label{equation: CYBE}
                    [\calr_{1, 2}(u), \calr_{1, 3}(u + v)] + [\calr_{1, 2}(u), \calr_{2, 3}(v)] + [\calr_{1, 3}(u + v), \calr_{2, 3}(v)] = 0
                \end{equation}
            Note also that:
                \begin{equation} \label{equation: classical_unitarity}
                    \calr(u) = -\calr(-u)
                \end{equation}
            and because of this, we characterise the Yangian classical r-matrix as being \textbf{unitary}.
            
            A general result of Etingof-Kazhdan tells us that the graded Lie bialgebra $(\g[t], \delta)$ admits a \say{graded quantisation}:
                $$\calY_{\hbar}(\g)$$
            of the aforementioned Lie bialgebra structure on $\g[t]$; in particular, this means that there is isomorphism of $\Z$-graded \textit{Hopf algebras}:
                $$\calU(\g[t]) \cong \calY_{\hbar}(\g)/\hbar$$
            with the comultiplication on $\calU(\g[t])$ being given by $\Box(X)$ for all $X \in \g[t]$. This graded quantisation is we call the \textbf{formal Yangian}, and by specialising $\hbar$ to any $\hbar_0 \in \bbC \setminus \{0\}$, one obtains the \textbf{Yangian}:
                $$\calY(\g)$$
            It turns out that, up to isomorphisms, the Yangian does not depend on the choice of $\hbar_0$ (hence why we do not specify this choice in the notation), and this is because the Yangian $\calY(\g)$ carries a natural $\Z$-filtration such that $\calY_{\hbar}(\g) \cong \bigoplus_{n \geq 0} \calY(\g) \hbar^n$ and $\gr \calY(\g) \cong \calU(\g[t]) \cong \calY_{\hbar}(\g)/\hbar$ (cf. \cite{drinfeld_original_yangian_paper}); as such, we refer to the Yangian as a \say{filtered quantisation} of $\calU(\g[t])$.

            Let:
                $$\calY_{\hbar}(\g)'$$
            be the $\bbC[\![\hbar]\!]$-submodule of $\Hom_{\bbC[\![\hbar]\!]}( \calY_{\hbar}(\g), \bbC[\![\hbar]\!] )$ spanned by so-called \say{tempered} $\bbC[\![\hbar]\!]$-linear functionals, which are elements $f := \sum_{m \geq 0} f_m$ such that $\lim_{m \to +\infty} \deg f_m = +\infty$.
                
            Such an element satisfies the \textbf{quantum Yang-Baxter equation (QYBE)}:
                \begin{equation} \label{equation: QYBE}
                    \calR_{1, 2}(u) \calR_{1, 3}(u + v) \calR_{2, 3}(v) = \calR_{2, 3}(v) \calR_{1, 3}(u + v) \calR_{1, 2}(u)
                \end{equation}
            wherein the subscript pairs indicate the pair of tensor copies that $\calR(u)$ is acting on.

            Finally, by letting 
                \begin{equation} \label{equation: RTT_relation}
                    \calR(u - v) T_1(u) T_2(v) = T_2(v) T_1(u) \calR(u - v)
                \end{equation}
            obtaining the so-called \textbf{transfer matrix}.
                $$T(u) \in \Mat_N( \calY(\g) )[\![u^{-1}]\!]$$

        \subsection{(Extended) untwisted Yangians}
            Following \cite[Definition 2.1]{guay_regelskis_twisted_yangians_for_symmetric_pairs_of_types_BCD}, we work with the following definition of extended untwisted Yangians associated to a rational solution $\calR(u)$ to the QYBE on $U^{\tensor 3}$.
            \begin{definition}[Extended untwisted Yangians] \label{def: extended_untwisted_yangians}
                The \textbf{extended untwisted Yangian} associated to the classical Lie algebra $\g_N$ and a rational solution $\calR(u)$ to the QYBE \eqref{equation: QYBE} is the associative algebra:
                    $$\calX(\g_N, \calR(u))$$
                generated by the coefficients of the matrix entries of the elements:
                    $$T(u) \in \Mat_N( \calX(\g_N)[\![u^{-1}]\!] )$$
                subjected to the RTT relation \eqref{equation: RTT_relation}.
            \end{definition}

            We will be fixing once and for all a solution:
                $$\calR(u) = 1 + \frac{P}{u} + \frac{Q}{u - \kappa}$$
            for $\kappa := \sgn(\g_N) 1 + \frac{N}{2}$, with:
                \begin{equation} \label{equation: BCD_signature}
                    \sgn(\g_N) =
                    \begin{cases}
                        \text{$-1$ if $\g_N$ is of either type $\sfB$ or $\sfD$}
                        \\
                        \text{$1$ if $\g_N$ is of type $\sfC$}
                    \end{cases}
                \end{equation}
            As such, we can abbreviate:
                $$\calX(\g_N) := \calX(\g_N, \calR)$$

            \begin{lemma}[Automorphisms of extended untwisted Yangians] \label{lemma: automorphisms_of_extended_untwisted_yangians}
                \begin{enumerate}
                    \item Let $f(u) \in \bbC[\![u^{-1}]\!]^{\x}$ be an invertible formal power series in $u^{-1}$; recall that, because $\bbC[\![u^{-1}]\!]$ is a local commutative ring with (unique) maximal ideal $u^{-1}\bbC[\![u^{-1}]\!]$, $f(u)$ must be of the form $1 + \sum_{r \geq 0} f^{(r)} u^{-r - 1}$. Then, the map:
                        $$\mu_f: \calX(\g_N)[\![u^{-1}]\!] \to \Mat_N(\calX(\g_N)[\![u^{-1}]\!])$$
                    given by:
                        $$\mu_f( T(u) ) := f(u) T(u)$$
                    defines an algebra automorphism of $\calX(\g_N)$.
                    \item Let $a \in \bbC$. Then, the map:
                        $$\tau_a: \Mat_N(\calX(\g_N)[\![u^{-1}]\!]) \to \Mat_N(\calX(\g_N)[\![u^{-1}]\!])$$
                    given by:
                        $$\tau_a( T(u) ) := T(u - a)$$
                    defines an algebra automorphism of $\calX(\g_N)$.
                \end{enumerate}
            \end{lemma}
                \begin{proof}
                    See \cite[Section 2]{guay_regelskis_twisted_yangians_for_symmetric_pairs_of_types_BCD}.
                \end{proof}

            \begin{lemma}[Quantum contractions] \label{lemma: quantum_contractions}
                The transfer matrix:
                    $$T(u) \in \Mat_N( \calX(\g_N) )[\![u^{-1}]\!]$$
                satisfies the \textbf{quantum contraction} formula:
                    \begin{equation} \label{equation: quantum_contraction}
                        T(u + \kappa)^t T(u) = T(u) T(u + \kappa)^t = z(u)
                    \end{equation}
                wherein $z(u) \in \calX(\g_N)[\![u^{-1}]\!]^{\x}$ is some invertible scalar series.
            \end{lemma}
                \begin{proof}
                    See \cite[Section 2]{guay_regelskis_twisted_yangians_for_symmetric_pairs_of_types_BCD}, Equation 2.11 in particular.
                \end{proof}
                
            This leads us to the following definition.
            \begin{definition}[Untwisted Yangians] \label{def: untwisted_yangians}
                The \textbf{untwisted Yangian} is given by:
                    $$\calY(\g_N) := \calX(\g_N)/\<z(u) - 1\>$$
            \end{definition}
            \begin{remark}[Unitarity]
                Through equation \eqref{equation: quantum_contraction}, we see that $\calY(\g_N)$ is equivalently the quotient of $\calX(\g_N)$ by the two-sided ideal generated by the relation $T(u + \kappa)^t T(u) - 1$ (or equivalently $T(u) T(u + \kappa)^t - 1$). Often, we will refer to these relations collectively as the \textbf{unitarity relation} or the \textbf{unitary condition}.
            \end{remark}
            \begin{lemma}[Centres of extended untwisted Yangians] \label{lemma: centres_of_extended_untwisted_yangians}
                \begin{enumerate}
                    \item The coefficients of the quantum contraction $z(u)$ from equation \eqref{equation: quantum_contraction} generated the centre $\calZ(\g_N) := \rmZ( \calX(\g_N) )$.
                    \item Moreover, the coefficients of $z(u)$ are algebraically independent from one another, and thus:
                        $$\calX(\g_N) \cong \calY(\g_N) \tensor \calZ(\g_N)$$
                \end{enumerate}
            \end{lemma}
                \begin{proof}
                    \begin{enumerate}
                        \item 
                        \item 
                    \end{enumerate}
                \end{proof}

    \section{The two algebras}
        Fix a symmetric pair $(\g_N, \vartheta)$ of type $\sfB, \sfC, \sfD$.
        
        \subsection{Boundary transfer matrices}
            \begin{definition}[Rational K-matrices] \label{def: boundary_quantum_yang_baxter_equations}
                Given a rational solution $\calR(u)$ to the QYBE, the resulting boundary quantum Yang-Baxter equation (bQYBE) with (meromorphic) solution $\calK(u)$ reads:
                    \begin{equation} \label{equation: boundary_quantum_yang_baxter_equations}
                        \calR_{1, 2}(u - v) \calK_1(u) \calR_{1, 2}(u + v) \calK_2(v) = \calK_2(v) \calR_{1, 2}(u + v) \calK_1(u) \calR_{1, 2}(u - v) 
                    \end{equation}
            \end{definition}
        
            \begin{definition}[Boundary transfer matrices] \label{def: boundary_transfer_matrices}
                
            \end{definition}
            \begin{lemma}[Reflection equations] \label{lemma: reflection_equations}
                The boundary transfer matrix $S(u)$ as in definition \ref{def: boundary_transfer_matrices} satisfy the following so-called \textbf{reflection equation}\footnote{Also called the quaternary relation, especially in the context of twisted Yangians. See e.g. \cite[Proposition 2.2.1]{molev_yangians_and_classical_lie_algebras}.}:
                    \begin{equation} \label{equation: reflection_equations}
                        \calR_{1, 2}(u - v) S_1(u) \calR_{1, 2}(u + v) S_2(v) = S_2(v) \calR_{1, 2}(u + v) S_1(u) \calR_{1, 2}(u - v) 
                    \end{equation}
            \end{lemma}
                \begin{proof}
                    
                \end{proof}

        \subsection{(Extended) twisted Yangians of type \texorpdfstring{$\sfB, \sfC, \sfD$}{}}
            \begin{definition}[(Extended) twisted Yangians associated to symmetric pairs] \label{def: (extended)_twisted_yangians}
                
            \end{definition}

            \begin{lemma}[Twisted quantum contractions] \label{lemma: twisted_quantum_contractions}
                The boundary transfer matrix:
                    $$S(u) \in \Mat_N( \calX^{\tw}(\g_N, \vartheta) )[\![u^{-1}]\!]$$
                satisfies the following so-called \textbf{twisted quantum contraction} formula:
                    \begin{equation} \label{equation: twisted_quantum_contraction}
                        S(u) S(-u) = z^{\tw}(u)
                    \end{equation}
                for some invertible \textit{even} scalar series $z^{\tw}(u) \in \calX^{\tw}(\g_N, \vartheta)$.
            \end{lemma}
                \begin{proof}
                    
                \end{proof}
            \begin{lemma}[Centres of extended twisted Yangians] \label{lemma: centres_of_extended_twisted_yangians}
                \begin{enumerate}
                    \item The coefficients of the twisted quantum contraction $z^{\tw}(u)$ from equation \eqref{equation: twisted_quantum_contraction} generated the centre $\calZ^{\tw}(\g_N, \vartheta) := \rmZ( \calX^{\tw}(\g_N, \vartheta) )$.
                    \item Moreover, the even coefficients of $z^{\tw}(u)$ are algebraically independent from one another, and thus:
                        $$\calX^{\tw}(\g_N, \vartheta) \cong \calY^{\tw}(\g_N, \vartheta) \tensor \calZ^{\tw}(\g_N, \vartheta)$$
                \end{enumerate}
            \end{lemma}
                \begin{proof}
                    \begin{enumerate}
                        \item See \cite[Corollary 3.5]{guay_regelskis_twisted_yangians_for_symmetric_pairs_of_types_BCD}.
                        \item See \cite[Corollary 3.6]{guay_regelskis_twisted_yangians_for_symmetric_pairs_of_types_BCD}.
                    \end{enumerate}
                \end{proof}
    
        \subsection{Reflection algebras of type \texorpdfstring{$\sfB, \sfC, \sfD$}{}}
            The following notions are due to Guay-Regelskis (see \cite{guay_regelskis_twisted_yangians_for_symmetric_pairs_of_types_BCD}).
            \begin{definition}[(Extended) reflection algebras] \label{def: (extended)_reflection_algebras} 
                \begin{enumerate}
                    \item 
                    \item 
                    \item 
                \end{enumerate}
            \end{definition}
            \begin{convention}[IMO reflection algebras] \label{conv: IMO_reflection_algebras}
                In \cite{isaev_molev_ogievetsky_fusion_for_brauer_algebras_2}, a notion of reflection algebra was also given. In the present document, it shall be denoted by:
                    $$\UXB(\g_N, \id)$$
                as it is obtained as the associative algebra generated by the coefficients of the matrix entries of $S(u)$, subjected to the $\BCDzero$ reflection equation and the unitary condition, but \textit{without} the symmetry relation (cf. \cite[Definition 3.1]{isaev_molev_ogievetsky_fusion_for_brauer_algebras_2}). Also, as proposition \ref{prop: mapping_IMO_reflection_algebras_to_extended_yangians} will point out, its generating matrix - which shall also be denoted by $B(u)$ - admits a somewhat different image when mapped to the extended Yangian $\calX(\g_N)$ from that of the Guay-Regelskis $\BCDzero$ unitary reflection algebra. So that we do not confuse it with the various kinds of reflection algebras (of type $\BCDzero$) in the sense of Guay-Regelskis that were mentioned above (particularly, with the unitary reflection algebra $\UB(\g_N, \id)$), we shall refer to $\UXB(\g_N, \id)$ as the \textbf{Isaev-Molev-Ogievetsky (IMO) reflection algebra}.
            \end{convention}
            \begin{remark}
                The IMO reflection algebra is not quite the same as the reflection algebra of Molev-Ragoucy from \cite{molev_ragoucy_representations_of_reflection_algebras}, which should correspond to the twisted Yangian of type $\sfA \romanthree$.
            \end{remark}
            From now on, let $\g_N$ be of one of the types $\sfB, \sfC, \sfD$ in the Cartan-Killing Classification, and let $\vartheta := \id$.

            \begin{proposition}[A homomorphism $\UXB(\g_N, \id) \to \calX(\g_N)$] \label{prop: mapping_IMO_reflection_algebras_to_extended_yangians}
                There is an algebra homomorphism:
                    \begin{equation} \label{equation: mapping_IMO_reflection_algebras_to_extended_yangians}
                        \Phi: \UXB(\g_N, \id)[\![u^{-1}]\!] \to \calX(\g_N)[\![u^{-1}]\!]
                    \end{equation}
                determined by:
                    $$\Phi(B(u)) := T\left(u - \frac12 \kappa\right) T\left(-u + \frac12 \kappa\right)^{-1}$$
                thus defining an algebra homomorphism $\UXB(\g_N, \id) \to \calX(\g_N)$.
            \end{proposition}
                \begin{proof}
                    See \cite[Proposition 3.2]{isaev_molev_ogievetsky_fusion_for_brauer_algebras_2}
                \end{proof}
            \begin{remark}
                Later on, we will see that the homomorphism \eqref{equation: mapping_IMO_reflection_algebras_to_extended_yangians} is neither injective nor surjective in general. See theorem \ref{theorem: IMO_reflection_algebras_vs_BCD0_twisted_yangians} and corollary \ref{coro: IMO_reflection_algebras_vs_BCD0_twisted_yangians} for more details.
            \end{remark}

        \subsection{IMO reflection algebras vs. twisted Yangians of type \texorpdfstring{$\BCDzero$}{}}
            In this subsection, we work only with the $\BCDzero$ symmetric pair:
                $$(\g_N, \vartheta) = (\g_N, \id)$$

            The central question that we would like to spend the subsection to answer is as follows.
            \begin{question}
                Does the algebra homomorphism \eqref{equation: mapping_IMO_reflection_algebras_to_extended_yangians} factor in the following manner:
                    \begin{equation} \label{diagram: mapping_IMO_reflection_algebras_to_extended_twisted_yangians}
                        \begin{tikzcd}
                            {\UXB(\g_N, \id)[\![u^{-1}]\!]} & {\calX^{\tw}(\g_N, \id)[\![u^{-1}]\!]} \\
                            & {\calX(\g_N)[\![u^{-1}]\!]}
                            \arrow["{\exists ? \Phi^{\tw}}", dashed, from=1-1, to=1-2]
                            \arrow["\Phi"', from=1-1, to=2-2]
                            \arrow[hook, from=1-2, to=2-2]
                        \end{tikzcd}
                    \end{equation}
                wherein the unlabelled arrow is the canonical algebra embedding; in other words, do matrix entries of:
                    $$\Phi(B(u)) := T\left(u - \frac12 \kappa\right) T\left(-u + \frac12 \kappa\right)^{-1}$$
                actually lie inside the subalgebra $\calX^{\tw}(\g_N, \id)[\![u^{-1}]\!] \subset \calX(\g_N)[\![u^{-1}]\!]$ ?
            \end{question}
        
            Homomorphisms map generators to generators, so let us first of all consider if and how the expression for $\Phi( B(u) )$ as above differs from that for the transfer matrix $S(u)$, whose entries generate $\calX^{\tw}(\g_N, \id)$. To this end, recall from definition \ref{def: (extended)_twisted_yangians} that:
                $$S(u) = T\left(u - \frac12 \kappa\right) T\left(-u + \frac12 \kappa\right)^t$$
            Indeed, this is different from the expression for $\Phi(B(u))$ from proposition \ref{prop: mapping_IMO_reflection_algebras_to_extended_yangians}, so one way for an algebra homomorphism $\Phi^{\tw}$ as in diagram \eqref{diagram: mapping_IMO_reflection_algebras_to_extended_twisted_yangians} to exist is for there to be an algebra automorphism $\beta^{\tw} \in \Aut_{\Assoc\Alg}( \calX^{\tw}(\g_N, \id)[\![u^{-1}]\!] )$ such that the equation $(\beta^{\tw} \circ \Phi)( B(u) ) = S(u)$ holds in $\Mat_N(\calX^{\tw}(\g_N, \id)[\![u^{-1}]\!])$. For this to hold, though, one must already have that $\im \Phi \subset \calX^{\tw}( \g_N, \id )[\![u^{-1}]\!]$, which is the very thing we are trying to verify, so instead, let us instead search for an algebra automorphism:
                $$\beta \in \Aut_{\Assoc\Alg}( \calX(\g_N)[\![u^{-1}]\!] )$$
            such that:
                \begin{equation} \label{equation: B_matrix_S_matrix_compatibility}
                    (\beta \circ \Phi)( B(u) ) = S(u)
                \end{equation}
            and then afterwards, we would take $\beta^{\tw} = \beta|_{ \calX^{\tw}( \g_N, \id )[\![u^{-1}]\!] }$ to get an automorphism of the subalgebra $\calX^{\tw}(\g_N, \id)[\![u^{-1}]\!] \subset \calX(\g_N)[\![u^{-1}]\!]$. The original question can now be rephrased in the following manner.
            \begin{question}
                Does there exist an algebra automorphism $\beta \in \Aut_{\Assoc\Alg}( \calX(\g_N)[\![u^{-1}]\!] )$ fitting into the following commutative diagram ?
                    \begin{equation} \label{diagram: mapping_IMO_reflection_algebras_to_extended_twisted_yangians_via_untwisted_automorphisms}
                        \begin{tikzcd}
                    	{\UXB(\g_N, \id)[\![u^{-1}]\!]} & {\calX^{\tw}(\g_N, \id)[\![u^{-1}]\!]} \\
                    	{\calX(\g_N)[\![u^{-1}]\!]} & {\calX(\g_N)[\![u^{-1}]\!]}
                    	\arrow["{\exists ? \Phi^{\tw}}", dashed, from=1-1, to=1-2]
                    	\arrow["\Phi"', from=1-1, to=2-1]
                    	\arrow[hook, from=1-2, to=2-2]
                    	\arrow["\beta", from=2-1, to=2-2]
                        \end{tikzcd}
                    \end{equation}
            \end{question}
            
            A natural case to consider is when $\beta$ is the inner automorphism defined on the transfer matrix $T(u) \in \Mat_N( \calX(\g_N)[\![u^{-1}]\!] )$ by:
                $$\beta( T(u) ) := b(u) \cdot T(u)$$
            for some invertible central scalar series:
                $$b(u) \in \calZ(\g_N)[\![u^{-1}]\!]^{\x}$$
            In this case, we have:
                $$
                    \begin{aligned}
                        (\beta \circ \Phi)( B(u) ) & = \beta\left( T\left(u - \frac12 \kappa\right) T\left(-u + \frac12 \kappa\right)^{-1} \right)
                        \\
                        & = \beta\left( T\left(u - \frac12 \kappa\right) \right) \beta\left( T\left(-u + \frac12 \kappa\right) \right)^{-1}
                        \\
                        & = b\left(u - \frac12 \kappa\right) T\left(u - \frac12 \kappa\right) \cdot T\left(-u + \frac12 \kappa\right)^{-1} b\left(-u + \frac12 \kappa\right)^{-1}
                        \\
                        & = b\left(u - \frac12 \kappa\right) b\left(-u + \frac12 \kappa\right)^{-1} \cdot T\left(u - \frac12 \kappa\right) T\left(-u + \frac12 \kappa\right)^{-1}
                        \\
                        & = b\left(u - \frac12 \kappa\right) b\left(-u + \frac12 \kappa\right)^{-1} \cdot \Phi( B(u) )
                    \end{aligned}
                $$
            For brevity, let:
                $$b^{\tw}(u) := b\left(u - \frac12 \kappa\right) b\left(-u + \frac12 \kappa\right)^{-1}$$
            and then equation \eqref{equation: B_matrix_S_matrix_compatibility} thus reads:
                $$S(u) = (\beta \circ \Phi)( B(u) ) = b^{\tw}(u) \cdot \Phi( B(u) )$$
            We must now check if this is compatible with the unitary condition:
                $$B(u) B(-u) = 1$$
            coming from the definition of $\UXB(\g_N, \id)$, as well as the twisted quantum contraction formula (see lemma \ref{lemma: twisted_quantum_contractions}):
                $$S(u) S(-u) = z^{\tw}(u)$$
            To this end, consider the following:
                $$
                    \begin{aligned}
                        z^{\tw}(u) & = S(u) S(-u)
                        \\
                        & = \left( b^{\tw}(u) \cdot \Phi( B(u) ) \right) \cdot \left( b^{\tw}(-u) \cdot \Phi( B(-u) ) \right)
                        \\
                        & = b^{\tw}(u) b^{\tw}(-u) \cdot \Phi( B(u) B(-u) )
                        \\
                        & = b^{\tw}(u) b^{\tw}(-u) \cdot \Phi(1)
                        \\
                        & = b^{\tw}(u) b^{\tw}(-u)
                    \end{aligned}
                $$
            This is to hold if the homomorphism $\Phi^{\tw}$ is to exist, and thus we are led to the following question.
            \begin{question} \label{question: square_roots_of_twsited_quantum_contractions}
                Is it possible to factorise:
                    \begin{equation} \label{equation: square_roots_of_twisted_quantum_contractions}
                        z^{\tw}(u) = b^{\tw}(u) b^{\tw}(-u)
                    \end{equation}
                for some $b^{\tw}(u) \in \calZ(\g_N)[\![u^{-1}]\!]^{\x}$, or better yet, for some $b^{\tw}(u) \in \calZ^{\tw}(\g_N, \id)[\![u^{-1}]\!]^{\x}$ ?
            \end{question}
            \begin{remark}
                This is consistent with the fact that $z^{\tw}(u)$ is an even formal power series \textit{a priori}.
            \end{remark}

            The following was noted already at the end of \cite[Section 2]{guay_regelskis_twisted_yangians_for_symmetric_pairs_of_types_BCD}, but no proof was provided, so we supply one here.
            \begin{lemma} \label{lemma: quantum_pfaffians}
                There exists an element:
                    $$y(u) \in \calZ(\g_N)[\![u^{-1}]\!]$$
                so that the quantum contraction $z(u) \in \calZ(\g_N)[\![u^{-1}]\!]^{\x}$ as in lemma \ref{lemma: quantum_contractions} factorises in the following manner:
                    \begin{equation} \label{equation: quantum_pfaffians}
                        z(u) = y(u) y(u + \kappa)
                    \end{equation}
            \end{lemma}
                \begin{proof}
                    First of all, since $z(u) \in \calX(\g_N)[\![u^{-1}]\!]^{\x}$ and since $\calX(\g_N)[\![u^{-1}]\!]^{\x}$ is multiplicatively closed, we can take $y(u) \in \calX(\g_N)[\![u^{-1}]\!]^{\x}$ too. Such a series is of the form:
                        $$y(u) = 1 + \sum_{r \geq 0} y^{(r)} u^{-r - 1}$$
                    \textit{a priori}, and thus:
                        $$z(u) = y(u) y(u + \kappa) = \left( 1 + \sum_{m \geq 0} y^{(m)} u^{-m - 1} \right) \left( 1 + \sum_{m \geq 0} y^{(m)} (u + \kappa)^{-m - 1} \right)$$
                    Now, we note that:
                        $$\frac{1}{u + \kappa} = \frac{u^{-1}}{1 + \kappa u^{-1}} = u^{-1} \sum_{n \geq 0} (-\kappa u^{-1})^n = \sum_{n \geq 0} (-\kappa)^n u^{-n - 1}$$
                    This then allows us to write:
                        $$
                            \begin{aligned}
                                z(u) & = \left( 1 + \sum_{m \geq 0} y^{(m)} u^{-m - 1} \right) \left( 1 + \sum_{m \geq 0} y^{(m)} (u + \kappa)^{-m - 1} \right)
                                \\
                                & = \left( 1 + \sum_{m \geq 0} y^{(m)} u^{-m - 1} \right) \left( 1 + \sum_{m \geq 0} y^{(m)} \left( \sum_{n \geq 0} (-\kappa)^n u^{-n - 1} \right)^{-m - 1} \right)
                                \\
                                & =
                                \begin{aligned}
                                    & 1 + \sum_{m \geq 0} y^{(m)} u^{-m - 1} + \sum_{m \geq 0} y^{(m)} \left( \sum_{n \geq 0} (-\kappa)^n u^{-n - 1} \right)^{-m - 1}
                                    \\
                                    & \quad + \left( \sum_{l \geq 0} y^{(l)} u^{-l - 1} \right)\left( \sum_{m \geq 0} y^{(m)} \left( \sum_{n \geq 0} (-\kappa)^n u^{-n - 1} \right)^{-m - 1} \right)
                                \end{aligned}
                                \\
                                & =
                                \begin{aligned}
                                    & (...)
                                    \\
                                    & \quad + \sum_{l \geq 0} \sum_{m \geq 0} y^{(l)} u^{-l - 1} \cdot y^{(m)} \left( \sum_{n \geq 0} (-\kappa)^n u^{-n - 1} \right)^{-m - 1}
                                \end{aligned}
                                \\
                                & =
                                \begin{aligned}
                                    & (...)
                                    \\
                                    & \quad + \sum_{l \geq 0} \sum_{m \geq 0} y^{(l)} y^{(m)} \cdot u^{-l - 1} \cdot \left( \frac{1}{u + \kappa} \right)^{-m - 1}
                                \end{aligned}
                                \\
                                & =
                                \begin{aligned}
                                    & (...)
                                    \\
                                    & \quad + \sum_{l \geq 0} \sum_{m \geq 0} y^{(l)} y^{(m)} \cdot u^{-l - 1} \cdot (u + \kappa)^{m + 1}
                                \end{aligned}
                                \\
                                & = 
                                \begin{aligned}
                                    & (...)
                                    \\
                                    & \quad + \sum_{l \geq 0} \sum_{m \geq 0} y^{(l)} y^{(m)} \cdot u^{-l - 1} \cdot \sum_{k = 0}^{m + 1} \binom{m + 1}{k} \kappa^k u^{m + 1 - k}
                                \end{aligned}
                                \\
                                & =
                                \begin{aligned}
                                    & (...)
                                    \\
                                    & \quad + \sum_{l \geq 0} \sum_{m \geq 0} y^{(l)} y^{(m)} \cdot \sum_{k = 0}^{m + 1} \binom{m + 1}{k} \kappa^k u^{m - l - k}
                                \end{aligned}
                            \end{aligned}
                        $$
                    Since $z(u) \in \calZ(\g_N)[\![u^{-1}]\!]^{\x}$, it can be written as:
                        $$z(u) = 1 + \sum_{m \geq 0} z^{(m)} u^{-m - 1}$$
                    with $z^{(m)} \in \calZ(\g_N)$, and since these coefficients are algebraically independent from one another and generate the centre $\calZ(\g_N)$. Through homogeneity, we then infer that:
                        $$z^{(0)} = 2y^{(0)}$$
                        $$\vdots$$
                        $$
                            z^{(m)} =
                            \begin{aligned}
                                & (1 + (-\kappa)^m) y^{(m)}
                                \\
                                & \quad + \left( \sum_{l \geq 0} \sum_{m \geq 0} y^{(l)} y^{(m)} \cdot \sum_{k = 0}^{m + 1} \binom{m + 1}{k} \kappa^k u^{m - l - k} \right)^{(m)} 
                            \end{aligned}
                        $$
                    wherein $(-)^{(m)}$ means the degree $-m - 1$ coefficient. This determines $y(u)$, so we have existence.

                    We also know from lemma \ref{lemma: centres_of_extended_untwisted_yangians} that the coefficients $z^{(m)} \in \calZ(\g_N)$, so by induction, one can show that $y^{(m)} \in \calZ(\g_N)$ necessarily as well, by arguing in the following manner. To this end, recall that $\frac{1}{u + \kappa} = \sum_{n \geq 0} (-\kappa)^n u^{-n - 1}$, which then gives us:
                        $$y(u + \kappa) = 1 + \sum_{m \geq 0} y^{(m)} \left( \sum_{n \geq 0} (-\kappa)^n u^{-n - 1} \right)^{-m - 1}$$
                    With this in mind, along with the fact that $z(u) \in \calZ(\g_N)[\![u^{-1}]\!]$, let us then consider the following for all $X(u) \in \calX(\g_N)[\![u^{-1}]\!]$:
                        $$
                            \begin{aligned}
                                0 & = [z(u), X(u)]
                                \\
                                & = [y(u) y(u + \kappa), X(u)]
                                \\
                                & =
                                \begin{aligned}
                                    & y(u) \cdot [y(u + \kappa), X(u)]
                                    \\
                                    & \quad + [y(u), X(u)] \cdot y(u + \kappa)
                                \end{aligned}
                                \\
                                & =
                                \begin{aligned}
                                    & \left( 1 + \sum_{l \geq 0} y^{(l)} u^{-l - 1} \right) \cdot \left[ 1 + \sum_{m \geq 0} y^{(m)} \left( \sum_{n \geq 0} (-\kappa)^n u^{-n - 1} \right)^{-m - 1}, X(u) \right]
                                    \\
                                    & \quad + \left[ 1 + \sum_{l \geq 0} y^{(l)} u^{-l - 1}, X(u) \right] \cdot \left( 1 + \sum_{m \geq 0} y^{(m)} \left( \sum_{n \geq 0} (-\kappa)^n u^{-n - 1} \right)^{-m - 1} \right) 
                                \end{aligned}
                                \\
                                & =
                                \begin{aligned}
                                    & \left( 1 + \sum_{l \geq 0} y^{(l)} u^{-l - 1} \right) \cdot \sum_{m \geq 0} \left[ y^{(m)} \left( \sum_{n \geq 0} (-\kappa)^n u^{-n - 1} \right)^{-m - 1}, X(u) \right]
                                    \\
                                    & \quad + \sum_{l \geq 0} \left[ y^{(l)} u^{-l - 1}, X(u) \right] \cdot \left( 1 + \sum_{m \geq 0} y^{(m)} \left( \sum_{n \geq 0} (-\kappa)^n u^{-n - 1} \right)^{-m - 1} \right) 
                                \end{aligned}
                            \end{aligned}
                        $$
                    From this, we see that we must have:
                        $$[y^{(m)}, X(u)] = 0$$
                    for all $m \geq 0$ and all $X(u) \in \calX(\g_N)[\![u^{-1}]\!]$, thus proving that:
                        $$y(u) \in \calZ(\g_N)[\![u^{-1}]\!]$$
                \end{proof}
            Once again, the following was already noted in \cite{guay_regelskis_twisted_yangians_for_symmetric_pairs_of_types_BCD}, particular at the beginning of Theorem 3.1, wherein the authors used the notations $w(u) = z^{\tw}(u)$ and $q(u) = y^{\tw}(u)$.
            \begin{lemma} \label{lemma: twisted_quantum_pfaffians}
                The twisted quantum contraction $z^{\tw}(u) \in \calZ^{\tw}(\g_N, \id)[\![u^{-1}]\!]^{\x}$ from lemma \ref{lemma: twisted_quantum_contractions} factorises in the following manner:
                    \begin{equation} \label{equation: twisted_quantum_pfaffians}
                        z^{\tw}(u) = y^{\tw}(u) y^{\tw}(u + \kappa)
                    \end{equation}
                wherein:
                    $$y^{\tw}(u) := y\left( u - \frac12 \kappa \right) y\left( -u + \frac12 \kappa \right)$$
                and $y(u) \in \calZ(\g_N)[\![u^{-1}]\!]^{\x}$ is as in lemma \ref{lemma: quantum_pfaffians}. Moreover, we have:
                    $$y^{\tw}(u) \in \calZ^{\tw}(\g_N, \id)[\![u^{-1}]\!]$$
            \end{lemma}
                \begin{proof}
                    From lemma \ref{lemma: twisted_quantum_contractions}, we know that $z^{\tw}(u) = z\left( u - \frac12 \kappa \right) z\left( -u + \frac12 \kappa \right)$. By combining this with equation \eqref{equation: quantum_pfaffians}, we then get:
                        $$
                            \begin{aligned}
                                z^{\tw}(u) & = z\left( u - \frac12 \kappa \right) z\left( -u + \frac12 \kappa \right)
                                \\
                                & = y\left( u - \frac12 \kappa \right) y\left( u + \frac12 \kappa \right) \cdot y\left( -u + \frac12 \kappa \right) y\left( -u + \frac32 \kappa \right)
                                \\
                                & = y\left( u - \frac12 \kappa \right) y\left( -u + \frac12 \kappa \right) \cdot y\left( u + \frac12 \kappa \right) y\left( -u + \frac32 \kappa \right)
                            \end{aligned}
                        $$
                    Setting $y^{\tw}(u) := y\left( u - \frac12 \kappa \right) y\left( -u + \frac12 \kappa \right)$ then yields us equation \eqref{equation: twisted_quantum_pfaffians}.

                    Then, by arguing as in the proof of lemma \ref{lemma: quantum_pfaffians}, one sees furthermore that $y^{\tw}(u) \in \calZ^{\tw}(\g_N, \id)[\![u^{-1}]\!]$.
                \end{proof}
            \begin{remark}
                We can rewrite equation \eqref{equation: twisted_quantum_pfaffians} into the following form, which has the advantage of being more symmetric and more clearly consistent with the evenness of $z^{\tw}(u)$:
                    \begin{equation} \label{equation: symmetrised_twisted_quantum_pfaffians}
                        z^{\tw}(u) = y^{\tw}(u) y^{\tw}(-u)
                    \end{equation}
            \end{remark}
            We can now answer question \ref{question: square_roots_of_twsited_quantum_contractions} in the affirmative by taking:
                $$b^{\tw}(u) := y^{\tw}(u)$$
            Let us now compute $b(u) \in \calZ(\g_N)[\![u^{-1}]\!]^{\x}$ in terms of $y(u) \in \calZ(\g_N)[\![u^{-1}]\!]^{\x}$; as these are invertible series, they can be written in the form:
                $$b(u) := 1 + \sum_{m \geq 0} b^{(m)} u^{-m - 1}$$
                $$y(u) := 1 + \sum_{m \geq 0} y^{(m)} u^{-m - 1}$$
            for some $b^{(m)}, y^{(m)} \in \calZ(\g_N)$. By convention, we have $b^{\tw}(u) := b\left( u - \frac12 \kappa \right) b\left( -u + \frac12 \kappa \right)^{-1}$ and $y^{\tw}(u) := y\left( u - \frac12 \kappa \right) y\left( -u + \frac12 \kappa \right)$, which are respectively equivalent to:
                $$b^{\tw}\left( -u - \frac12 \kappa \right) = b(-u) b(u)^{-1}$$
                $$y^{\tw}\left( -u - \frac12 \kappa \right) = y(-u) y(u)$$
            Next, let us compute $b(u)^{-1}$; to this end, set:
                $$b(u)^{-1} := 1 + \sum_{n \geq 0} c^{(n)} u^{-n - 1}$$
            for some $c^{(n)} \in \calZ(\g_N)$ (again, the $0^{th}$ degree term is $1$ as the series is invertible) and then consider the following:
                $$
                    \begin{aligned}
                        1 & = b(u) b(u)^{-1}
                        \\
                        & = \left( 1 + \sum_{m \geq 0} b^{(m)} u^{-m - 1} \right) \left( 1 + \sum_{n \geq 0} c^{(n)} u^{-n - 1} \right)
                        \\
                        & = 1 + \sum_{m \geq 0} b^{(m)} u^{-m - 1} + \sum_{n \geq 0} c^{(n)} u^{-m - 1} + \left( \sum_{m \geq 0} b^{(m)} u^{-m - 1} \right) \left( \sum_{n \geq 0} c^{(n)} u^{-n - 1} \right)
                        \\
                        & = 1 + \sum_{m \geq 0} ( b^{(m)} + c^{(m)} ) u^{-m - 1} + \sum_{m \geq 0} \sum_{n \geq 0} b^{(m)} c^{(n)} u^{-m - n - 2}
                        \\
                        & = 1 + \sum_{l \geq 0} \left( b^{(l)} + c^{(l)} + \sum_{ \substack{m, n \geq 0 \\ m + n = l} } b^{(m)} c^{(n)} \right) u^{-l - 1}
                    \end{aligned}
                $$
            By homogeneity, we then have for all $l \geq 0$ that $b^{(l)} + c^{(l)} + \sum_{ \substack{m, n \geq 0 \\ m + n = l} } b^{(m)} c^{(n)} = 0$, or equivalently, the following recursive formula for the coefficients $c^{(l)}$ of $b(u)^{-1}$:
                $$c^{(l)} = -b^{(l)} - \sum_{m = 0}^l b^{(m)} c^{(l - m)}$$
            This then allows us to compute:
                $$
                    \begin{aligned}
                        b(-u) b(u)^{-1} & = \left( 1 + \sum_{m \geq 0} b^{(m)} (-u)^{-m - 1} \right) \left( 1 + \sum_{n \geq 0} c^{(n)} u^{-n - 1} \right)
                        \\
                        & = \left( 1 + \sum_{m \geq 0} b^{(m)} (-u)^{-m - 1} \right) \left( 1 + \sum_{n \geq 0} c^{(n)} u^{-n - 1} \right)
                        \\
                        & = 1 + \sum_{m \geq 0} b^{(m)} (-u)^{-m - 1} + \sum_{n \geq 0} c^{(n)} u^{-n - 1} + \left( \sum_{m \geq 0} b^{(m)} (-u)^{-m - 1} \right) \left( \sum_{n \geq 0} c^{(n)} u^{-n - 1} \right)
                        \\
                        & = 1 + \sum_{m \geq 0} ( (-1)^{-m - 1} b^{(m)} + c^{(n)} ) u^{-m - 1} + \sum_{m \geq 0} \sum_{n \geq 0} (-1)^{-m - 1} b^{(m)} c^{(n)} u^{-m - n - 2}
                        \\
                        & = 1 + \sum_{l \geq 0} \left( (-1)^{-l - 1} b^{(l)} + c^{(l)} + \sum_{ \substack{m, n \geq 0\\m + n = l} } (-1)^{-m - 1} b^{(m)} c^{(n)} \right) u^{-l - 1}
                        \\
                        & = 1 + \sum_{l \geq 0} \left( (-1)^{-l - 1} b^{(l)} + \left( -b^{(l)} - \sum_{m = 0}^l b^{(m)} c^{(l - m)} \right) + \sum_{m = 0}^l (-1)^{-m - 1} b^{(m)} c^{(l - m)} \right) u^{-l - 1}
                        \\
                        & = 1 + \sum_{l \geq 0} \left( 2b^{(2l + 1)} - \sum_{ \substack{ 0 \leq m \leq l \\ \text{$m$ even} } } b^{(m)} c^{(l - m)} \right) u^{-l - 1}
                    \end{aligned}
                $$
            At the same time, we have:
                $$
                    \begin{aligned}
                        y(-u) y(u) & = \left( 1 + \sum_{m \geq 0} y^{(m)} (-u)^{-m - 1} \right) \left( 1 + \sum_{n \geq 0} y^{(n)} u^{-n - 1} \right)
                        \\
                        & = 1 + \sum_{m \geq 0} y^{(m)} (-u)^{-m - 1} + \sum_{n \geq 0} y^{(n)} u^{-n - 1} + \left( \sum_{m \geq 0} y^{(m)} (-u)^{-m - 1} \right) \left( \sum_{n \geq 0} y^{(n)} u^{-n - 1} \right)
                        \\
                        & = 1 + \sum_{m \geq 0} ( (-1)^{-m - 1} y^{(m)} + y^{(m)} ) u^{-m - 1} + \sum_{m \geq 0} \sum_{n \geq 0} (-1)^{-m - 1} y^{(m)} y^{(n)} u^{-m - n - 2}
                        \\
                        & = 1 + \sum_{l \geq 0} \left( (-1)^{-l - 1} y^{(l)} + y^{(l)} + \sum_{ \substack{m, n \geq 0\\m + n = l} } (-1)^{-m - 1} y^{(m)} y^{(n)} \right) u^{-l - 1}
                        \\
                        & = 1 + \sum_{l \geq 0} \left( 2y^{(2l + 1)} + \sum_{0 \leq m \leq l} (-1)^{-m - 1} y^{(m)} y^{(l - m)} \right) u^{-l - 1}
                    \end{aligned}
                $$
            As we are supposed to have:
                $$b(-u) b(u)^{-1} = y(-u) y(u)$$
            we now see - via homogeneity - that the coefficients of $b(u)$ are determined by those of $y(u)$ by:
                \begin{equation} \label{equation: b_cofficients_in_terms_of_y_coefficients}
                    2b^{(2l + 1)} - \sum_{ \substack{ 0 \leq m \leq l \\ \text{$m$ even} } } b^{(m)} c^{(l - m)} = 2y^{(2l + 1)} + \sum_{0 \leq m \leq l} (-1)^{-m - 1} y^{(m)} y^{(l - m)}
                \end{equation}
                
            \begin{proposition}[A homomorphism $\UXB(\g_N, \id) \to \calX^{\tw}(\g_N, \id)$] \label{prop: mapping_IMO_reflection_algebras_to_extended_untwisted_yangians_via_extended_twisted_yangians}
                The algebra homomorphism $\Phi$ from proposition \ref{prop: mapping_IMO_reflection_algebras_to_extended_yangians} induces an algebra homomorphism:
                    $$\Phi^{\tw}: \UXB(\g_N, \id)[\![u^{-1}]\!] \to \calX^{\tw}(\g_N, \id)[\![u^{-1}]\!]$$
                given by:
                    $$\Phi^{\tw}( B(u) ) = S(u)$$
                and fitting into the commutative diagram \eqref{diagram: mapping_IMO_reflection_algebras_to_extended_twisted_yangians_via_untwisted_automorphisms} (and thus also diagram \eqref{diagram: mapping_IMO_reflection_algebras_to_extended_twisted_yangians}).
            \end{proposition}
                \begin{proof}
                    This is a direct consequence of lemma \ref{lemma: twisted_quantum_pfaffians}. See the discussion above.
                \end{proof}

            Let us conclude the subsection by considering whether the composition:
                \begin{equation} \label{diagram: mapping_IMO_reflection_algebras_to_BCD0_twisted_yangians}
                    \begin{tikzcd}
                        {\UXB(\g_N, \id)[\![u^{-1}]\!]} & {\calX^{\tw}(\g_N, \id)[\![u^{-1}]\!]} & {\calY^{\tw}(\g_N, \id)[\![u^{-1}]\!]}
                        \arrow["{\Phi^{\tw}}", dashed, from=1-1, to=1-2]
                        \arrow[two heads, from=1-2, to=1-3]
                    \end{tikzcd}
                \end{equation}
            wherein the unlabelled arrow is the canonical quotient map, is an isomorphism of algebras. 

            Now, in the process of proving \cite[Theorem 4.1]{guay_regelskis_twisted_yangians_for_symmetric_pairs_of_types_BCD}, the authors proved Lemma 4.3, which states that the boundary transfer matrices $S(u) \in \Mat_N( \calX^{\tw}(\g_N, \id)[\![u^{-1}]\!] )$ satisfy the symmetry relation \eqref{equation: symmetry_relation}. This leads us to make the following claim.
            \begin{theorem} \label{theorem: IMO_reflection_algebras_vs_BCD0_twisted_yangians}
                The algebra homomorphism $\Phi^{\tw}$ from proposition \ref{prop: mapping_IMO_reflection_algebras_to_extended_untwisted_yangians_via_extended_twisted_yangians} fits into the following commutative diagram:
                    $$
                        \begin{tikzcd}
                    	{\UXB(\g_N, \id)[\![u^{-1}]\!]} & {\calX^{\tw}(\g_N, \id)[\![u^{-1}]\!]} \\
                    	{\UB(\g_N, \id)[\![u^{-1}]\!]} & {\calY^{\tw}(\g_N, \id)[\![u^{-1}]\!]}
                    	\arrow["{{\Phi^{\tw}}}", from=1-1, to=1-2]
                    	\arrow[two heads, from=1-1, to=2-1]
                    	\arrow[two heads, from=1-2, to=2-2]
                    	\arrow["\phi^{\tw}", from=2-1, to=2-2]
                        \end{tikzcd}
                    $$
                wherein:
                \begin{itemize}
                    \item the left vertical arrow is the canonical quotient map by the ideal generated by the symmetry relation \eqref{equation: symmetry_relation},
                    \item the right vertical arrow is the canonical quotient map by the ideal generated by the unitarity relation, and
                    \item the bottom arrow $\phi^{\tw}: \UB(\g_N, \id)[\![u^{-1}]\!] \xrightarrow[]{\cong} \calY^{\tw}(\g_N, \id)[\![u^{-1}]\!]$ is the isomorphism from \cite[Theorem 4.1]{guay_regelskis_twisted_yangians_for_symmetric_pairs_of_types_BCD} (see also \cite[Equation 4.32]{guay_regelskis_twisted_yangians_for_symmetric_pairs_of_types_BCD}), which we recall to be given by $\phi^{\tw}( B(u) ) := S(u)$. 
                \end{itemize}
            \end{theorem}
                \begin{proof}
                    This is a straightforward diagram chase.
                \end{proof}
            \begin{corollary}[IMO reflection algebras vs. $\BCDzero$ twisted Yangians] \label{coro: IMO_reflection_algebras_vs_BCD0_twisted_yangians}
                The composite homomorphism \eqref{diagram: mapping_IMO_reflection_algebras_to_BCD0_twisted_yangians} is surjective, with kernel equal to the ideal of $\UXB(\g_N, \id)$ generated by the following version of the symmetry relation \eqref{equation: symmetry_relation}:
            \end{corollary}
    
    \addcontentsline{toc}{section}{References}
    \printbibliography

\end{document}