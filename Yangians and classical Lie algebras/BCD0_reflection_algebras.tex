\documentclass[a4paper, 11pt]{article}

%\usepackage[center]{titlesec}

\usepackage{amsfonts, amssymb, amsmath, amsthm, amsxtra}

\usepackage{foekfont}

\usepackage{MnSymbol}

\usepackage{pdfrender, xcolor}
%\pdfrender{StrokeColor=black,LineWidth=.4pt,TextRenderingMode=2}

%\usepackage{minitoc}
%\setcounter{section}{-1}
%\setcounter{tocdepth}{}
%\setcounter{minitocdepth}{}
%\setcounter{secnumdepth}{}

\usepackage{graphicx}

\usepackage[english]{babel}
\usepackage[utf8]{inputenc}
%\usepackage{mathpazo}
%\usepackage{eucal}
\usepackage{eufrak}
\usepackage{bbm}
\usepackage{bm}
\usepackage{csquotes}
\usepackage[nottoc]{tocbibind}
\usepackage{appendix}
\usepackage{float}
\usepackage[T1]{fontenc}
\usepackage[
    left = \flqq{},% 
    right = \frqq{},% 
    leftsub = \flq{},% 
    rightsub = \frq{} %
]{dirtytalk}

\usepackage{imakeidx}
\makeindex

%\usepackage[dvipsnames]{xcolor}
\usepackage{hyperref}
    \hypersetup{
        colorlinks=true,
        linkcolor=teal,
        filecolor=pink,      
        urlcolor=teal,
        citecolor=magenta
    }
\usepackage{comment}

% You would set the PDF title, author, etc. with package options or
% \hypersetup.

\usepackage[backend=biber, style=alphabetic, sorting=nty]{biblatex}
    \addbibresource{bibliography.bib}
\renewbibmacro{in:}{}

\raggedbottom

\usepackage{mathrsfs}
\usepackage{mathtools} 
\mathtoolsset{showonlyrefs} 
%\usepackage{amsthm}
\renewcommand\qedsymbol{$\blacksquare$}

\usepackage{tikz-cd}
\tikzcdset{scale cd/.style={every label/.append style={scale=#1},
    cells={nodes={scale=#1}}}}
\usepackage{tikz}
\usepackage{setspace}
\usepackage[version=3]{mhchem}
\parskip=0.1in
\usepackage[margin=25mm]{geometry}

\usepackage{listings, lstautogobble}
\lstset{
	language=matlab,
	basicstyle=\scriptsize\ttfamily,
	commentstyle=\ttfamily\itshape\color{gray},
	stringstyle=\ttfamily,
	showstringspaces=false,
	breaklines=true,
	frameround=ffff,
	frame=single,
	rulecolor=\color{black},
	autogobble=true
}

\usepackage{todonotes,tocloft,xpatch,hyperref}

% This is based on classicthesis chapter definition
\let\oldsec=\section
\renewcommand*{\section}{\secdef{\Sec}{\SecS}}
\newcommand\SecS[1]{\oldsec*{#1}}%
\newcommand\Sec[2][]{\oldsec[\texorpdfstring{#1}{#1}]{#2}}%

\newcounter{istodo}[section]

% http://tex.stackexchange.com/a/61267/11984
\makeatletter
%\xapptocmd{\Sec}{\addtocontents{tdo}{\protect\todoline{\thesection}{#1}{}}}{}{}
\newcommand{\todoline}[1]{\@ifnextchar\Endoftdo{}{\@todoline{#1}}}
\newcommand{\@todoline}[3]{%
	\@ifnextchar\todoline{}
	{\contentsline{section}{\numberline{#1}#2}{#3}{}{}}%
}
\let\l@todo\l@subsection
\newcommand{\Endoftdo}{}

\AtEndDocument{\addtocontents{tdo}{\string\Endoftdo}}
\makeatother

\usepackage{lipsum}

%   Reduce the margin of the summary:
\def\changemargin#1#2{\list{}{\rightmargin#2\leftmargin#1}\item[]}
\let\endchangemargin=\endlist 

%   Generate the environment for the abstract:
%\newcommand\summaryname{Abstract}
%\newenvironment{abstract}%
    %{\small\begin{center}%
    %\bfseries{\summaryname} \end{center}}

\newtheorem{theorem}{Theorem}[section]
    \numberwithin{theorem}{subsection}
\newtheorem{proposition}{Proposition}[section]
    \numberwithin{proposition}{subsection}
\newtheorem{lemma}{Lemma}[section]
    \numberwithin{lemma}{subsection}
\newtheorem{claim}{Claim}[section]
    \numberwithin{claim}{subsection}
\newtheorem{question}{Question}[section]
    \numberwithin{question}{subsection}

\theoremstyle{definition}
    \newtheorem{definition}{Definition}[section]
        \numberwithin{definition}{subsection}

\theoremstyle{remark}
    \newtheorem{remark}{Remark}[section]
        \numberwithin{remark}{subsection}
    \newtheorem{example}{Example}[section]
        \numberwithin{example}{subsection}    
    \newtheorem{convention}{Convention}[section]
        \numberwithin{convention}{subsection}
    \newtheorem{corollary}{Corollary}[section]
        \numberwithin{corollary}{subsection}

\numberwithin{equation}{section}

\setcounter{section}{-1}

\renewcommand{\cong}{\simeq}
\newcommand{\ladjoint}{\dashv}
\newcommand{\radjoint}{\vdash}
\newcommand{\<}{\langle}
\renewcommand{\>}{\rangle}
\newcommand{\ndiv}{\hspace{-2pt}\not|\hspace{5pt}}
\newcommand{\cond}{\blacktriangle}
\newcommand{\decond}{\triangle}
\newcommand{\solid}{\blacksquare}
\newcommand{\ot}{\leftarrow}
\renewcommand{\-}{\text{-}}
\renewcommand{\mapsto}{\leadsto}
\renewcommand{\leq}{\leqslant}
\renewcommand{\geq}{\geqslant}
\renewcommand{\setminus}{\smallsetminus}
\makeatletter
\DeclareRobustCommand{\cev}[1]{%
  {\mathpalette\do@cev{#1}}%
}
\newcommand{\do@cev}[2]{%
  \vbox{\offinterlineskip
    \sbox\z@{$\m@th#1 x$}%
    \ialign{##\cr
      \hidewidth\reflectbox{$\m@th#1\vec{}\mkern4mu$}\hidewidth\cr
      \noalign{\kern-\ht\z@}
      $\m@th#1#2$\cr
    }%
  }%
}
\makeatother

\newcommand{\N}{\mathbb{N}}
\newcommand{\Z}{\mathbb{Z}}
\newcommand{\Q}{\mathbb{Q}}
\newcommand{\R}{\mathbb{R}}
\newcommand{\bbC}{\mathbb{C}}
\NewDocumentCommand{\x}{e{_^}}{%
  \mathbin{\mathop{\times}\displaylimits
    \IfValueT{#1}{_{#1}}
    \IfValueT{#2}{^{#2}}
  }%
}
\NewDocumentCommand{\pushout}{e{_^}}{%
  \mathbin{\mathop{\sqcup}\displaylimits
    \IfValueT{#1}{_{#1}}
    \IfValueT{#2}{^{#2}}
  }%
}
\newcommand{\supp}{\operatorname{supp}}
\newcommand{\im}{\operatorname{im}}
\newcommand{\coker}{\operatorname{coker}}
\newcommand{\id}{\mathrm{id}}
\newcommand{\chara}{\operatorname{char}}
\newcommand{\trdeg}{\operatorname{trdeg}}
\newcommand{\rank}{\operatorname{rank}}
\newcommand{\trace}{\operatorname{tr}}
\newcommand{\length}{\operatorname{length}}
\newcommand{\height}{\operatorname{ht}}
\renewcommand{\span}{\operatorname{span}}
\newcommand{\e}{\epsilon}
\newcommand{\p}{\mathfrak{p}}
\newcommand{\q}{\mathfrak{q}}
\newcommand{\m}{\mathfrak{m}}
\newcommand{\n}{\mathfrak{n}}
\newcommand{\calF}{\mathcal{F}}
\newcommand{\calG}{\mathcal{G}}
\newcommand{\calO}{\mathcal{O}}
\newcommand{\F}{\mathbb{F}}
\DeclareMathOperator{\lcm}{lcm}
\newcommand{\gr}{\operatorname{gr}}
\newcommand{\vol}{\mathrm{vol}}
\newcommand{\ord}{\operatorname{ord}}
\newcommand{\projdim}{\operatorname{proj.dim}}
\newcommand{\injdim}{\operatorname{inj.dim}}
\newcommand{\flatdim}{\operatorname{flat.dim}}
\newcommand{\globdim}{\operatorname{glob.dim}}
\renewcommand{\Re}{\operatorname{Re}}
\renewcommand{\Im}{\operatorname{Im}}
\newcommand{\sgn}{\operatorname{sgn}}
\newcommand{\coad}{\operatorname{coad}}

\newcommand{\Ad}{\mathrm{Ad}}
\newcommand{\GL}{\mathrm{GL}}
\newcommand{\SL}{\mathrm{SL}}
\newcommand{\PGL}{\mathrm{PGL}}
\newcommand{\PSL}{\mathrm{PSL}}
\newcommand{\Sp}{\mathrm{Sp}}
\newcommand{\GSp}{\mathrm{GSp}}
\newcommand{\GSpin}{\mathrm{GSpin}}
\newcommand{\rmO}{\mathrm{O}}
\newcommand{\SO}{\mathrm{SO}}
\newcommand{\SU}{\mathrm{SU}}
\newcommand{\rmU}{\mathrm{U}}
\newcommand{\rmu}{\mathrm{u}}
\newcommand{\rmV}{\mathrm{V}}
\newcommand{\gl}{\mathfrak{gl}}
\renewcommand{\sl}{\mathfrak{sl}}
\newcommand{\diag}{\mathfrak{diag}}
\newcommand{\pgl}{\mathfrak{pgl}}
\newcommand{\psl}{\mathfrak{psl}}
\newcommand{\fraksp}{\mathfrak{sp}}
\newcommand{\gsp}{\mathfrak{gsp}}
\newcommand{\gspin}{\mathfrak{gspin}}
\newcommand{\frako}{\mathfrak{o}}
\newcommand{\so}{\mathfrak{so}}
\newcommand{\su}{\mathfrak{su}}
%\newcommand{\fraku}{\mathfrak{u}}
\newcommand{\Spec}{\operatorname{Spec}}
\newcommand{\Spf}{\operatorname{Spf}}
\newcommand{\Spm}{\operatorname{Spm}}
\newcommand{\Spv}{\operatorname{Spv}}
\newcommand{\Spa}{\operatorname{Spa}}
\newcommand{\Spd}{\operatorname{Spd}}
\newcommand{\Proj}{\operatorname{Proj}}
\newcommand{\Gr}{\mathrm{Gr}}
\newcommand{\Hecke}{\mathrm{Hecke}}
\newcommand{\Sht}{\mathrm{Sht}}
\newcommand{\Quot}{\mathrm{Quot}}
\newcommand{\Hilb}{\mathrm{Hilb}}
\newcommand{\Pic}{\mathrm{Pic}}
\newcommand{\Div}{\mathrm{Div}}
\newcommand{\Jac}{\mathrm{Jac}}
\newcommand{\Alb}{\mathrm{Alb}} %albanese variety
\newcommand{\Bun}{\mathrm{Bun}}
\newcommand{\loopspace}{\mathbf{\Omega}}
\newcommand{\suspension}{\mathbf{\Sigma}}
\newcommand{\tangent}{\mathrm{T}} %tangent space
\newcommand{\Eig}{\mathrm{Eig}}
\newcommand{\Cox}{\mathrm{Cox}} %coxeter functors
\newcommand{\rmK}{\mathrm{K}} %Killing form
\newcommand{\km}{\mathfrak{km}} %kac-moody algebras
\newcommand{\Dyn}{\mathrm{Dyn}} %associated Dynkin quivers
\newcommand{\Car}{\mathrm{Car}} %cartan matrices of finite quivers

\newcommand{\Ring}{\mathrm{Ring}}
\newcommand{\Cring}{\mathrm{CRing}}
\newcommand{\Alg}{\mathrm{Alg}}
\newcommand{\Leib}{\mathrm{Leib}} %leibniz algebras
\newcommand{\Fld}{\mathrm{Fld}}
\newcommand{\Sets}{\mathrm{Sets}}
\newcommand{\Equiv}{\mathrm{Equiv}} %equivalence relations
\newcommand{\Cat}{\mathrm{Cat}}
\newcommand{\Grp}{\mathrm{Grp}}
\newcommand{\Ab}{\mathrm{Ab}}
\newcommand{\Sch}{\mathrm{Sch}}
\newcommand{\Coh}{\mathrm{Coh}}
\newcommand{\QCoh}{\mathrm{QCoh}}
\newcommand{\Perf}{\mathrm{Perf}} %perfect complexes
\newcommand{\Sing}{\mathrm{Sing}} %singularity categories
\newcommand{\Desc}{\mathrm{Desc}}
\newcommand{\Sh}{\mathrm{Sh}}
\newcommand{\Psh}{\mathrm{PSh}}
\newcommand{\Fib}{\mathrm{Fib}}
\renewcommand{\mod}{\-\mathrm{mod}}
\newcommand{\comod}{\-\mathrm{comod}}
\newcommand{\bimod}{\-\mathrm{bimod}}
\newcommand{\Vect}{\mathrm{Vect}}
\newcommand{\Rep}{\mathrm{Rep}}
\newcommand{\Grpd}{\mathrm{Grpd}}
\newcommand{\Arr}{\mathrm{Arr}}
\newcommand{\Esp}{\mathrm{Esp}}
\newcommand{\Ob}{\mathrm{Ob}}
\newcommand{\Mor}{\mathrm{Mor}}
\newcommand{\Mfd}{\mathrm{Mfd}}
\newcommand{\Riem}{\mathrm{Riem}}
\newcommand{\RS}{\mathrm{RS}}
\newcommand{\LRS}{\mathrm{LRS}}
\newcommand{\TRS}{\mathrm{TRS}}
\newcommand{\TLRS}{\mathrm{TLRS}}
\newcommand{\LVRS}{\mathrm{LVRS}}
\newcommand{\LBRS}{\mathrm{LBRS}}
\newcommand{\Spc}{\mathrm{Spc}}
\newcommand{\Top}{\mathrm{Top}}
\newcommand{\Topos}{\mathrm{Topos}}
\newcommand{\Nil}{\mathfrak{nil}}
\newcommand{\J}{\mathfrak{J}}
\newcommand{\Stk}{\mathrm{Stk}}
\newcommand{\Pre}{\mathrm{Pre}}
\newcommand{\simp}{\mathbf{\Delta}}
\newcommand{\Res}{\mathrm{Res}}
\newcommand{\Ind}{\mathrm{Ind}}
\newcommand{\Pro}{\mathrm{Pro}}
\newcommand{\Mon}{\mathrm{Mon}}
\newcommand{\Comm}{\mathrm{Comm}}
\newcommand{\Fin}{\mathrm{Fin}}
\newcommand{\Assoc}{\mathrm{Assoc}}
\newcommand{\Semi}{\mathrm{Semi}}
\newcommand{\Co}{\mathrm{Co}}
\newcommand{\Loc}{\mathrm{Loc}}
\newcommand{\Ringed}{\mathrm{Ringed}}
\newcommand{\Haus}{\mathrm{Haus}} %hausdorff spaces
\newcommand{\Comp}{\mathrm{Comp}} %compact hausdorff spaces
\newcommand{\Stone}{\mathrm{Stone}} %stone spaces
\newcommand{\Extr}{\mathrm{Extr}} %extremely disconnected spaces
\newcommand{\Ouv}{\mathrm{Ouv}}
\newcommand{\Str}{\mathrm{Str}}
\newcommand{\Func}{\mathrm{Func}}
\newcommand{\Crys}{\mathrm{Crys}}
\newcommand{\LocSys}{\mathrm{LocSys}}
\newcommand{\Sieves}{\mathrm{Sieves}}
\newcommand{\pt}{\mathrm{pt}}
\newcommand{\Graphs}{\mathrm{Graphs}}
\newcommand{\Lie}{\mathrm{Lie}}
\newcommand{\Env}{\mathrm{Env}}
\newcommand{\Ho}{\mathrm{Ho}}
\newcommand{\rmD}{\mathrm{D}}
\newcommand{\Cov}{\mathrm{Cov}}
\newcommand{\Frames}{\mathrm{Frames}}
\newcommand{\Locales}{\mathrm{Locales}}
\newcommand{\Span}{\mathrm{Span}}
\newcommand{\Corr}{\mathrm{Corr}}
\newcommand{\Monad}{\mathrm{Monad}}
\newcommand{\Var}{\mathrm{Var}}
\newcommand{\sfN}{\mathrm{N}} %nerve
\newcommand{\Diam}{\mathrm{Diam}} %diamonds
\newcommand{\co}{\mathrm{co}}
\newcommand{\ev}{\mathrm{ev}}
\newcommand{\bi}{\mathrm{bi}}
\newcommand{\Nat}{\mathrm{Nat}}
\newcommand{\Hopf}{\mathrm{Hopf}}
\newcommand{\Dmod}{\mathrm{D}\mod}
\newcommand{\Perv}{\mathrm{Perv}}
\newcommand{\Sph}{\mathrm{Sph}}
\newcommand{\Moduli}{\mathrm{Moduli}}
\newcommand{\Pseudo}{\mathrm{Pseudo}}
\newcommand{\Lax}{\mathrm{Lax}}
\newcommand{\Strict}{\mathrm{Strict}}
\newcommand{\Opd}{\mathrm{Opd}} %operads
\newcommand{\Shv}{\mathrm{Shv}}
\newcommand{\Char}{\mathrm{Char}} %CharShv = character sheaves
\newcommand{\Huber}{\mathrm{Huber}}
\newcommand{\Tate}{\mathrm{Tate}}
\newcommand{\Affd}{\mathrm{Affd}} %affinoid algebras
\newcommand{\Adic}{\mathrm{Adic}} %adic spaces
\newcommand{\Rig}{\mathrm{Rig}}
\newcommand{\An}{\mathrm{An}}
\newcommand{\Perfd}{\mathrm{Perfd}} %perfectoid spaces
\newcommand{\Sub}{\mathrm{Sub}} %subobjects
\newcommand{\Ideals}{\mathrm{Ideals}}
\newcommand{\Isoc}{\mathrm{Isoc}} %isocrystals
\newcommand{\Ban}{\-\mathrm{Ban}} %Banach spaces
\newcommand{\Fre}{\-\mathrm{Fr\acute{e}}} %Frechet spaces
\newcommand{\Ch}{\mathrm{Ch}} %chain complexes
\newcommand{\Pure}{\mathrm{Pure}}
\newcommand{\Mixed}{\mathrm{Mixed}}
\newcommand{\Hodge}{\mathrm{Hodge}} %Hodge structures
\newcommand{\Mot}{\mathrm{Mot}} %motives
\newcommand{\KL}{\mathrm{KL}} %category of Kazhdan-Lusztig modules
\newcommand{\Pres}{\mathrm{Pres}} %presentable categories
\newcommand{\Noohi}{\mathrm{Noohi}} %category of Noohi groups
\newcommand{\Inf}{\mathrm{Inf}}
\newcommand{\LPar}{\mathrm{LPar}} %Langlands parameters
\newcommand{\ORig}{\mathrm{ORig}} %overconvergent sites
\newcommand{\Quiv}{\mathrm{Quiv}} %quivers
\newcommand{\Def}{\mathrm{Def}} %deformation functors
\newcommand{\Root}{\mathrm{Root}}
\newcommand{\gRep}{\mathrm{gRep}}
\newcommand{\Higgs}{\mathrm{Higgs}}
\newcommand{\BGG}{\mathrm{BGG}}

\newcommand{\Aut}{\mathrm{Aut}}
\newcommand{\Inn}{\mathrm{Inn}}
\newcommand{\Out}{\mathrm{Out}}
\newcommand{\der}{\mathfrak{der}} %derivations on Lie algebras
\newcommand{\frakend}{\mathfrak{end}}
\newcommand{\aut}{\mathfrak{aut}}
\newcommand{\inn}{\mathfrak{inn}} %inner derivations
\newcommand{\out}{\mathfrak{out}} %outer derivations
\newcommand{\Stab}{\mathrm{Stab}}
\newcommand{\Cent}{\mathrm{Cent}}
\newcommand{\Norm}{\mathrm{Norm}}
\newcommand{\stab}{\mathfrak{stab}}
\newcommand{\cent}{\mathfrak{cent}}
\newcommand{\norm}{\mathfrak{norm}}
\newcommand{\Rad}{\operatorname{Rad}}
\newcommand{\Transporter}{\mathrm{Transp}} %transporter between two subsets of a group
\newcommand{\Conj}{\mathrm{Conj}}
\newcommand{\Diag}{\mathrm{Diag}}
\newcommand{\Gal}{\mathrm{Gal}}
\newcommand{\bfG}{\mathbf{G}} %absolute Galois group
\newcommand{\Frac}{\mathrm{Frac}}
\newcommand{\Ann}{\mathrm{Ann}}
\newcommand{\Val}{\mathrm{Val}}
\newcommand{\Chow}{\mathrm{Chow}}
\newcommand{\Sym}{\mathrm{Sym}}
\newcommand{\End}{\mathrm{End}}
\newcommand{\Mat}{\mathrm{Mat}}
\newcommand{\Diff}{\mathrm{Diff}}
\newcommand{\Autom}{\mathrm{Autom}}
\newcommand{\Artin}{\mathrm{Artin}} %artin maps
\newcommand{\sk}{\mathrm{sk}} %skeleton of a category
\newcommand{\eqv}{\mathrm{eqv}} %functor that maps groups $G$ to $G$-sets
\newcommand{\Inert}{\mathrm{Inert}}
\newcommand{\Fil}{\mathrm{Fil}}
\newcommand{\Prim}{\mathfrak{Prim}}
\newcommand{\Nerve}{\mathrm{N}}
\newcommand{\Hol}{\mathrm{Hol}} %holomorphic functions %holonomy groups
\newcommand{\Bi}{\mathrm{Bi}} %Bi for biholomorphic functions
\newcommand{\chev}{\mathfrak{chev}} %chevalley relations
\newcommand{\bfLie}{\mathbf{Lie}} %non-reduced lie algebra associated to generalised cartan matrices
\newcommand{\frakLie}{\mathfrak{Lie}} %reduced lie algebra associated to generalised cartan matrices
\newcommand{\frakChev}{\mathfrak{Chev}} 
\newcommand{\Rees}{\operatorname{Rees}}
\newcommand{\Dr}{\mathrm{Dr}} %Drinfeld's quantum double 

\renewcommand{\projlim}{\varprojlim}
\newcommand{\indlim}{\varinjlim}
\newcommand{\colim}{\operatorname{colim}}
\renewcommand{\lim}{\operatorname{lim}}
\newcommand{\toto}{\rightrightarrows}
%\newcommand{\tensor}{\otimes}
\NewDocumentCommand{\tensor}{e{_^}}{%
  \mathbin{\mathop{\otimes}\displaylimits
    \IfValueT{#1}{_{#1}}
    \IfValueT{#2}{^{#2}}
  }%
}
\NewDocumentCommand{\singtensor}{e{_^}}{%
  \mathbin{\mathop{\odot}\displaylimits
    \IfValueT{#1}{_{#1}}
    \IfValueT{#2}{^{#2}}
  }%
}
\NewDocumentCommand{\hattensor}{e{_^}}{%
  \mathbin{\mathop{\hat{\otimes}}\displaylimits
    \IfValueT{#1}{_{#1}}
    \IfValueT{#2}{^{#2}}
  }%
}
\NewDocumentCommand{\semidirect}{e{_^}}{%
  \mathbin{\mathop{\rtimes}\displaylimits
    \IfValueT{#1}{_{#1}}
    \IfValueT{#2}{^{#2}}
  }%
}
\newcommand{\eq}{\operatorname{eq}}
\newcommand{\coeq}{\operatorname{coeq}}
\newcommand{\Hom}{\mathrm{Hom}}
\newcommand{\Maps}{\mathrm{Maps}}
\newcommand{\Tor}{\mathrm{Tor}}
\newcommand{\Ext}{\mathrm{Ext}}
\newcommand{\Isom}{\mathrm{Isom}}
\newcommand{\stalk}{\mathrm{stalk}}
\newcommand{\RKE}{\operatorname{RKE}}
\newcommand{\LKE}{\operatorname{LKE}}
\newcommand{\oblv}{\mathrm{oblv}}
\newcommand{\const}{\mathrm{const}}
\newcommand{\free}{\mathrm{free}}
\newcommand{\adrep}{\mathrm{ad}} %adjoint representation
\newcommand{\NL}{\mathbb{NL}} %naive cotangent complex
\newcommand{\pr}{\operatorname{pr}}
\newcommand{\Der}{\mathrm{Der}}
\newcommand{\Frob}{\mathrm{Fr}} %Frobenius
\newcommand{\frob}{\mathrm{f}} %trace of Frobenius
\newcommand{\bfpt}{\mathbf{pt}}
\newcommand{\bfloc}{\mathbf{loc}}
\DeclareMathAlphabet{\mymathbb}{U}{BOONDOX-ds}{m}{n}
\newcommand{\0}{\mymathbb{0}}
\newcommand{\1}{\mathbbm{1}}
\newcommand{\2}{\mathbbm{2}}
\newcommand{\Jet}{\mathrm{Jet}}
\newcommand{\Split}{\mathrm{Split}}
\newcommand{\Sq}{\mathrm{Sq}}
\newcommand{\Zero}{\mathrm{Z}}
\newcommand{\SqZ}{\Sq\Zero}
\newcommand{\lie}{\mathfrak{lie}}
\newcommand{\y}{\mathrm{y}} %yoneda
\newcommand{\Sm}{\mathrm{Sm}}
\newcommand{\AJ}{\phi} %abel-jacobi map
\newcommand{\act}{\mathrm{act}}
\newcommand{\ram}{\mathrm{ram}} %ramification index
\newcommand{\inv}{\mathrm{inv}}
\newcommand{\Spr}{\mathrm{Spr}} %the Springer map/sheaf
\newcommand{\Refl}{\mathrm{Refl}} %reflection functor]
\newcommand{\HH}{\mathrm{HH}} %Hochschild (co)homology
\newcommand{\Poinc}{\mathrm{Poinc}}
\newcommand{\Simpson}{\mathrm{Simpson}}

\newcommand{\bbU}{\mathbb{U}}
\newcommand{\V}{\mathbb{V}}
\newcommand{\calU}{\mathcal{U}}
\newcommand{\calW}{\mathcal{W}}
\newcommand{\rmI}{\mathrm{I}} %augmentation ideal
\newcommand{\bfV}{\mathbf{V}}
\newcommand{\C}{\mathcal{C}}
\newcommand{\D}{\mathcal{D}}
\newcommand{\T}{\mathscr{T}} %Tate modules
\newcommand{\calM}{\mathcal{M}}
\newcommand{\calN}{\mathcal{N}}
\newcommand{\calP}{\mathcal{P}}
\newcommand{\calQ}{\mathcal{Q}}
\newcommand{\A}{\mathbb{A}}
\renewcommand{\P}{\mathbb{P}}
\newcommand{\calL}{\mathcal{L}}
\newcommand{\E}{\mathcal{E}}
\renewcommand{\H}{\mathbf{H}}
\newcommand{\scrS}{\mathscr{S}}
\newcommand{\calX}{\mathcal{X}}
\newcommand{\calY}{\mathcal{Y}}
\newcommand{\calZ}{\mathcal{Z}}
\newcommand{\calS}{\mathcal{S}}
\newcommand{\calR}{\mathcal{R}}
\newcommand{\scrX}{\mathscr{X}}
\newcommand{\scrY}{\mathscr{Y}}
\newcommand{\scrZ}{\mathscr{Z}}
\newcommand{\calA}{\mathcal{A}}
\newcommand{\calB}{\mathcal{B}}
\renewcommand{\S}{\mathcal{S}}
\newcommand{\B}{\mathbb{B}}
\newcommand{\bbD}{\mathbb{D}}
\newcommand{\G}{\mathbb{G}}
\newcommand{\horn}{\mathbf{\Lambda}}
\renewcommand{\L}{\mathbb{L}}
\renewcommand{\a}{\mathfrak{a}}
\renewcommand{\b}{\mathfrak{b}}
\renewcommand{\c}{\mathfrak{c}}
\renewcommand{\t}{\mathfrak{t}}
\renewcommand{\r}{\mathfrak{r}}
\newcommand{\fraku}{\mathfrak{u}}
\newcommand{\bbX}{\mathbb{X}}
\newcommand{\frakw}{\mathfrak{w}}
\newcommand{\frakG}{\mathfrak{G}}
\newcommand{\frakH}{\mathfrak{H}}
\newcommand{\frakE}{\mathfrak{E}}
\newcommand{\frakF}{\mathfrak{F}}
\newcommand{\g}{\mathfrak{g}}
\newcommand{\h}{\mathfrak{h}}
\renewcommand{\k}{\mathfrak{k}}
\newcommand{\z}{\mathfrak{z}}
\newcommand{\fraki}{\mathfrak{i}}
\newcommand{\frakj}{\mathfrak{j}}
\newcommand{\del}{\partial}
\newcommand{\bbE}{\mathbb{E}}
\newcommand{\scrO}{\mathscr{O}}
\newcommand{\bbO}{\mathbb{O}}
\newcommand{\scrA}{\mathscr{A}}
\newcommand{\scrB}{\mathscr{B}}
\newcommand{\scrF}{\mathscr{F}}
\newcommand{\scrG}{\mathscr{G}}
\newcommand{\scrM}{\mathscr{M}}
\newcommand{\scrN}{\mathscr{N}}
\newcommand{\scrP}{\mathscr{P}}
\newcommand{\frakS}{\mathfrak{S}}
\newcommand{\frakT}{\mathfrak{T}}
\newcommand{\calI}{\mathcal{I}}
\newcommand{\calJ}{\mathcal{J}}
\newcommand{\scrI}{\mathscr{I}}
\newcommand{\scrJ}{\mathscr{J}}
\newcommand{\scrK}{\mathscr{K}}
\newcommand{\calK}{\mathcal{K}}
\newcommand{\scrV}{\mathscr{V}}
\newcommand{\scrW}{\mathscr{W}}
\newcommand{\bbS}{\mathbb{S}}
\newcommand{\scrH}{\mathscr{H}}
\newcommand{\bfA}{\mathbf{A}}
\newcommand{\bfB}{\mathbf{B}}
\newcommand{\bfC}{\mathbf{C}}
\renewcommand{\O}{\mathbb{O}}
\newcommand{\calV}{\mathcal{V}}
\newcommand{\scrR}{\mathscr{R}} %radical
\newcommand{\rmZ}{\mathrm{Z}} %centre of algebra
\newcommand{\rmC}{\mathrm{C}} %centralisers in algebras
\newcommand{\bfGamma}{\mathbf{\Gamma}}
\newcommand{\scrU}{\mathscr{U}}
\newcommand{\rmW}{\mathrm{W}} %Weil group
\newcommand{\frakM}{\mathfrak{M}}
\newcommand{\frakN}{\mathfrak{N}}
\newcommand{\frakB}{\mathfrak{B}}
\newcommand{\frakX}{\mathfrak{X}}
\newcommand{\frakY}{\mathfrak{Y}}
\newcommand{\frakZ}{\mathfrak{Z}}
\newcommand{\frakU}{\mathfrak{U}}
\newcommand{\frakR}{\mathfrak{R}}
\newcommand{\frakP}{\mathfrak{P}}
\newcommand{\frakQ}{\mathfrak{Q}}
\newcommand{\sfX}{\mathsf{X}}
\newcommand{\sfY}{\mathsf{Y}}
\newcommand{\sfZ}{\mathsf{Z}}
\newcommand{\sfS}{\mathsf{S}}
\newcommand{\sfT}{\mathsf{T}}
\newcommand{\sfOmega}{\mathsf{\Omega}} %drinfeld p-adic upper-half plane
\newcommand{\rmA}{\mathrm{A}}
\newcommand{\rmB}{\mathrm{B}}
\newcommand{\calT}{\mathcal{T}}
\newcommand{\sfA}{\mathsf{A}}
\newcommand{\sfD}{\mathsf{D}}
\newcommand{\sfE}{\mathsf{E}}
\newcommand{\frakL}{\mathfrak{L}}
\newcommand{\K}{\mathrm{K}}
\newcommand{\rmT}{\mathrm{T}}
\newcommand{\bfv}{\mathbf{v}}
\newcommand{\bfg}{\mathbf{g}}
\newcommand{\frakV}{\mathfrak{V}}
\newcommand{\frakv}{\mathfrak{v}}
\newcommand{\bfn}{\mathbf{n}}
\renewcommand{\o}{\mathfrak{o}}

\newcommand{\aff}{\mathrm{aff}}
\newcommand{\ft}{\mathrm{ft}} %finite type
\newcommand{\fp}{\mathrm{fp}} %finite presentation
\newcommand{\fr}{\mathrm{fr}} %free
\newcommand{\tft}{\mathrm{tft}} %topologically finite type
\newcommand{\tfp}{\mathrm{tfp}} %topologically finite presentation
\newcommand{\tfr}{\mathrm{tfr}} %topologically free
\newcommand{\aft}{\mathrm{aft}}
\newcommand{\lft}{\mathrm{lft}}
\newcommand{\laft}{\mathrm{laft}}
\newcommand{\cpt}{\mathrm{cpt}}
\newcommand{\cproj}{\mathrm{cproj}}
\newcommand{\qc}{\mathrm{qc}}
\newcommand{\qs}{\mathrm{qs}}
\newcommand{\lcmpt}{\mathrm{lcmpt}}
\newcommand{\red}{\mathrm{red}}
\newcommand{\fin}{\mathrm{fin}}
\newcommand{\fd}{\mathrm{fd}} %finite-dimensional
\newcommand{\gen}{\mathrm{gen}}
\newcommand{\petit}{\mathrm{petit}}
\newcommand{\gros}{\mathrm{gros}}
\newcommand{\loc}{\mathrm{loc}}
\newcommand{\glob}{\mathrm{glob}}
%\newcommand{\ringed}{\mathrm{ringed}}
%\newcommand{\qcoh}{\mathrm{qcoh}}
\newcommand{\cl}{\mathrm{cl}}
\newcommand{\et}{\mathrm{\acute{e}t}}
\newcommand{\fet}{\mathrm{f\acute{e}t}}
\newcommand{\profet}{\mathrm{prof\acute{e}t}}
\newcommand{\proet}{\mathrm{pro\acute{e}t}}
\newcommand{\Zar}{\mathrm{Zar}}
\newcommand{\fppf}{\mathrm{fppf}}
\newcommand{\fpqc}{\mathrm{fpqc}}
\newcommand{\orig}{\mathrm{orig}} %overconvergent topology
\newcommand{\smooth}{\mathrm{sm}}
\newcommand{\sh}{\mathrm{sh}}
\newcommand{\op}{\mathrm{op}}
\newcommand{\cop}{\mathrm{cop}}
\newcommand{\open}{\mathrm{open}}
\newcommand{\closed}{\mathrm{closed}}
\newcommand{\geom}{\mathrm{geom}}
\newcommand{\alg}{\mathrm{alg}}
\newcommand{\sober}{\mathrm{sober}}
\newcommand{\dR}{\mathrm{dR}}
\newcommand{\rad}{\mathfrak{rad}}
\newcommand{\discrete}{\mathrm{discrete}}
%\newcommand{\add}{\mathrm{add}}
%\newcommand{\lin}{\mathrm{lin}}
\newcommand{\Krull}{\mathrm{Krull}}
\newcommand{\qis}{\mathrm{qis}} %quasi-isomorphism
\newcommand{\ho}{\mathrm{ho}} %homotopy equivalence
\newcommand{\sep}{\mathrm{sep}}
\newcommand{\unr}{\mathrm{unr}}
\newcommand{\tame}{\mathrm{tame}}
\newcommand{\wild}{\mathrm{wild}}
\newcommand{\nil}{\mathrm{nil}}
\newcommand{\defm}{\mathrm{defm}}
\newcommand{\Art}{\mathrm{Art}}
\newcommand{\Noeth}{\mathrm{Noeth}}
\newcommand{\affd}{\mathrm{affd}}
%\newcommand{\adic}{\mathrm{adic}}
\newcommand{\pre}{\mathrm{pre}}
\newcommand{\coperf}{\mathrm{coperf}}
\newcommand{\perf}{\mathrm{perf}}
\newcommand{\perfd}{\mathrm{perfd}}
\newcommand{\rat}{\mathrm{rat}}
\newcommand{\cont}{\mathrm{cont}}
\newcommand{\dg}{\mathrm{dg}}
\newcommand{\almost}{\mathrm{a}}
%\newcommand{\stab}{\mathrm{stab}}
\newcommand{\heart}{\heartsuit}
\newcommand{\proj}{\mathrm{proj}}
\newcommand{\qproj}{\mathrm{qproj}}
\newcommand{\pd}{\mathrm{pd}}
\newcommand{\crys}{\mathrm{crys}}
\newcommand{\prisma}{\mathrm{prisma}}
\newcommand{\FF}{\mathrm{FF}}
\newcommand{\sph}{\mathrm{sph}}
\newcommand{\lax}{\mathrm{lax}}
\newcommand{\weak}{\mathrm{weak}}
\newcommand{\strict}{\mathrm{strict}}
\newcommand{\mon}{\mathrm{mon}}
\newcommand{\sym}{\mathrm{sym}}
\newcommand{\lisse}{\mathrm{lisse}}
\newcommand{\an}{\mathrm{an}}
\newcommand{\ad}{\mathrm{ad}}
\newcommand{\sch}{\mathrm{sch}}
\newcommand{\rig}{\mathrm{rig}}
\newcommand{\pol}{\mathrm{pol}}
\newcommand{\plat}{\mathrm{flat}}
\newcommand{\proper}{\mathrm{proper}}
\newcommand{\compl}{\mathrm{compl}}
\newcommand{\non}{\mathrm{non}}
\newcommand{\access}{\mathrm{access}}
\newcommand{\comp}{\mathrm{comp}}
\newcommand{\tstructure}{\mathrm{t}} %t-structures
\newcommand{\pure}{\mathrm{pure}} %pure motives
\newcommand{\mixed}{\mathrm{mixed}} %mixed motives
\newcommand{\num}{\mathrm{num}} %numerical motives
\newcommand{\ess}{\mathrm{ess}}
\newcommand{\topological}{\mathrm{top}}
\newcommand{\convex}{\mathrm{cvx}}
\newcommand{\locconvex}{\mathrm{lcvx}}
\newcommand{\ab}{\mathrm{ab}} %abelian extensions
\newcommand{\inj}{\mathrm{inj}}
\newcommand{\surj}{\mathrm{surj}} %coverage on sets generated by surjections
\newcommand{\eff}{\mathrm{eff}} %effective Cartier divisors
\newcommand{\Weil}{\mathrm{Weil}} %weil divisors
\newcommand{\lex}{\mathrm{lex}}
\newcommand{\rex}{\mathrm{rex}}
\newcommand{\AR}{\mathrm{A\-R}}
\newcommand{\cons}{\mathrm{c}} %constructible sheaves
\newcommand{\tor}{\mathrm{tor}} %tor dimension
\newcommand{\semisimple}{\mathrm{ss}}
\newcommand{\connected}{\mathrm{connected}}
\newcommand{\cg}{\mathrm{cg}} %compactly generated
\newcommand{\nilp}{\mathrm{nilp}}
\newcommand{\isg}{\mathrm{isg}} %isogenous
\newcommand{\qisg}{\mathrm{qisg}} %quasi-isogenous
\newcommand{\irr}{\mathrm{irr}} %irreducible represenations
\newcommand{\simple}{\mathrm{simple}} %simple objects
\newcommand{\indecomp}{\mathrm{indecomp}}
\newcommand{\preproj}{\mathrm{preproj}}
\newcommand{\preinj}{\mathrm{preinj}}
\newcommand{\reg}{\mathrm{reg}}
\renewcommand{\ss}{\mathrm{ss}}

%prism custom command
\usepackage{relsize}
\usepackage[bbgreekl]{mathbbol}
\usepackage{amsfonts}
\DeclareSymbolFontAlphabet{\mathbb}{AMSb} %to ensure that the meaning of \mathbb does not change
\DeclareSymbolFontAlphabet{\mathbbl}{bbold}
\newcommand{\prism}{{\mathlarger{\mathbbl{\Delta}}}}
\renewcommand{\simpleroots}{\mathbb{I}}
\newcommand{\romanzero}{\mathsf{0}}
\newcommand{\romanone}{\mathsf{I}}
\newcommand{\romantwo}{\mathsf{II}}
\newcommand{\romanthree}{\mathsf{III}}
\newcommand{\BCDzero}{\sfB\sfC\sfD\romanzero}
\newcommand{\UB}{\mathcal{UB}}
\newcommand{\XB}{\mathcal{XB}}
\newcommand{\ZB}{\mathcal{ZB}}

\begin{document}

    \title{Isaev-Molev-Ogievetsky reflection algebras vs. $\BCDzero$ twisted Yangians}
    
    \author{Dat Minh Ha}
    \maketitle
    
    \begin{abstract}
        We compare the constructions of the reflection algebras of Isaev-Molev-Ogievetsky (IMO) with the twisted Yangian - in the sense of Guay-Regelskis - of type $\BCDzero$.
    \end{abstract}
    
    {
    \hypersetup{} 
    %\dominitoc
    \tableofcontents %sort sections alphabetically
    }

    \section{Notations}
        \subsection{Generating series}

        \subsection{Symmetric pairs of classical Lie algebras}
            All throughout:
                $$\g_N \subset \gl_N$$
            will be used to denote a finite-dimensional simple Lie algebra over $\bbC$. This Lie algebra has a natural action on $U := \bbC^{\oplus N}$ (the so-called \say{vector representation}), which shall always be equipped with the standard basis:
                $$\{e_i\}_{1 \leq i \leq N}$$
            consisting of vectors with $1$ in the $i^{th}$ entry and $0$ elsewhere. By finite-dimensionality, we identify:
                $$U^{\tensor 2} \xrightarrow[]{\cong} \End(U)$$
            and then write:
                $$E_{i, j}$$
            for the image of $e_i \tensor e_j$ under the identification above; explicitly, $E_{i, j}$ is the $N \x N$ matrix with $1$ in the $(i, j)^{th}$ entry and $0$ elsewhere.

    \section{Recollections about Yangians}
        \subsection{Transfer matrices in the absence of boundary conditions}
            Let:
                $$\calR(u) \in \End(U)^{\tensor 2}(\!(u)\!)$$
            be a spectral solution to the quantum Yang-Baxter equation (QYBE) on $U(\!(u_1)\!) \tensor U(\!(u_2)\!) \tensor U(\!(u_3)\!)$, which reads:
                \begin{equation} \label{equation: QYBE}
                    \calR_{1, 2}(u_1 - u_2) \calR_{1, 3}(u_1 - u_3) \calR_{2, 3}(u_2 - u_3) = \calR_{2, 3}(u_2 - u_3) \calR_{1, 3}(u_1 - u_3) \calR_{1, 2}(u_1 - u_2)
                \end{equation}
            wherein the subscript pairs indicate the pair of tensor copies that $\calR(u)$ is acting on. From such a solution, one can extract the so-called \textbf{transfer matrix}:
                $$T(u) \in $$

        \subsection{Extended untwisted Yangians}
            Following \cite[Definition 2.1]{guay_regelskis_twisted_yangians_for_symmetric_pairs_of_types_BCD}, we work with the following definition of extended untwisted Yangians associated to a spectral rational solution $\calR(u)$ to the QYBE on $U^{\tensor 3}$.
            \begin{definition}[Extended untwisted Yangians] \label{def: extended_untwisted_yangians}
                
            \end{definition}
            \begin{definition}[Untwisted Yangians] \label{def: untwisted_yangians}
                
            \end{definition}

            From now on, assume that $\g_N$ is of one of the types $\sfB, \sfC, \sfD$. We will be fixing once and for all a solution:
                $$\calR(u) = 1 + \frac{P}{u} + \frac{Q}{u - \kappa}$$
            for $\kappa := \sgn(\g_N) 1 + \frac{N}{2}$, with:
                \begin{equation} \label{equation: BCD_signature}
                    \sgn(\g_N) =
                    \begin{cases}
                        \text{$1$ if $\g_N$ is of type $\sfC$}
                        \\
                        \text{$-1$ if $\g_N$ is of either type $\sfB$ or $\sfD$}
                    \end{cases}
                \end{equation}
            As such, we can abbreviate:
                $$\calX(\g_N) := \calX(\g_N, \calR)$$
                $$\calY(\g_N) := \calY(\g_N, \calR)$$

            \begin{lemma}[Automorphisms of extended untwisted Yangians] \label{lemma: automorphisms_of_extended_untwisted_yangians}
                \begin{enumerate}
                    \item Let $f(u) \in \bbC[\![u^{-1}]\!]^{\x}$ be an invertible formal power series in $u^{-1}$; recall that, because $\bbC[\![u^{-1}]\!]$ is a local commutative ring with (unique) maximal ideal $u^{-1}\bbC[\![u^{-1}]\!]$, $f(u)$ must be of the form $1 + \sum_{r \geq 0} f^{(r)} u^{-r - 1}$. Then, the map:
                        $$\mu_f: \calX(\g_N)[\![u^{-1}]\!] \to \Mat_N(\calX(\g_N)[\![u^{-1}]\!])$$
                    given by:
                        $$\mu_f( T(u) ) := f(u) T(u)$$
                    defines an algebra automorphism of $\calX(\g_N)$.
                    \item Let $a \in \bbC$. Then, the map:
                        $$\tau_a: \Mat_N(\calX(\g_N)[\![u^{-1}]\!]) \to \Mat_N(\calX(\g_N)[\![u^{-1}]\!])$$
                    given by:
                        $$\tau_a( T(u) ) := T(u - a)$$
                    defines an algebra automorphism of $\calX(\g_N)$.
                    \item Let $A \in \gl_N$ be such that $AA^t = 1$. Then, the map:
                        $$\alpha_A: \Mat_N(\calX(\g_N)[\![u^{-1}]\!]) \to \Mat_N(\calX(\g_N)[\![u^{-1}]\!])$$
                    given by:
                        $$\alpha_A( T(u) ) := A T(u) A^t$$
                    defines an algebra automorphism of $\calX(\g_N)$.
                \end{enumerate}
            \end{lemma}
                \begin{proof}
                    
                \end{proof}

            \begin{lemma}[Quantum contractions] \label{lemma: quantum_contractions}
                There is a distinguished invertible series:
                    $$z(u) \in \calX(\g_N)[\![u^{-1}]\!]^{\x}$$
                called the \textbf{quantum contraction} of the transfer matrix $T(u)$, which is such that:
                    \begin{equation} \label{equation: quantum_contraction}
                        T(u + \kappa)^t T(u) = T(u) T(u + \kappa)^t = z(u)
                    \end{equation}
                Moreover, the coefficients of $z(u)$ generated the centre $\calZ(\g_N) := \rmZ( \calX(\g_N) )$.
            \end{lemma}
                \begin{proof}
                    
                \end{proof}
            \begin{corollary}
                There is an algebra isomorphism:
                    $$\calX(\g_N) \cong \calY(\g_N) \tensor \calZ(\g_N)$$
                thereby giving us the realisation:
                    $$\calY(\g_N) \cong \calX(\g_N)/\<z(u) - 1\>$$
            \end{corollary}

    \section{The two algebras}
        \subsection{Transfer matrices with boundary conditions}
    
        \subsection{Twisted Yangians of type \texorpdfstring{$\BCDzero$}{}}
            \begin{definition}[Twisted Yangians associated to symmetric pairs] \label{def: twisted_yangians_associated_to_symmetric_pairs}
                
            \end{definition}
    
        \subsection{The IMO reflection algebras}
            In this subsection, let $\g_N$ be of one of the types $\sfB, \sfC, \sfD$ in the Cartan-Killing Classification, and let $\vartheta := \id$.
        
            \begin{definition}[IMO reflection algebras] \label{def: IMO_reflection_algebras}
                The \textbf{reflection algebra} of Isaev-Molev-Ogievetsky, associated to the symmetric pair $(\g_N, \id)$ of type $\BCDzero$, is the associative algebra:
                    $$\calB(\g_N, \id)$$
                generated by the coefficients of the entries of the boundary transfer matrix $B(u) \in \End(U \tensor U)[\![u^{-1}]\!]$. These generators are subjected to the following so-called \textbf{reflection equation}:
                    \todo[inline]{Reflection equation}
                In \cite{isaev_molev_ogievetsky_fusion_for_brauer_algebras_2} imposed also the following version of unitarity:
                    $$B(u) B(-u) = 1$$
                In order to stay consistent with the notations from \cite{guay_regelskis_twisted_yangians_for_symmetric_pairs_of_types_BCD}, let us denote the resulting algebra by:
                    $$\UB(\g_N, \id)$$
                and refer to it as the \textbf{unitary reflection algebra}.
            \end{definition}
            \begin{remark}
                This is not quite the same as the reflection algebra of Molev-Ragoucy from \cite{molev_ragoucy_representations_of_reflection_algebras}, which should correspond to the twisted Yangian of type $\sfA \romanthree$.
            \end{remark}

            \begin{proposition}[A homomorphism $\UB(\g_N, \id) \to \calX(\g_N)$] \label{prop: mapping_IMO_reflection_algebras_to_extended_yangians}
                There is an algebra homomorphism:
                    \begin{equation} \label{equation: mapping_IMO_reflection_algebras_to_extended_yangians}
                        \varphi: \UB(\g_N, \id) \to \calX(\g_N)
                    \end{equation}
                given by:
                    $$\varphi(B(u)) := T\left(u - \frac12 \kappa\right) T\left(-u + \frac12 \kappa\right)^{-1}$$
                Moreover, this homomorphism is not injective in general.
            \end{proposition}
                \begin{proof}
                    For a proof that the given map is a well-defined algebra homomorphism, we refer the reader to \cite[Proposition 3.2]{isaev_molev_ogievetsky_fusion_for_brauer_algebras_2}. 
                \end{proof}

        \subsection{Comparison}
            We would now like to see if the algebra homomorphism:
                $$\varphi: \UB(\g_N, \id) \to \calX(\g_N)$$
            from proposition \ref{prop: mapping_IMO_reflection_algebras_to_extended_yangians} factors in the following manner:
                \begin{equation} \label{equation: mapping_IMO_reflection_algebras_to_extended_untwisted_yangians_via_extended_twisted_yangians}
                    \begin{tikzcd}
                	{\UB(\g_N, \id)} & {\calX^{\tw}(\g_N, \id)} \\
                	& {\calX(\g_N)}
                	\arrow["{\exists ? \varphi^{\tw}}", dashed, from=1-1, to=1-2]
                	\arrow["\varphi"', from=1-1, to=2-2]
                	\arrow[hook, from=1-2, to=2-2]
                    \end{tikzcd}
                \end{equation}
            wherein the unlabelled arrow is the canonical algebra embedding. To this end, recall first of all that the boundary transfer matrix $S(u) \in \Mat_N(\calX^{\tw}(\g_N, \id)[\![u^{-1}]\!])$ are realisable as the following element of $\Mat_N(\calX(\g_N)[\![u^{-1}]\!])$:
                $$S(u) = T\left(u - \frac12 \kappa\right) T\left(-u + \frac12 \kappa\right)^t$$
            At the same time, we know from proposition \ref{prop: mapping_IMO_reflection_algebras_to_extended_yangians} that the image of the generators $B(u) \in \UB(\g_N, \id)[\![u^{-1}]\!]$ in $\calX(\g_N)$ is:
                $$\varphi( B(u) ) = T\left(u - \frac12 \kappa\right) T\left(-u + \frac12 \kappa\right)^{-1}$$
            As generators are mapped to generators by homomorphisms, in order for there to be an algebra homomorphism:
                $$\varphi^{\tw}: \UB(\g_N, \id) \to \calX^{\tw}(\g_N, \id)$$
            there must exist an algebra automorphism:
                $$b(u) \in \Aut_{\Assoc\Alg}( \Mat_N(\calX^{\tw}(\g_N, \id)[\![u^{-1}]\!]) )$$
            such that the following holds in $\Mat_N(\calX^{\tw}(\g_N, \id)[\![u^{-1}]\!])$:
                $$
                    \begin{aligned}
                        \varphi(B(u)) & = T\left(u - \frac12 \kappa\right) T\left(-u + \frac12 \kappa\right)^{-1}
                        \\
                        & = b(u) \cdot T\left(u - \frac12 \kappa\right) T\left(-u + \frac12 \kappa\right)^t
                        \\
                        & = b(u) \cdot S(u)
                    \end{aligned}
                $$
            wherein $b(u) \cdot$ means the application of the automorphism.
            
            To check if this is true, we begin by recalling that, as a result of equation \eqref{equation: quantum_contraction}, we have the following version of quantum contraction for the boundary transfer matrix:
                $$S(u) S(-u) = z^{\tw}(u)$$
            for some even and invertible central series $z^{\tw}(u) \in \calZ^{\tw}(\g_N, \id)[\![u^{-1}]\!]^{\x}$.
            
            Consider now the case when $b(u)$ is an inner automorphism, particularly when there exists some invertible central series:
                $$b(u) \in \calZ^{\tw}(\g_N, \id)[\![u^{-1}]\!]^{\x}$$
            that acts on elements of $\Mat_N(\calX^{\tw}(\g_N, \id)[\![u^{-1}]\!])$ by left-multiplication (equivalently, right-multiplication), we must require that:
                $$
                    \begin{aligned}
                        \varphi( B(u) B(-u) ) & = T\left(u - \frac12 \kappa\right) T\left(-u + \frac12 \kappa\right)^{-1} \cdot T\left(-u - \frac12 \kappa\right) T\left(u + \frac12 \kappa\right)^{-1}
                        \\
                        & = b(u) S(u) \cdot b(-u) S(-u)
                        \\
                        & = b(u) b(-u) \cdot S(u) S(-u)
                        \\
                        & = b(u) b(-u) \cdot z^{\tw}(u)
                    \end{aligned}
                $$
            At the same time, the construction of the unitary reflection algebra $\UB(\g_N, \id)$ stipulates that:
                $$B(u) B(-u) = 1$$
            so in fact, we have:
                $$\varphi( B(u) B(-u) ) = 1 = b(u) b(-u) \cdot z^{\tw}(u)$$
            Therefore, we see that when $b(u)$ is an inner automorphism, the existence of $\varphi^{\tw}$ as in diagram \eqref{equation: mapping_IMO_reflection_algebras_to_extended_untwisted_yangians_via_extended_twisted_yangians} is equivalent to the following.
            \begin{question} \label{question: boundary_quantum_contraction_factorisation}
                Is it possible to factor:
                    \begin{equation} \label{equation: boundary_quantum_contraction_factorisation}
                        z^{\tw}(u)^{-1} = b(u) b(-u) \in \calZ^{\tw}(\g_N, \id)[\![u^{-1}]\!]^{\x}
                    \end{equation}
                for some $b(u) \in \calZ^{\tw}(\g_N, \id)[\![u^{-1}]\!]^{\x}$ ?
            \end{question}
            \begin{remark}
                This is consistent with the fact that $z^{\tw}(u)$ is an even formal power series \textit{a priori}.
            \end{remark}
            
            To this end, note that because $S(u) = T\left(u - \frac12 \kappa\right) T\left(-u + \frac12 \kappa\right)^t$ and because we have the quantum contraction formula \eqref{equation: quantum_contraction}, we can expand the quantum contraction of the boundary transfer matrix $S(u)$ as:
                $$
                    \begin{aligned}
                        z^{\tw}(u) & = S(u) S(-u)
                        \\
                        & = T\left(u - \frac12 \kappa\right) T\left(-u + \frac12 \kappa\right)^t \cdot T\left(-u - \frac12 \kappa\right) T\left(u + \frac12 \kappa\right)^t
                        \\
                        & = T\left(u - \frac12 \kappa\right) \cdot z\left(-u - \frac12 \kappa\right) \cdot T\left(u + \frac12 \kappa\right)^t
                        \\
                        & = z\left(-u - \frac12 \kappa\right) \cdot T\left(u - \frac12 \kappa\right) T\left(u + \frac12 \kappa\right)^t
                        \\
                        & = z\left(-u - \frac12 \kappa\right) z\left(u - \frac12 \kappa\right)
                    \end{aligned}
                $$
            Next, recall from \cite[Section 2, p. 8]{arnaudon_molev_ragoucy_R_matrix_presentation_for_yangians} that applying the antipode $\sigma: \calX(\g_N) \to \calX(\g_N)^{\op, \cop}$ of the canonical Hopf algebra structure on $\calX(\g_N)$ to the transfer matrix $T(u)$ yields\footnote{It is understood that the antipode is applied matrix entry-wise.}:
                $$\sigma( T(u) ) = z(u)^{-1} T(u + \kappa)^t$$
            or equivalently, the following:
                $$\sigma\left( T\left( \pm u - \frac12\kappa \right) \right) = z\left( \pm u - \frac12\kappa \right)^{-1} T\left( \pm u + \frac12\kappa \right)^t$$
                $$\sigma\left( T\left( \pm u + \frac12\kappa \right) \right) = z\left( \pm u + \frac12\kappa \right)^{-1} T\left( \pm u + \frac32\kappa \right)^t$$
            The two facts then imply together that the following holds in $\Mat_N(\calX^{\tw}(\g_N, \id)[\![u^{-1}]\!])$:
                $$
                    \begin{aligned}
                        \sigma( S(u) ) & = \sigma\left( T\left(u - \frac12 \kappa\right) T\left(-u + \frac12 \kappa\right)^t \right)
                        \\
                        & = \sigma\left( T\left(-u + \frac12 \kappa\right)^t \right) \sigma\left( T\left(u - \frac12 \kappa\right) \right)
                        \\
                        & = z\left( -u + \frac12\kappa \right)^{-1} T\left( -u + \frac32\kappa \right) \cdot z\left( u + \frac12\kappa \right)^{-1} T\left( u + \frac12\kappa \right)^t
                        \\
                        & = z\left( -u + \frac12\kappa \right)^{-1} z\left( u + \frac12\kappa \right)^{-1} \cdot \left( T\left( u + \frac12\kappa \right) T\left( -u + \frac32\kappa \right)^t \right)^t
                        \\
                        & = z^{\tw}(u + \kappa)^{-1} \cdot S(u + \kappa)^t
                    \end{aligned}
                $$
            with the last equality holding because $z^{\tw}(u)$ is an even series. Through this, we infer that in order for question \ref{question: boundary_quantum_contraction_factorisation} to have an answer in the positive, we must require that:
                \begin{equation} \label{equation: applying_antipode_to_boundary_transfer_matrix}
                    \sigma( S(u) ) = b(u + \kappa)b(-u + \kappa) \cdot S(u + \kappa)^t
                \end{equation}
            At the same time, the following holds in $\Mat_N(\calX(\g_N)[\![u^{-1}]\!])$:
                $$\sigma(T(u)) = T(u)^{-1}$$
            so we also have:
                $$\sigma( S(u) ) = S(u)^{-1}$$
            and thus:
                $$b(u + \kappa)b(-u + \kappa) \cdot S(u + \kappa)^t = S(u)^{-1}$$
            This is equivalent to:
                $$( b(-u + \kappa) S(u + \kappa)^t )^{-1} = b(u + \kappa) S(u)$$

            \begin{proposition}[A homomorphism $\UB(\g_N, \id) \to \calX^{\tw}(\g_N, \id)$] \label{prop: mapping_IMO_reflection_algebras_to_extended_untwisted_yangians_via_extended_twisted_yangians}
                
            \end{proposition}
                \begin{proof}
                    
                \end{proof}

            Let us conclude the subsection by considering whether the composition:
                $$
                    \begin{tikzcd}
                    {\UB(\g_N, \id)} & {\calX^{\tw}(\g_N, \id)} & {\calY^{\tw}(\g_N, \id)}
                    \arrow["{\varphi^{\tw}}", dashed, from=1-1, to=1-2]
                    \arrow[two heads, from=1-2, to=1-3]
                    \end{tikzcd}
                $$
            wherein the unlabelled arrow is the canonical quotient map, is an isomorphism of algebras. 
    
    \addcontentsline{toc}{section}{References}
    \printbibliography

\end{document}