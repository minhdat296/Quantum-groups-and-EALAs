\input{book preambles}

\setcounter{chapter}{-1}
\setcounter{section}{-1}

\renewcommand{\cong}{\simeq}
\newcommand{\ladjoint}{\dashv}
\newcommand{\radjoint}{\vdash}
\newcommand{\<}{\langle}
\renewcommand{\>}{\rangle}
\newcommand{\ndiv}{\hspace{-2pt}\not|\hspace{5pt}}
\newcommand{\cond}{\blacktriangle}
\newcommand{\solid}{\blacksquare}
\newcommand{\ot}{\leftarrow}
\renewcommand{\-}{\text{-}}
\renewcommand{\mapsto}{\leadsto}
\renewcommand{\leq}{\leqslant}
\renewcommand{\geq}{\geqslant}
\renewcommand{\setminus}{\smallsetminus}
\makeatletter
\DeclareRobustCommand{\cev}[1]{%
  {\mathpalette\do@cev{#1}}%
}
\newcommand{\do@cev}[2]{%
  \vbox{\offinterlineskip
    \sbox\z@{$\m@th#1 x$}%
    \ialign{##\cr
      \hidewidth\reflectbox{$\m@th#1\vec{}\mkern4mu$}\hidewidth\cr
      \noalign{\kern-\ht\z@}
      $\m@th#1#2$\cr
    }%
  }%
}
\makeatother

\newcommand{\N}{\mathbb{N}}
\newcommand{\Z}{\mathbb{Z}}
\newcommand{\Q}{\mathbb{Q}}
\newcommand{\R}{\mathbb{R}}
\newcommand{\bbC}{\mathbb{C}}
\NewDocumentCommand{\x}{e{_^}}{%
  \mathbin{\mathop{\times}\displaylimits
    \IfValueT{#1}{_{#1}}
    \IfValueT{#2}{^{#2}}
  }%
}
\NewDocumentCommand{\pushout}{e{_^}}{%
  \mathbin{\mathop{\sqcup}\displaylimits
    \IfValueT{#1}{_{#1}}
    \IfValueT{#2}{^{#2}}
  }%
}
\newcommand{\supp}{\operatorname{supp}}
\newcommand{\im}{\operatorname{im}}
\newcommand{\coker}{\operatorname{coker}}
\newcommand{\id}{\mathrm{id}}
\newcommand{\chara}{\operatorname{char}}
\newcommand{\trdeg}{\operatorname{trdeg}}
\newcommand{\rank}{\operatorname{rank}}
\newcommand{\trace}{\operatorname{tr}}
\newcommand{\length}{\operatorname{length}}
\newcommand{\height}{\operatorname{height}}
\renewcommand{\span}{\operatorname{span}}
\newcommand{\e}{\epsilon}
\newcommand{\p}{\mathfrak{p}}
\newcommand{\q}{\mathfrak{q}}
\newcommand{\m}{\mathfrak{m}}
\newcommand{\n}{\mathfrak{n}}
\newcommand{\calF}{\mathcal{F}}
\newcommand{\calG}{\mathcal{G}}
\newcommand{\calO}{\mathcal{O}}
\newcommand{\F}{\mathbb{F}}
\DeclareMathOperator{\lcm}{lcm}
\newcommand{\gr}{\operatorname{gr}}
\newcommand{\vol}{\mathrm{vol}}
\newcommand{\ord}{\operatorname{ord}}

\newcommand{\GL}{\operatorname{GL}}
\newcommand{\SL}{\operatorname{SL}}
\newcommand{\Sp}{\operatorname{Sp}}
\newcommand{\GSp}{\operatorname{GSp}}
\newcommand{\GSpin}{\operatorname{GSpin}}
\newcommand{\opO}{\operatorname{O}}
\newcommand{\SO}{\operatorname{SO}}
\newcommand{\SU}{\operatorname{SU}}
\newcommand{\opU}{\operatorname{U}}
\newcommand{\Spec}{\mathrm{Spec}}
\newcommand{\Spf}{\mathrm{Spf}}
\newcommand{\Spm}{\mathrm{Spm}}
\newcommand{\Spv}{\mathrm{Spv}}
\newcommand{\Spa}{\mathrm{Spa}}
\newcommand{\Spd}{\mathrm{Spd}}
\newcommand{\Proj}{\mathrm{Proj}}
\newcommand{\Gr}{\mathrm{Gr}}
\newcommand{\Hecke}{\mathrm{Hecke}}
\newcommand{\Sht}{\mathrm{Sht}}
\newcommand{\Quot}{\mathrm{Quot}}
\newcommand{\Hilb}{\mathrm{Hilb}}
\newcommand{\Pic}{\mathrm{Pic}}
\newcommand{\Div}{\mathrm{Div}}
\newcommand{\Jac}{\mathrm{Jac}}
\newcommand{\Alb}{\mathrm{Alb}} %albanese variety
\newcommand{\Bun}{\mathrm{Bun}}
\newcommand{\loopspace}{\mathbf{\Omega}}
\newcommand{\suspension}{\mathbf{\Sigma}}
\newcommand{\tangent}{\mathrm{T}} %tangent space
\newcommand{\Eig}{\mathrm{Eig}}

\newcommand{\Ring}{\mathrm{Ring}}
\newcommand{\Cring}{\mathrm{CRing}}
\newcommand{\Alg}{\mathrm{Alg}}
\newcommand{\Leib}{\mathrm{Leib}} %leibniz algebras
\newcommand{\Fld}{\mathrm{Fld}}
\newcommand{\Sets}{\mathrm{Sets}}
\newcommand{\Cat}{\mathrm{Cat}}
\newcommand{\Grp}{\mathrm{Grp}}
\newcommand{\Ab}{\mathrm{Ab}}
\newcommand{\Sch}{\mathrm{Sch}}
\newcommand{\Coh}{\mathrm{Coh}}
\newcommand{\QCoh}{\mathrm{QCoh}}
\newcommand{\Desc}{\mathrm{Desc}}
\newcommand{\Sh}{\mathrm{Sh}}
\newcommand{\Psh}{\mathrm{PSh}}
\newcommand{\Fib}{\mathrm{Fib}}
\renewcommand{\mod}{\-\mathrm{mod}}
\newcommand{\comod}{\-\mathrm{comod}}
\newcommand{\bimod}{\-\mathrm{bimod}}
\newcommand{\Vect}{\mathrm{Vect}}
\newcommand{\Rep}{\mathrm{Rep}}
\newcommand{\Grpd}{\mathrm{Grpd}}
\newcommand{\Arr}{\mathrm{Arr}}
\newcommand{\Esp}{\mathrm{Esp}}
\newcommand{\Ob}{\mathrm{Ob}}
\newcommand{\Mor}{\mathrm{Mor}}
\newcommand{\Mfd}{\mathrm{Mfd}}
%\newcommand{\LR}{\mathrm{LR}}
%\newcommand{\RSpc}{\mathrm{RSpc}}
\newcommand{\Spc}{\mathrm{Spc}}
\newcommand{\Top}{\mathrm{Top}}
\newcommand{\Topos}{\mathrm{Topos}}
\newcommand{\Nil}{\mathfrak{Nil}}
\newcommand{\J}{\mathfrak{J}}
\newcommand{\Stk}{\mathrm{Stk}}
\newcommand{\Pre}{\mathrm{Pre}}
\newcommand{\simp}{\mathbf{\Delta}}
\newcommand{\Ind}{\mathrm{Ind}}
\newcommand{\Pro}{\mathrm{Pro}}
\newcommand{\Mon}{\mathrm{Mon}}
\newcommand{\Comm}{\mathrm{Comm}}
\newcommand{\Fin}{\mathrm{Fin}}
\newcommand{\Assoc}{\mathrm{Assoc}}
\newcommand{\Co}{\mathrm{Co}}
\newcommand{\Loc}{\mathrm{Loc}}
\newcommand{\Ringed}{\mathrm{Ringed}}
\newcommand{\Comp}{\mathrm{Comp}} %compact hausdorff spaces
\newcommand{\Stone}{\mathrm{Stone}} %stone spaces
\newcommand{\sfExt}{\mathrm{Ext}} %extremely disconnected spaces
\newcommand{\Ouv}{\mathrm{Ouv}}
\newcommand{\Str}{\mathrm{Str}}
\newcommand{\Func}{\mathrm{Func}}
\newcommand{\Crys}{\mathrm{Crys}}
\newcommand{\LocSys}{\mathrm{LocSys}}
\newcommand{\Sieves}{\mathrm{Sieves}}
\newcommand{\pt}{\mathrm{pt}}
\newcommand{\Graphs}{\mathrm{Graphs}}
\newcommand{\Lie}{\mathrm{Lie}}
\newcommand{\Env}{\mathrm{Env}}
\newcommand{\Ho}{\mathrm{Ho}}
\newcommand{\rmD}{\mathrm{D}}
\newcommand{\Cov}{\mathrm{Cov}}
\newcommand{\Frames}{\mathrm{Frames}}
\newcommand{\Locales}{\mathrm{Locales}}
\newcommand{\Span}{\mathrm{Span}}
\newcommand{\Corr}{\mathrm{Corr}}
\newcommand{\Monad}{\mathrm{Monad}}
\newcommand{\Var}{\mathrm{Var}}
\newcommand{\sfN}{\mathrm{N}} %nerve
\newcommand{\Dia}{\mathrm{Dia}}
\newcommand{\co}{\mathrm{co}}
\newcommand{\ev}{\mathrm{ev}}
\newcommand{\bi}{\mathrm{bi}}
\newcommand{\Nat}{\mathrm{Nat}}
\newcommand{\Hopf}{\mathrm{Hopf}}
\newcommand{\Dmod}{\mathrm{D}\mod}
\newcommand{\Perv}{\mathrm{Perv}}
\newcommand{\Sph}{\mathrm{Sph}}
\newcommand{\Moduli}{\mathrm{Moduli}}
\newcommand{\Pseudo}{\mathrm{Pseudo}}
\newcommand{\Lax}{\mathrm{Lax}}
\newcommand{\Strict}{\mathrm{Strict}}
\newcommand{\Opd}{\mathrm{Opd}} %operads
\newcommand{\Shv}{\mathrm{Shv}}
\newcommand{\Char}{\mathrm{Char}} %CharShv = character sheaves
\newcommand{\Huber}{\mathrm{Huber}}
\newcommand{\Tate}{\mathrm{Tate}}
\newcommand{\Ad}{\mathrm{Ad}} %adic spaces
\newcommand{\Perfd}{\mathrm{Perfd}} %perfectoid spaces
\newcommand{\Sub}{\mathrm{Sub}} %subobjects
\newcommand{\Ideals}{\mathrm{Ideals}}
\newcommand{\Isoc}{\mathrm{Isoc}}
\newcommand{\Ban}{\-\mathrm{Ban}} %Banach spaces
\newcommand{\Fre}{\-\mathrm{Fre}} %Frechet spaces
\newcommand{\Ch}{\mathrm{Ch}} %chain complexes
\newcommand{\Mot}{\mathrm{Mot}} %motives
\newcommand{\KL}{\mathrm{KL}} %category of Kazhdan-Lusztig modules
\newcommand{\Pres}{\mathrm{Pres}} %presentable categories
\newcommand{\Noohi}{\mathrm{Noohi}} %category of Noohi groups
\newcommand{\Inf}{\mathrm{Inf}}

\newcommand{\Aut}{\mathrm{Aut}}
\newcommand{\Inn}{\mathrm{Inn}}
\newcommand{\Out}{\mathrm{Out}}
\newcommand{\frakgl}{\mathfrak{gl}}
\newcommand{\der}{\mathfrak{der}} %derivations on Lie algebras
\newcommand{\inn}{\mathfrak{inn}} %inner derivations
\newcommand{\out}{\mathfrak{out}} %outer derivations
\newcommand{\Stab}{\mathrm{Stab}}
\newcommand{\Cent}{\mathrm{Cent}}
\newcommand{\Norm}{\mathrm{Norm}}
\newcommand{\Rad}{\mathrm{Rad}}
\newcommand{\Transporter}{\mathrm{Transp}} %transporter between two subsets of a group
\newcommand{\Conj}{\mathrm{Conj}}
\newcommand{\Diag}{\mathrm{Diag}}
\newcommand{\Gal}{\mathrm{Gal}}
\newcommand{\bfG}{\mathbf{G}} %absolute Galois group
\newcommand{\Frac}{\mathrm{Frac}}
\newcommand{\Ann}{\mathrm{Ann}}
\newcommand{\Val}{\mathrm{Val}}
\newcommand{\Chow}{\mathrm{Chow}}
\newcommand{\Sym}{\mathrm{Sym}}
\newcommand{\End}{\mathrm{End}}
\newcommand{\Mat}{\mathrm{Mat}}
\newcommand{\Diff}{\mathrm{Diff}}
\newcommand{\Autom}{\mathrm{Autom}}
\newcommand{\Artin}{\mathrm{Artin}} %artin maps
\newcommand{\sk}{\mathrm{sk}} %skeleton of a category
\newcommand{\eqv}{\mathrm{eqv}} %functor that maps groups $G$ to $G$-sets
\newcommand{\Inert}{\mathrm{Inert}}
\newcommand{\Fil}{\mathrm{Fil}}

\newcommand{\colim}{\operatorname{colim} \:}
\renewcommand{\lim}{\operatorname{lim} \:}
\newcommand{\toto}{\rightrightarrows}
%\newcommand{\tensor}{\otimes}
\NewDocumentCommand{\tensor}{e{_^}}{%
  \mathbin{\mathop{\otimes}\displaylimits
    \IfValueT{#1}{_{#1}}
    \IfValueT{#2}{^{#2}}
  }%
}
\newcommand{\eq}{\operatorname{eq}}
\newcommand{\coeq}{\operatorname{coeq}}
\newcommand{\Hom}{\mathrm{Hom}}
\newcommand{\Maps}{\mathrm{Maps}}
\newcommand{\Tor}{\mathrm{Tor}}
\newcommand{\Ext}{\mathrm{Ext}}
\newcommand{\Isom}{\mathrm{Isom}}
\newcommand{\stalk}{\mathrm{stalk}}
\newcommand{\RKE}{\operatorname{RKE}}
\newcommand{\LKE}{\operatorname{LKE}}
\newcommand{\oblv}{\mathrm{oblv}}
\newcommand{\const}{\mathrm{const}}
%\newcommand{\forget}{\mathrm{forget}}
\newcommand{\adrep}{\mathrm{ad}} %adjoint representation
\newcommand{\NL}{\mathbb{NL}} %naive cotangent complex
\newcommand{\pr}{\operatorname{pr}}
\newcommand{\Der}{\mathrm{Der}}
\newcommand{\Frob}{\mathrm{Frob}} %Frobenius
\newcommand{\frob}{\mathrm{f}} %trace of Frobenius
\newcommand{\bfpt}{\mathbf{pt}}
\newcommand{\bfloc}{\mathbf{loc}}
\DeclareMathAlphabet{\mymathbb}{U}{BOONDOX-ds}{m}{n}
\newcommand{\0}{\mymathbb{0}}
\newcommand{\1}{\mathbbm{1}}
\newcommand{\2}{\mathbbm{2}}
\newcommand{\Jet}{\mathrm{Jet}}
\newcommand{\Split}{\mathrm{Split}}
\newcommand{\Sq}{\mathrm{Sq}}
\newcommand{\Zero}{\mathrm{Z}}
\newcommand{\SqZ}{\Sq\Zero}
\newcommand{\frakLie}{\mathfrak{Lie}}
\newcommand{\y}{\mathrm{y}} %yoneda
\newcommand{\Sm}{\mathrm{Sm}}
\newcommand{\AJ}{\phi} %abel-jacobi map
\newcommand{\act}{\mathrm{act}}
\newcommand{\ram}{\mathrm{ram}} %ramification index
\newcommand{\inv}{\mathrm{inv}}

\newcommand{\bbU}{\mathbb{U}}
\newcommand{\V}{\mathbb{V}}
\newcommand{\U}{\mathrm{U}}
\newcommand{\calU}{\mathcal{U}}
\newcommand{\calW}{\mathcal{W}}
\newcommand{\rmI}{\mathrm{I}} %augmentation ideal
\newcommand{\bfV}{\mathbf{V}}
\newcommand{\C}{\mathcal{C}}
\newcommand{\D}{\mathcal{D}}
\newcommand{\T}{\mathscr{T}} %Tate modules
\newcommand{\calM}{\mathcal{M}}
\newcommand{\calN}{\mathcal{N}}
\newcommand{\calP}{\mathcal{P}}
\newcommand{\calQ}{\mathcal{Q}}
\newcommand{\A}{\mathbb{A}}
\renewcommand{\P}{\mathbb{P}}
\newcommand{\calL}{\mathcal{L}}
\newcommand{\E}{\mathcal{E}}
\renewcommand{\H}{\mathbf{H}}
\newcommand{\scrS}{\mathscr{S}}
\newcommand{\calX}{\mathcal{X}}
\newcommand{\calY}{\mathcal{Y}}
\newcommand{\calZ}{\mathcal{Z}}
\newcommand{\calS}{\mathcal{S}}
\newcommand{\calR}{\mathcal{R}}
\newcommand{\scrX}{\mathscr{X}}
\newcommand{\scrY}{\mathscr{Y}}
\newcommand{\scrZ}{\mathscr{Z}}
\newcommand{\calA}{\mathcal{A}}
\newcommand{\calB}{\mathcal{B}}
\newcommand{\sfT}{\mathrm{T}}
\renewcommand{\S}{\mathcal{S}}
\newcommand{\B}{\mathbb{B}}
\newcommand{\bbD}{\mathbb{D}}
\newcommand{\G}{\mathbb{G}}
\newcommand{\horn}{\mathbf{\Lambda}}
\renewcommand{\L}{\mathbb{L}}
\renewcommand{\a}{\mathfrak{a}}
\renewcommand{\b}{\mathfrak{b}}
\renewcommand{\t}{\mathfrak{t}}
\renewcommand{\r}{\mathfrak{r}}
\newcommand{\bbX}{\mathbb{X}}
\newcommand{\g}{\mathfrak{g}}
\newcommand{\h}{\mathfrak{h}}
\renewcommand{\k}{\mathfrak{k}}
\newcommand{\del}{\partial}
\newcommand{\bbE}{\mathbb{E}}
\newcommand{\scrO}{\mathscr{O}}
\newcommand{\bbO}{\mathbb{O}}
\newcommand{\scrA}{\mathscr{A}}
\newcommand{\scrB}{\mathscr{B}}
\newcommand{\scrF}{\mathscr{F}}
\newcommand{\scrG}{\mathscr{G}}
\newcommand{\scrM}{\mathscr{M}}
\newcommand{\scrN}{\mathscr{N}}
\newcommand{\scrP}{\mathscr{P}}
\newcommand{\frakS}{\mathfrak{S}}
\newcommand{\calI}{\mathcal{I}}
\newcommand{\calJ}{\mathcal{J}}
\newcommand{\scrK}{\mathscr{K}}
\newcommand{\calK}{\mathcal{K}}
\newcommand{\scrV}{\mathscr{V}}
\newcommand{\bbS}{\mathbb{S}}
\newcommand{\scrH}{\mathscr{H}}
\newcommand{\bfB}{\mathbf{B}}
\newcommand{\Witt}{W}
%\newcommand{\bfA}{\mathbf{A}}
\renewcommand{\O}{\mathbb{O}}
\newcommand{\calV}{\mathcal{V}}
\newcommand{\scrR}{\mathscr{R}} %radical
\newcommand{\rmZ}{\mathrm{Z}} %centre of algebra
\newcommand{\bfGamma}{\mathbf{\Gamma}}
\newcommand{\scrU}{\mathscr{U}}
\newcommand{\rmW}{\mathrm{W}} %Weil group
\newcommand{\frakM}{\mathfrak{M}}
\newcommand{\frakN}{\mathfrak{N}}
\newcommand{\frakX}{\mathfrak{X}}
\newcommand{\frakY}{\mathfrak{Y}}
\newcommand{\frakZ}{\mathfrak{Z}}

\newcommand{\aff}{\mathrm{aff}}
\newcommand{\ft}{\mathrm{ft}} %finite type
\newcommand{\fp}{\mathrm{fp}} %finite presentation
\newcommand{\aft}{\mathrm{aft}}
\newcommand{\lft}{\mathrm{lft}}
\newcommand{\laft}{\mathrm{laft}}
\newcommand{\cmpt}{\mathrm{cmpt}}
\newcommand{\qc}{\mathrm{qc}}
\newcommand{\qs}{\mathrm{qs}}
\newcommand{\lcmpt}{\mathrm{lcmpt}}
%\newcommand{\conv}{\mathrm{conv}}
\newcommand{\red}{\mathrm{red}}
\newcommand{\fin}{\mathrm{fin}}
\newcommand{\gen}{\mathrm{gen}}
\newcommand{\petit}{\mathrm{petit}}
\newcommand{\gros}{\mathrm{gros}}
\newcommand{\loc}{\mathrm{loc}}
\newcommand{\glob}{\mathrm{glob}}
%\newcommand{\ringed}{\mathrm{ringed}}
\newcommand{\qcoh}{\mathrm{qcoh}}
\newcommand{\cl}{\mathrm{cl}}
\newcommand{\et}{\mathrm{\acute{e}t}}
\newcommand{\fet}{\mathrm{f\acute{e}t}}
\newcommand{\profet}{\mathrm{prof\acute{e}t}}
\newcommand{\proet}{\mathrm{pro\acute{e}t}}
\newcommand{\Zar}{\mathrm{Zar}}
\newcommand{\fppf}{\mathrm{fppf}}
\newcommand{\fpqc}{\mathrm{fpqc}}
\newcommand{\smooth}{\mathrm{sm}}
\newcommand{\sh}{\mathrm{sh}}
\newcommand{\op}{\mathrm{op}}
\newcommand{\open}{\mathrm{open}}
\newcommand{\closed}{\mathrm{closed}}
\newcommand{\geom}{\mathrm{geom}}
\newcommand{\alg}{\mathrm{alg}}
\newcommand{\sober}{\mathrm{sober}}
\newcommand{\dR}{\mathrm{dR}}
\newcommand{\rad}{\mathrm{rad}}
\newcommand{\discrete}{\mathrm{discrete}}
%\newcommand{\add}{\mathrm{add}}
%\newcommand{\lin}{\mathrm{lin}}
\newcommand{\Krull}{\mathrm{Krull}}
\newcommand{\qis}{\mathrm{qis}} %quasi-isomorphism
\newcommand{\ho}{\mathrm{ho}} %homotopy equivalence
\newcommand{\sep}{\mathrm{sep}}
\newcommand{\unr}{\mathrm{unr}}
\newcommand{\tame}{\mathrm{tame}}
\newcommand{\wild}{\mathrm{wild}}
\newcommand{\nil}{\mathrm{nil}}
\newcommand{\defm}{\mathrm{defm}}
\newcommand{\Art}{\mathrm{Art}}
\newcommand{\Noeth}{\mathrm{Noeth}}
\newcommand{\affd}{\mathrm{affd}}
%\newcommand{\adic}{\mathrm{adic}}
\newcommand{\pre}{\mathrm{pre}}
\newcommand{\perf}{\mathrm{perf}}
\newcommand{\perfd}{\mathrm{perfd}}
\newcommand{\rat}{\mathrm{rat}}
\newcommand{\cont}{\mathrm{cont}}
\newcommand{\dg}{\mathrm{dg}}
\newcommand{\almost}{\mathrm{a}}
%\newcommand{\stab}{\mathrm{stab}}
\newcommand{\heart}{\heartsuit}
\newcommand{\proj}{\mathrm{proj}}
\newcommand{\qproj}{\mathrm{qproj}}
\newcommand{\pd}{\mathrm{pd}}
\newcommand{\crys}{\mathrm{crys}}
\newcommand{\prisma}{\mathrm{prisma}}
\newcommand{\FF}{\mathrm{FF}}
\newcommand{\sph}{\mathrm{sph}}
\newcommand{\lax}{\mathrm{lax}}
\newcommand{\weak}{\mathrm{weak}}
\newcommand{\strict}{\mathrm{strict}}
\newcommand{\mon}{\mathrm{mon}}
\newcommand{\sym}{\mathrm{sym}}
\newcommand{\lisse}{\mathrm{lisse}}
\newcommand{\an}{\mathrm{an}}
\newcommand{\ad}{\mathrm{ad}}
\newcommand{\sch}{\mathrm{sch}}
\newcommand{\rig}{\mathrm{rig}}
\newcommand{\pol}{\mathrm{pol}}
\newcommand{\plat}{\mathrm{flat}}
\newcommand{\proper}{\mathrm{proper}}
\newcommand{\compl}{\mathrm{compl}}
\newcommand{\non}{\mathrm{non}}
\newcommand{\access}{\mathrm{access}}
\newcommand{\comp}{\mathrm{comp}}
\newcommand{\tstructure}{\mathrm{t}} %t-structures
\newcommand{\pure}{\mathrm{pure}} %pure motives
\newcommand{\mixed}{\mathrm{mixed}} %mixed motives
\newcommand{\num}{\mathrm{num}} %numerical motives
\newcommand{\ess}{\mathrm{ess}}
\newcommand{\topological}{\mathrm{top}}
\newcommand{\convex}{\mathrm{cv}}
\newcommand{\ab}{\mathrm{ab}} %abelian extensions
\newcommand{\surj}{\mathrm{surj}} %coverage on sets generated by surjections
\newcommand{\eff}{\mathrm{eff}} %effective Cartier divisors
\newcommand{\Weil}{\mathrm{Weil}} %weil divisors
\newcommand{\lex}{\mathrm{lex}}
\newcommand{\rex}{\mathrm{rex}}
\newcommand{\AR}{\mathrm{A\-R}}
\newcommand{\cons}{\mathrm{c}} %constructible sheaves
\newcommand{\tor}{\mathrm{tor}} %tor dimension
\newcommand{\semisimple}{\mathrm{ss}}

%prism custom command
\usepackage{relsize}
\usepackage[bbgreekl]{mathbbol}
\usepackage{amsfonts}
\DeclareSymbolFontAlphabet{\mathbb}{AMSb} %to ensure that the meaning of \mathbb does not change
\DeclareSymbolFontAlphabet{\mathbbl}{bbold}
\newcommand{\prism}{{\mathlarger{\mathbbl{\Delta}}}}

\begin{document}

	\title{Quiver varieties and quantum groups}
	
	\author{Dat Minh Ha}
	\maketitle
	
	\begin{abstract}
	    
	\end{abstract}
	
	{
      \hypersetup{} 
      \dominitoc
      \tableofcontents %sort sections alphabetically
    }
    
    \section{Introduction}
    \subsection{Notations and conventions}
        All throughout, we work over the field of complex numbers $\bbC$, although any algebraically closed field of characteristic $0$ will suffice, as we only need this assumption so that certain operators will be diagonalisable. 

        We fix once and for all a finite-dimensional simple Lie algebra $\g$. Its Cartan matrix will be denoted by $C$, and we will use standard notations (cf. e.g. \cite{humphreys_lie_algebras} and \cite{kac_infinite_dimensional_lie_algebras}) for all of the other data that usually accompanies $\g$, e.g. a choice of simple roots, a subsequently defined root system, etc. We note that to specify these data, we will have to choose a non-degenerate and invariant symmetric bilinear form on $\g$, which shall be denoted by $\kappa$.

        The \textbf{affine Lie algebra} attached to the Lie $2$-cocycle $\kappa \in H^2_{\Lie}(\g[v^{\pm 1}], \bbC)$ shall be:
            $$\tilde{\g}_{\kappa} := \uce(\g[v^{\pm 1}])$$
        on which the Lie bracket is given by:
            $$[xf, yg]_{\tilde{\g}_{\kappa}} := [x, y]_{\g} fg + \kappa(x, y) g df$$
        Here, we have used a result by Kassel to identify:
            $$\tilde{\g}_{\kappa} \cong \g[v^{\pm 1}] \oplus \Omega^1_{\bbC[v^{\pm 1}]/\bbC}/d\bbC[v^{\pm 1}] \cong \g[v^{\pm 1}] \oplus \bbC c_{\kappa}$$
        (see \cite{kassel_universal_central_extensions_of_lie_algebras}). We also know, per a result of Garland (see \cite{garland_arithmetics_of_loop_groups}), that $\tilde{\g}_{\kappa}$ is - up to isomorphisms - the only central extension of $\g[v^{\pm 1}]$ (i.e. $H^2_{\Lie}(\g[v^{\pm 1}], \bbC)$); isomorphisms are given by rescaling $\kappa$. One also says that $\tilde{\g}_{\kappa}$ is the affine Lie algebra at \textbf{level} $\kappa$.

        The \textbf{untwisted affine Kac-Moody algebra} at level $\kappa$ (as in \cite[Chapter 7]{kac_infinite_dimensional_lie_algebras}) will then be:
            $$\hat{\g}_{\kappa} := \tilde{\g}_{\kappa} \rtimes \bbC D_{\kappa}$$
        where $D_{\kappa} \in \der(\tilde{\g}_{\kappa})$ can be shown to be equal to $v\frac{d}{dv}$. This algebra will be endowed with the non-degenerate and invariant symmetric bilinear form given by:
            $$(xf, yg)_{\hat{\g}_{\kappa}} := \kappa(x, y) \Res(g df)$$
            $$(c_{\kappa}, D_{\kappa})_{\hat{\g}_{\kappa}} := 1$$
        Using this bilinear form, one can construct the Cartan matrix $\hat{C}$ of $\hat{\g}_{\kappa}$ and related combinatorial data, like the affinisation of the root system of $\g$ and so on.
            
        Recall in particular that for $\hat{\g}$ (or indeed, for any Kac-Moody algebra) there is the \textbf{principal $\Z$-grading} of type $\vec{s} := (s_i)_{i \in \hat{\simpleroots}} \in \Z_{\geq 0} \cdot \hat{\simpleroots}$, given on the Chevalley-Serre generators by:
            $$\deg x_i^{\pm} := \pm s_i$$
            $$\deg h_i := 0$$
        for all $i \in \hat{\simpleroots}$. When $\vec{s} := \vec{1} := (1, ..., 1)$, we recover the usual root height grading. 

    \subsection{Vertex algebras}

    \subsection{Why vertex operators ?}
        From a representation-theoretic point of view, one possible motivation for the use of vertex algebras comes from simply trying to write down some kind of \say{affine Casimir element}. There are many reasons to care about such elements, though one prominent reason is that they generate the centre of some completion of $\rmU(\hat{\g}_{\kappa})$ (what we will denote by $\tilde{\rmU}_{\kappa}$ down below), and hence are useful understanding affine central characters. 
        
        Firstly, recall that for the affine Kac-Moody algebra $\hat{\g}_{\kappa}$, one shall obtain a root space decomposition of the following form, after choosing a Cartan subalgebra $\hat{\h}_{\kappa}$:
            $$\hat{\g}_{\kappa} \cong \hat{\h}_{\kappa} \oplus \bigoplus_{\alpha \in \hat{\Phi}} \hat{\g}_{\kappa, \alpha}$$
        in which:
        \begin{itemize}
            \item the real roots - of which there are finitely many - are all equally of multiplicity $1$, while
            \item the imaginary roots - of which there are infinitely many, as their root spaces are identified with $\h \tensor v^n$ for all $n \in \Z \setminus \{0\}$ - do not necessarily have multiplicity $1$ (indeed, when $\g \not \cong \sl_2$, $\dim \h > 1$).
        \end{itemize}
        Because of this, if we were to insist on writing down the \textit{na\"ive} guess for an affine (quadratic) Casimir element, i.e. something of the form:
            $$\hat{\sfr}_{\text{na\"ive}} := \sum_{(i, n) \in \simpleroots \x \Z} h_i v^n \tensor h^i v^{-n} + \sum_{\alpha \in \Re \hat{\Phi}^+} (x_{\alpha}^+ \tensor x_{\alpha}^- + x_{\alpha}^- \tensor x_{\alpha}^+)$$
        then we would have a hard time trying to make sense of the summand:
            $$\hat{\sfr}_{\text{na\"ive}}^{\Im} := \sum_{(i, n) \in \simpleroots \x (\Z \setminus \{0\})} h_i v^n \tensor h^i v^{-n}$$
        which is a sum over basis-dual basis vectors of the direct sum $\bigoplus_{n \in \Z \setminus \{0\}} \h \tensor v^n$ of all the imaginary root spaces. Worse still, this is a \say{bad} infinite sum, in the sense that the infinite in both the positive and negative directions (recall that formal Laurent series contain infinitely many terms only in either direction, not both). In other words, the expression $\hat{\sfr}_{\text{na\"ive}}^{\Im}$ is properly an element of $\hat{\g}_{\kappa}[\![v^{\pm 1}]\!]$, and this very problematic because $\bbC[\![v^{\pm 1}]\!]$ - the vector space spanned by so-called \say{formal distributions} $\sum_{n \in \Z} a_n v^n$ where, crucially, there can be infinitely many $a_n \not = 0$ - is \textit{not} an associative $\bbC$-algebra at all, so we can not regard $\hat{\g}_{\kappa}[\![v^{\pm 1}]\!]$ as a current algebra!

        If, instead, we were to regard the elements $h_i, h^i$ as operators on some $\hat{\g}_{\kappa}$-module, say $V$, then we can consider:
            $$\hat{\sfr}_{V, \text{na\"ive}}^{\Im}(z) := \sum_{(i, n) \in \simpleroots \x (\Z \setminus \{0\})} h_i z^n \tensor h^i z^{-n} \in \End(V)[\![z^{\pm 1}]\!]$$
        Even though this is still a formal distribution (this time with coefficients in the associative algebra $\End(V)$), it is known that $\End(V)[\![z^{\pm 1}]\!]$ can be used for constructing a vertex algebra structure on the underlying vector space of $V$, and hence expressions like $\hat{\sfr}_V^{\Im}(z)$ can be dealt with as vertex operators.

    \subsection{Generating series}
        To have a better sense of how to proceed from here, let us firstly introduce the physicist's perspective on infinite-dimensional Lie algebras with loop realisations, namely the apprroach via \textbf{generating series}. For untwisted affine Kac-Moody algebras, this ultimately originates from the fact that the extra \say{affinising} Chevalley-Serre generators:
            $$x_0^{\pm} := x_{\theta}^{\pm} v^{\mp 1}, h_0 := [x_{\theta}^+, x_{\theta}^-]$$
        are given in terms of the finite-type generators (elements of $\{x_i^{\pm}, h_i\}_{i \in \simpleroots}$); namely, the highest/lowest root vectors $x_{\theta}^{\pm}$ can be written as ordered monomials in $x_i^{\pm}$. The physical perspective is then, that $x_0^{\pm}, h_0$ ought be obtained as certain \say{residues} of formal distributions:
            $$J(z) := \sum_{n \in \Z} J^{(n)} z^{-n - 1}, J^{(n)} := J v^n$$
        associated to elements $J \in \g$ (and why we have decided to write $z^{-n - 1}$ instead of $z^n$ will become clear shortly); these formal distributions are usually called \textbf{associated fields} or \textbf{generating series} when $J$ is a Chevalley-Serre generator of $\g$.

        Now, to see how commutator relations are given, let us firstly fix a basis $\{J_{\alpha}\}_{\alpha = 1}^{\dim \g}$ for $\g$ and then consider the following:
            $$[J_{\alpha}^{(m)}, J_{\beta}^{(n)}] = \sum_{r = 1}^{\dim \g} C_{\alpha, \beta}^{\gamma} J_{\gamma}^{(m + n)} + m \delta_{m + n, 0} \kappa(J_{\alpha}, J_{\beta}) c_{\kappa}$$
        wherein $C_{\alpha, \beta}^{\gamma} \in \bbC$ are the structural constants. Next, let us recall that the formal Dirac distribution in a formal variable $w$ centered at $z$ is the element of $\bbC[\![w^{\pm 1}, z^{\pm 1}]\!]$ that is given by:
            $$\1(w - z) := \sum_{n \in \Z} w^n z^{-n - 1}$$
        and crucially, this formal distribution satisfies the following identity for all $J(w) \in \g[\![w^{\pm 1}]\!]$:
            $$f(z) = \Res_{w = 0} f(w) \1(w - z)$$
        We can then perform the following computations, which allows us to pass to associated fields:
            $$
                \begin{aligned}
                    & [J_{\alpha}(w), J_{\beta}(z)]
                    \\
                    := & \sum_{m \in \Z} \sum_{n \in \Z} [J_{\alpha}^{(m)} w^{-m - 1}, J_{\beta}^{(n)} z^{-n - 1}]
                    \\
                    = & \sum_{m \in \Z} \sum_{n \in \Z} \sum_{\gamma = 1}^{\dim \g} C_{\alpha, \beta}^{\gamma} J_{\gamma}^{(m + n)} w^{-m - 1} z^{-n - 1} + \kappa(J_{\alpha}, J_{\beta}) c_{\kappa} \sum_{m \in \Z} \sum_{n \in \Z} m \delta_{m + n, 0} w^{-m - 1} z^{-n - 1}
                    \\
                    = & \sum_{m \in \Z} \sum_{n \in \Z} \sum_{\gamma = 1}^{\dim \g} C_{\alpha, \beta}^{\gamma} J_{\gamma}^{(m + n)} w^{-m - n - 1} w^n z^{-n - 1} + \kappa(J_{\alpha}, J_{\beta}) c_{\kappa} \sum_{m \in \Z} m w^{-m - 1} z^{m - 1}
                    \\
                    = & \sum_{\gamma = 1}^{\dim \g} C_{\alpha, \beta}^{\gamma} J_{\gamma}(w) \1(w - z) + \kappa(J_{\alpha}, J_{\beta}) c_{\kappa} \del_z \sum_{m \in \Z} w^{-m - 1} z^m
                    \\
                    = & [J_{\alpha}, J_{\beta}](w) \1(w - z) + \kappa(J_{\alpha}, J_{\beta}) c_{\kappa} \del_z \1(z - w)
                \end{aligned}
            $$
        where $\del_z \in \End( \bbC[\![z^{\pm 1}]\!] )$ is the formal partial derivative, given by $\del_z \sum_{m \in \Z} a_m z^m := \sum_{m \in \Z} m a_m z^{m - 1}$. By applying $\Res_{w = 0}$ to both sides, we then get:
            $$
                \begin{aligned}
                    & \Res_{w = 0} [J_{\alpha}(w), J_{\beta}(z)]
                    \\
                    = & \Res_{w = 0} [J_{\alpha}, J_{\beta}](w) \1(w - z) + \kappa(J_{\alpha}, J_{\beta}) c_{\kappa} \Res_{w = 0} \del_z \1(z - w)
                    \\
                    = & [J_{\alpha}, J_{\beta}](z) + \kappa(J_{\alpha}, J_{\beta}) c_{\kappa} 
                \end{aligned}
            $$
        One can then choose the basis vectors $J_{\alpha}$ to be orthonormal to one another, and then evaluate at $z = 0$ to get $c_{\kappa}$. 

        Now, observe in particular, that for all $x, y \in \g$, the commutators of fields:
            $$[x(w), y(z)] = [x, y](w) \1(w, z) + \kappa(x, y) \del_w \1(w - z)$$
        are all \textbf{mutually local} as the formal Dirac distribution satisfies:
            $$\forall N \in \Z_{\geq 0}: N > n \implies (w - z)^N \del_w^n \1(w - z) = 0$$
        This suggests to us that vertex algebra structures on $\tilde{\g}_{\kappa}$-modules $V$ - say, specified by a Lie algebra homomorphism $\rho: \tilde{\g}_{\kappa} \to \End(V)$ - can be specified by linear maps:
            $$\rho_z: \g[\![z^{\pm 1}]\!] \to \End(V)[\![z^{\pm 1}]\!]$$
        preserving field commutators, as the mutual locality between fields that is required to happen on the $\End(V)[\![z^{\pm 1}]\!]$ side as a part of the vertex algebra definition is already satisfied on the $\tilde{\g}_{\kappa}[\![z^{\pm 1}]\!]$ side. There is also the so-called \textbf{Vacuum Axiom}, which for now can be simply understood as requiring that $V$ is a highest-weight module on which the central charge $c_{\kappa}$ acts as $1$ (this is more-or-less a normalising condition); we will be offering a physical explanation later on, when discussing the so-called \say{Heisenberg vacuum module}. Finally, the \textbf{Translation Axiom}, which requires that there is a formal derivation $T$ acting on fields:
            $$J(z) := \sum_{m \in \Z} J^{(m)} z^{-m - 1} \in \End(V)[\![z^{\pm 1}]\!]$$
        via commutators, i.e.:
            $$[T, J(z)] := [T, J^{(m)}](z)$$
        where on the RHS, we regard $T$ as a derivation on $\g[v^{\pm 1}]$ or on $\g(\!(v)\!)$, depending on the situation at hand; typically, one takes $T := \id_{\g} \tensor D$ for some derivation $D$ on $\bbC[v^{\pm 1}]$ or $\bbC(\!(v)\!)$. In combination, these requirements suggest to us that a good candidates for a $\tilde{\g}_{\kappa}$-module that can support vertex algebra structures is the induced module:
            $$\V_{\kappa} := \tilde{\rmU}_{\kappa} \tensor_{\tilde{\rmU}_{\kappa}^+} \bbC c_{\kappa}$$
        Down below, we shall see that, indeed, this module does carry a unique vertex algebra structure.
    
    \part{Representations of quivers and quiver varieties}
        \chapter{Associative algebras of low homological dimensions}
            \begin{abstract}
                In this chapter, we study representations of associative algebras of low homological dimensions. In particular, we are interested in the cases of dimension $0$ (i.e. semi-simple algebras) and dimensions $\leq 1$ (i.e. hereditary algebras, such as path algebras of quivers) and would like to understand what these prescribed conditions can lead to as far as the categories of representations of these algebras are concerned, e.g. whether or not there might be finitely many isomorphism classes of simple or indecomposable objects, or if the category might be semi-simple. 
            \end{abstract}
            
            \minitoc
            
            \section{Semi-simple algebras and homological dimension \texorpdfstring{$0$}{}}
    \subsection{(Semi-)simplicity and (in)decomposability}
        In this subsection, we recall and attempt to re-conceptualise the notions of (semi-)simplicity and (in)decomposability. Our approach is categorical, as our goal is to establish the fact that certain classical results on (semi-)simple and (in)decomposable modules/representations are actually just special cases of some basic categorical patterns. 
    
        \subsubsection{(Semi)-simplicity and the Jordan-H\"older Theorem}
            \begin{definition}[(Semi-)simple objects] \label{def: (semi)_simple_objects}
                Let $\C$ be a category with enough monomorphisms and epimorphisms and zero objects. A \textit{non-zero} object $X \in \Ob(\C)$ is said to be \textbf{simple} if and only if any monomorphism $\iota: X' \to X$ and any epimorphism $\pi: X \to X''$ is either zero (i.e. it factors through a zero object) or the identity $\id_X$. If $\C$ additionally has enough products, and if a (non-zero) object $X$ admits a product decomposition:
                    $$X \cong \prod_{i \in I} X_i$$
                into simple sub-objects $X_i \subseteq X$, then $X$ will be said to be \textbf{semi-simple}; a category wherein all objects are semi-simple is itself called \textbf{semi-simple}.
            \end{definition}
            \begin{example}[Irreducible representations]
                (Semi-)simplicity is not a relative notion, i.e. it does not behave well under base-change. For instance, if $A$ is an associative algebra over a field $k$, then there can be irreducible $k$-linear representations $V$ which, while being finitely generated as an $A$-module, is of infinite dimension as a $k$-vector space at the same time.
            \end{example}
            \begin{remark}[(Semi-)simple objects in abelian categories]
                If $\calA$ is an abelian category, then because $f: M \to N$ is a kernel (respectively, a cokernel) if and only if it is a monomorphism (respectively, an epimorphism), one might characterise simple objects of $\calA$ as those such that the kernel of any morphism:
                    $$\iota: M' \to M$$
                and and the cokernel of any morphism:
                    $$\pi: M \to M''$$
                are identically zero. 
            \end{remark}
            \begin{example}[Modules over semi-simple rings] \label{example: modules_over_semi_simple_rings}
                \cite[Theorem 2.5]{lam_first_course_in_noncommutative_rings} The following are equivalent are equivalent for any associative ring $R$:
                    \noindent
                    \begin{enumerate}
                        \item $R$ is left/right-semi-simple (as a left/right-module over itself).
                        \item All shorts exact sequences of left/right-$R$-modules split.
                        \item All left/right-$R$-modules are semi-simple.
                        \item All finitely generated left/right-$R$-modules are semi-simple\footnote{In fact, the category ${}^lR\mod^{\fin}$ (respectively ${}^rR\mod^{\fin}$) of finitely generated left/right-$R$-modules is - by definition (cf. definition \ref{def: finite_linear_categories}) - an instance of a finite (abelian) $\Z$-linear category, meaning that it is Jordan-H\"older (cf. definition \ref{def: jordan_holder_categories}) and Krull-Schmidt (cf. definition \ref{def: krull_schmidt_categories}).}. 
                        \item All cyclic left/right-$R$-modules are semi-simple. 
                    \end{enumerate}
                Furthermore, an associative ring is left-semi-simple if and only if it is right-semi-simple.  
            \end{example}
            \begin{lemma}[Schur's Lemma for abelian categories] \label{lemma: schur_lemma_for_abelian_categories}
                Let $\calA$ be an abelian category and $f: M \to N$ be a non-zero morphism between two simple objects $M$ and $N$. Such a morphism must then be an isomorphism.
            \end{lemma}
                \begin{proof}
                    Since both $M$ and $N$ are simple objects of $\calA$ and since $\calA$ is abelian, the lemma comes from consideration of the following commutative diagram and from the assumption that $f: M \to N$ is a non-zero morphism:
                        $$
                            \begin{tikzcd}
                            	0 & M & 0 \\
                            	0 & N & 0
                            	\arrow["f", from=1-2, to=2-2]
                            	\arrow[from=1-1, to=2-1]
                            	\arrow[from=2-1, to=2-2]
                            	\arrow["{\ker f}", from=1-1, to=1-2]
                            	\arrow[from=1-2, to=1-3]
                            	\arrow["{\coker f}", from=2-2, to=2-3]
                            	\arrow[from=1-3, to=2-3]
                            	\arrow["\lrcorner"{anchor=center, pos=0.125, rotate=180}, draw=none, from=2-3, to=1-2]
                            	\arrow["\lrcorner"{anchor=center, pos=0.125}, draw=none, from=1-1, to=2-2]
                            \end{tikzcd}
                        $$
                \end{proof}
            \begin{corollary}[Endomorphism algebras of simple objects] \label{coro: endomorphism_algebras_of_simple_objects_in_abelian_categories}
                Let $\calA$ be an abelian category and let $M$ be a simple object. Then $\calA(M, M)$ will be a division ring (with respect to compositions of endomorphisms on $M$), and if $N \not \cong M$ is another simply object, then $\calA(M, N) \cong 0$. 
            \end{corollary}
                \begin{proof}
                    Both assertions are direct consequences of lemma \ref{lemma: schur_lemma_for_abelian_categories}.
                \end{proof}
            \begin{example}[Irreducible complex representations]
                Let $k$ be a commutative ring, let $A$ be an associative $k$-algebra (e.g. one might consider the group algebra $A := k\<G\>$ of some abstract group $G$), and let $V$ and $W$ be two irreducible $k$-linear representations of $A$, viewed as simple (left-)$A$-modules. Then, any $A$-linear map $f: M \to N$ (or equivalently, any map of $k$-linear representations of $A$) will either be zero or an isomorphism; when $k$ is an algebraically closed field of characteristic $0$ (e.g. $k \cong \bbC$), such an isomorphism will also have to be given by (left-)multiplication by a scalar $a \in k$ (the scalar $1_k$ corresponds to the identity map)\footnote{For more details, see remark \ref{prop: endomorphism_algebras_of_simple_objects_in_locally_finite_linear_categories}}.
            \end{example}
                
            \begin{definition}[Lengths and Jordan-H\"older series] \label{def: lengths_of_objects_and_jordan_holder_series}
                Let $\C$ be a category with enough kernels and cokernels. An object $X \in \Ob(\C)$ is then said to be of \textbf{length} $n$ (for some $n \in \N$) if and only if there exists a so-called \textbf{Jordan-H\"older filtration} of \textit{normal} monomorphisms:
                    $$0 =: X_0 \subseteq X_1 \subseteq X_2 \subseteq ... \subseteq X_n \subseteq X$$
                such that the quotients (i.e. cokernels) $X_{i + 1}/X_i$ are simple for all $0 \leq i \leq n - 1$; these quotients are commonly called \textbf{Jordan-H\"older factors}. Such a filtration is said to be of \textbf{multiplicity} $m \geq 0$ if and only if the number of isomorphic factors is $m$. 
            \end{definition}
            \begin{convention}
                Zero objects are of length $0$.
            \end{convention}
            \begin{remark}
                Obviously, semi-simple objects are of finite lengths. 
            \end{remark}
            \begin{theorem}[Uniqueness of Jordan-H\"older filtrations] \label{theorem: jordan_holder_theorem}
                Let $\C$ be a category with kernels and cokernels; assume also that $\C$ has enough cokernels. Next, consider an object $X \in \Ob(\C)$ that is of some finite length $n \geq 0$. Then, any Jordan-H\"older filtration of $X$ must be of length $n$.
            \end{theorem}
                \begin{proof}
                    Let $X_{\bullet} := \{X_i\}_{0 \leq i \leq n}$ and $Y_{\bullet} := \{Y_j\}_{0 \leq j \leq m}$ be two Jordan-H\"older filtrations of $X$ and consider some $f_{\bullet} \in \C^{\N}(X_{\bullet}, Y_{\bullet})$. Taking cokernels is functorial, so let us consider the resulting map between factors $\coker f_{\bullet} \in \C^{\N}(X_{\bullet + 1}/X_{\bullet}, Y_{\bullet + 1}/Y_{\bullet})$. By Schur's Lemma (cf. lemma \ref{lemma: schur_lemma_for_abelian_categories}), this is either zero or an isomorphism term-wise, which tells us that $m = n$. 
                \end{proof}
            \begin{example}
                The Jordan-H\"older theorem holds in any semi-abelian category (such as that of groups) and more particularly, any abelian category (such as categories of modules over rings or categories of linear representations).
            \end{example}
            \begin{definition}[Jordan-H\"older categories] \label{def: jordan_holder_categories}
                A \textbf{Jordan-H\"older category} is one wherein every object is of finite length.
            \end{definition}
        
        \subsubsection{Decomposability and the Krull-Schmidt Theorem}
            \begin{definition}[(In)decomposable objects] \label{def: (in)decomposable_objects}
                An object $X$ of a category $\C$ with enough products is said to be \textbf{decomposable} if there exists a family $\{X_i\}_{i \in I}$ of sub-objects $X_i \subseteq X$ such that:
                    $$X \cong \prod_{i \in I} X_i$$
                Otherwise, $X$ is said to be \textbf{indecomposable}.
            \end{definition}
            \begin{remark}
                Semi-simple objects are decomposable, while simple objects are always indecomposable. There can - however - be indecomposable objects which are not simple (e.g. any non-field commutative ring as a module over itself). 
            \end{remark}
            \begin{lemma}[Characterisations of local associative algebras] \label{lemma: characterisations_of_local_associative_algebras}
                \cite[Theorem 19.1]{lam_first_course_in_noncommutative_rings} Let $k$ be a commutative ring. The following are equivalent for any non-zero associative $k$-algebra $(A, +, \cdot)$:
                    \noindent
                    \begin{enumerate}
                        \item $A$ has a unique maximal left-ideal.
                        \item $A$ has a unique maximal right-ideal. 
                        \item $A/\rad(A)$ is a division $k$-algebra (with $\rad(A)$ denoting the Jacobson radical of $A$).
                        \item $A \setminus A^{\x}$ is an $A$-ideal.
                        \item $(A \setminus A^{\x}, +)$ is a group. 
                    \end{enumerate}
            \end{lemma}
            \begin{definition}[Local associative algebras] \label{def: local_associative_algebras}
                An associative algebra $A$ over a commutative ring $k$ is said to be \textbf{local} if and only if it satisfies one of the equivalent conditions of lemma \ref{lemma: characterisations_of_local_associative_algebras}. Note that this is a two-sided notion.
            \end{definition}
            \begin{proposition}[Endormophism algebras of indecomposable objects] \label{prop: endomorphism_algebras_of_indecomposable_objects}
                For some commutative ring $k$, let $\C$ be a $k$-linear category with enough products and let $Z$ be an indecomposable object. Then $\C(Z, Z)$ will be a local associative $k$-algebra. 
            \end{proposition}
                \begin{proof}
                    
                \end{proof}
            \begin{definition}[Krull-Schmidt categories] \label{def: krull_schmidt_categories}
                Let $k$ be a commutative ring. A \textbf{$k$-linear Krull-Schmidt category} is a category is a $k$-linear category in which every object is decomposable into \textit{finitely many} indecomposable factors.
            \end{definition}
            
            \begin{theorem}[Krull-Schmidt Decomposition Theorem] \label{theorem: krull_schmidt_theorem}
                Suppose that $\C$ is a category with enough (co)kernels and products, and that $X \in \Ob(\C)$ is an object of finite length. Then, there exists a (necessarily finite) product decomposition:
                    $$X \cong \prod_{i \in I} X_i$$
                of $X$ into indecomposable sub-objects $X_i \subseteq X$, which is unique up to isomorphisms.
            \end{theorem}
                \begin{proof}
                    Suppose to the contrary that there exists a finite-length object $X \in \Ob(\C)$ which does not admit a (finite) product decomposition into indecomposable sub-objects. Particularly, this means that $X$ is not indecomposable (because if so $X$ would just be its own unary product decomposition into indecomposable subobjects), meaning that by definition, $X$ is decomposable. Let:
                        $$X \cong \prod_{j \in J^{(1)}} Z_j^{(1)}$$
                    be a (necessarily finite) product decomposition of $X$ into sub-objects $Z_j^{(1)} \subseteq X$; the sub-objects $Z_j^{(1)}$ can not be simultaneously indecomposable, so there must exist at least one $j_1 \in J^{(1)}$ such that $Z_{j_1}^{(1)}$ is decomposable. By repeating this argument, one obtains the a (finite) product decomposition of $X$ as follows:
                        $$X \cong \left( \prod_{j \in J^{(1)} \setminus j_1} Z_j^{(1)} \right) \x Z_{j_1}^{(1)} \cong \left( \prod_{j \in J^{(1)} \setminus j_1} Z_j^{(1)} \right) \x \left( \prod_{j \in J^{(2)} \setminus j_2} Z_j^{(2)} \right) \x Z_{j_2}^{(2)} \cong ...$$
                    which yields the following \textit{infinite} filtration on $X$:
                        $$X \supseteq Z_{j_1}^{(1)} \supseteq Z_{j_2}^{(2)} \supseteq ...$$
                    This contradicts the hypothesis that $X$ is of finite length, and therefore $X$ is decomposable into finitely many indecomposable factors.
                \end{proof}
            \begin{example}
                The Krull-Schmidt Decomposition Theorem is, particularly, applicable within the category of groups and any category of modules over an arbitrarily given associative ring. More specifically (and perhaps more pertinent to our needs), the theorem is holds within categories of representations (e.g. of groups, Lie algebras, Hopf algebras, and even of monoids\footnote{Observe that all such objects admit universally defined }). 
            \end{example}
            \begin{example}[Cyclic decomposition of finitely generated modules over PIDs] \label{example: cyclic_decomposition_theorem_for_finitely_generated_modules_over_PIDs}
                An enhanced specialisation of the Krull-Schmidt Decomposition Theorem is the Cyclic Decomposition Theorem for finitely generated modules over PIDs $R$. Such modules are certainly of finite length (which are equal to the cardinalities of their generating sets) and as such - per the Krull-Schmidt Decomposition Theorem - admits a finite product (in fact, direct sum) decomposition into indecomposable modules. The additional lemma to prove for the Cyclic Decomposition Theorem is that over PIDs, indecomposable modules are always cyclic\footnote{Though the converse is not necessarily true (e.g. $\Z/6\Z \cong \Z/2\Z \oplus \Z/3\Z$ as a $\Z$-module).}; more particularly, one has:
                    \begin{enumerate}
                        \item a finitely generated module over a given PID is torsion-free if and only if it is free, and 
                        \item finitely generated torsion modules over PIDs $R$ can be decomposed as:
                            $$\bigoplus_{\p \in \Spec R} (R/\p)^{\oplus e_{\p}}$$
                        for some finitely supported set of powers $\{e_{\p} \in \N \mid \p \in \Spec R\}$.
                    \end{enumerate}
                
                A trivial yet interesting corollary of the Cyclic Decomposition Theorem is that over a given PID $R$, the canonical short exact sequence:
                    $$0 \to \Tor_R(M) \to M \to M/\Tor_R(M) \to 0$$
                splits for any finitely generated $R$-module $M$ (here, $\Tor_R(M)$ denotes the torsion $R$-submodules of $M$).
                
                In turn, the Cyclic Decomposition Theorem admits its own special case, that being the Chinese Remainder Theorem: given any integer $N$ and a fixed prime factorsiation $N := \prod_{i \in I} p_i^{e_i}$ thereof, one has an isomorphism of abelian groups as follows:
                    $$\Z/N\Z \cong \bigoplus_{i \in I} (\Z/p_i\Z)^{e_i}$$
            \end{example}
            
        \subsubsection{(Locally) finite abelian categories}
            \begin{definition}[Locally finite linear categories] \label{def: locally_finite_linear_categories}
                Let $k$ be a commutative ring. A $k$-linear category $\E$ is said to be \textbf{locally finite}\footnote{Also called \textbf{Artinian}.} if:
                    \begin{enumerate}
                        \item for all objects $X, Y \in \Ob(\C)$, the hom-space $\E(X, Y)$ is finitely generated as a $k$-module, and
                        \item every object of $\E$ is of finite-length.
                    \end{enumerate}
            \end{definition}
            \begin{remark}[Locally finite linear categories are Jordan-H\"older and Krull-Schmidt] \label{remark: locally_finite_linear_categories_are_jordan_holder_and_krull_schmidt}
                Clearly, every locally finite linear category is simultaneously Jordan-H\"older (this is tautological) and Krull-Schmidt (because otherwise, there would exist objects of infinite length; cf. the proof of theorem \ref{theorem: krull_schmidt_theorem}).
            \end{remark}
            \begin{proposition}[Endomorphism algebras of simple objects in locally finite linear categories] \label{prop: endomorphism_algebras_of_simple_objects_in_locally_finite_linear_categories}
                Let $k$ be a commutative and $\E$ be a locally finite $k$-linear abelian category. Then $\E(Z, Z)$ will be a division $k$-algebra for all simple objects $Z \in \Ob(\E)$ and in particular, when $k$ is an algebraically closed field, one has $\E(Z, Z) \cong k$.
            \end{proposition}
                \begin{proof}
                    The first assertion is just from corollary \ref{coro: endomorphism_algebras_of_simple_objects_in_abelian_categories}. The second comes from the fact if $D$ is a division algebra which is finite-dimensional over some algebraically closed field $k$, then in fact $D \cong k$ (to prove this, suppose that there exists an element $\alpha \in D \setminus k$ and then consider the field extension\footnote{Note that because $k$ is a $k$-subalgebra of the centre $\rmZ(D) \subseteq D$ per the definition of algebras over a ring, the element $\alpha$ must commute with all elements of $k$, and therefore $k(\alpha)$ is actually a $k$-algebra that is a field by construction.} $k(\alpha)/k$, which is \textit{a priori} algebraic\footnote{Since $k(\alpha)$ is finite-dimensional as a $k$-vector space, by virtue of being a $k$-vector subspace of $D$ (which is finite-dimensional over $k$ by assumption).}: since $k$ is algebraically closed, the minimal polynomial of $\alpha$ must be $(x - \alpha) \in k[x]$, but this in turn implies - via the hypothesis that $k$ is algebraically closed and the fact that the extension $k(\alpha)/k$ is algebraic - that $\alpha \in k$, which is clearly a contradiction, and thus $D \cong k$).
                \end{proof}
                
            \begin{definition}[Finite linear categories] \label{def: finite_linear_categories}
                For any commutative ring $k$, a $k$-linear (abelian) category $\E$ is said to be \textbf{finite} if and only if there exists an \textit{exact} equivalence:
                    $$F: \E \to {}^lA\mod$$
                to the category $A\mod$ of left-modules\footnote{Which are necessarily finitely generated.} over some Artinian $k$-algebra $A$, which we shall call the algebra of \textbf{coefficients} of $\E$.
            \end{definition}
            \begin{proposition}[Finite linear categories are locally finite] \label{prop: finite_linear_categories_are_locally_finite}
                Let $k$ be a commutative ring. Then, any finite $k$-linear category will also be locally finite. 
            \end{proposition}
                \begin{proof}
                    Combine definitions \ref{def: locally_finite_linear_categories} and \ref{def: finite_linear_categories}, and note that categories of finitely generated modules over finite $k$-algebras are indeed locally finite as $k$-linear categories. 
                \end{proof}
            \begin{proposition}[Finite linear categories have enough projectives] \label{prop: finite_linear_categories_have_enough_projectives}
                Let $k$ be a commutative ring. Every finite $k$-linear category has enough projectives.
            \end{proposition}
                \begin{proof}
                    This is tautologically obvious because every left-module $M$ over a ring $A$ admits a free (hence projective) resolution $\e_{\bullet}: P_{\bullet} \to M$; the existence of the two-term complex $\e_1: P_1 \to M \to 0$ then tells us that indeed, finite linear categories have enough projectives. 
                \end{proof}
            
            \begin{lemma}[Simple modules over semi-simple rings] \label{lemma: simple_modules_over_semi_simple_rings}
                Let $R$ be a semi-simple associative ring. Then there are only finitely many isomorphism classes of simple left-$R$-modules. 
            \end{lemma}
                \begin{proof}
                    For the sake of simplicity (and not at the expense of generality), suppose that the given semi-simple ring $R$ admits a product decomposition (in the category of rings) as follows:
                        $$R \cong A \x B$$
                    (wherein, of course, $A$ and $B$ are simple). It is then easy to see that every left-$R$-module $M$ decomposes as:
                        $$M \cong P \oplus N$$
                    wherein $P := (1_A, 0_B)M$ is a left-$A$-module and $N := (0_A, 1_B)M$ is a left-$B$-module. It then follows that $M$ is simple if either $P$ is simple and $N \cong 0$, or \textit{vice versa}.; this means that the number of simple left-$R$-modules is the sum of the number of simple left-modules over $A$ and those over $B$.
                    
                    From here, our task becomes to prove that there are only finitely many isomorphism classes of simple left-modules over a given simple ring $R$. For this, recall first of all that because $R$ is simple (hence semi-simple), every simple left-$R$-module must be left-cyclic (cf. example \ref{example: modules_over_semi_simple_rings}). In other words, simple left-$R$-modules are precisely the left-ideals of $R$, and since $R$ is simple, the only isomorphism class of such modules is that of ${}_R(0)$. Thus we are done. 
                \end{proof}
            \begin{example}[$1$-dimensional vector spaces]
                There is exactly one isomorphism class of simple vector spaces over a fixed field $k$ (which is indeed semi-simple, since fields do not admit any non-zero proper ideal), which is precisely the isomorphism class of $1$-dimensional $k$-vector spaces. 
            \end{example}
            \begin{example}[Irreducible representations]
                If $A$ is an associative algebra over a commutative ring $k$, then there will only be finitely many isomorphism classes of irreducible $k$-linear representations on $A$. 
            \end{example}
            \begin{remark}
                The assumption that $k$ is a field (as opposed to being any commutative ring) is crucial to validity of the following proposition. This is because we need the algebra of coefficient $A$ to be semi-simple. 
            \end{remark}
            \begin{proposition}[Simple objects in finite linear categories] \label{prop: simple_objects_in_finite_linear_categories}
                Let $k$ be a field. Every finite $k$-linear category has finitely many isomorphism classes of simple objects. 
            \end{proposition}
                \begin{proof}
                    Combine lemma \ref{lemma: simple_modules_over_semi_simple_rings} with proposition \ref{prop: semi_simple_iff_trivial_jacobson_radical_and_artinian}.
                \end{proof}
        
    \subsection{Homological dimensions}
        \subsubsection{Projective and injective dimensions; global dimensions}
            \begin{lemma}[Schaunel's Lemma] \label{lemma: schaunel_lemma}
                \cite[\href{https://stacks.math.columbia.edu/tag/00O3}{Tag 00O3}]{stacks}
            \end{lemma}
                \begin{proof}
                    
                \end{proof}
        
            \begin{definition}[Projective and injective dimensions] \label{def: projective_and_injective_dimensions}
                Let $\calA$ be an abelian category and $M \in \Ob(\calA)$ some arbitrary object therein. $M$ is said to be of \textbf{projective/injective dimension} $d \in \N \cup \{\infty\}$ if and only if any projective/injective resolution of $M$ is quasi-isomorphic to a projective/injective resolution of length $d$.
            \end{definition}
            \begin{example}
                Projective/injective modules are of projective/dimension $0$.
            \end{example}
            \begin{remark}
                Note that an object $M \in \Ob(\calA)$ of an abelian category $\calA$ is of injective dimension $d$ if and only if it is of projective dimension $d$ in $\calA^{\op}$. For this reason, the results below will only be stated in terms of projectives instead of projectives and injectives respectively.
            \end{remark}
            \begin{convention}[Resolutions]
                Let $\calA$ be an abelian category and $M \in \Ob(\calA)$ some arbitrary object therein. Then, whenever we want to consider a resolution of $M$ by projectives (supposing that such a resolution exists), we shall mean a long exact sequence of the form:
                    $$
                        M_{\bullet} :=
                        \left\{
                            \begin{tikzcd}
                            	\cdots & {M_{-2}} & {M_{-1}} & M & 0
                            	\arrow[from=1-1, to=1-2]
                            	\arrow["{\del_2}", from=1-2, to=1-3]
                            	\arrow["{\del_1}", from=1-3, to=1-4]
                            	\arrow[from=1-4, to=1-5]
                            \end{tikzcd}
                        \right\}
                    $$
                In particular, note that $M$ is placed in degree $0$, so usually, we will write:
                    $$M_{\bullet}[0] := M$$
                Also, note the \textit{cohomological} indexing.
            \end{convention}
            \begin{lemma}
                (Cf. \cite[\href{https://stacks.math.columbia.edu/tag/00O5}{Tag 00O5}]{stacks}) Let $\calA$ be an abelian category with enough projectives and let $(M_{\bullet}, \del_{\bullet}) \in \Ob(\rmD^{[-e, 0]}_{\proj}(\calA))$ be a projective resolution. Suppose also that the $M_{\bullet}[0]$ is of some finite projective dimension $d \in \N$. Then, the kernel $\ker(\del_{-e}: M_{-e} \to M_{-(e - 1)})$ will be a projective object of $\calA$.
            \end{lemma}
                \begin{proof}
                    
                \end{proof}
            \begin{proposition}
                (Cf. \cite[\href{https://stacks.math.columbia.edu/tag/0CXC}{Tag 0CXC}]{stacks}) Let $\calA$ be an abelian category with enough projectives and let $M \in \Ob(\calA)$ be an object; also let $d \in \N$ be a fixed natural number. Then, the following are equivalent:
                    \begin{enumerate}
                        \item $\projdim M \leq d$.
                        \item $M$ admits a projective resolution of length $d$.
                        \item 
                        \item 
                    \end{enumerate}
            \end{proposition}
                \begin{proof}
                    
                \end{proof}
            \begin{proposition}[Computing projective dimensions using $\Ext^*$] \label{prop: computing_projective_dimensions_using_Ext}
                Let $\calA$ be an abelian category with enough projectives and $M \in \Ob(\calA)$ be an arbitrary object thereof. Also, fix a natural number $d \in \N$. Then, the following are equivalent:
                    \begin{enumerate}
                        \item $\projdim M \leq d$.
                        \item $\Ext_{\calA}^i(M, N) \cong 0$ for all $N \in \Ob(\calA)$ and all $i \geq d + 1$.
                        \item $\Ext_{\calA}^{d + 1}(M, N) \cong 0$ for all $N \in \Ob(\calA)$.
                    \end{enumerate}
            \end{proposition}
                \begin{proof}
                    
                \end{proof}
            \begin{proposition}[Projective dimensions and short exact sequences] \label{prop: projective_dimensions_and_short_exact_sequences}
                
            \end{proposition}
                \begin{proof}
                    
                \end{proof}
                
            \begin{definition}[Global dimensions] \label{def: global_dimensions}
                The \textbf{global dimension} of an abelian category $\calA$ (often assumed to have enough projectives), denoted by $\globdim \calA$, is given by:
                    $$\globdim \calA := \sup_{M \in \Ob(\calA)} \projdim M$$
            \end{definition}
            \begin{convention}[Global dimensions of rings] \label{conv: global_dimensions_of_rings}
                When $\calA$ is the category of left/right-module over some ring $R$, one might speak instead of the \textbf{left/right-global dimension} of the ring $R$ itself instead of the global dimension of the abelian category of left/right-$R$-modules. 
            \end{convention}
            
            \begin{lemma}[Projective and injective modules over semi-simple rings] \label{lemma: projective_and_injective_modules_over_semi_simple_rings}
                \cite[Theorem 2.8]{lam_first_course_in_noncommutative_rings} The following are equivalent for any given associative ring $R$:
                    \begin{enumerate}
                        \item $R$ is left/right semi-simple.
                        \item Left/right-$R$-modules are projective (respectively, injective).
                        \item Finitely generated left/right-$R$-modules are projective (respectively, injective).
                        \item Cyclic left/right-$R$-modules are projective (respectively, injective).
                    \end{enumerate}
            \end{lemma}
            \begin{corollary}
                The category of left/right modules over a semi-simple ring has enough projectives and injectives.
            \end{corollary}
            \begin{corollary}[(Co)homology of modules over semi-simple rings] \label{coro: (co)homology_of_modules_over_semi_simple_rings}
                Let $R$ be a semi-simple associative ring and let $M$ be an arbitrary right-$R$-module. Then:
                    \begin{enumerate}
                        \item $\Ext_R^i(M, N) \cong 0$ for all right-$R$-modules $N$ and all indices $i > 0$. 
                        \item $\Tor^R_i(M, P) \cong 0$ for all left-$R$-modules $P$ and all indices $i > 0$. 
                    \end{enumerate}
            \end{corollary}
            \begin{definition}[Jacobson radicals] \label{def: jacobson_radicals}
                The \textbf{Jacobson radical} of an associative ring $R$ is the two-sided $R$-ideal, denoted by $\rad(R)$, that is maximal among two-sided $R$-ideal whose elements all act as $0$ on simple $R$-modules.
            \end{definition}
            \begin{remark}[Jacobson radicals as intersections of kernels] \label{remark: jacobson_radicals_as_intersections_of_kernels}
                Another way to view the Jacobson radical of a given associative $k$-algebra $A$ (for any commutative ring $k$) is that:
                    $$\rad(A) \cong \bigcap_{(\rho, V) \in \Ob(\Rep_k^{\irr}(A))} \ker \rho$$
                The kernels $\ker \rho$ are two-sided $A$-ideals, which ensures that $\rad(A)$ is also a two-sided $A$-ideal.
            \end{remark}
            \begin{proposition}[Jacobson radicals are nilpotent] \label{prop: jacobson_radicals_are_nilpotent} 
                The Jacobson radical is maximal among nilpotent two-sided ideals. 
            \end{proposition}
                \begin{proof}
                    
                \end{proof}
            \begin{example}[Two-sided ideals in semi-simple rings are trivial] \label{example: two_sided_ideals_in_semi_simple_rings_are_trivial}
                Let $R$ be a semi-simple associative ring and let $M$ be an arbitrary right-$R$-module. Because $\Tor^R_i(M, P) \cong 0$ for all left-$R$-modules $P$ and all indices $i > 0$, take $P \cong {}_RR/I$ for any two-sided $R$-ideal $I$ (e.g. the Jacobson radical $\rad(R)$) to see that:
                    $$\Tor^R_1(M, {}_RR/I) \cong \{x \in M \mid \forall a \in I: xa = 0\} \cong 0$$
                In particular, one has:
                    $$\Tor_1^R(R_R, {}_RR/I) \cong I \cong 0$$
                which tells us that semi-simple rings do not admit any non-trivial two-sided ideals (so for instance, $\rad(R) = 0$ when $R$ is semi-simple). 
                
                The converse statement is true by definition, so we have obtained a characterisation of semi-simple rings via their two-sided ideals (or rather, the lack thereof). This will be used in our proof of Maschke's Theorem (cf. theorem \ref{theorem: maschke_theorem}).
            \end{example}
            \begin{example}[Ideals in matrix rings and semi-simplicity of matrix rings]
                If $R$ is an associative ring and $n > 0$ is any positive integer then there are bijective functions:
                    $$\Mat_n(-): \{\text{left/right/two-sided ideals of $R$}\} \to \{\text{left/right/two-sided ideals of $\Mat_n(R)$}\}$$
                In particular, this tells us that $\Mat_n(R)$ is (semi-)simple if and only if $R$ is (semi-)simple, e.g. $R$ is a product of division rings (so in particular, fields and products thereof).
                
                We should also note that this is actually a very special case of something much more general and much deeper, called \say{Morita equivalence}: essentially, two associative rings $R$ and $S$ are Morita-equivalent if and only if there is an (adjoint and additive) equivalence\footnote{An equivalent version holds for right-modules.}:
                    $$
                        \begin{tikzcd}
                        	{{}^lR\mod} & {{}^lS\mod}
                        	\arrow[""{name=0, anchor=center, inner sep=0}, "{\Hom_S(P, -)}"', bend right, from=1-1, to=1-2]
                        	\arrow[""{name=1, anchor=center, inner sep=0}, "{- \tensor_R P}"', bend right, from=1-2, to=1-1]
                        	\arrow["\dashv"{anchor=center, rotate=-90}, draw=none, from=1, to=0]
                        \end{tikzcd}
                    $$
                wherein $P$ is an $(S, R)$-bimodule which is a finitely generated projective generator of the category of left/right-$R$-modules, such that $S \cong \End_R(P)$. For $n \x n$ matrices, one takes:
                    $$P := R^{\oplus n}$$
                so that:
                    $$S \cong \End_R(R^{\oplus n}) \cong \Mat_n(R)$$
            \end{example}
            \todo[inline]{Write something about left/right-global dimensions of semi-simple rings here}
    
        \subsubsection{Flat dimensions, weak dimensions, and the Artin-Wedderburn Theorem}
            \begin{remark}
                Some other results that could also be included here are lemma \ref{lemma: simple_modules_over_semi_simple_rings} and example \ref{example: modules_over_semi_simple_rings} (which in particular informs us of the trivial fact that left/right-semi-simple algebras are left/right Noetherian and left/right Artinian; cf. \cite[Corollary 2.6]{lam_first_course_in_noncommutative_rings}).
            \end{remark}
            
            \begin{theorem}[Artin-Wedderburn] \label{theorem: artin_wedderburn}
                Let $k$ be a commutative ring and $A$ be a left/right-semi-simple associative $k$-algebra. Then, $A$ admits a finite product decomposition as follows, indexed by the (finitely many) irreducible $k$-linear representations of $A$:
                    $$A \cong \prod_{V \in \Ob(\Rep_k^{\irr}(A))} \End_k(V)$$
            \end{theorem}
                \begin{proof}
                    
                \end{proof}
            \begin{corollary}[The Artin-Wedderburn Theorem over fields] \label{coro: artin_wedderburn_over_fields}
                
            \end{corollary}
                \begin{proof}
                    
                \end{proof}
            \begin{proposition}[Artinian rings with trivial Jacobson radicals are semi-simple] \label{prop: semi_simple_iff_trivial_jacobson_radical_and_artinian}
                An Artinian ring $R$ is semi-simple if and only if its Jacobson radical $\rad(R)$ is zero.
            \end{proposition}
                \begin{proof}
                    
                \end{proof}
            \begin{corollary}[The Sum-Of-Squares Formula for semi-simple algebras] \label{coro: sum_of_squares_formula_for_semi_simple_algebras}
                Let $A$ be an associative algebra over some field $k$. Then:
                    $$\dim_k(A) \geq \sum_{V \in \Ob(\Rep_k^{\irr}(A))} (\dim_k V)^2$$
                and equality occurs if and only if $A$ is semi-simple.
            \end{corollary}
        
    \subsection{Representations of finite groups}
        In this subsection, we turn our focus to representations of finite groups, which we view as modules over group algebras. Our goal is to tabulate a list of important properties, as well as to give a guide on how one might perform explicit computations on these representations.
        
        \subsubsection{Group cohomology; induced and restricted representations}
            \begin{definition}[Group cohomology] \label{def: group_cohomology}
                Let $\E$ be a topos wherein the category $\Ab(\E)$ of internal abelian groups has enough projectives; recall that $\Ab(\E)$ is \textit{a priori} an infinite $\Z_{\E}$-linear tensor category\footnote{This is like \cite[Definition 4.1.1]{EGNO}, but without the local finiteness condition.}, so let us denote its monoidal unit by $\Z_{\E}$. Simultaneously, fix a group object $G \in \Ob(\Grp(\E))$. Then, the functor of \textbf{$\Ab(\E)$-valued group cohomology} of $G$, denoted by:
                    $$H^*_{\Grp(\E)}(G, -)$$
                shall be nothing but:
                    $$\Ext^*_{\Z_{\E}\<G\>}(\Z_{\E}, -)$$
                wherein we view $\Z_{\E}$ as a left-$\Z_{\E}\<G\>$-module via augmentation.
            \end{definition}
            \begin{remark}[How to compute group cohomologies ?]
                The reason that we have insisted in definition \ref{def: group_cohomology} that $\Ab(\E)$ must have enough enough projectives is that in order to compute the cohomology groups:
                    $$H^i_{\Grp(\E)}(G, M) := \Ext^i_{\Z_{\E}\<G\>}(\Z_{\E}, M)$$
                (for any abelian group $M \in \Ob(\Ab(\E))$), one shall have to choose some projective resolution $(P_{\bullet}, \e_{\bullet})$ for $\Z_{\E} \in \Ob(\Ab(\E))$ (which really ought to be viewed as an injective resolution in $\Ab(\E)^{\op}$) and then calculate:
                    $$\Ext^i_{\Z_{\E}\<G\>}(\Z_{\E}, M) := H^i\left( \R\Hom_{\Z_{\E}\<G\>}(P_{\bullet}, M) \right)$$
                Since by definition, one views $\Z_{\E}$ as a left-$\Z_{\E}\<G\>$-module in this context via the augmentation map:
                    $$\e: \Z_{\E}\<G\> \to \Z_{\E}$$
                (given by $\e(f) := f(1)$ for all sections $f \in \Z_{\E}\<G\>$), and because this map is \textit{a priori} surjective (hence an epimorphism, since we are in an abelian category), a good choice of projective resolution $(P_{\bullet}, \e_{\bullet})$ for $\Z_{\E} \in \Ob(\Ab(\E))$ is:
                    $$
                        \begin{tikzcd}
                        	\cdots & {P_{-2}} & {P_{-1} := \Z_{\E}\<G\>} & {P_0 := \Z_{\E}} & 0
                        	\arrow["{\e_0}", from=1-4, to=1-5]
                        	\arrow["{\e := \e_{-1}}", from=1-3, to=1-4]
                        	\arrow["{\e_{-2}}", from=1-2, to=1-3]
                        	\arrow["{\e_{-3}}", from=1-1, to=1-2]
                        \end{tikzcd}
                    $$
                wherein $P_{-n}$ can be conveniently taken to be the free cover\footnote{Which always exists and is projective as an object of $\Ab(\E)$ should one assume the Axiom of Choice.} of $\ker \e_{-(n - 1)}$ for all $n \geq 2$.
            \end{remark}
            \begin{remark}[Group cohomology with coefficients in modules over general rings]
                
            \end{remark}
            \begin{proposition}[Low-dimensional group cohomologies] \label{prop: low_dimensional_group_cohomologies}
                Let $G$ be a group internal to a topos $\E$ wherein the infinite tensor category $(\Ab(\E), \tensor_{\Z_{\E}}, \Z_{\E})$ of internal abelian groups has enough projectives. Then, we have the following results for the low-dimensional $\Ab(\E)$-valued cohomologies of $G$:
                    \begin{enumerate}
                        \item $H^0_{\Grp(\E)}(G, -) \cong (-)^G$ is the functor of $G$-fixed points.
                        \item $H^1_{\Grp(\E)}(G, -) \cong $
                        \item $H^2_{\Grp(\E)}(G, M) \cong \{ \text{isomorphism classes of \textit{central} extensions of $G$ by $M$} \}$ for all $M \in \Ob(\Ab(\E))$.
                    \end{enumerate}
            \end{proposition}
                \begin{proof}
                    
                \end{proof}
            \begin{corollary}[Criteria for being semi-simple and hereditary] \label{coro: criteria_for_being_semi_simeple_and_hereditary}
                Let $G$ be a group internal to a topos $\E$ wherein the infinite tensor category $(\Ab(\E), \tensor_{\Z_{\E}}, \Z_{\E})$ of internal abelian groups has enough projectives. Also, let $k$ be an associative algebra internal to $\Ab(\E)$ (i.e. an associative $\Z_{\E}$-algebra).
                    \begin{enumerate}
                        \item The category ${}^lk\<G\>\mod^{\fin}$ of finitely generated left-$\Z_{\E}\<G\>$-modules\footnote{... or equivalently, the category $\Rep_k^{\fin}(G)$ of finite-rank $k$-linear $G$-representations.} is semi-simple (i.e. $\globdim {}^lk\<G\>\mod^{\fin} = 0$) if and only if the fixed-point functor $(-)^G$ is exact.
                        \item The category ${}^lk\<G\>\mod^{\fin}$ of finitely generated left-$\Z_{\E}\<G\>$-modules is hereditary (i.e. $\globdim {}^lk\<G\>\mod^{\fin} \leq 1$) if and only if there are no non-trivial central extensions of $G$ by any $M \in \Ob(k\mod)$.
                    \end{enumerate}
            \end{corollary}
                \begin{proof}
                    \noindent
                    \begin{enumerate}
                        \item Combine lemma \ref{lemma: projective_and_injective_modules_over_semi_simple_rings} with proposition \ref{prop: computing_projective_dimensions_using_Ext}.
                        \item Apply proposition \ref{prop: computing_projective_dimensions_using_Ext} directly to definition \ref{def: hereditary_abelian_categories}.
                    \end{enumerate}
                \end{proof}
        
        \subsubsection{Maschke's Theorem and characters}
            \begin{lemma}[Projectivity and restriction of scalars] \label{lemma: projectivity_and_restriction_of_scalars}
                Let $R$ be a ring, $S$ be a semi-simple associative $R$-algebra of finite presentation that is also an $R$-coalgebra\footnote{E.g. over some base ring $k$, one can take $S := k\<G\>$ for some abstract group $G$ and $R := k\<H\>$ for some subgroup $H \leq G$; for a concrete example, take $R := k\<1\> \cong k$ and suppose that the associative $S$-algebra structure thereon is given by the augmentation map $\e: k\<G\> \to k$.}, so that $R$ will carry an $(S, R)$-bimodule. Then, the restriction-of-scalars functor:
                    $$\Hom_S(R, -): {}^lS\mod \to {}^lR\mod$$
                will preserve projective objects.
            \end{lemma}
                \begin{proof}
                    
                \end{proof}
            \begin{theorem}[Maschke's Theorem] \label{theorem: maschke_theorem}
                Let $G$ be group and $k$ be a ring. Then the group $k$-algebra $k\<G\>$ will be semi-simple as a $k$-algebra (not just as a $k$-module!) if and only if:
                    \begin{enumerate}
                        \item $k$ is semi-simple ring,
                        \item $G$ is finite, and
                        \item $n$ is invertible in $k$.
                    \end{enumerate}
            \end{theorem}
                \begin{proof}
                    Assume first of all that $k\<G\>$ is semi-simple as an associative $k$-algebra; note that via the augmentation map $\e: k\<G\> \to k$ defined by $\e(f) := f(1)$ for all $f \in k\<G\>$, $k$ is also an associative $k\<G\>$-algebra. Then, by lemma \ref{lemma: projectivity_and_restriction_of_scalars}, projective left-$k\<G\>$-modules will be projective as left-$k$-modules as well. However, since $k\<G\>$ is a semi-simple ring, all left-modules over it are projective (cf. lemma \ref{lemma: simple_modules_over_semi_simple_rings}).
                    
                    Conversely, assume that the three listed conditions hold. Then, using corollary \ref{coro: criteria_for_being_semi_simeple_and_hereditary}, we shall aim to show that the fixed-point functor $(-)^G$ is exact; in fact, it shall suffice to only show that $(-)^G$ is right-exact, since it is \textit{a priori} left-exact by virtue of being a corepresentable on an abelian category, namely ${}^lk\<G\>\mod$ (according to proposition \ref{prop: low_dimensional_group_cohomologies}, $(-)^G \cong \Hom_{k\<G\>}(k, -)$).  
                \end{proof}
            \begin{example}[What happens when $|G| \mid \chara k$ ?]
                
            \end{example}
            \begin{corollary}[The Sum-Of-Squares Formula for finite groups] \label{coro: sum_of_squares_formula_for_finite_groups}
                Let $G$ be a finite group and $k$ is a field wherein $|G|$ is invertible. Then there are finitely many irreducible $k$-linear representations of $G$ (i.e. simple left-$k\<G\>$-modules), and:
                    $$|G| = \sum_{V \in \Ob(\Rep_k^{\irr}(G))} (\dim_k V)^2$$
            \end{corollary}
                \begin{proof}
                    Note that $|G| = \dim_k k\<G\>$ when $\chara k$ does not divide $|G|$, so one can simply apply the Artin-Wedderburn Theorem (cf. theorem \ref{theorem: artin_wedderburn}) to the semi-simple $k$-algebra $k\<G\>$.
                \end{proof}
                
        \subsubsection{Characters and the classification of irreducible representations of finite groups}
        
        \subsubsection{Representations of finite flat group schemes}
        
    \subsection{Nakayama rings}
            
            \section{Quivers, hereditary rings, and homological dimension \texorpdfstring{$\leq 1$}{}}
    \subsection{Quivers}
        \subsubsection{Generalities about quivers and their representations}
            We begin by introducing so-called \say{quivers}, which is more-or-less a relaxed version of directed graphs which turn out to be rather important within representation-theoretic contexts. Our approach is that quivers are certain very simple categories, and as such their representations ought to be viewed as particular instances of representations of categories. Specifically, this means we shall be studying the various algebraic properties of quiver representations via modules over their so-called \say{category algebras}, so that notions of say, finite-type-ness of quivers, could be discussed using purely ring-theoretic arguments.
        
            \begin{definition}[Quivers] \label{def: quivers}
                For any category $\C$, a \textbf{$\C$-valued quiver} $Q$ is a diagram in $\C$ of the following form:
                    $$
                        \begin{tikzcd}
                            Q_1 \arrow[r, "t"', shift right=2] \arrow[r, "s", shift left=2] \arrow["\id_{Q_1}"', loop, distance=2em, in=215, out=145] & Q_0 \arrow["\id_{Q_0}"', loop, distance=2em, in=35, out=325]
                        \end{tikzcd}
                    $$
                wherein $Q_1, Q_0 \in \Ob(\C)$ respectively are known as the objects of \textbf{arrows/vertices} and \textbf{objects/edges}, and the arrows $s, t: Q_1 \toto Q_0$ are known as the \textbf{source} and \textbf{target} morphisms. 
            \end{definition}
            \begin{remark}[The category of quivers]
                Equivalently, one might think of $\C$-valued quivers as $\C$-valued presheaves on the category:
                    $$
                        \begin{tikzcd}
                            {\1} \arrow[r, "t"', shift right=2] \arrow[r, "s", shift left=2] \arrow["\id_{\1}"', loop, distance=2em, in=215, out=145] & {\0} \arrow["\id_{\0}"', loop, distance=2em, in=35, out=325]
                        \end{tikzcd}
                    $$
                As such, for an arbitrarily fixed target category $\C$, one obtains a category $\Quiv(\C)$ of $\C$-valued quivers and natural transformations between them. In particular, when $\C \cong \Sets$, one obtains the presheaf topos $\Quiv$ of $\Sets$-valued quivers. 
            \end{remark}
            \begin{example}[Quivers internal to $\Sets$]
                A $\Sets$-valued quiver is actually nothing but a so-called \textbf{directed graph}: such a quiver is a quadruple $(Q_1, Q_0, s, t)$ wherein $Q_1$ is a set of directed edges, $Q_0$ is the set of vertices of those direct edges, and $s, t: Q_1 \toto Q_0$ are the assignments of the sources and targets vertices (i.e. beginning and endpoints) to the aforementioned directed edges. One important detail to note here is that between two given vertices, there may be many directed edges, and there might also be loops onto the same vertex, as follows:
                    $$
                        \begin{tikzcd}
                                                                                                                                           &         & \bullet \arrow[loop, distance=2em, in=125, out=55]             &                                                        & \bullet \\
                            \bullet \arrow[r, shift right=2] \arrow[r, shift left=2] \arrow[loop, distance=2em, in=215, out=145] \arrow[r] & \bullet & \bullet \arrow[r, shift right=2] \arrow[l] \arrow[u] \arrow[d] & \bullet \arrow[ru] \arrow[rd] \arrow[l, shift right=2] &         \\
                                                                                                                                           &         & \bullet \arrow[loop, distance=2em, in=305, out=235]            &                                                        & \bullet
                        \end{tikzcd}
                    $$
                These diagrams of sets and functions, however, need not be commutative: as such, one is usually interested in the free categories associated to quivers, which are nothing but those quivers with compositions and identities added on (cf. proposition \ref{prop: free_quivers}).
                
                To be very specific, Dynkin diagrams are examples of $\Sets$-valued quivers (cf. definition \ref{def: dynkin_quivers}). 
            \end{example}
            \begin{proposition}[Sub-object classifier for $\Quiv$] \label{prop: sub_object_classifier_for_topos_of_quivers}
                
            \end{proposition}
                \begin{proof}
                    
                \end{proof}
            \begin{definition}[The double negation topology]
                
            \end{definition}
            \begin{proposition}[A separatedness criterion for quivers] \label{prop: separatedness_criterion_for_quivers}
                
            \end{proposition}
                \begin{proof}
                    
                \end{proof}
                
            \begin{proposition}[Free quivers] \label{prop: free_quivers}
                The evident forgetful functor:
                    $$\oblv: 1\-\Cat_1 \to \Quiv$$
                (which forgets compositions and the associativity of said compositions) admits a left-adjoint:
                    $$[-]: \Quiv \to 1\-\Cat_1$$
                which shall be known as the \textbf{free quiver} functor. Explicitly, this functor assigns to each ($\Sets$-valued) quiver its associated free category.
            \end{proposition}
                \begin{proof}
                    
                \end{proof}
            \begin{definition}[Quiver representations] \label{def: quiver_representations}
                Given a quiver $Q \in \Ob(\Quiv)$, a commutative ring $k$, and a $k$-linear tensor category\footnote{Aside from the obvious example of $k\mod^{\heart}$ (and $k$-linear tensor subcategories thereof, like $(k\mod^{\fin})^{\heart}$), one might also consider categories such as the derived category of $k$-modules $k\mod$.} $\calV$, its category of $k$-linear representations, denoted by $\Rep_{\calV}(Q)$, is the functor category $\Func([Q], \calV)$.
            \end{definition}
            \begin{remark}[Basic properties of quiver representations]
                Given a quiver $Q \in \Ob(\Quiv)$, a commutative ring $k$, and a $k$-linear tensor category $\calV$, its category of $k$-linear representations $\Rep_{\calV}(Q)$ will also be a $k$-linear tensor category.
                
                Of course, one could consider representations with values in categories of a more general kind than linear tensor $1$-categories, such as symmetric monoidal stable $\infty$-categories. 
            \end{remark}
            \begin{convention}
                Often, we will consider the case $\calV \cong k\mod$ for some commutative ring $k$, and in which case, we shall write $\Rep_k(Q)$ for the category of $k$-linear representations of $Q$. Some important subcategories therein are:
                    $$\Rep_k^{\fin}(Q)$$
                    $$\Rep_k^{\irr}(Q)$$
                    $$\Rep_k^{\red}(Q)$$
                which are the categories of finite-rank $k$-linear representations of $Q$, of irreducible representations, and that of indecomposable representations.
            \end{convention}
            
            Let us now move on to the notion of so-called \say{quiver algebras} and discuss the roles that they play in the representation theory of $\Sets$-valued quivers. 
            \begin{definition}[Category algebras] \label{def: category_algebras}
                Let $k$ be an associative ring and $\C$ be a category. Then, the \textbf{category $k$-algebra} $k\<\C\>$ of the given category $\C$ shall be the free $k$-algebra:
                    $$k\<\C_1\> := k\<\Mor(\C)\>$$
                whose underlying (left-)$k$-module is the free (left-)$k$-module on the set $\C_1 := \Mor(\C)$ of morphisms of $\C$ and whose (associative and unital) multiplication is given by compositions of arrows in $\C_1$ (if two arrows are not composable then their product will be $0$).
            \end{definition}
            \begin{convention}[Path algebras ?]
                Sometimes the category $k$-algebra (for some ring $k$) of the free category associated to a given $\Sets$-valued quiver $Q$ is also called the path algebra of that quiver. It is commonly denoted simply by $k\<Q\>$ as opposed to $k\<[Q]\>$.
            \end{convention}
            \begin{remark}[A path algebra functor] \label{remark: path_algebra_functor}
                When $k$ is commutative, one has a tautological equivalence of categories:
                    $$k\bimod \cong k\mod$$
                which gives us access to the tensor $k$-algebra construction, which is \textit{a priori} a left-adjoint:
                    $$
                        \begin{tikzcd}
                        	{k\-\Assoc\Alg} & k\mod
                        	\arrow[""{name=0, anchor=center, inner sep=0}, "{(-)^{\tensor}}"', bend right, from=1-2, to=1-1]
                        	\arrow[""{name=1, anchor=center, inner sep=0}, "\oblv"', bend right, from=1-1, to=1-2]
                        	\arrow["\dashv"{anchor=center, rotate=-90}, draw=none, from=0, to=1]
                        \end{tikzcd}
                    $$
                Using this, one sees that category $k$-algebras arise functorially in the following manner:
                    $$
                        \begin{tikzcd}
                        	{k\-\Assoc\Alg} && k\mod \\
                        	\\
                        	{1\-\Cat_1} && \Sets
                        	\arrow[""{name=0, anchor=center, inner sep=0}, "{(-)^{\tensor}}"', bend right, from=1-3, to=1-1]
                        	\arrow[""{name=1, anchor=center, inner sep=0}, "\oblv"', bend right, from=1-1, to=1-3]
                        	\arrow["{(-)_1}", from=3-1, to=3-3]
                        	\arrow["{k\<-\> := ( k^{\oplus (-)_1} )^{\tensor}}", dashed, from=3-1, to=1-1]
                        	\arrow[""{name=2, anchor=center, inner sep=0}, "{k^{\oplus (-)}}", bend left, from=3-3, to=1-3]
                        	\arrow[""{name=3, anchor=center, inner sep=0}, "\oblv", bend left, from=1-3, to=3-3]
                        	\arrow["\dashv"{anchor=center, rotate=-90}, draw=none, from=0, to=1]
                        	\arrow["\dashv"{anchor=center}, draw=none, from=2, to=3]
                        \end{tikzcd}
                    $$
                Now, since the functors:
                    $$(-)^{\tensor}: k\mod \to k\-\Assoc\Alg$$
                and:
                    $$k^{\oplus (-)}: \Sets \to k\mod$$
                are left-adjoints, and because:
                    $$(-)_1: 1\-\Cat_1 \to \Sets$$
                preserves coproducts and pushouts, one sees that given any pair of categories $\C, \C'$ (which need not be disjoint), one has:
                    $$k\<\C \cup \C'\> \cong k\<\C\> \tensor_k k\<\C'\>$$
                Since the construction of free categories on (small) quivers also comes from a left-adjoint, namely:
                    $$[-]: \Quiv \to 1\-\Cat$$
                (cf. proposition \ref{prop: free_quivers}), one also has\footnote{One does not have to worry about the formal existence of $Q \cup Q'$ within the category $\Quiv$ of $\Sets$-valued quivers, since it is a topos by construction.}:
                    $$k\<Q \cup Q'\> \cong k\<Q\> \tensor_k k\<Q'\>$$
            \end{remark}
            \begin{example}[Path algebras of discrete quivers]
                The simplest example of a quiver is the trivial loop:
                    \begin{figure}[H]
                        \centering
                            $$
                                \begin{tikzcd}
                                    \bullet \arrow["\id"', loop, distance=2em, in=35, out=325]
                                \end{tikzcd}
                            $$
                        \caption{The trivial quiver}
                        \label{fig: trivial_quiver}
                    \end{figure}
                It is easy to see that the path algebra of this quiver (over some fixed ring $k$) is:
                    $$k\<e\>/\<e^2 = e\> \cong k$$
                    
                Going off in a different direction\footnote{Pun may or may not have been intended.}, consider the following quiver $Q := (Q_1, Q_0, s, t)$ (i.e. a discrete category\footnote{Let's ignore potential set-theoretic issues for now!}):
                    $$
                        \begin{tikzcd}
                            \cdots & \bullet_i \arrow["\id_{\bullet_i}"', loop, distance=2em, in=125, out=55] & \bullet_j \arrow["\id_{\bullet_j}"', loop, distance=2em, in=125, out=55] & \bullet_k \arrow["\id_{\bullet_k}"', loop, distance=2em, in=125, out=55] & \cdots
                        \end{tikzcd}
                    $$
                Its path algebra is:
                    $$k\<\{e_i\}_{i \in Q_0}\>\/\<\forall i \in Q_0: e_i^2 = e_i, \forall i, j \in Q_0: e_i e_j = e_j e_i = \delta_{ij}\>$$
                (where $\delta_{ij}$ is the Kronecker delta), which we recognise as being isomorphic to the product:
                    $$\prod_{i \in Q_0} k$$
                taken in the category of associative $k$-algebras.
            \end{example}
            \begin{example}[Path algebras of $\sfA_n$ quivers] \label{example: path_algebras_of_A_n_quivers}
                For a simple yet non-trivial example of quiver algebras, consider the following quiver, commonly known as the \say{$\sfA_2$ quiver}:
                    \begin{figure}[H]
                        \centering
                        $$
                            \begin{tikzcd}
                                {\bullet_1} \arrow["\id_{\bullet_1}"', loop, distance=2em, in=215, out=145] \arrow[r] & {\bullet_2} \arrow["\id_{\bullet_2}"', loop, distance=2em, in=35, out=325]
                            \end{tikzcd}
                        $$
                        \caption{The $\sfA_2$ quiver}
                        \label{fig: A_2_quiver}
                    \end{figure}
                which has a rather small path algebra, isomorphic to:
                    $$k\<e_1, e_2, x\>/\<e_i^2 = e_i, e_i e_j = e_j e_i = \delta_{ij}, x^2 = 0, e_2 x = x e_1 = x\>$$
                One observation to make is that there is an associative $k$-algebra isomorphism given by:
                    $$e_1 \mapsto \begin{pmatrix} 1 & 0 \\ 0 & 0 \end{pmatrix}$$
                    $$e_2 \mapsto \begin{pmatrix} 0 & 0 \\ 0 & 1 \end{pmatrix}$$
                    $$x \mapsto \begin{pmatrix} 0 & 0 \\ 1 & 0 \end{pmatrix}$$
                between this path $k$-algebra and the $k$-algebra:
                    $$\b_2^-(k) := \left\{ \begin{pmatrix} * & 0 \\ * & * \end{pmatrix} \in \Mat_2(k) \right\}$$
                of lower-triangular $2 \x 2$ matrices with entries in $k$.
                    
                Next, consider the general $\sfA_n$ quiver\footnote{Henceforth we shall start omiting the identity paths from depictions of quivers.} (for some finite positive integer $n$):
                    \begin{figure}[H]
                        \centering
                            $$
                                \begin{tikzcd}
                                	{\bullet_1} & {\bullet_2} & {\bullet_3} & \cdots
                                	\arrow["{x_{12}}", from=1-1, to=1-2]
                                	\arrow["{x_{23}}", from=1-2, to=1-3]
                                	\arrow["{x_{34}}", from=1-3, to=1-4]
                                \end{tikzcd}
                            $$
                        \caption{The $\sfA_n$ quiver}
                        \label{fig: A_n_quiver}
                    \end{figure}
                Observe that the set of morphisms $[\sfA_n]_1$ of the free category $[\sfA_n]$ on $\sfA_n$ is nothing but $\{x_{ij}\}_{1 \leq i \leq j \leq n}$ (also, note that $e_i = x_{ii}$ for all $1 \leq i \leq n$). Now, by letting $\1_{ij}$ denote the $n \x n$ matrix with $1$ at the $ij^{th}$ entry and $0$ everywhere else, and consider the map:
                    $$k\<\sfA_n\> \to \Mat_n(k)$$
                    $$x_{ij} \mapsto \1_{ij}$$
                one then sees that $k\<\sfA_n\>$ is isomorphic to the $k$-algebra:
                    $$\b_n^-(k)$$
                of lower-triangular $n \x n$ matrices with entries in $k$.
            \end{example}
            \begin{example}[Path algebras of quivers with loops]
                If we were to add a non-trivial loop $x: \bullet \to \bullet$ to the trivial quiver $(\{\bullet\}, \{\id\})$ to yield:
                    $$
                        \begin{tikzcd}
                            \bullet \arrow["x"', loop, distance=2em, in=35, out=325]
                        \end{tikzcd}
                    $$
                then the resulting path algebra will be:
                    $$k\<e, x\>/\<e^2 = e, ex = xe = x\> \cong k\<x\>$$
                Of course, when $k$ is a commutative ring, then we have a tautological isomorphism $k\<x\> \cong k[x]$, but this does not really help us flesh out the nature of the path algebra of this loop quiver.
                    
                More generally, we can consider the quiver with one vertex and a set $\{x_i\}_{i \in I}$ of (possibly infinitely) many loops on that vertex:
                    $$
                        \begin{tikzcd}
                            \bullet \arrow["x_1"', loop, distance=2em, in=305, out=235] \arrow["x_2"', loop, distance=2em, in=35, out=325] \arrow["x_3"', loop, distance=2em, in=125, out=55] \arrow["\cdots"', loop, distance=2em, in=215, out=145]
                        \end{tikzcd}
                    $$
                then the resulting path algebra will be:
                    $$k\<\{x_i\}_{i \in I}\>$$
                because while one can free form compositions $x_j \circ x_i \circ ...$ of the endomorphisms/loops $x_i \in \End(\bullet)$, there is not a reason to expect that $x_j x_i = x_i x_j$. 
            \end{example}
            \begin{example}[Path algebras of non-simply-laced quivers]
                Consider the so-called \say{Kronecker quiver}:
                    \begin{figure}[H]
                        \centering
                            $$
                                \begin{tikzcd}
                                	{\bullet_1} & {\bullet_2}
                                	\arrow["x_2"', shift right=1, from=1-1, to=1-2]
                                	\arrow["x_1", shift left=1, from=1-1, to=1-2]
                                \end{tikzcd}
                            $$
                        \caption{The Kronecker quiver}
                        \label{fig: kronecker_quiver}
                    \end{figure}
                which is easily recognisable as the (non-disjoint) union of two copies of the $\sfA_2$ quiver (cf. example \ref{example: path_algebras_of_A_n_quivers}). One sees that its path $k$-algebra is:
                    $$k\<e_1, e_2, x_1, x_2\>/\<e_i^2 = e_i, e_i e_j = e_j e_i = \delta_{ij}, x_m x_n = 0, e_2 x_m = x_m e_1 = x_m\>$$
                which can be shown to be isomorphic to the $k$-algebra\footnote{$\b_2^-(k)$ is the $k$-algebra of lower-triangular $2 \x 2$ matrices with entries in $k$; cf. example \ref{example: path_algebras_of_A_n_quivers}.}:
                    $$k\<\sfA_2 \cup \sfA_2\> \cong k\<\sfA_2\> \tensor_k k\<\sfA_2\> \cong \b_2^-(k) \tensor_k \b_2^-(k)$$
                using the fact that the path $k$-algebra functor:
                    $$k\<-\>: \Quiv \to k\-\Assoc\Alg$$
                preserves pushouts\footnote{In $\Quiv$, these are nothing but unions of quivers} (cf. remark \ref{remark: path_algebra_functor}) whenever $k$ is commutative; when $k$ is noncommutative, one can still show that the isomorphism holds using the fact that $\b_2^-(k)$ is a $k$-bimodule.
            \end{example}
            \begin{example}[Path algebras of $\sfD_n$ quivers]
                For any $n \geq 2$, the so-called \say{$\sfD_n$ quiver} is of the form:
                    \begin{figure}[H]
                        \centering
                            $$
                                \begin{tikzcd}
                                	&&&& {\bullet_n} \\
                                	{\bullet_1} & {\bullet_2} & \cdots & {\bullet_{n - 2}} \\
                                	&&&& {\bullet_{n - 1}}
                                	\arrow["{x_{12}}", from=2-1, to=2-2]
                                	\arrow["{x_{23}}", from=2-2, to=2-3]
                                	\arrow["{x_{n - 2, n - 3}}", from=2-3, to=2-4]
                                	\arrow["{x_{n, n - 2}}", from=2-4, to=1-5]
                                	\arrow["{x_{n - 1, n - 2}}"', from=2-4, to=3-5]
                                \end{tikzcd}
                            $$
                        \caption{The $\sfD_n$ quiver}
                        \label{fig: D_n_quiver}
                    \end{figure}
                (when $n = 2, 3$, note that $\sfD_n = \sfA_n$, so there is nothing new in those cases, and we shall assume hereonafter that $n \geq 3$). Observe that for all $n \geq 3$, the $\sfD_n$ quiver is the union of the $\sfA_{n - 2}$ quiver with two copies of the $\sfA_2$ quiver, both at the $(n - 2)^{th}$ vertex and in such a manner that:
                    $$x_{n - 1, n - 2} x_{n - 2, n - 3} \not = 0$$
                    $$x_{n, n - 2} x_{n - 2, n - 3} \not = 0$$
                and with this observation in mind, one sees that:
                    $$k\<\sfD_n\> \cong k\<\sfA_{n - 2}\> \tensor_k k\<\sfA_2\> \tensor_k k\<\sfA_2\> \tensor_k k\<\sfA_3\>/$$
            \end{example}
            \begin{example}[Path algebras of $\sfE_n$ quivers]
                
            \end{example}
            \begin{proposition}[Quiver representations are modules over quiver algebras] \label{prop: quiver_representations_are_modules_over_quiver_algebras}
                Let $k$ be a commutative ring. Then, there is an exact and monoidal equivalence of $k$-linear categories (fibred over $k\mod$):
                    $$\Rep_{k}(Q) \to {}^lk\<Q\>\mod$$
            \end{proposition}
                \begin{proof}
                    
                \end{proof}
            \begin{remark}
                The equivalence of categories:
                    $$\Rep_{k}(Q) \to {}^lk\<Q\>\mod$$
                from proposition \ref{prop: quiver_representations_are_modules_over_quiver_algebras} has many important further properties, which all come from the very definition of linear representations of associative algebras (in this case, path algebras of quivers) themselves. For instance, this functor preserves (semi-)simplicity and (in)decomposability.
                
                As for further properties coming not from the definition of representations themselves, but rather from the basic property of this equivalence of categories, one has that the functor also preserves lengths and more generally, ascending and descending chains of objects, as a consequence of being exact. In particular, this means that should a representation of a given quiver $Q$ is finitely generated as a $k$-module (i.e. \say{finite-dimensional}), then the same would also be true for the corresponding left-$k\<Q\>$-module.
            \end{remark}
            
        \subsubsection{Admissible ideals; quivers with relations}
            \begin{convention}
                From now on, all quivers shall be $\Sets$-valued and therefore we shall simply say \say{quiver} for brevity.
            \end{convention}
            \begin{convention}
                Let us henceforth assume that $k$ is field (i.e. a simple commutative ring). For the most part, this assumption is made so that given any quiver $Q := (Q_1, Q_0, s, t)$, the corresponding path $k$-algebra $k\<Q\>$ will admit $\<Q\> := \<[Q]_1\>$ as a maximal two-sided ideal, commonly referred to as the \say{arrow ideal} of $Q$.
            \end{convention}
            
            \begin{definition}[Admissible ideals of path algebras] \label{def: admissible_ideals_of_path_algebras}
                The path $k$-algebra of any given quiver is said to be \textbf{admissible} if and only if it is admissible in the sense of definition \ref{def: pre_admissible_and_pre_adic_rings}, and said to be \textbf{bounded} if and only if its ideal of definition is a finite power of the arrow ideal.
            \end{definition}
            
        \subsubsection{The Cartan-Tits quadratic form of a quiver; Grothendieck groups, Euler characteristics, and dimension vectors of quiver representations}
            \begin{definition}[Cartan-Tits quadratic forms of finite quivers] \label{def: cartan_tits_quadratic_forms_of_finite_quivers}
                Let $Q := (Q_1, Q_0, s, t)$ be a finite quiver\footnote{With value in any category} and suppose that $n := |Q_0|$; also, let us give the set $Q_0$ of vertices a \textit{fixed} enumeration $\{v_1, ..., v_n\}$. To such a quiver, one can associate a so-called \textbf{adjacency matrix}:
                    $$R_Q \in \Mat_n(\Z)$$
                given by:
                    $$R_Q := (r_{ij} := |\{f \in Q_1 \mid s(f) = v_i, t(f) = v_j\}|)_{1 \leq i, j \leq n}$$
                (that is, the entries $r_{ij}$ of $R_Q$ are the numbers of \textit{undirected} edges from the vertex $v_i$ to the vertex $v_j$). From this matrix, one can define the \textbf{Cartan matrix} of $Q$:
                    $$A_Q := 2I_n - R_Q$$
                along with a $\Z$-bilinear form, which we shall call the \textbf{Cartan-Tits form}:
                    $$B_Q: \Z^{\oplus n} \x \Z^{\oplus n} \to \Z$$
                    $$(x, y) \mapsto x^{\top} A_Q y$$                
            \end{definition}
            \begin{definition}[Dynkin quivers] \label{def: dynkin_quivers}
                A \textbf{Dynkin quiver} is a \textit{connected} and \textit{simply laced} finite quiver whose Cartan-Tits form is positive-definite.
            \end{definition}
            \begin{proposition}[Evenness of Cartan-Tits forms of Dynkin quivers]
                Let $\Gamma := (\Gamma_1, \Gamma_0, s, t)$ be a Dynkin quiver and consider its Cartan matrix $A_{\Gamma}$. Then for all $x \in \Z^{\oplus n}$, $B_{\Gamma}(x, x)$ is even.  
            \end{proposition}
                \begin{proof}
                    Let $n := |\Gamma_0|$ and pick a basis for $\Z^{\oplus n}$ in order to write:
                        $$
                            \begin{aligned}
                                B_{\Gamma}(x, x) & = x^{\top} A_{\Gamma} x
                                \\
                                & = \sum_{1 \leq i, j \leq n} x_i a_{ij} x_j
                                \\
                                & = \sum_{1 \leq i, j \leq n} x_i (2\delta_{ij} - r_{ij}) x_j \text{(wherein $\delta_{ij}$ is the Kronecker delta)}
                                \\
                                & = 2\sum_{1 \leq i, j \leq n} x_i \delta_{ij} x_j - \sum_{1 \leq i, j \leq n} x_i (1 - \delta_{ij}) r_{ij} x_j \text{($r_{ij}$ are the entries the adjacency matrix of $\Gamma$)}
                                \\
                                & = 2\sum_{1 \leq i \leq n} x_i^2 - 2\sum_{1 \leq i < j \leq n} x_i r_{ij} x_j
                            \end{aligned}
                        $$
                    for all $x \in \Z^{\oplus n}$, wherein the last line is due to the fact that the number of undirected edges from a vertex $v_i$ to another $v_j$ is equal to the number of undirected directed edges from $v_j$ to $v_i$, which in turn is a result of the assumption that by virtue of being a Dynkin quiver, $\Gamma$ is connected and simply laced\footnote{Note how this argument fails if $\Gamma$ is either not simply laced or not connected.}. Clearly, $B_{\Gamma}(v, v)$ is even for all $x \in \Z^{\oplus n}$.
                \end{proof}
            \begin{definition}[Roots] \label{def: roots_of_dynkin_quivers}
                A root of a \textit{Dynkin} quiver $\Gamma := (\Gamma_1, \Gamma_0, s, t)$ is a vector:
                    $$\alpha \in \Z^{\oplus n}$$
                (where $n := |\Gamma_0|$) such that:
                    $$B_{\Gamma}(\alpha, \alpha) = 2$$
                The set of all roots of $\Gamma$ is denoted by $\Phi_{\Gamma}$, and often called the \textbf{root system} of $\Gamma$.
            \end{definition}
            \begin{example}[Simple roots] \label{example: simple_roots}
                Let $\Gamma := (\Gamma_1, \Gamma_0, s, t)$ be a Dynkin quiver and set $n := \Gamma_0$. Also, pick a basis $\{e_1, ..., e_n\}$ for $\Z^{\oplus n}$. Then the vectors of the form:
                    $$\alpha_i := \delta_{ij} e_j$$
                (for $1 \leq i, j \leq n$) are in fact roots of $\Gamma$ and furthermore, they form a basis for $\Z^{\oplus n}$.
            \end{example}
            \begin{lemma}[Roots are exclusively either negative or positive] \label{lemma: roots_are_exclusively_either_negative_or_positive}
                Let $\Gamma := (\Gamma_1, \Gamma_0, s, t)$ be a Dynkin quiver and set $n := \Gamma_0$. Also, pick a basis $\{e_1, ..., e_n\}$ for $\Z^{\oplus n}$ and choose a root $\alpha$ and write it in terms of the simple roots $\{\alpha_i\}_{1 \leq i \leq n}$ as:
                    $$\alpha := \sum_{1 \leq i \leq n} c_i \alpha_i$$
                Then either $c_i \geq 0$ or $c_i \leq 0$ for all $1 \leq i \leq n$ simultaneously.
            \end{lemma}
                \begin{proof}
                    
                \end{proof}
            \begin{definition}[Positive and negative roots] \label{def: negative_and_positive_roots}
                Let $\Gamma := (\Gamma_1, \Gamma_0, s, t)$ be a Dynkin quiver and set $n := \Gamma_0$. Also, pick a basis $\{e_1, ..., e_n\}$ for $\Z^{\oplus n}$ and choose a root $\alpha$ and write it in terms of the simple roots $\{\alpha_i\}_{1 \leq i \leq n}$ as:
                    $$\alpha := \sum_{1 \leq i \leq n} c_i \alpha_i$$
                Then we say that $\alpha$ is \textbf{positive} if and only if $c_i \geq 0$ for all $1 \leq i \leq n$ and \textbf{negative} if and only if $c_i \leq 0$ for all $1 \leq i \leq n$. The set of positive (respectively, negative) roots of the given Dynkin quiver $\Gamma$ is denoted by $\Phi_{\Gamma}^+$ (respectively, $\Phi_{\Gamma}^-$).
            \end{definition}
            \begin{remark}
                Obviously, one has:
                    $$\Phi_{\Gamma} = \Phi_{\Gamma}^+ \cup \Phi_{\Gamma}^-$$
                for all Dynkin quivers $\Gamma$.
            \end{remark}
            
            Now that we are familiar with Cartan-Tits quadratic forms, let us see how they arise naturally as Euler characteristics of representations of quivers (cf. theorem \ref{theorem: cartan_tits_quadratic_forms_as_euler_characteristics}). 
            \begin{convention}[Jordan-H\"older multiplicities]
                Recall that by Schur's Lemma (cf. lemma \ref{lemma: schur_lemma_for_abelian_categories}), all morphisms between simple objects of any given abelian category are either zero or isomorphisms. Recall also, that by the Jordan-H\"older Theorem (cf. theorem \ref{theorem: jordan_holder_theorem}), Finite-length objects have Jordan-H\"older filtrations which are unique up to isomorphisms, which in particular implies that the simple factors are also unique up to isomorphisms. As such, one may speak of the \textbf{multiplicity} (cf. definition \ref{def: lengths_of_objects_and_jordan_holder_series}) of \textit{any} given simple object $M \in \Ob(\E)$ in any finite ($\Z$-linear) abelian category $\E$ (cf. definition \ref{def: finite_linear_categories}) within the Jordan-H\"older series of some other object $X \in \Ob(\E)$, which we shall denote by:
                    $$[X : M]$$
                Note that $X$ is necessarily of finite length per remark \ref{remark: locally_finite_linear_categories_are_jordan_holder_and_krull_schmidt} and the fact that finite abelian categories are locally finite by definition (cf. proposition \ref{prop: finite_linear_categories_are_locally_finite}).
            \end{convention}
            \begin{definition}[Grothendieck groups] \label{def: grothendieck_groups}
                
            \end{definition}
            \begin{remark}
                
            \end{remark}
            
            \begin{definition}[Euler characteristics] \label{def: euler_characteristics}
                
            \end{definition}
            \begin{example}[Euler characteristics of quiver representations] \label{example: euler_characteristics_of_quiver_representations}
                
            \end{example}
            
            \begin{definition}[Dimension vectors of quiver representations] \label{def: dimension_vectors_of_quiver_representations}
                Let $k$ be a commutative ring, let $Q := (Q_1, Q_0, s, t)$ be a finite quiver, and consider the finite $k$-linear abelian category $\Rep_k^{\fin}(Q)$ of finite-rank $k$-linear representations of $Q$. We then call the assignment:
                    $$\rank_k: \Rep_k^{\fin}(Q) \to \Z^{\oplus Q_0}$$
                    $$\calF \mapsto \sum_{v \in Q_0} \rank_k \calF(v) e_v$$
                the assignment of \textbf{rank vectors} (or \textbf{dimension vectors} when $k$ is a field) to finite-rank $k$-linear representations of $Q$, wherein the set $\{e_v\}_{v \in Q_0}$ is some choice of basis for the free $\Z$-module $\Z^{\oplus Q_0}$.
            \end{definition}
            \begin{remark}
                Let $k$ be a field and let $Q := (Q_1, Q_0, s, t)$ be a finite quiver. It is not hard to notice that taking rank vectors of finite-rank $k$-linear representations of $Q$ is additive, i.e.:
                    $$\rank_k (\calF \oplus \calF') = \rank_k \calF + \rank_k \calF'$$
                for all $\calF, \calF' \in \Ob(\Rep_k^{\fin}(Q))$. 
            \end{remark}
            \begin{lemma}[Representations of finite quivers are determined by dimension vectors] \label{lemma: representations_of_finite_quivers_are_determined_by_dimension_vectors}
                Let $k$ be a field and let $Q := (Q_1, Q_0, s, t)$ be a finite quiver. There is then a $\Z$-module isomorphism:
                    $$\rank_k: \K_0(\Rep_k^{\fin}(Q)) \to \Z^{\oplus Q_0}$$
            \end{lemma}
                \begin{proof}
                    
                \end{proof}
            \begin{corollary}[Irreducible representations of finite quivers are labelled by vertices] \label{coro: irreducible_representations_of_finite_quivers_are_labelled_by_vertices}
                Let $k$ be a commutative ring and $Q := (Q_1, Q_0, s, t)$ be a finite quiver. Then, there are choices of bijections:
                    $$|\Rep_k^{\irr}(\Gamma)| \cong Q_0$$
            \end{corollary}
                \begin{proof}
                    Since the Grothendieck group $\K_0(\Rep_k^{\fin}(Q))$ is freely generated as a $\Z$-module by the set of isomorphism classes of irreducible $k$-linear representations of $Q$ (i.e. the simple objects of $\Rep_k^{\fin}(Q)$), this is an immediate consequence of lemma \ref{lemma: representations_of_finite_quivers_are_determined_by_dimension_vectors}.
                \end{proof}
            \begin{theorem}[Cartan-Tits quadratic forms as Euler characteristics] \label{theorem: cartan_tits_quadratic_forms_as_euler_characteristics}
                Let $k$ be a commutative ring and let $Q := (Q_1, Q_0, s, t)$ be a finite quiver, and let us denote the Cartan-Tits quadratic form of this quiver by $B_Q$ (cf. definition \ref{def: cartan_tits_quadratic_forms_of_finite_quivers}). Then, there is a commutative diagram of $\Z$-linear maps as follows:
                    $$
                        \begin{tikzcd}
                        	{\K_0(\Rep_k^{\fin}(Q))} & {\Z^{\oplus Q_0}} \\
                        	& \Z
                        	\arrow["{\rank_k}", from=1-1, to=1-2]
                        	\arrow["{B_Q}", from=1-2, to=2-2]
                        	\arrow["\chi_Q"', from=1-1, to=2-2]
                        \end{tikzcd}
                    $$
                wherein $\chi_Q := \chi_{\Rep_k^{\fin}(Q)}$ and it is understood that:
                    $$B_Q(\rank_k [\calF]) := B_Q(\rank_k [\calF], \rank_k [\calF])$$
                for all isomorphism classes $[\calF] \in \K_0(\Rep_k^{\fin}(Q))$.
            \end{theorem}
                \begin{proof}
                    
                \end{proof}
            
        \subsubsection{(Co)reflection functors and Gabriel's Theorem}
            We have now come to Gabriel's Theorem. The proof thereof relies heavily on constructions known as \say{(co)reflection functors} (cf. definition \ref{def: (co)reflection_functors}) which can be thought of as gadgets that help us ensure that the representation theory of a quiver is independent of its orientation\footnote{So for instance, once we have proven Gabriel's Theorem, we can simply think of Dynkin quivers as those whose underlying undirected graph is a Dynkin diagram, and this notion is completely captured by the Cartan-Tits quadratic forms associated to these undirected graphs.}. 
            \begin{convention}
                From now on, we shall be working over a field $k$. \todo[inline]{Explain why working over a field is necessary.}
            \end{convention}
            \begin{definition}[(Co)sinks] \label{def: (co)sinks}
                
            \end{definition}
            \begin{convention}
                If $Q := (Q_1, Q_0, s, t)$ is a quiver and $\sigma \in Q_0$ is a (co)sink, then we shall implicitly write $(Q, \sigma)^{\op}$ for the quiver wherein one reverses all the arrow into/out of $\sigma$ and keep the remaining arrows unchanged.
            \end{convention}
            \begin{definition}[(Co)reflection functors] \label{def: (co)reflection_functors}
                Let $Q := (Q_1, Q_0, s, t)$ be a quiver and $\sigma \in Q_0$ be a fixed vertex. 
                    \begin{itemize}
                        \item If $v$ is a sink, then one can define the so-called \textbf{reflection functor}:
                            $$\Refl_{Q, \sigma}: \Rep_k(Q) \to \Rep_k(Q)$$
                            $$(\calF: [Q] \to k\mod) \mapsto \left( \Refl_{Q, \sigma}(\calF): [Q] \to k\mod: v \mapsto \begin{cases} \text{$\calF(v)$ if $v \in Q_0 \setminus \{\sigma\}$} \\ \text{$\underset{v \in [Q]_{/\sigma}}{\lim} \calF(v)$ if $v = \sigma$} \end{cases} \right)$$
                        \item If $v$ is a cosink, then one can instead define the \textbf{coreflection functor}:
                            $$\co\Refl_{Q, \sigma}: \Rep_k(Q) \to \Rep_k(Q)$$
                            $$(\calF: [Q] \to k\mod) \mapsto \left( \Refl_{Q, \sigma}(\calF): [Q] \to k\mod: v \mapsto \begin{cases} \text{$\calF(v)$ if $v \in Q_0 \setminus \{\sigma\}$} \\ \text{$\underset{v \in {}^{\sigma/}[Q]}{\colim} \calF(v)$ if $v = \sigma$} \end{cases} \right)$$
                    \end{itemize}
            \end{definition}
            \begin{convention}
                Let $Q := (Q_1, Q_0, s, t)$ be a quiver and $\sigma \in Q_0$ be a fixed vertex. Then, for all $\calF \in \Ob(\Rep_k(Q))$, we shall abuse notations slightly and write:
                    $$\Refl_{Q, \sigma}(\calF) := \underset{v \in [Q]_{/\sigma}}{\lim} \calF(v)$$
                    $$\co\Refl_{Q, \sigma}(\calF) := \underset{v \in {}^{\sigma/}[Q]}{\colim} \calF(v)$$
                when there is no risk of confusion. Keep in mind, over the remarks to come, that this abuse of notations of ours actually does not obscure any important categorical properties of the (co)reflection functors such as exactness.
            \end{convention}
            \begin{remark}[Categorical properties of (co)reflection functors] \label{remark: categorical_properties_of_(co)reflection_functors}
                Immediately, one sees that (co)reflection functors preserve (co)limits of representations. More specifically, if $Q := (Q_1, Q_0, s, t)$ is a quiver and $\sigma \in Q_0$ is a fixed vertex, and if:
                    $$I \to \Rep_k(Q)$$
                    $$i \mapsto \calF_i$$
                is a diagram of $k$-linear representations of $Q$, then clearly:
                    $$\Refl_{Q, \sigma}\left( \underset{i \in I^{\op}}{\lim} \calF_i \right) \cong \underset{i \in I}{\lim} \Refl_{Q, \sigma}(\calF)$$
                and:
                    $$\co\Refl_{Q, \sigma}\left( \underset{i \in I}{\colim} F_i \right) \cong \underset{i \in I}{\colim} \co\Refl_{Q, \sigma}(F)$$
                as a result of (co)limits commuting with one another. Of course, all the (co)limits at play here do indeed exist, since $\Rep_k(Q)$ is both complete and cocomplete as a result of $k\mod$ being so.
            \end{remark}
            \begin{remark}[(Co)reflection functors of finite quivers are exact] \label{remark: (co)reflection_functors_of_finite_quivers_are_exact}
                If:
                    $$Q := (Q_1, Q_0, s, t)$$
                is a \textit{finite} quiver with a fixed vertex $\sigma \in Q_0$ then the corresponding (co)reflection functors:
                    $$\Refl_{Q, \sigma}, \co\Refl_{Q, \sigma}: \Rep_k(Q) \to \Rep_k(Q)$$
                will obviously be defined via finite (co)limits. Because there is an exact equivalence of abelian categories as follows (cf. proposition \ref{prop: quiver_representations_are_modules_over_quiver_algebras}):
                    $$\Rep_k(Q) \to {}^lk\<Q\>\mod$$
                the aforementioned reflection functor $\Refl_{Q, \sigma}$ and coreflection functor $\co\Refl_{Q, \sigma}$ are, respectively, left-exact and right-exact. If, additionally, the diagram $[Q]_{/\sigma}$ is cofiltered (respectively, if ${}^{\sigma/}[Q]$ is filtered), such as when $Q$ is a Dynkin quiver\footnote{Cf. figures \ref{fig: A_n_quiver} and \ref{fig: D_n_quiver}.}, then one will be able to use the fact that (co)filtered limits are right-exact (respectively, filtered colimits are left-exact) in categories of modules over rings in order to see that in such a situation, both $\Refl_{Q, \sigma}$ and $\co\Refl_{Q, \sigma}$ are simultaneously left and right-exact (i.e. exact), and hence will \textit{preserve simple objects} (i.e. irreducible representations).
            \end{remark}
                
            \begin{theorem}[Gabriel's theorem on Dynkin quivers] \label{theorem: gabriel_theorem_on_dynkin_quivers}
                Let $k$ be a field and $\Gamma$ be a connected finite quiver. $\Gamma$ is then a Dynkin quiver if and only if $\Rep_k(\Gamma)$ has finitely many isomorphism classes of indecomposable objects.
            \end{theorem}
                \begin{proof}
                    
                \end{proof}
            \begin{corollary}[Indecomposable representations of finite quivers are labelled by positive roots] \label{coro: indecomposable_representations_of_finite_quivers_are_labelled_by_positive_roots}
                Let $k$ be a field and $\Gamma$ be a Dynkin quiver. Then, there is a bijection:
                    $$\dim_k: |\Rep_k^{\red}(\Gamma)| \to \Phi_{\Gamma}^+$$
                wherein $\Phi_{\Gamma}^+$ denotes the set of positive roots of $\Gamma$ (cf. definition \ref{def: negative_and_positive_roots}).
            \end{corollary}
                \begin{proof}
                    Via theorem \ref{theorem: cartan_tits_quadratic_forms_as_euler_characteristics}, one sees that it suffices to show that:
                        $$\chi_{\Gamma}([\calF]) = 2$$
                    for all isomorphism classes of indecomposable representations $[\calF] \in |\Rep_k^{\red}(\Gamma)|$, as this demonstrates that indeed, the dimension vectors of the irreducible representations of $\Gamma$ are precisely the positive roots of the Cartan-Tits quadratic form $B_{\Gamma}$ (and we have already known that representations of finite quivers are uniquely identified by their dimension vectors; cf. lemma \ref{lemma: representations_of_finite_quivers_are_determined_by_dimension_vectors}).
                \end{proof}
                
            \begin{example}[Gabriel's Theorem for $\sfA_n$ quivers]
                
            \end{example}
            \begin{example}[Gabriel's Theorem for $\sfD_n$ quivers]
                
            \end{example}
            \begin{example}[Gabriel's Theorem for $\sfE_n$ quivers]
                
            \end{example}
            \begin{example}[Gabriel's Theorem fails for the loop quiver]
                
            \end{example}
            \begin{example}[Gabriel's Theorem fails for the Kronecker quiver]
                
            \end{example}
        
    \subsection{Tilting modules and (derived) Morita equivalences}
        \subsubsection{(Partial) tilting objects}
        
        \subsubsection{Separating and splitting tilting objects}
        
        \subsubsection{Torsion induced by tilting objects}
        
    \subsection{Hereditary rings}    
        \subsubsection{Hereditary algebras and hereditary categories}
            \begin{definition}[Hereditary abelian categories] \label{def: hereditary_abelian_categories}
                An abelian category $\calA$ is said to be \textbf{hereditary}\footnote{If only the full subcategory $\calA^{\fin}$ of finite-length objects is of global dimension $\leq 1$ (i.e. if $\calA^{\fin}$ is hereditary) then one says that the larger abelian category $\calA$ is \textbf{semi-hereditary}.} if and only if $\globdim \calA \leq 1$.
            \end{definition}
            \begin{convention}[Left/right-hereditary rings] \label{conv: left/right_hereditary_rings}
                When $\calA$ is the category of left/right-module over some ring $R$, one might say that the ring $R$ itself is \textbf{left/right-hereditary}\footnote{A ring is left/right semi-hereditary if and only if all finitely generated modules over it are of projective dimension $\leq 1$.} whenever $\calA$ itself is hereditary.
            \end{convention}
            \begin{proposition}[Kernels in hereditary abelian categories are projective] \label{prop: kernels_in_hereditary_abelian_categories_are_projective}
                Let $\calA$ be a hereditary abelian category. Then, every kernel (i.e. sub-object) therein will be projective.
            \end{proposition}
                \begin{proof}
                    
                \end{proof}
            \begin{corollary}[Kaplansky's Theorem] \label{coro: kaplansky_theorem}
                If $R$ is a left/right-hereditary ring then every submodule of a free left/right-$R$-module will be a direct sum of left/right-$R$-ideals. As a direct result of this, every submodule of a projective left/right-$R$-module is also projective\footnote{Hence the terminology \say{hereditary}.}.
            \end{corollary}
            \begin{example}
                Any left/right-semi-simple category (hence any semi-simple algebra) is left/right-hereditary. Concrete examples of hereditary algebras include finite-dimensional matrix rings over division rings (cf. theorem \ref{theorem: artin_wedderburn}) and group algebras $k\<G\>$ of finite groups $G$ over semi-simple rings $k$ such that $\chara k \nmid |G|$ (cf. theorem \ref{theorem: maschke_theorem}).
                
                As for algebras which are strictly left/right-hereditary and not left/right-semi-simple, consider the algebra $\b_2^-(k)$ of lower-triangular $2 \x 2$ matrices over a field $k$.
            \end{example}
            
        \subsubsection{Quivers associated to finite-dimensional algebra}
            \begin{definition}[Basic algebras] \label{def: basic_algebras}
                An associative algebra $A$ over some commutative ring $k$ is said to be \textbf{basic} if and only if the semi-simple\footnote{Cf. proposition \ref{prop: semi_simple_iff_trivial_jacobson_radical_and_artinian}} $k$-algebra $A/\rad(A)$ admits an Artin-Wedderburn Decomposition (cf. theorem \ref{theorem: artin_wedderburn}) into finitely many copies of $k$, i.e.:
                    $$A/\rad(A) \cong \prod_{i = 1}^d k$$
                wherein the product is taken in the category of associative $k$-algebras.
            \end{definition}
            \begin{remark}
                Obviously, if $A$ is a basic algebra over some commutative ring $k$ then $A$ will be of finite rank as a $k$-module, which is equal to the number of copies of $k$ in the Artin-Wedderburn Decomposition of $A/\rad(A)$; that is to say, if:
                    $$A/\rad(A) \cong \prod_{i = 1}^d k$$
                then:
                    $$\rank_k A = d$$
            \end{remark}
            \begin{remark}[Simple modules over basic algebras are $1$-dimensional] \label{remark: simple_modules_over_basic_algebras_are_one_dimensional}
                Let $k$ be a commutative ring and $A$ be a basic $k$-algebra. By the Artin-Wedderburn Theorem (cf. theorem \ref{theorem: artin_wedderburn}), one sees that every simple (left/right-)$A$-modules are of rank $1$ as $k$-modules. Among other things, this means that in order to compute $\rad(A)$, which by definition contains all elements $a \in A$ which act as $0$ on simple (left/right-)$A$-modules, one needs to only check whether or not said elements act as $0$ on the underlying ring $k$.
            \end{remark}
            \begin{definition}[Primitive idempotents] \label{def: primitive_idempotents}
                An idempotent ring element $e \in R$ (i.e. $e^2 = e$) is said to be \textbf{primitive} if and only if the left-$R$-ideal ${}_R\<e\>$ ((or equivalently, the right-$R$-ideal $\<e\>_R$) is an indecomposable left/right-$R$-module.
            \end{definition}
            \begin{proposition}[Primitive idempotents are indecomposable] \label{prop: primitive_idempotents_are_indecomposable}
                \cite[Proposition 21.8]{lam_first_course_in_noncommutative_rings} Let $R$ be an associative ring and $e \in R$ be an idempotent element therein. Then the following are equivalent:
                    \begin{enumerate}
                        \item $e$ is primitive.
                        \item The ring $eRe$ has no non idempotent elements aside from $0$ and $1$.
                        \item There does not exist a decomposition $e := \alpha + \beta$ of $e$ into non-zero orthogonal (i.e. $\alpha \beta = \beta \alpha = 0$) idempotents $\alpha, \beta \in R$.
                    \end{enumerate}
            \end{proposition}
            \begin{example}
                Consider the matrix ring $\Mat_2(k)$ over some field $k$ of characteristic $0$. The element $\begin{pmatrix} 1 & 0 \\ 0 & 0 \end{pmatrix}$ is first of all idempotent, but moreoever, primitive, as the ring $\begin{pmatrix} 1 & 0 \\ 0 & 0 \end{pmatrix} \Mat_2(k) \begin{pmatrix} 1 & 0 \\ 0 & 0 \end{pmatrix}$ has no idempotents aside from the additive and multiplicative identities: one can easily show via some obvious computations that:
                    $$\begin{pmatrix} 1 & 0 \\ 0 & 0 \end{pmatrix} \Mat_2(k) \begin{pmatrix} 1 & 0 \\ 0 & 0 \end{pmatrix} = \left\{ \begin{pmatrix} a & 0 \\ 0 & 0 \end{pmatrix} \: \bigg| \: \forall a \in k: a^2 = a\right\}$$
                but since $k$ is a field of characteristic $0$, there are no elements $a \in k$ such that $a^2 = a$ aside from $0$ and $1$. On the other hand, the identity matrix $\begin{pmatrix} 1 & 0 \\ 0 & 1 \end{pmatrix}$ (which is also idempotent) is \textit{not} primitive, since one can easily write:
                    $$\begin{pmatrix} 1 & 0 \\ 0 & 1 \end{pmatrix} = \begin{pmatrix} 1 & 0 \\ 0 & 0 \end{pmatrix} + \begin{pmatrix} 0 & 0 \\ 0 & 1 \end{pmatrix}$$
                and it is trivial to check that the two summands are idempotent and orthogonal to one another.
            \end{example}
            \begin{remark}[Idempotents split] \label{remark: idempotents_split}
                
            \end{remark}
            \begin{definition}[Associated basic algebras] \label{def: associative_basic_algebras}
                Let $A$ be a finite algebra over a commutative ring $k$. To such a $k$-algebra, one can construct a basic one, denoted by ${}^bA$, which is given by:
                    $${}^bA := e_A A e_A$$
                wherein:
                    $$e_A := \sum_{e \in E} e$$
                for some set $E$ of pair-wise orthogonal\footnote{For all $e, e' \in E$, one has $ee' = e'e = 0$ whenever $e \not = e'$} primitive idempotents such that $|E| \leq \rank A$ and for all $e, e' \in E$, one has ${}_A\<e\> \not \cong {}_A\<e'\>$ (or $\<e\>_A \not \cong \<e'\>_A$) if $e \not = e'$. For the sake of brevity, let us call $E$ a \textbf{over-basis} of $A$ as a $k$-module.
            \end{definition}
            \begin{example}
                Let $k$ be a field of characteristic $0$ and consider the $k$-algebra $\Mat_2(k)$ of $2 \x 2$ matrices with entries in $k$; within this algebra, consider the over-basis\footnote{We will let the reader check that the elements herein are pair-wise orthogonal idempotents.}:
                    $$E := \left\{ \begin{pmatrix} 0 & 0 \\ 0 & 0 \end{pmatrix}, \begin{pmatrix} 1 & 0 \\ 0 & 0 \end{pmatrix}, \begin{pmatrix} 0 & 0 \\ 0 & 1 \end{pmatrix}, \begin{pmatrix} 1 & 0 \\ 0 & 1 \end{pmatrix} \right\}$$
                
            \end{example}
            
            \begin{definition}
                A quiver is said to have a ring-theoretic property $\calP$ over some base ring $k$ if and only if its quiver $k$-algebra has property $\calP$.
            \end{definition}
            \begin{example}[Quivers of finite type] \label{example: quivers_of_finite_type}
                A quiver $Q$ is said to be of finite type over some ring $k$ if and only if its path $k$-algebra $k\<Q\>$ is of finite type as a $k$-algebra.
            \end{example}
            \begin{proposition}[Quivers of finite type are of finite presentations] \label{prop: quivers_of_finite_type_are_of_finite_presentations}
                Let $k$ be a ring and $Q$ be a quiver. Then $Q$ is of finite type if and only if it is of finite presentation over $k$.
            \end{proposition}
                \begin{proof}
                    If $k\<Q\>$ is of finite presentation then obviously it is of fintie type. Conversely, if $k\<Q\>$ is of finite presentation, then by definition the set $[Q]_1$ of morphisms of the free category $[Q]$ on $Q$ will have to be finite. These finitely many morphisms will generate finitely many relations on the noncommutative free $k$-algebra $k\<\{v_i\}_{1 \leq i \leq |Q_1|}\>$ (which we note to be of finite type over $k$), thereby making the algebra $k\<Q\>$ finitely presented over $k$.
                \end{proof}
            \begin{definition}[Simply laced quivers] \label{def: simply_laced_quivers}
                A $\Sets$-valued quiver $Q := (Q_1, Q_0, s, t)$ is \textbf{simply laced} if and only if none of its vertices has self-loops: that is, for all $v \in Q_0$, one has:
                    $$\{f \in Q_1 \mid s(f) = t(f) = v\} = \{\id_v\}$$
            \end{definition}
            \begin{remark}
                Equivalently, one might characterise the simply laced quivers $Q := (Q_1, Q_0, s, t)$ as those wherein all non-identity paths $f \in Q_1$ generate square-free elements $f \in k\<Q\>$ (i.e. elements such that $f^2 = 0$). This is because the lack of non-identity loops implies that there is no non-identity path that can be composed with itself. 
            \end{remark}
            \begin{definition}[Connected quivers] \label{def: connected_quivers}
                A $\Sets$-valued quiver $Q: \{\1, \0, s, t\}^{\op} \to \Sets$ is said to be \textbf{connected} if and only if for all $v, w \in Q(\0)$, there exists $f \in Q(\1)$ such that $s(f) = v$ and $t(f) = w$ (i.e. it has no isolated vertices).
            \end{definition}
            \begin{remark}
                Equivalently, one can say that a $\Sets$-valued (or maybe with values in any category $\C$ with enough coproducts and sub-objects) quiver is connected if and only if it can not be written as the disjoint union of two sub-quivers. 
            \end{remark}
            
        \subsubsection{Revisiting Gabriel's Theorem}
            
            \section{Derived categories of coherent sheaves and quivers}
            
        \chapter{Quiver varieties and affine Lie algebras}
            \begin{abstract}
                
            \end{abstract}
            
            \minitoc
            
            \section{Nakajima quiver varieties}
    \subsection{\textit{Pr\'elude}: \texorpdfstring{$\Quot$}{} and \texorpdfstring{$\Hilb$}{} schemes}

    \subsection{Moduli spaces of quiver representations}
        \begin{convention}
            Throughout this subsection, we work over a ground field $k$. All schemes, unless stated to be otherwise, shall be over $k$.
        \end{convention}
        \begin{convention}
            If $K$ is a field then by $\bar{K}$ we will usually mean the algebraic closure of $K$. When there are reasons to confuse $\bar{K}$ for the separable closure of $K$, then we shall write $K^{\alg}$ and $K^{\sep}$ instead. 
        \end{convention}
        
        \subsubsection{GIT quotients parametrising quiver representations}
            \begin{convention}
                Let $Q := (Q_1, Q_0, s, t)$ be a finite quiver and fix some vector $v \in \N^{Q_0}$. Then, by $\Rep_k^v(Q)$ we shall mean the set of \textit{all} (not only isomorphism classes) $k$-linear representations $\calF \in \Ob(\Rep_k(Q))$ such that:
                    $$\dim_k \calF = v$$
            \end{convention}
            \begin{proposition}[Dimensions of representation spaces of finite quivers] \label{prop: dimensions_of_representation_spaces_of_finite_quivers}
                Let $Q := (Q_1, Q_0, s, t)$ be a finite quiver and fix some vector $v \in \N^{Q_0}$. Then, $\Rep_k^v(Q)$ has a natural structure of a $k$-vector space; furthermore, we have:
                    $$\dim_k \Rep_k^v(Q) = v^{\top} R_Q v$$
                wherein $R_Q$ is the adjacency matrix of $Q$ (cf. definition \ref{def: cartan_quadratic_forms_of_finite_quivers}).
            \end{proposition}
                \begin{proof}
                            
                \end{proof}
            \begin{corollary}[Action of general linear groups on spaces of quiver representations] \label{coro: general_linear_group_action_on_quiver_representations}
                Let $Q := (Q_1, Q_0, s, t)$ be a finite quiver and fix some vector $v \in \N^{Q_0}$. Then, there is a natural of the group $\GL_v(k) := \prod_{i \in Q_0} \GL_{v_i}(k)$ on $\Rep_k^v(Q)$ via conjugations. 
            \end{corollary}
                \begin{proof}
                    
                \end{proof}
            \begin{remark}[Action of general linear groups on spaces of quiver representations] \label{remark: general_linear_group_action_on_quiver_representations}
                Let $Q := (Q_1, Q_0, s, t)$ be a finite quiver and fix some vector $v \in \N^{Q_0}$. Then, observe that because the (central\footnote{Observe that $\rmZ(\GL_v(k)) = \GL_1(k)^{Q_0}$}) subgroup $\GL_1(k) \leq \GL_1(k)^{Q_0} \leq \GL_v(k)$ acts via conjugations on $\Rep_k^v(Q)$ by non-zero $k$-scalar multiples of the identity matrix, it acts trivially on $\Rep_k^v(Q)$. As such, one obtains an induced action of $\PGL_v(k)$ on $\Rep_k^v(Q)$ via conjugations. Observe for later, that both $\GL_v(k)$ and $\PGL_v(k)$ are the groups of $k$-points of geometrically reductive group $k$-schemes (namely $\GL_v$ and $\PGL_v$)
            \end{remark}
            \begin{remark}[Moduli spaces of quiver representations of given dimensions] \label{remark: moduli_spaces_of_quiver_representations_of_given_dimensions} 
                Let $Q := (Q_1, Q_0, s, t)$ be a finite quiver and fix some vector $v \in \N^{Q_0}$. In light of remark \ref{remark: general_linear_group_action_on_quiver_representations}, we can now construct a moduli space of isomorphism classes $[\calF] \in \Rep_k^v(Q)$ in the following manner: since $\Rep_k^v(Q)$ is a finite-dimensional $k$-vector space, it can be regarded as a finite-rank vector bundle over $\Spec k$, namely:
                    $$\calN_{Q, v} := \Spec\left( \Sym_k \Rep_k^v(Q) \right)$$
                In fact, one readily sees that $\calN_{Q, v}$ is an affine scheme that is finite over $\Spec k$. Also, observe that wse have isomorphisms:
                    $$\Sym_k \Rep_k^v(Q) \cong \Hom_k(\Rep_k^v(Q), k)$$
                so the functor-of-points:
                    $$\calN_{Q, v}: k\-\Comm\Alg \to \Sets$$
                is given by:
                    $$\calN_{Q, v}(-) \cong \Hom_k(\Rep_k^v(Q), -)$$
            \end{remark}
            Remark \ref{remark: moduli_spaces_of_quiver_representations_of_given_dimensions} can be formalised in the following manner:
            \begin{proposition}[]
                Let $Q := (Q_1, Q_0, s, t)$ be a finite quiver and fix some vector $v \in \N^{Q_0}$. If $k$ is algebraically closed then there will be a bijection:
                    $$\overline{\calN_{Q, v}}(k) \cong [\Rep_k^v(Q)]$$
            \end{proposition}
                \begin{proof}
                    
                \end{proof}
            
            Let us now revisit Gabriel's Theorem (cf. theorems \ref{theorem: gabriel_theorem_dynkin_implies_representation_finite} and \ref{theorem: gabriel_theorem_representation_finite_implies_dynkin}). Here, we would like to investigate 
            \begin{definition}[Orbit-finite quivers] \label{def: orbit_finite_quivers}
                Let $Q := (Q_1, Q_0, s, t)$ be a finite quiver. Such a quiver is said to be ($k$-linearly) \textbf{orbit-finite} if and only if for all vectors $v \in \N^{Q_0}$, the GIT quotient $\calN_{Q, v} /\!/ \GL_v$ is finite over $\Spec k$.
            \end{definition}
            \begin{remark}
                Let $Q := (Q_1, Q_0, s, t)$ be a finite quiver. By \'etale descent (cf. \cite[\href{https://stacks.math.columbia.edu/tag/04DH}{Tag 04DH}]{stacks}), one sees that $Q$ is equivalently orbit-finite if and only if for all finite extensions $k'/k$, there are only finitely many $\GL_v(k')$-orbits inside 
            \end{remark}
            \begin{definition}[Representation-finite quivers] \label{def: representation_finite_quivers}
                Let $Q := (Q_1, Q_0, s, t)$ be a finite quiver. It is said to be ($k$-linearly) \textbf{representation-finite} if and only if $\Rep_k(Q)$ has only finitely many isomorphism classes of indecomposable objects. 
            \end{definition}
            \begin{theorem}[Gabriel's theorem (representation-finite implies Dynkin)] \label{theorem: geometric_gabriel_theorem_representation_finite_implies_dynkin}
                Let $k$ be a field and $\Gamma$ be a connected and acyclic finite quiver. $\Gamma$ is then a Dynkin\footnote{Cf. definition \ref{def: dynkin_quivers}.} quiver if it $k$-linearly representation-finite.
            \end{theorem}
                \begin{proof}
                    Following definition \ref{def: dynkin_quivers}, we shall seek to show that the Tits quadratic form $q_{\Gamma}$ (cf. definition \ref{def: tits_quadratic_forms}) is positive-definite, i.e. $q_{\Gamma}(v) \geq 0$ for all $v \in \N^{\Gamma_0}$ with equality occurring if and only if $v = 0$. To this end, observe first of all that:
                        $$q_{\Gamma}(v) = \sum_{1 \leq i \leq |\Gamma_0|} v_i^2 - \sum_{1 \leq i \leq j \leq |\Gamma_0|} v_i v_j = \dim \GL_v - \dim \calN_{\Gamma, v}$$
                \end{proof}
        
            \begin{theorem}[Gabriel's theorem (Dynkin implies representation-finite)] \label{theorem: geometric_gabriel_theorem_dynkin_implies_representation_finite}
                Let $k$ be a field and $\Gamma$ be a Dynkin quiver. Then $\Gamma$ will have finitely many (finite-dimensional) indecomposable $k$-linear representations; in particular, there is a bijection:
                    $$\dim_k: [\Rep_k^{\red}(\Gamma)] \to \Phi_{\Gamma}^+$$
                between the set of isomorphism classes of indecomposable $k$-linear representations of $\Gamma$ and that of positive roots\footnote{Cf. definition \ref{def: negative_and_positive_roots}.} of $\Gamma$.
            \end{theorem}
                \begin{proof}
                    
                \end{proof}
                
            \begin{example}[Gabriel's Theorem for $\sfA_n$ quivers]
                
            \end{example}
            \begin{example}[Gabriel's Theorem for $\sfD_n$ quivers]
                
            \end{example}
            \begin{example}[Gabriel's Theorem for $\sfE_n$ quivers]
                
            \end{example}
            \begin{example}[Gabriel's Theorem fails for the loop quiver]
                
            \end{example}
            \begin{example}[Gabriel's Theorem fails for the Kronecker quiver]
                
            \end{example}
            
        \subsubsection{(Semi-)stability of quiver representations}
    
        \subsubsection{Framings}
    
        \subsubsection{Moment maps, symplectic singularities, and Hamiltonian reductions}
            
            \section{The Geometric McKay Correspondence}
            
            \section{Representations of Kac-Moody algebras via (co)homology of quiver varieties}
            
        \begin{appendices}
            \chapter{Reductive groups}
                \begin{abstract}
                    
                \end{abstract}
                
                \minitoc
                
                \section{Algebraic groups}
    \subsection{Group schemes}
        \subsubsection{Definition and general properties}
            \begin{definition}[Group schemes] \label{def: group_schemes}
                A \textbf{group scheme} over a given scheme $S$ is a group object in the category of $S$-schemes. Note that this is a well-defined notion, as the category of $S$-schemes has all finite products.
                
                In more details, a group scheme over $S$ is an $S$-scheme $G$ equipped with morphisms $m: G \x_S G \to G$, $e: S \to G$, and $i: G \to G$ such that the following diagrams\footnote{The isomorphism in the second diagram is the canonical one and the unnamed arrow in the third diagram is the structural morphism of $G$ over $S$.} commute:
                    $$
                        \begin{tikzcd}
                        	{G \x_S G \x_S G} & {G \x_S G} \\
                        	{G \x_S G} & G
                        	\arrow["{\id_{G/S} \x_S m}"', from=1-1, to=2-1]
                        	\arrow["m", from=2-1, to=2-2]
                        	\arrow["{m \x_S \id_{G/S}}", from=1-1, to=1-2]
                        	\arrow["m", from=1-2, to=2-2]
                        \end{tikzcd}
                    $$
                    $$
                        \begin{tikzcd}
                        	{G \x_S S} && {S \x_S G} \\
                        	{G \x_S G} && {G \x_S G} \\
                        	& G
                        	\arrow["m"', from=2-1, to=3-2]
                        	\arrow["{e \x_S \id_{G/S}}", from=1-3, to=2-3]
                        	\arrow["m", from=2-3, to=3-2]
                        	\arrow["{\id_{G/S} \x_S e}"', from=1-1, to=2-1]
                        	\arrow["\cong", from=1-1, to=1-3]
                        \end{tikzcd}
                    $$
                    $$
                        \begin{tikzcd}
                        	& G \\
                        	{G \x_S G} && {G \x_S G} \\
                        	& S \\
                        	{G \x_S G} && {G \x_S G} \\
                        	& G
                        	\arrow["m", from=4-3, to=5-2]
                        	\arrow["m"', from=4-1, to=5-2]
                        	\arrow["{\id_{G/S} \x_S i}"', from=2-1, to=4-1]
                        	\arrow["{i \x_S \id_{G/S}}", from=2-3, to=4-3]
                        	\arrow["{\Delta_{G/S}}"', from=1-2, to=2-1]
                        	\arrow["{\Delta_{G/S}}", from=1-2, to=2-3]
                        	\arrow["e", from=3-2, to=5-2]
                        	\arrow[from=1-2, to=3-2]
                        \end{tikzcd}
                    $$
                As is the case in general categories with enough pullbacks, there is a (non-full) subcategory $\Grp\Sch_{/S} \subset \Sch_{/S}$ spanned by group $S$-schemes and homomorphisms between them, i.e. morphisms $\varphi: (H, m_H, e_H, i_H) \to (G, m_G, e_G, i_G)$ such that the following diagram commutes:
                    $$
                        \begin{tikzcd}
                        	{H \x_S H} & {G \x_S G} & {} \\
                        	H & G
                        	\arrow["{m_H}"', from=1-1, to=2-1]
                        	\arrow["{m_G}", from=1-2, to=2-2]
                        	\arrow["\varphi", from=2-1, to=2-2]
                        	\arrow["{\varphi \x_S \varphi}", from=1-1, to=1-2]
                        \end{tikzcd}
                    $$
            \end{definition}
            \begin{remark}[Group scheme homomorphisms are group homomorphisms]
                Again, as in the case in general categories with enough pullbacks, group scheme homomorphisms preserve identities and inverses. This is easy to check.
            \end{remark}
            \begin{remark}[Pullbacks of group schemes] \label{remark: pullbacks_of_group_schemes}
                If $f: S' \to S$ is a morphism of schemes and $G$ is a group scheme over $S$, then the pullback $G \x_{S, f} S'$ will be group scheme over $S'$. This is an easy consequence of the fact that limits commute.
            \end{remark}
            \begin{remark}[Open and closed subgroup schemes] \label{remark: open_and_closed_subgroup_schemes}
                Because limits commute, should $G$ be a group scheme over a given base scheme $S$ and $\iota: H \hookrightarrow G$ be a monomorphism of $S$-schemes (such as open immersions or closed immersions), then $H$ should be a subgroup $S$-scheme of $G$. One thing that is worth checking is that the identity map $e_G: S \to G$ is a monomorphism and therefore the trivial group $S$-scheme $S$ is a subgroup $S$-scheme of $G$.   
            \end{remark}
            \begin{definition}[Actions of group schemes] \label{def: actions_of_group_schemes}
                Let $S$ be a scheme, $X$ be an $S$-scheme, and $G$ a group $S$-scheme. An \textbf{left-action}\footnote{Right-actions are defined analogously. We let our dear reader figure this out for themselves.} of $G$ on $X$ over $S$ is thus a morphism of $S$-schemes $\alpha: G \x_S X \to X$ such that the following diagrams commute:
                    $$
                        \begin{tikzcd}
                        	{G \x_S G \x_S X} & {G \x_S X} \\
                        	{G \x_S X} & X
                        	\arrow["{m \x_S \id_{X/S}}"', from=1-1, to=2-1]
                        	\arrow["\alpha", from=2-1, to=2-2]
                        	\arrow["{\id_{G/S} \x_S \alpha}", from=1-1, to=1-2]
                        	\arrow["\alpha", from=1-2, to=2-2]
                        \end{tikzcd}
                    $$
                    $$
                        \begin{tikzcd}
                        	X \\
                        	{G \x_S X} & X
                        	\arrow["\alpha", from=2-1, to=2-2]
                        	\arrow["{e \x_S \id_{X/S}}"', from=1-1, to=2-1]
                        	\arrow["{\id_{X/S}}", from=1-1, to=2-2]
                        \end{tikzcd}
                    $$
            \end{definition}
            \begin{definition}[Action groupoids] \label{def: action_groupoids}
                Let $S$ be a scheme and let $G$ be a group $S$-scheme which acts on an $S$-scheme $X$ via $\alpha: G \x_S X \to X$. Then, the \textbf{action groupoid} of the $G$-action $\alpha$ shall be the span $\alpha, \pr_2: G \x_S X \toto X$. 
            \end{definition}
            \begin{remark}[Action groupoids are internal groupoids]
                Definition \ref{def: action_groupoids}, as it stands, does not give us a legitimate groupoid internal to the category of $S$-schemes. This, however, is not hard to check: for any group $S$-scheme $(G, m, e, i)$ acting on an $S$-scheme $X$ via $\alpha: G \x_S X \to X$, simply take the inversion map to be the morphism of $S$-schemes $i \x_S \id_{X/S}: G \x_S X \to G \x_S X$.
            \end{remark}
            \begin{convention}[Action groupoids of group schemes] \label{conv: action_groupoids_of_group_schemes}
                Usually, the action groupoid of a given group $S$-scheme $G$ is taken to be the one wherein the action $\alpha: G \x_S G \to G$ is the multiplication map.
            \end{convention}
            
            Before we can discuss properties of group schemes, let us take a brief detour and discuss certain relevant properties of diagonals of (morphisms of) schemes, particularly how they pertain to separatedness.
            \begin{definition}[(Quasi-)separatedness] \label{def: (quasi)_separatedness}
                A morphism $f: X \to S$ of schemes is said to be \textbf{separated} (respectively, \textbf{quasi-separated}) if and only if its diagonal $\Delta_{X/S}$ is closed (respectively, quasi-compact). 
            \end{definition}
            \begin{remark}
                Obviously, being separated implies being quasi-separated.
            \end{remark}
            \begin{lemma}[Diagonals of affines are closed] \label{lemma: diagonals_of_affines_are_closed}
                The diagonal of an affine morphism is closed.
            \end{lemma}
                \begin{proof}
                    
                \end{proof}
            \begin{proposition}[Diagonals are locally closed immersions] \label{prop: diagonals_of_schemes_are_locally_closed_immersions}
                The diagonal of any morphism of schemes is a locally closed immersion.
            \end{proposition}
                \begin{proof}
                    
                \end{proof}
            \begin{corollary}[Affines are separated] \label{coro: affines_are_separated}
                Affine schemes are always separated. More generally, affine morphisms are separated. 
            \end{corollary}
            \begin{example}[A scheme that is not quasi-separated] \label{example: a_scheme_that_is_not_quasi_separated}
                Let $k$ be a field and let $X$ be the $k$-scheme given by gluing two copies of $\Spec k[x_1, x_2, ...]$ along the complement of the closed subscheme $\Spec k \cong \Spec k[x_1, x_2, ...]/(x_1, x_2, ...)$ inside $\Spec k[x_1, x_2, ...]$, i.e. as the following canonical pushout of $k$-schemes:
                    $$
                        \begin{tikzcd}
                        	{\Spec k[x_1, x_2, ...] \setminus \Spec k} & {\Spec k[x_1, x_2, ...]} \\
                        	{\Spec k[x_1, x_2, ...]} & X
                        	\arrow[from=1-1, to=2-1]
                        	\arrow[from=1-1, to=1-2]
                        	\arrow[from=2-1, to=2-2]
                        	\arrow[from=1-2, to=2-2]
                        	\arrow["\lrcorner"{anchor=center, pos=0.125, rotate=180}, draw=none, from=2-2, to=1-1]
                        \end{tikzcd}
                    $$
                We thus see that the topological preimage of $\Spec k[x_1, x_2, ...] \x_{\Spec k} \Spec k[x_1, x_2, ...]$ under the diagonal morphism $\Delta_{\Spec k[x_1, x_2, ...]/\Spec k}$ is the complement $\Spec k[x_1, x_2, ...] \setminus \Spec k$, which is very clearly not quasi-compact.
            \end{example}
            \begin{proposition}[Permanence of (quasi-)separatedness] \label{prop: permanence_of_quasi_separatedness}
                \noindent
                \begin{enumerate}
                    \item (Quasi-)separatedness is preserved by compositions. In fact, (quasi-)separatedness the 2-out-of-3 property, i.e. for any given commutative triangle of schemes as follows:
                        $$
                            \begin{tikzcd}
                            	X & Y \\
                            	& Z
                            	\arrow["f", from=1-1, to=1-2]
                            	\arrow["g", from=1-2, to=2-2]
                            	\arrow["h"', from=1-1, to=2-2]
                            \end{tikzcd}
                        $$
                    if any two of the three arrows are (quasi-)separated morphisms then the remaining one will also be a (quasi-)separated morphism.
                    \item (Quasi-)separatedness is preserved by base-changes.
                \end{enumerate}
            \end{proposition}
                \begin{proof}
                    
                \end{proof}
            \begin{lemma}[A topological criterion for (quasi-)separatedness] \label{lemma: topological_criterion_for_(quasi)_separatedness}
                
            \end{lemma}
                \begin{proof}
                    
                \end{proof}
            \begin{lemma}[An algebraic criterion for (quasi-)separatedness] \label{lemma: algebraic_criterion_for_(quasi)_separatedness}
                
            \end{lemma}
                \begin{proof}
                    
                \end{proof}
            \begin{proposition}[A scheme-theoretic criterion for (quasi)-separatedness] \label{prop: scheme_theoretic_criterion_for_(quasi)_separatedness}
                Let $S$ be a scheme, let $X, Y$ be $S$-schemes, and let $f: T \to S$ be a morphism of schemes. Then, the canonically induced morphism $X \x_T Y \to X \x_S Y$ is an immersion which will be closed (respectively, quasi-compact) if $f: T \to S$ is separated (respectively, quasi-separated).  
            \end{proposition}
                \begin{proof}
                    
                \end{proof}
            \begin{corollary}[Sections are immersions] \label{coro: sections_are_immersions}
                Let $f: X \to S$ be a morphism of schemes and $s: S \to X$ be a section thereof, i.e. a morphism such that $f \circ s = \id_S$. Then, $s: S \to X$ shall be an immersion that is closed (respectively, quasi-compact) when $f: X \to S$ is separated (respectively, quasi-separated).
            \end{corollary}
            
            We are now ready to establish a criterion for a given group scheme to be (quasi-)separated.
            \begin{proposition}[A criterion for (quasi-)separatedness for group schemes] \label{prop: (quasi)_separatedness_criterion_for_group_schemes}
                Let $S$ be a scheme and $(G, m, e, i)$ be a group $S$-scheme. Then, $G$ is separated (respectively, quasi-separated) over $S$ if and only if the identity $e: S \to G$ is a closed immersion (respectively, quasi-compact).
            \end{proposition}
                \begin{proof}
                    Suppose first of all that the identity morphism $e: S \to G$ is a closed immersion (respectively, quasi-compact) and recall that by defintion, a scheme is separated (respectively, quasi-separated) if and only if its diagonal is a closed immersion (respectively, quasi-compact), meaning that we shall have to show that $\Delta_{G/S}: G \to G \x_S G$ is a closed immersion (respectively, quasi-compact). To that end, consider the following diagram (wherein the unnamed arrow is the structural morphism defining $G$ as an $S$-scheme):
                        $$
                            \begin{tikzcd}
                            	G & {G \x_S G} \\
                            	S & G
                            	\arrow["{\Delta_{G/S}}", from=1-1, to=1-2]
                            	\arrow["e", from=2-1, to=2-2]
                            	\arrow[from=1-1, to=2-1]
                            	\arrow["{m \circ (i \x_S \id_{G/S})}", from=1-2, to=2-2]
                            \end{tikzcd}
                        $$
                    It is not hard to check that this is a pullback square in the category of $S$-schemes, and since closed immersions (respecitvely, quasi-compactness) are preserved by pullbacks (cf. \cite[\href{https://stacks.math.columbia.edu/tag/01JY}{Tag 01JY}]{stacks} and respectively, \cite[\href{https://stacks.math.columbia.edu/tag/01K5}{Tag 01K5}]{stacks}), $e: S \to G$ being a closed immersion implies that $\Delta_{G/S}: G \to G \x_S G$ is also a closed immersion (respectively, quasi-compact). By definition, this means that $G$ is separated (respectively, quasi-separated) over $S$.
                    
                    Conversely, suppose that $G$ is separated (respectively, quasi-separated) over $S$. Then, note that because $e: S \to G$ is, by definition, a section of the structural morphism $G \to S$ that defines $G$ as an $S$-scheme, one can apply corollary \ref{coro: sections_are_immersions} directly to see that $e: S \to G$ must be a closed immersion (respectively, quasi-compact).
                \end{proof}
        
        \subsubsection{Group schemes over fields and algebraic groups}
            \begin{convention}
                Henceforth, we work over a fixed field $k$.
            \end{convention}
            
            \begin{proposition}[Group schemes over fields are separated] \label{prop: group_schemes_over_fields_are_separated}
                Any group scheme $(G, m, e, i)$ over $\Spec k$ is separated. 
            \end{proposition}
                \begin{proof}
                    From proposition \ref{prop: (quasi)_separatedness_criterion_for_group_schemes}, we know that $G$ is separated over $\Spec k$ if and only if the identity morphism $e: \Spec k \to G$ is a closed immersion, but this is self-evident.
                \end{proof}
            
            \begin{proposition}[Multiplication maps of group schemes over fields are open] \label{prop: multiplication_maps_of_group_schemes_over_fields_are_open}
                Suppose that $(G, m, e, i)$ is a group scheme over $\Spec k$. Then, the multiplication map $m: G \x_{\Spec k} G \to G$ is open.
            \end{proposition}
                \begin{proof}
                    
                \end{proof}
            
            \begin{lemma}[Irreducibility, quasi-compactness, and connectedness] \label{lemma: irreducibility_quasi_compactness_connectedness_of_group_schemes_over_fields}
                For group schemes over fields, connectedness implies irreducibility, which in turn implies quasi-compactness. 
            \end{lemma}
                \begin{proof}
                    
                \end{proof} 
            \begin{proposition}[Existence, uniqueness, and geometric irreducibility of connected components of the identity] \label{prop: existence_of_identity_components_of_group_schemes_over_fields}
                Let $G$ be a group scheme over $\Spec k$. Then:
                    \begin{enumerate}
                        \item at all points $g \in |G|$, the corresponding stalk of the structure sheaf $\calO_{G, g}$ has a unique minimal prime ideal, and
                        \item there is a unique geometrically irreducible connected $k$-subscheme $G^{\circ} \subseteq G$ such that $e \in |G^{\circ}|$, called the \textbf{connected component of the identity} or simply the \textbf{identity component}.
                    \end{enumerate}
            \end{proposition}
                \begin{proof}
                    \noindent
                    \begin{enumerate}
                        \item 
                        \item 
                    \end{enumerate}
                \end{proof}
            \begin{corollary}[Identity components are quasi-compact over fields]
                Let $G$ be group scheme over $\Spec k$. Then the connected component of the identity $G^{\circ}$, by virtue of being connected, is quasi-compact. 
            \end{corollary}
                
            \begin{lemma}[Nilradicals of tensor products] \label{lemma: nilradicals_of_tensor_products}
                Let $k$ be a field and let $R, S$ be commutative $k$-algebras. Then, ${}^{\red}(R \tensor_k S) \cong {}^{\red}R \tensor_k {}^{\red}S$.
            \end{lemma}
                \begin{proof}
                    
                \end{proof}
            \begin{proposition}[Associated reduced group scheme] \label{prop: associated_reduced_group_scheme}
                Let $G$ be a group scheme over $\Spec k$. Then the associated reduced scheme ${}^{\red}G$ is a closed subgroup scheme of $G$.
            \end{proposition}
                \begin{proof}
                    That ${}^{\red}G$ is a closed subscheme over $\Spec k$ of $G$ is obvious by construction. One can then apply lemma \ref{lemma: nilradicals_of_tensor_products} directly to show that ${}^{\red}G$ inherits its group structure from $G$.
                \end{proof}
                
            \begin{proposition}[Immersions of group schemes over fields are closed] \label{prop: subgroup_schemes_over_fields_are_closed}
                Let $k$ be a field and $\varphi: H \to G$ be a group $k$-scheme homomorphism. If $\varphi$ is an immersion of $k$-schemes then it will in fact be a closed immersion.
            \end{proposition}
                \begin{proof}
                    
                \end{proof}
            \begin{remark}
                Proposition \ref{prop: subgroup_schemes_over_fields_are_closed} does not imply that any subscheme of group schemes over fields is always closed, only that subgroup schemes are. For instance, for any field $k$, $(\GL_n)_k$ admits $(\SL_n)_k$ as a closed subscheme over $\Spec k$, but not $\Spec k[x_1, x_2, ..., x_{n^2}]\left[\frac{1}{\det}, \frac{1}{x_{12}}\right]$, which is very obviously open.
            \end{remark}
            \begin{corollary}[Identity components are closed over fields]
                Let $G$ be group scheme over $\Spec k$. Then its identity component $G^{\circ}$, by virtue of being a subscheme of $G$, is a closed subgroup scheme of $G$ over $\Spec k$.
            \end{corollary}
                
        \subsubsection{On the smoothness of algebraic groups}
            Now, one of the most oustanding properties of group schemes is that in many cases, they are smooth; in fact, algebraic groups over fields of characteristic $0$ are always smooth, and those over fields of positive characteristics are smooth under rather mild hypotheses. To be able to establish these results pertaining to smoothness, however, one will have to conduct some analysis of differential forms on group schemes. Actually, thanks to proposition \ref{prop: (quasi)_separatedness_criterion_for_group_schemes}, it shall suffice to understand the behaviour of modules of K\"ahler differentials associated to (closed) immersions.
            
            We begin by recalling the notion of the \textbf{conormal sheaf} of an immersion, along with some relevant material on the behaviour of quasi-coherent sheaves with respect to closed immersions. 
            \begin{lemma}[Quasi-coherent sheaves on closed subschemes] \label{lemma: quasi_coherent_sheaves_on_closed_subschemes}
                \cite[\href{https://stacks.math.columbia.edu/tag/01QY}{Tag 01QY}]{stacks} Suppose that $X$ is a scheme and that $i: Z \hookrightarrow X$ is a closed subscheme therein, corresponding to a quasi-coherent ideal sheaf $\calI_{Z/X} \subset \calO_X$. In such a situation, the pushforward functor $i_*: \QCoh(Z)\to \QCoh(X)$ will be an exact full faithful embedding whose essential image are quasi-coherent $\calO_X$-modules $\calF \in \QCoh(X)$ such that $\calI_{Z/X} \calF = 0$.
            \end{lemma}
            \begin{convention}[Closures and boundaries] \label{conv: closures_and_boundaries}
                From now on, if $i: Z \hookrightarrow X$ be an immersion of ringed spaces, then we shall denote the its topological closure by $\bar{i}: \bar{Z} \hookrightarrow X$ and boundary by $\del Z := \bar{Z} \setminus Z$. 
            \end{convention}
            \begin{definition}[(Co)normal sheaves of closed immersions] \label{def: (co)normal_sheaves_of_closed_immersions}
                Suppose that $X$ is a scheme and that $i: Z \hookrightarrow X$ is a locally closed subscheme therein, corresponding to a quasi-coherent ideal sheaf $\calI_{Z/X} \subset \calO_X$. The \textbf{conormal sheaf} associated to the locally closed immersion $i: Z \hookrightarrow X$ is the quasi-coherent $\calO_Z$-module $\calN_{Z/X}^{\vee} \cong i^*(\calI_{Z/X}/\calI_{Z/X}^2)$. The dual notion is that of so-called \textbf{normal sheaves}: the normal sheaf associated to a locally closed immersion $i: Z \hookrightarrow X$ is the module-theoretic dual of $\calN_{Z/X}^{\vee}$, i.e. the quasi-coherent $\calO_Z$-module $\calN_{Z/X} \cong \Hom_{\calO_Z}(i^*(\calI_{Z/X}/\calI_{Z/X}^2), \calO_Z)$. 
                
                It is also possible to define (co)normal sheaves for locally closed immersions: should $Z \subseteq X$ be a locally closed subscheme then its associated (co)normal sheaf could be defined with respect to the closed immersion $Z \subseteq X \setminus \del Z$.
            \end{definition} 
            \begin{remark}[Extension-by-zero of conormal sheaves]
                Let $i: Z \hookrightarrow X$ be a closed immersion. Because the ideal sheaf $\calI_{Z/X}/\calI_{Z/X}^2$ is annihilated by $\calI_{Z/X}$, one gets by lemma \ref{lemma: quasi_coherent_sheaves_on_closed_subschemes} (which can be understood to imply that the adjunction counit $\eta_{Z/X}: \id_{\QCoh(X)} \to i_*i^*$ is a natural isomorphism), that there is a canonical isomorphism of quasi-coherent $\calO_X$-modules:
                    $$\eta_{Z/X}(\calI_{Z/X}/\calI_{Z/X}^2): \calI_{Z/X}/\calI_{Z/X}^2 \to i_*i^*(\calI_{Z/X}/\calI_{Z/X}^2)$$
                This, in turn, induces a canonical isomorphism $\calI_{Z/X}/\calI_{Z/X}^2 \cong i_*\calN_{Z/X}^{\vee}$ of quasi-coherent $\calO_X$-modules.
            \end{remark}
            The following result illustrates the necessity for the introduction of conormal sheaves. 
            \begin{proposition}[Conormal sheaves of diagonals] \label{prop: conormal_sheaves_of_diagonals}
                \cite[\href{https://stacks.math.columbia.edu/tag/08S2}{Tag 08S2}]{stacks} Let $S$ be a scheme, let $X$ be an $S$-scheme. Then, the conormal sheaf $\calN_{\Delta_{X/S}}^{\vee}$ of the diagonal\footnote{Which is indeed a locally closed immersion by proposition \ref{prop: diagonals_of_schemes_are_locally_closed_immersions}, and therefore on can define $\calN_{\Delta_{X/S}}^{\vee}$ along the induced closed immersion $X \subseteq (X \x_S X) \setminus \del \Delta_{X/S}(X)$.} $\Delta_{X/S}: X \to X \x_S X$ is canonically isomorphic (as a quasi-coherent $\calO_X$-module) to the module of relative K\"ahler differentials $\Omega^1_{X/S}$.
            \end{proposition}
                \begin{proof}
                    
                \end{proof}
            \begin{proposition}[Flat base-changes of conormal sheaves] \label{prop: flat_base_changes_of_conormal_sheaves}
                Let $f: X \to X'$ be a flat morphism of schemes and $i': Z' \hookrightarrow X'$ is a closed immersion of schemes, and consider the following pullback square:
                    $$
                        \begin{tikzcd}
                        	Z & X \\
                        	{Z'} & {X'}
                        	\arrow["{f|_Z}"', from=1-1, to=2-1]
                        	\arrow["{i'}", hook, from=2-1, to=2-2]
                        	\arrow["f", from=1-2, to=2-2]
                        	\arrow["i", hook, from=1-1, to=1-2]
                        	\arrow["\lrcorner"{anchor=center, pos=0.125}, draw=none, from=1-1, to=2-2]
                        \end{tikzcd}
                    $$
                Then, there is a canonical isomorphism $(f|_Z)^* \calN_{Z'/X'}^{\vee} \to \calN_{Z/X}^{\vee}$ of quasi-coherent $\calO_Z$-modules. 
            \end{proposition}
                \begin{proof}
                    
                \end{proof}
            \begin{lemma}[A flatness criterion for action groupoids of group schemes] \label{lemma: flatness_criterion_for_action_groupoids_of_group_schemes}
                For some given scheme $S$, suppose that $\psi: T \to G$ is a morphism from a flat $S$-scheme $T$ to a group $S$-scheme $(G, m, e, i)$. Then, the composition $m \circ (\psi \x_S \id_{G/S}): T \x_S G \to G$ is flat as well.
            \end{lemma}
                \begin{proof}
                    
                \end{proof}
            \begin{corollary}
                The action groupoid of a flat group scheme (cf. convention \ref{conv: action_groupoids_of_group_schemes}) is flat. More generally, if for some fixed base scheme $S$, $G$ is a group $X$-scheme for some $S$-scheme $X$ that is flat over $S$ and acts on $X$ via $\alpha: G \x_S X \to X$ then the associated action groupoid $\alpha, \pr_2: G \x_S X \toto X$ will also be flat.
            \end{corollary}
            \begin{proposition}[Differential forms on group schemes] \label{prop: differential_forms_on_group_schemes}
                Let $(G, m, e, i)$ be a flat group scheme over a given scheme $S$ and that $\pi: G \to S$ is its structural morphism. 
            \end{proposition}
                \begin{proof}
                    
                \end{proof}
                
        \subsubsection{Epi-mono factorisations for linear algebraic groups}
            \begin{definition}[Scheme-theoretic images] \label{def: scheeme_theoretic_images}
                Let $f: Y \to X$ be a morphism of schemes. Then the category $({}^{Y/}\Sch_{/X})_{\closed}$ of commutative triangles:
                    $$
                        \begin{tikzcd}
                        	& Y \\
                        	Z & X
                        	\arrow["f", from=1-2, to=2-2]
                        	\arrow["", hook, from=2-1, to=2-2]
                        	\arrow[""', from=1-2, to=2-1]
                        \end{tikzcd}
                    $$
                wherein $Z \hookrightarrow X$ is a closed immersion (i.e. the category of $X$-schemes under $Y$ and closed immersions between them) has an initial object $\im f$ (cf. \cite[\href{https://stacks.math.columbia.edu/tag/01R6}{Tag 01R6}]{stacks}), which we call the \textbf{scheme-theoretic image} of the given morphism $f: Y \to X$. We might also write $f(Y)$ instead of $\im f$.
            \end{definition}
            \begin{proposition}[Morphisms surject onto images] \label{prop: morphisms_surject_onto_images}
                Let $f: Y \to X$ be a morphism of schemes. Then $\im f: Y \to f(Y)$ is surjective.
            \end{proposition}
                \begin{proof}
                    
                \end{proof}
            \begin{lemma}[The Closed Orbit Lemma] \label{lemma: the_closed_orbit_lemma}
                Let $k$ be an algebraically closed field and let $f: G' \to G$ be a homomorphism of smooth linear algebraic groups over $\Spec k$. Then the image $f(G')$ will be a smooth closed subgroup $k$-scheme of $G$.
            \end{lemma}
                \begin{proof}
                    
                \end{proof}
            \begin{proposition}[Surjections onto images of linear algebraic group homomorphisms are faithfully flat] \label{prop: surjections_onto_images_of_linear_algebraic_group_homomorphisms_are_faithfully_flat}
                Let $k$ be an algebraically closed field and let $f: G' \to G$ be a homomorphism of smooth linear algebraic groups over $\Spec k$. Then $\im f: G' \to f(G')$ is faithfully flat. 
            \end{proposition}
                \begin{proof}
                    
                \end{proof}
            \begin{definition}[Kernels] \label{def: kernels_of_homomorphisms_of_group_schemes}
                Suppose that $S$ is a scheme and that $\varphi: H \to G$ is a homomorphism of group presheaves on $(\Sch_{/S})_{\fppf}$. Then, thanks to the category of presheaves of groups on $(\Sch_{/S})_{\fppf}$ having all finite pullbacks\footnote{Which comes from the fact that $\Psh((\Sch_{/S})_{\fppf})$ has all (finite) pullbacks and that presheaves of groups on $(\Sch_{/S})_{\fppf}$ are defined internally via the monoidal structure on $\Psh((\Sch_{/S})_{\fppf})$ induced by binary products.} as well as initial objects $1_S$ (namely those isomorphic to $S$, as every group $S$-scheme is equipped with a uniquely defined identity morphism $e: S \to G$ by definition), one can define $\ker \varphi$ has the following pullback of presheaves of groups on $(\Sch_{/S})_{\fppf}$:
                    $$
                        \begin{tikzcd}
                        	{\ker \varphi} & {1_S} \\
                        	H & G
                        	\arrow[from=1-2, to=2-2]
                        	\arrow["\varphi", from=2-1, to=2-2]
                        	\arrow[from=1-1, to=2-1]
                        	\arrow[from=1-1, to=1-2]
                        	\arrow["\lrcorner"{anchor=center, pos=0.125}, draw=none, from=1-1, to=2-2]
                        \end{tikzcd}
                    $$
            \end{definition}
            \begin{corollary}[First Isomorphism Theorem for linear algebraic groups] \label{coro: first_isomorphism_theorem_for_linear_algebraic_groups}
                Let $k$ be an algebraically closed field and let $f: G' \to G$ be a homomorphism of smooth linear algebraic groups over $\Spec k$. Then there is an isomorphism of $k$-schemes:
                    $$G'/\ker f \cong \im f$$
            \end{corollary}
                \begin{proof}
                    
                \end{proof}
            \begin{proposition}[Monomorphisms between linear algebraic groups] \label{prop: monomorphisms_between_linear_algebraic_groups}
                Let $k$ be an algebraically closed field and let $f: G' \to G$ be a homomorphism of smooth linear algebraic groups over $\Spec k$. If $\ker f$ is finite over $\Spec k$ then $\im f: G' \to f(G')$ will be finite flat. In particular, if $\ker f$ is trivial then $f: G' \to G$ will be a closed immersion; if $f: G' \to G$ is also surjective and $\ker f$ is trivial then $f$ will be an isomorphism of group $k$-schemes.
            \end{proposition}
                \begin{proof}
                    
                \end{proof}
                
    \subsection{Representations and cohomology of group schemes}
        \subsubsection{Linear representations of group schemes}
            \begin{definition}[Algebras, coalgebras, and bialgebras] \label{def: algebras_coalgebras_and_bialgebras} 
                Let $(\calV, \tensor, \1)$ be a monoidal category. 
                    \begin{itemize}
                        \item \textbf{(Algebras):} A(n) (associative and unital) \textbf{algebra} $A$ internal to $\calV$ is a monoid object of $\calV$, i.e. one equipped with a so-called multiplication $\nabla: A \tensor A \to A$ and unit $\eta: \1 \to A$, which together satisfy the following commutative diagrams:
                            $$
                                \begin{tikzcd}
                                	{A \tensor A \tensor A} & {A \tensor A} \\
                                	{A \tensor A} & {A}
                                	\arrow["{\id_A \tensor \nabla}"', from=1-1, to=2-1]
                                	\arrow["{\nabla}", from=2-1, to=2-2]
                                	\arrow["{\nabla \tensor \id_A}", from=1-1, to=1-2]
                                	\arrow["{\nabla}", from=1-2, to=2-2]
                                \end{tikzcd}
                            $$
                            $$
                                \begin{tikzcd}
                                	{\1 \tensor A} & {A \tensor A} \\
                                	{A \tensor A} & {A}
                                	\arrow["{\nabla}", from=1-2, to=2-2]
                                	\arrow["{\nabla}", from=2-1, to=2-2]
                                	\arrow["{\id_A \tensor \eta}"', from=1-1, to=2-1]
                                	\arrow["{\eta \tensor \id_A}", from=1-1, to=1-2]
                                \end{tikzcd}
                            $$
                        \item \textbf{(Coalgebras):} A (coassociative and counital) \textbf{coalgebra} internal to $\calV$ is then a comonoid object of $\calV$, or in other words, an monoid object in $\calV^{\op}$. Typically, the so-called multiplication and counit maps on a coalgebra will be denoted by $\Delta$ and $\e$.
                        \item \textbf{(Bialgebras):} A(n) (associative, coassociative, unital, and counital) \textbf{bialgebra} internal to $\calV$ is an object that is simultaneously an (associative and unital) algebra and a (coassociative and counital) coalgebra.
                    \end{itemize}
                Algebras, coalgebras, and bialgebras internal to a monoidal category $\calV$ form subcategories, which we will denote, respectively, by $\Assoc\Alg(\calV), \co\Assoc\Alg(\calV)$, and $\bi\Alg(\calV)$.
            \end{definition}
            \begin{convention}
                Unless stated otherwise, every algebra shall be assumed to be associative and unital for now, and every coalgebra shall likewise be taken as being coassociative and counital. 
            \end{convention}
            \begin{definition}[(Co)commutativity] \label{def: (co)commutativity}
                Suppose that $(\calV, \tensor, \1)$ is a monoidal category and that $A$ is an object equipped with a braiding $\tau_{A, A}: A \tensor A \to A \tensor A$. Should $A$ be an algebra $(A, \nabla, \eta)$, then it shall be \textbf{commutative} in the event that the following diagram commutes:
                    $$
                        \begin{tikzcd}
                        	{A \tensor A} && {A \tensor A} \\
                        	& A
                        	\arrow["{\tau_{A, A}}", from=1-1, to=1-3]
                        	\arrow["\nabla"', from=1-1, to=2-2]
                        	\arrow["\nabla", from=1-3, to=2-2]
                        \end{tikzcd}
                    $$
                Dually, a \textbf{cocommutative} coalgebra in $\calV$ is a commutative algebra in $\calV^{\op}$.
            \end{definition}
            \begin{definition}[Hopf algebras] \label{def: hopf_algebras}
                Let $(\calV, \tensor, \1)$ be a monoidal category. Then, a bialgebra $(H, \nabla, \eta, \Delta, \e)$ is called a \textbf{Hopf algebra} if there exists a so-called \textbf{antipode} $\sigma: H \to H$ for which the following diagram commutes:
                    $$
                        \begin{tikzcd}
                        	& {H \tensor H} && {H \tensor H} \\
                        	{H} && {\1} && {H} \\
                        	& {H \tensor H} && {H \tensor H}
                        	\arrow["{\Delta}"', from=2-1, to=3-2]
                        	\arrow["{\id_H \tensor \sigma}"', from=3-2, to=3-4]
                        	\arrow["{\Delta}", from=2-1, to=1-2]
                        	\arrow["{\sigma \tensor \id_H}", from=1-2, to=1-4]
                        	\arrow["{\nabla}", from=1-4, to=2-5]
                        	\arrow["{\nabla}"', from=3-4, to=2-5]
                        	\arrow["{\e}", from=2-1, to=2-3]
                        	\arrow["{\eta}", from=2-3, to=2-5]
                        \end{tikzcd}
                    $$
                There is an obvious category of Hopf algebras internal to $\calV$, which we shall denote by $\Hopf\Alg(\calV)$.
            \end{definition}
            \begin{example}[Hopf algebra structures on global sections of group schemes] \label{example: hope_algebra_structures_on_global_sections_of_group_schemes}
                Suppose that $S$ is a scheme, which can be taken to be some affine scheme $\Spec k$ without any loss of generality. In addition, suppose that $G$ is a group scheme over $\Spec k$. One sees that the global section $k[G]$ has a natural structure of a commutative Hopf algebra internal to the symmetric\footnote{Symmetry is important here, because it implies that every object is equipped with an invertible braiding.} monoidal category of $k$-modules. Furthermore, $k[G]$ is cocommutative if and only if $G$ is abelian. In fact, there are the following (monoidal) adjoint equivalences, compatible in the obvious manner:
                    $$
                        \begin{tikzcd}
                        	{k\-\Comm\Hopf\Alg^{\op}} & {\Grp\Sch(\Spec k)}
                        	\arrow[""{name=0, anchor=center, inner sep=0}, "{\Spec }"', bend right, from=1-1, to=1-2]
                        	\arrow[""{name=1, anchor=center, inner sep=0}, "\Gamma"', bend right, from=1-2, to=1-1]
                        	\arrow["\dashv"{anchor=center, rotate=-90}, draw=none, from=1, to=0]
                        \end{tikzcd}
                    $$
                    $$
                        \begin{tikzcd}
                        	{k\-\co\Comm\Comm\Hopf\Alg^{\op}} & {\Comm\Grp\Sch(\Spec k)}
                        	\arrow[""{name=0, anchor=center, inner sep=0}, "{\Spec }"', bend right, from=1-1, to=1-2]
                        	\arrow[""{name=1, anchor=center, inner sep=0}, "\Gamma"', bend right, from=1-2, to=1-1]
                        	\arrow["\dashv"{anchor=center, rotate=-90}, draw=none, from=1, to=0]
                        \end{tikzcd}
                    $$
            \end{example}
            
            Let us now move on to discussing linear representations of affine group schemes, which we know to correspond to commutative Hopf algebras (cf. example \ref{example: hope_algebra_structures_on_global_sections_of_group_schemes}). In particular, linear representations of affine group schemes shall be considered as so-called \textbf{comodules} over commutative Hopf algebras, which is an interpretation of representations of group schemes that will be of tremendous usefulness later on when we wish to discuss the functoriality of representations with respect to group scheme homomorphisms, which can then be simply thought of as extensions and restrictions of scalars. 
            
            \begin{definition}[Equivariant monoidal categories] \label{def: equivariant_monoidal_categories}
                Let $\calA$ and $\calV$ be monoidal categories. Then, the category $\calV^{\calA}$ of $\calA$-equivariant objects of $\calV$ is the category $\Mon\Func(\calA, \calV)$ of monoidal functors from $\calA$ into $\calV$.
            \end{definition}
            \begin{definition}[Linear representations of group schemes] \label{def: linear_representations_of_group_schemes}
                Suppose that $S$ is a scheme and $G$ is a group scheme over $S$. Additionally, let $\calV$ be a triangulated monoidal category fibred over $(\Sch_{/S})_{\fppf}$ (which we note to also be monoidal). Then, the category of \textbf{$\calV$-valued representations of $G$} is 
            \end{definition}
            \begin{definition}[Modules and comodules] \label{def: modules_and_comodules}
                Suppose that $(\calV, \tensor, \1)$ is a monoidal category. 
                    \begin{itemize}
                        \item \textbf{(Modules)} A \textbf{left-module} over an algebra\footnote{Again, assumed to be associative and unital unless specifically stated otherwise.} $(A, \nabla, \eta)$ (also known as a left-$A$-module) is an pair $(M, \lambda)$ consisting of an object $M \in \Ob(\calV)$ and a morphism $\lambda: A \tensor M \to M$ for which the following diagrams commute:
                            $$
                                \begin{tikzcd}
                                	{A \tensor A \tensor M} & {A \tensor M} \\
                                	{A \tensor M} & M
                                	\arrow["{\id_A \tensor \lambda}"', from=1-1, to=2-1]
                                	\arrow["\lambda", from=2-1, to=2-2]
                                	\arrow["{\nabla \tensor \id_M}", from=1-1, to=1-2]
                                	\arrow["\lambda", from=1-2, to=2-2]
                                \end{tikzcd}
                            $$
                            $$
                                \begin{tikzcd}
                                	{1 \tensor M} & {A \tensor M} \\
                                	& M
                                	\arrow["{\eta \tensor \id_M}", from=1-1, to=1-2]
                                	\arrow["\lambda", from=1-2, to=2-2]
                                	\arrow[from=1-1, to=2-2]
                                \end{tikzcd}
                            $$
                        \item \textbf{(Comodules):} A \textbf{left-comodule} over a coalgebra $A$ is an object $M \in \Ob(\calV)$ which is a left-$A$-module in $\calV^{\op}$. 
                    \end{itemize}
                Give any (co)algebra object $A \in \Ob(\calV)$, there is an evident category of left $A$-(co)modules internal to $\calV$ in the above sense, which we denote by ${}^lA\mod$ (respectively, ${}^lA\comod$).
            \end{definition}
            \begin{proposition}
                Suppose that $G$ is an affine group scheme over a commutative ring $k$. Then, $\Rep_k(G)$ shall be monoidal-equivalent to the category of $k[G]$-comodules internal to the monoidal category $k\mod$ of $k$-modules\footnote{Note that this definition makes sense since $k[G]$, by virtue of being a Hopf algebra object of the monoidal category $k\mod$, is a coalgebra internal to $k\mod$ \textit{a fortiori}.}.
            \end{proposition}
                \begin{proof}
                    
                \end{proof}
            
            In order to be able to effectively discuss linear representations of non-affine group schemes (i.e. those of group schemes instead of merely of constant abstract groups) in a manner mimicking the affine theory, a theory of equivariant quasi-coherent sheaves is needed, as we think of them as generalised linear-algebraic spaces upon which our group schemes shall act. In order to do this, we shall be making use of the fact that structure sheaves of group schemes have natural interpretations as Hopf algebra objects in order to consider equivariant quasi-coherent sheaves as comodules over these special bialgebra objects. 
            
        \subsubsection{Induced and restricted representations}
        
        \subsubsection{Cohomology of group schemes}
        
    \subsection{Algebraic Lie theory}
        \subsubsection{Tangent spaces and Lie algebras}
        
        \subsubsection{Differential operators, distributions, and arithmetic Lie algebras}
        
        \subsubsection{Formal groups}
            
                \section{Preliminary constructions}
    \subsection{Normalisers, centralisers, and quotients}
        \subsubsection{Transporters and homomorphisms}
            \begin{definition}[Adjoint actions] \label{def: adjoint_actions}
                \noindent
                \begin{enumerate}
                    \item \textbf{(Adjoint actions of groups):} The \textbf{adjoint action} of a group $G$ on a subset $X \subseteq G$ is given by:
                        $$\Ad_G: G \to \Aut(X)$$
                        $$g \mapsto \left(\Ad_G(g): X \to X: x \mapsto gxg^{-1}\right)$$
                    \item \textbf{(Adjoint actions of group schemes):} Suppose that $G$ is a group scheme over a given base scheme $S$ and that $i: X \hookrightarrow G$ is a monomorphism of $S$-schemes. The \textbf{adjoint action} of $G$ on $X$ is thus a natural transformation $\Ad_G: G \to \Aut(X/S)$ given by $\Ad_{G(T)}$ at each $T \in \Ob((\Sch_{/S})_{\fppf})$.
                \end{enumerate}
            \end{definition}
            \begin{remark}[Representability of automorphism groups of schemes] \label{remark: representability_of_automorphism_groups_of_schemes}
                One reason that we needed the notion of kernels of homomorphisms of presheaves of groups on $\Psh((\Sch_{/S})_{\fppf})$ (with $S$ being some fixed base scheme) rather than the more concrete\footnote{Not that the two differ much from a categorical point of view.} notion of kernels of homomorphisms between group $S$-schemes is because it is not always guaranteed that for any $S$-scheme $X$, the automorphism group $\Aut(X/S)$ is representable by an $S$-scheme. 
            \end{remark}
            \begin{definition}[Normalisers] \label{def: normalisers}
                Suppose that $S$ is a scheme, $G$ is a group $S$-scheme and $i: X \hookrightarrow G$ is a monomorphism of $S$-schemes. Then, the \textbf{normaliser} of $X$ inside $G$ (or rather, associated to the monomorphism $i: X \hookrightarrow G$), denoted by $\Norm_{G/S}(X)$, is the scheme-theoretic kernel $\ker \Ad_G$ of the adjoint action of $G$ on the subscheme $X$.
            \end{definition}
            \begin{definition}[Transporters] \label{def: transporters}
                Suppose that $S$ is a scheme and $G$ is a group scheme over $S$. In addition, suppose that we are given two monomorphisms $i: X \hookrightarrow G$ and $j: Y \hookrightarrow G$ of $S$-schemes. The \textbf{transporter} from $X$ to $Y$ inside $G$ (or rather, from the monomorphism $i: X \hookrightarrow G$ to the monomorphism $j: Y \hookrightarrow G$) is thus the following pullback in the category of group $S$-schemes:
                    $$
                        \begin{tikzcd}
                        	{\Transporter_{G/S}(X, Y)} & {\Norm_{G/S}(Y)} \\
                        	{\Norm_{G/S}(X)} & G
                        	\arrow[from=1-1, to=2-1]
                        	\arrow[from=2-1, to=2-2]
                        	\arrow[from=1-1, to=1-2]
                        	\arrow[from=1-2, to=2-2]
                        	\arrow["\lrcorner"{anchor=center, pos=0.125}, draw=none, from=1-1, to=2-2]
                        \end{tikzcd}
                    $$
            \end{definition}
            \begin{remark}[Transporters and normalisers]
                It is not hard to see that given a fixed group $S$-scheme $G$ and a fixed monomorphism $i: X \hookrightarrow G$, we have a natural isomorphism $\Norm_{G/S}(X) \cong \Transporter_{G/S}(X, X)$. 
            \end{remark}
            \begin{lemma}[Representability of automorphism groups of schemes] \label{lemma: representability_of_automorphism_groups_of_schemes}
                
            \end{lemma}
                \begin{proof}
                    
                \end{proof}
            \begin{corollary}[Representability of normalisers and transporters] \label{coro: representability_of_normalisers_and_transporteres}
                Let $S$ be a scheme and $G$ be a group scheme over $S$. Then, the 
            \end{corollary}
                \begin{proof}
                    
                \end{proof}
                
        \subsubsection{Centralisers}
        
        \subsubsection{Quotients}
        
    \subsection{Groups of multiplicative type}
        \subsubsection{Diagonalisability and split tori}
            \begin{definition}[Diagonalisable group schemes] \label{def: diagonalisable_group_schemes}
                Suppose that $\Lambda$ is a commutative group. Then, over a commutative ring $k$, one can define an associated (affine) group $k$-scheme $\Diag(\Lambda/k) \cong \Spec k\<\Lambda\>$, called the \textbf{diagonalisation} of $\Lambda$, by putting a cocommutative and commutative Hopf algebra structure on $k\<\Lambda\>$ via the comultiplication $\Delta(\lambda) := \lambda \tensor \lambda$, counit $\e(\lambda) := 1$, and antipode $\sigma(\lambda) = \lambda^{-1}$.
                
                A group scheme over $\Spec k$ that is isomorphic to one of the form $\Diag(\Lambda/k)$ is said to be \textbf{diagonalisable}\footnote{As we shall see, this is the same as requiring that diagonalisable group schemes be in the essential image of the functor $\Diag: \Comm\Grp \to \Grp\Sch(\Spec k)$.} over $\Spec k$, and in doing so one obtains a full subcategory $\Diag\Grp\Sch^{\aff}_{/\Spec k}$ spanned by diagonalisable group schemes over $\Spec k$.
            \end{definition}
            \begin{remark}
                It should be noted that over non-affine bases, one can certainly construct diagonalisable non-affine group schemes. However, one does not lose any amount of generality by working over non-affine bases, so this is not an issue we will have to worry about. Nevertheless, we write the \say{$\aff$} superscript to put emphasis on this subtlety. Diagonalisable algebraic groups, as they are always defined over fields, will instead always be affine.
            \end{remark}
            \begin{remark}[Diagonalisable algebraic groups] \label{remark: diagonalisable_algebraic_groups}
                Fix a commutative ring $k$. While it is rather obvious that there is a functor:
                    $$\Diag: \Comm\Grp^{\op} \to \Diag\Grp\Sch^{\aff}_{/\Spec k}$$
                    $$\Lambda \mapsto \Spec k\<\Lambda\>$$
                coming from the fact that homomorphisms of group algebras one can actually say much more. For instance, it commutes with base-changes by construction: for every homomorphism between commutative rings $k \to k'$, one has:
                    $$\Diag(\Lambda/k') \cong \Spec k'\<\Lambda\> \cong \Spec \left(k\<\Lambda\> \tensor_k k'\right) \cong \Spec k\<\Lambda\> \x_{\Spec k} \Spec k' \cong \Diag(\Lambda/k) \x_{\Spec k} \Spec k'$$
                In addition, observe that if a given commutative group $\Lambda$ were to be finitely generated\footnote{Henceforth, we shall refer to finitely generated groups as groups of finite type so that the terminology will line up with those of modules and algebras.} (say, by $\lambda_1, ..., \lambda_n$) then its group $k$-algebra $k\<\Lambda\>$ would be of finite type as a (commutative) $k$-algebra by virtue of being isomorphic to $k[\lambda_1, ..., \lambda_n]$ (note that the variables $\lambda_1, ..., \lambda_n$ might not be $k[x_1, ..., x_n]$-linearly independent), which in turn implies that the associated diagonalisable group scheme $\Diag(\Lambda/k) \cong \Spec k\<\Lambda\>$ is of finite type over $\Spec k$; as a result, $\Diag(\Lambda/k)$ is an algebraic group over $\Spec k$ whenever $\Lambda$ is finitely generated (and of course, when $k$ is a field). Since algebraic groups form a full subcategory of the category of group schemes, the above analysis leads to a functor:
                    $$\Diag: (\Comm\Grp^{\ft})^{\op} \to \Diag\Alg\Grp^{\aff}_{/\Spec k}$$
                Of course, this functor also commutes with base-changes. 
            \end{remark}
            \begin{definition}[Characters] \label{def: characters_of_algebraic_groups}
                The group of \textbf{characters} of an algebraic group $G$ over a given field $k$, commonly denoted by $\bbX(G/k)$, is the (commutative) group of homomorphisms $\chi: G \to \G_m$.
            \end{definition}
            \begin{proposition}[Pontryagin Duality for algebraic groups] \label{prop: pontryagin_duality_for_algebraic_groups}
                For a fixed field $k$, the functor $\Diag: (\Comm\Grp^{\ft})^{\op} \to \Diag\Alg\Grp^{\aff}_{/\Spec k}$ is fully faithful and essentially surjective, with quasi-inverse given by the character functor\footnote{Which by definition is nothing but the representable functor $\Diag\Alg\Grp^{\aff}_{/\Spec k}(-, \G_m)$ (note that this is well-defined because $\G_m$ is diagonalisable as it is isomorphic to $\Spec k\<\Z\>$).}:
                    $$\bbX: (\Diag\Alg\Grp^{\aff}_{/\Spec k})^{\op} \to \Comm\Grp^{\ft}$$
            \end{proposition}
                \begin{proof}
                    
                \end{proof}
            \begin{corollary}[Diagonalisation is exact] \label{coro: diagonalisation_is_exact}
                For a fixed field $k$, the functor:
                    $$\Diag: (\Comm\Grp^{\ft})^{\op} \to \Diag\Alg\Grp^{\aff}_{/\Spec k}$$
                is exact. 
            \end{corollary}
                \begin{proof}
                    
                \end{proof}
            \begin{corollary}[The Fundamental Theorem of Diagonalisable Algebraic Groups] \label{coro: the_fundamental_theorem_of_diagonalisable_algebraic_groups}
                Suppose that $\Lambda$ is a commutative group of finite type which decomposes\footnote{By the Fundamental Theorem of Finitely Generated Abelian Groups} into free and torsion factors as follows (note that we are supposing that $\rank_{\Z} \Lambda = r$):
                    $$\Lambda \cong \Z^{\oplus r} \oplus \bigoplus_{i = 1}^s (\Z/p_i\Z)^{\oplus e_i}$$
                wherein $p_1, ..., p_s$ are certain primes and $e_i$ are positive integer exponents. In addition, fix a field $k$. Then, by exactness, the diagonalisable algebraic group $\Diag(\Lambda/k)$ decomposes as:
                    $$\Diag(\Lambda/k) \cong \Diag(\Z/k)^r \x_{\Spec k} \prod_{i = 1}^s \Diag((\Z/p_i\Z)/k)^{e_i} \cong \G_m^r \x_{\Spec k} \prod_{i = 1}^s \mu_{p_i}^{e_i}$$
            \end{corollary}
            
            \begin{definition}[Algebraic tori] \label{def: algebraic_tori}
                An \textbf{algebraic torus} over a given field $k$ is a commutative algebraic group scheme $T$ over $\Spec k$, such that $T_{\bar{k}}$ (for some choice of algebraic closure $\bar{k}/k$) is diagonalisable as a commutative algebraic group scheme over $\Spec \bar{k}$.
            \end{definition}
            \begin{remark}
                By corollary \ref{coro: the_fundamental_theorem_of_diagonalisable_algebraic_groups} and the fact that algebraic groups are of finite type over their base fields by definition, one sees that an algebraic torus $T$ over a given field $k$ is equivalently an algebraic group over $\Spec k$ such that $T_{\bar{k}} \cong (\G_m)_{\bar{k}}^r$, where $\bar{k}/k$ is an algebraic closure and $r$ is a positive integer.  
            \end{remark}
            
        \subsubsection{Group schemes of multiplicative type}
            \begin{definition}[Group schemes of multiplicative type] \label{def: group_schemes_of_multiplicative_type}
                Let $S$ be a scheme and $G$ be a group scheme over $S$. We say that $G$ is of \textbf{multiplicative type} if fpqc-locally, it is diagonalisable.     
            \end{definition}
            
        \subsubsection{Classification of group shcemes of multiplicative type}
        
    \subsection{Maximal tori, Weyl groups, and Cartan subgroups}
        \subsubsection{Maximal tori}
        
        \subsubsection{Weyl groups}
        
        \subsubsection{Cartan subgroups}
                
    \subsection{Unipotent group schemes}
                
                \section{Reductive group schemes and geometric reductivity}
    \subsection{Classical reductive groups}
        \begin{convention} \label{conv: classical_reductive_groups_ground_field}
            Throughout, we shall be working with smooth affine group schemes over an algebraically closed field $k$, though we do not assume that these they are Zariski-connected.
        \end{convention}
        \begin{remark}[Regarding the smoothness assumption]
            If $\chara k = 0$ (respectively, $\chara k = p$ for some prime $p$ and $k$ is perfect) then we will not have to make the extra assumption of smoothness for affine group schemes that are locally of finite type (respectively, locally of finite type and reduced), since such group schemes are already smooth \textit{a priori} (cf. \cite[\href{https://stacks.math.columbia.edu/tag/047N}{Tag 047N} and \href{https://stacks.math.columbia.edu/tag/047P}{Tag 047P}]{stacks}).
        \end{remark}
        \begin{remark}[Embeddings into $\GL_n$]
            Recall also that every so-called \say{linear algebraic group} over $k$ (i.e. group $k$-schemes which are affine and of finite type over $\Spec k$) are closed subschemes of $(\GL_n)_k$, whose functor of points is represented by the affine $k$-scheme:
                $$\Spec k[x_{11}, x_{12}, ..., x_{nn}][1/\det]$$
            wherein $\det := \det(x_{11}, x_{12}, ..., x_{n^2})$ is the polynomial that returns the determinant of the matrix $(x_{ij})_{1 \leq i, j \leq n} \in \Mat_n(k)$. This is because should $G$ be a linear algebraic group over $k$, then the fact that it is of finite type over $\Spec k$ means that it is defined by some $k[x_{11}, x_{12}, ..., x_{nn}][1/\det]$-ideal $I$ via:
                $$G \cong \Spec k[x_{11}, x_{12}, ..., x_{nn}][1/\det]/I$$
        \end{remark}
    
        \subsubsection{Solvable and reductive linear algebraic groups}
        
        \subsubsection{Roots and coroots}
        
        \subsubsection{Root data and root space decompositions}
        
        \subsubsection{Positive roots and parabolic subgroups}
        
        \subsubsection{Based root data and pinnings}
    
    \subsection{Reductive group schemes}
                
                \input{Chapters/Appendices/Reductive groups/roots and coroots}
                
                \section{Split reductive groups}
    \subsection{Parabolic subgroups}
    
    \subsection{Existence theorems and isogenies}
    
    \subsection{Automorphism schemes}
    
    \subsection{Grothendieck's theorem on tori}
        \begin{theorem}[Grothendieck's theorem on tori] \label{theorem: grothendieck_theorem_on_tori}
            Let $k$ be a field. Any smooth affine group scheme over $\Spec k$ contains a $k$-torus $T$ such that $T_{\bar{k}}$ is a maximal torus of $G_{\bar{k}}$.
        \end{theorem}
            \begin{proof}
            
            \end{proof}
        \end{appendices}
    
    \part{Tensor categories and quantum groups}
        \chapter{Quantum groups and module categories}
            \begin{abstract}
                
            \end{abstract}
            
            \minitoc
            
            \section{Hopf algebras and quantum groups}
    \subsection{\texorpdfstring{$q$}{}-analogues and quantum groups arising from Lie algebras}
        \subsubsection{\texorpdfstring{$q$}{}-analogues}
            \begin{definition}[$q$-analogues] \label{def: q_analogues}
                This shall be a somewhat informal definition of the notion of $q$-analogues, namely due to the fact that the notion itself is not a mathematical one but is instead more sociological in nature. Roughly speaking, a \textbf{$q$-analogue} is an analogue of a mathematical definition or result via the introduction of a parameter $q$ (sometimes referred to as a \say{quantum parameter}, and hence the notation), such that in the \say{classical limit} (also referred to as the \say{classical case} or the \say{limiting case}) $q \to 1$, the original definition or result is recovered. 
            \end{definition}
            \begin{example}
                Suppose that $k$ is a commutative ring of characteristic $0$ and consider the polynomial ring $k[x, y]$. A natural $q$-analogue to introduce would then be the noncommutative polynomial ring $k\<x, y\>/\<xy - q yx\>$: obviously, in the limit $q \to 1$, the ideal $\<xy - q yx\>$ becomes zero, and one recovers the commutative polynomial ring $k[x, y]$.
            \end{example}
            
            \begin{remark}[Affine group schemes and Hopf algebras]
                For definition \ref{def: quantum_groups}, let us first recall that for every commutative ring $k$, there is a monoidal equivalence between the symmetric monoidal category $\Grp\Sch_{/\Spec k}^{\aff}$ of affine group schemes over $\Spec k$ and the opposite $k\-\Comm\Hopf\Alg^{\op}$ of the symmetric monoidal category of commutative Hopf algebra objects internal to the symmetric monoidal category of $k$-modules. The strategy is then to define quantum groups (at least in the sense of Drinfeld-Jimbo) as $q$-analogues (perhaps with more quantum parameters) of commutative Hopf algebras. The resulting objects will generally neither be commutative nor cocommutative, but they will still be Hopf algebras. 
            \end{remark}
            \begin{definition}[Quantum groups] \label{def: quantum_groups}
                An \textbf{affine quantum group}
            \end{definition}
            
            \begin{definition}[Quantum groups associated to simple Lie algebras] \label{}
                
            \end{definition}
        
        \subsubsection{Quantum universal enveloping algebras}
        
\section{Quasi-triangular bialgebras and modules over monoidal categories}
    
            
            \input{Chapters/Quantum groups and module categories/quasi-triangular bialgebras}
        
        \chapter{Braided and chiral monoidal categories}
            \begin{abstract}
                
            \end{abstract}
            
            \minitoc
            
            \input{Chapters/Braided and chiral categories/braided monoidal categories}
            
            \section{(Quantum) vertex algebras}
    \subsection{Vertex operator algebras}
        \subsubsection{The classical approach to vertex operator algebras}
    
        \subsubsection{Vertex operator algebras as "singular" commutative algebras}
            \begin{convention}
                Let us now fix a commutative ring $k$ along with a $k$-linear infinite\footnote{This means that we do not require that $\calA$ is locally finite as a $k$-linear category, unlike in \cite[Definition 4.1.1]{EGNO}. Perhaps one could either require that $\calA$ is locally Noetherian or Artinian as a $k$-linear category (that is to say, that the hom-spaces are Noetherian or Artinian $k$-modules, respectively), but we are not certain that this is the weakest finiteness assumption that one could impose upon $\calA$.} tensor category:
                    $$(\calA, \tensor, \1)$$
                (this means that $\E$ is a $k$-linear abelian rigid monoidal category such that firstly, the monoidal structure $\tensor: \calA \x \calA \to \calA$ is $k$-bilinear on hom-spaces and secondly, that $\calA(\1, \1) \cong k$) wherein colimits are flat. An example of such a category is the category $k\mod$ of $k$-modules. 
            \end{convention}
            \begin{remark}
                It is easy (via object-wise considerations of tensor products and hom-spaces, respectively) to see that the functor category\footnote{Here, $\Fin\Sets$ is understood to be equipped with the canonical monoidal structure given by disjoint unions (the monoidal unit is $\varnothing$).}:
                    $$\Mon\Func(\Fin\Sets^{\op}, \calA)$$
                are symmetric monoidal and $k$-linear\footnote{We shall explain later why we care about this category in particular.}. In particular, this means that it makes sense to consider coalgebras internal to either of these categories; specifically, we want to consider the monoidal functor\footnote{We shall let our dear readers try to convince themselves that $H^{\tensor (-)}$ is indeed functorial. It is certainly monoidal.}:
                    $$H^{\tensor (-)}: \Fin\Sets^{\op} \to \Comm\co\Comm\Alg(\calA)$$
                    $$I \mapsto H^{\tensor I}$$
                where:
                    $$H \in \Ob(\Comm\co\Comm\Alg(\calA))$$
                is a fixed commutative and cocommutative bialgebra object of $\calA$, and note that:
                    $$H^{\tensor (-)} \in \Ob(\Comm\co\Comm\Alg(\Mon\Func(\Fin\Sets^{\op}, \calA)))$$
                precisely because $H$ has been chosen to be a commutative and cocommutative bialgebra object of $\calA$. 
                
                If we make the further assumption that $\calA$ is monoidal-closed, then we can consider actions of $H^{\tensor (-)}$ on objects $F \in \Ob(\Mon\Func(\Fin\Sets^{\op}, \calA))$: specifically, these are associative algebra\footnote{In the sense of associative algebras internal to $\Mon\Func(\Fin\Sets^{\op}, \calA)$} homomorphisms:
                    $$\alpha: H^{\tensor (-)} \to \End_{\Mon\Func(\Fin\Sets^{\op}, \calA)}(F)$$
                The point here is that the category of interest to us is:
                    $$\Mon\Func(\Fin\Sets^{\op}, \calA)^{H^{\tensor (-)}}$$
                i.e. that of $H^{\tensor (-)}$-equivariant monoidal functors from $\Fin\Sets^{\op}$ to $\calA$. This category is also $k$-linear and symmetric monoidal, so if we were to consider commutative algebra objects:
                    $$S \in \Ob(\Comm\Alg(\Mon\Func(\Fin\Sets^{\op}, \calA)^{H^{\tensor (-)}}))$$
                internal to it, we will be able to subsequently consider the category of $S$-modules internal to $\Mon\Func(\Fin\Sets^{\op}, \calA)^{H^{\tensor (-)}}$. This category, in turn, is also $k$-linear and symmetric monoidal: its monoidal structure is given by:
                    $$- \tensor_S -: S\mod \x S\mod \to S\mod$$
                    $$(F, G) \mapsto \left(I \mapsto F(I) \tensor_{S(I)} G(I)\right)$$
            \end{remark}
            \begin{example}
                Fix a field $k$ and let $\calA := k\mod$; particularly, note that colimits in this category are indeed flat (every vector space is free, after all). One could consider, for instance, the commutative and cocommutative $k$-bialgebra:
                    $$k[\del] \in \Ob(k\-\Comm\co\Comm\Alg)$$
                (that this is a cocommutative coalgebra internal to $k\mod$ comes from the fact that it is the global section of the additive group $k$-scheme $(\G_a)_k \cong \A^1_k$). The resulting functor $k[\del]^{\tensor (-)}$ then sends $I \in \Ob(\Fin\Sets^{\op})$ to the $k$-vector space $k[\del_I] := k[\{\del_i\}_{i \in I}]$, which is indeed a commutative and cocommutative $k$-bialgebra.
                
                An example of an object $F \in \Ob(k\-\Comm\co\Comm\Alg)$ with a $k[\del]^{\tensor (-)}$-action is the functor:
                    $$k[z^{\pm 1}]^{\tensor (-)}: \Fin\Sets^{\op} \to k\mod$$
                    $$I \mapsto k[\{(z_i - z_j)^{\pm 1}\}_{i, j \in I, i \not = j}]$$
                An explicit example of such an action is the action of $k[\{\del_{z_i}\}_{i \in I}]$ on $k[\{(z_i - z_j)^{\pm 1}\}_{i, j \in I, i \not = j}]$ via partial differentiation. Now, $k[z^{\pm 1}]^{\tensor (-)}$ is actually a commutative $k[\del]^{\tensor (-)}$-algebra as well, so the ultimate category of interest shall be that of $k[z^{\pm 1}]^{\tensor (-)}$-modules internal to the $k$-linear symmetric monoidal category $\Mon\Func(\Fin\Sets^{\op}, k\mod)^{k[\del]^{\tensor (-)}}$. 
            \end{example}
            
            \begin{definition}[Borcherd's singular tensor product] \label{def: singular_tensor_products}
                Let $k$ be a commutative ring, let $\calA$ be a $k$-linear infinite tensor category, let $H \in \Ob(\Comm\co\Comm\Alg(\calA))$ be a commutative and cocommutative bialgebra internal to $\calA$, and let $S \in \Ob(\Comm\Alg( \Mon\Func(\Fin\Sets, \calA)^{H^{\tensor (-)}} ))$ be a commutative algebra object over which we shall be considering modules. One can then define an additional monoidal structure on $S\mod$, which is:
                    $$\singtensor: S\mod \x S\mod \to S\mod$$
            \end{definition}
        
        \subsubsection{Examples of vertex operator algebras}
    
    \subsection{Quantum vertex operator algebras}
        
        \chapter{Fusion categories}
            \begin{abstract}
                
            \end{abstract}
            
            \minitoc
            
            \section{Fusion categories over characteristic \texorpdfstring{$0$}{}}
    \begin{convention}
        Throughout the section, we work over a fixed algebraically closed field $k$ of characteristic $0$.
    \end{convention}

    \subsection{Structure and dimension theories for fusion categories}
        \subsubsection{Definition, global dimensions, and Oceanu Rigidity}
            \begin{definition}[Fusion categories] \label{def: fusion_categories}
                A \textbf{(multi-)fusion category} over $k$ is a finite semi-simple $k$-linear (multi-)tensor category $(\calV, \tensor, \1)$ with\footnote{Note that this condition is redundant for fusion categories, since we already have $\End_{\calV}(\1) \cong k$ when $(\calV, \tensor, \1)$ is a tensor category.} $\End_{\calV}(\1) \cong k$.
            \end{definition}
            \begin{remark}[Semi-simplicity of monoidal units of multi-fusion categories] \label{remark: semi_simplicity_of_monoidal_units_of_multi_fusion_categories}
                Suppose that $(\calV, \tensor, \1)$ is a multi-fusion category over $k$ and consider the monoidal unit $\1$. Then, it is easy to see that $\1$ is necessarily semi-simple. 
            \end{remark}
            
        \subsubsection{Duality}
        
        \subsubsection{Pseudo-unitary fusion categories}
        
        \subsubsection{Integral and weakly integral fusion categories}

    \subsection{Tannakian categories and Deligne's theorem}
    
    \subsection{Group-theoretic fusion categories}
            
            \section{Fusion categories over characteristics \texorpdfstring{$p > 0$}{}}
            
            \input{Chapters/Fusion categories/braided fusion categories}
        
        \begin{appendices}
            \chapter{Structure of finite tensor categories}
                \begin{abstract}
                
                \end{abstract}
                
                \minitoc
            
                \section{Tensor categories}
    \subsection{Tensor categories}
        \begin{convention}
            Fix an algebraically closed field $k$.
        \end{convention}
        
        \subsubsection{Multi-tensor categories}
            \begin{definition}[Multi-tensor categories] \label{def: mutli_tensor_categories}
                A \textbf{multi-tensor category} over $k$ is a locally finite $k$-linear abelian rigid monoidal category $(\calV, \tensor, \1)$ wherein the bifunctor $\tensor$ is $k$-bilinear on morphisms, i.e. for every triple of objects $V, V', V'' \in \Ob(\calV)$, the induced morphism $\tensor_{V, V', V''}: \calV(V, V') \x \calV(V', V'') \to \calV(V, V'')$ will be $k$-bilinear if it exists. If we have - in addition - that $\End_{\calV}(\1) \cong k$ then $(\calV, \tensor, \1)$ will be known as a \textbf{tensor category} over $k$.
            \end{definition}
            \begin{example}
                \noindent
                \begin{itemize}
                    \item The category $k\-\Vect^{\fin}$ of finite-dimensional $k$-vector spaces is a tensor category\footnote{In fact, this is a fusion category over $k$ (cf. definition \ref{def: fusion_categories}).} over $k$.
                    \item This is definitely not the case for the category $k\-\Vect$ of all $k$-vector spaces, due to the fact that given any pair of infinite-dimensional $k$-vector spaces $V, W$, the hom-space $\Hom_k(V, W)$ would be a finite-dimensional $k$-vector spaces, meaning that $k\-\Vect$ is not even locally finite.
                \end{itemize}
            \end{example}
            \begin{example}
                Let $A$ be a finite-dimensional semi-simple associative and unital algebra over $k$ and consider the category $A\bimod^{\fin}$ of finitely generated $A$-bimodules. 
            \end{example}
            \begin{example}
                \noindent
                \begin{itemize}
                    \item The category $\Rep_k^{\fin}(G)$ of finite-dimensional $k$-linear representation of a group $G$ can also be easily shown to be a tensor category\footnote{This is also a fusion category over $k$. In fact, one ought to view $k\-\Vect^{\fin}$ as the category of finite-dimensional $k$-linear representations of the trivial group $1$.} over $k$. To make a slight generalisation, one could assume that $G$ is an affine group scheme over $\Spec k$ and consider the category $\Vect^{\fin}(X/k)^G$ of $G$-equivariant finite-rank vector bundles on a $k$-scheme $X$ with a $G$-action over $\Spec k$.
                    \item The category $\Rep_k^{\fin}(\g)$ of finite-dimensional $k$-linear representations of a Lie $k$-algebra $\g$ is also a tensor category over $k$, due to being canonically isomorphic to to the category of finite-type left-$\scrU(\g)$-modules. 
                \end{itemize}
            \end{example}
            
            
    
    \subsection{Grothendieck rings and dimension theory for tensor categories}
    
    \subsection{Tensor products of tensor categories}
                
                \input{Chapters/Appendices/Tensor categories/finite tensor categories}
        \end{appendices}
	
	\addcontentsline{toc}{section}{References}
	\printbibliography

\end{document}