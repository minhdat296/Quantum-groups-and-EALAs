\section{Quantisation of infinite-dimensional Lie bialgebras}
    In this section, we discuss how infinite-dimensional Lie bialgebras can also be quantised, thus giving an answer in the affirmative to the question of whether or not all Lie bialgebras can be quantised.
    
    In the first subsection, we discuss the necessary modifications to the quantisation scheme for finite-dimensional Lie bialgebras that have to be made so that the theory can be extended to incorporate general infinite-dimensional Lie bialgebras. For the most part, these will merely be discussions surrounding topological subtleties that arise in the infinite-dimensional setting, as there is little more that one can say at this level of generality.
    
    Then, in the latter subsections, we shall be discussing a method for quantising Lie bialgebras of functions on punctured curves. It will be shown, that due to a connection between such Lie bialgebras and the classification of connected algebraic groups of dimension $1$, one can circumvent the use of Drinfeld associators when quantising such Lie bialgebras, thus making these quantisations particularly explicitly realisable. Historically, this formalism was proposed initially by N. Yu. Reshetikhin, L. A. Takhtajan, and L. D. Faddeev in the context of their work on the quantum inverse scattering method, and thus sometimes bears the name \say{FRT}. However, it is equally common to refer to this procedure as the \say{RTT formalism}, referring to a particularly important relation in the construction of the quantum groups that arise in this manner.

    \subsection{Quantisation of general infinite-dimensional Lie bialgebras}

    \subsection{Factorisation}

    \subsection{Meromorphic classical r-matrices on punctured curves; (co)pseudo-triangularity}
        \begin{convention}
            As we will be relying on the classification of connected algebraic groups of dimension $1$ over an algebraically closed field (as in subsection \ref{subsection: classification_of_1_dimensional_algebraic_groups}), we shall have to assume from now on that the ground field $k$ is algebraically closed. 
        \end{convention}

    \subsection{Quantum groups associated to punctured curves}