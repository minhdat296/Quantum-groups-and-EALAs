\section{Algebras and coalgebras}
    \subsection{Algebras}
        Let us begin by giving a brief recollection of how the notion of algebras over rings is treated traditionally. Recall that for a fixed base commutative ring $k$ (usually a field or $\Z$), an algebra over $k$ is a pair:
            $$(A, \rho_A)$$
        consisting of a ring $A$ and a ring homomorphism with central image:
            $$\rho_A: k \to A$$
        \begin{question}
            Give three non-trivial examples of noncommutative algebras.
        \end{question}
        A $k$-algebra homomorphism:
            $$\varphi: (A, \rho_A) \to (B, \rho_B)$$
        between two $k$-algebras $(A, \rho_A), (B, \rho_B)$ is a ring homomorphism $\varphi: A \to B$ fitting into the following commutative diagram:
            $$
                \begin{tikzcd}
            	A & B \\
            	k
            	\arrow["\varphi", from=1-1, to=1-2]
            	\arrow["{\rho_A}", from=2-1, to=1-1]
            	\arrow["{\rho_B}"', from=2-1, to=1-2]
                \end{tikzcd}
            $$
        \begin{question}
            Prove that a ring homomorphism $\varphi: A \to B$ between two $k$-algebras $(A, \rho_A), (B, \rho_B)$ is a $k$-algebra homomorphism if and only if it is $k$-linear. Show that when $k \cong \Z$, we recover the usual notion of ring homomorphisms. Conclude, then, that $\Z$ is an initial object of $\Ring$ (the category of rings).
        \end{question}
        \begin{question}
            Prove that, given a $k$-module $V$, the ring of $k$-linear endomorphisms $\End_k(V)$ (on which multiplication is given by compositions) is a $k$-algebra. Prove also, that any subring of $\End_k(V)$ is also a $k$-algebra and that conversely, given any $k$-algebra, there exist a $k$-module $V$ along with an injective $k$-algebra homomorphism $A \hookrightarrow \End_k(V)$ (i.e. a \say{faithful representation} of $A$). 
        \end{question}
        An $A$-module is a pair:
            $$(V, \varrho_V)$$
        consisting of a $k$-module $V$ along with a $k$-algebra homomorphism, usually called an \say{action} or \say{representation} of $A$ on $V$:
            $$\varrho_V: A \to \End_k(V)$$
        A homomorphism of $A$-modules:
            $$\varphi: (V, \varrho_V) \to (W, \varrho_W)$$
        consists of an underlying $k$-linear map $\varphi: V \to W$ which \say{intertwines} with the actions $\varrho_V$ and $\varrho_W$, which is to say that the following diagram commutes for any $x \in A$:
            $$
                \begin{tikzcd}
            	V & W \\
            	V & W
            	\arrow["\varphi", from=1-1, to=1-2]
            	\arrow["{\varrho_V(x)}"', from=1-1, to=2-1]
            	\arrow["{\varrho_W(x)}", from=1-2, to=2-2]
            	\arrow["\varphi", from=2-1, to=2-2]
                \end{tikzcd}
            $$
        Since $A$ is not assumed to be commutative in general, we need to make a distinction between so-called \say{left} and \say{right-actions}. Recall that by the universal property of free modules, every module over any ring (hence over any algebra) admits a presentation\footnote{In the literature, sometimes people say \say{presentation} when they really mean \say{free presentation}, which is when the kernel of the quotient map is also free. We will refrain from conflating these two terms.}, which is to say that any module can be written as a quotient of a free module. For instance, for $(V, \rho)$ as above, there exists a set $I$ and a surjective $A$-module homomorphism:
            $$\pi: A^{\oplus I} \to V$$
        which allows us to regard elements of $V$ as $I$-tuples:
            $$(x_i)_{i \in I}, x_i \in A$$
        that vanish if and only if they lie in $\ker \pi$; elements of $\ker \pi$ are called the \say{relations} defining $V$. A left/right-action of $A$ on $V$ is then specified by whether elements of $A$ scalar-multiply into the tuples $(x_i)_{i \in I}$ from the left or the right. If both occurs simultaneously, we say that $V$ is a two-sided module or $(A, A)$-bimodule. Similarly, one can even define $(A, B)$-bimodules for two possibly different $k$-algebras $A, B$, with it being understood that $A$ acts from the left and $B$ acts from the right.
        \begin{convention}
            For an algebra $A$, the categories of left-$A$-modules (respectively, of right-$A$-modules) shall be denoted by:
                $${}^lA\mod$$
            (respectively, by ${}^rA\mod$).
        \end{convention}
        \begin{question}
            Let $A$ be a $k$-algebra and $V, W$ be two left/right-$A$-modules. Prove that $\Mor_{ {}^lA\mod }(V, W)$, the set of left/right-$A$-module homomorphisms $V \to W$ is generally only a $k$-module, not a left/right-$A$-module, unless $A$ is commutative, in which case $\Mor_{ {}^lA\mod }(V, W)$ will automatically be a two-sided $A$-module.

            We will usually write $\Mor_{ {}^lA\mod }$ instead of $\Hom_A$ when we need to put emphasis on this fact. 
        \end{question}
        \begin{question}
            Let $A$ be a $k$-algebra and $(V, \varrho_V)$ be a left-$A$-module.
            \begin{enumerate}
                \item Prove that the full $k$-linear dual:
                    $$V^* := \Hom_k(V, k)$$
                has a natural right-$A$-module structure induced by $\varrho_V$ (and \textit{vice versa}).
                \item Now, suppose that the underlying $k$-module $V$ is graded by some abelian group $\Gamma$, i.e. we can decompose $V$ into $k$-submodules as:
                    $$V := \bigoplus_{\gamma \in \Gamma} V_{\gamma}$$
                Prove that the graded dual:
                    $$V^{\star} := \bigoplus_{\gamma \in \Gamma} V_{\gamma}^*$$
                also has a natural right-$A$-module structure induced by $\varrho_V$ (and \textit{vice versa}). If, moreover, $A$ is a $\Gamma$-graded algebra - meaning that $A := \bigoplus_{\gamma} A_{\gamma}$ as a $k$-module and $A_{\gamma} A_{\gamma'} \subseteq A_{\gamma \gamma'}$ for all $\gamma, \gamma' \in \Gamma$ - and if $\varrho_V: A \to \End_k(V)$ is a $\Gamma$-graded action - meaning that $A_{\gamma} V_{\lambda} \subseteq V_{\gamma + \lambda}$ for all $\gamma, \lambda \in \Gamma$ - then the aforementioned right-$A$-module structure on $V^{\star}$ will also be $\Gamma$-graded.
            \end{enumerate}
        \end{question}
        As such, we see that for any $k$-algebra $A$, there exists the (linear) duality and graded duality functors:
            $$(-)^*, (-)^{\star}: {}^lA\mod \to {}^rA\mod$$
        (and \textit{vice versa}, going from right-modules to left-modules), and it is usually a fruitful endeavour to investigate when these functors are equivalences of categories. The famous duality principle of Beilinson-Gel'fand-Gel'fand (BGG) is along these lines (see e.g. \cite[Chapter 3]{humphreys_category_O} and \cite[Sections 2.6 and 6.7]{moody_pianzola_lie_algebras_with_triangular_decompositions}). We would also like to remark that in practice, it is very common for $k$-algebras to be infinitely generated as $k$-modules. Considering graded duals is one way to get around that infinitely generated modules are not isomorphic to their full linear duals, and even this strategy is reliant on whether or not the graded components are finitely generated over $k$.
        \begin{example}
            Highest-weight modules over a finite-dimensional semi-simple Lie algebra $\g$ over an algebraically closed field of characteristic $0$ are left-modules. In fact, they are graded as left-$\g$-modules\footnote{Or, to not abuse terminologies, left-$\rmU(\g)$-modules, with $\rmU(\g)$ being the universal enveloping algebra of $\g$.} by the root lattice of $\g$ (or more crudely, by the underlying abelian group of the linear dual of \say{the} Cartan subalgebra of $\g$). As such, one usually considers graded duals of highest-weight $\g$-modules, as opposed to full linear duals.  
        \end{example}
        
        Next, let us discuss the matter of tensor products over noncommutative rings, which is rather less straightforward than what happens over commutative rings. Given two $A$-modules $(V, \varrho_V), (W, \varrho_W)$, one can first of all form their \say{tensor product} over the previously fixed commutative ring $k$, $V \tensor_k W$, and by definition, this is the object of $k\mod$ that corepresents the functor:
            $$\Bil_k(V \x W, -): k\mod \to \Sets$$
        mapping $k$-modules $P \in \Ob(k\mod)$ to the set $\Bil_k(V \x W, P)$ of $k$-bilinear maps $V \x W \to P$, i.e. we have a natural isomorphism of functors $k\mod \to \Sets$:
            $$\Bil_k(V \x W, -) \xrightarrow[]{\cong} \Hom_k(V \tensor_k W, -)$$
        (cf. \cite[\href{https://stacks.math.columbia.edu/tag/00CX}{Tag 00CX}]{stacks-project}). If $A$ is a commutative algebra, then we can just repeat the construction above, replacing $k$ with $A$, but if $A$ is noncommutative, then we will need to be careful. If $(V, \varrho_V), (W, \varrho_W)$ are two left-$A$-modules, then on the $k$-module $V \tensor_k W$, there will be a naturally induced left-$A$-module structure:
            $$\varrho_{V \tensor_k W}: A \to \End_k(V \tensor_k W)$$
        given by:
            $$\varrho_{V \tensor_k W}(x) \cdot (v \tensor w) := (\varrho_V(x) \cdot v) \tensor w = v \tensor (\varrho_W(x) \cdot w)$$
        On the other hand, if we work with right-modules, then the action will be given by:
            $$\varrho_{V \tensor_k W}(x) \cdot (v \tensor w) := v \tensor (w \cdot \varrho_W(x)) = (v \cdot \varrho_W(x)) \tensor w$$
        When there is no ambiguity as to whether $V \tensor_k W$ is a left or right-$A$-module, we will write:
            $$V \tensor_A W$$
        to denote $(V \tensor_k W, \rho_{V \tensor_k W})$ defined as above. In either case, one crucial detail that distinguishes tensor products over noncommutative rings and over commutative is that when $A$ is non-commutative, we generally have:
            $$V \tensor_A W \not \cong W \tensor_A V$$
        This is an important example of an asymmetric monoidal structure, namely on the category of left/right-$A$-modules.
        \begin{question}
            Prove that the monoidal unit, as one would expect, is the algebra $A$ itself. Namely, prove that there are $A$-module isomorphisms:
                $$V \cong V \tensor_A A \cong A \tensor_A V \cong V$$
            for any $A$-module $V$.
        \end{question}
        \begin{question}
            Let $A, B$ be two $k$-algebras and let $V$ be an $A$-module, while $W$ be a $B$-module. Prove that:
            \begin{enumerate}
                \item if $V$ is a left-$A$-module while $W$ is a right-$B$-module, then $V \tensor_k W$ will have a naturally induced $(A, B)$-bimodule structure, and
                \item if $V$ is a right-$A$-module while $W$ is a left-$B$-module, then $V \tensor_k W$ will have a naturally induced $(B, A)$-bimodule structure.
            \end{enumerate}
            In these cases, we may write $V \tensor_{A, B} W$ and $V \tensor_{B, A} W$, respectively. Note that even when $A = B$, if $V, W$ do not have the same directionality then $V \tensor_A W \not \cong V \tensor_{A, A} W$, for the former may even be ill-defined (explain why).
        \end{question}
        One of the most important facts about tensor products is that they give rise to an adjunction, usually called the tensor-hom adjunction.
        \begin{question}[Tensor-hom adjunction]
            Let $A, B$ be two $k$-algebras and let $U$ be a left-$A$-module, $V$ be an $(A, B)$-bimodule, and $W$ be a left-$B$-module. Prove that there is a $k$-module isomorphism as follows, natural in $U, V, W$:
                $$\Mor_{ {}^lB\mod }(U \tensor_A V, W) \cong \Mor_{ {}^lA\mod }(U, \Mor_{ {}^lB\mod }(V, W))$$
            State and prove the version for right-modules.
        \end{question}
        This result allows us to give the following important diagrammatic descriptions of algebras and modules over them. 
        \begin{question}
            \begin{enumerate}
                \item Prove that to give a $k$-algebra $A$ is the same as to give $k$-linear maps:
                    $$\mu: A^{\tensor 2} \to A$$
                    $$\eta: k \to A$$
                called the \say{multiplication} and \say{unit} maps that fit into the following commutative diagrams in $k\mod$:
                \begin{itemize}
                    \item \textbf{(Associativity):}
                        $$
                            \begin{tikzcd}
                        	{A^{\tensor 2} \tensor_k A} && {A \tensor_k A^{\tensor 2}} \\
                        	{A^{\tensor 2}} && {A^{\tensor 2}} \\
                        	& A
                        	\arrow["{\alpha_{A, A, A}}", from=1-1, to=1-3]
                        	\arrow["{\mu \tensor \id_A}"', from=1-1, to=2-1]
                        	\arrow["{\id_A \tensor \mu}", from=1-3, to=2-3]
                        	\arrow["\mu"', from=2-1, to=3-2]
                        	\arrow["\mu", from=2-3, to=3-2]
                            \end{tikzcd}
                        $$
                    wherein $\alpha_{A, A, A}: A \tensor_k A^{\tensor 2} \xrightarrow[]{\cong} A^{\tensor 2}$ is the canonical isomorphism arising from the associativity of tensor products, usually called the \say{associator} for the triple of objects $(A, A, A)$. 
                    \item \textbf{(Left and right-unitality):}
                        $$
                            \begin{tikzcd}
                        	{k \tensor_k A} & {A^{\tensor 2}} & {A \tensor_k k} \\
                        	& A
                        	\arrow["{\eta \tensor \id_A}", from=1-1, to=1-2]
                        	\arrow["\cong"', from=1-1, to=2-2]
                        	\arrow["\mu"{description}, from=1-2, to=2-2]
                        	\arrow["{\id_A \tensor \eta}"', from=1-3, to=1-2]
                        	\arrow["\cong", from=1-3, to=2-2]
                            \end{tikzcd}
                        $$
                    wherein the unlabelled isomorphisms are the canonical ones. 
                \end{itemize}
                If the algebra is commutative, then the following commutative diagram in $k\mod$ must also hold:
                    $$
                        \begin{tikzcd}
                    	{A^{\tensor 2}} && {A^{\tensor 2}} \\
                    	& A
                    	\arrow["{\tau_{A, A}}", from=1-1, to=1-3]
                    	\arrow["\mu"', from=1-1, to=2-2]
                    	\arrow["\mu", from=1-3, to=2-2]
                        \end{tikzcd}
                    $$
                wherein $\tau_{A, A}: A^{\tensor 2} \xrightarrow[]{\cong} A^{\tensor 2}$ is the isomorphism coming from the commutativity of tensor products over commutative rings.
                \item Prove that to give a left-action $A \to \End_k(V)$ of a $k$-algebra $A$ on a $k$-module $V$ is equivalent to giving a left-$A$-module homomorphism:
                    $$\varrho_V: A \tensor_k V \to V$$
                fitting into the following commutative diagrams in $k\mod$:
                \begin{itemize}
                    \item \textbf{(Left-associativity):}
                        $$
                            \begin{tikzcd}
                        	{A^{\tensor 2} \tensor_k V} && {A \tensor_k (A \tensor_k V)} \\
                        	{A \tensor_k V} && {A \tensor_k V} \\
                        	& V
                        	\arrow["{\alpha_{A, A, V}}", from=1-1, to=1-3]
                        	\arrow["{\mu \tensor \id_V}"', from=1-1, to=2-1]
                        	\arrow["{\id_A \tensor \varrho_V}", from=1-3, to=2-3]
                        	\arrow["{\varrho_V}"', from=2-1, to=3-2]
                        	\arrow["{\varrho_V}", from=2-3, to=3-2]
                            \end{tikzcd}
                        $$
                    wherein $\alpha_{A, A, V}$ is the associator isomorphism for the triple of objects $(A, A, V)$.
                    \item \textbf{(Left-unitality):}
                        $$
                            \begin{tikzcd}
                        	{k \tensor_k V} & {A \tensor_k V} \\
                        	& V
                        	\arrow["{\eta \tensor \id_V}", from=1-1, to=1-2]
                        	\arrow["\cong"', from=1-1, to=2-2]
                        	\arrow["{\varrho_V}", from=1-2, to=2-2]
                            \end{tikzcd}
                        $$
                    wherein the unlabelled isomorphisms are the canonical ones.
                \end{itemize}
                    
                State and prove the version for right-modules.
            \end{enumerate}
        \end{question}

        This leads us to the following \textit{vast} abstraction of the notion of algebras over commutative rings, internal to any monoidal category.
        \begin{definition}[Algebras in monoidal categories] \label{def: algebras_in_monoidal_categories}
            Let $(\C, \tensor, \1, \alpha)$ be a monoidal category. The subcategory:
                $$\Assoc\Alg(\C)$$
            (or $\Assoc\Alg(\C, \tensor, \1)$, to put emphasis on the monoidal structure) whose objects are \textbf{algebra objects} and whose morphisms are \textbf{algebra homomorphisms} is the category consisting of objects:
                $$A \in \Ob(\C)$$
            equipped with morphisms:
                $$\mu: A^{\tensor 2} \to A$$
                $$\eta: \1 \to A$$
            called the \textbf{multiplication} and \textbf{counit}, respectively, fitting into the following commutative diagrams:
            \begin{itemize}
                \item \textbf{(Associativity):}
                    $$
                        \begin{tikzcd}
                        {A^{\tensor 2} \tensor A} && {A \tensor A^{\tensor 2}} \\
                        {A^{\tensor 2}} && {A^{\tensor 2}} \\
                        & A
                        \arrow["{\alpha_{A, A, A}}", from=1-1, to=1-3]
                        \arrow["{\mu \tensor \id_A}"', from=1-1, to=2-1]
                        \arrow["{\id_A \tensor \mu}", from=1-3, to=2-3]
                        \arrow["\mu"', from=2-1, to=3-2]
                        \arrow["\mu", from=2-3, to=3-2]
                        \end{tikzcd}
                    $$
                wherein $\alpha_{A, A, A}: A \tensor A^{\tensor 2} \xrightarrow[]{\cong} A^{\tensor 2}$ is the canonical isomorphism arising from the associativity of tensor products, usually called the \say{associator} for the triple of objects $(A, A, A)$. 
                \item \textbf{(Left and right-unitality):}
                    $$
                        \begin{tikzcd}
                        {\1 \tensor A} & {A^{\tensor 2}} & {A \tensor \1} \\
                        & A
                        \arrow["{\eta \tensor \id_A}", from=1-1, to=1-2]
                        \arrow["\cong"', from=1-1, to=2-2]
                        \arrow["\mu"{description}, from=1-2, to=2-2]
                        \arrow["{\id_A \tensor \eta}"', from=1-3, to=1-2]
                        \arrow["\cong", from=1-3, to=2-2]
                        \end{tikzcd}
                    $$
                wherein the unlabelled isomorphisms are the canonical ones. 
            \end{itemize}
            If the monoidal structure on $\C$ is moreover symmetric, then we can also define \textbf{commutative algebra objects} by furthermore requiring that the following commutative diagram holds:
                $$
                    \begin{tikzcd}
                    {A^{\tensor 2}} && {A^{\tensor 2}} \\
                    & A
                    \arrow["{\tau_{A, A}}", from=1-1, to=1-3]
                    \arrow["\mu"', from=1-1, to=2-2]
                    \arrow["\mu", from=1-3, to=2-2]
                    \end{tikzcd}
                $$
            wherein $\tau_{A, A}: A^{\tensor 2} \xrightarrow[]{\cong} A^{\tensor 2}$ is the symmetriser for the pair of objects $(A, A)$. The subcategorry of $\Assoc\Alg(\C)$ spanned by commutative algebra objects is denoted by:
                $$\Comm\Alg(\C)$$
        \end{definition}
        \begin{question}
            Let $(\C, \tensor, \1, \alpha)$ be a monoidal category. Show that $\Assoc\Alg(\C)$ is a non-full subcategory of $\C$, and that $\Comm\Alg(\C)$ is a full subcategory of $\Assoc\Alg(\C)$.
        \end{question}

        \begin{definition}[Modules over algebras in monoidal categories] \label{def: modules_over_algebras_in_monoidal_categories}
            Let $(\C, \tensor, \1, \alpha)$ be a monoidal category and let $(A, \mu, \eta)$ be an algebra object thereof. The subcategory:
                $${}^lA\mod$$
            of $\C$ whose objects are \textbf{left-$A$-modules} and whose morphisms are \textbf{left-$A$-module homomorphisms} is spanned by objects:
                $$V \in \Ob(\C)$$
            equipped with an \textbf{left-action} morphism:
                $$\varrho_V: A \tensor V \to V$$
            fitting into the following commutative diagrams:
            \begin{itemize}
                \item \textbf{(Left-associativity):}
                    $$
                        \begin{tikzcd}
                        {A^{\tensor 2} \tensor V} && {A \tensor (A \tensor V)} \\
                        {A \tensor V} && {A \tensor V} \\
                        & V
                        \arrow["{\alpha_{A, A, V}}", from=1-1, to=1-3]
                        \arrow["{\mu \tensor \id_V}"', from=1-1, to=2-1]
                        \arrow["{\id_A \tensor \varrho_V}", from=1-3, to=2-3]
                        \arrow["{\varrho_V}"', from=2-1, to=3-2]
                        \arrow["{\varrho_V}", from=2-3, to=3-2]
                        \end{tikzcd}
                    $$
                \item \textbf{(Left-unitality):}
                    $$
                        \begin{tikzcd}
                        {\1 \tensor V} & {A \tensor V} \\
                        & V
                        \arrow["{\eta \tensor \id_V}", from=1-1, to=1-2]
                        \arrow["\cong"', from=1-1, to=2-2]
                        \arrow["{\varrho_V}", from=1-2, to=2-2]
                        \end{tikzcd}
                    $$
            \end{itemize}
        \end{definition}
        \begin{question}
            Define right-modules over algebra objects in a general monoidal category, following definition \ref{def: modules_over_algebras_in_monoidal_categories}.
        \end{question}

        \begin{lemma}
            
        \end{lemma}
            \begin{proof}
                
            \end{proof}

    \subsection{Coalgebras and bialgebras}
        \begin{definition}[Coalgebras in monoidal categories] \label{def: coalgebras_in_monoidal_categories}
            \todo[inline]{Coalgebras in monoidal categories}
        \end{definition}
        
        \begin{definition}[Bialgebras in monoidal categories] \label{def: bialgebras_in_monoidal_categories}
            \todo[inline]{Bialgebras in monoidal categories}
        \end{definition}

        \begin{definition}[Hopf algebras in monoidal categories] \label{def: hopf_algebras_in_monoidal_categories}
            \todo[inline]{Hopf algebras in monoidal categories}
        \end{definition}

        \begin{lemma}[Tensor products of modules over Hopf algebras] \label{lemma: tensor_products_of_modules_over_hopf_algebras}
            
        \end{lemma}
            \begin{proof}
                
            \end{proof}

        \begin{example}[Cocommutative bialgebras from affine group schemes]
            
        \end{example}

        \begin{example}[Quantum groups: non-cocommutative bialgebras]
            
        \end{example}