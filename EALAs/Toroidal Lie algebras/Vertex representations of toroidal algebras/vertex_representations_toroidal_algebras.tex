\section{Vertex representations of simply laced toroidal Lie algebras}
    \subsection{Chevalley-Serre presentation for \texorpdfstring{$\toroidal$}{}}
        We begin by demonstrating how the toroidal Lie algebra $\toroidal := \uce(\tilde{\g}[t^{\pm 1}])$ admits a description as a Lie algebra generated by a set:
            $$\{x_{i, r}^{\pm}, h_{i, r}\}_{(i, r) \in \hat{\simpleroots} \x \Z}$$
        subjected to relations in the style of Chevalley-Serre; this can be seen as an affinisation of the Chevalley-Serre presentation for the affine Lie algebra $\tilde{\g}$. Then, we will identify the Heisenberg algebra $\tilde{\h}_{\hat{Q}}$ associated with the affine root lattice $\hat{Q}$ with the Lie subalgebra of $\toroidal$ generated by the elements $h_{i, r}$. We will also be giving descriptions of these Lie algebras in terms of formal distributions, which will be useful for the calculations with vertex operators later on. 

    \subsection{\texorpdfstring{$\toroidal$}{}-action on \texorpdfstring{$\V_{\hat{Q}}$}{}}
        Since we now know that there is a Lie algebra embedding of $\tilde{\h}_{\hat{Q}}$ into $\toroidal$, let us investigate whether the toroidal Lie algebra $\toroidal$ may act on the $\tilde{\h}_{\hat{Q}}$-module $\V_{\hat{Q}}$, much like how the affine Lie algebra $\tilde{\g}$ is known to act on $\V_Q$ by Frenkel-Kac-Segal. We will accomplish this by relying on the fact that $\V_{\hat{Q}}$ has a (unique) vertex algebra structure\footnote{Such a construction exists for all (even) lattices, not just $\hat{Q}$.} 