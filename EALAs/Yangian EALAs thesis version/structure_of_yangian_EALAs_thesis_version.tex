\section{Structure of \texorpdfstring{$\gamma$}{}-extended toroidal Lie algebras}
    \subsection{Centres of \texorpdfstring{$\gamma$}{}-extended toroidal Lie algebras}
        Fix an arbitrary $\gamma$-extended toroidal Lie algebra $\extendedtoroidal$.
    
        Using lemma \ref{lemma: yangian_div_zero_vector_fields_basic_properties} in conjunction with lemma \ref{lemma: derivation_action_on_toroidal_centres}, we can now also explicitly compute the commutation relations between the basis elements of $\divzero$ and $\z(\toroidal)$.
        \begin{lemma}[Explicit commutators between basis elements of $\divzero$ and $\z(\toroidal)$] \label{lemma: explicit_commutators_between_central_basis_elements_and_derivations}
            In the Lie algebra $\extendedtoroidal$, one has the following commutation relations between elements of $\z(\toroidal)$ and those of $\divzero$. Namely, for all $D \in \divzero$, the following relations hold:
                $$
                    \forall (a, b) \in \Z^2: [D, K_{a, b}]_{\extendedtoroidal} =
                    \begin{cases}
                        \text{$((b - 1)r - sa) K_{a - r, b - s - 1} + \delta_{(r, s + 1), (a, b)} \left( r c_v + s c_t \right)$ if $D = D_{r, s}$}
                        \\
                        \text{$a K_{a, b - 1}$ if $D_v$}
                        \\
                        \text{$b K_{a, b - 1}$ if $D_t$}
                    \end{cases}
                $$
                $$[D, c_v]_{\extendedtoroidal} = [D, c_t]_{\extendedtoroidal} = 0$$
        \end{lemma}
            \begin{proof}
                For this, we shall be making use of invariance again, namely:
                    $$(D, [D', K]_{\extendedtoroidal})_{\extendedtoroidal} = ([D, D']_{\extendedtoroidal}, K)_{\extendedtoroidal}$$
                for all $D, D' \in \divzero$ and all $K \in \z(\toroidal)$, and how the brackets $[D, D']_{\extendedtoroidal}$ are given explicitly (cf. lemma \ref{lemma: yangian_div_zero_vector_fields_basic_properties}) as well as how basis elements $K \in \z(\toroidal)$ pair with basis elements of $\divzero$ in the construction of $(-, -)_{\extendedtoroidal}$. Without any loss of generality, we can assume that $D, D' \in \divzero$ and $K \in \z(\toroidal)$ are basis elements, i.e.:
                    $$D, D' \in \{D_{r, s}\}_{(r, s) \in \Z^2} \cup \{D_v, D_t\}$$
                    $$K \in \{K_{a, b}\}_{(a, b) \in \Z^2} \cup \{c_v, c_t\}$$
                and then perform the computations case-by-case, for which we shall recall from example \ref{example: toroidal_lie_algebras_centres} that:
                    $$
                        K_{a, b} :=
                        \begin{cases}
                            \text{$\frac1b v^{a - 1} t^b \bar{d}v$ if $(a, b) \in \Z \x (\Z \setminus \{0\})$}
                            \\
                            \text{$-\frac1a v^a t^{-1} \bar{d}t$ if $(a, b) \in (\Z \setminus \{0\}) \x \{0\}$}
                            \\
                            \text{$0$ if $(a, b) = (0, 0)$}
                        \end{cases}
                    $$
                    $$c_v := v^{-1} \bar{d}v, c_t := t^{-1} \bar{d}t$$
                and from lemma \ref{lemma: yangian_div_zero_vector_fields_basic_properties}, that:
                    $$[D_v, D_t] = 0$$
                    $$[D_v, D_{r, s}] = r D_{r, s + 1}$$
                    $$[D_t, D_{r, s}] = D_{r, s + 1}$$
                    $$[D_{\alpha, \beta}, D_{r, s}] = (\beta r - s \alpha) D_{\alpha + r, \beta + s + 1}$$
                For what follows, let:
                    $$D := \sum_{(\alpha, \beta) \in \Z^2} \lambda_{\alpha, \beta} D_{\alpha, \beta} + \lambda_v D_v + \lambda_t D_t$$
                for some $\lambda_{\alpha, \beta}, \lambda_v, \lambda_t \in \bbC$.
                \begin{enumerate}
                    \item Assume firstly that $K = K_{a, b}$.
                    \begin{enumerate}
                        \item If $D' = D_{r, s}$, then we shall have that:
                            $$
                                \begin{aligned}
                                    & ( D, [D_{r, s}, K_{a, b}]_{\extendedtoroidal} )_{\extendedtoroidal}
                                    \\
                                    = & ( [D, D_{r, s}]_{\extendedtoroidal}, K_{a, b} )_{\extendedtoroidal}
                                    \\
                                    = & \sum_{(\alpha, \beta) \in \Z^2} (\beta r - s \alpha) \lambda_{\alpha, \beta} \delta_{(\alpha + r, \beta + s + 1), (a, b)} + \delta_{(r, s + 1), (a, b)} \left( r\lambda_v + s\lambda_t \right)
                                    \\
                                    = & ((b - s - 1) r - s (a - r)) \lambda_{a - r, b - s - 1} + \delta_{(r, s + 1), (a, b)} \left( r\lambda_v + s\lambda_t \right)
                                    \\
                                    = & ((b - 1)r - sa) \lambda_{a - r, b - s - 1} + \delta_{(r, s + 1), (a, b)} \left( r\lambda_v + s\lambda_t \right)
                                \end{aligned}
                            $$
                        from which we are able to conclude that:
                            $$[D_{r, s}, K_{a, b}]_{\extendedtoroidal} = ((b - 1)r - sa) K_{a - r, b - s - 1} + \delta_{(r, s + 1), (a, b)} \left( r c_v + s c_t \right)$$
                        \item If $D' = D_v$, then:
                            $$
                                \begin{aligned}
                                    & ( D, [D_v, K_{a, b}]_{\extendedtoroidal} )_{\extendedtoroidal}
                                    \\
                                    = & ( [D, D_v]_{\extendedtoroidal}, K_{a, b} )_{\extendedtoroidal}
                                    \\
                                    = & \sum_{(\alpha, \beta) \in \Z^2} \alpha \lambda_{\alpha, \beta} \delta_{(\alpha, \beta + 1), (a, b)} 
                                    \\
                                    = & a \lambda_{a, b - 1}
                                \end{aligned}
                            $$
                        from which we are able to conclude that:
                            $$[D_v, K_{a, b}]_{\extendedtoroidal} = aK_{a, b - 1}$$   
                        \item Finally, if $D' = D_t$, then:
                            $$
                                \begin{aligned}
                                    & ( D, [D_t, K_{a, b}]_{\extendedtoroidal} )_{\extendedtoroidal}
                                    \\
                                    = & ( [D, D_t]_{\extendedtoroidal}, K_{a, b} )_{\extendedtoroidal}
                                    \\
                                    = & \sum_{(\alpha, \beta) \in \Z^2} \beta \lambda_{\alpha, \beta} \delta_{(\alpha, \beta + 1), (a, b)} 
                                    \\
                                    = & b \lambda_{a, b - 1}
                                \end{aligned}
                            $$
                        from which we are able to conclude that:
                            $$[D_t, K_{a, b}]_{\extendedtoroidal} = b K_{a, b - 1}$$
                    \end{enumerate}
                    \item If $K = c_v$ or $K = c_t$ then simply note that because:
                        $$[D, D']_{\extendedtoroidal} \in \bigoplus_{(\alpha, \beta) \in \Z^2} \bbC D_{\alpha, \beta}$$
                    for all $D' \in \divzero$, we shall have that:
                        $$[D, K]_{\extendedtoroidal} = 0$$
                    for all $D := \sum_{(\alpha, \beta) \in \Z^2} \lambda_{\alpha, \beta} D_{\alpha, \beta} + \lambda_v D_v + \lambda_t D_t \in \divzero$.
                \end{enumerate}
            \end{proof}

        \todo[inline]{Rewrote proof of centre being $\bbC c_v \oplus \bbC c_t$.}
        \begin{proposition}[Centres of $\gamma$-extended toroidal Lie algebras] \label{prop: centres_of_yangian_extended_toroidal_lie_algebras}
            The centre of $\extendedtoroidal$ is given by:
                $$\z(\extendedtoroidal) = \bbC c_v \oplus \bbC c_t$$
        \end{proposition}
            \begin{proof}
                From example \ref{example: toroidal_lie_algebras_centres} and lemma \ref{lemma: explicit_commutators_between_central_basis_elements_and_derivations}, one sees that:
                    $$c_v, c_t \in \z(\extendedtoroidal)$$
                and hence:
                    $$\z(\extendedtoroidal) \supset \bbC c_v \oplus \bbC c_t$$
                To show the other containment, it shall suffice - due to lemma \ref{lemma: explicit_commutators_between_central_basis_elements_and_derivations} - to demonstrate that $\z(\extendedtoroidal) \subset \z(\toroidal)$; in fact, because $[\toroidal, \z(\toroidal)]_{\extendedtoroidal} = [\toroidal, \z(\toroidal)]_{\toroidal} = 0$ and because $\g_{[2]}$ is centre-less, it shall suffice to show that there does \textit{not} exist $D \in \divzero$ such that:
                    $$[D, \g_{[2]}]_{\extendedtoroidal} = 0$$
                To this end, recall from lemma \ref{lemma: no_polynomial_terms_for_derivation_action_on_multiloop_algebras} that because $\extendedtoroidal$ is a $\gamma$-extended toroidal Lie algebra, we have that:
                    $$[D, \g_{[2]}]_{\extendedtoroidal} = \rho(D) \cdot \g_{[2]}$$
                with $\rho: \divzero \to \der(\toroidal)$ as in corollary \ref{coro: a_fixed_yangian_div_zero_vector_field_action}. Because of this, all we need to do now is to prove that there does not exist $D \in \divzero$ so that:
                    $$\forall f \in A: D(f) = 0$$
                Suppose for the sake of deriving a contradiction that such an element $D \in \divzero$ does exist. This would imply that for all $f, g \in A$, one would have that:
                    $$(D, g \bar{d}f)_{\extendedtoroidal} = \gamma( g D(f) ) = 0$$
                meaning that such an element $D \in \divzero$ must be orthogonal to every element of $\z(\toroidal)$ (since $\z(\toroidal)$ is spanned by elements of the form $g \bar{d}f$). But this is clearly false, because of the construction of the bilinear form $(-, -)_{\extendedtoroidal}$ (cf. corollary \ref{coro: pairing_yangian_div_zero_vector_fields_and_cyclic_1_forms}), and so we have a contradiction. Therefore:
                    $$\z(\extendedtoroidal) \subset \bbC c_v \oplus \bbC c_t$$
                as well, and hence:
                    $$\z(\extendedtoroidal) = \bbC c_v \oplus \bbC c_t$$
                as claimed.
            \end{proof}
        \begin{remark}
            It is rather interesting that:
                $$\z(\extendedtoroidal) \cong \bbC c_v \oplus \bbC c_t$$
            as this is in good analogy with the affine Kac-Moody case, where the centre of $\hat{\g}$ is $1$-dimensional, namely spanned by $c_v$ (cf. example \ref{example: affine_lie_algebras_centres}).
        \end{remark}

    \subsection{The Witt algebra}
        Recall from lemma \ref{lemma: yangian_div_zero_vector_fields_basic_properties} that:
            $$\forall (r, s) \in \Z^2: D_{r, s} = s v^{-r + 1} t^{-s - 1} \del_v - r v^{-r} t^{-s} \del_t$$
            $$D_v = -v t^{-1} \del_v$$
            $$D_t = -\del_t$$
        and from lemma \ref{lemma: yangian_div_zero_vector_fields_basic_properties} that the commutation relations that these basis elements of $\divzero$ satisfy are:
            $$[D_v, D_t] = 0$$
            $$[D_v, D_{r, s}] = r D_{r, s + 1}$$
            $$[D_t, D_{r, s}] = D_{r, s + 1}$$
            $$[D_{a, b}, D_{r, s}] = (br - sa) D_{a + r, b + s + 1}$$
        (given for all $(r, s), (a, b) \in \Z^2$). With these information in mind, one sees that the following vector subspace of $\divzero$:
            $$\frakw := \bigoplus_{r \in \Z} \bbC D_{r, -1}$$
        is actually a Lie subalgebra, as the basis elements satisfy the following commutators:
            $$[D_{a, -1}, D_{r, -1}] = (a - r) D_{a + r, -1}$$
        given for all $a, r \in \Z$; note also that we have $D_v, D_t \not \in \frakw$ because:
            $$[D_v, D_{r, -1}] = r D_{r, 0} \not \in \frakw$$
            $$[D_t, D_{r, -1}] = D_{r, 0} \not \in \frakw$$    
        for all $r \in \Z$. Interestingly, these are precisely the commutation relations satisfied by the elements of the following basis of the Lie algebra $\der(\bbC[v^{\pm 1}])$:
            $$\{ d_r := -v^r D_{\aff} \}_{r \in \Z}$$
        (where $D_{\aff} := v \frac{d}{dv}$ is the \say{untwisted affine Kac-Moody derivation} as in subsection \ref{subsection: a_fixed_untwisted_affine_kac_moody_algebra}) and in light of this, we make the following observation:
        \begin{lemma}[A copy of the Witt algebra inside $\divzero$] \label{lemma: a_copy_of_the_witt_algebra_inside_the_lie_algebra_of_yangian_div_zero_vector_fields}
            There is an isomorphism of Lie algebras:
                $$\der(\bbC[v^{\pm 1}]) \xrightarrow[]{\cong} \frakw$$
            given by:
                $$d_r \mapsto D_{r, -1}$$
            This identifies a copy of $\der(\bbC[v^{\pm 1}])$ inside $\divzero$ as a Lie subalgebra. 
        \end{lemma}