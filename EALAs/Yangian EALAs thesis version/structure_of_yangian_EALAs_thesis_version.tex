\section{Structure of \texorpdfstring{$\gamma$}{}-extended toroidal Lie algebras}
    \subsection{Centres of \texorpdfstring{$\gamma$}{}-extended toroidal Lie algebras}
        \begin{question}
            What is the centre $\hat{\z}_{[2]} := \z( \extendedtoroidal )$ ? This ought to be smaller than $\z(\toroidal)$ somehow, since elements of $\z(\toroidal)$ need not be central in $\extendedtoroidal$. 
        \end{question}
        \begin{remark}[Computing the centre without computing all the brackets ...]
            Since $\g_{[2]}$ is centreless, we have that:
                $$\hat{\z}_{[2]} = \z( \z(\toroidal) \oplus \divzero )$$
            As $\z(\toroidal)$ is an abelian Lie algebra, this implies that in order to compute $\hat{\z}_{[2]}$, it suffices to explicitly compute the commutators of the form:
                $$[D, K]_{\extendedtoroidal}, [D, D']_{\extendedtoroidal}$$
            for $D, D' \in \divzero$ and $K \in \z(\toroidal)$, to see which ones vanish. However, this is rather tedious and not very insightful.
            
            An alternative method is as follows: exploiting the fact that the symmetric bilinear form $(-, -)_{\extendedtoroidal}$ is both invariant and non-degenerate, we can characterise the centre $\hat{\z}_{[2]}$ as the Lie ideal of $\extendedtoroidal$ containing elements $Z$ such that:
                $$0 = ([Z, X]_{\extendedtoroidal}, Y)_{\extendedtoroidal} = (Z, [X, Y]_{\extendedtoroidal})_{\extendedtoroidal}$$
            for any $X, Y \in \extendedtoroidal$, with the first equality holding thanks to the fact that $Z$ is supposed to commute with every other element of $\extendedtoroidal$ by assumption of being central. We are thus left with the task of finding elements:
                $$Z \in \extendedtoroidal$$
            such that:
                $$(Z, [\extendedtoroidal, \extendedtoroidal]_{\extendedtoroidal})_{\extendedtoroidal} = 0$$
            Since brackets of the form:
                $$[X, Y]_{\extendedtoroidal}, [D, D']_{\extendedtoroidal}$$
            (for some $X, Y \in \g_{[2]}$ and some $D, D' \in \divzero$) are generally non-zero, their elements can not be central in $\extendedtoroidal$. As such, we have narrowed the scope of our search down to:
                $$\hat{\z}_{[2]} \subset \z(\toroidal)$$

            Another way to see that:
                $$\hat{\z}_{[2]} \subset \z(\toroidal)$$
            is to use the fact that $\extendedtoroidal$ is a Lie algebra extension of $\divzero$ by $\toroidal$ (cf. proposition \ref{prop: yangian_extended_toroidal_lie_algebras}). This tells us that the centre of $\extendedtoroidal$ ought to lie inside that of $\toroidal$, i.e.:
                $$\hat{\z}_{[2]} \subset \z(\toroidal) = \z(\toroidal)$$
            as per proposition \ref{prop: twisted_semi_direct_product_criterion}.
        \end{remark}

        Using lemma \ref{lemma: yangian_div_zero_vector_fields_basic_properties} in conjunction with lemma \ref{lemma: derivation_action_on_toroidal_centres}, we can now also explicitly compute the commutation relations between the basis elements of $\divzero$ and $\z(\toroidal)$.
        \begin{lemma}[Explicit commutators between basis elements of $\divzero$ and $\z(\toroidal)$] \label{lemma: explicit_commutators_between_central_basis_elements_and_derivations}
            In the Lie algebra $\extendedtoroidal$, one has the following commutation relations between elements of $\z(\toroidal)$ and those of $\divzero$. Namely, for all $D \in \divzero$, the following relations hold:
                $$
                    \forall (a, b) \in \Z^2: [D, K_{a, b}]_{\extendedtoroidal} =
                    \begin{cases}
                        \text{$((b - 1)r - sa) K_{a - r, b - s - 1} + \delta_{(r, s + 1), (a, b)} \left( r c_v + s c_t \right)$ if $D = D_{r, s}$}
                        \\
                        \text{$a K_{a, b - 1}$ if $D_v$}
                        \\
                        \text{$b K_{a, b - 1}$ if $D_t$}
                    \end{cases}
                $$
                $$[D, c_v]_{\extendedtoroidal} = [D, c_t]_{\extendedtoroidal} = 0$$
        \end{lemma}
            \begin{proof}
                For this, we shall be making use of invariance again, namely:
                    $$(D, [D', K]_{\extendedtoroidal})_{\extendedtoroidal} = ([D, D']_{\extendedtoroidal}, K)_{\extendedtoroidal}$$
                for all $D, D' \in \divzero$ and all $K \in \z(\toroidal)$, and how the brackets $[D, D']_{\extendedtoroidal}$ are given explicitly (cf. lemma \ref{lemma: yangian_div_zero_vector_fields_basic_properties}) as well as how basis elements $K \in \z(\toroidal)$ pair with basis elements of $\divzero$ in the construction of $(-, -)_{\extendedtoroidal}$. Without any loss of generality, we can assume that $D, D' \in \divzero$ and $K \in \z(\toroidal)$ are basis elements, i.e.:
                    $$D, D' \in \{D_{r, s}\}_{(r, s) \in \Z^2} \cup \{D_v, D_t\}$$
                    $$K \in \{K_{a, b}\}_{(a, b) \in \Z^2} \cup \{c_v, c_t\}$$
                and then perform the computations case-by-case, for which we shall recall from example \ref{example: toroidal_lie_algebras_centres} that:
                    $$
                        K_{a, b} :=
                        \begin{cases}
                            \text{$\frac1b v^{a - 1} t^b \bar{d}v$ if $(a, b) \in \Z \x (\Z \setminus \{0\})$}
                            \\
                            \text{$-\frac1a v^a t^{-1} \bar{d}t$ if $(a, b) \in (\Z \setminus \{0\}) \x \{0\}$}
                            \\
                            \text{$0$ if $(a, b) = (0, 0)$}
                        \end{cases}
                    $$
                    $$c_v := v^{-1} \bar{d}v, c_t := t^{-1} \bar{d}t$$
                and from lemma \ref{lemma: yangian_div_zero_vector_fields_basic_properties}, that:
                    $$[D_v, D_t] = 0$$
                    $$[D_v, D_{r, s}] = r D_{r, s + 1}$$
                    $$[D_t, D_{r, s}] = D_{r, s + 1}$$
                    $$[D_{\alpha, \beta}, D_{r, s}] = (\beta r - s \alpha) D_{\alpha + r, \beta + s + 1}$$
                For what follows, let:
                    $$D := \sum_{(\alpha, \beta) \in \Z^2} \lambda_{\alpha, \beta} D_{\alpha, \beta} + \lambda_v D_v + \lambda_t D_t$$
                for some $\lambda_{\alpha, \beta}, \lambda_v, \lambda_t \in \bbC$.
                \begin{enumerate}
                    \item Assume firstly that $K = K_{a, b}$.
                    \begin{enumerate}
                        \item If $D' = D_{r, s}$, then we shall have that:
                            $$
                                \begin{aligned}
                                    & ( D, [D_{r, s}, K_{a, b}]_{\extendedtoroidal} )_{\extendedtoroidal}
                                    \\
                                    = & ( [D, D_{r, s}]_{\extendedtoroidal}, K_{a, b} )_{\extendedtoroidal}
                                    \\
                                    = & \sum_{(\alpha, \beta) \in \Z^2} (\beta r - s \alpha) \lambda_{\alpha, \beta} \delta_{(\alpha + r, \beta + s + 1), (a, b)} + \delta_{(r, s + 1), (a, b)} \left( r\lambda_v + s\lambda_t \right)
                                    \\
                                    = & ((b - s - 1) r - s (a - r)) \lambda_{a - r, b - s - 1} + \delta_{(r, s + 1), (a, b)} \left( r\lambda_v + s\lambda_t \right)
                                    \\
                                    = & ((b - 1)r - sa) \lambda_{a - r, b - s - 1} + \delta_{(r, s + 1), (a, b)} \left( r\lambda_v + s\lambda_t \right)
                                \end{aligned}
                            $$
                        from which we are able to conclude that:
                            $$[D_{r, s}, K_{a, b}]_{\extendedtoroidal} = ((b - 1)r - sa) K_{a - r, b - s - 1} + \delta_{(r, s + 1), (a, b)} \left( r c_v + s c_t \right)$$
                        \item If $D' = D_v$, then:
                            $$
                                \begin{aligned}
                                    & ( D, [D_v, K_{a, b}]_{\extendedtoroidal} )_{\extendedtoroidal}
                                    \\
                                    = & ( [D, D_v]_{\extendedtoroidal}, K_{a, b} )_{\extendedtoroidal}
                                    \\
                                    = & \sum_{(\alpha, \beta) \in \Z^2} \alpha \lambda_{\alpha, \beta} \delta_{(\alpha, \beta + 1), (a, b)} 
                                    \\
                                    = & a \lambda_{a, b - 1}
                                \end{aligned}
                            $$
                        from which we are able to conclude that:
                            $$[D_v, K_{a, b}]_{\extendedtoroidal} = aK_{a, b - 1}$$   
                        \item Finally, if $D' = D_t$, then:
                            $$
                                \begin{aligned}
                                    & ( D, [D_t, K_{a, b}]_{\extendedtoroidal} )_{\extendedtoroidal}
                                    \\
                                    = & ( [D, D_t]_{\extendedtoroidal}, K_{a, b} )_{\extendedtoroidal}
                                    \\
                                    = & \sum_{(\alpha, \beta) \in \Z^2} \beta \lambda_{\alpha, \beta} \delta_{(\alpha, \beta + 1), (a, b)} 
                                    \\
                                    = & b \lambda_{a, b - 1}
                                \end{aligned}
                            $$
                        from which we are able to conclude that:
                            $$[D_t, K_{a, b}]_{\extendedtoroidal} = b K_{a, b - 1}$$
                    \end{enumerate}
                    \item If $K = c_v$ or $K = c_t$ then simply note that because:
                        $$[D, D']_{\extendedtoroidal} \in \bigoplus_{(\alpha, \beta) \in \Z^2} \bbC D_{\alpha, \beta}$$
                    for all $D' \in \divzero$, we shall have that:
                        $$[D, K]_{\extendedtoroidal} = 0$$
                    for all $D := \sum_{(\alpha, \beta) \in \Z^2} \lambda_{\alpha, \beta} D_{\alpha, \beta} + \lambda_v D_v + \lambda_t D_t \in \divzero$.
                \end{enumerate}
            \end{proof}
        
        \begin{proposition}[Centres of $\gamma$-extended toroidal Lie algebras] \label{prop: centres_of_yangian_extended_toroidal_lie_algebras}
            The centre $\hat{\z}_{[2]}$ is a two-dimensional (abelian) Lie subalgebra of $\z(\toroidal)$, spanned by $c_v$ and $c_t$. 
        \end{proposition}
            \begin{proof}
                Since we know that:
                    $$\hat{\z}_{[2]} \subset \z(\toroidal)$$
                and that the only possibly non-zero bracket with elements of $\z(\toroidal)$ are elements of $[\divzero, \z(\toroidal)]_{\extendedtoroidal}$, and since we also know from lemma \ref{lemma: explicit_commutators_between_central_basis_elements_and_derivations} that:
                    $$[\divzero, K]_{\extendedtoroidal} = 0 \iff K \in \bbC c_v \oplus \bbC c_v$$
                we can conclude immediately that:
                    $$\hat{\z}_{[2]} = \bbC c_v \oplus \bbC c_t$$
            \end{proof}
        \begin{remark}
            It is rather interesting that:
                $$\hat{\z}_{[2]} \cong \bbC c_v \oplus \bbC c_t$$
            as this is in good analogy with the affine Kac-Moody case, where the centre of $\hat{\g}$ is $1$-dimensional, namely spanned by $c_v$ (cf. example \ref{example: affine_lie_algebras_centres}).
        \end{remark}

    \subsection{The Witt algebra}