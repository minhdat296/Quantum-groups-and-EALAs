\section{Construction of \texorpdfstring{$\gamma$}{}-extended toroidal Lie algebras}
    \subsection{Outline and statement of the main result}
        Using the linear map:
            $$\gamma: A \to \bbC$$
        as in subsection \ref{subsection: yangian_div_zero_vector_fields}, we can construct an \textit{invariant} and \textit{non-degenerate} symmetric bilinear form on $\g_{[2]}$, given as follows:
            $$(xf, yg)_{\g_{[2]}} := (x, y)_{\g} \gamma(fg)$$
        for all $x, y \in \g$ and all $f, g \in A$, and let us remind the reader that we abbreviate the pure tensors by:
            $$xf := x \tensor f$$
        for all $x \in \g$ and $f \in A$. Next, let us fix an \textit{invariant} extension (in the sense of remark \ref{remark: extending_bilinear_forms_to_central_extensions}):
            $$(-, -)_{\toroidal}$$
        of the bilinear form $(-, -)_{\g_{[2]}}$ to $\toroidal$. \textit{A priori}, it is degenerate, with:
            $$\Rad(-, -)_{\toroidal} := \{ K \in \toroidal \mid \forall X \in \toroidal (K, X)_{\toroidal} = 0 \} = \z(\toroidal)$$
        To fix this degeneracy of $(-, -)_{\toroidal}$, a \textit{non-degenerate} extension thereof to the vector space $\extendedtoroidal := \toroidal \oplus \z(\toroidal)^{\star}$.
        \todo[inline]{The non-degenerate bilinear form on $\extendedtoroidal$ as on p. 1 of \lstinline{Ch2_revamped}.}
        \begin{proposition}[A non-degenerate bilinear form extending $(-, -)_{\toroidal}$] \label{prop: extended_toroidal_bilinear_form}
            On the vector space $\extendedtoroidal := \toroidal \oplus \z(\toroidal)^{\star}$, there is a non-degenerate and symmetric bilinear form $(-, -)_{\extendedtoroidal}$ given by:
                $$( X + D, Y + D' )_{\extendedtoroidal} := (X, Y)_{\toroidal} + \varphi(D')( \pi_{\z(\toroidal)}(X) ) + \varphi(D)( \pi_{\z(\toroidal)}(Y) )$$
            for all $X, Y \in \toroidal$ and all $D, D' \in \divzero$, extending the bilinear form $(-, -)_{\toroidal}$ on $\toroidal$. Here, $\pi_{\z(\toroidal)}: \extendedtoroidal \to \z(\toroidal)$ denotes the canonical projection and $\varphi: \divzero \xrightarrow[]{\cong} \z(\toroidal)^{\star}$ is as in proposition \ref{prop: yangian_div_zero_vector_fields_are_graded_dual_to_toroidal_centre}.
        \end{proposition}
            \begin{proof}
                Clear from corollary \ref{coro: pairing_yangian_div_zero_vector_fields_and_cyclic_1_forms}.
            \end{proof}

        \todo[inline]{Definition 1.1 in \lstinline{Ch2_revamped}.}
        \begin{definition}[$\gamma$-extended toroidal Lie algebras] \label{def: yangian_extended_toroidal_lie_algebras}
            A \textbf{$\gamma$-extended toroidal Lie algebra} or \textbf{$\gamma$-extended toroidal Lie algebra} is quadruple:
                $$(\fraky, \mu, [-, -]_{\fraky}, (-, -)_{\fraky})$$
            consisting of:
            \begin{itemize}
                \item a vector space $\fraky$;
                \item an isomorphism of vector spaces:
                    $$\mu: \fraky \xrightarrow[]{\cong} \toroidal \oplus \z(\toroidal)^{\star}$$
                \item a Lie bracket $[-, -]_{\fraky}: \bigwedge^2 \fraky \to \fraky$, such that the canonical inclusion of vector spaces $\toroidal \subset \fraky$ is a Lie algebra monomorphism;
                \item a non-degenerate symmetric bilinear form\footnote{... and we note the subtlety that the construction of the bilinear form $(-, -)_{\extendedtoroidal}$ depends on $\gamma$ (which is not reflected in our notations).}:
                    $$(-, -)_{\fraky} := (-, -)_{\extendedtoroidal} \circ \mu^{\tensor 2}$$
                that is \textit{invariant} with respect to the Lie bracket $[-, -]_{\fraky}$.
            \end{itemize}
        \end{definition}

        \todo[inline]{$\divzero$-action on $\toroidal$}
        \begin{convention} \label{conv: a_fixed_yangian_div_zero_vector_field_action} 
            Let us fix a $\divzero$-module structure on $\toroidal$, originating naturally from a natural action of $\divzero$ on $A$, and we do this by fixing module structures on $\g_{[2]}$ and $\z(\toroidal)$. Let $\divzero$ act on $A$ by evaluations of derivations $D \in \divzero$ on elements $f \in A$:
                $$D \cdot f := D(f)$$
            This induces the following obvious $\divzero$-action on $\g_{[2]}$, given by:
                $$D \cdot xf := x D(f)$$
            for all $x \in \g$ and all $f \in A$. It also induces an action of $\divzero$ on $\z(\toroidal) \cong \bar{\Omega}_{[2]}$ by \say{Lie derivatives}, i.e.:
                $$D \cdot f\bar{d}g := Df \bar{d}g + f \bar{d}(D(g))$$
            As such, we have a $\divzero$-action:
                $$\rho: \divzero \to \gl(\toroidal)$$
            on $\toroidal \cong \g_{[2]} \oplus \z(\toroidal)$.
        \end{convention}
        \begin{remark}
            The action $\rho: \divzero \to \gl(\toroidal)$ as in convention \ref{conv: a_fixed_yangian_div_zero_vector_field_action} is by Lie derivations (cf. definition \ref{def: lie_derivations}). For us, this means that we are allowed to form the semi-direct products:
                $$\toroidal \rtimes \divzero$$
            (cf. example \ref{example: lie_algebra_semi_direct_products}).
        \end{remark}

        Our main result concerning $\gamma$-extended toroidal Lie algebras is as follows:
        \begin{theorem} \label{theorem: yangian_extended_toroidal_lie_algebras_main_theorem}
            A given Lie algebra $(\fraky, [-, -]_{\fraky})$ is a $\gamma$-extended toroidal Lie algebra if and only if the following conditions are satisfied simultaneously:
            \begin{enumerate}
                \item there is a vector space isomorphism:
                    $$\toroidal \rtimes^{\sigma} \divzero \xrightarrow[\cong]{\id_{\toroidal} \oplus \varphi} \fraky$$
                with $\varphi$ as in proposition \ref{prop: yangian_div_zero_vector_fields_are_graded_dual_to_toroidal_centre};
                \item for all $D \in \divzero$ and all $X \in \toroidal$, one has that:
                    $$[D, X]_{\fraky} = \rho(D) \cdot X$$
                (i.e. the action of $\divzero$ on $\toroidal$ is independent of $\sigma$) with $\rho: \divzero \to \gl(\toroidal)$ as in convention \ref{conv: a_fixed_yangian_div_zero_vector_field_action};

                \todo[inline]{Is this second condition the same as what was meant by "the action of $\divzero$ on $\toroidal$ is independent of the choice of extension" ?}
                \item the $2$-cocycle $\sigma \in Z^2_{\Lie}(\divzero, \z(\toroidal))$ is such that:
                    $$(\sigma(D, D'), D'')_{\extendedtoroidal} = (D, \sigma(D', D''))_{\extendedtoroidal}$$
                for any triple of elements $D, D', D'' \in \divzero$. 
            \end{enumerate}
        \end{theorem}
        \begin{corollary}
            $\toroidal$ is a Lie algebra ideal of any $\gamma$-extended toroidal Lie algebra.
        \end{corollary}

    \subsection{Which twisted semi-direct products are \texorpdfstring{$\gamma$}{}-extended toroidal Lie algebras ?}
        \todo[inline]{Proof of the "if" direction of the main theorem. This was not concisely stated before, so most of the materials in this subsection is completely new.}
    
        In this subsection, we prove the \say{if} direction of theorem \ref{theorem: yangian_extended_toroidal_lie_algebras_main_theorem}. To this end, let us firstly fix a twisted semi-direct product:
            $$\toroidal \rtimes^{\sigma} \divzero$$
        (whose Lie bracket shall be denoted by $[-, -]_{\sigma}$) such that:
            $$[D, X]_{\fraky} = \rho(D) \cdot X$$
        (for all $X \in \toroidal$ and $D \in \divzero$, and with $\rho: \divzero \to \gl(\toroidal)$ being as in convention \ref{conv: a_fixed_yangian_div_zero_vector_field_action}) and such that:
            $$(\sigma(D, D'), D'')_{\extendedtoroidal} = (D, \sigma(D', D''))_{\extendedtoroidal}$$
        for all $D, D', D'' \in \divzero$.

        \begin{remark}[What do we need to show ?]
            Obviously, there is a vector space isomorphism between $\fraky$ and $\toroidal \oplus \z(\toroidal)^{\star}$, namely:
                $$\id_{\toroidal} \oplus \varphi: \fraky \to \toroidal \oplus \z(\toroidal)^{\star}$$
            with $\varphi: \divzero \to \z(\toroidal)$ as in proposition \ref{prop: yangian_div_zero_vector_fields_are_graded_dual_to_toroidal_centre}. It is also clear (per definition \ref{def: twisted_semi_direct_products}), that the canonical inclusion of vector spaces $\toroidal \subset \fraky$ is a Lie algebra monomorphism. As such, the only thing to prove is that the non-degenerate symmetric bilinear form given by:
                $$(-, -)_{\toroidal} \circ (\id_{\toroidal} \oplus \varphi)^{\tensor 2}$$
            is invariant with respect to the Lie bracket $[-, -]_{\sigma}$.
        \end{remark}

        In accordance with the notations above, let us write $[-, -]_0$ for the bracket on the semi-direct product $\toroidal \rtimes \divzero$.
        \begin{lemma} \label{lemma: semi_direct_product_of_toroidal_lie_algebras_with_div_zero_vector_fields_are_yangian_extended_toroidal_lie_algebras}
            Let $\divzero$ act on $\toroidal$ via $\rho$ as in convention \ref{conv: a_fixed_yangian_div_zero_vector_field_action}. The semi-direct product $\toroidal \rtimes \divzero$ - corresponding to the $2$-cocycle:
                $$0 \in Z^2_{\Lie}(\divzero, \z(\toroidal))$$
            - is a $\gamma$-extended toroidal Lie algebra in the sense of definition \ref{def: yangian_extended_toroidal_lie_algebras}.
        \end{lemma}
            \begin{proof}
                We would like to show that, for all $Z + D'' \in \toroidal \oplus \divzero$, we shall have that:
                    $$\left( [X + D, Y + D']_0, Z + D'' \right)_{\extendedtoroidal} = \left( X + D, [Y + D', Z + D'']_0 \right)_{\extendedtoroidal}$$
                for all $X, Y, Z \in \toroidal$ and all $D, D', D'' \in \divzero$.
                
                From example \ref{example: lie_algebra_semi_direct_products}, we know that the semi-direct product Lie bracket $[-, -]_0$ is given by:
                    $$[X + D, Y + D']_0 := [X, Y]_{\toroidal} + \rho(D)(Y) - \rho(D')(X) + [D, D']$$
                for all $X, Y \in \toroidal$ and all $D, D' \in \divzero$. Using this, consider the following:
                    $$
                        \begin{aligned}
                            & \left( [X + D, Y + D']_0, Z + D'' \right)_{\extendedtoroidal}
                            \\
                            = & \left( [X, Y]_{\toroidal} + \rho(D)(Y) - \rho(D')(X) + [D, D'], Z + D'' \right)_{\extendedtoroidal}
                            \\
                            = & \left( [X, Y]_{\toroidal}, Z + D'' \right)_{\extendedtoroidal} + \left(\rho(D)(Y) - \rho(D')(X), Z + D'' \right)_{\extendedtoroidal} + \left( [D, D'], Z + D'' \right)_{\extendedtoroidal}
                            \\
                            = & \left( [X, Y]_{\toroidal}, Z + D'' \right)_{\extendedtoroidal} + \left(\rho(D)(Y) - \rho(D')(X), Z + D'' \right)_{\extendedtoroidal} + \left( [D, D'], Z \right)_{\extendedtoroidal}
                        \end{aligned}
                    $$
                Similarly, we have that:
                    $$
                        \begin{aligned}
                            & \left( X + D, [Y + D', Z + D'']_0 \right)_{\extendedtoroidal}
                            \\
                            = & \left( X + D, [Y, Z]_{\toroidal} \right)_{\extendedtoroidal} + \left( X + D, \rho(D')(Z) - \rho(D'')(Y) \right)_{\extendedtoroidal} + \left(X, [D', D'']\right)_{\extendedtoroidal}
                        \end{aligned}
                    $$
                Because $(-, -)_{\extendedtoroidal}|_{\Sym^2 \toroidal} = (-, -)_{\toroidal}$ by construction (cf. proposition \ref{prop: extended_toroidal_bilinear_form}), and because $(-, -)_{\toroidal}$ is invariant (cf. remark \ref{remark: extending_bilinear_forms_to_central_extensions}), we have that:
                    $$([X, Y]_{\toroidal}, Z)_{\extendedtoroidal} = ([X, Y]_{\toroidal}, Z)_{\toroidal} = (X, [Y, Z]_{\toroidal})_{\toroidal} = (X, [Y, Z]_{\toroidal})_{\extendedtoroidal}$$
                and so:
                    $$( [X, Y]_{\toroidal}, Z + D'' )_{\extendedtoroidal} = (X, [Y, Z]_{\toroidal})_{\extendedtoroidal} + ( [X, Y]_{\toroidal}, D'')_{\extendedtoroidal}$$
                and likewise, we have that:
                    $$\left( X + D, [Y, Z]_{\toroidal} \right)_{\extendedtoroidal} = (X, [Y, Z]_{\toroidal})_{\extendedtoroidal} + (D, [Y, Z]_{\toroidal})_{\extendedtoroidal} = (X, [Y, Z]_{\toroidal})_{\extendedtoroidal} + (D, [Y, Z]_{\toroidal})_{\extendedtoroidal}$$
                As such, we have that:
                    $$
                        \begin{aligned}
                            & 
                            \begin{aligned}
                                & \left( \left( [X, Y]_{\toroidal}, Z + D'' \right)_{\extendedtoroidal} + \left(\rho(D)(Y) - \rho(D')(X), Z + D'' \right)_{\extendedtoroidal} + \left( [D, D'], Z \right)_{\extendedtoroidal} \right)
                                \\
                                - & \left( \left( X + D, [Y, Z]_{\toroidal} \right)_{\extendedtoroidal} + \left( X + D, \rho(D')(Z) - \rho(D'')(Y) \right)_{\extendedtoroidal} + \left(X, [D', D'']\right)_{\extendedtoroidal} \right)
                            \end{aligned}
                            \\
                            & =
                            \begin{aligned}
                                & \left( ( [X, Y]_{\toroidal}, D'')_{\extendedtoroidal} - (D, [Y, Z]_{\toroidal})_{\extendedtoroidal} \right)
                                \\
                                + & \left( \left(\rho(D)(Y) - \rho(D')(X), Z + D'' \right)_{\extendedtoroidal} - \left( X + D, \rho(D')(Z) - \rho(D'')(Y) \right)_{\extendedtoroidal} \right)
                                \\
                                + & \left( \left( [D, D'], Z \right)_{\extendedtoroidal} - \left(X, [D', D'']\right)_{\extendedtoroidal} \right)
                            \end{aligned}
                        \end{aligned}
                    $$
                \textit{We would like to show that this is equal to $0$.}

                Let us now perform a case-by-case analysis.
                \begin{enumerate}
                    \item Firstly, consider the case where $X, Y, Z \in \g_{[2]}$. Without any loss of generality, let us now assume that:
                        $$X := xf, Y := yg, Z := zh$$
                    for some $x, y, z \in \g$ and some $f, g, h \in A$. Per this assumption, we shall have via the fact that $(\g_{[2]}, \divzero)_{\extendedtoroidal} := 0$ that:
                        $$( [X, Y]_{\toroidal}, D'')_{\extendedtoroidal} = (x, y)_{\g} ( g \bar{d}f, D'')_{\extendedtoroidal} = (x, y)_{\g} \varphi(D'')(g \bar{d}f) = (x, y)_{\g} \gamma( g D''(f) )$$
                    and likewise, that:
                        $$(D, [Y, Z]_{\toroidal})_{\extendedtoroidal} = (y, z)_{\g} \gamma( h D(g) )$$
                    At the same time, we have that:
                        $$
                            \begin{aligned}
                                & \left(\rho(D)(Y) - \rho(D')(X), Z + D'' \right)_{\extendedtoroidal}
                                \\
                                = & \left( y D(g) - x D'(f), zh + D'' \right)_{\extendedtoroidal}
                                \\
                                = & \left( y D(g) - x D'(f), zh \right)_{\extendedtoroidal}
                                \\
                                = & (y, z)_{\g} \gamma( D(g) h ) - (x, z)_{\g} \gamma( D'(f) h )
                            \end{aligned}
                        $$
                    and likewise, that:
                        $$\left( X + D, \rho(D')(Z) - \rho(D'')(Y) \right)_{\extendedtoroidal} = (x, z)_{\g} \gamma( f D'(h) ) - (x, y)_{\g} \gamma( f D''(g) )$$
                    Finally, our assumption that $X, Y, Z \in \g_{[2]}$ implies that:
                        $$([D, D'], Z)_{\extendedtoroidal} = (X, [D', D''])_{\extendedtoroidal} = 0$$
                    By putting everything together, we shall have that:
                        $$
                            \begin{aligned}
                                &
                                    \begin{aligned}
                                    & \left( ( [X, Y]_{\toroidal}, D'')_{\extendedtoroidal} - (D, [Y, Z]_{\toroidal})_{\extendedtoroidal} \right)
                                    \\
                                    + & \left( \left(\rho(D)(Y) - \rho(D')(X), Z + D'' \right)_{\extendedtoroidal} - \left( X + D, \rho(D')(Z) - \rho(D'')(Y) \right)_{\extendedtoroidal} \right)
                                    \\
                                    + & \left( \left( [D, D'], Z \right)_{\extendedtoroidal} - \left(X, [D', D'']\right)_{\extendedtoroidal} \right)
                                \end{aligned}
                                \\
                                = & 
                                \begin{aligned}
                                    & \left( (x, y)_{\g} \gamma( g D''(f) ) - (y, z)_{\g} \gamma( h D(g) ) \right)
                                    \\
                                    + & \left( \left( (y, z)_{\g} \gamma( D(g) h ) - (x, z)_{\g} \gamma( D'(f) h ) \right) - \left( (x, z)_{\g} \gamma( f D'(h) ) - (x, y)_{\g} \gamma( f D''(g) ) \right) \right)
                                \end{aligned}
                                \\
                                = & (x, y)_{\g} \gamma( g D''(f) + f D''(g) ) - (x, z)_{\g} \gamma( D'(f) h + f D'(h) ) + (y, z)_{\g} \gamma( D(g) h - h D(g) )
                                \\
                                = & (x, y)_{\g} \gamma( D''(fg) ) - (x, z)_{\g} \gamma( D'(fh) )
                                \\
                                = & 0
                            \end{aligned}
                        $$
                    with the last equality holding because of the definition of $\divzero$, which implies in particular that:
                        $$\gamma( D''(fg) ) = \gamma( D'(fh) ) = 0$$
                    (cf. lemma \ref{lemma: yangian_div_zero_vector_fields_basis}).
                    \item Next, consider the case where $X, Y, Z \in \z(\toroidal)$, and without loss of generality, let us assume that:
                        $$X := f_1\bar{d}g_1, Y := f_2\bar{d}g_2, Z := f_3\bar{d}g_3$$
                    for some $f_i, g_i \in A$ ($i \in \{1, 2, 3\}$). Immediately, we have that:
                        $$[X, Y]_{\toroidal} = [Y, Z]_{\toroidal} = 0$$
                    and so:
                        $$([X, Y]_{\toroidal}, D'')_{\extendedtoroidal} = (D, [Y, Z]_{\toroidal})_{\extendedtoroidal} = 0$$
                    Next, using the fact that $\rho(\z(\toroidal)) \subseteq \z(\toroidal)$ and the fact that $(\z(\toroidal), \z(\toroidal))_{\extendedtoroidal} = 0$, we shall obtain:
                        $$
                            \begin{aligned}
                                & \left(\rho(D)(Y) - \rho(D')(X), Z + D'' \right)_{\extendedtoroidal}
                                \\
                                = & \left( \left( D(f_2) \bar{d}g_2 + f_2 \bar{d}( D(g_2) ) \right) - \left( D'(f_1) \bar{d}g_1 + f_1 \bar{d}( D'(g_1) ) \right), D'' \right)_{\extendedtoroidal}
                                \\
                                = & \varphi(D'')\left( \left( D(f_2) \bar{d}g_2 + f_2 \bar{d}( D(g_2) ) \right) - \left( D'(f_1) \bar{d}g_1 + f_1 \bar{d}( D'(g_1) ) \right) \right)
                                \\
                                = & \gamma\left(
                                    \left( D(f_2) D''(g_2) + f_2 D''( D(g_2) ) \right) - \left( D'(f_1) D''(g_1) + f_1 D''( D'(g_1) ) \right)
                                \right)
                            \end{aligned}
                        $$
                    and likewise, that:
                        $$
                            \begin{aligned}
                                & \left( X + D, \rho(D')(Z) - \rho(D'')(Y) \right)_{\extendedtoroidal}
                                \\
                                = & \gamma\left(
                                \left( D'(f_3) D(g_3) + f_3 D( D'(g_3) ) \right) - \left( D''(f_2) D(g_2) + f_2 D( D''(g_2) ) \right)
                        \right)
                            \end{aligned}
                        $$
                    We have also that:
                        $$\left( [D, D'], Z \right)_{\extendedtoroidal} = \varphi([D, D'])( f_3 \bar{d} g_3 ) = \gamma(f_3 [D, D'](g_3))$$
                    and likewise, that:
                        $$\left(X, [D', D'']\right) = \gamma(f_1 [D', D''](g_1))$$
                    By putting everything together, we shall get that:
                        $$
                            \begin{aligned}
                                &
                                \begin{aligned}
                                    & \left( ( [X, Y]_{\toroidal}, D'')_{\extendedtoroidal} - (D, [Y, Z]_{\toroidal})_{\extendedtoroidal} \right)
                                    \\
                                    + & \left( \left(\rho(D)(Y) - \rho(D')(X), Z + D'' \right)_{\extendedtoroidal} - \left( X + D, \rho(D')(Z) - \rho(D'')(Y) \right)_{\extendedtoroidal} \right)
                                    \\
                                    + & \left( \left( [D, D'], Z \right)_{\extendedtoroidal} - \left(X, [D', D'']\right)_{\extendedtoroidal} \right)
                                \end{aligned}
                                \\
                                = &
                                \begin{aligned}
                                    & 0
                                    \\
                                    + & \gamma\left( \left( D(f_2) D''(g_2) + f_2 D''( D(g_2) ) \right) - \left( D'(f_1) D''(g_1) + f_1 D''( D'(g_1) ) \right) \right)
                                    \\
                                    - & \gamma\left( \left( D'(f_3) D(g_3) + f_3 D( D'(g_3) ) \right) - \left( D''(f_2) D(g_2) + f_2 D( D''(g_2) ) \right) \right)
                                    \\
                                    + & \gamma( f_3 [D, D'](g_3) - f_1 [D', D''](g_1) )
                                \end{aligned}
                                \\
                                = &
                                \begin{aligned}
                                    & -\gamma\left( f_1 D''( D'(g_1) ) + D'(f_1) D''(g_1) + f_1[D', D''](g_1) \right)
                                    \\
                                    + & \gamma\left( D(f_2) D''(g_2) + f_2 D''( D(g_2) ) + D''(f_2) D(g_2) + f_2 D( D''(g_2) ) \right)
                                    \\
                                    - & \gamma\left( D'(f_3) D(g_3) + f_3 D( D'(g_3) ) - f_3 [D, D'](g_3) \right)
                                \end{aligned}
                                \\
                                = &
                                \begin{aligned}
                                    & -\gamma\left( D'(f_1) D''(g_1) + f_1 D' (D''(g_1)) \right)
                                    \\
                                    + & \gamma\left( D(f_2) D''(g_2) + f_2 D( D''(g_2) ) \right) + \gamma( D''(f_2) D(g_2) + f_2 D''( D(g_2) ) )
                                    \\
                                    - & \gamma\left( D'(f_3) D(g_3) + f_3 D'(D(g_3)) \right)
                                \end{aligned}
                                \\
                                = &
                                \begin{aligned}
                                    & -\gamma\left( D'( f_1 D''(g_1) ) \right)
                                    \\
                                    + & \gamma\left( D( f_2 D''(g_2) ) \right) + \gamma\left( D''( f_2 D(g_2) ) \right)
                                    \\
                                    - & \gamma\left( D'( f_3 D(g_3) ) \right)
                                \end{aligned}
                                \\
                                = & 0
                            \end{aligned}
                        $$
                    wherein the last equality holds because of the construction of $\divzero$ (cf. lemma \ref{lemma: yangian_div_zero_vector_fields_basis}).
                \end{enumerate}

                Recall now that, per the construction of the action $\rho: \divzero \to \gl(\toroidal)$ (cf. convention \ref{conv: a_fixed_yangian_div_zero_vector_field_action}), we know that $\toroidal$ splits into a direct sum of $\divzero$-submodules $\g_{[2]} \oplus \z(\toroidal)$. We see thus that, the case-by-case analysis above is sufficient for concluding that:
                    $$
                        \begin{aligned}
                            & \left( ( [X, Y]_{\toroidal}, D'')_{\extendedtoroidal} - (D, [Y, Z]_{\toroidal})_{\extendedtoroidal} \right)
                            \\
                            + & \left( \left(\rho(D)(Y) - \rho(D')(X), Z + D'' \right)_{\extendedtoroidal} - \left( X + D, \rho(D')(Z) - \rho(D'')(Y) \right)_{\extendedtoroidal} \right)
                            \\
                            + & \left( \left( [D, D'], Z \right)_{\extendedtoroidal} - \left(X, [D', D'']\right)_{\extendedtoroidal} \right)
                        \end{aligned}
                        = 0
                    $$
                As stated above, this is equivalent to that :
                    $$\left( [X + D, Y + D']_0, Z + D'' \right)_{\extendedtoroidal} = \left( X + D, [Y + D', Z + D'']_0 \right)_{\extendedtoroidal}$$
                for all $X, Y, Z \in \toroidal$ and all $D, D', D'' \in \divzero$. This implies that the semi-direct product $\toroidal \rtimes \divzero$ is a $\gamma$-extended toroidal Lie algebra, as claimed.
            \end{proof}
        \begin{theorem} \label{theorem: yangian_criterion_for_toroidal_cocycles}
            Assume that a given toroidal $2$-cocycle:
                $$\sigma \in Z^2_{\Lie}(\divzero, \z(\toroidal))$$
            that satisfies:
                $$(\sigma(D, D'), D'')_{\extendedtoroidal} = (D, \sigma(D', D''))_{\extendedtoroidal}$$
            for all $D, D', D'' \in \divzero$. Then the corresponding twisted semi-direct product $\toroidal \rtimes^{\sigma} \divzero$ will be a $\gamma$-extended toroidal Lie algebra in the sense of definition \ref{def: yangian_extended_toroidal_lie_algebras}.
        \end{theorem}
            \begin{proof}
                Pick arbitrary elements $X, Y, Z \in \toroidal$ and $D, D', D'' \in \divzero$.
                
                To begin, consider the following:
                    $$
                        \begin{aligned}
                            ([X + D', Y + D'']_{\sigma}, Z + D'')_{\extendedtoroidal} & = ([X + D, Y + D']_0 + \sigma(D, D'), Z)_{\extendedtoroidal}
                            \\
                            & = ([X + D, Y + D']_0, Z + D'')_{\extendedtoroidal} + (\sigma(D, D'), Z + D'')_{\extendedtoroidal}
                            \\
                            & = (X + D, [Y + D', Z + D'']_0)_{\extendedtoroidal} + (\sigma(D, D'), Z + D'')_{\extendedtoroidal}
                            \\
                            & = (X + D, [Y + D', Z + D'']_0)_{\extendedtoroidal} + (\sigma(D, D'), D'')_{\extendedtoroidal}
                        \end{aligned}
                    $$
                where the last equality is becuase $\sigma(D, D') \in \z(\toroidal)$. Similarly, we have that:
                    $$(X + D', [Y + D'', Z + D'']_{\sigma})_{\extendedtoroidal} = ([X + D, Y + D']_0, Z + D'')_{\extendedtoroidal} + (D, \sigma(D', D''))_{\extendedtoroidal}$$
                As lemma \ref{lemma: semi_direct_product_of_toroidal_lie_algebras_with_div_zero_vector_fields_are_yangian_extended_toroidal_lie_algebras} shows, the semi-direct product $\toroidal \rtimes \divzero$ is a $\gamma$-extended toroidal Lie algebra, so we have that:
                    $$(X + D, [Y + D', Z + D'']_0)_{\extendedtoroidal} = ([X + D, Y + D'], Z + D''_0)_{\extendedtoroidal}$$
                We have also assumed:
                    $$(\sigma(D, D'), D'')_{\extendedtoroidal} = (D, \sigma(D', D''))_{\extendedtoroidal}$$
                Together, these two facts imply that:
                    $$([X + D', Y + D'']_{\sigma}, Z + D'')_{\extendedtoroidal} = (X + D', [Y + D'', Z + D'']_{\sigma})_{\extendedtoroidal}$$
                which is precisely what we need so we are done.
            \end{proof}

        We have thus proven the \say{if} implication in theorem \ref{theorem: yangian_extended_toroidal_lie_algebras_main_theorem}.

    \subsection{\texorpdfstring{$\gamma$}{}-extended toroidal Lie algebras are twisted semi-direct products}
        \todo[inline]{The proof of the "only if" direction of theorem 2.1 in \lstinline{Ch2_revamped} is contained entirely within this subsection.}

        \todo[inline]{Rephrased and reproved that $[\divzero, \g_{[2]}]_{\extendedtoroidal} \subseteq \g_{[2]} \oplus \z(\toroidal)$}
        \begin{lemma}[$\divzero$ acts on $\g_{[2]}$ by derivations] \label{lemma: derivation_action_on_multiloop_algebras}
            For any $D \in \divzero$ and any $x \in \g, f \in A$, we have that:
                $$[D, xf]_{\extendedtoroidal} = x D(f) + K( D, xf )$$
            for some yet-unknown $K( D, xf ) \in \z(\toroidal)$. As such, we have that:
                $$[\divzero, \g_{[2]}]_{\extendedtoroidal} \subseteq \g_{[2]} \oplus \z(\toroidal)$$
            as of now\footnote{In corollary \ref{coro: derivation_action_on_multiloop_algebras}, it will be shown that in fact, the $\z(\toroidal)$-summand $K( D, xf )$ vanishes.}.
        \end{lemma}
            \begin{proof}
                Consider firstly the following, for any $D \in \divzero$ and any $x, y \in \g, f, g \in A$:
                    $$( [D, xf]_{\extendedtoroidal}, yg )_{\extendedtoroidal} = ( D, [xy, fg]_{\toroidal} )_{\extendedtoroidal} = (x, y)_{\g} ( D, g\bar{d}f )_{\extendedtoroidal} = (x, y)_{\g} \gamma( g D(f) )$$
                At the same time, we have that:
                    $$( x D(f), yg )_{\extendedtoroidal} = (x, y)_{\g} \gamma( g D(f) )$$
                Clearly, then, we have that:
                    $$( [D, xf]_{\extendedtoroidal}, yg )_{\extendedtoroidal} = ( x D(f), yg )_{\extendedtoroidal}$$
                Since $yg \in \g_{[2]}$ is arbitrary and since $(\z(\toroidal), \g_{[2]})_{\extendedtoroidal} = 0$, the above implies via the non-degeneracy of $(-, -)_{\extendedtoroidal}$ that there exists some $K(D, xf) \in \z(\toroidal)$ such that:
                    $$[D, xf]_{\extendedtoroidal} = x D(f) + K(D, xf)$$
            \end{proof}
    
        \begin{lemma}[$\divzero$ acts on $\z(\toroidal)$ by Lie derivatives] \label{lemma: derivation_action_on_toroidal_centres}
            Elements of $\divzero$ act on those of $\z(\toroidal)$ as Lie derivatives. This is to say that, the elements $D \in \divzero$ act on the generating elements $f \bar{d}g \in \z(\toroidal)$ (for some $f, g \in A$) in the following manner:
                $$[D, f \bar{d}g]_{\extendedtoroidal} = \xi_D(f) \bar{d}g + f \bar{d}(\xi_D(g))$$
            where $\xi_D \in \der(A)$ is a derivation on $A$ determined uniquely by $A$ (well-defined because $\divzero$ is a vector subspace of $\der(A)$ per lemma \ref{lemma: derivation_action_on_multiloop_algebras}). In particular, this means that:
                $$[\divzero, \z(\toroidal)]_{\extendedtoroidal} \subseteq \z(\toroidal)$$
        \end{lemma}
            \begin{proof}
                Without any loss of generality, let us consider the following for any $h, h' \in \h$ so that\footnote{We can make this assumption because ultimately, elements of $\z(\toroidal)$ do not depend on those of $\g$.}:
                    $$(h, h')_{\g} = 1$$
                any $f, g \in A$, and any $D \in \divzero$:
                    $$[ D, [h f, h' g]_{\toroidal} ]_{\extendedtoroidal} = [ D, f \bar{d}( g ) ]_{\extendedtoroidal}$$
                At the same time, we have via the Jacobi identity that:
                    $$
                        \begin{aligned}
                            [ D, [h f, h' g]_{\toroidal} ]_{\extendedtoroidal} & = [ h f, [D, h' g]_{\extendedtoroidal} ]_{\toroidal} + [ [D, h f]_{\extendedtoroidal}, h' g ]_{\toroidal}
                            \\
                            & = [ h f, h' D( g ) ]_{\toroidal} + [ h D( f ), h' g ]_{\toroidal}
                            \\
                            & = f \bar{d}( D( g ) ) + D( f ) \bar{d}(g)
                        \end{aligned}
                    $$
                One thus sees that:
                    $$[ D, f \bar{d}( g ) ]_{\extendedtoroidal} = f \bar{d}( D( g ) ) + D( f ) \bar{d}(g)$$
                and since the element $f \bar{d}( g )$ is central (via the map $\e$ mentioned earlier), this gives another description of:
                    $$[ \divzero, \z(\toroidal) ]_{\extendedtoroidal}$$
                With this in mind, we return quickly to lemma \ref{lemma: derivation_action_on_multiloop_algebras}; there, we previously demonstrated that:
                    $$[ \divzero, \g_{[2]} ]_{\extendedtoroidal} \subseteq \g_{[2]} \oplus \z(\toroidal)$$
                but we claim now that the following stronger fact holds:
                    $$[ \divzero, \g_{[2]} ]_{\extendedtoroidal} \subseteq \g_{[2]}$$
                To see why this is the case, suppose firstly that for any $D \in \divzero$, any $X := x f \in \g_{[2]}$ (for some $f \in A$), there is $K(X) \in \z(\toroidal)$ depending on $X$ (and indeed, such a $K(X)$ exists by lemma \ref{lemma: derivation_action_on_multiloop_algebras}) such that:
                    $$[ D, X ]_{\extendedtoroidal} = x D( f ) + K(X)$$
                Next, pick an arbitrary element $\xi \in \divzero$ and then consider the following:
                    $$( [ D, X ]_{\extendedtoroidal}, \xi )_{\extendedtoroidal} = (D(X) + K(X), \xi)_{\extendedtoroidal} = (K(X), \xi)_{\extendedtoroidal}$$
                wherein the last equality holds as a consequence of the fact that:
                    $$( \g_{[2]}, \divzero )_{\extendedtoroidal} = 0$$
                per the construction of the bilinear form $(-, -)_{\extendedtoroidal}$.
            \end{proof}
        \begin{corollary}[Toroidal Lie algebras are ideals] \label{coro: toroidal_lie_algebras_are_ideals}
            With respect to the bracket $[-, -]_{\extendedtoroidal}$, the vector subspace $\toroidal$ is actually a Lie ideal of $\extendedtoroidal$.
        \end{corollary}

        We have now yielded the following intermediate conclusion:
        \begin{proposition}[$\toroidal$ as a $\divzero$-module] \label{prop: toroidal_lie_algebras_as_modules_over_div_zero_vector_field_lie_algebras}
            The $\divzero$-module structure:
                $$\rho: \divzero \to \gl(\toroidal)$$
            is such that:
                $$[\divzero, \toroidal]_{\extendedtoroidal} = \rho(\divzero) \cdot \toroidal$$
            regardless of our choice of the Lie bracket $[-, -]_{\extendedtoroidal}$ on the vector space $\extendedtoroidal$.
        \end{proposition}

        Finally, let us investigate how the brackets of the form:
            $$[D, D']_{\extendedtoroidal}$$
        are given, for all $D, D' \in \divzero$.
        \begin{proposition}[How does $\divzero$ act on itself] \label{prop: lie_bracket_on_orthogonal_complement_of_toroidal_centre}
            We have that:
                $$[ \divzero, \divzero ]_{\extendedtoroidal} \subset \z(\toroidal) \oplus \divzero$$
            i.e. the $\g_{[2]}$-summand of any commutator of the kind $[D, D']_{\extendedtoroidal}$ (for any two $D, D' \in \divzero$) actually vanishes. Furthermore, neither the $\z(\toroidal)$- nor the $\divzero$-summand of those commutators $[D, D']_{\extendedtoroidal}$ necessarily vanish in general. 
        \end{proposition}
            \begin{proof}
                For convenience, we will be abbreviating $\h_{[2]} := \h[v^{\pm}, t^{\pm 1}]$ and $\n^{\pm}_{[2]} := \n^{\pm}[v^{\pm}, t^{\pm 1}]$, with $\n^{\pm} := \bigoplus_{\alpha \in \Phi^{\pm}} \g_{\alpha}$ being the direct sums of the positive/negative roots spaces of $\g$, as usual.
            
                Pick arbitrary elements $D, D' \in \divzero$ and set:
                    $$[D, D']_{\extendedtoroidal} := X(D, D') + Z(D, D') + \xi(D, D')$$
                for some $X(D, D') \in \g_{[2]}, Z(D, D') \in \z(\toroidal)$, and $\xi(D, D') \in \divzero$ depending on $D, D'$. Pick also a test element $y g \in \g_{[2]}$, for some arbitrary $y \in \g$ and $g \in A$ and set:
                    $$[D, y g]_{\extendedtoroidal} := y D( g ) + K(D, Y)$$
                    $$[D', y g]_{\extendedtoroidal} := y D'( g ) + K(D', Y)$$
                for some $K(D, Y) \in \z(\toroidal)$ depending on $Y$ (cf. lemma \ref{lemma: derivation_action_on_multiloop_algebras}).
                
                Via the Jacobi identity, we get that:
                    $$
                        \begin{aligned}
                            & [ [D, D']_{\extendedtoroidal}, y g ]_{\extendedtoroidal}
                            \\
                            = & [ D, [ D', y g ]_{\extendedtoroidal} ]_{\extendedtoroidal} + [ D', [ y g, D ]_{\extendedtoroidal} ]_{\extendedtoroidal}
                            \\
                            = & [ D, y D'( g ) + K(D', Y) ]_{\extendedtoroidal} - [ D', y D( g ) + K(D, Y) ]_{\extendedtoroidal}
                            \\
                            = & \left( y D( D'(g) ) + K(DD', Y) + [ D, K(D', Y) ]_{\extendedtoroidal} \right) - \left( y D'( D(g) ) + K(D'D, Y) + [ D', K(D, Y) ]_{\extendedtoroidal} \right)
                            \\
                            = & y (DD' - D'D)( g ) + ( K(DD', Y) - K(D'D, Y) ) + ( [ D, K(D', Y) ]_{\extendedtoroidal} - [ D', K(D, Y) ]_{\extendedtoroidal} )
                        \end{aligned}
                    $$
                for some $K(DD', Y), K(D'D, Y) \in \z(\toroidal)$ such that:
                    $$[ D, y D'( g ) ]_{\extendedtoroidal} := y D( D'( g ) ) + K(DD', Y)$$
                    $$[ D', y D( g ) ]_{\extendedtoroidal} := y D( D'( g ) ) + K(D'D, Y)$$
                At the same time, we have that:
                    $$
                        \begin{aligned}
                            & [ [D, D']_{\extendedtoroidal}, y g ]_{\extendedtoroidal}
                            \\
                            = & [ X(D, D') + Z(D, D') + \xi(D, D') , y g ]_{\extendedtoroidal}
                            \\
                            = & [ X(D, D') + \xi(D, D') , y g ]_{\extendedtoroidal}
                            \\
                            = & [ X(D, D') , y g ]_{\extendedtoroidal} + \left( y \xi(D, D')(g) + K_{\xi(D, D'), Y} \right)
                        \end{aligned}
                    $$
                wherein the second equality holds thanks to the fact that $[\z(\toroidal), \g_{[2]}]_{\extendedtoroidal} = 0$, and $K_{\xi(D, D'), Y} \in \z(\toroidal)$ is some element (cf. lemma \ref{lemma: derivation_action_on_multiloop_algebras}). Combining the two observations together then yields:
                    $$
                        \begin{aligned}
                            & [ X(D, D') , y g ]_{\extendedtoroidal} + \left( y \xi(D, D')(g) + K_{\xi(D, D'), Y} \right)
                            \\
                            = & y (DD' - D'D)( g ) + ( K(DD', Y) - K(D'D, Y) ) + ( [ D, K(D', Y) ]_{\extendedtoroidal} - [ D', K(D, Y) ]_{\extendedtoroidal} )
                        \end{aligned}
                    $$
                There exists $K_{X(D, D'), Y} \in \z(\toroidal)$ such that:
                    $$[ X(D, D') , y g ]_{\extendedtoroidal} = [ X(D, D') , Y ]_{\extendedtoroidal} = [X(D, D'), Y]_{\g_{[2]}} + K_{X(D, D'), Y}$$
                using which we can write:
                    $$
                        \begin{aligned}
                            & [X(D, D'), Y]_{\g_{[2]}} - y \left( ( DD' - D'D) - \xi(D, D') \right)( g )
                            \\
                            = & \left( [ D, K(D', Y) ]_{\extendedtoroidal} - [ D', K(D, Y) ]_{\extendedtoroidal} \right) - \left( K_{X(D, D'), Y} + K_{\xi(D, D'), Y} \right)
                        \end{aligned}
                    $$
                    
                We note right away that the LHS lies entirely in $\g_{[2]}$, whereas the RHS is an element of $\z(\toroidal)$ due to the fact that $[\divzero, \z(\toroidal)]_{\extendedtoroidal} \subseteq \z(\toroidal)$ (cf. lemma \ref{lemma: derivation_action_on_toroidal_centres}), which tells us that $[ D, K(D', Y) ]_{\extendedtoroidal}, [ D', K(D, Y) ]_{\extendedtoroidal} \in \z(\toroidal)$ in particular. Because $\g_{[2]}$ is centreless (as $\g$ is simple and the Lie bracket on $\g_{[2]}$ is given by extension of scalars), this observation subsequently implies that the LHS must vanish, i.e.:
                    $$[X(D, D'), Y]_{\g_{[2]}} - y \left( ( DD' - D'D) - \xi(D, D') \right)( g ) = 0$$
                Because we have by construction that:
                    $$DD' - D'D - \xi(D, D') \in \divzero$$
                we now make the following claim: \textit{if we fix some arbitrary $E \in \g_{[2]}$ and some $P \in \divzero$ then:}
                    $$\forall H := h \varphi \in \g_{[2]}: [E, H]_{\g_{[2]}} = h P( \varphi ) \implies E = 0$$

                Using the root space decomposition for $\g$ (cf. theorem \ref{theorem: root_space_decomposition_for_finite_dimensional_simple_lie_algebras}), we see that if $h \in \h$ then we then will have that $[E, H]_{\g_{[2]}} \in \n^{\pm}_{[2]}$, but at the same time, that $h P(\varphi) \in \h_{[2]}$. The only way for this to be true is that $[E, H]_{\g_{[2]}} = 0$, which is the case if and only if $E = 0$. If $h \in \n^{\pm}$, then $[E, H]_{\g_{[2]}} \in \n^{\pm}_{[2]} \oplus \h_{[2]}$ and the $\h_{[2]}$-summand will be non-zero in general; at the same time, $h P(\varphi) \in \n^{\pm}_{[2]}$ in this case, and again, the only way for these to facts to be true simultaneously is that $E = 0$ necessarily. 

                Apply the claim to the fact that:
                    $$[X(D, D'), Y]_{\g_{[2]}} = y \left( ( DD' - D'D) - \xi(D, D') \right)( g )$$
                - and again, note that $( DD' - D'D) - \xi(D, D') \in \divzero$ - then yields:
                    $$X(D, D') = 0$$
                precisely as desired. 
            \end{proof}
        \begin{corollary}
            For any $D, D' \in \divzero$, the $\divzero$-summand of $[D, D']_{\extendedtoroidal}$ is nothing but the commutator $DD' - D'D$.
        \end{corollary}
        \begin{corollary}[$\gamma$-extended toroidal Lie algebras are extensions] \label{coro: yangian_extended_toroidal_lie_algebras_are_extensions}
            When endowed with a Lie bracket, $\extendedtoroidal$ is necessarily a Lie algebra extension of $\divzero$ by $\divzero$.
        \end{corollary}
            \begin{proof}
                We know from corollary \ref{coro: toroidal_lie_algebras_are_ideals} that $\toroidal$ is a Lie ideal of $\extendedtoroidal$, and that there is a canonical surjective map $\pi: \extendedtoroidal \to \divzero$ such that $\pi([D, D']_{\extendedtoroidal}) = [D, D']$ with the RHS denoting the usual commutator $DD' - D'D$. As such, we have a short exact sequence of Lie algebras:
                    $$0 \to \toroidal \to \extendedtoroidal \to \divzero \to 0$$
                i.e. an extension.
            \end{proof}
        \begin{corollary}[\texorpdfstring{$\z(\toroidal)$}{}-summands of elements of \texorpdfstring{$[\divzero, \g_{[2]}]_{\extendedtoroidal}$}{}] \label{coro: derivation_action_on_multiloop_algebras}
            The action of $\divzero$ on $\g_{[2]}$ as in lemma \ref{lemma: derivation_action_on_multiloop_algebras} satisfies:
                $$[\divzero, \g_{[2]}]_{\extendedtoroidal} \subseteq \g_{[2]}$$
            Explicitly, the basis elements $D_{r, s}, D_v, D_t \in \divzero$ thus act on generators of $\g_{[2]}$ (i.e. monomials of the form $x v^m t^p$ for some $x \in \g$ and some $(m, p) \in \Z^2$) in the following manner:
                $$[D_{r, s}, x v^m t^p]_{\extendedtoroidal} = (rp - ms) x v^{m - r} t^{p - s - 1}$$
                $$[D_v, x v^m t^p]_{\extendedtoroidal} = -m x v^m t^{p - 1}$$
                $$[D_t, x v^m t^p]_{\extendedtoroidal} = -p x v^m t^{p - 1}$$
        \end{corollary}
            \begin{proof}
                From lemma \ref{lemma: derivation_action_on_multiloop_algebras}, we know that given some $D \in \divzero$ and some $x \in \g$ and $f \in A$, there shall exist $K(D, xf) \in \z(\toroidal)$ (depending on the choices of $D$ and $x, f$) such that:
                    $$[D, xf]_{\extendedtoroidal} = x D(f) + K(D, xf)$$
                Next, consider the following:
                    $$( \divzero, x D(f) + K(D, xf) )_{\extendedtoroidal} = ( \divzero, [D, xf]_{\extendedtoroidal} )_{\extendedtoroidal} = ( [\divzero, D]_{\extendedtoroidal}, xf )_{\extendedtoroidal} = 0$$
                where the second equality holds thanks to invariance, and the third equality holds due to a combination of the fact that $[\divzero, \divzero]_{\extendedtoroidal} \subset \z(\toroidal) \oplus \divzero$ (cf. proposition \ref{prop: lie_bracket_on_orthogonal_complement_of_toroidal_centre}) and the fact that $(\z(\toroidal) \oplus \divzero, \g_{[2]})_{\extendedtoroidal}$ per the construction of the bilinear form $(-, -)_{\extendedtoroidal}$. We also have the following, again per the construction of the bilinear form $(-, -)_{\extendedtoroidal}$:
                    $$( \divzero, x D(f) + K(D, xf) )_{\extendedtoroidal} = ( \divzero, K(D, xf) )_{\extendedtoroidal}$$
                which means that:
                    $$( \divzero, K(D, xf) )_{\extendedtoroidal} = 0$$
                The non-degeneracy of $(-, -)_{\extendedtoroidal}$ (or more particularly, the fact that $( \divzero, \z(\toroidal) )_{\extendedtoroidal} \not = 0$) then implies that:
                    $$K(D, xf) = 0$$
                necessarily. This means that, indeed, we have that:
                    $$[D, xf]_{\extendedtoroidal} = x D(f)$$
                for all $D \in \divzero$ and all $x \in \g$ and all $f \in A$. Since $\g_{[2]}$ is generated by elements of the form $xf$, this implies that:
                    $$[\divzero, \g_{[2]}]_{\extendedtoroidal} \subseteq \g_{[2]}$$
                as claimed. 
            \end{proof}
            
        We have thus obtained the following result:
        \begin{theorem}[$\gamma$-extended toroidal Lie algebras are twisted semi-direct products] \label{theorem: yangian_extended_toroidal_lie_algebras}
            If $\fraky$ is a $\gamma$-extended toroidal Lie algebra in the sense of definition \ref{def: yangian_extended_toroidal_lie_algebras}, then there will exist some Lie $2$-cocycle $\sigma \in Z^2_{\Lie}(\divzero, \z(\toroidal))$ so that:
                $$\toroidal \rtimes^{\sigma} \divzero \xrightarrow[\cong]{\id_{\toroidal} \oplus \varphi} \fraky$$
            where $\varphi: \divzero \xrightarrow[]{\cong} \z(\toroidal)^{\star}$ being as in proposition \ref{prop: yangian_div_zero_vector_fields_are_graded_dual_to_toroidal_centre}.
        \end{theorem}
            \begin{proof}
                Combine proposition \ref{prop: toroidal_lie_algebras_as_modules_over_div_zero_vector_field_lie_algebras} with corollary \ref{coro: yangian_extended_toroidal_lie_algebras_are_extensions}, and then apply proposition \ref{prop: twisted_semi_direct_product_criterion}.
            \end{proof}

        It remains to prove the following result:
        \todo[inline]{Dr. Wendlandt: Is this what you meant by the "invariance part" of the proof ?}
        \begin{proposition} \label{prop: yangian_cocycle_invariance}
            If $\fraky \cong \toroidal \rtimes^{\sigma} \divzero$ is a $\gamma$-extended toroidal Lie algebra (corresponding to some $2$-cocycle $\sigma \in Z^2_{\Lie}(\divzero, \z(\toroidal))$), then:
                $$(\sigma(D, D'), D'')_{\extendedtoroidal} = (D, \sigma(D', D''))_{\extendedtoroidal}$$
            for all $D, D', D'' \in \divzero$.
        \end{proposition}
            \begin{proof}
                Per definition \ref{def: twisted_semi_direct_products}, we know that:
                    $$\sigma(\xi, \xi') = [\xi, \xi']_{\fraky} - [\xi, \xi]$$
                for all $\xi, \xi' \in \divzero$, meaning that we have that:
                    $$
                        \begin{aligned}
                            (\sigma(D, D'), D'')_{\extendedtoroidal} & = ([D, D]_{\fraky} - [D, D'], D'')_{\extendedtoroidal}
                            \\
                            & = ([D, D]_{\fraky}, D'')_{\extendedtoroidal} 
                            \\
                            & = (D, [D', D'']_{\fraky})
                            \\
                            & = (D, [D', D'']_{\fraky} - [D', D''])_{\extendedtoroidal}
                            \\
                            & = (D, \sigma(D', D''))_{\extendedtoroidal}
                        \end{aligned}
                    $$
                which is precisely as needed.
            \end{proof}

        Together, theorem \ref{theorem: yangian_extended_toroidal_lie_algebras} and proposition \ref{prop: yangian_cocycle_invariance} imply the \say{only if} direction of theorem \ref{theorem: yangian_extended_toroidal_lie_algebras_main_theorem}.