\section{Construction of Yangian extended toroidal Lie algebras}
    \subsection{Yangian divergence-zero vector fields}
        \todo[inline]{Computed basis of $\divzero$.}
        \begin{lemma}[Yangian divergence-zero vector fields] \label{lemma: yangian_div_zero_vector_fields_basis}
            Let $\divzero$ be the vector subspace of $\der(A)$ defined as follows:
                $$\divzero := \{ D \in \der(A) \mid \forall f \in A: \gamma(D(f)) = 0 \}$$
            A basis for this vector subspace is then the set:
                $$\{D_{r, s}\}_{(r, s) \in \Z^2} \cup \{D_v, D_t\}$$
            whose elements are given in terms of the partial derivatives $\del_v := \frac{\del}{\del v}$ and $\del_t := \frac{\del}{\del t}$ by:
                $$D_{r, s} := -s v^{-r + 1} t^{-s - 1} \del_v + r v^{-r} t^{-s} \del_t$$
                $$D_v := -v t^{-1} \del_v$$
                $$D_t := -\del_t$$
        \end{lemma}
            \begin{proof}
                Any element $D \in \divzero$ is, of course, an element of $\der(A) \cong A \del_v \oplus A \del_t$, and hence can be written as:
                    $$D := \sum_{(a, b) \in \Z^2} \left( \lambda_{a, b} v^a t^b \del_v + \mu_{a, b} v^a t^b \del_t \right)$$
                for some $\lambda_{a, b}, \mu_{a, b} \in \bbC$. Consider then the following, where $f \in A$ is arbitrary:
                    $$0 =\gamma(D(f)) = \sum_{(a, b) \in \Z^2} \left( \lambda_{a, b} \gamma(v^r t^s \del_v f) + \mu_{a, b} \gamma(v^a t^b \del_t f) \right)$$
                Without any loss of generality, we can take $f \in A$ to be a basis element, i.e. $f := v^m t^p$ for some $(m, p) \in \Z^2$. Doing so yields:
                    $$
                        \begin{aligned}
                            0 & = \sum_{(a, b) \in \Z^2} \left( \lambda_{a, b} \gamma(v^a t^b \del_v(v^m t^p)) + \mu_{a, b} \gamma(v^a t^b \del_t(v^m t^p)) \right)
                            \\
                            & = \sum_{(a, b) \in \Z^2} \left( m\lambda_{a, b} \gamma(v^{a + m - 1} t^{b + p}) + p \mu_{a, b} \gamma( v^{a + m} t^{b + p - 1} ) \right)
                            \\
                            & = -\sum_{(a, b) \in \Z^2} \left( m \lambda_{a, b} \delta_{(a + m - 1, b + p), (0, -1)} + p \mu_{a, b} \delta_{(a + m, b + p - 1), (0, -1)} \right)
                            \\
                            & = -\sum_{(a, b) \in \Z^2} \left( m \lambda_{a, b} \delta_{(a + m, b + p), (1, -1)} + p \mu_{a, b} \delta_{(a + m, b + p), (0, 0)} \right)
                            \\
                            & = -\left( m \lambda_{-m + 1, -p - 1} + p \mu_{-m, -p} \right)
                        \end{aligned} 
                    $$
                for all $(m, p) \in \Z^2$. From this, one sees that:
                    $$D = \sum_{(r, s) \in \Z^2} \lambda_{r, s} D_{r, s} + \lambda_v D_v + \lambda_t D_t$$
                for some $\lambda_{r, s}, \lambda_v, \lambda_t \in \bbC$, where:
                    $$D_{r, s} := -s v^{-r + 1} t^{-s - 1} \del_v + r v^{-r} t^{-s} \del_t$$
                    $$D_v := -v t^{-1} \del_v$$
                    $$D_t := -\del_t$$
                These elements are clearly linearly independent, so we are done.
            \end{proof}
        \todo[inline]{Showed that $\divzero \cong \z_{[2]}^{\star}$ as $\Z^2$-graded vector spaces.}
        \begin{lemma}[Commutators of Yangian divergence-zero vector fields] \label{lemma: commutators_of_yangian_div_zero_vector_fields}
            $\divzero$ is a Lie subalgebra of $\der(A)$. In particular, the basis elements of $\divzero$ satify the following commutation relations:
                $$[D_v, D_t] = 0$$
                $$[D_v, D_{r, s}] = -r D_{r, s + 1}$$
                $$[D_t, D_{r, s}] = -s D_{r, s + 1}$$
                $$[D_{a, b}, D_{r, s}] = (br - as) D_{a + r, b + s + 1}$$
        \end{lemma}
            \begin{proof}
                \begin{enumerate}
                    \item Since we know that:
                        $$D_v = -vt^{-1} \del_v, D_t = -\del_t$$
                    (cf. lemma \ref{lemma: yangian_div_zero_vector_fields_basis}), it is therefore trivial that:
                        $$[D_v, D_t] = 0$$
                    \item From lemma \ref{lemma: yangian_div_zero_vector_fields_basis}, we know that:
                        $$D_v(v^m t^p) = -m v^m t^{p - 1}$$
                        $$D_{r, s}(v^m t^p) = ( rp - ms ) v^{m - r} t^{p - s - 1}$$
                    From this, we infer that:
                        $$
                            \begin{aligned}
                                & [D_v, D_{r, s}](v^m t^p)
                                \\
                                = & D_v( D_{r, s}(v^m t^p) ) - D_{r, s}( D_v(v^m t^p) )
                                \\
                                = & (rp - ms) D_v( v^{m - r} t^{p - s - 1} ) + m D_{r, s}( v^m t^{p - 1} )
                                \\
                                = & -(m - r)(rp - ms) v^{m - r} t^{p - s - 2} + (ms - r(p - 1)) m v^{m - r} t^{p - s - 2}
                                \\
                                = & -r(rp - m(s + 1)) v^{m - r} t^{p - (s + 1) - 1}
                                \\
                                = & -r D_{r, s + 1}(v^m t^p)
                            \end{aligned}
                        $$
                    and hence:
                        $$[D_v, D_{r, s}] = -r D_{r, s + 1}$$
                    \item Likewise, we have that:
                        $$[D_t, D_{r, s}] = -s D_{r, s + 1}$$
                    \item Using the same method, we can show that:
                        $$[D_{a, b}, D_{r, s}] = (br - as) D_{a + r, b + s + 1}$$
                \end{enumerate}
            \end{proof}
        \begin{corollary}[$\Z^2$-grading on Yangian divergence-zero vector fields] \label{coro: yangian_div_zero_vector_fields_are_graded}
            $\divzero$ is a $\Z^2$-graded Lie subalgebra of the $\Z^2$-graded Lie algebra $\der(A)$. Namely, the grading on $\divzero$ is given by\footnote{The reason for this choice of grading will become clear after proposition \ref{prop: yangian_div_zero_vector_fields_are_graded_dual_to_toroidal_centre}, which asserts that $\divzero \cong \z_{[2]}^{\star}$ as $\Z^2$-graded vector spaces; recall that the latter has a natural $\Z^2$-grading given by $\deg K_{r, s} = (r, s), \deg c_v = \deg c_t = 0$.}:
                $$\deg D_{r, s} := (-r, -s - 1)$$
                $$\deg D_v = \deg D_t = (0, -1)$$
        \end{corollary}
        \begin{proposition}[Yangian divergence-zero vector fields are graded-dual to toroidal centre] \label{prop: yangian_div_zero_vector_fields_are_graded_dual_to_toroidal_centre}
            There is a $\Z^2$-graded vector space isomorphism:
                $$\varphi: \divzero \xrightarrow[]{\cong} \z_{[2]}^{\star}$$
            determined by:
                $$\varphi(D)( f\bar{d}g ) := \gamma( f D(g) )$$
            for all $D \in \divzero$. This identifies the basis $\{D_{r, s}\}_{(r, s) \in \Z^2} \cup \{D_v, D_t\}$ of $\divzero$ as being $\Z^2$-graded dual to the basis $\{K_{r, s}\}_{(r, s) \in \Z^2} \cup \{c_v, c_t\}$
        \end{proposition}
            \begin{proof}
                First of all, let us prove that $\varphi: \divzero \xrightarrow[]{\cong} \z_{[2]}^{\star}$ as given is graded, and it is enough to check this on the basis elements: we claim that, because:
                    $$\deg D_{r, s} = (-r, -s - 1)$$
                    $$\deg D_v = \deg D_t = (0, 0)$$
                (cf. corollary \ref{coro: yangian_div_zero_vector_fields_are_graded}) and because:
                    $$\deg K_{a, b} = (a, b)$$
                    $$\deg c_v = \deg c_t = (0, 0)$$
                (cf. remark \ref{remark: Z^2_grading_on_toroidal_centres}), we ought to have that:
                    $$\varphi(D_{r, s})(K_{a, b}) = \delta_{(a - r, b - s - 1), (-1, -1)}$$
                    $$\varphi(D_v)(c_v) = \varphi(D_t)(c_t) = 1$$
                for $D_{r, s}, D_v, D_t$ to be identified - respectively - as $\Z^2$-graded dual basis elements corresponding to $K_{r, s}, c_v, c_t$; a straightforward dimension argument will then show that $\varphi$ must be a vector space isomorphism. Indeed, we have that:
                    $$
                        \begin{aligned}
                            \varphi(D_{r, s})(K_{a, b}) & = 
                            \begin{cases}
                                \text{$\gamma\left( \frac1b v^{a - 1} t^b D_{r, s}(v) \right)$ if $(a, b) \in \Z \x (\Z \setminus \{0\})$}
                                \\
                                \text{$\gamma\left( -\frac1a v^a t^{-1} \bar{d}t \right)$ if $(a, b) \in (\Z \setminus \{0\}) \x \{0\}$}
                                \\
                                \text{$0$ if $(a, b) = (0, 0)$}
                            \end{cases}
                            \\
                            & = 
                            \begin{cases}
                                \text{$\gamma\left( -\frac{s}{b} v^{a - r} t^{b - s - 1} \right)$ if $(a, b) \in \Z \x (\Z \setminus \{0\})$}
                                \\
                                \text{$\gamma\left( -\frac{r}{a} v^{a - r} t^{-s - 1} \right)$ if $(a, b) \in (\Z \setminus \{0\}) \x \{0\}$}
                                \\
                                \text{$0$ if $(a, b) = (0, 0)$}
                            \end{cases}
                            \\
                            & = \delta_{(a - r, b - s - 1), (-1, -1)}
                        \end{aligned}
                    $$
                as well as:
                    $$\varphi(D_v)(c_v) = \gamma(v^{-1} D_v(v)) = \gamma( -t^{-1} ) = 1$$
                    $$\varphi(D_t)(c_t) = \gamma(t^{-1} D_t(t)) = \gamma( -t^{-1} ) = 1$$
                Since:
                    $$\divzero \cong \bigoplus_{(r, s) \in \Z^2} \bbC D_{r, s} \oplus \bbC D_v \oplus \bbC D_t$$
                    $$\z_{[2]}^{\star} \cong \bigoplus_{(r, s) \in \Z^2} (\bbC K_{r, s})^* \oplus (\bbC c_v)^* \oplus (\bbC c_t)^*$$
                the computations above are enough to show that:
                    $$\varphi: \divzero \to \z_{[2]}^{\star}$$
                as given is a vector space isomorphism.
            \end{proof}
        \begin{corollary}[Lie brackets on graded-duals of the toroidal centres] \label{coro: lie_bracket_on_graded_dual_of_toroidal_centres}
            $\z_{[2]}^{\star}$ is naturally a Lie algebra via the vector space isomorphism $\varphi: \divzero \xrightarrow[]{\cong} \z_{[2]}^{\star}$.
        \end{corollary}
        \begin{corollary}[A non-degenerate bilinear form extending $(-, -)_{\toroidal}$] \label{coro: pairing_yangian_div_zero_vector_fields_and_cyclic_1_forms}
            On the vector space $\z_{[2]} \oplus \divzero$, there is a natural non-degenerate symmetric bilinear form $\<-, -\>$ given by:
                $$( K, D )_{\varphi} := \varphi(D)(K)$$
            for all $K \in \z_{[2]}$ and all $D \in \divzero$. This induces a non-degenerate symmetric bilinear form $(-, -)_{\toroidal \oplus \divzero}$ on the vector space $\toroidal \oplus \divzero$, given by:
                $$(xf, yg)_{\toroidal \oplus \divzero} := (xf, yg)_{\g_{[2]}}$$
                $$(K, D)_{\toroidal \oplus \divzero} := (K, D)_{\varphi}$$
                $$(xf, K)_{\toroidal \oplus \divzero} := (xf, K)_{\toroidal} = 0$$
                $$(xf, D)_{\toroidal \oplus \divzero} := 0$$
            for all $x, y \in \g, f, g \in A$ and $K \in \z_{[2]}, D \in \divzero$, extending the bilinear form $(-, -)_{\toroidal}$.
        \end{corollary}
        \begin{convention}
            Since we now have a vector space isomorphism:
                $$\toroidal \oplus \divzero \xrightarrow[\cong]{\id_{\toroidal} \oplus \varphi} \toroidal \oplus \z_{[2]}^{\star} =: \extendedtoroidal$$
            we shall abuse notations and equate $(-, -)_{\toroidal \oplus \divzero}$ with $(-, -)_{\extendedtoroidal}$.
        \end{convention}
        \begin{remark}
            Note that the pairing $(-, -)_{\toroidal}$ as in corollary \ref{coro: pairing_yangian_div_zero_vector_fields_and_cyclic_1_forms}, when regarded as an element of $\z_{[2]} \tensor_{\bbC} \z_{[2]}^{\star}$, which has a canonical $\Z^2$-grading coming from those on $\z_{[2]}$ and $\z_{[2]}^{\star}$, has total degree $-1$ due to the choice of $\Z^2$-grading on $\divzero$ (and hence on $\z_{[2]}^{\star}$, thanks to proposition \ref{prop: yangian_div_zero_vector_fields_are_graded_dual_to_toroidal_centre}) that was made in corollary \ref{coro: yangian_div_zero_vector_fields_are_graded}.
        \end{remark}

    \subsection{Yangian extended toroidal Lie algebras are twisted semi-direct products}
        Now that we have identified an isomorphism:
            $$\varphi: \divzero \xrightarrow[]{\cong} \z_{[2]}^{\star}$$
        let us subsequently consider the vector space:
            $$\toroidal \oplus \divzero$$
        which is isomorphic to $\extendedtoroidal := \toroidal \oplus \z_{[2]}^{\star}$ via $\id_{\toroidal} \oplus \varphi$. From now on, it will always be equipped with the non-degenerate bilinear form $(-, -)_{\extendedtoroidal}$ constructed in corollary \ref{coro: pairing_yangian_div_zero_vector_fields_and_cyclic_1_forms}. We would like to identify all Yangian extended toroidal Lie algebra structures $[-, -]_{\extendedtoroidal}$ on the vector space $\extendedtoroidal$, in the sense of definition \ref{def: yangian_extended_toroidal_lie_algebras}, which is the same as identifying all Lie algebra structures on $\toroidal \oplus \divzero$ with respect to which . As mentioned earlier, our claim is that all such Lie algebra structures are isomorphic to twisted semi-direct products:
            $$\toroidal \rtimes^{\sigma} \divzero$$
        whose corresponding $2$-cocycle $\sigma \in Z^2_{\Lie}(\divzero, \toroidal)$ satisfy a certain property (which will be discussed in the next subsection).

        To characterise Yangian extended toroidal Lie algebras as certain twisted semi-direct products of the form $\toroidal \rtimes^{\sigma} \divzero$, we shall apply proposition \ref{prop: twisted_semi_direct_product_criterion}. This result requires two inputs:
        \begin{enumerate}
            \item We shall need to firstly show that $\toroidal$ is a $\divzero$-module - say, determined by a Lie algebra homomorphism $\rho: \divzero \to \gl(\toroidal)$ - such that:
                $$[\divzero, \toroidal]_{\extendedtoroidal} = \rho(\divzero) \cdot \toroidal$$
            where a linear section $\divzero \to \extendedtoroidal$ has been implicitly chosen.
            \item Secondly, we will need to show that Yangian extended toroidal Lie algebras are isomorphic as Lie algebras to certain extensions of $\divzero$ by $\toroidal$, for which we shall need to show that for any $D, D' \in \divzero$, the $\divzero$-summand of the bracket:
                $$[D, D']_{\divzero}$$
            is nothing but the usual commutator. This gives us a surjective Lie algebra homomorphism:
                $$\extendedtoroidal \to \divzero$$
            and therefore a Lie algebra extension:
                $$0 \to \toroidal \to \extendedtoroidal \to \divzero \to 0$$
        \end{enumerate}

        The first assertion is now trivial, in light of proposition \ref{prop: yangian_div_zero_vector_fields_are_graded_dual_to_toroidal_centre}. Nevertheless, let us package it into a lemma for organisational purposes and due to its importance.
        \begin{lemma}[Yangian extended toroidal Lie algebras are extensions] \label{lemma: yangian_extended_toroidal_lie_algebras_are_extensions}
            Any Lie algebra structure on the vector space $\extendedtoroidal$ shall give rise to a Lie algebra extension:
                $$0 \to \toroidal \to \fraky \to \divzero \to 0$$
            where the underlying vector space of $\fraky$ is isomorphic to $\extendedtoroidal$. 
        \end{lemma}

        Let us now prove the second assertion. 

        First of all, we shall need to fix a $\divzero$-module structure on $\toroidal$, which we shall do by fixing module structures on $\g_{[2]}$ and $\z_{[2]}$.
        \begin{convention}
            Let $\divzero$ act on $A$ in the canonical manner, i.e. evaluations of derivations $D \in \divzero$ on elements $f \in A$:
                $$D \cdot f := D(f)$$
            This induces the following obvious $\divzero$-action on $\g_{[2]}$, given by:
                $$D \cdot xf := x D(f)$$
            for all $x \in \g$ and all $f \in A$. It also induces an action of $\divzero$ on $\z_{[2]} \cong \bar{\Omega}_{[2]}$ by \say{Lie derivatives}, i.e.:
                $$D \cdot f\bar{d}g := Df \bar{d}g + f \bar{d}(D(g))$$
            As such, we have a $\divzero$-action:
                $$\rho: \divzero \to \gl(\toroidal)$$
            on $\toroidal \cong \g_{[2]} \oplus \z_{[2]}$.
        \end{convention}

        To begin, let us see that elements of $\divzero$ are certain derivations on $D$.
        \todo[inline]{Rephrased and reproved that $[\divzero, \g_{[2]}]_{\extendedtoroidal} \subseteq \g_{[2]} \oplus \z_{[2]}$}
        \begin{lemma}[$\divzero$ acts on $\g_{[2]}$ by derivations] \label{lemma: derivation_action_on_multiloop_algebras}
            For any $D \in \divzero$ and any $x \in \g, f \in A$, we have that:
                $$[D, xf]_{\extendedtoroidal} = x D(f) + K( D, xf )$$
            for some yet-unknown $K( D, xf ) \in \z_{[2]}$. As such, we have that:
                $$[\divzero, \g_{[2]}]_{\extendedtoroidal} \subseteq \g_{[2]} \oplus \z_{[2]}$$
            as of now\footnote{In corollary \ref{coro: derivation_action_on_multiloop_algebras}, it will be shown that in fact, the $\z_{[2]}$-summand $K( D, xf )$ vanishes.}.
        \end{lemma}
            \begin{proof}
                Consider firstly the following, for any $D \in \divzero$ and any $x, y \in \g, f, g \in A$:
                    $$( [D, xf]_{\extendedtoroidal}, yg )_{\extendedtoroidal} = ( D, [xy, fg]_{\toroidal} )_{\extendedtoroidal} = (x, y)_{\g} ( D, g\bar{d}f )_{\extendedtoroidal} = (x, y)_{\g} \gamma( g D(f) )$$
                where the last equality is due to corollary \ref{coro: pairing_yangian_div_zero_vector_fields_and_cyclic_1_forms}. At the same time, we have that:
                    $$( x D(f), yg )_{\extendedtoroidal} = (x, y)_{\g} \gamma( g D(f) )$$
                Clearly, then, we have that:
                    $$( [D, xf]_{\extendedtoroidal}, yg )_{\extendedtoroidal} = ( x D(f), yg )_{\extendedtoroidal}$$
                Since $yg \in \g_{[2]}$ is arbitrary and since $(\z_{[2]}, \g_{[2]})_{\extendedtoroidal} = 0$, the above implies via the non-degeneracy of $(-, -)_{\extendedtoroidal}$ that there exists some $K(D, xf) \in \z_{[2]}$ such that:
                    $$[D, xf]_{\extendedtoroidal} = x D(f) + K(D, xf)$$
            \end{proof}
    
        As we know how $\divzero$ acts on $\g_{[2]}$ (cf. lemma \ref{lemma: derivation_action_on_multiloop_algebras}), it remains to see how it acts on $\z_{[2]}$ to completely determine its action on $\toroidal$. 
        \begin{lemma}[$\divzero$ acts on $\z_{[2]}$ by Lie derivatives] \label{lemma: derivation_action_on_toroidal_centres}
            Elements of $\divzero$ act on those of $\z_{[2]}$ as Lie derivatives. This is to say that, the elements $D \in \divzero$ act on the generating elements $f \bar{d}g \in \z_{[2]}$ (for some $f, g \in A$) in the following manner:
                $$[D, f \bar{d}g]_{\extendedtoroidal} = \xi_D(f) \bar{d}g + f \bar{d}(\xi_D(g))$$
            where $\xi_D \in \der(A)$ is a derivation on $A$ determined uniquely by $A$ (well-defined because $\divzero$ is a vector subspace of $\der(A)$ per lemma \ref{lemma: derivation_action_on_multiloop_algebras}). In particular, this means that:
                $$[\divzero, \z_{[2]}]_{\extendedtoroidal} \subseteq \z_{[2]}$$
        \end{lemma}
            \begin{proof}
                Without any loss of generality, let us consider the following for any $h, h' \in \h$ so that\footnote{We can make this assumption because ultimately, elements of $\z_{[2]}$ do not depend on those of $\g$.}:
                    $$(h, h')_{\g} = 1$$
                any $f, g \in A$, and any $D \in \divzero$:
                    $$[ D, [h f, h' g]_{\toroidal} ]_{\extendedtoroidal} = [ D, f \bar{d}( g ) ]_{\extendedtoroidal}$$
                At the same time, we have via the Jacobi identity that:
                    $$
                        \begin{aligned}
                            [ D, [h f, h' g]_{\toroidal} ]_{\extendedtoroidal} & = [ h f, [D, h' g]_{\extendedtoroidal} ]_{\toroidal} + [ [D, h f]_{\extendedtoroidal}, h' g ]_{\toroidal}
                            \\
                            & = [ h f, h' D( g ) ]_{\toroidal} + [ h D( f ), h' g ]_{\toroidal}
                            \\
                            & = f \bar{d}( D( g ) ) + D( f ) \bar{d}(g)
                        \end{aligned}
                    $$
                One thus sees that:
                    $$[ D, f \bar{d}( g ) ]_{\extendedtoroidal} = f \bar{d}( D( g ) ) + D( f ) \bar{d}(g)$$
                and since the element $f \bar{d}( g )$ is central (via the map $\e$ mentioned earlier), this gives another description of:
                    $$[ \divzero, \z_{[2]} ]_{\extendedtoroidal}$$
                With this in mind, we return quickly to lemma \ref{lemma: derivation_action_on_multiloop_algebras}; there, we previously demonstrated that:
                    $$[ \divzero, \g_{[2]} ]_{\extendedtoroidal} \subseteq \g_{[2]} \oplus \z_{[2]}$$
                but we claim now that the following stronger fact holds:
                    $$[ \divzero, \g_{[2]} ]_{\extendedtoroidal} \subseteq \g_{[2]}$$
                To see why this is the case, suppose firstly that for any $D \in \divzero$, any $X := x f \in \g_{[2]}$ (for some $f \in A$), there is $K(X) \in \z_{[2]}$ depending on $X$ (and indeed, such a $K(X)$ exists by lemma \ref{lemma: derivation_action_on_multiloop_algebras}) such that:
                    $$[ D, X ]_{\extendedtoroidal} = x D( f ) + K(X)$$
                Next, pick an arbitrary element $\xi \in \divzero$ and then consider the following:
                    $$( [ D, X ]_{\extendedtoroidal}, \xi )_{\extendedtoroidal} = (D(X) + K(X), \xi)_{\extendedtoroidal} = (K(X), \xi)_{\extendedtoroidal}$$
                wherein the last equality holds as a consequence of the fact that:
                    $$( \g_{[2]}, \divzero )_{\extendedtoroidal} = 0$$
                per the construction of the bilinear form $(-, -)_{\extendedtoroidal}$ as in corollary \ref{coro: pairing_yangian_div_zero_vector_fields_and_cyclic_1_forms}.
            \end{proof}
        \begin{corollary}[Toroidal Lie algebras are ideals]
            With respect to the bracket $[-, -]_{\extendedtoroidal}$, the vector subspace $\toroidal$ is actually a Lie ideal of $\extendedtoroidal$.
        \end{corollary}
        Using lemma \ref{lemma: commutators_of_yangian_div_zero_vector_fields} in conjunction with lemma \ref{lemma: derivation_action_on_toroidal_centres}, we can now also explicitly compute the commutation relations between the basis elements of $\divzero$ and $\z_{[2]}$.
        \begin{lemma}[Explicit commutators between basis elements of $\divzero$ and $\z_{[2]}$] \label{lemma: explicit_commutators_between_central_basis_elements_and_derivations}
            In the Lie algebra $\extendedtoroidal$, one has the following commutation relations between elements of $\z_{[2]}$ and those of $\divzero$. Namely, for all $D \in \divzero$, the following relations hold:
                $$
                    \forall (a, b) \in \Z^2: [D, K_{a, b}]_{\extendedtoroidal} =
                    \begin{cases}
                        \text{$((b - 1)r - sa) K_{a - r, b - s - 1} + \delta_{(r, s + 1), (a, b)} \left( r c_v + s c_t \right)$ if $D = D_{r, s}$}
                        \\
                        \text{$a K_{a, b - 1}$ if $D_v$}
                        \\
                        \text{$b K_{a, b - 1}$ if $D_t$}
                    \end{cases}
                $$
                $$[D, c_v]_{\extendedtoroidal} = [D, c_t]_{\extendedtoroidal} = 0$$
        \end{lemma}
            \begin{proof}
                For this, we shall be making use of invariance again, namely:
                    $$(D, [D', K]_{\extendedtoroidal})_{\extendedtoroidal} = ([D, D']_{\extendedtoroidal}, K)_{\extendedtoroidal}$$
                for all $D, D' \in \divzero$ and all $K \in \z_{[2]}$, and how the brackets $[D, D']_{\extendedtoroidal}$ are given explicitly (cf. lemma \ref{lemma: commutators_of_yangian_div_zero_vector_fields}) as well as how basis elements $K \in \z_{[2]}$ pair with basis elements of $\divzero$ in the construction of $(-, -)_{\extendedtoroidal}$ (cf. corollary \ref{coro: pairing_yangian_div_zero_vector_fields_and_cyclic_1_forms}). Without any loss of generality, we can assume that $D, D' \in \divzero$ and $K \in \z_{[2]}$ are basis elements, i.e.:
                    $$D, D' \in \{D_{r, s}\}_{(r, s) \in \Z^2} \cup \{D_v, D_t\}$$
                    $$K \in \{K_{a, b}\}_{(a, b) \in \Z^2} \cup \{c_v, c_t\}$$
                and then perform the computations case-by-case, for which we shall recall from example \ref{example: toroidal_lie_algebras_centres} that:
                    $$
                        K_{a, b} :=
                        \begin{cases}
                            \text{$\frac1b v^{a - 1} t^b \bar{d}v$ if $(a, b) \in \Z \x (\Z \setminus \{0\})$}
                            \\
                            \text{$-\frac1a v^a t^{-1} \bar{d}t$ if $(a, b) \in (\Z \setminus \{0\}) \x \{0\}$}
                            \\
                            \text{$0$ if $(a, b) = (0, 0)$}
                        \end{cases}
                    $$
                    $$c_v := v^{-1} \bar{d}v, c_t := t^{-1} \bar{d}t$$
                and from lemma \ref{lemma: commutators_of_yangian_div_zero_vector_fields}, that:
                    $$[D_v, D_t] = 0$$
                    $$[D_v, D_{r, s}] = r D_{r, s + 1}$$
                    $$[D_t, D_{r, s}] = D_{r, s + 1}$$
                    $$[D_{\alpha, \beta}, D_{r, s}] = (\beta r - s \alpha) D_{\alpha + r, \beta + s + 1}$$
                For what follows, let:
                    $$D := \sum_{(\alpha, \beta) \in \Z^2} \lambda_{\alpha, \beta} D_{\alpha, \beta} + \lambda_v D_v + \lambda_t D_t$$
                for some $\lambda_{\alpha, \beta}, \lambda_v, \lambda_t \in \bbC$.
                \begin{enumerate}
                    \item Assume firstly that $K = K_{a, b}$.
                    \begin{enumerate}
                        \item If $D' = D_{r, s}$, then we shall have that:
                            $$
                                \begin{aligned}
                                    & ( D, [D_{r, s}, K_{a, b}]_{\extendedtoroidal} )_{\extendedtoroidal}
                                    \\
                                    = & ( [D, D_{r, s}]_{\extendedtoroidal}, K_{a, b} )_{\extendedtoroidal}
                                    \\
                                    = & \sum_{(\alpha, \beta) \in \Z^2} (\beta r - s \alpha) \lambda_{\alpha, \beta} \delta_{(\alpha + r, \beta + s + 1), (a, b)} + \delta_{(r, s + 1), (a, b)} \left( r\lambda_v + s\lambda_t \right)
                                    \\
                                    = & ((b - s - 1) r - s (a - r)) \lambda_{a - r, b - s - 1} + \delta_{(r, s + 1), (a, b)} \left( r\lambda_v + s\lambda_t \right)
                                    \\
                                    = & ((b - 1)r - sa) \lambda_{a - r, b - s - 1} + \delta_{(r, s + 1), (a, b)} \left( r\lambda_v + s\lambda_t \right)
                                \end{aligned}
                            $$
                        from which we are able to conclude that:
                            $$[D_{r, s}, K_{a, b}]_{\extendedtoroidal} = ((b - 1)r - sa) K_{a - r, b - s - 1} + \delta_{(r, s + 1), (a, b)} \left( r c_v + s c_t \right)$$
                        \item If $D' = D_v$, then:
                            $$
                                \begin{aligned}
                                    & ( D, [D_v, K_{a, b}]_{\extendedtoroidal} )_{\extendedtoroidal}
                                    \\
                                    = & ( [D, D_v]_{\extendedtoroidal}, K_{a, b} )_{\extendedtoroidal}
                                    \\
                                    = & \sum_{(\alpha, \beta) \in \Z^2} \alpha \lambda_{\alpha, \beta} \delta_{(\alpha, \beta + 1), (a, b)} 
                                    \\
                                    = & a \lambda_{a, b - 1}
                                \end{aligned}
                            $$
                        from which we are able to conclude that:
                            $$[D_v, K_{a, b}]_{\extendedtoroidal} = aK_{a, b - 1}$$   
                        \item Finally, if $D' = D_t$, then:
                            $$
                                \begin{aligned}
                                    & ( D, [D_t, K_{a, b}]_{\extendedtoroidal} )_{\extendedtoroidal}
                                    \\
                                    = & ( [D, D_t]_{\extendedtoroidal}, K_{a, b} )_{\extendedtoroidal}
                                    \\
                                    = & \sum_{(\alpha, \beta) \in \Z^2} \beta \lambda_{\alpha, \beta} \delta_{(\alpha, \beta + 1), (a, b)} 
                                    \\
                                    = & b \lambda_{a, b - 1}
                                \end{aligned}
                            $$
                        from which we are able to conclude that:
                            $$[D_t, K_{a, b}]_{\extendedtoroidal} = b K_{a, b - 1}$$
                    \end{enumerate}
                    \item If $K = c_v$ or $K = c_t$ then simply note that because:
                        $$[D, D']_{\extendedtoroidal} \in \bigoplus_{(\alpha, \beta) \in \Z^2} \bbC D_{\alpha, \beta}$$
                    for all $D' \in \divzero$, we shall have that:
                        $$[D, K]_{\extendedtoroidal} = 0$$
                    for all $D := \sum_{(\alpha, \beta) \in \Z^2} \lambda_{\alpha, \beta} D_{\alpha, \beta} + \lambda_v D_v + \lambda_t D_t \in \divzero$.
                \end{enumerate}
            \end{proof}
        We have now yielded the following intermediate conclusion:
        \begin{proposition}[$\toroidal$ as a $\divzero$-module] \label{prop: toroidal_lie_algebras_as_modules_over_div_0_vector_field_lie_algebras}
            The $\divzero$-module structure:
                $$\rho: \divzero \to \gl(\toroidal)$$
            is such that:
                $$[\divzero, \toroidal]_{\extendedtoroidal} = \rho(\divzero) \cdot \toroidal$$
        \end{proposition}

        Finally, let us investigate how the brackets of the form:
            $$[D, D']_{\extendedtoroidal}$$
        are given. Should it be true that the $\divzero$-summand of the brackets of the form $[D, D']_{\extendedtoroidal}$ are the usual commutators $[D, D']$, we will be able to identify $\extendedtoroidal$ as a Lie algebra extension of $\divzero$ by $\toroidal$. It turns out that this is indeed true and in fact, we have that:
            $$[D, D']_{\extendedtoroidal} \in \z_{[2]} \oplus \divzero$$
        which suggests to us that the difference:
            $$\sigma(D, D') := [D, D']_{\extendedtoroidal} - [D, D']$$
        gives rise to a Lie $2$-cocycle $\sigma \in Z^2_{\Lie}(\divzero, \z_{[2]})$. By combining this with proposition \ref{prop: toroidal_lie_algebras_as_modules_over_div_0_vector_field_lie_algebras}, we shall get the desired result, i.e. that $\extendedtoroidal$ is isomorphic to a twisted semi-direct product $\toroidal \rtimes^{\sigma} \divzero$. 
        \begin{proposition}[How does $\divzero$ act on itself] \label{prop: lie_bracket_on_orthogonal_complement_of_toroidal_centre}
            We have that:
                $$[ \divzero, \divzero ]_{\extendedtoroidal} \subset \z_{[2]} \oplus \divzero$$
            i.e. the $\g_{[2]}$-summand of any commutator of the kind $[D, D']_{\extendedtoroidal}$ (for any two $D, D' \in \divzero$) actually vanishes. Furthermore, neither the $\z_{[2]}$- nor the $\divzero$-summand of those commutators $[D, D']_{\extendedtoroidal}$ necessarily vanish in general. 
        \end{proposition}
            \begin{proof}
                For convenience, we will be abbreviating $\h_{[2]} := \h[v^{\pm}, t^{\pm 1}]$ and $\n^{\pm}_{[2]} := [v^{\pm}, t^{\pm 1}]$, with $\n^{\pm} := \bigoplus_{\alpha \in \Phi^{\pm}} \g_{\alpha}$ being the direct sums of the positive/negative roots spaces of $\g$, as usual.
            
                Pick arbitrary elements $D, D' \in \divzero$ and set:
                    $$[D, D']_{\extendedtoroidal} := X(D, D') + Z(D, D') + \xi(D, D')$$
                for some $X(D, D') \in \g_{[2]}, Z(D, D') \in \z_{[2]}$, and $\xi(D, D') \in \divzero$ depending on $D, D'$. Pick also a test element $y g \in \g_{[2]}$, for some arbitrary $y \in \g$ and $g \in A$ and set:
                    $$[D, y g]_{\extendedtoroidal} := y D( g ) + K(D, Y)$$
                    $$[D', y g]_{\extendedtoroidal} := y D'( g ) + K(D', Y)$$
                for some $K(D, Y) \in \z_{[2]}$ depending on $Y$ (cf. lemma \ref{lemma: derivation_action_on_multiloop_algebras}).
                
                Via the Jacobi identity, we get that:
                    $$
                        \begin{aligned}
                            & [ [D, D']_{\extendedtoroidal}, y g ]_{\extendedtoroidal}
                            \\
                            = & [ D, [ D', y g ]_{\extendedtoroidal} ]_{\extendedtoroidal} + [ D', [ y g, D ]_{\extendedtoroidal} ]_{\extendedtoroidal}
                            \\
                            = & [ D, y D'( g ) + K(D', Y) ]_{\extendedtoroidal} - [ D', y D( g ) + K(D, Y) ]_{\extendedtoroidal}
                            \\
                            = & \left( y D( D'(g) ) + K(DD', Y) + [ D, K(D', Y) ]_{\extendedtoroidal} \right) - \left( y D'( D(g) ) + K(D'D, Y) + [ D', K(D, Y) ]_{\extendedtoroidal} \right)
                            \\
                            = & y (DD' - D'D)( g ) + ( K(DD', Y) - K(D'D, Y) ) + ( [ D, K(D', Y) ]_{\extendedtoroidal} - [ D', K(D, Y) ]_{\extendedtoroidal} )
                        \end{aligned}
                    $$
                for some $K(DD', Y), K(D'D, Y) \in \z_{[2]}$ such that:
                    $$[ D, y D'( g ) ]_{\extendedtoroidal} := y D( D'( g ) ) + K(DD', Y)$$
                    $$[ D', y D( g ) ]_{\extendedtoroidal} := y D( D'( g ) ) + K(D'D, Y)$$
                At the same time, we have that:
                    $$
                        \begin{aligned}
                            & [ [D, D']_{\extendedtoroidal}, y g ]_{\extendedtoroidal}
                            \\
                            = & [ X(D, D') + Z(D, D') + \xi(D, D') , y g ]_{\extendedtoroidal}
                            \\
                            = & [ X(D, D') + \xi(D, D') , y g ]_{\extendedtoroidal}
                            \\
                            = & [ X(D, D') , y g ]_{\extendedtoroidal} + \left( y \xi(D, D')(g) + K_{\xi(D, D'), Y} \right)
                        \end{aligned}
                    $$
                wherein the second equality holds thanks to the fact that $[\z_{[2]}, \g_{[2]}]_{\extendedtoroidal} = 0$, and $K_{\xi(D, D'), Y} \in \z_{[2]}$ is some element (cf. lemma \ref{lemma: derivation_action_on_multiloop_algebras}). Combining the two observations together then yields:
                    $$
                        \begin{aligned}
                            & [ X(D, D') , y g ]_{\extendedtoroidal} + \left( y \xi(D, D')(g) + K_{\xi(D, D'), Y} \right)
                            \\
                            = & y (DD' - D'D)( g ) + ( K(DD', Y) - K(D'D, Y) ) + ( [ D, K(D', Y) ]_{\extendedtoroidal} - [ D', K(D, Y) ]_{\extendedtoroidal} )
                        \end{aligned}
                    $$
                There exists $K_{X(D, D'), Y} \in \z_{[2]}$ such that:
                    $$[ X(D, D') , y g ]_{\extendedtoroidal} = [ X(D, D') , Y ]_{\extendedtoroidal} = [X(D, D'), Y]_{\g_{[2]}} + K_{X(D, D'), Y}$$
                using which we can write:
                    $$
                        \begin{aligned}
                            & [X(D, D'), Y]_{\g_{[2]}} - y \left( ( DD' - D'D) - \xi(D, D') \right)( g )
                            \\
                            = & \left( [ D, K(D', Y) ]_{\extendedtoroidal} - [ D', K(D, Y) ]_{\extendedtoroidal} \right) - \left( K_{X(D, D'), Y} + K_{\xi(D, D'), Y} \right)
                        \end{aligned}
                    $$
                    
                We note right away that the LHS lies entirely in $\g_{[2]}$, whereas the RHS is an element of $\z_{[2]}$ due to the fact that $[\divzero, \z_{[2]}]_{\extendedtoroidal} \subseteq \z_{[2]}$ (cf. lemma \ref{lemma: derivation_action_on_toroidal_centres}), which tells us that $[ D, K(D', Y) ]_{\extendedtoroidal}, [ D', K(D, Y) ]_{\extendedtoroidal} \in \z_{[2]}$ in particular. Because $\g_{[2]}$ is centreless (as $\g$ is simple and the Lie bracket on $\g_{[2]}$ is given by extension of scalars), this observation subsequently implies that the LHS must vanish, i.e.:
                    $$[X(D, D'), Y]_{\g_{[2]}} - y \left( ( DD' - D'D) - \xi(D, D') \right)( g ) = 0$$
                Because we have by construction that:
                    $$DD' - D'D - \xi(D, D') \in \divzero$$
                we now make the following claim: \textit{if we fix some arbitrary $E \in \g_{[2]}$ and some $P \in \divzero$ then:}
                    $$\forall H := h \varphi \in \g_{[2]}: [E, H]_{\g_{[2]}} = h P( \varphi ) \implies E = 0$$

                Using the root space decomposition for $\g$, we see that if $h \in \h$ then we then will have that $[E, H]_{\g_{[2]}} \in \n^{\pm}_{[2]}$, but at the same time, that $h P(\varphi) \in \h_{[2]}$. The only way for this to be true is that $[E, H]_{\g_{[2]}} = 0$, which is the case if and only if $E = 0$. If $h \in \n^{\pm}$, then $[E, H]_{\g_{[2]}} \in \n^{\pm}_{[2]} \oplus \h_{[2]}$ and the $\h_{[2]}$-summand will be non-zero in general; at the same time, $h P(\varphi) \in \n^{\pm}_{[2]}$ in this case, and again, the only way for these to facts to be true simultaneously is that $E = 0$ necessarily. 

                Apply the claim to the fact that:
                    $$[X(D, D'), Y]_{\g_{[2]}} = y \left( ( DD' - D'D) - \xi(D, D') \right)( g )$$
                - and again, note that $( DD' - D'D) - \xi(D, D') \in \divzero$ - then yields:
                    $$X(D, D') = 0$$
                precisely as desired. 
            \end{proof}
        \begin{corollary}
            For any $D, D' \in \divzero$, the $\divzero$-summand of $[D, D']_{\extendedtoroidal}$ is nothing but the commutator $DD' - D'D$.
        \end{corollary} 
        \begin{corollary}[\texorpdfstring{$\z_{[2]}$}{}-summands of elements of \texorpdfstring{$[\divzero, \g_{[2]}]_{\extendedtoroidal}$}{}] \label{coro: derivation_action_on_multiloop_algebras}
            The action of $\divzero$ on $\g_{[2]}$ as in lemma \ref{lemma: derivation_action_on_multiloop_algebras} satisfies:
                $$[\divzero, \g_{[2]}]_{\extendedtoroidal} \subseteq \g_{[2]}$$
            Explicitly, the basis elements $D_{r, s}, D_v, D_t \in \divzero$ thus act on generators of $\g_{[2]}$ (i.e. monomials of the form $x v^m t^p$ for some $x \in \g$ and some $(m, p) \in \Z^2$) in the following manner:
                $$[D_{r, s}, x v^m t^p]_{\extendedtoroidal} = (rp - ms) x v^{m - r} t^{p - s - 1}$$
                $$[D_v, x v^m t^p]_{\extendedtoroidal} = -m x v^m t^{p - 1}$$
                $$[D_t, x v^m t^p]_{\extendedtoroidal} = -p x v^m t^{p - 1}$$
        \end{corollary}
            \begin{proof}
                From lemma \ref{lemma: derivation_action_on_multiloop_algebras}, we know that given some $D \in \divzero$ and some $x \in \g$ and $f \in A$, there shall exist $K(D, xf) \in \z_{[2]}$ (depending on the choices of $D$ and $x, f$) such that:
                    $$[D, xf]_{\extendedtoroidal} = x D(f) + K(D, xf)$$
                Next, consider the following:
                    $$( \divzero, x D(f) + K(D, xf) )_{\extendedtoroidal} = ( \divzero, [D, xf]_{\extendedtoroidal} )_{\extendedtoroidal} = ( [\divzero, D]_{\extendedtoroidal}, xf )_{\extendedtoroidal} = 0$$
                where the second equality holds thanks to invariance, and the third equality holds due to a combination of the fact that $[\divzero, \divzero]_{\extendedtoroidal} \subset \z_{[2]} \oplus \divzero$ (cf. proposition \ref{prop: lie_bracket_on_orthogonal_complement_of_toroidal_centre}) and the fact that $(\z_{[2]} \oplus \divzero, \g_{[2]})_{\extendedtoroidal}$ per the construction of the bilinear form $(-, -)_{\extendedtoroidal}$ (cf. corollary \ref{coro: pairing_yangian_div_zero_vector_fields_and_cyclic_1_forms}). We also have the following, again per the construction of the bilinear form $(-, -)_{\extendedtoroidal}$:
                    $$( \divzero, x D(f) + K(D, xf) )_{\extendedtoroidal} = ( \divzero, K(D, xf) )_{\extendedtoroidal}$$
                which means that:
                    $$( \divzero, K(D, xf) )_{\extendedtoroidal} = 0$$
                The non-degeneracy of $(-, -)_{\extendedtoroidal}$ (or more particularly, the fact that $( \divzero, \z_{[2]} )_{\extendedtoroidal} \not = 0$) then implies that:
                    $$K(D, xf) = 0$$
                necessarily. This means that, indeed, we have that:
                    $$[D, xf]_{\extendedtoroidal} = x D(f)$$
                for all $D \in \divzero$ and all $x \in \g$ and all $f \in A$. Since $\g_{[2]}$ is generated by elements of the form $xf$, this implies that:
                    $$[\divzero, \g_{[2]}]_{\extendedtoroidal} \subseteq \g_{[2]}$$
                as claimed. 
            \end{proof}
        
        In summary, we have yielded the following result, which amounts to one half of theorem \ref{theorem: yangian_extended_toroidal_lie_algebras_preliminary_version} stated at the beginning of this chapter.
        \todo[inline]{First half of main theorem}
        \begin{theorem}[Yangian extended toroidal Lie algebras are twisted semi-direct products] \label{theorem: yangian_extended_toroidal_lie_algebras}
            If $\fraky$ is a Yangian extended toroidal Lie algebra in the sense of definition \ref{def: yangian_extended_toroidal_lie_algebras}, then there will exist some Lie $2$-cocycle $\sigma \in Z^2_{\Lie}(\divzero, \z_{[2]})$ so that:
                $$\fraky \cong \toroidal \rtimes^{\sigma} \divzero$$
        \end{theorem}

    \subsection{Centres of Yangian extended toroidal Lie algebras}
        Let us conclude this section with the following question, which is natural now that we have a solid handle on how the Lie bracket on $\extendedtoroidal$ is given:
        \begin{question}
            What is the centre $\hat{\z}_{[2]} := \z( \extendedtoroidal )$ ? This ought to be smaller than $\z_{[2]}$ somehow, since elements of $\z_{[2]}$ need not be central in $\extendedtoroidal$. 
        \end{question}
        \begin{remark}[Computing the centre without computing all the brackets ...]
            Since $\g_{[2]}$ is centreless, we have that:
                $$\hat{\z}_{[2]} = \z( \z_{[2]} \oplus \divzero )$$
            As $\z_{[2]}$ is an abelian Lie algebra, this implies that in order to compute $\hat{\z}_{[2]}$, it suffices to explicitly compute the commutators of the form:
                $$[D, K]_{\extendedtoroidal}, [D, D']_{\extendedtoroidal}$$
            for $D, D' \in \divzero$ and $K \in \z_{[2]}$, to see which ones vanish. However, this is rather tedious and not very insightful.
            
            An alternative method is as follows: exploiting the fact that the symmetric bilinear form $(-, -)_{\extendedtoroidal}$ is both invariant and non-degenerate, we can characterise the centre $\hat{\z}_{[2]}$ as the Lie ideal of $\extendedtoroidal$ containing elements $Z$ such that:
                $$0 = ([Z, X]_{\extendedtoroidal}, Y)_{\extendedtoroidal} = (Z, [X, Y]_{\extendedtoroidal})_{\extendedtoroidal}$$
            for any $X, Y \in \extendedtoroidal$, with the first equality holding thanks to the fact that $Z$ is supposed to commute with every other element of $\extendedtoroidal$ by assumption of being central. We are thus left with the task of finding elements:
                $$Z \in \extendedtoroidal$$
            such that:
                $$(Z, [\extendedtoroidal, \extendedtoroidal]_{\extendedtoroidal})_{\extendedtoroidal} = 0$$
            Since brackets of the form:
                $$[X, Y]_{\extendedtoroidal}, [D, D']_{\extendedtoroidal}$$
            (for some $X, Y \in \g_{[2]}$ and some $D, D' \in \divzero$) are generally non-zero, their elements can not be central in $\extendedtoroidal$. As such, we have narrowed the scope of our search down to:
                $$\hat{\z}_{[2]} \subset \z_{[2]}$$

            Another way to see that:
                $$\hat{\z}_{[2]} \subset \z_{[2]}$$
            is to use the fact that $\extendedtoroidal$ is a Lie algebra extension of $\divzero$ by $\toroidal$ (cf. theorem \ref{theorem: yangian_extended_toroidal_lie_algebras}). This tells us that the centre of $\extendedtoroidal$ ought to lie inside that of $\toroidal$, i.e.:
                $$\hat{\z}_{[2]} \subset \z(\toroidal) = \z_{[2]}$$
            as per proposition \ref{prop: twisted_semi_direct_product_criterion}.
        \end{remark}
        \begin{proposition}[Centres of Yangian extended toroidal Lie algebras] \label{prop: centres_of_yangian_extended_toroidal_lie_algebras}
            The centre $\hat{\z}_{[2]}$ is a two-dimensional (abelian) Lie subalgebra of $\z_{[2]}$, spanned by $c_v$ and $c_t$. 
        \end{proposition}
            \begin{proof}
                Since we know that:
                    $$\hat{\z}_{[2]} \subset \z_{[2]}$$
                and that the only possibly non-zero bracket with elements of $\z_{[2]}$ are elements of $[\divzero, \z_{[2]}]_{\extendedtoroidal}$, and since we also know from lemma \ref{lemma: explicit_commutators_between_central_basis_elements_and_derivations} that:
                    $$[\divzero, K]_{\extendedtoroidal} = 0 \iff K \in \bbC c_v \oplus \bbC c_v$$
                we can conclude immediately that:
                    $$\hat{\z}_{[2]} = \bbC c_v \oplus \bbC c_t$$
            \end{proof}
        \begin{remark}
            It is rather interesting that:
                $$\hat{\z}_{[2]} \cong \bbC c_v \oplus \bbC c_t$$
            as this is in good analogy with the affine Kac-Moody case, where the centre of $\hat{\g}$ is $1$-dimensional, namely spanned by $c_v$ (cf. example \ref{example: affine_lie_algebras_centres}).
        \end{remark}