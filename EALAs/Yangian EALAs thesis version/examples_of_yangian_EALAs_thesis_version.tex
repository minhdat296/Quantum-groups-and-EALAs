\section{Examples of \texorpdfstring{$\gamma$}{}-extended toroidal Lie algebras}
    \subsection{Explicit (counter-)examples of $\gamma$-invariant toroidal cocycles}
        For convenience, let us fix the following terminologies (which have already been eluded to in the statement of theorem \ref{theorem: yangian_extended_toroidal_lie_algebras_main_theorem}).
        \begin{definition}[$\gamma$-invariant toroidal $2$-cocycles] \label{def: yangian_toroidal_cocycles}
            Any $2$-cocyle $\sigma \in Z^2_{\Lie}(\divzero, \toroidal)$ shall be referred to as a \textbf{toroidal $2$-cocycle}.
            
            Any toroidal $2$-cocycle $\sigma$ such that $\toroidal \rtimes^{\sigma} \divzero$ is a $\gamma$-extended toroidal Lie algebra shall be called a \textbf{$\gamma$-invariant toroidal $2$-cocycle}.
        \end{definition}
        \begin{remark}
            Because Lie $2$-cocycles are central (cf. corollary \ref{coro: 2_cocycles_are_central}), toroidal $2$-cocycles are the same as elements of $Z^2_{\Lie}( \divzero, \z(\toroidal) )$.
        \end{remark}
        
        In theorem \ref{theorem: yangian_extended_toroidal_lie_algebras_main_theorem} (see also proposition \ref{prop: yangian_extended_toroidal_lie_algebras_are_twisted_semi_direct_products}) we have provided a criterion for a given toroidal $2$-cocycle to be $\gamma$-invariant in the sense of definition \ref{def: yangian_toroidal_cocycles} above, namely that:
            $$( \sigma(D, D'), D'' )_{\extendedtoroidal} = ( D, \sigma(D', D'') )_{\extendedtoroidal}$$
        for all $D, D', D'' \in \divzero$. In this subsection, let us apply it to some known toroidal $2$-cocycles from \cite{billig_energy_momentum_tensor} to check whether or not they are $\gamma$-invariant.

        Let us begin by introducing the two toroidal $2$-cocycles we will be working with.
        \begin{example} \label{example: billig_toroidal_cocycles}
            In \cite[p. 5, below Equation 1.3]{billig_energy_momentum_tensor}, it was noted that there are at least $2$-cocyles:
                $$\sigma_1, \sigma_2 \in Z^2_{\Lie}( \der(A), \z(\toroidal) )$$
            These are given in terms of the following basis of $\der(A)$\footnote{Note how we are identifying $\der(A) \cong A v\del_v \oplus A t\del_t$ as opposed to the usual identification $\der(A) \cong A \del_v \oplus A \del_t$.}:
                $$\{ v^{r_v} t^{r_t} \cdot v \del_v \}_{(r_v, r_t) \in \Z^2} \cup \{ v^{m_v} t^{m_t} \cdot t \del_t \}_{(m_v, m_t) \in \Z^2}$$
            by the following formulae:
                $$\sigma_1( v^{r_v} t^{r_t} \cdot x \del_x, v^{m_v} t^{m_t} \cdot y \del_y ) := r_y m_x \cdot v^{r_v} t^{r_t} \bar{d}( v^{m_v} t^{m_t} )$$
                $$\sigma_2( v^{r_v} t^{r_t} \cdot x \del_x, v^{m_v} t^{m_t} \cdot y \del_y ) := r_x m_y \cdot v^{r_v} t^{r_t} \bar{d}( v^{m_v} t^{m_t} )$$
            where $x, y \in \{v, t\}$ are symbolic placeholders.
            
            As a quick aside, let us note that the first cocycle $\sigma_1$ was known as far back as \cite{moody_rao_yokonuma_vertex_representations_of_toroidal_lie_algebras}, but we are not aware of the history of $\sigma_2$ beyond its appearance in \cite{billig_energy_momentum_tensor}.
        \end{example}

        We would like to give descriptions of $\sigma_1, \sigma_2$ as elements of $Z^2_{\Lie}(\divzero, \z(\toroidal))$ instead of elements of $Z^2_{\Lie}(\der(A), \z(\toroidal))$, which is the same as computing the domain restrictions of these two cocycles from $\bigwedge^2 \der(A)$ down to the vector subspace $\bigwedge^2 \divzero$, which can be done by computing the values of $\sigma_1, \sigma_2$ on pairs of basis elements of $\divzero$. Since lemma \ref{lemma: yangian_div_zero_vector_fields_basic_properties}, we have known how the elements of the \say{standard} basis:
            $$\{D_{r, s}\}_{(r, s) \in \Z^2} \cup \{D_v, D_t\}$$
        of $\divzero$ are given in terms of the partial derivatives $\del_v, \del_t$ (or more accurately, in terms of the aforementioned basis of $\der(A)$) by:
            $$D_{r, s} = -s v^{-r + 1} t^{-s - 1} \del_v + r v^{-r} t^{-s} \del_t = -s v^{-r} t^{-s - 1} \cdot v\del_v + r v^{-r} t^{-s - 1} \cdot t\del_t$$
            $$D_v = -v t^{-1} \del_v = -t^{-1} \cdot v\del_v$$
            $$D_t = -\del_t = -t^{-1} \cdot t\del_t$$
        (to rewrite the expressions slightly in terms of the currently employed basis for $\der(A)$). Knowing this allows us to compute the values of $\sigma_1$ and $\sigma_2$ on pairs of these basis elements.
        \begin{lemma}[Values of $\sigma_1$ and $\sigma_2$ on pairs of basis elements of $\divzero$] \label{lemma: billig_toroidal_cocycles_on_yangian_div_zero_vector_fields}
            In what follows, let $i \in \{1, 2\}$. The values of the toroidal $2$-cocycles $\sigma_1, \sigma_2$ from example \ref{example: billig_toroidal_cocycles} on pairs of elements of the basis $\{D_{r, s}\}_{(r, s) \in \Z^2} \cup \{D_v, D_t\}$ of $\divzero$ (cf. lemma \ref{lemma: yangian_div_zero_vector_fields_basic_properties}) are:
                $$\sigma_i(D_{r, s}, D_{a, b}) = N_i(r, s, a, b) \left( ( r(b + 1) - a(s + 1) )K_{-r - a, -s - b - 2} - \delta_{ (-a - r, -b - s - 2), (0, 0) } (r c_v + (s + 1) c_t) \right)$$
            where:
                $$
                    N_i(r, s, a, b) =
                    \begin{cases}
                        \text{$2 rsab - ( (sa)^2 + s a^2 ) - ( (rb)^2 + r^2 b ) + rsa + rab + ra$ if $i = 1$}
                        \\
                        \text{$ra$ if $i = 2$}
                    \end{cases}
                $$
            and:
                $$\sigma_i(D_v, D_{r, s}) = \delta_{i, 1} r^3 K_{-r, -s - 2}$$
                $$\sigma_i(D_t, D_{r, s}) = \delta_{i, 1} r^2 s K_{-r, -s - 2}$$
                $$\sigma_i(D_v, D_t) = 0$$
        \end{lemma}
            \begin{proof}
                \begin{enumerate}
                    \item Firstly, let us compute $\sigma_i(D_{r, s}, D_{a, b})$. For this, consider the following:
                        $$
                            \begin{aligned}
                                & \sigma_i(D_{r, s}, D_{a, b})
                                \\
                                = & \sigma_i( -s v^{-r + 1} t^{-s - 1} \del_v + r v^{-r} t^{-s} \del_t, -b v^{-a + 1} t^{-b - 1} \del_v + a v^{-a} t^{-b} \del_t )
                                \\
                                = & \sigma_i( s v^{-r} t^{-s - 1} \cdot v\del_v - r v^{-r} t^{-s - 1} \cdot t \del_t, b v^{-a} t^{-b - 1} \cdot v\del_v - a v^{-a} t^{-b - 1} \cdot t \del_t )
                                \\
                                = & s \sigma_i( v^{-r} t^{-s - 1} \cdot v\del_v, b v^{-a} t^{-b - 1} \cdot v\del_v - a v^{-a} t^{-b - 1} \cdot t \del_t ) - r \sigma_i( v^{-r} t^{-s - 1} \cdot t \del_t, b v^{-a} t^{-b - 1} \cdot v\del_v - a v^{-a} t^{-b - 1} \cdot t \del_t )
                                \\
                                = &
                                \begin{aligned}
                                    & s b \cdot \sigma_i( v^{-r} t^{-s - 1} \cdot v\del_v, v^{-a} t^{-b - 1} \cdot v\del_v )
                                    \\
                                    - & s a \cdot \sigma_i( v^{-r} t^{-s - 1} \cdot v\del_v, v^{-a} t^{-b - 1} \cdot t \del_t )
                                    \\
                                    - & r b \cdot \sigma_i( v^{-r} t^{-s - 1} \cdot t \del_t, v^{-a} t^{-b - 1} \cdot v\del_v )
                                    \\
                                    + & r a \cdot \sigma_i( v^{-r} t^{-s - 1} \cdot t \del_t, v^{-a} t^{-b - 1} \cdot t \del_t )
                                \end{aligned}
                                \\
                                = & N_i(r, s, a, b) v^{-r} t^{-s - 1} \bar{d}( v^{-a} t^{-b - 1} )
                            \end{aligned}
                        $$
                    where:
                        $$
                            \begin{aligned}
                                N_i(r, s, a, b) & = 
                                sbra
                                - sa \left( \delta_{i, 1} a(s + 1) + \delta_{i, 2} (b + 1) r \right) 
                                - rb \left( \delta_{i, 1} (b + 1) r + \delta_{i, 2} a (s + 1) \right)
                                + r a (s + 1) (b + 1)
                                \\
                                & = 
                                \begin{cases}
                                    \text{$
                                        sbra
                                        - s a^2 (s + 1) 
                                        - r^2 b (b + 1)
                                        + r a (s + 1) (b + 1)
                                    $if $i = 1$}
                                    \\
                                    \text{$
                                        sbra
                                        - sa (b + 1) r
                                        - rb a (s + 1)
                                        + r a (s + 1) (b + 1)
                                    $ if $i = 2$}
                                \end{cases}
                                \\
                                & = 
                                \begin{cases}
                                    \text{$
                                        sbra
                                        - ( (sa)^2 + s a^2 ) 
                                        - ( (rb)^2 + r^2 b ) 
                                        + rasb + rsa + rab + ra
                                    $ if $i = 1$}
                                    \\
                                    \text{$
                                        sbra
                                        - (sabr + sar)
                                        - (rbas + rba)
                                        + rasb + rsa + rab + ra
                                    $ if $i = 2$}
                                \end{cases}
                                \\
                                & = 
                                \begin{cases}
                                    \text{$2 rsab - ( (sa)^2 + s a^2 ) - ( (rb)^2 + r^2 b ) + rsa + rab + ra$ if $i = 1$}
                                    \\
                                    \text{$ra$ if $i = 2$}
                                \end{cases}
                            \end{aligned}
                        $$
                    
                    Now, recall from example \ref{example: toroidal_lie_algebras_centres} that any element:
                        $$v^n t^q \bar{d}(v^m t^p) \in \z(\toroidal)$$
                    can be written in terms of the basis elements of $\z(\toroidal)$ in the following manner:
                        $$v^n t^q \bar{d}(v^m t^p) = \delta_{(m, p) + (n, q), (0, 0)} ( n c_v + q c_t ) + (np - mq) K_{m + n, p + q}$$
                    Using this, we shall be able to conclude that:
                        $$\sigma_i(D_{r, s}, D_{a, b}) = N_i(r, s, a, b) \left( -\delta_{ (-a - r, -b - s - 2), (0, 0) } (r c_v + (s + 1) c_t) + ( r(b + 1) - a(s + 1) )K_{-r - a, -s - b - 2} \right)$$
                    \item Secondly, we shall be computing $\sigma_i(D_v, D_{r, s})$. Consider the following:
                        $$
                            \begin{aligned}
                                & \sigma_i(D_{r, s}, D_v)
                                \\
                                = & \sigma_i( -s v^{-r + 1} t^{-s - 1} \del_v + r v^{-r} t^{-s} \del_t, -v t^{-1} \del_v )
                                \\
                                = & \sigma_i( s v^{-r} t^{-s - 1} \cdot v \del_v - r v^{-r} t^{-s - 1} t \del_t, t^{-1} \cdot v \del_v )
                                \\
                                = & s \sigma_i( v^{-r} t^{-s - 1} \cdot v \del_v, t^{-1} \cdot v \del_v ) - r\sigma_i( v^{-r} t^{-s - 1} \cdot t \del_t, t^{-1} \cdot v \del_v )
                                \\
                                = & \left( s \left( \delta_{i, 1} (-r) \cdot 0 + \delta_{i, 2} (-r) \cdot 0 \right) - r\left( \delta_{i, 1} (-r) \cdot (-1) + \delta_{i, 2} (-s - 1) \cdot 0 \right) \right) v^{-r} t^{-s - 1} \bar{d}(t^{-1}) 
                                \\
                                = & -\delta_{i, 1} r^2 v^{-r} t^{-s - 1} \bar{d}(t^{-1})
                            \end{aligned}
                        $$
                    Now, recall from example \ref{example: toroidal_lie_algebras_centres} that any element:
                        $$v^n t^q \bar{d}(v^m t^p) \in \z(\toroidal)$$
                    can be written in terms of the basis elements of $\z(\toroidal)$ in the following manner:
                        $$v^n t^q \bar{d}(v^m t^p) = \delta_{(m, p) + (n, q), (0, 0)} ( n c_v + q c_t ) + (np - mq) K_{m + n, p + q}$$
                    Using this, we shall get that:
                        $$
                            \begin{aligned}
                                & \sigma_i(D_{r, s}, D_v)
                                \\
                                = & -r^2 \left( -\delta_{(r, s), (0, -2)} ( r c_v + (s + 3) c_t ) - r K_{-r, -s - 2} \right)
                                \\
                                = &
                                \begin{cases}
                                    \text{$0$ if $(r, s) \in \{0\} \x \Z$}
                                    \\
                                    \text{$r^3 K_{-r, -s - 2}$ if $(r, s) \in (\Z \setminus \{0\}) \x \Z$}
                                \end{cases}
                                \\
                                = & \delta_{i, 1} r^3 K_{-r, -s - 2}
                            \end{aligned}
                        $$
                    \item Using similar methods as in the previous case, we shall get that:
                        $$\sigma_i(D_{r, s}, D_t) = r^2s K_{-r, -s - 2}$$
                    \item Finally, we have that:
                        $$
                            \begin{aligned}
                                & \sigma_i(D_v, D_t)
                                \\
                                = & \sigma_i(-v t^{-1} \del_v, -\del_t)
                                \\
                                = & \sigma_i(t^{-1} \cdot v \del_v, t^{-1} t \del_t)
                                \\
                                = & 0
                            \end{aligned}
                        $$
                \end{enumerate}
            \end{proof}

        One can now use the criterion given in proposition \ref{proposition: twisted_semi_direct_products_are_yangian_extended_toroidal_lie_algebras} (see also theorem \ref{theorem: yangian_extended_toroidal_lie_algebras_main_theorem}) to verify whether or not the cocycles $\sigma_1, \sigma_2$ are $\gamma$-invariant in the sense of definition \ref{def: yangian_toroidal_cocycles}.
        \todo[inline]{I decided to dedicated a separate proposition for discussing whether or not $\sigma_1, \sigma_2$ are $\gamma$-invariant, as the proofs of this proposition and of proposition \ref{prop: cohomological_non_triviality_of_billig_toroidal_cocycles} are both quite long.}
        \begin{proposition}[$\gamma$-invariance of Billig's toroidal $2$-cocycles] \label{prop: invariance_of_billig_toroidal_cocycles}
            Of the two toroidal $2$-cocycles:
                $$\sigma_1, \sigma_2 \in Z^2_{\Lie}(\divzero, \z(\toroidal))$$
            discussed in example \ref{example: billig_toroidal_cocycles}, $\sigma_1$ is $\gamma$-invariant in the sense of definition \ref{def: yangian_toroidal_cocycles}, while $\sigma_2$ fails to be so.
        \end{proposition}
            \begin{proof}
                \begin{enumerate}
                    \item Firstly, using the fact that:
                        $$\sigma_i(D_{r, s}, D_{a, b}) = N_i(r, s, a, b) \left( -\delta_{ (-a - r, -b - s - 2), (0, 0) } (r c_v + (s + 1) c_t) + ( r(b + 1) - a(s + 1) )K_{-r - a, -s - b - 2} \right)$$
                    we shall get that:
                        $$
                            \left( \sigma_i(D_{r, s}, D_{a, b}), D \right)_{\extendedtoroidal} =
                            \begin{cases}
                                \text{$N_i(r, s, a, b) ( r(b + 1) - a(s + 1) ) \delta_{(-r - a, -s - b - 2), (\alpha, \beta)}$ if $D = D_{\alpha, \beta}$}
                                \\
                                \text{$-N_i(r, s, a, b) \delta_{(r, s), -(a, b)} r$ if $D = D_v$}
                                \\
                                \text{$-N_i(r, s, a, b) \delta_{(r, s), -(a, b)} (s + 1)$ if $D = D_t$}
                            \end{cases}
                        $$
                    At the same time, using the fact that:
                        $$\sigma_i(D_{a, b}, D_v) = -\delta_{i, 1} a^3 K_{-a, -b - 2}$$
                        $$\sigma_i(D_{a, b}, D_t) = - \delta_{i, 1} a^2b K_{-a, -b - 2}$$
                    we have that:
                        $$
                            \begin{aligned}
                                \left( D_{r, s}, \sigma_i(D_{a, b}, D) \right)_{\extendedtoroidal} =
                                \begin{cases}
                                    \text{$N_i(r, s, \alpha, \beta) ( r(\beta + 1) - \alpha(s + 1) ) \delta_{(r, s), (-a - \alpha, -b - \beta - 2)}$ if $D = D_{\alpha, \beta}$}
                                    \\
                                    \text{$-\delta_{i, 1} a^3 \delta_{(r, s), (-a, -b - 2)}$ if $D = D_v$}
                                    \\
                                    \text{$-\delta_{i, 1} a^2 b \delta_{(r, s), (-a, -b - 2)}$ if $D = D_t$}
                                \end{cases}
                            \end{aligned}
                        $$
                    We can thus conclude immediately that $\sigma_2$ is \textit{not} invariant, as:
                        $$\left( \sigma_2(D_{r, s}, D_{a, b}), D \right)_{\extendedtoroidal} \not = \left( D_{r, s}, \sigma_2(D_{a, b}, D) \right)_{\extendedtoroidal}$$
                    when $D \in \{D_v, D_t\}$. As such, let us focus on $\sigma_1$ from now on, for which we now have:
                        $$\left( \sigma_1(D_{r, s}, D_{a, b}), D \right)_{\extendedtoroidal} \not = \left( D_{r, s}, \sigma_1(D_{a, b}, D) \right)_{\extendedtoroidal}$$
                    for all $D \in \divzero$.
                    \item Secondly, using the fact that:
                        $$\sigma_1(D_{r, s}, D_v) = r^3 K_{-r, -s - 2}$$
                    we shall get that:
                        $$
                            \left( \sigma_1(D_{r, s}, D_v), D \right)_{\extendedtoroidal} =
                            \begin{cases}
                                \text{$r^3 \delta_{(-r, -s - 2), (\alpha, \beta)}$ if $D = D_{\alpha, \beta}$}
                                \\
                                \text{$0$ if $D = D_v$}
                                \\
                                \text{$0$ if $D = D_t$}
                            \end{cases}
                        $$
                    At the same time, knowing that:
                        $$\sigma_1(D_v, D_t) = 0$$
                    we see that:
                        $$
                            \begin{aligned}
                                \left( D_{r, s}, \sigma_1(D_v, D) \right)_{\extendedtoroidal} =
                                \begin{cases}
                                    \text{$-\alpha^3 \delta_{(r, s), (-\alpha, -\beta - 2)}$ if $D = D_{\alpha, \beta}$}
                                    \\
                                    \text{$0$ if $D = D_v$}
                                    \\
                                    \text{$0$ if $D = D_t$}
                                \end{cases}
                            \end{aligned}
                        $$
                    We thus have:
                        $$\left( \sigma_1(D_{r, s}, D_v), D \right)_{\extendedtoroidal} = \left( D_{r, s}, \sigma_1(D_v, D) \right)_{\extendedtoroidal}$$
                    for all $D \in \divzero$.
                    \item Next, by using the fact that:
                        $$\sigma_1(D_{r, s}, D_t) = r^2 s K_{-r, -s - 2}$$
                    we shall get that:
                        $$
                            \left( \sigma_1(D_{r, s}, D_t), D \right)_{\extendedtoroidal} =
                            \begin{cases}
                                \text{$r^2 s \delta_{(-r, -s - 2), (\alpha, \beta)}$ if $D = D_{\alpha, \beta}$}
                                \\
                                \text{$0$ if $D = D_v$}
                                \\
                                \text{$0$ if $D = D_t$}
                            \end{cases}
                        $$
                    At the same time, we have that:
                        $$
                            \begin{aligned}
                                \left( D_{r, s}, \sigma_1(D_t, D) \right)_{\extendedtoroidal} =
                                \begin{cases}
                                    \text{$-\alpha^2 \beta \delta_{(r, s), (-\alpha, -\beta - 2)}$ if $D = D_{\alpha, \beta}$}
                                    \\
                                    \text{$0$ if $D = D_v$}
                                    \\
                                    \text{$0$ if $D = D_t$}
                                \end{cases}
                            \end{aligned}
                        $$
                    By combining these two observations, one is able to conclude furthermore that:
                        $$\left( \sigma_1(D_{r, s}, D_t), D \right)_{\extendedtoroidal} = \left( D_{r, s}, \sigma_1(D_t, D) \right)_{\extendedtoroidal}$$
                    for all $D \in \divzero$.
                    \item Lastly, since:
                        $$\sigma_1(D_v, D_t) = 0$$
                    we automatically have that:
                        $$( \sigma_1(D_v, D_t), D )_{\extendedtoroidal} = ( D_v, \sigma_1(D_t, D) )_{\extendedtoroidal}$$
                    for all $D \in \divzero$.
                \end{enumerate}
                We have therefore shown that $\sigma_1$ is $\gamma$-invariant in the sense of definition \ref{def: yangian_toroidal_cocycles}. 
            \end{proof}

    \subsection{Cohomological (non-)triviality}
        Whether or not the cocycles:
            $$\sigma_1, \sigma_2 \in Z^2_{\Lie}(\divzero, \z(\toroidal))$$
        might be cohomologous to $0$ (cf. definition \ref{def: lie_algebra_cohomology}) - and hence whether or not they might give rise to extensions that are isomorphic to the semi-direct product $\toroidal \rtimes \divzero$ - is a much subtler issue. One way to tackle this problem is to \textit{firstly} check whether or not their restrictions to a particular Lie subalgebra of $\extendedtoroidal$ is cohomologous to $0$. We remark right away that simply checking that these restrictions are non-cohomologous to $0$ is \textit{not} sufficient for concluding that $\sigma_1$ and $\sigma_2$ are non-zero elements of $H^2_{\Lie}(\divzero, \z(\toroidal))$.

        We begin our analysis by recalling that the Witt algebra:
            $$\der(\bbC[v^{\pm 1}]) \cong \bigoplus_{r \in \Z} \bbC d_r$$
        (where $d_r := -v^r \cdot v\frac{d}{dv}$) is known to possess a \textit{non-trivial} UCE:
            $$\frakv := \der(\bbC[v^{\pm 1}]) \oplus^{\eta} \bbC c_{\frakv}$$
        called the \textbf{Virasoro algebra} (cf. \cite[Sections 9.13 and 9.14]{kac_infinite_dimensional_lie_algebras} and \cite[Section 1.3]{kac_raina_rozhkovskaya_bombay_lectures_on_highest_weight_modules_of_infinite_dimensional_lie_algebras}), whose corresponding $2$-cocycle $\eta \in Z^2_{\Lie}( \der(\bbC[v^{\pm 1}]), \bbC c_{\frakv} )$ is given by:
            $$\eta(d_r, d_a) := \delta_{r + a, 0} (r^3 - r) c_{\frakv}$$
        for all $r, a \in \Z$; in particular, this means this $\eta$ is \textit{non-cohomologous} to $0$. Moreover, the Virasoro algebra is the only non-trivial central extension of the Witt algebra.
        \begin{lemma}[$H^2_{\Lie}$ of the Witt algebra] \label{lemma: H^2_of_witt_algebra}
            (Cf. \cite[Proposition 1.3]{kac_raina_rozhkovskaya_bombay_lectures_on_highest_weight_modules_of_infinite_dimensional_lie_algebras}) Let $\bbC c_{\frakv}$ be viewed as a trivial $\der(\bbC[v^{\pm 1}])$-module. Then:
                $$\dim_{\bbC} H^2_{\Lie}( \der(\bbC[v^{\pm 1}]), \bbC c_{\frakv} ) = 1$$
        \end{lemma}

        Since we know now that there is an isomorphism of Lie algebras $\der(\bbC[v^{\pm 1}]) \xrightarrow[]{\cong} \frakw$ given by $d_r \mapsto D_{r, -1}$ (cf. lemma \ref{lemma: a_copy_of_the_witt_algebra_inside_the_lie_algebra_of_yangian_div_zero_vector_fields}), it makes sense to regard $\sigma_1$ and $\sigma_2$ as $2$-cocycles of $\der(\bbC[v^{\pm 1}])$ (with either values in $\bbC c_{\frakv}$ or all of $\z(\toroidal)$) after restricting their domains to $\bigwedge^2 \frakw$. We shall now attempt to show, using lemma \ref{lemma: H^2_of_witt_algebra}, that because:
            $$\sigma_1|_{\bigwedge^2 \frakw}(D_{r, -1}, D_{a, -1}) = \sigma_2|_{\bigwedge^2 \frakw}(D_{r, -1}, D_{a, -1}) = \delta_{r + a, 0} r^3 c_v$$
        they must be non-zero as elements of $H^2_{\Lie}(\frakw, \bbC c_v)$. This will be the content of lemma \ref{lemma: restrictions_of_billig_toroidal_cocycles_are_coboundary}.

        We begin with the following lemma:
        \begin{lemma}[A Virasoro $2$-coboundary] \label{lemma: a_virasoro_coboundary}
            Let:
                $$\eta': \bigwedge^2 \der(\bbC[v^{\pm 1}]) \to \bbC c_{\frakv}$$
            be the function given by:
                $$\eta'(d_r, d_a) := \delta_{r + a, 0} r c_{\frakv}$$
            This is a $2$-cocycle of $\der(\bbC[v^{\pm 1}])$ with values in $\bbC c_{\frakv}$. From this, one sees that:
                $$\eta + \eta': \bigwedge^2 \der(\bbC[v^{\pm 1}]) \to \bbC c_{\frakv}$$
            (which is given by $(\eta + \eta')(d_r, d_a) := \delta_{r + a, 0} r^3 c_{\frakv}$) is also a $2$-cocycle of $\der(\bbC[v^{\pm 1}])$ with values in $\bbC c_{\frakv}$. 

            Furthermore, we have that:
                $$\eta' \in B^2_{\Lie}( \der(\bbC[v^{\pm 1}]), \bbC c_{\frakv} )$$
            with notations as in definition \ref{def: lie_cocycles_and_coboundaries}.
        \end{lemma}
            \begin{proof}
                It is clear from the construction of $\eta'$ that it is linear and skew-symmetric; the only non-trivial thing to prove is that $\eta'$ satisfies the Jacobi identity in the sense of definition \ref{def: twisted_semi_direct_products}. To this end, simply consider the following, for all $i, j, k \in \Z$:
                    $$
                        \begin{aligned}
                            & \eta'([d_i, d_j], d_k) + \eta'([d_k, d_i], d_j) + \eta'([d_j, d_k], d_i)
                            \\
                            = & (i - j) \eta'(d_{i + j}, d_k) + (k - i) \eta'(d_{k + i}, d_j) + (j - k) \eta'(d_{j + k}, d_i)
                            \\
                            = & \delta_{i + j + k, 0} \left( (i - j) (i + j) + (k - i) (k + i) + (j - k) (j + k) \right) c_{\frakv}
                            \\
                            = & 0
                        \end{aligned}
                    $$
                    
                To show that $\eta'$ is a Lie $2$-coboundary in the sense of definition \ref{def: lie_cocycles_and_coboundaries}, we must show that there exists a linear map:
                    $$\tilde{\eta'}: \der(\bbC[v^{\pm 1}]) \to \bbC c_{\frakv}$$
                which is merely a linear map, such that:
                    $$\eta(d_i, d_j) = \tilde{\eta'}([d_i, d_j])$$
                (cf. examples \ref{example: lie_cocycles_and_coboundaries_with_trivial_coefficients} and \ref{example: low_degree_lie_cocycles_and_coboundaries_with_trivial_coefficients}). The RHS is nothing but:
                    $$\tilde{\eta'}([d_i, d_j]) = (i - j) \tilde{\eta'}(d_{i + j}) c_{\frakv}$$
                while by construction, the LHS is:
                    $$\eta(d_i, d_j) := \delta_{i + j, 0} i c_{\frakv}$$
                and hence:
                    $$\delta_{i + j, 0} i = (i - j) \tilde{\eta'}(d_{i + j})$$
                By setting $j = 0$, we then see that:
                    $$i (\tilde{\eta'}(d_i) - \delta_{i, 0}) = 0$$
                for all $i \in \Z$, which in turn implies that:
                    $$\tilde{\eta'}(d_i) = \delta_{i, 0}$$
                The sought-for linear map $\tilde{\eta}: \der(\bbC[v^{\pm 1}]) \to \bbC c_{\frakv}$ is thus defined, and hence exists.
            \end{proof}
        \begin{lemma}[Restricting $\sigma_1$ and $\sigma_2$ to the Witt algebra] \label{lemma: billig_toroidal_cocycles_on_the_witt_algebra}
            Let $\eta' \in Z^2_{\Lie}(\der(\bbC[v^{\pm 1}]), \bbC c_{\frakv})$ be as in lemma \ref{lemma: a_virasoro_coboundary}. The Lie $2$-cocycle:
                $$\eta + \eta' \in Z^2_{\Lie}(\der(\bbC[v^{\pm 1}]), \bbC c_{\frakv})$$
            given by:
                $$(\eta + \eta')(d_r, d_a) = \delta_{r + a, 0} r^3 c_{\frakv}$$
            for all $r, a \in \Z$, is not cohomologous to $0$, i.e. its image under the canonical projection:
                $$Z^2(\der(\bbC[v^{\pm 1}]), \bbC c_{\frakv}) \to H^2(\der(\bbC[v^{\pm 1}]), \bbC c_{\frakv})$$
            (cf. definition \ref{def: complexes_and_cohomology}) is non-zero.

            Furthermore, we have that:
                $$\sigma_i|_{\bigwedge^2 \frakw} = \eta + \eta'$$
            and consequently, the equivalence classes of domain restrictions $\sigma_i|_{\bigwedge^2 \frakw}$ (where $i \in \{1, 2\}$) in $H^2_{\Lie}( \frakw, \bbC c_v )$ are also non-zero.
        \end{lemma}
            \begin{proof}
                Because we know that $\dim_{\bbC} H^2(\der(\bbC[v^{\pm 1}]), \bbC c_{\frakv}) = 1$ (cf. lemma \ref{lemma: H^2_of_witt_algebra}) and that $\eta' \in B^2(\der(\bbC[v^{\pm 1}]), \bbC c_{\frakv})$ (cf. lemma \ref{lemma: a_virasoro_coboundary}), $\eta + \eta'$ must be cohomologous to $\eta$, which is not cohomologous to $0$ as shown in the proof of lemma \ref{lemma: H^2_of_witt_algebra}.

                To prove the consequential statement, simply note that:
                    $$\sigma_i(D_{r, s}, D_{a, b}) = N_i(r, s, a, b) \left( -\delta_{ (-a - r, -b - s - 2), (0, 0) } (r c_v + (s + 1) c_t) + ( r(b + 1) - a(s + 1) )K_{-r - a, -s - b - 2} \right)$$
                where:
                    $$
                        N_i(r, s, a, b) =
                        \begin{cases}
                            \text{$2 rsab - ( (sa)^2 + s a^2 ) - ( (rb)^2 + r^2 b ) + rsa + rab + ra$ if $i = 1$}
                            \\
                            \text{$ra$ if $i = 2$}
                        \end{cases}
                    $$
                Setting $s = b = -1$ then yields:
                    $$\sigma_i(D_{r, -1}, D_{a, -1}) = -N_i(r, -1, a, -1) \delta_{r + a, 0} r c_v$$
                where now, we have that:
                    $$N_i(r, -1, a, -1) = ra$$
                for both $i = 1$ and $i = 2$, and hence:
                    $$\sigma_i(D_{r, -1}, D_{a, -1}) = \delta_{r + a, 0} r^3 c_v = (\eta + \eta')(d_r, d_a)$$
            \end{proof}
        \begin{remark}[The Virasoro central element]
            Because we now know that:
                $$\sigma_i(D_{r, -1}, D_{a, -1}) = \delta_{r + a, 0} r^3 c_v$$
            it is clear that the Virasoro central element $c_{\frakv}$ is nothing but $c_v$.
        \end{remark}

        Now, even though it might be tempting to conclude right away that because of lemma \ref{lemma: billig_toroidal_cocycles_on_the_witt_algebra}, which tells us that:
            $$\sigma_i|_{\bigwedge^2 \frakw}: \bigwedge^2 \frakw \to \bbC c_v$$
        is cohomologous to $\eta + \eta' \in Z^2_{\Lie}(\frakw, \bbC c_v)$, which is not $2$-coboundary, it must then also be true that $\sigma_i \not \in B^2_{\Lie}(\divzero, \z(\toroidal))$. However, the subtlety here is that because $\z(\toroidal)$ is non-trivial as a module over $\divzero$ (and likewise, over the Lie subalgebra $\frakw \subset \divzero$), unlike $\bbC c_v$, one would have to actually check whether or not the restricted toroidal $2$-cocycle:
            $$\sigma_i|_{\bigwedge^2 \frakw}: \bigwedge^2 \divzero \to \z(\toroidal)$$
        is $2$-coboundary. Our claim is that, somewhat surprisingly, it is in fact $2$-coboundary, for both $i = 1$ and $i = 2$.
        \todo[inline]{$\sigma_1|_{\bigwedge^2 \frakw}$ and $\sigma_2|_{\bigwedge^2 \frakw}$ are $2$-coboundary.}
        \begin{lemma} \label{lemma: restrictions_of_billig_toroidal_cocycles_are_coboundary}
            Let $i \in \{1, 2\}$. As $\z(\toroidal)$ is a (\textit{non-trivial}) $\divzero$-module via Lie derivatives (cf. lemma \ref{lemma: derivation_action_on_toroidal_centres}), it is also a module over the Lie subalgebra $\frakw$ of $\divzero$, determined by the same action. Then:
                $$\sigma_i|_{ \bigwedge^2 \frakw } \in B^2_{\Lie}(\frakw, \z(\toroidal))$$
            i.e. they have zero images under the canonical projection $Z^2_{\Lie}(\frakw, \z(\toroidal)) \to H^2_{\Lie}(\frakw, \z(\toroidal))$. 
        \end{lemma}
            \begin{proof}
                Let $i \in \{1, 2\}$. Also, let us regard the $\divzero$-action on $\z(\toroidal)$ as a Lie algebra homomorphism:
                    $$\rho: \divzero \to \gl( \z(\toroidal) )$$
                Recall from lemma \ref{lemma: derivation_action_on_toroidal_centres} that this is given by:
                    $$\rho(D) := [D, -]_{\extendedtoroidal}$$
                for all $D \in \divzero$, which is well-defined because we know that $[\divzero, \z(\toroidal)]_{\extendedtoroidal} \subseteq \z(\toroidal)$ (cf. \textit{loc. cit.}).
            
                Per example \ref{example: low_degree_lie_coboundaries_with_non-trivial_coefficients}, it suffices to prove the existence of a linear map:
                    $$\tau: \divzero \to \z(\toroidal)$$
                (i.e. an element of $C_1(\divzero, \z(\toroidal))$ in the notations of remark \ref{remark: simplified_chevalley_eilenberg_complexes}) such that:
                    $$
                        \begin{aligned}
                            \sigma_i|_{\bigwedge^2 \frakw}(D, D') & = \rho(D) \cdot \tau(D') - \rho(D') \cdot \tau(D) - \tau([D, D'])
                            \\
                            & = [D, \tau(D')]_{\extendedtoroidal} - [D', \tau(D)]_{\extendedtoroidal} - \tau([D, D'])
                        \end{aligned}
                    $$
                We note that since $\sigma_1|_{\bigwedge^2 \frakw} = \sigma_2|_{\bigwedge^2 \frakw}$, the sought-for map $\tau$ should be independent of $i$. 
                    
                For what follows, let us recall:
                \begin{itemize}
                    \item from lemma \ref{lemma: billig_toroidal_cocycles_on_yangian_div_zero_vector_fields} that:
                        $$\sigma_i|_{ \bigwedge^2 \frakw }(D_{r, -1}, D_{a, -1}) = \delta_{r, -a} r^3 c_v$$
                    \item from lemma \ref{lemma: yangian_div_zero_vector_fields_basic_properties} that:
                        $$[D_{a, b}, D_{r, s}] = (br - sa) D_{a + r, b + s + 1}$$
                    \item and from lemma \ref{lemma: explicit_commutators_between_central_basis_elements_and_derivations} that:
                        $$[D_{r, s}, K_{\alpha, \beta}]_{\extendedtoroidal} = ((\beta - 1)r - s\alpha) K_{\alpha - r, \beta - s - 1} + \delta_{(r, s + 1), (\alpha, \beta)} \left( r c_v + s c_t \right)$$
                        $$[D, c_v]_{\extendedtoroidal} = [D, c_t]_{\extendedtoroidal} = 0$$
                    for all $D \in \divzero$.
                \end{itemize}
                Now, since it suffices to prove only that $\tau$ exists, let us assume that it is graded, which means that for every $\alpha \in \Z$, the following is true:
                    $$\tau(D_{\alpha, -1}) = \lambda_{\alpha} K_{-\alpha, 0} + \delta_{\alpha, 0}( \mu_{\alpha} c_v + \nu_{\alpha} c_t )$$
                for some $\lambda_{\alpha}, \mu_{\alpha}, \nu_{\alpha} \in \bbC$, which is because:
                    $$\deg D_{-\alpha, -\beta - 1} = \deg K_{\alpha, \beta} = (\alpha, \beta)$$
                    $$\deg c_v = \deg c_t = (0, 0)$$
                (cf. remark \ref{remark: Z^2_grading_on_toroidal_centres} and corollary \ref{coro: yangian_div_zero_vector_fields_are_graded}). When $D := D_{r, -1}$ and $D' := D_{a, -1}$, the above shall imply that:
                    $$
                        \begin{aligned}
                            & (r - a) \left( \lambda_{a + r} K_{-a - r, 0} + \delta_{a + r, 0}( \mu_{a + r} c_v + \nu_{a + r} c_t ) \right)
                            \\
                            = & (r - a) \tau( D_{r + a, -1} )
                            \\
                            = & \tau( [D_{r, -1}, D_{a, -1}] )
                            \\
                            = & [ D_{r, -1}, \tau(D_{a, -1}) ]_{\extendedtoroidal} - [ D_{a, -1}, \tau(D_{r, -1}) ]_{\extendedtoroidal} - \sigma_i|_{\bigwedge^2 \frakw}(D_{r, -1}, D_{a, -1})
                            \\
                            = & \lambda_a [ D_{r, -1}, K_{-a, 0} ]_{\extendedtoroidal} -  \lambda_r [ D_{a, -1}, K_{-r, 0} ]_{\extendedtoroidal} - \delta_{r + a, 0} r^3 c_v
                            \\
                            = &
                            \lambda_a \left(
                                -(a + r) K_{-a - r, 0} + \delta_{(r, 0), (-a, 0)} \left( r c_v - c_t \right)
                            \right)
                            +
                            \lambda_r \left(
                                -(a + r) K_{-a - r, 0} + \delta_{(a, 0), (-r, 0)} \left( a c_v - c_t \right)
                            -
                            \delta_{r + a, 0} r^3 c_v
                            \right)
                            \\
                            = &
                            -(a + r) ( \lambda_a + \lambda_r ) K_{-a - r, 0}
                            +
                            \delta_{a + r, 0} \left( (r \lambda_a + a \lambda_r - r^3) c_v - (\lambda_a + \lambda_r)c_t \right)
                        \end{aligned}
                    $$
                from which we gather that:
                    $$(r - a) \lambda_{a + r} = -(a + r) ( \lambda_a + \lambda_r )$$
                    $$\delta_{r + a, 0} (r - a) \mu_{a + r} = \delta_{r + a, 0} (r \lambda_a + a \lambda_r - r^3)$$
                    $$\delta_{r + a, 0} (r - a) \nu_{a + r} = -\delta_{r + a, 0}(\lambda_a + \lambda_r)$$
                When $r + a = 0$ and $r \not = 0$ (which together imply in particular that $r \not = a$), the second and third shall become:
                    $$2r \mu_0 = r \lambda_{-r} - r\lambda_r - r^3$$
                    $$2r \nu_0 = -\lambda_{-r} - \lambda_r$$
                from which we get the following system of linear equations:
                    $$
                        \begin{cases}
                            \lambda_{-r} - \lambda_r = r^2 + 2\mu_0
                            \\
                            \lambda_{-r} + \lambda_r = -2\nu_0 r
                        \end{cases}
                    $$
                Solving this system yields:
                    $$\lambda_r = -\frac12 r^2 - \nu_0 r - \mu_0$$
                As such, we have arrived at:
                    $$
                        \begin{aligned}
                            & \tau( D_{r, -1} )
                            \\
                            = & \left( -\frac12 r^2 - \nu_0 r - \mu_0 \right) K_{-r, 0} + \delta_{r, 0} ( \mu_r c_v + \nu_r c_t )
                            \\
                            = & \left( -\frac12 r^2 - \nu_0 r - \mu_0 \right) K_{-r, 0} + \delta_{r, 0} ( \mu_0 c_v + \nu_0 c_t )
                        \end{aligned}
                    $$
                where $\mu_0, \nu_0 \in \bbC$ are undetermined, and we have thus shown that linear maps $\tau \in C_1(\frakw, \z(\toroidal))$ such that $d_1^{\z(\toroidal)}(\tau) = \sigma_i|_{\bigwedge^2 \frakw}$ exist.
            \end{proof}

        The following result is our final conclusion regarding whether or not the toroidal $2$-cocycles $\sigma_1$ and $\sigma_2$ are cohomologous to $0$.
        \begin{proposition}[Are the toroidal $2$-cocycles $\sigma_1, \sigma_2$ cohomologous to $0$ ?] \label{prop: cohomological_non_triviality_of_billig_toroidal_cocycles}
            Let:
                $$\sigma_1, \sigma_2 \in Z^2_{\Lie}(\divzero, \z(\toroidal) )$$
            be the toroidal $2$-cocycles as in example \ref{example: billig_toroidal_cocycles}. $\sigma_1$ is \textit{not} cohomologous to $0$ (i.e. the image of $\sigma_1$ under the canonical projection $Z^2_{\Lie}(\divzero, \z(\toroidal) ) \to H^2_{\Lie}(\divzero, \z(\toroidal) )$ is non-zero), while $\sigma_2$ is, i.e.:
                $$\sigma_1 \not \in B^2_{\Lie}(\divzero, \z(\toroidal))$$
                $$\sigma_2 \in B^2_{\Lie}(\divzero, \z(\toroidal))$$
            
            Moreover, there is a one-parameter family:
                $$\{\tau_2^{\kappa}\}_{\kappa \in \bbC}$$
            of pre-image $\tau_2^{\kappa} \in C_1(\divzero, \z(\toroidal))$ of $\sigma_2$ under $d_1^{\z(\toroidal)}: C_1(\divzero, \z(\toroidal)) \to C_2(\divzero, \z(\toroidal))$\footnote{See remark \ref{remark: simplified_chevalley_eilenberg_complexes} and definition \ref{def: lie_algebra_cohomology} for notations.}. Each $\tau_2^{\kappa}$ is given by:
                $$\tau_2^{\kappa}(D_{r, s}) = 2r K_{-r, -s - 1} + \delta_{(r, s), (0, -1)} ( \kappa c_v - 2c_t )$$
                $$\tau_2^{\kappa}(D_v) = \mu_v K_{0, -1}$$
                $$\tau_2^{\kappa}(D_t) = \mu_t K_{0, -1}$$
            where the coefficients $\mu_v, \mu_t \in \bbC$ are undetermined.
        \end{proposition}
        \begin{remark}[Some recollection]
            The proof of proposition \ref{prop: cohomological_non_triviality_of_billig_toroidal_cocycles} is somewhat computation-heavy, so for the sake of visual clarity and convenience, let us recall for the proof the following facts that have been earlier in this chapter.
            \begin{itemize}
                \item Firstly, recall from remark \ref{remark: Z^2_grading_on_toroidal_centres} and corollary \ref{coro: yangian_div_zero_vector_fields_are_graded} that:
                    $$\deg D_{-a, -b - 1} = \deg K_{a, b} = (a, b)$$
                    $$\deg D_v = \deg D_t = (0, -1)$$
                    $$\deg c_v = \deg c_t = (0, 0)$$
                \item Secondly, let us recall from lemma \ref{lemma: yangian_div_zero_vector_fields_basic_properties} that:
                    $$[D_v, D_{r, s}] = -r D_{r, s + 1}$$
                    $$[D_t, D_{r, s}] = -s D_{r, s + 1}$$
                    $$[D_{a, b}, D_{r, s}] = (br - sa) D_{a + r, b + s + 1}$$
                \item Finally, from lemma \ref{lemma: explicit_commutators_between_central_basis_elements_and_derivations}, we know that:
                    $$[D, K_{\alpha, \beta}]_{\extendedtoroidal} =
                        \begin{cases}
                            \text{$((\beta - 1)r - s\alpha) K_{\alpha - r, \beta - s - 1} + \delta_{(r, s + 1), (\alpha, \beta)} \left( r c_v + s c_t \right)$ if $D = D_{r, s}$}
                            \\
                            \text{$-\alpha K_{\alpha, \beta - 1}$ if $D_v$}
                            \\
                            \text{$-\beta K_{\alpha, \beta - 1}$ if $D_t$}
                        \end{cases}
                    $$
                    $$[D, c_v]_{\extendedtoroidal} = [D, c_t]_{\extendedtoroidal} = 0$$
                for all $D \in \divzero$.
            \end{itemize}
        \end{remark}
        We present now the proof of proposition \ref{prop: cohomological_non_triviality_of_billig_toroidal_cocycles}.
            \begin{proof}
                Even though it suffices to prove the existence of an element:
                    $$\tau_i \in C_1(\divzero, \z(\toroidal))$$
                (i.e. a linear map; cf. remark \ref{remark: simplified_chevalley_eilenberg_complexes}) such that:
                    $$\tau_i([D, D']) = [D, \tau_i(D')]_{\extendedtoroidal} - [D', \tau_i(D)]_{\extendedtoroidal} - \sigma_i(D, D')$$
                we shall prove a stronger statement, namely the existence of $\tau_i \in C_1(\divzero, \z(\toroidal))$ that is \textit{graded} with respect to the $\Z^2$-gradings on $\divzero$ and $\z(\toroidal)$. We will do this by firstly assuming that such a graded $\tau_i$ exists, deriving an explicit closed-form formula for it, and then verifying that such a $\tau_i$ must then be $2$-coboundary.

                To that end, assume that $\tau_i$ as described above indeed exists. We will then have:
                \begin{itemize}
                    \item coefficients:
                        $$\lambda_{r, s}, \alpha'_{r, s}, \alpha_{r, s}, \beta_{r, s} \in \bbC$$
                    such that:
                        $$
                            \begin{aligned}
                                \tau_i(D_{r, s}) & = \lambda_{r, s} K_{-r, -s - 1} + \delta_{(r, s), (0, -1)} ( \alpha'_{r, s} K_{r, 0} + \alpha_{r, s} c_v + \beta_{r, s} c_t )
                                \\
                                & = \lambda_{r, s} K_{-r, -s - 1} + \delta_{(r, s), (0, -1)} ( \alpha_{r, s} c_v + \beta_{r, s} c_t )
                            \end{aligned}
                        $$
                    where the second equality holds because $K_{0, 0} = 0$ (cf. example \ref{example: toroidal_lie_algebras_centres}), 
                    \item as well as coefficients:
                        $$\mu_v, \mu_t \in \bbC$$
                    such that:
                        $$\tau_i(D_v) = \mu_v K_{0, -1}$$
                        $$\tau_i(D_t) = \nu_t K_{0, -1}$$
                \end{itemize}
                
                \todo[inline]{Fixed the general expressions for $\tau_i(D_v)$ and for $\tau_i(D_t)$.}
                \begin{itemize}
                    \item We attempt first of all to compute $\tau_i(D_{r, s})$. For this, we refer the reader to lemma \ref{lemma: billig_toroidal_cocycles_on_yangian_div_zero_vector_fields} for the following fact:
                        $$\sigma_i(D_v, D_{r, s}) = \delta_{i, 1} r^3 K_{-r, -s - 2}$$
                    This gives the following\footnote{Using $D_t$ in place of $D_v$ would yield the same conclusion.}:
                        $$
                            \begin{aligned}
                                & r \tau_i(D_{r, s + 1})
                                \\
                                = & \tau_i([D_v, D_{r, s}])
                                \\
                                = & [D_v, \tau_i(D_{r, s})]_{\extendedtoroidal} - [D_{r, s}, \tau_i(D_v)]_{\extendedtoroidal} - \sigma_i(D_v, D_{r, s})
                                \\
                                = & [D_v, \lambda_{r, s} K_{-r, -s - 1}]_{\extendedtoroidal} - [D_{r, s}, \mu_v c_v + \mu_t c_t]_{\extendedtoroidal} - \delta_{i, 1} r^3 K_{-r, -s - 2}
                                \\
                                = & r \lambda_{r, s} K_{-r, -s - 2} - \delta_{i, 1} r^3 K_{-r, -s - 2}
                                \\
                                = & ( r \lambda_{r, s} - \delta_{i, 1} r^3 ) K_{-r, -s - 2}
                            \end{aligned}
                        $$
                    From this, we infer that when $r \not = 0$, we have that:
                        $$\tau_i(D_{r, s + 1}) = (\lambda_{r, s} - \delta_{i, 1} r^2) K_{-r, -s - 2}$$
                    but at the same time, we have the following per our initial assumption that $\tau_i$ is graded:
                        $$\tau_i(D_{r, s + 1}) = \lambda_{r, s + 1} K_{-r, -s - 2} + \delta_{(r, s + 1), (0, -1)} ( \alpha_{r, s + 1} c_v + \beta_{r, s + 1} c_t )$$
                    and so we have:
                        $$\lambda_{r, s + 1} = \lambda_{r, s} - \delta_{i, 1} r^2$$
                    whenever $r \not = 0$ (in which case ${(r, s + 1), (0, -1)} = 0$).
                    
                    Next, consider the following:
                        $$
                            \begin{aligned}
                                & (br - sa) \tau_i(D_{r + a, b + s + 1})
                                \\
                                = & \tau_i( [D_{a, b}, D_{r, s}] )
                                \\
                                = & [D_{a, b}, \tau_i(D_{r, s})]_{\extendedtoroidal} - [D_{r, s}, \tau_i(D_{a, b})]_{\extendedtoroidal} - \sigma_i(D_{a, b}, D_{r, s})
                                \\
                                = & \lambda_{r, s} [D_{a, b}, K_{-r, -s - 1}]_{\extendedtoroidal} - \lambda_{a, b} [D_{r, s}, K_{-a, -b - 1}]_{\extendedtoroidal} + \sigma_i(D_{r, s}, D_{a, b})
                            \end{aligned}
                        $$
                    Using lemma \ref{lemma: explicit_commutators_between_central_basis_elements_and_derivations}, we get that:
                        $$[D_{a, b}, K_{-r, -s - 1}]_{\extendedtoroidal} = -\left( a(s + 2) - br \right) K_{-a - r, -b - s - 2} + \delta_{(a + r, b + s + 2), (0, 0)} \left( a c_v + b c_t \right)$$
                        $$[D_{r, s}, K_{-a, -b - 1}]_{\extendedtoroidal} = -\left( (b + 2) r - as \right) K_{-a - r, -b - s - 2} + \delta_{(a + r, b + s + 2), (0, 0)} \left( r c_v + s c_t \right)$$
                    and from lemma \ref{lemma: billig_toroidal_cocycles_on_yangian_div_zero_vector_fields}, we know that:
                        $$\sigma_i(D_{r, s}, D_{a, b}) = N_i(r, s, a, b) \left( ( r(b + 1) - a(s + 1) )K_{-r - a, -s - b - 2} - \delta_{ (a + r, b + s + 2), (0, 0) } (r c_v + (s + 1) c_t) \right)$$
                    where $N_i(r, s, a, b)$ is as in \textit{loc. cit.} Simultaneously, these fact imply that:
                    \todo[inline]{I think the mistakes were in some of the indices here. I also forgot to include $\lambda_{r, s}$ and $\lambda_{a, b}$.}
                        $$
                            \begin{aligned}
                                & (br - sa) \tau_i(D_{r + a, b + s + 1})
                                \\
                                = &
                                \begin{aligned}
                                    & \left( -\lambda_{r, s} \left( a(s + 2) - br \right) + \lambda_{a, b} \left( (b + 2) r - as \right) - N_i(r, s, a, b)\left( r(b + 1) - a(s + 1) \right) \right) K_{-a - r, -b - s - 2}
                                    \\
                                    + & \delta_{(a + r, b + s + 2), (0, 0)} \left( \lambda_{r, s} a - \lambda_{a, b} r + N_i(r, s, a, b) r \right) c_v
                                    \\
                                    + & \delta_{(a + r, b + s + 2), (0, 0)} \left( \lambda_{r, s} b - \lambda_{a, b} s + N_i(r, s, a, b) (s + 1) \right) c_t
                                \end{aligned}
                                \\
                                = & (br - sa) \left( \lambda_{a + r, b + s + 1} K_{-a - r, -b - s - 2} + \delta_{(a + r, b + s + 1), (0, -1)}( \alpha_{a + r, b + s + 1} c_v + \beta_{a + r, b + s + 1} c_t ) \right)
                            \end{aligned}
                        $$
                    from which it can be inferred that:
                        $$
                            \lambda_{a + r, b + s + 1}
                            =
                            -\lambda_{r, s} \left( a(s + 2) - br \right) + \lambda_{a, b} \left( (b + 2) r - as \right) - N_i(r, s, a, b)\left( r(b + 1) - a(s + 1) \right)
                        $$
                        $$
                            \delta_{(a + r, b + s + 1), (0, -1)} \alpha_{a + r, b + s + 1}
                            =
                            \delta_{(a + r, b + s + 2), (0, 0)} \left( \lambda_{r, s} a - \lambda_{a, b} r + N_i(r, s, a, b) r \right)
                        $$
                        $$
                            \delta_{(a + r, b + s + 1), (0, -1)} \beta_{a + r, b + s + 1}
                            =
                            \delta_{(a + r, b + s + 2), (0, 0)} \left( \lambda_{r, s} b - \lambda_{a, b} s + N_i(r, s, a, b) (s + 1) \right)
                        $$
                    By evaluating the first equation at $(a, b) = (0, 0)$ and then combining the result with the fact that $\lambda_{r, s + 1} = \lambda_{r, s} - \delta_{i, 1} r^2$ (as shown above), we will get that:
                        $$
                            \begin{aligned}
                                \lambda_{r, s + 1} & = \lambda_{r, s} - \delta_{i, 1} r^2
                                \\
                                = & 2r \lambda_{0, 0} - r N_i(r, s, 0, 0)
                                \\
                                = & 2r \lambda_{0, 0}
                            \end{aligned}
                        $$
                    (where we have used the fact that $N_i(r, s, 0, 0) = 0$), which in turn implies that:
                        $$\lambda_{r, s} = \delta_{i, 1} r^2 + 2r \lambda_{0, 0}$$
                    We thus have that:
                        $$\tau_i(D_{r, s + 1}) = (-\delta_{i, 1} r^2 + \lambda_{r, s}) K_{-r, -s - 2} = 2r \lambda_{0, 0} K_{-r, -s - 2}$$
                    and hence:
                        $$\tau_i(D_{r, s}) = 2r \lambda_{0, 0} K_{-r, -s - 1}$$
                    This clearly does \textit{not} depend on $i$, so already, we can conclude that only one of either $\sigma_1$ or $\sigma_2$ can not be $2$-coboundary.

                    To check this, let us evaluate the following at $(a, b) = (0, -1)$:
                        $$
                            \begin{aligned}
                                & (br - sa) \tau_i( D_{r + a, b + s + 1} )
                                \\
                                = & 
                                \begin{aligned}
                                    & \left( -\lambda_{r, s} \left( a(s + 2) - br \right) + \lambda_{a, b} \left( (b + 2) r - as \right) - N_i(r, s, a, b)\left( r(b + 1) - a(s + 1) \right) \right) K_{-a - r, -b - s - 2}
                                    \\
                                    + & \delta_{(a + r, b + s + 2), (0, 0)} \left( \lambda_{r, s} a - \lambda_{a, b} r + N_i(r, s, a, b) r \right) c_v
                                    \\
                                    + & \delta_{(a + r, b + s + 2), (0, 0)} \left( \lambda_{r, s} b - \lambda_{a, b} s + N_i(r, s, a, b) (s + 1) \right) c_t
                                \end{aligned}    
                            \end{aligned}
                        $$
                    in order to get:
                        $$
                            \begin{aligned}
                                & -r \tau_i(D_{r, s}) =
                                \\
                                = & ( -\lambda_{r, s} r + \lambda_{0, -1} r ) K_{-r, -s - 1} - \delta_{(r, s), (0, -1)} \lambda_{0, -1} r c_v - \delta_{(r, s), (0, -1)} ( \lambda_{r, s} + \lambda_{0, -1} s ) c_t
                                \\
                                = & -r(\lambda_{r, s} - \lambda_{0, -1}) K_{-r, -s - 1} - \delta_{(r, s), (0, -1)} \left( r c_v + ( \lambda_{r, s} + \lambda_{0, -1} s ) c_t \right)
                            \end{aligned}
                        $$
                    (note that $N_i(r, s, 0, -1) = 0$). When $r \not = 0$, this is equivalent to:
                        $$\tau_i(D_{r, s}) = (\lambda_{r, s} - \lambda_{0, -1}) K_{-r, -s - 1}$$
                   (note how the second term vanishes when $r \not = 0$) and therefore, we have that:
                        $$\lambda_{r, s} K_{-r, -s - 1} = ( \delta_{i, 1} r^2 + 2r \lambda_{0, 0} ) K_{-r, -s - 1}$$
                    Through with the previous conclusion that:
                        $$\tau_i(D_{r, s}) = 2r \lambda_{0, 0} K_{-r, -s - 1}$$
                    the above implies that \textit{only when $i = 2$ can we have a well-defined linear map:}
                        $$\tau_2 \in C_1( \divzero, \z(\toroidal) )$$
                    We can thus conclude also, that:
                        $$\sigma_1 \not \in B^2_{\Lie}( \divzero, \z(\toroidal) )$$

                    \todo[inline]{Everything below this line needs revision}
                    As such, let us only work with $i = 2$ from now on. Now, let us return to the initial assumption that $\tau_2$ is graded, which means that:
                        $$\tau_2(D_{r, s}) = \lambda_{r, s} K_{-r, -s - 1} + \delta_{(r, s), (0, -1)} ( \alpha_{r, s} c_v + \beta_{r, s} c_t )$$
                    (as discussed at the beginning), we see that when $(r, s) = (0, -1)$, we shall have that:
                        $$\tau_2(D_{0, -1}) = \alpha_{0, -1} c_v + \beta_{0, -1} c_t$$
                    In lemma \ref{lemma: restrictions_of_billig_toroidal_cocycles_are_coboundary}, we have seen that:
                        $$\tau_2|_{\frakw}(D_{0, -1}) = \mu_0 c_v + \nu c_t$$
                    for some undetermined $\mu_0, \nu_0 \in \bbC$. Hence, we have that:
                        $$\mu_0 = \alpha_{0, -1}, \nu_0 = \beta_{0, -1}$$
                    We claim now that $\beta_{0, -1}$ is actually determinable, while $\alpha_{0, -1}$ is not. Thus, let us set:
                        $$\alpha_{0, -1} := \kappa$$
                    where $\kappa \in \bbC$ is a parameter. This leads to a one-parameter family:
                        $$\{\tau_2^{\kappa}\}_{\kappa \in \bbC}$$
                    of elements $\tau_2^{\kappa} \in C_1(\divzero, \z(\toroidal))$ such that $d_1^{\z(\toroidal)}(\tau_2^{\kappa}) = \sigma_2$. Indeed, if we consider the equations:
                        $$\delta_{(a + r, b + s + 1), (0, -1)} \alpha_{a + r, b + s + 1} = \delta_{(a + r, b + s + 2), (0, 0)} (a + r) + \delta_{(a + r, b + s), (0, 0)} r N_i(r, s, a, b)$$
                        $$\delta_{(a + r, b + s + 1), (0, -1)} \beta_{a + r, b + s + 1} = \delta_{(a + r, b + s + 2), (0, 0)} (b + s) + \delta_{(a + r, b + s), (0, 0)} (s + 1) N_i(r, s, a, b)$$
                    that we arrived at earlier, and then evaluate these equations at $(a, b) = (r, s) = (0, -1)$, we will obtain:
                        $$0 = 0$$
                        $$\beta_{0, -1} = -2$$
                    The first equation is tautological, so $\alpha_{0, -1}$ remains undetermined, and our claim is proven.
                    \todo[inline]{Double-check until here.}

                    In conclusion, to check whether $\sigma_2$ is $2$-coboundary, $\sigma_1$ is not, we can use the following family of elements $\tau_2^{\kappa} \in C_1(\divzero, \z(\toroidal))$:
                        $$\tau_2^{\kappa}(D_{r, s}) = $$
                    Of course, it remains still to verify that each $\tau_2^{\kappa}$ satisfies the $2$-coboundary equation:
                        $$\tau_2^{\kappa}([D, D']) = [D, \tau_2^{\kappa}(D')]_{\extendedtoroidal} - [D', \tau_2^{\kappa}(D)]_{\extendedtoroidal} - \sigma_2(D, D')$$
                    (for all $D, D' \in \divzero$). We will get back to this shortly, after computing $\tau_2^{\kappa}(D_v)$ and $\tau_2^{\kappa}(D_t)$.
                    \item Next, consider $\tau_2^{\kappa}(D_v)$. Because there do not exist elements $D, D' \in \divzero$ so that $D_v = [D, D']$ and because $[D_v, D_t] = 0$ (cf. lemma \ref{lemma: yangian_div_zero_vector_fields_basic_properties}), it is impossible to determine $\tau_2^{\kappa}(D_v)$ through the use of the $2$-coboundary equation:
                        $$\tau_2^{\kappa}([D, D']) = [D, \tau_2^{\kappa}(D')]_{\extendedtoroidal} - [D', \tau_2^{\kappa}(D)]_{\extendedtoroidal} - \sigma_2(D, D')$$
                    (cf. example \ref{example: low_degree_lie_coboundaries_with_non-trivial_coefficients}). As such, we can let the coefficients $\mu_v \in \bbC$ in the expression:
                        $$\tau_i(D_v) = \mu_v K_{0, -1}$$
                    be arbitrary.
                    \item Likewise, we can let the coefficients $\mu_v \in \bbC$ in the expression:
                        $$\tau_2^{\kappa}(D_t) = \mu_t K_{0, -1}$$
                    be arbitrary.
                \end{itemize}

                Now, it remains to check whether or not the $2$-boundary equation:
                    $$\tau_2^{\kappa}([D, D']) = [D, \tau_2^{\kappa}(D')]_{\extendedtoroidal} - [D', \tau_2^{\kappa}(D)]_{\extendedtoroidal} - \sigma_2(D, D')$$
                is satisfied for all $D, D' \in \divzero$. Without loss of generality, we can assume again that $D, D' \in \divzero$ are basis elements.
                \begin{itemize}
                    \item Firstly, consider $D := D_{r, s}$ and $D' := D_{a, b}$. Since we now know that:
                        $$\tau_2^{\kappa}(D_{\alpha, \beta}) = 2\alpha K_{-\alpha, -\beta - 1} + \delta_{(\alpha, \beta), (0, -1)} ( \kappa c_v - 2c_t )$$
                    for all $(\alpha, \beta) \in \Z^2$, we thus have that:
                        $$\tau_2([ D_{r, s}, D_{a, b} ]) = (br - sa) \tau_2(D_{a + r, b + s + 1}) = (br - sa) \left( 2(a + r) K_{-a - r, -b - s - 2} + \delta_{(a + r, b + s + 1), (0, -1)} ( \kappa c_v - 2c_t ) \right)$$
                    as well as:
                        $$
                            \begin{aligned}
                                & [D_{r, s}, \tau_2(D_{a, b})]_{\extendedtoroidal} - [D_{a, b}, \tau_2(D_{r, s})]_{\extendedtoroidal} - \sigma_2(D_{r, s}, D_{a, b})
                                \\
                                = &
                                \begin{aligned}
                                    & 2a [ D_{r, s}, K_{-a, -b - 1} ]_{\extendedtoroidal}
                                    \\
                                    - & 2r [ D_{a, b}, K_{-r, -s - 1} ]_{\extendedtoroidal}
                                    \\
                                    - & ra \left( ( r(b + 1) - a(s + 1) )K_{-r - a, -s - b - 2} -\delta_{ (-a - r, -b - s - 2), (0, 0) } (r c_v + (s + 1) c_t) \right)
                                \end{aligned}
                                \\
                                = &
                                \begin{aligned}
                                    & 2a \left( (-(b + 2)r + sa) K_{-a - r, -b - s - 2} + \delta_{(r, s + 1), (-a, -b - 1)} \left( r c_v + s c_t \right) \right)
                                    \\
                                    - & 2r \left( (-(s + 2)a + br) K_{-a - r, -b - s - 2} + \delta_{(a, b + 1), (-r, -s - 1)} \left( a c_v + b c_t \right) \right)
                                    \\
                                    - & ra \left( ( r(b + 1) - a(s + 1) )K_{-r - a, -s - b - 2} + \delta_{(a + r, b + s + 1), (0, -1)} (r c_v + (s + 1) c_t) \right)
                                \end{aligned}
                                \\
                                = &
                                \begin{aligned}
                                    & \left( 2(a - r) (sa - br) - ra ( as - br + a - r ) \right) K_{-a - r, -b - s - 2}
                                    \\
                                    + & \delta_{(a + r, b + s + 1), (0, -1)} \left( -r^2 a c_v + ( 2a(s + 1) + 2r(s + 2) - ra(s + 1) ) c_t \right)
                                \end{aligned}
                                \\
                                = &
                                \begin{aligned}
                                    & \left( 2(a - r) (sa - br) - ra ( as - br + a - r ) \right) K_{-a - r, -b - s - 2}
                                    \\
                                    + & \delta_{(a + r, b + s + 1), (0, -1)} \left( r^3 c_v + ( -2r + r^2(s + 1) ) c_t \right)
                                \end{aligned}
                            \end{aligned}
                        $$
                    \item 
                    \item 
                \end{itemize}
            \end{proof}
        \begin{remark}
            Through proposition \ref{prop: invariance_of_billig_toroidal_cocycles}, we have seen that the toroidal $2$-cocycle:
                $$\sigma_2$$
            is \textit{not} $\gamma$-invariant in the sense of definition \ref{def: yangian_toroidal_cocycles}. However, in light of theorem \ref{prop: cohomological_non_triviality_of_billig_toroidal_cocycles}, we now see that this non-$\gamma$-invariance of $\sigma_2$ holds \textit{despite} the fact that it is cohomologous to $0$. Therefore, one can not conclude from observing that a toroidal $2$-cocycle $\sigma$ is cohomologous to $0$ that $\sigma$ is $\gamma$-invariant, but when $\sigma = 0$ as elements of $Z^2_{\Lie}(\divzero, \z(\toroidal))$, then it will indeed be true that $\sigma$ is $\gamma$-invariant, as we know that the semi-direct product $\toroidal \rtimes \divzero$ is an instance of a $\gamma$-extended toroidal Lie algebra.
        \end{remark}