\section{Examples of \texorpdfstring{$\gamma$}{}-extended toroidal Lie algebras}
    \subsection{Statement of the main theorem}
        For convenience, let us fix the following terminologies (which have already been eluded to in the statement of theorem \ref{theorem: yangian_extended_toroidal_lie_algebras_main_theorem}).
        \begin{definition}[$\gamma$-invariant toroidal $2$-cocycles] \label{def: yangian_toroidal_cocycles}
            Any $2$-cocyle $\sigma \in Z^2_{\Lie}(\divzero, \z(\toroidal))$ shall be referred to as a \textbf{toroidal $2$-cocycle}.
            
            Any toroidal $2$-cocycle $\sigma$ such that $\toroidal \rtimes^{\sigma} \divzero$ is a $\gamma$-extended toroidal Lie algebra shall be called a \textbf{$\gamma$-invariant toroidal $2$-cocycle}.
        \end{definition}
        
        In this section, we seek to produce examples of $\gamma$-extended toroidal Lie algebras beyond the semi-direct product $\toroidal \rtimes \divzero$ (which we have shown to be a $\gamma$-extended Lie algebra in lemma \ref{lemma: semi_direct_product_of_toroidal_lie_algebras_with_div_zero_vector_fields_are_yangian_extended_toroidal_lie_algebras}). We will do this through examination of some toroidal $2$-cocycles that have been known from \cite{billig_energy_momentum_tensor} (see also: \cite{billig_a_module_category_over_toroidal_EALAs}).

        Let us begin by introducing the two toroidal $2$-cocycles we will be working with. In \cite[p. 5, below Equation 1.3]{billig_energy_momentum_tensor}, it was noted that there are at least $2$-cocyles:
            $$\sigma_1, \sigma_2 \in Z^2_{\Lie}( \der(A), \z(\toroidal) )$$
        These are given in terms of the following basis of $\der(A)$:
            $$\{ v^{r_v} t^{r_t} \cdot v \del_v \}_{(r_v, r_t) \in \Z^2} \cup \{ v^{m_v} t^{m_t} \cdot t \del_t \}_{(m_v, m_t) \in \Z^2}$$
        by the following formulae:
            \begin{equation} \label{equation: billig_toroidal_cocycles}
                \begin{aligned}
                    & \sigma_1( v^{r_v} t^{r_t} \cdot x \del_x, v^{m_v} t^{m_t} \cdot y \del_y ) := r_y m_x \cdot v^{r_v} t^{r_t} \bar{d}( v^{m_v} t^{m_t} )
                    \\
                    & \sigma_2( v^{r_v} t^{r_t} \cdot x \del_x, v^{m_v} t^{m_t} \cdot y \del_y ) := r_x m_y \cdot v^{r_v} t^{r_t} \bar{d}( v^{m_v} t^{m_t} )
                \end{aligned}
            \end{equation}
        where $x, y \in \{v, t\}$ are symbolic placeholders.
        
        As a quick aside, let us note that the first cocycle $\sigma_1$ was known as far back as \cite{moody_rao_n_toroidal_vertex_representations}, but we are not aware of the history of $\sigma_2$ beyond its appearance in first \cite{billig_energy_momentum_tensor}.

        \begin{remark}
            Before we state the main theorem of this section, let us remind the reader of a few relevant notions from Lie algebra cohomology, and for more details, we refer the reader to appendix \ref{section: lie_algebra_cohomology_appendix}.
            
            If $\a$ is a Lie algebra and $M$ is an $\a$-module then into the vector space of Lie $2$-cocycles $Z^2_{\Lie}(\a, M)$ of $\a$ with values in $M$, there will be a linear map:
                $$d_1^M: \Hom_{\bbC}(\a, M) \to Z^2_{\Lie}(\a, M)$$
            called the \textbf{differential}, given by:
                $$d_1^M(\tau)(X, Y) = X \cdot \tau(Y) - Y \cdot \tau(X) - \tau([X, Y])$$
            for all linear maps $\tau \in \Hom_{\bbC}(\a, M)$ and all $X, Y \in \a$. The $2$-cocycles that are in the image of $d_1^M$ are called \textbf{$2$-coboundaries}. Also, one typically writes:
                $$B^2_{\Lie}(\a, M) := \im d_1^M$$
            as well as
                $$H^2_{\Lie}(\a, M) := Z^2_{\Lie}(\a, M)/B^2_{\Lie}(\a, M)$$
            This latter vector space is called the \textbf{$2^{nd}$ Lie algebra cohomology} of $\a$ with coefficients in $M$, and note that a $2$-cocycle is $2$-coboundary if and only if its image under the canonical projection $Z^2_{\Lie}(\a, M) \to H^2_{\Lie}(\a, M)$ is zero; in that case, the cocycle might also be be said to be \textbf{cohomologus} to $0$.
        \end{remark}

        Our main result concerning $\sigma_1$ and $\sigma_2$ is as below:
        \begin{theorem}[Non-trivial (counter-)examples of $\gamma$-extended toroidal Lie algebras] \label{theorem: billig_cocycle_main_theorem}
            Consider the $2$-cocycles:
                $$\sigma_1, \sigma_2 \in Z^2_{\Lie}(\divzero, \z(\toroidal))$$
            as in equation \eqref{equation: billig_toroidal_cocycles}.
            \begin{enumerate}
                \item $\sigma_1$ is $\gamma$-invariant in the sense of definition \ref{def: yangian_toroidal_cocycles}, and has non-zero image under the canonical projection $Z^2_{\Lie}(\divzero, \z(\toroidal)) \to H^2_{\Lie}(\divzero, \z(\toroidal))$, and hence $\toroidal \rtimes^{\sigma_1} \divzero$ is a $\gamma$-extended toroidal Lie algebra that is \textit{not} isomorphic to $\toroidal \rtimes \divzero$.
                \item On the other hand, $\sigma_2$ is \textit{not} $\gamma$-invariant, and has zero image under the canonical projection $Z^2_{\Lie}(\divzero, \z(\toroidal)) \to H^2_{\Lie}(\divzero, \z(\toroidal))$, and hence $\toroidal \rtimes^{\sigma_2} \divzero$ is not a $\gamma$-extended toroidal Lie algebra.

                Moreover, there is a one-parameter family $\{\tau_2^{\kappa}\}_{\kappa \in \bbC}$ of graded linear maps $\tau_2^{\kappa} \in C_1(\divzero, \z(\toroidal))$ of $\sigma_2$ under $d_1^{\z(\toroidal)}: C_1(\divzero, \z(\toroidal)) \to C_2(\divzero, \z(\toroidal))$\footnote{See remark \ref{remark: simplified_chevalley_eilenberg_complexes} and definition \ref{def: lie_algebra_cohomology} for notations. In particular, we have that $C_i(\a, M) := \Hom_{\bbC}\left(\bigwedge^i \a, M \right)$ for all Lie algebras $\a$ and all $\a$-modules $M$.}, whose elements are given by:
                    $$\tau_2^{\kappa}(D_{r, s}) = \left( \frac12 r^2 + r\kappa \right) K_{-r, -s - 2} + \delta_{(r, s), (0, -1)} \kappa c_t$$
                    $$\tau_2^{\kappa}(D_v) = 0$$
                    $$\tau_2^{\kappa}(D_t) = \kappa K_{0, -1}$$
            \end{enumerate}
        \end{theorem}
            \begin{proof}[Proof outline]
                Our proof of theorem \ref{theorem: billig_cocycle_main_theorem} will end up being rather long and computational, and therefore will be split up into multiple sections, culminating particularly in propositions \ref{prop: invariance_of_billig_toroidal_cocycles}, \ref{prop: sigma_1_is_not_coboundary}, and \ref{prop: cohomological_non_triviality_of_billig_toroidal_cocycles}. Below is an outline of our strategy.
                \begin{enumerate}
                    \item Firstly, we need to give descriptions of $\sigma_1, \sigma_2$ as elements of $Z^2_{\Lie}(\divzero, \z(\toroidal))$ instead of elements of $Z^2_{\Lie}(\der(A), \z(\toroidal))$, which is the same as computing the domain restrictions of these two cocycles from $\bigwedge^2 \der(A)$ down to the vector subspace $\bigwedge^2 \divzero$. This is done in lemma \ref{lemma: billig_toroidal_cocycles_on_yangian_div_zero_vector_fields}.
                    \item Using the $\gamma$-invariance criterion given in proposition \ref{proposition: twisted_semi_direct_products_are_yangian_extended_toroidal_lie_algebras}, we can then check that $\sigma_1$ is $\gamma$-invariant while $\sigma_2$ is not by directly performing computations.
                    \item Checking whether or not the images of $\sigma_1$ and of $\sigma_2$ might be zero under the canonical projection $Z^2_{\Lie}(\divzero, \z(\toroidal)) \to H^2_{\Lie}(\divzero, \z(\toroidal))$ (cf. definition \ref{def: lie_algebra_cohomology}) requires the following two steps, to be carried out in sequence. Let $i \in \{1, 2\}$.
                    \begin{enumerate}
                        \item Firstly, we shall argue that should $\sigma_i$ be $2$-coboundary, then an element:
                            $$\tau_i \in C_1(\divzero, \z(\toroidal))$$
                        (cf. remark \ref{remark: simplified_chevalley_eilenberg_complexes}) such that:
                            $$d_1^{\z(\toroidal)}(\tau_i) = \sigma_i$$
                        (cf. definition \ref{def: lie_algebra_cohomology}) will have to be graded\footnote{Recall that both $\divzero$ and $\z(\toroidal)$ are $\Z^2$-graded vector spaces (cf. example \ref{example: toroidal_lie_algebras_centres} and ccorollary \ref{coro: yangian_div_zero_vector_fields_are_graded}).} and be necessarily given by a certain formula, in which the only freedom lies in a parameter $\kappa \in \bbC$; incidentally, this is why we have written $\tau_2^{\kappa}$ for the pre-images along $d_1^{\z(\toroidal)}$ of $\sigma_2$ in theorem \ref{theorem: billig_cocycle_main_theorem}. This is the content of proposition \ref{prop: sigma_1_is_not_coboundary}. During the process, we will see that $\sigma_1$ is actually \textit{not} $2$-coboundary, and hence has non-zero image in $H^2_{\Lie}(\divzero, \z(\toroidal))$.
                        \item Secondly, to verify that $\sigma_2$ is $2$-coboundary, we will need to check that it satisfies:
                            $$\tau_2^{\kappa}([D, D']) = [ D, \tau_2^{\kappa}(D') ]_{\extendedtoroidal} - [ D', \tau_2^{\kappa}(D) ]_{\extendedtoroidal} - \sigma_2(D, D')$$
                        for all $D, D' \in \divzero$ (cf. example \ref{example: low_degree_lie_coboundaries_with_non_trivial_coefficients}). This will be done in proposition \ref{prop: cohomological_non_triviality_of_billig_toroidal_cocycles}, which concludes our proof.
                    \end{enumerate}
                \end{enumerate}
            \end{proof}

        Along the way (cf. remark \ref{remark: billig_toroidal_cocycles_on_the_witt_algebra}), we will also make note of a curious phenomenon, whereby after being restricted down to the homomorphic image $\frakw \subset \divzero$ of the Witt algebra $\der(\bbC[z^{\pm 1}])$ inside $\divzero$ (cf. proposition \ref{prop: a_copy_of_the_witt_algebra_inside_the_lie_algebra_of_yangian_div_zero_vector_fields}), both $\sigma_1$ and $\sigma_2$ will become equal to a $2$-cocycle of $\frakw$ with values in the trivial module $\bbC c_v$ that is \textit{not} cohomologous to $0$. The $2$-cocycles $\sigma_1$ and $\sigma_2$ therefore give rise to the unique non-trivial central extension of the Witt algebra (which happens also to be its UCE), i.e. the Virasoro algebra.
    
    \subsection{\texorpdfstring{$\gamma$}{}-invariance}
        Since lemma \ref{lemma: yangian_div_zero_vector_fields_basic_properties}, we have known how the elements of the basis:
            $$\{D_{r, s}\}_{(r, s) \in \Z^2} \cup \{D_v, D_t\}$$
        of $\divzero$ are given in terms of the partial derivatives $\del_v, \del_t$ (or more accurately, in terms of the aforementioned basis of $\der(A)$) by:
            $$D_{r, s} = -s v^{-r + 1} t^{-s - 1} \del_v + r v^{-r} t^{-s} \del_t = -s v^{-r} t^{-s - 1} \cdot v\del_v + r v^{-r} t^{-s - 1} \cdot t\del_t$$
            $$D_v = -v t^{-1} \del_v = -t^{-1} \cdot v\del_v$$
            $$D_t = -\del_t = -t^{-1} \cdot t\del_t$$
        (to rewrite the expressions slightly in terms of the currently employed basis for $\der(A)$). Knowing this allows us to compute the values of $\sigma_1$ and $\sigma_2$ on pairs of these basis elements.
        \begin{lemma}[Values of $\sigma_1$ and $\sigma_2$ on pairs of basis elements of $\divzero$] \label{lemma: billig_toroidal_cocycles_on_yangian_div_zero_vector_fields}
            In what follows, let $i \in \{1, 2\}$.
            \begin{enumerate}
                \item The values of the toroidal $2$-cocycles $\sigma_1, \sigma_2$ as in equation \eqref{equation: billig_toroidal_cocycles} on pairs of elements of the basis $\{D_{r, s}\}_{(r, s) \in \Z^2} \cup \{D_v, D_t\}$ of $\divzero$ (cf. lemma \ref{lemma: yangian_div_zero_vector_fields_basic_properties}) are:
                $$\sigma_i(D_{r, s}, D_{a, b}) = N_i(r, s, a, b) v^{-r} t^{-s - 1} \bar{d}( v^{-a} t^{-b - 1} )$$
                where:
                    \begin{equation} \label{equation: billig_cocycles_coefficient}
                        N_i(r, s, a, b) =
                        \begin{cases}
                            \text{$ra(2sb + s + b + 1) - ( (sa)^2 + s a^2 ) - ( (rb)^2 + r^2 b )$ if $i = 1$}
                            \\
                            \text{$ra$ if $i = 2$}
                        \end{cases}
                    \end{equation}
                and:
                    $$\sigma_i(D_{r, s}, D_v) = -\delta_{i, 1} r^2 v^{-r} t^{-s - 1} \bar{d}(t^{-1})$$
                    $$\sigma_i(D_{r, s}, D_t) = -\delta_{i, 1} rs v^{-r} t^{-s - 1} \bar{d}(t^{-1})$$
                    $$\sigma_i(D_v, D_t) = 0$$
                \item Furthermore, the restriction of both $\sigma_1$ and $\sigma_2$ to the Lie subalgebra $\frakw \cong \der(\bbC[z^{\pm 1}])$ (cf. proposition \ref{prop: a_copy_of_the_witt_algebra_inside_the_lie_algebra_of_yangian_div_zero_vector_fields}) of $\divzero$ coincides with the $2$-cocycle $\sigma_{\frakv} \in Z^2_{\Lie}(\frakw, \bbC c_v)$ given by:
                    $$\sigma_{\frakv}(D_{r, -1}, D_{a, -1}) := \delta_{r + a, 0} r^3 c_v$$
                for every $r, a \in \Z$.
            \end{enumerate}
        \end{lemma}
            \begin{proof}
                Even though the second statement will turn out to be more-or-less a corollary of the first identity in the first statement, let us nevertheless prove the two claims separately, for the sake of clarity.
                \begin{enumerate}
                    \item
                    \begin{enumerate}
                        \item Firstly, let us compute $\sigma_i(D_{r, s}, D_{a, b})$. For this, consider the following:
                            $$
                                \begin{aligned}
                                    & \sigma_i(D_{r, s}, D_{a, b})
                                    \\
                                    = & \sigma_i( -s v^{-r + 1} t^{-s - 1} \del_v + r v^{-r} t^{-s} \del_t, -b v^{-a + 1} t^{-b - 1} \del_v + a v^{-a} t^{-b} \del_t )
                                    \\
                                    = & \sigma_i( s v^{-r} t^{-s - 1} \cdot v\del_v - r v^{-r} t^{-s - 1} \cdot t \del_t, b v^{-a} t^{-b - 1} \cdot v\del_v - a v^{-a} t^{-b - 1} \cdot t \del_t )
                                    \\
                                    = &
                                        s \sigma_i( v^{-r} t^{-s - 1} \cdot v\del_v, b v^{-a} t^{-b - 1} \cdot v\del_v - a v^{-a} t^{-b - 1} \cdot t \del_t )
                                        \\
                                        & \qquad - r \sigma_i( v^{-r} t^{-s - 1} \cdot t \del_t, b v^{-a} t^{-b - 1} \cdot v\del_v - a v^{-a} t^{-b - 1} \cdot t \del_t )
                                    \\
                                    = &
                                        s b \cdot \sigma_i( v^{-r} t^{-s - 1} \cdot v\del_v, v^{-a} t^{-b - 1} \cdot v\del_v )
                                        \\
                                        & \qquad - s a \cdot \sigma_i( v^{-r} t^{-s - 1} \cdot v\del_v, v^{-a} t^{-b - 1} \cdot t \del_t )
                                        \\
                                        & \qquad - r b \cdot \sigma_i( v^{-r} t^{-s - 1} \cdot t \del_t, v^{-a} t^{-b - 1} \cdot v\del_v )
                                        \\
                                        & \qquad + r a \cdot \sigma_i( v^{-r} t^{-s - 1} \cdot t \del_t, v^{-a} t^{-b - 1} \cdot t \del_t )
                                    \\
                                    = & N_i(r, s, a, b) v^{-r} t^{-s - 1} \bar{d}( v^{-a} t^{-b - 1} )
                                \end{aligned}
                            $$
                        where:
                            $$
                                \begin{aligned}
                                    & N_i(r, s, a, b)
                                    \\
                                    = & 
                                    sbra
                                    - sa \left( \delta_{i, 1} a(s + 1) + \delta_{i, 2} (b + 1) r \right) 
                                    - rb \left( \delta_{i, 1} (b + 1) r + \delta_{i, 2} a (s + 1) \right)
                                    + r a (s + 1) (b + 1)
                                    \\
                                    = & 
                                    \begin{cases}
                                        \text{$
                                            sbra
                                            - s a^2 (s + 1) 
                                            - r^2 b (b + 1)
                                            + r a (s + 1) (b + 1)
                                        $if $i = 1$}
                                        \\
                                        \text{$
                                            sbra
                                            - sa (b + 1) r
                                            - rb a (s + 1)
                                            + r a (s + 1) (b + 1)
                                        $ if $i = 2$}
                                    \end{cases}
                                    \\
                                    = & 
                                    \begin{cases}
                                        \text{$
                                            sbra
                                            - ( (sa)^2 + s a^2 ) 
                                            - ( (rb)^2 + r^2 b ) 
                                            + rasb + rsa + rab + ra
                                        $ if $i = 1$}
                                        \\
                                        \text{$
                                            sbra
                                            - (sabr + sar)
                                            - (rbas + rba)
                                            + rasb + rsa + rab + ra
                                        $ if $i = 2$}
                                    \end{cases}
                                    \\
                                    = & 
                                    \begin{cases}
                                        \text{$2 rsab - ( (sa)^2 + s a^2 ) - ( (rb)^2 + r^2 b ) + rsa + rab + ra$ if $i = 1$}
                                        \\
                                        \text{$ra$ if $i = 2$}
                                    \end{cases}
                                    \\
                                    = &
                                    \begin{cases}
                                        \text{$ra(2sb + s + b + 1) - ( (sa)^2 + s a^2 ) - ( (rb)^2 + r^2 b )$ if $i = 1$}
                                        \\
                                        \text{$ra$ if $i = 2$}
                                    \end{cases}
                                \end{aligned}
                            $$
                        \item Secondly, we shall be computing $\sigma_i(D_v, D_{r, s})$. Consider the following:
                            $$
                                \begin{aligned}
                                    & \sigma_i(D_{r, s}, D_v)
                                    \\
                                    = & \sigma_i( -s v^{-r + 1} t^{-s - 1} \del_v + r v^{-r} t^{-s} \del_t, -v t^{-1} \del_v )
                                    \\
                                    = & \sigma_i( s v^{-r} t^{-s - 1} \cdot v \del_v - r v^{-r} t^{-s - 1} t \del_t, t^{-1} \cdot v \del_v )
                                    \\
                                    = & s \sigma_i( v^{-r} t^{-s - 1} \cdot v \del_v, t^{-1} \cdot v \del_v ) - r\sigma_i( v^{-r} t^{-s - 1} \cdot t \del_t, t^{-1} \cdot v \del_v )
                                    \\
                                    = & \left( s \left( \delta_{i, 1} (-r) \cdot 0 + \delta_{i, 2} (-r) \cdot 0 \right) - r\left( \delta_{i, 1} (-r) \cdot (-1) + \delta_{i, 2} (-s - 1) \cdot 0 \right) \right) v^{-r} t^{-s - 1} \bar{d}(t^{-1}) 
                                    \\
                                    = & -\delta_{i, 1} r^2 v^{-r} t^{-s - 1} \bar{d}(t^{-1})
                                \end{aligned}
                            $$
                        \item Using similar methods as in the previous case, we shall get that:
                            $$\sigma_i(D_{r, s}, D_t) = -\delta_{i, 1} rs v^{-r} t^{-s - 1} \bar{d}(t^{-1})$$
                        \item Finally, we have that:
                            $$\sigma_i(D_v, D_t) = \sigma_i(-v t^{-1} \del_v, -\del_t) = \sigma_i(t^{-1} \cdot v \del_v, t^{-1} t \del_t) = 0$$
                    \end{enumerate}
                    \item Using the fact that any element:
                        $$v^n t^q \bar{d}(v^m t^p) \in \z(\toroidal)$$
                    can be written in terms of the basis elements of $\z(\toroidal)$ in the following manner:
                        $$v^n t^q \bar{d}(v^m t^p) = \delta_{(m, p) + (n, q), (0, 0)} ( n c_v + q c_t ) + (np - mq) K_{m + n, p + q}$$
                    we can rewrite:
                        $$
                            \begin{aligned}
                                & \sigma_i(D_{r, s}, D_{a, b})
                                \\
                                = & N_i(r, s, a, b) \left( ( r(b + 1) - a(s + 1) )K_{-r - a, -s - b - 2} - \delta_{ (-a - r, -b - s - 2), (0, 0) } (r c_v + (s + 1) c_t) \right)
                            \end{aligned}
                        $$
                    Setting $s = b = -1$ in the equation above then yields:
                        $$\sigma_i(D_{r, -1}, D_{a, -1}) = -N_i(r, -1, a, -1) \delta_{r + a, 0} r c_v$$
                    where now, we have that:
                        $$N_i(r, -1, a, -1) = ra$$
                    for both $i = 1$ and $i = 2$, and hence:
                        $$\sigma_i(D_{r, -1}, D_{a, -1}) = \delta_{r + a, 0} r^3 c_v$$
                    as claimed.
                \end{enumerate}
            \end{proof}
        Aside from being a prerequisite for verifying if $\sigma_1$ and/or $\sigma_2$ might be $\gamma$-invariant, lemma \ref{lemma: billig_toroidal_cocycles_on_yangian_div_zero_vector_fields} has the following trivial but important consequence:
        \begin{corollary}[$\sigma_1$ and $\sigma_2$ are $\Z^2$-graded] \label{coro: billig_toroidal_cocycles_are_graded}
            When regarded as linear maps from $\bigwedge^2 \divzero$ to $\z(\toroidal)$, both with their $\Z^2$-gradings (cf. example \ref{example: toroidal_lie_algebras_centres} and corollary \ref{coro: yangian_div_zero_vector_fields_are_graded}), both $\sigma_1$ and $\sigma_2$ as in equation \eqref{equation: billig_toroidal_cocycles} are graded. 
        \end{corollary}
        \begin{remark}
            Despite the way that lemma \ref{lemma: billig_toroidal_cocycles_on_yangian_div_zero_vector_fields} has been stated, which was for the sake of conciseness, the form of the non-trivial values of $\sigma_1$ and $\sigma_2$ on pairs of basis elements of $\divzero$ that we will actually be using in our proofs are as follows:
                $$
                    \begin{aligned}
                        & \sigma_i(D_{r, s}, D_{a, b})
                        \\
                        = & N_i(r, s, a, b) \left( ( r(b + 1) - a(s + 1) )K_{-r - a, -s - b - 2} - \delta_{ (-a - r, -b - s - 2), (0, 0) } (r c_v + (s + 1) c_t) \right)
                    \end{aligned}
                $$
                $$\sigma_i(D_v, D_{r, s}) = \delta_{i, 1} r^3 K_{-r, -s - 2}$$
                $$\sigma_i(D_t, D_{r, s}) = \delta_{i, 1} r^2 s K_{-r, -s - 2}$$
            To prove that these identities hold true, we only need to recall from example \ref{example: toroidal_lie_algebras_centres} that any element:
                $$v^n t^q \bar{d}(v^m t^p) \in \z(\toroidal)$$
            can be written in terms of the basis elements of $\z(\toroidal)$ in the following manner:
                $$v^n t^q \bar{d}(v^m t^p) = \delta_{(m, p) + (n, q), (0, 0)} ( n c_v + q c_t ) + (np - mq) K_{m + n, p + q}$$
        \end{remark}
        \begin{remark}[An appearance of the Virasoro algebra] \label{remark: billig_toroidal_cocycles_on_the_witt_algebra}
            A rather well-known fact (cf. e.g. \cite[Proposition 1.3]{kac_raina_rozhkovskaya_bombay_lectures_on_highest_weight_modules_of_infinite_dimensional_lie_algebras}) is that:
                $$H^2_{\Lie}(\der(\bbC[z^{\pm 1}], \bbC)) \cong \bbC$$
            and hence the Witt algebra admits a non-trivial UCE by a $1$-dimensional centre (cf. theorem \ref{theorem: H^2_of_lie_algebras_and_abelian_extensions}), typically called the \textbf{Virasoro algebra}\footnote{... and hence the notation $\sigma_{\frakv}$.}. This UCE corresponds precisely to the $2$-cocycle $\sigma_{\frakv} \in Z^2_{\Lie}(\frakw, c_v)$ as in lemma \ref{lemma: billig_toroidal_cocycles_on_yangian_div_zero_vector_fields}. 
        \end{remark}

        One can now use the criterion given in proposition \ref{proposition: twisted_semi_direct_products_are_yangian_extended_toroidal_lie_algebras} (see also theorem \ref{theorem: yangian_extended_toroidal_lie_algebras_main_theorem}) to verify whether or not the cocycles $\sigma_1, \sigma_2$ are $\gamma$-invariant in the sense of definition \ref{def: yangian_toroidal_cocycles}.
        \begin{proposition}[$\gamma$-invariance of Billig's toroidal $2$-cocycles] \label{prop: invariance_of_billig_toroidal_cocycles}
            Of the two toroidal $2$-cocycles:
                $$\sigma_1, \sigma_2 \in Z^2_{\Lie}(\divzero, \z(\toroidal))$$
            as in equation \eqref{equation: billig_toroidal_cocycles}, $\sigma_1$ is $\gamma$-invariant in the sense of definition \ref{def: yangian_toroidal_cocycles}, while $\sigma_2$ fails to be so.
        \end{proposition}
            \begin{proof}
                \begin{enumerate}
                    \item Firstly, using the fact that:
                        $$
                            \begin{aligned}
                                & \sigma_i(D_{r, s}, D_{a, b})
                                \\
                                = & N_i(r, s, a, b) \left( ( r(b + 1) - a(s + 1) )K_{-r - a, -s - b - 2} - \delta_{ (-a - r, -b - s - 2), (0, 0) } (r c_v + (s + 1) c_t) \right)
                            \end{aligned}
                        $$
                    we shall get that:
                        $$
                            \left( \sigma_i(D_{r, s}, D_{a, b}), D \right)_{\extendedtoroidal} =
                            \begin{cases}
                                \text{$N_i(r, s, a, b) ( r(b + 1) - a(s + 1) ) \delta_{(-r - a, -s - b - 2), (\alpha, \beta)}$ if $D = D_{\alpha, \beta}$}
                                \\
                                \text{$-N_i(r, s, a, b) \delta_{(r, s), -(a, b)} r$ if $D = D_v$}
                                \\
                                \text{$-N_i(r, s, a, b) \delta_{(r, s), -(a, b)} (s + 1)$ if $D = D_t$}
                            \end{cases}
                        $$
                    At the same time, using the fact that:
                        $$\sigma_i(D_{a, b}, D_v) = -\delta_{i, 1} a^3 K_{-a, -b - 2}$$
                        $$\sigma_i(D_{a, b}, D_t) = - \delta_{i, 1} a^2b K_{-a, -b - 2}$$
                    we have that:
                        $$
                            \begin{aligned}
                                \left( D_{r, s}, \sigma_i(D_{a, b}, D) \right)_{\extendedtoroidal} =
                                \begin{cases}
                                    \text{$N_i(r, s, \alpha, \beta) ( r(\beta + 1) - \alpha(s + 1) ) \delta_{(r, s), (-a - \alpha, -b - \beta - 2)}$ if $D = D_{\alpha, \beta}$}
                                    \\
                                    \text{$-\delta_{i, 1} a^3 \delta_{(r, s), (-a, -b - 2)}$ if $D = D_v$}
                                    \\
                                    \text{$-\delta_{i, 1} a^2 b \delta_{(r, s), (-a, -b - 2)}$ if $D = D_t$}
                                \end{cases}
                            \end{aligned}
                        $$
                    We can thus conclude immediately that $\sigma_2$ is \textit{not} invariant, as:
                        $$\left( \sigma_2(D_{r, s}, D_{a, b}), D \right)_{\extendedtoroidal} \not = \left( D_{r, s}, \sigma_2(D_{a, b}, D) \right)_{\extendedtoroidal}$$
                    when $D \in \{D_v, D_t\}$. As such, let us focus on $\sigma_1$ from now on, for which we now have:
                        $$\left( \sigma_1(D_{r, s}, D_{a, b}), D \right)_{\extendedtoroidal} \not = \left( D_{r, s}, \sigma_1(D_{a, b}, D) \right)_{\extendedtoroidal}$$
                    for all $D \in \divzero$.
                    \item Secondly, using the fact that:
                        $$\sigma_1(D_{r, s}, D_v) = r^3 K_{-r, -s - 2}$$
                    we shall get that:
                        $$
                            \left( \sigma_1(D_{r, s}, D_v), D \right)_{\extendedtoroidal} =
                            \begin{cases}
                                \text{$r^3 \delta_{(-r, -s - 2), (\alpha, \beta)}$ if $D = D_{\alpha, \beta}$}
                                \\
                                \text{$0$ if $D = D_v$}
                                \\
                                \text{$0$ if $D = D_t$}
                            \end{cases}
                        $$
                    At the same time, knowing that:
                        $$\sigma_1(D_v, D_t) = 0$$
                    we see that:
                        $$
                            \begin{aligned}
                                \left( D_{r, s}, \sigma_1(D_v, D) \right)_{\extendedtoroidal} =
                                \begin{cases}
                                    \text{$-\alpha^3 \delta_{(r, s), (-\alpha, -\beta - 2)}$ if $D = D_{\alpha, \beta}$}
                                    \\
                                    \text{$0$ if $D = D_v$}
                                    \\
                                    \text{$0$ if $D = D_t$}
                                \end{cases}
                            \end{aligned}
                        $$
                    We thus have:
                        $$\left( \sigma_1(D_{r, s}, D_v), D \right)_{\extendedtoroidal} = \left( D_{r, s}, \sigma_1(D_v, D) \right)_{\extendedtoroidal}$$
                    for all $D \in \divzero$.
                    \item Next, by using the fact that:
                        $$\sigma_1(D_{r, s}, D_t) = r^2 s K_{-r, -s - 2}$$
                    we shall get that:
                        $$
                            \left( \sigma_1(D_{r, s}, D_t), D \right)_{\extendedtoroidal} =
                            \begin{cases}
                                \text{$r^2 s \delta_{(-r, -s - 2), (\alpha, \beta)}$ if $D = D_{\alpha, \beta}$}
                                \\
                                \text{$0$ if $D = D_v$}
                                \\
                                \text{$0$ if $D = D_t$}
                            \end{cases}
                        $$
                    At the same time, we have that:
                        $$
                            \begin{aligned}
                                \left( D_{r, s}, \sigma_1(D_t, D) \right)_{\extendedtoroidal} =
                                \begin{cases}
                                    \text{$-\alpha^2 \beta \delta_{(r, s), (-\alpha, -\beta - 2)}$ if $D = D_{\alpha, \beta}$}
                                    \\
                                    \text{$0$ if $D = D_v$}
                                    \\
                                    \text{$0$ if $D = D_t$}
                                \end{cases}
                            \end{aligned}
                        $$
                    By combining these two observations, one is able to conclude furthermore that:
                        $$\left( \sigma_1(D_{r, s}, D_t), D \right)_{\extendedtoroidal} = \left( D_{r, s}, \sigma_1(D_t, D) \right)_{\extendedtoroidal}$$
                    for all $D \in \divzero$.
                    \item Lastly, since:
                        $$\sigma_1(D_v, D_t) = 0$$
                    we automatically have that:
                        $$( \sigma_1(D_v, D_t), D )_{\extendedtoroidal} = ( D_v, \sigma_1(D_t, D) )_{\extendedtoroidal}$$
                    for all $D \in \divzero$.
                \end{enumerate}
                We have therefore shown that $\sigma_1$ is $\gamma$-invariant in the sense of definition \ref{def: yangian_toroidal_cocycles}. 
            \end{proof}

    \subsection{Cohomological non-triviality}
        Whether or not the cocycles:
            $$\sigma_1, \sigma_2 \in Z^2_{\Lie}(\divzero, \z(\toroidal))$$
        might be cohomologous to $0$ (cf. definition \ref{def: lie_algebra_cohomology}) - and hence whether or not they might give rise to extensions that are isomorphic to the semi-direct product $\toroidal \rtimes \divzero$ - is a much subtler issue. Remark \ref{remark: billig_toroidal_cocycles_on_the_witt_algebra} in fact does not imply anything about whether or not the images in $H^2_{\Lie}(\divzero, \z(\toroidal))$ of $\sigma_1$ and of $\sigma_2$ might vanish. The subtlety here is that because $\z(\toroidal)$ is non-trivial as a module over $\divzero$ (and likewise, over the Lie subalgebra $\frakw \subset \divzero$), unlike $\bbC c_v$, one would have to actually check whether or not the restricted toroidal $2$-cocycle:
            $$\sigma_i|_{\bigwedge^2 \frakw}: \bigwedge^2 \divzero \to \z(\toroidal)$$
        is $2$-coboundary.

        Now, although in order to check if $\sigma_i$ is $2$-coboundary, it suffices to prove the existence of an element:
            $$\tau_i \in C_1(\divzero, \z(\toroidal))$$
        (i.e. a linear map; cf. remark \ref{remark: simplified_chevalley_eilenberg_complexes}) such that:
            $$\tau_i([D, D']) = [D, \tau_i(D')]_{\extendedtoroidal} - [D', \tau_i(D)]_{\extendedtoroidal} - \sigma_i(D, D')$$
        we shall prove a stronger statement, namely the existence of $\tau_i \in C_1(\divzero, \z(\toroidal))$ that is \textit{graded} with respect to the $\Z^2$-gradings on $\divzero$ and $\z(\toroidal)$. We will do this by firstly assuming that such a graded $\tau_i$ exists, deriving an explicit closed-form formula for it, and then verifying whether or not $d_1^{\z(\toroidal)}(\tau_i) = \sigma_i$. Our final result, proposition \ref{prop: cohomological_non_triviality_of_billig_toroidal_cocycles} will as such be proven firstly through the use of proposition \ref{prop: sigma_1_is_not_coboundary} below.
        \begin{remark}[Some recollection]
            The proofs of propositions \ref{prop: sigma_1_is_not_coboundary} and \ref{prop: cohomological_non_triviality_of_billig_toroidal_cocycles} down below are somewhat computation-heavy, so for the sake of clarity and convenience, let us recall for the proof the following facts that have been known since earlier in this chapter.
            \begin{itemize}
                \item Firstly, recall from example \ref{example: toroidal_lie_algebras_centres} and corollary \ref{coro: yangian_div_zero_vector_fields_are_graded} that:
                    $$\deg D_{-a, -b - 1} = \deg K_{a, b} = (a, b)$$
                    $$\deg D_v = \deg D_t = (0, -1)$$
                    $$\deg c_v = \deg c_t = (0, 0)$$
                \item Secondly, let us recall from lemma \ref{lemma: yangian_div_zero_vector_fields_basic_properties} that:
                    $$[D_v, D_{r, s}] = r D_{r, s + 1}$$
                    $$[D_t, D_{r, s}] = s D_{r, s + 1}$$
                    $$[D_{a, b}, D_{r, s}] = (br - sa) D_{a + r, b + s + 1}$$
                \item Thirdly, from lemma \ref{lemma: explicit_commutators_between_central_basis_elements_and_derivations}, we know that:
                    $$[D, K_{\alpha, \beta}]_{\extendedtoroidal} =
                        \begin{cases}
                            \text{$((\beta - 1)a - b\alpha) K_{\alpha - a, \beta - b - 1} + \delta_{(a, b + 1), (\alpha, \beta)} \left( a c_v + b c_t \right)$ if $D = D_{a, b}$}
                            \\
                            \text{$\alpha K_{\alpha, \beta - 1}$ if $D = D_v$}
                            \\
                            \text{$\beta K_{\alpha, \beta - 1}$ if $D = D_t$}
                        \end{cases}
                    $$
                    $$[D, c_v]_{\extendedtoroidal} = [D, c_t]_{\extendedtoroidal} = 0$$
                for all $D \in \divzero$.
                \item Finally, we will need to know how $\sigma_1$ and $\sigma_2$ are given explicitly on the basis elements of $\divzero$. For this, we refer the reader to lemma \ref{lemma: billig_toroidal_cocycles_on_yangian_div_zero_vector_fields} at the top of this section.
            \end{itemize}
        \end{remark}
        \begin{proposition} \label{prop: sigma_1_is_not_coboundary}
            The toroidal $2$-cocycle $\sigma_1 \in Z^2_{\Lie}(\divzero, \z(\toroidal) )$ as in equation \eqref{equation: billig_toroidal_cocycles} is \textit{not} $2$-coboundary, i.e. its image under the canonical projection $Z^2_{\Lie}(\divzero, \z(\toroidal) ) \to H^2_{\Lie}(\divzero, \z(\toroidal) )$ is non-zero.
        \end{proposition}
            \begin{proof}
                Let us suppose that $\sigma_i$ (for either $i = 1$ or $i = 2$, or perhaps even neither) is $2$-coboundary, say:
                    $$d_1^{\z(\toroidal)}(\tau_i) = \sigma_i$$
                for some $\tau_i \in C_1(\divzero, \z(\toroidal))$. Recall firstly, from example \ref{example: toroidal_lie_algebras_centres} and corollary \ref{coro: yangian_div_zero_vector_fields_are_graded}, respectively, that there are $\Z^2$-gradings on $\z(\toroidal)$ and $\divzero$. Secondly, recall from corollary \ref{coro: billig_toroidal_cocycles_are_graded} that $\sigma_i$ is graded with respect to these $\Z^2$-gradings. Thirdly, recall from lemma \ref{lemma: explicit_commutators_between_central_basis_elements_and_derivations} (cf. corollary \ref{coro: a_fixed_yangian_div_zero_vector_field_action}) that the $\divzero$ acts homogeneously on $\z(\toroidal)$. We can therefore apply lemma \ref{lemma: graded_2_coboundaries}, which tells us that we can assume without any loss of generality that $\tau_i$ is a graded linear map (with respect to the $\Z^2$-gradings on $\divzero$ and on $\z(\toroidal)$).
            
                Assume thus that $\tau_i$ is graded. Firstly, this assumption  implies that there are coefficients:
                    $$\lambda_{r, s}, \mu, \kappa \in \bbC$$
                such that:
                    $$\tau_i(D_{r, s}) = \lambda_{r, s} K_{-r, -s - 1} + \delta_{(r, s), (0, -1)} ( \mu c_v + \kappa c_t )$$
                whenever $(r, s) \not = (0, 0)$, where the second equality holds because $K_{0, 0} = 0$ (cf. example \ref{example: toroidal_lie_algebras_centres}). Likewise, there shall be coefficients:
                    $$\nu_v, \nu_t \in \bbC$$
                such that:
                    $$\tau_i(D_v) = \nu_v K_{0, -1}$$
                    $$\tau_i(D_t) = \nu_t K_{0, -1}$$

                We attempt first of all to compute $\tau_i(D_{r, s})$. For this, we refer the reader to lemma \ref{lemma: billig_toroidal_cocycles_on_yangian_div_zero_vector_fields} for the following fact:
                    $$\sigma_i(D_v, D_{r, s}) = -\delta_{i, 1} r^3 K_{-r, -s - 2}$$
                This gives the following:
                    \begin{equation} \label{equation: tau_i_on_commutator_of_D_v_and_D_rs}
                        \begin{aligned}
                            & r \tau_i(D_{r, s + 1})
                            \\
                            = & \tau_i([D_v, D_{r, s}])
                            \\
                            = & [D_v, \tau_i(D_{r, s})]_{\extendedtoroidal} - [D_{r, s}, \tau_i(D_v)]_{\extendedtoroidal} - \sigma_i(D_v, D_{r, s})
                            \\
                            = & [D_v, \lambda_{r, s} K_{-r, -s - 1}]_{\extendedtoroidal} - [D_{r, s}, \nu_v K_{0, -1}]_{\extendedtoroidal} + \delta_{i, 1} r^3 K_{-r, -s - 2}
                            \\
                            = & r \lambda_{r, s} K_{-r, -s - 2} + \nu_v \left( -2r K_{-r, -s - 2} + \delta_{(r, s + 1), (0, -1)} \left( r c_v + s c_t \right) \right) + \delta_{i, 1} r^3 K_{-r, -s - 2}
                            \\
                            = & \left( r \lambda_{r, s} - 2r \nu_v + \delta_{i, 1} r^3 \right) K_{-r, -s - 2} + \nu_v \delta_{(r, s + 2), (0, 0)} \left( r c_v + s c_t \right)
                        \end{aligned}
                    \end{equation}
                From this, we infer that when $r \not = 0$, we have that:
                    $$\tau_i(D_{r, s + 1}) = \left( \lambda_{r, s} - 2\nu_v + \delta_{i, 1} r^2 \right) K_{-r, -s - 2}$$
                but at the same time, we have the following per our initial assumption that $\tau_i$ is graded:
                    $$\tau_i(D_{r, s + 1}) = \lambda_{r, s + 1} K_{-r, -s - 2} + \delta_{(r, s + 1), (0, -1)} ( \alpha_{r, s + 1} c_v + \beta_{r, s + 1} c_t )$$
                and so we have:
                    $$\lambda_{r, s + 1} = \lambda_{r, s} - 2\nu_v + \delta_{i, 1} r^2$$
                whenever $r \not = 0$ (in which case $\delta_{(r, s + 1), (0, -1)} = 0$). This recursive formula is equivalent to the following, valid for all $(r, s) \in \Z^2 \setminus ( \{0\} \x \Z )$:
                    \begin{equation} \label{equation: lambda_rs_coefficients_recursion}
                        \lambda_{r, s} = \lambda_{r, 0} + (-2\nu_v + \delta_{i, 1} r^2) s
                    \end{equation}
                From this, one infers that in order to determined $\lambda_{r, s}$ when $r \not = 0$, it suffices to only determine $\lambda_{r, 0}$.
                
                Next, consider the following:
                    $$
                        \begin{aligned}
                            & (br - sa) \tau_i(D_{a + r, b + s + 1})
                            \\
                            = & \tau_i( [D_{a, b}, D_{r, s}] )
                            \\
                            = & [D_{a, b}, \tau_i(D_{r, s})]_{\extendedtoroidal} - [D_{r, s}, \tau_i(D_{a, b})]_{\extendedtoroidal} - \sigma_i(D_{a, b}, D_{r, s})
                            \\
                            = & \lambda_{r, s} [D_{a, b}, K_{-r, -s - 1}]_{\extendedtoroidal} - \lambda_{a, b} [D_{r, s}, K_{-a, -b - 1}]_{\extendedtoroidal} + \sigma_i(D_{r, s}, D_{a, b})
                        \end{aligned}
                    $$
                Using lemma \ref{lemma: explicit_commutators_between_central_basis_elements_and_derivations}, we get that:
                    $$[D_{a, b}, K_{-r, -s - 1}]_{\extendedtoroidal} = -\left( a(s + 2) - br \right) K_{-a - r, -b - s - 2} + \delta_{(a + r, b + s + 2), (0, 0)} \left( a c_v + b c_t \right)$$
                    $$[D_{r, s}, K_{-a, -b - 1}]_{\extendedtoroidal} = -\left( (b + 2) r - as \right) K_{-a - r, -b - s - 2} + \delta_{(a + r, b + s + 2), (0, 0)} \left( r c_v + s c_t \right)$$
                and from lemma \ref{lemma: billig_toroidal_cocycles_on_yangian_div_zero_vector_fields}, we know that:
                    $$\sigma_i(D_{r, s}, D_{a, b}) = N_i(r, s, a, b) \left( ( r(b + 1) - a(s + 1) )K_{-r - a, -s - b - 2} - \delta_{ (a + r, b + s + 2), (0, 0) } (r c_v + (s + 1) c_t) \right)$$
                where $N_i(r, s, a, b)$ is as in \textit{loc. cit.} Simultaneously, these fact imply that:
                    \begin{equation} \label{equation: coboundary_equation_D_rs_D_ab}
                        \begin{aligned}
                            & (br - sa) \tau_i(D_{a + r, b + s + 1})
                            \\
                            = & (br - sa) \left( \lambda_{a + r, b + s + 1} K_{-a - r, -b - s - 2} + \delta_{(a + r, b + s + 1), (0, -1)}( \mu c_v + \kappa c_t ) \right)
                            \\
                            = &
                                \left( -\lambda_{r, s} \left( a(s + 2) - br \right) + \lambda_{a, b} \left( (b + 2) r - as \right) + N_i(r, s, a, b)\left( r(b + 1) - a(s + 1) \right) \right) K_{-a - r, -b - s - 2}
                                \\
                                & \qquad + \delta_{(a + r, b + s + 2), (0, 0)} \left( \lambda_{r, s} a - \lambda_{a, b} r - N_i(r, s, a, b) r \right) c_v
                                \\
                                & \qquad + \delta_{(a + r, b + s + 2), (0, 0)} \left( \lambda_{r, s} b - \lambda_{a, b} s - N_i(r, s, a, b) (s + 1) \right) c_t
                            \\
                            = &
                                \left( -\lambda_{r, s} \left( a(s + 2) - br \right) + \lambda_{a, b} \left( (b + 2) r - as \right) + N_i(r, s, a, b)\left( r(b + 1) - a(s + 1) \right) \right) K_{-a - r, -b - s - 2}
                                \\
                                & \qquad - \delta_{(a + r, b + s + 2), (0, 0)} r \left( \lambda_{r, s} + \lambda_{-r, -s - 2} + N_i(r, s, -r, -s - 2) \right) c_v
                                \\
                                & \qquad - \delta_{(a + r, b + s + 2), (0, 0)} \left( \lambda_{r, s} (s + 2) + \lambda_{-r, -s - 2} s + N_i(r, s, -r, -s - 2) (s + 1) \right) c_t
                        \end{aligned}
                    \end{equation}
                from which it can be inferred - by comparing the coefficients of $K_{-a - r, -b - s - 2}$ - that:
                    \begin{equation} \label{equation: lambda_rs_coefficients}
                        \begin{aligned}
                            & (br - sa) \lambda_{a + r, b + s + 1}
                            \\
                            = & -\lambda_{r, s} \left( a(s + 2) - br \right) + \lambda_{a, b} \left( (b + 2) r - as \right) + N_i(r, s, a, b)\left( r(b + 1) - a(s + 1) \right)
                        \end{aligned}
                    \end{equation}
                As mentioned above (cf. equation \eqref{equation: lambda_rs_coefficients_recursion}), we would like to determine $\lambda_{r, 0}$ when $r \not = 0$. To this end, let us evaluate equation \eqref{equation: lambda_rs_coefficients} at:
                    $$b = 0, s = -1$$
                Doing so yields:
                    $$
                        \begin{aligned}
                            & a \lambda_{a + r, 0}
                            \\
                            = & -\lambda_{r, -1} a + \lambda_{a, 0} \left( 2r + a \right) + N_i(r, -1, a, 0) r
                            \\
                            = & -(\lambda_{r, 0} - \delta_{i, 1} r^2) a + \lambda_{a, 0} \left( 2r + a \right) + N_i(r, -1, a, 0) r
                        \end{aligned}
                    $$
                When $r + a = 0$, this will turn into:
                    $$0 = -r \lambda_{0, 0} = (\lambda_{r, 0} - \delta_{i, 1} r^2) r + \lambda_{-r, 0} r + N_i(r, -1, -r, 0) r$$
                and since the current assumption is that $r \not = 0$, the above is equivalent to:
                    $$
                        \begin{aligned}
                            & 0
                            \\
                            = & (\lambda_{r, 0} - \delta_{i, 1} r^2) + \lambda_{-r, 0} + N_i(r, -1, -r, 0)
                            \\
                            = & (\lambda_{r, 0} + \lambda_{-r, 0}) - \delta_{i, 1} r^2 + N_i(r, -1, -r, 0)
                        \end{aligned}
                    $$
                and hence:
                    $$
                        \begin{aligned}
                            & \lambda_{r, 0} + \lambda_{-r, 0}
                            \\
                            = & \delta_{i, 1} r^2 - N_i(r, -1, -r, 0)
                            \\
                            = & \delta_{i, 1} r^2 + \delta_{i, 2} r^2
                        \end{aligned}
                    $$
                (to see why the second equality holds, see equation \eqref{equation: billig_cocycles_coefficient}, from which one sees that $N_1(r, -1, -r, 0) = 0$ and $N_2(r, -1, -r, 0) = -r^2$). Using equation \eqref{equation: lambda_rs_coefficients_recursion}, we then get that:
                    $$
                        \begin{aligned}
                            & \lambda_{r, -1} + \lambda_{-r, -1}
                            \\
                            = & ( \lambda_{r, 0} + \lambda_{-r, 0} ) - 2(-2\nu_v + \delta_{i, 1} r^2)
                            \\
                            = & ( \delta_{i, 1} r^2 + \delta_{i, 2} r^2 ) - 2\delta_{i, 1} r^2 + 4\nu_v
                            \\
                            = & \delta_{i, 1} r^2 - \delta_{i, 2} r^2 + 4\nu_v
                            \\
                            = & (-1)^{\delta_{i, 1}} r^2 + 4\nu_v
                        \end{aligned}
                    $$
                Next, let us compute $\lambda_{r, -1} - \lambda_{-r, -1}$ (and during the process, we will also be able to compute the coefficients $\mu, \kappa$), so that we can establish a linear system in the variables $\lambda_{\pm r, -1}$. To this end, let us evaluate equation \eqref{equation: coboundary_equation_D_rs_D_ab} when $r + a = 0$ and $b = s = -1$. Doing so yields:
                    $$
                        \begin{aligned}
                            & 2r \tau_i(D_{0, -1})
                            \\
                            = & r\left( \lambda_{r, -1} + \lambda_{-r, -1} + N_i(r, -1, -r, -1) \right) c_v + \left( \lambda_{r, -1} - \lambda_{-r, -1} \right) c_t
                            \\
                            = & r\left( \lambda_{r, -1} + \lambda_{-r, -1} - r^2 \right) c_v + \left( \lambda_{r, -1} - \lambda_{-r, -1} \right) c_t
                            \\
                            = & r\left( (-1)^{\delta_{i, 1}} r^2 + 4\nu_v - r^2 \right) c_v + \left( \lambda_{r, -1} - \lambda_{-r, -1} \right) c_t
                            \\
                            = & r\left( ((-1)^{\delta_{i, 1}} - 1) r^2 + 4\nu_v \right) c_v + \left( \lambda_{r, -1} - \lambda_{-r, -1} \right) c_t
                        \end{aligned}
                    $$
                (and note that from equation \eqref{equation: billig_cocycles_coefficient}, it can be inferred that $N_1(r, -1, -r, -1) = N_2(r, -1, -r, -1) = -r^2$, which gives the second equality). At the same time, we have per the graded-ness assumption on $\tau_i$ that:
                    $$\tau_i(D_{0, -1}) = \mu c_v + \kappa c_t$$
                Neither $\mu, \kappa \in \bbC$ depends on $r$, so we must have the following, by simplying comparing the coefficients of $c_v$ and of $c_t$:
                    $$\lambda_{r, -1} + \lambda_{-r, -1} - r^2 = (-1)^{\delta_{i, 2}} r^2 + 4\nu_v = 2\mu$$
                    $$\lambda_{r, -1} - \lambda_{-r, -1} = 2r\kappa$$
                Observe now that:
                    $$(-1)^{\delta_{i, 1}} - 1 = -2\delta_{i, 1}$$
                meaning that only when $i \not = 1$ (so $i = 2$ for us) is the expression $((-1)^{\delta_{i, 1}} - 1) r^2$ \textit{independent} of $r$, which is what we would like to have, since $\mu$ is not dependent on $r$. As such, we can immediately rule out the case $i = 1$. This proves our claim that $\sigma_1 \in Z^2_{\Lie}(\divzero, \z(\toroidal) )$ is \textit{not} $2$-coboundary.
            \end{proof}
        \begin{corollary}[A non-trivial $\gamma$-extended toroidal Lie algebra]
            The twisted semi-direct product $\toroidal \rtimes^{\sigma_1} \divzero$ is an example of a $\gamma$-extended toroidal Lie algebra that is \textit{not} isomorphic to the semi-direct product $\toroidal \rtimes \divzero$.
        \end{corollary}
            \begin{proof}
                Combine proposition \ref{prop: sigma_1_is_not_coboundary} with the fact that $\sigma_1$ is $\gamma$-invariant, shown in proposition \ref{prop: invariance_of_billig_toroidal_cocycles}.
            \end{proof}

        The following result is our final conclusion regarding whether or not the toroidal $2$-cocycles $\sigma_1$ and $\sigma_2$ are cohomologous to $0$.
        \begin{proposition} \label{prop: cohomological_non_triviality_of_billig_toroidal_cocycles}
            The toroidal $2$-cocycle $\sigma_2 \in Z^2_{\Lie}(\divzero, \z(\toroidal) )$ as in equation \eqref{equation: billig_toroidal_cocycles} is $2$-coboundary. Furthermore, there exists a one-parameter family $\{\tau_2^{\kappa}\}_{\kappa \in \bbC}$ of graded linear maps $\tau_2^{\kappa}: \divzero \to \z(\toroidal)$ such that $d_1^{\z(\toroidal)}(\tau_2^{\kappa}) = \sigma_2$. Each $\tau_2^{\kappa}$ is given by:
                $$\tau_2^{\kappa}(D_{r, s}) = \left( \frac12 r^2 + r\kappa \right) K_{-r, -s - 2} + \delta_{(r, s), (0, -1)} \kappa c_t$$
                $$\tau_2^{\kappa}(D_v) = 0$$
                $$\tau_2^{\kappa}(D_t) = \kappa K_{0, -1}$$
        \end{proposition}
            \begin{proof}
                We will prove this proposition by firstly pre-supposing that $\sigma_2$ is $2$-coboundary and then identifying explicit formulae for $1$-cochains $\tau_2^{\kappa} \in C_1(\divzero, \z(\toroidal))$ such that $d_1^{\z(\toroidal)}(\tau_2^{\kappa}) = \sigma_2$; along the way, we will also see that such linear maps $\tau_2^{\kappa}$ are indeed parametrised by $\kappa \in \bbC$, hence the notation. Afterwards, we will check that such $1$-cochains indeed give rise to a Lie $2$-coboundary that is equal to $\sigma_2$ by verifying that equation \eqref{equation: coboundary_verification_tau_2} holds.
                
                To this end, we begin by letting:
                    $$\tau_2 \in C_1(\divzero, \z(\toroidal))$$
                be graded, as in the proof of proposition \ref{prop: sigma_1_is_not_coboundary} (and we refer the reader there for an explanation of why $\tau_2$ can be assumed to be graded; see also lemma \ref{lemma: graded_2_coboundaries}), and by letting:
                    $$\lambda_{r, s}, \mu, \kappa, \nu_v, \nu_t \in \bbC$$
                also be as in the same proof. Recall also from \textit{loc. cit.} that we have arrived at the equation:
                    $$\lambda_{r, -1} + \lambda_{-r, -1} - r^2 = (-1)^{\delta_{i, 2}} r^2 + 4\nu_v = 2\mu$$
                which gives:
                    $$\mu = 2\nu_v$$
                when $i = 2$, which is what we are currerntly working with.
                
                We thus have the following linear system:
                    $$
                        \begin{cases}
                            \lambda_{r, -1} + \lambda_{-r, -1} = r^2 + 2\mu = r^2 + 2 \nu_v
                            \\
                            \lambda_{r, -1} - \lambda_{-r, -1} = 2r\kappa
                        \end{cases}
                    $$
                which can be solved to yield:
                    $$\lambda_{r, -1} = \frac12 r^2 + r\kappa + \mu = \frac12 r^2 + r\kappa + 2\nu_v$$
                Plugging this into equation \eqref{equation: lambda_rs_coefficients_recursion} then yields the following whenever $r \not = 0$:
                    $$
                        \begin{aligned}
                            & \lambda_{r, s}
                            \\
                            = & \lambda_{r, 0} - 2\nu_v s
                            \\
                            = & \lambda_{r, -1} - 2 \nu_v (s + 1)
                            \\
                            = & \left( \frac12 r^2 + r\kappa + 2\nu_v \right) - 2 \nu_v (s + 1)
                            \\
                            = & \frac12 r^2 + r\kappa - 2s \nu_v
                        \end{aligned} 
                    $$
                with no constrains on $\kappa \in \bbC$. 

                Next, let us verify whether there might be any constraint on either $\nu_v$ or $\nu_t$. Observe next that when $r = 0$, we have that:
                    $$0 = r \tau_2(D_{r, s + 1}) = \nu_v \delta_{s, -2} sc_t$$
                (cf. equation \eqref{equation: tau_i_on_commutator_of_D_v_and_D_rs}) and hence it is necessary that:
                    $$\nu_v = 0$$
                To determine $\nu_t$, recall from lemma \ref{lemma: billig_toroidal_cocycles_on_yangian_div_zero_vector_fields} that:
                    $$\sigma_2(D_t, D_{r, s}) = 0$$
                and hence:
                    $$
                        \begin{aligned}
                            & s \tau_2(D_{r, s + 1})
                            \\
                            = & \tau_2([D_t, D_{r, s}])
                            \\
                            = & [D_t, \tau_2(D_{r, s})]_{\extendedtoroidal} - [D_{r, s}, \tau_2(D_t)]_{\extendedtoroidal} 
                            \\
                            = & [D_t, \lambda_{r, s} K_{-r, -s - 1}]_{\extendedtoroidal} - [D_{r, s}, \nu_t K_{0, -1}]_{\extendedtoroidal}
                            \\
                            = & (s + 1) \lambda_{r, s} K_{-r, -s - 2} + \nu_t \left( -2r K_{-r, -s - 2} + \delta_{(r, s + 1), (0, -1)} \left( r c_v + s c_t \right) \right)
                            \\
                            = & \left( (s + 1) \lambda_{r, s} - 2r \nu_t \right) K_{-r, -s - 2} + \nu_t \delta_{(r, s + 2), (0, 0)} \left( r c_v + s c_t \right)
                        \end{aligned}
                    $$
                If $(r, s) = (0, -2)$ then the above will imply that:
                    $$-2\tau_2(D_{0, -1}) = -2\nu_t c_t$$
                But at the same time, we have that:
                    $$\tau_2(D_{0, -1}) = \mu c_v + \kappa c_t = -2\nu_v c_v + \kappa c_t = \kappa c_t$$
                and hence:
                    $$\nu_t = \kappa$$
                    
                Therefore, we can conclude that:
                    \begin{equation} \label{equation: lambda_rs_formula}
                        \lambda_{r, s} = \frac12 r^2 + r\kappa
                    \end{equation}
                    $$\nu_v = 0$$
                    $$\nu_t = \kappa$$
                and hence we have found a one-parameter family $\{\tau_2^{\kappa}\}_{\kappa \in \bbC}$ of graded linear maps $\tau_2^{\kappa}: \divzero \to \z(\toroidal)$ such that $d_1^{\z(\toroidal)}(\tau_2^{\kappa}) = \sigma_2$.
            
                Now, it remains to check whether or not the $2$-boundary equation:
                    \begin{equation} \label{equation: coboundary_verification_tau_2}
                        \tau_2^{\kappa}([D, D']) = [D, \tau_2^{\kappa}(D')]_{\extendedtoroidal} - [D', \tau_2^{\kappa}(D)]_{\extendedtoroidal} - \sigma_2(D, D')
                    \end{equation}
                is satisfied for all $D, D' \in \divzero$. Without loss of generality, we can assume again that $D, D' \in \divzero$ are basis elements.
                \begin{enumerate}
                    \item Firstly, let us verify if it is true that:
                        \begin{equation} \label{equation: coboundary_equation_D_rs_D_ab_verification}
                            \tau_2^{\kappa}([D_{a, b}, D_{r, s}]) = [D_{a, b}, \tau_2^{\kappa}(D_{r, s})]_{\extendedtoroidal} - [D_{r, s}, \tau_2^{\kappa}(D_{a, b})]_{\extendedtoroidal} - \sigma_2(D_{a, b}, D_{r, s})
                        \end{equation}
                    We will perform a case-by-case verification, as in each of these cases, certain terms in equation \eqref{equation: coboundary_equation_D_rs_D_ab} (and hence in equation \eqref{equation: coboundary_equation_D_rs_D_ab_verification}) would vanish.
                    \begin{enumerate}
                        \item To begin, not that when $a = r = 0$, the LHS and the RHS of equation \eqref{equation: coboundary_equation_D_rs_D_ab_verification} will both be $0$ (cf. equation \eqref{equation: coboundary_equation_D_rs_D_ab}), and hence equal to one another.
                        \item When $a + r \not = 0$ and $a, r \not = 0$, the LHS of equation \eqref{equation: coboundary_equation_D_rs_D_ab_verification} will become:
                            $$
                                \begin{aligned}
                                    & \tau_2^{\kappa}([D_{a, b}, D_{r, s}])
                                    \\
                                    = & (br - sa) \tau_2^{\kappa}(D_{a + r, b + s + 1})
                                    \\
                                    = & (br - sa) \left( \frac12 (a + r)^2 + (a + r)\kappa \right) K_{-a - r, -b - s - 1}
                                \end{aligned}
                            $$
                        while the RHS will become:
                            $$
                                \begin{aligned}
                                    & [D_{a, b}, \tau_2^{\kappa}(D_{r, s})]_{\extendedtoroidal} - [D_{r, s}, \tau_2^{\kappa}(D_{a, b})]_{\extendedtoroidal} - \sigma_2(D_{a, b}, D_{r, s})
                                    \\
                                    = & \left( -\lambda_{r, s} \left( a(s + 2) - br \right) + \lambda_{a, b} \left( (b + 2) r - as \right) + N_2(r, s, a, b)\left( r(b + 1) - a(s + 1) \right) \right) K_{-a - r, -b - s - 2}
                                    \\
                                    = &
                                        -\left( \frac12 r^2 + r\kappa \right) \left( a(s + 2) - br \right) K_{-a - r, -b - s - 2}
                                        \\
                                        & \qquad + \left( \frac12 a^2 + a\kappa \right) \left( (b + 2) r - as \right) K_{-a - r, -b - s - 2} 
                                        \\
                                        & \qquad + ra\left( r(b + 1) - a(s + 1) \right) K_{-a - r, -b - s - 2}
                                \end{aligned}
                            $$
                        which can now be easily checked to be equal to the LHS.
                        \item When $r + a = 0$ but $b + s + 2 \not = 0$, the LHS of equation \eqref{equation: coboundary_equation_D_rs_D_ab_verification} will be equal to:
                            $$
                                \begin{aligned}
                                    & \tau_2^{\kappa}([D_{-r, b}, D_{r, s}])
                                    \\
                                    = & \left( \lambda_{r, s} \left( r(s + 2) + br \right) + \lambda_{-r, b} \left( (b + 2) r + rs \right) + N_2(r, s, -r, b)\left( r(b + 1) + r(s + 1) \right) \right) K_{0, -b - s - 2}
                                    \\
                                    = & r (b + s + 2) \left( (\lambda_{r, s} + \lambda_{r, b}) - r^2 \right) K_{0, -b - s - 2}
                                \end{aligned}
                            $$
                        while the RHS will become:
                            $$
                                \begin{aligned}
                                    & [D_{-r, b}, \tau_2^{\kappa}(D_{r, s})]_{\extendedtoroidal} - [D_{r, s}, \tau_2^{\kappa}(D_{-r, b})]_{\extendedtoroidal} - \sigma_2(D_{-r, b}, D_{r, s})
                                    \\
                                    = & r (b + s + 2) \left( r^2 - \kappa + N_2(r, s, -r, b) \right) K_{0, -b - s - 2}
                                    \\
                                    = & r (b + s + 2) \left( r^2 - \kappa - r^2 \right) K_{0, -b - s - 2}
                                    \\
                                    = & -r (b + s + 2) \kappa K_{0, -b - s - 2}
                                \end{aligned}
                            $$
                        For equation \eqref{equation: coboundary_equation_D_rs_D_ab_verification} to be true, we must then have that:
                            $$\lambda_{r, s} + \lambda_{r, b} - r^2 = -\kappa$$
                        Because we have covered the case $a = r = 0$ (which does give $a + r = 0$), let us assume now that $r \not = 0$. In that case, the equation above will become:
                            $$r^2 - \kappa - r^2 = -\kappa$$
                        which is clearly true.
                        \item Lastly, consider the case where $a + r = 0$ and $b + s + 2 = 0$. In this case, the LHS of equation \eqref{equation: coboundary_equation_D_rs_D_ab_verification} will become:
                            $$-2r \tau_2^{\kappa}(D_{0, -1}) = -2r \kappa c_t$$
                        while the RHS will become:
                            $$
                                \begin{aligned}
                                    & [D_{-r, -s - 2}, \tau_2^{\kappa}(D_{r, s})]_{\extendedtoroidal} - [D_{r, s}, \tau_2^{\kappa}(D_{-r, -s - 2})]_{\extendedtoroidal} - \sigma_2(D_{-r, -s - 2}, D_{r, s})
                                    \\
                                    = &
                                        -r\left( \lambda_{r, s} + \lambda_{-r, -s - 2} + N_2(r, s, -r, -s - 2) \right) c_v
                                        \\
                                        & \qquad - \left( \lambda_{r, s} (s + 2) + \lambda_{-r, -s - 2} s + N_2(r, s, -r, -s - 2) (s + 1) \right) c_t
                                    \\
                                    = &
                                        -r \left( \left( \frac12 r^2 + r \kappa \right) + \left( \frac12 r^2 - r \kappa \right) - r^2\right) c_v
                                        \\
                                        & \qquad - \left( \left( \frac12 r^2 + r \kappa \right) (s + 2) + \left( \frac12 r^2 - r \kappa \right) s - r^2 (s + 1) \right) c_t
                                    \\
                                    = & -2r \kappa c_t
                                \end{aligned}
                            $$
                        so clearly the LHS and RHS are equal to one another.
                    \end{enumerate}
                    \item Next, recall from lemma \ref{lemma: billig_toroidal_cocycles_on_yangian_div_zero_vector_fields} that:
                        $$\sigma_2(D_v, D_{r, s}) = 0$$
                    and then consider the following for $r \not = 0$ (the case $r = 0$ is trivial):
                        $$
                            \begin{aligned}
                                & \tau_2^{\kappa}([D_v, D_{r, s}])
                                \\
                                = & r\tau_2^{\kappa}(D_{r, s + 1})
                                \\
                                = & r \left(\frac12 r^2 + r \kappa \right) K_{-r, -s - 2}
                            \end{aligned}
                        $$
                    along with the following (also for $r \not = 0$):
                        $$
                            \begin{aligned}
                                & [D_v, \tau_2^{\kappa}(D_{r, s})]_{\extendedtoroidal} - [D_{r, s}, \tau_2^{\kappa}(D_v)]_{\extendedtoroidal}
                                \\
                                = & \left(\frac12 r^2 + r \kappa \right)[D_v, K_{-r, -s - 1}]_{\extendedtoroidal}
                                \\
                                = & r \left(\frac12 r^2 + r \kappa \right) K_{-r, -s - 2}
                            \end{aligned}
                        $$
                    and hence the LHS and RHS of equation \eqref{equation: coboundary_verification_tau_2} are indeed equal when $D = D_v$ and $D' = D_{r, s}$.
                    \item The same method as above can be used to show that the LHS and RHS of equation \eqref{equation: coboundary_verification_tau_2} are indeed equal when $D = D_t$ and $D' = D_{r, s}$.
                    \item Finally, let us verify that the LHS and RHS of equation \eqref{equation: coboundary_verification_tau_2} are indeed equal when $D = D_v$ and $D' = D_t$. Since:
                        $$[D_v, D_t] = -D_{0, 1}$$
                    the LHS of said equation will vanish:
                        $$\tau_2^{\kappa}([D_v, D_t]) = -\tau_2^{\kappa}(D_{0, 1}) = 0$$
                    Recall also from lemma \ref{lemma: billig_toroidal_cocycles_on_yangian_div_zero_vector_fields} that:
                        $$\sigma_2(D_v, D_t) = 0$$
                    and therefore it suffices to check if it is true that:
                        $$[D_v, \tau_2^{\kappa}(D_t)]_{\extendedtoroidal} - [D_t, \tau_2^{\kappa}(D_v)]_{\extendedtoroidal} = 0$$
                    From proposition \ref{prop: sigma_1_is_not_coboundary}, we know that:
                        $$\tau_2^{\kappa}(D_v) = 0, \tau_2^{\kappa}(D_t) = \kappa K_{0, -1}$$
                    so the LHS of the equation above reduces down to:
                        $$\kappa [D_v, K_{0, -1}]_{\extendedtoroidal}$$
                    Since $[D_v, K_{0, -1}]_{\extendedtoroidal} = 0 \cdot K_{0, -2} = 0$, the above is certainly equal to $0$, as needed.
                \end{enumerate}
                In conclusion, we have shown that $\sigma_2$ is indeed $2$-coboundary and hence cohomologous to $0$.
            \end{proof}
        \begin{remark}
            Through proposition \ref{prop: invariance_of_billig_toroidal_cocycles}, we have seen that the toroidal $2$-cocycle:
                $$\sigma_2$$
            is \textit{not} $\gamma$-invariant in the sense of definition \ref{def: yangian_toroidal_cocycles}. However, in light of proposition \ref{prop: cohomological_non_triviality_of_billig_toroidal_cocycles}, we now see that this non-$\gamma$-invariance of $\sigma_2$ holds \textit{despite} the fact that it is cohomologous to $0$. Therefore, one can \textit{not} conclude from observing that a toroidal $2$-cocycle $\sigma$ is \textit{cohomologous} to $0$ that $\sigma$ is $\gamma$-invariant, but when the \textit{equality} $\sigma = 0$ holds in $Z^2_{\Lie}(\divzero, \z(\toroidal))$, then it will indeed be true that $\sigma$ is $\gamma$-invariant, as we know that the semi-direct product $\toroidal \rtimes \divzero$ is an instance of a $\gamma$-extended toroidal Lie algebra.
        \end{remark}