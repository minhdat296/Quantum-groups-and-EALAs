\section{Examples of \texorpdfstring{$\gamma$}{}-extended toroidal Lie algebras}
    \subsection{Explicit (counter-)examples of $\gamma$-invariant toroidal cocycles}
        For convenience, let us fix the following terminologies (which have already been eluded to in the statement of theorem \ref{theorem: yangian_extended_toroidal_lie_algebras_main_theorem}).
        \begin{definition}[$\gamma$-invariant toroidal $2$-cocycles] \label{def: yangian_toroidal_cocycles}
            Any $2$-cocyle $\sigma \in Z^2_{\Lie}(\divzero, \toroidal)$ shall be referred to as a \textbf{toroidal $2$-cocycle}.
            
            Any toroidal $2$-cocycle $\sigma$ such that $\toroidal \rtimes^{\sigma} \divzero$ is a $\gamma$-extended toroidal Lie algebra shall be called a \textbf{$\gamma$-invariant toroidal $2$-cocycle}.
        \end{definition}
        \begin{remark}
            Because Lie $2$-cocycles are central (cf. corollary \ref{coro: 2_cocycles_are_central}), toroidal $2$-cocycles are the same as elements of $Z^2_{\Lie}( \divzero, \z(\toroidal) )$.
        \end{remark}
        
        In theorem \ref{theorem: yangian_extended_toroidal_lie_algebras_main_theorem} (see also proposition \ref{prop: yangian_extended_toroidal_lie_algebras_are_twisted_semi_direct_products}) we have provided a criterion for a given toroidal $2$-cocycle to be $\gamma$-invariant in the sense of definition \ref{def: yangian_toroidal_cocycles} above, namely that:
            $$( \sigma(D, D'), D'' )_{\extendedtoroidal} = ( D, \sigma(D', D'') )_{\extendedtoroidal}$$
        for all $D, D', D'' \in \divzero$. In this subsection, let us apply it to some known toroidal $2$-cocycles from \cite{billig_energy_momentum_tensor} to check whether or not they are $\gamma$-invariant.

        Let us begin by introducing the two toroidal $2$-cocycles we will be working with.
        \begin{example} \label{example: billig_toroidal_cocycles}
            In \cite[p. 5, below Equation 1.3]{billig_energy_momentum_tensor}, it was noted that there are at least $2$-cocyles:
                $$\sigma_1, \sigma_2 \in Z^2_{\Lie}( \der(A), \z(\toroidal) )$$
            These are given in terms of the following basis of $\der(A)$\footnote{Note how we are identifying $\der(A) \cong A v\del_v \oplus A t\del_t$ as opposed to the usual identification $\der(A) \cong A \del_v \oplus A \del_t$.}:
                $$\{ v^{r_v} t^{r_t} \cdot v \del_v \}_{(r_v, r_t) \in \Z^2} \cup \{ v^{m_v} t^{m_t} \cdot t \del_t \}_{(m_v, m_t) \in \Z^2}$$
            by the following formulae:
                $$\sigma_1( v^{r_v} t^{r_t} \cdot x \del_x, v^{m_v} t^{m_t} \cdot y \del_y ) := r_y m_x \cdot v^{r_v} t^{r_t} \bar{d}( v^{m_v} t^{m_t} )$$
                $$\sigma_2( v^{r_v} t^{r_t} \cdot x \del_x, v^{m_v} t^{m_t} \cdot y \del_y ) := r_x m_y \cdot v^{r_v} t^{r_t} \bar{d}( v^{m_v} t^{m_t} )$$
            where $x, y \in \{v, t\}$ are symbolic placeholders.
            
            As a quick aside, let us note that the first cocycle $\sigma_1$ was known as far back as \cite{moody_rao_yokonuma_vertex_representations_of_toroidal_lie_algebras}, but we are not aware of the history of $\sigma_2$ beyond its appearance in \cite{billig_energy_momentum_tensor}.
        \end{example}

        We would like to give descriptions of $\sigma_1, \sigma_2$ as elements of $Z^2_{\Lie}(\divzero, \z(\toroidal))$ instead of elements of $Z^2_{\Lie}(\der(A), \z(\toroidal))$, which is the same as computing the domain restrictions of these two cocycles from $\bigwedge^2 \der(A)$ down to the vector subspace $\bigwedge^2 \divzero$, which can be done by computing the values of $\sigma_1, \sigma_2$ on pairs of basis elements of $\divzero$. Since lemma \ref{lemma: yangian_div_zero_vector_fields_basic_properties}, we have known how the elements of the \say{standard} basis:
            $$\{D_{r, s}\}_{(r, s) \in \Z^2} \cup \{D_v, D_t\}$$
        of $\divzero$ are given in terms of the partial derivatives $\del_v, \del_t$ (or more accurately, in terms of the aforementioned basis of $\der(A)$) by:
            $$D_{r, s} = -s v^{-r + 1} t^{-s - 1} \del_v + r v^{-r} t^{-s} \del_t = -s v^{-r} t^{-s - 1} \cdot v\del_v + r v^{-r} t^{-s - 1} \cdot t\del_t$$
            $$D_v = -v t^{-1} \del_v = -t^{-1} \cdot v\del_v$$
            $$D_t = -\del_t = -t^{-1} \cdot t\del_t$$
        (to rewrite the expressions slightly in terms of the currently employed basis for $\der(A)$). Knowing this allows us to compute the values of $\sigma_1$ and $\sigma_2$ on pairs of these basis elements.
        \begin{lemma}[Values of $\sigma_1$ and $\sigma_2$ on pairs of basis elements of $\divzero$] \label{lemma: billig_toroidal_cocycles_on_yangian_div_zero_vector_fields}
            In what follows, let $i \in \{1, 2\}$. The values of the toroidal $2$-cocycles $\sigma_1, \sigma_2$ from example \ref{example: billig_toroidal_cocycles} on pairs of elements of the basis $\{D_{r, s}\}_{(r, s) \in \Z^2} \cup \{D_v, D_t\}$ of $\divzero$ (cf. lemma \ref{lemma: yangian_div_zero_vector_fields_basic_properties}) are:
                $$\sigma_i(D_{r, s}, D_{a, b}) = N_i(r, s, a, b) \left( ( r(b + 1) - a(s + 1) )K_{-r - a, -s - b - 2} - \delta_{ (-a - r, -b - s - 2), (0, 0) } (r c_v + (s + 1) c_t) \right)$$
            where:
                \begin{equation} \label{equation: billig_cocycles_coefficient}
                    N_i(r, s, a, b) =
                    \begin{cases}
                        \text{$2 rsab - ( (sa)^2 + s a^2 ) - ( (rb)^2 + r^2 b ) + rsa + rab + ra$ if $i = 1$}
                        \\
                        \text{$ra$ if $i = 2$}
                    \end{cases}
                \end{equation}
            and:
                $$\sigma_i(D_v, D_{r, s}) = \delta_{i, 1} r^3 K_{-r, -s - 2}$$
                $$\sigma_i(D_t, D_{r, s}) = \delta_{i, 1} r^2 s K_{-r, -s - 2}$$
                $$\sigma_i(D_v, D_t) = 0$$
        \end{lemma}
            \begin{proof}
                \begin{enumerate}
                    \item Firstly, let us compute $\sigma_i(D_{r, s}, D_{a, b})$. For this, consider the following:
                        $$
                            \begin{aligned}
                                & \sigma_i(D_{r, s}, D_{a, b})
                                \\
                                = & \sigma_i( -s v^{-r + 1} t^{-s - 1} \del_v + r v^{-r} t^{-s} \del_t, -b v^{-a + 1} t^{-b - 1} \del_v + a v^{-a} t^{-b} \del_t )
                                \\
                                = & \sigma_i( s v^{-r} t^{-s - 1} \cdot v\del_v - r v^{-r} t^{-s - 1} \cdot t \del_t, b v^{-a} t^{-b - 1} \cdot v\del_v - a v^{-a} t^{-b - 1} \cdot t \del_t )
                                \\
                                = & s \sigma_i( v^{-r} t^{-s - 1} \cdot v\del_v, b v^{-a} t^{-b - 1} \cdot v\del_v - a v^{-a} t^{-b - 1} \cdot t \del_t ) - r \sigma_i( v^{-r} t^{-s - 1} \cdot t \del_t, b v^{-a} t^{-b - 1} \cdot v\del_v - a v^{-a} t^{-b - 1} \cdot t \del_t )
                                \\
                                = &
                                \begin{aligned}
                                    & s b \cdot \sigma_i( v^{-r} t^{-s - 1} \cdot v\del_v, v^{-a} t^{-b - 1} \cdot v\del_v )
                                    \\
                                    - & s a \cdot \sigma_i( v^{-r} t^{-s - 1} \cdot v\del_v, v^{-a} t^{-b - 1} \cdot t \del_t )
                                    \\
                                    - & r b \cdot \sigma_i( v^{-r} t^{-s - 1} \cdot t \del_t, v^{-a} t^{-b - 1} \cdot v\del_v )
                                    \\
                                    + & r a \cdot \sigma_i( v^{-r} t^{-s - 1} \cdot t \del_t, v^{-a} t^{-b - 1} \cdot t \del_t )
                                \end{aligned}
                                \\
                                = & N_i(r, s, a, b) v^{-r} t^{-s - 1} \bar{d}( v^{-a} t^{-b - 1} )
                            \end{aligned}
                        $$
                    where:
                        $$
                            \begin{aligned}
                                N_i(r, s, a, b) & = 
                                sbra
                                - sa \left( \delta_{i, 1} a(s + 1) + \delta_{i, 2} (b + 1) r \right) 
                                - rb \left( \delta_{i, 1} (b + 1) r + \delta_{i, 2} a (s + 1) \right)
                                + r a (s + 1) (b + 1)
                                \\
                                & = 
                                \begin{cases}
                                    \text{$
                                        sbra
                                        - s a^2 (s + 1) 
                                        - r^2 b (b + 1)
                                        + r a (s + 1) (b + 1)
                                    $if $i = 1$}
                                    \\
                                    \text{$
                                        sbra
                                        - sa (b + 1) r
                                        - rb a (s + 1)
                                        + r a (s + 1) (b + 1)
                                    $ if $i = 2$}
                                \end{cases}
                                \\
                                & = 
                                \begin{cases}
                                    \text{$
                                        sbra
                                        - ( (sa)^2 + s a^2 ) 
                                        - ( (rb)^2 + r^2 b ) 
                                        + rasb + rsa + rab + ra
                                    $ if $i = 1$}
                                    \\
                                    \text{$
                                        sbra
                                        - (sabr + sar)
                                        - (rbas + rba)
                                        + rasb + rsa + rab + ra
                                    $ if $i = 2$}
                                \end{cases}
                                \\
                                & = 
                                \begin{cases}
                                    \text{$2 rsab - ( (sa)^2 + s a^2 ) - ( (rb)^2 + r^2 b ) + rsa + rab + ra$ if $i = 1$}
                                    \\
                                    \text{$ra$ if $i = 2$}
                                \end{cases}
                            \end{aligned}
                        $$
                    
                    Now, recall from example \ref{example: toroidal_lie_algebras_centres} that any element:
                        $$v^n t^q \bar{d}(v^m t^p) \in \z(\toroidal)$$
                    can be written in terms of the basis elements of $\z(\toroidal)$ in the following manner:
                        $$v^n t^q \bar{d}(v^m t^p) = \delta_{(m, p) + (n, q), (0, 0)} ( n c_v + q c_t ) + (np - mq) K_{m + n, p + q}$$
                    Using this, we shall be able to conclude that:
                        $$\sigma_i(D_{r, s}, D_{a, b}) = N_i(r, s, a, b) \left( -\delta_{ (-a - r, -b - s - 2), (0, 0) } (r c_v + (s + 1) c_t) + ( r(b + 1) - a(s + 1) )K_{-r - a, -s - b - 2} \right)$$
                    \item Secondly, we shall be computing $\sigma_i(D_v, D_{r, s})$. Consider the following:
                        $$
                            \begin{aligned}
                                & \sigma_i(D_{r, s}, D_v)
                                \\
                                = & \sigma_i( -s v^{-r + 1} t^{-s - 1} \del_v + r v^{-r} t^{-s} \del_t, -v t^{-1} \del_v )
                                \\
                                = & \sigma_i( s v^{-r} t^{-s - 1} \cdot v \del_v - r v^{-r} t^{-s - 1} t \del_t, t^{-1} \cdot v \del_v )
                                \\
                                = & s \sigma_i( v^{-r} t^{-s - 1} \cdot v \del_v, t^{-1} \cdot v \del_v ) - r\sigma_i( v^{-r} t^{-s - 1} \cdot t \del_t, t^{-1} \cdot v \del_v )
                                \\
                                = & \left( s \left( \delta_{i, 1} (-r) \cdot 0 + \delta_{i, 2} (-r) \cdot 0 \right) - r\left( \delta_{i, 1} (-r) \cdot (-1) + \delta_{i, 2} (-s - 1) \cdot 0 \right) \right) v^{-r} t^{-s - 1} \bar{d}(t^{-1}) 
                                \\
                                = & -\delta_{i, 1} r^2 v^{-r} t^{-s - 1} \bar{d}(t^{-1})
                            \end{aligned}
                        $$
                    Now, recall from example \ref{example: toroidal_lie_algebras_centres} that any element:
                        $$v^n t^q \bar{d}(v^m t^p) \in \z(\toroidal)$$
                    can be written in terms of the basis elements of $\z(\toroidal)$ in the following manner:
                        $$v^n t^q \bar{d}(v^m t^p) = \delta_{(m, p) + (n, q), (0, 0)} ( n c_v + q c_t ) + (np - mq) K_{m + n, p + q}$$
                    Using this, we shall get that:
                        $$
                            \begin{aligned}
                                & \sigma_i(D_{r, s}, D_v)
                                \\
                                = & -r^2 \left( -\delta_{(r, s), (0, -2)} ( r c_v + (s + 3) c_t ) - r K_{-r, -s - 2} \right)
                                \\
                                = &
                                \begin{cases}
                                    \text{$0$ if $(r, s) \in \{0\} \x \Z$}
                                    \\
                                    \text{$r^3 K_{-r, -s - 2}$ if $(r, s) \in (\Z \setminus \{0\}) \x \Z$}
                                \end{cases}
                                \\
                                = & \delta_{i, 1} r^3 K_{-r, -s - 2}
                            \end{aligned}
                        $$
                    \item Using similar methods as in the previous case, we shall get that:
                        $$\sigma_i(D_{r, s}, D_t) = r^2s K_{-r, -s - 2}$$
                    \item Finally, we have that:
                        $$
                            \begin{aligned}
                                & \sigma_i(D_v, D_t)
                                \\
                                = & \sigma_i(-v t^{-1} \del_v, -\del_t)
                                \\
                                = & \sigma_i(t^{-1} \cdot v \del_v, t^{-1} t \del_t)
                                \\
                                = & 0
                            \end{aligned}
                        $$
                \end{enumerate}
            \end{proof}

        One can now use the criterion given in proposition \ref{proposition: twisted_semi_direct_products_are_yangian_extended_toroidal_lie_algebras} (see also theorem \ref{theorem: yangian_extended_toroidal_lie_algebras_main_theorem}) to verify whether or not the cocycles $\sigma_1, \sigma_2$ are $\gamma$-invariant in the sense of definition \ref{def: yangian_toroidal_cocycles}.
        \todo[inline]{I decided to dedicated a separate proposition for discussing whether or not $\sigma_1, \sigma_2$ are $\gamma$-invariant, as the proofs of this proposition and of proposition \ref{prop: cohomological_non_triviality_of_billig_toroidal_cocycles} are both quite long.}
        \begin{proposition}[$\gamma$-invariance of Billig's toroidal $2$-cocycles] \label{prop: invariance_of_billig_toroidal_cocycles}
            Of the two toroidal $2$-cocycles:
                $$\sigma_1, \sigma_2 \in Z^2_{\Lie}(\divzero, \z(\toroidal))$$
            discussed in example \ref{example: billig_toroidal_cocycles}, $\sigma_1$ is $\gamma$-invariant in the sense of definition \ref{def: yangian_toroidal_cocycles}, while $\sigma_2$ fails to be so.
        \end{proposition}
            \begin{proof}
                \begin{enumerate}
                    \item Firstly, using the fact that:
                        $$\sigma_i(D_{r, s}, D_{a, b}) = N_i(r, s, a, b) \left( -\delta_{ (-a - r, -b - s - 2), (0, 0) } (r c_v + (s + 1) c_t) + ( r(b + 1) - a(s + 1) )K_{-r - a, -s - b - 2} \right)$$
                    we shall get that:
                        $$
                            \left( \sigma_i(D_{r, s}, D_{a, b}), D \right)_{\extendedtoroidal} =
                            \begin{cases}
                                \text{$N_i(r, s, a, b) ( r(b + 1) - a(s + 1) ) \delta_{(-r - a, -s - b - 2), (\alpha, \beta)}$ if $D = D_{\alpha, \beta}$}
                                \\
                                \text{$-N_i(r, s, a, b) \delta_{(r, s), -(a, b)} r$ if $D = D_v$}
                                \\
                                \text{$-N_i(r, s, a, b) \delta_{(r, s), -(a, b)} (s + 1)$ if $D = D_t$}
                            \end{cases}
                        $$
                    At the same time, using the fact that:
                        $$\sigma_i(D_{a, b}, D_v) = -\delta_{i, 1} a^3 K_{-a, -b - 2}$$
                        $$\sigma_i(D_{a, b}, D_t) = - \delta_{i, 1} a^2b K_{-a, -b - 2}$$
                    we have that:
                        $$
                            \begin{aligned}
                                \left( D_{r, s}, \sigma_i(D_{a, b}, D) \right)_{\extendedtoroidal} =
                                \begin{cases}
                                    \text{$N_i(r, s, \alpha, \beta) ( r(\beta + 1) - \alpha(s + 1) ) \delta_{(r, s), (-a - \alpha, -b - \beta - 2)}$ if $D = D_{\alpha, \beta}$}
                                    \\
                                    \text{$-\delta_{i, 1} a^3 \delta_{(r, s), (-a, -b - 2)}$ if $D = D_v$}
                                    \\
                                    \text{$-\delta_{i, 1} a^2 b \delta_{(r, s), (-a, -b - 2)}$ if $D = D_t$}
                                \end{cases}
                            \end{aligned}
                        $$
                    We can thus conclude immediately that $\sigma_2$ is \textit{not} invariant, as:
                        $$\left( \sigma_2(D_{r, s}, D_{a, b}), D \right)_{\extendedtoroidal} \not = \left( D_{r, s}, \sigma_2(D_{a, b}, D) \right)_{\extendedtoroidal}$$
                    when $D \in \{D_v, D_t\}$. As such, let us focus on $\sigma_1$ from now on, for which we now have:
                        $$\left( \sigma_1(D_{r, s}, D_{a, b}), D \right)_{\extendedtoroidal} \not = \left( D_{r, s}, \sigma_1(D_{a, b}, D) \right)_{\extendedtoroidal}$$
                    for all $D \in \divzero$.
                    \item Secondly, using the fact that:
                        $$\sigma_1(D_{r, s}, D_v) = r^3 K_{-r, -s - 2}$$
                    we shall get that:
                        $$
                            \left( \sigma_1(D_{r, s}, D_v), D \right)_{\extendedtoroidal} =
                            \begin{cases}
                                \text{$r^3 \delta_{(-r, -s - 2), (\alpha, \beta)}$ if $D = D_{\alpha, \beta}$}
                                \\
                                \text{$0$ if $D = D_v$}
                                \\
                                \text{$0$ if $D = D_t$}
                            \end{cases}
                        $$
                    At the same time, knowing that:
                        $$\sigma_1(D_v, D_t) = 0$$
                    we see that:
                        $$
                            \begin{aligned}
                                \left( D_{r, s}, \sigma_1(D_v, D) \right)_{\extendedtoroidal} =
                                \begin{cases}
                                    \text{$-\alpha^3 \delta_{(r, s), (-\alpha, -\beta - 2)}$ if $D = D_{\alpha, \beta}$}
                                    \\
                                    \text{$0$ if $D = D_v$}
                                    \\
                                    \text{$0$ if $D = D_t$}
                                \end{cases}
                            \end{aligned}
                        $$
                    We thus have:
                        $$\left( \sigma_1(D_{r, s}, D_v), D \right)_{\extendedtoroidal} = \left( D_{r, s}, \sigma_1(D_v, D) \right)_{\extendedtoroidal}$$
                    for all $D \in \divzero$.
                    \item Next, by using the fact that:
                        $$\sigma_1(D_{r, s}, D_t) = r^2 s K_{-r, -s - 2}$$
                    we shall get that:
                        $$
                            \left( \sigma_1(D_{r, s}, D_t), D \right)_{\extendedtoroidal} =
                            \begin{cases}
                                \text{$r^2 s \delta_{(-r, -s - 2), (\alpha, \beta)}$ if $D = D_{\alpha, \beta}$}
                                \\
                                \text{$0$ if $D = D_v$}
                                \\
                                \text{$0$ if $D = D_t$}
                            \end{cases}
                        $$
                    At the same time, we have that:
                        $$
                            \begin{aligned}
                                \left( D_{r, s}, \sigma_1(D_t, D) \right)_{\extendedtoroidal} =
                                \begin{cases}
                                    \text{$-\alpha^2 \beta \delta_{(r, s), (-\alpha, -\beta - 2)}$ if $D = D_{\alpha, \beta}$}
                                    \\
                                    \text{$0$ if $D = D_v$}
                                    \\
                                    \text{$0$ if $D = D_t$}
                                \end{cases}
                            \end{aligned}
                        $$
                    By combining these two observations, one is able to conclude furthermore that:
                        $$\left( \sigma_1(D_{r, s}, D_t), D \right)_{\extendedtoroidal} = \left( D_{r, s}, \sigma_1(D_t, D) \right)_{\extendedtoroidal}$$
                    for all $D \in \divzero$.
                    \item Lastly, since:
                        $$\sigma_1(D_v, D_t) = 0$$
                    we automatically have that:
                        $$( \sigma_1(D_v, D_t), D )_{\extendedtoroidal} = ( D_v, \sigma_1(D_t, D) )_{\extendedtoroidal}$$
                    for all $D \in \divzero$.
                \end{enumerate}
                We have therefore shown that $\sigma_1$ is $\gamma$-invariant in the sense of definition \ref{def: yangian_toroidal_cocycles}. 
            \end{proof}

    \subsection{Cohomological (non-)triviality}
        Whether or not the cocycles:
            $$\sigma_1, \sigma_2 \in Z^2_{\Lie}(\divzero, \z(\toroidal))$$
        might be cohomologous to $0$ (cf. definition \ref{def: lie_algebra_cohomology}) - and hence whether or not they might give rise to extensions that are isomorphic to the semi-direct product $\toroidal \rtimes \divzero$ - is a much subtler issue. One way to tackle this problem is to \textit{firstly} check whether or not their restrictions to a particular Lie subalgebra of $\extendedtoroidal$ is cohomologous to $0$. We remark right away that simply checking that these restrictions are non-cohomologous to $0$ is \textit{not} sufficient for concluding that $\sigma_1$ and $\sigma_2$ are non-zero elements of $H^2_{\Lie}(\divzero, \z(\toroidal))$.

        We begin our analysis by recalling that the Witt algebra:
            $$\der(\bbC[v^{\pm 1}]) \cong \bigoplus_{r \in \Z} \bbC d_r$$
        (where $d_r := -v^r \cdot v\frac{d}{dv}$) is known to possess a \textit{non-trivial} UCE:
            $$\frakv := \der(\bbC[v^{\pm 1}]) \oplus^{\eta} \bbC c_{\frakv}$$
        called the \textbf{Virasoro algebra} (cf. \cite[Sections 9.13 and 9.14]{kac_infinite_dimensional_lie_algebras} and \cite[Section 1.3]{kac_raina_rozhkovskaya_bombay_lectures_on_highest_weight_modules_of_infinite_dimensional_lie_algebras}), whose corresponding $2$-cocycle $\eta \in Z^2_{\Lie}( \der(\bbC[v^{\pm 1}]), \bbC c_{\frakv} )$ is given by:
            $$\eta(d_r, d_a) := \delta_{r + a, 0} (r^3 - r) c_{\frakv}$$
        for all $r, a \in \Z$; in particular, this means this $\eta$ is \textit{non-cohomologous} to $0$. Moreover, the Virasoro algebra is the only non-trivial central extension of the Witt algebra.
        \begin{lemma}[$H^2_{\Lie}$ of the Witt algebra] \label{lemma: H^2_of_witt_algebra}
            (Cf. \cite[Proposition 1.3]{kac_raina_rozhkovskaya_bombay_lectures_on_highest_weight_modules_of_infinite_dimensional_lie_algebras}) Let $\bbC c_{\frakv}$ be viewed as a trivial $\der(\bbC[v^{\pm 1}])$-module. Then:
                $$\dim_{\bbC} H^2_{\Lie}( \der(\bbC[v^{\pm 1}]), \bbC c_{\frakv} ) = 1$$
        \end{lemma}

        Since we know now that there is an isomorphism of Lie algebras $\der(\bbC[v^{\pm 1}]) \xrightarrow[]{\cong} \frakw$ given by $d_r \mapsto D_{r, -1}$ (cf. lemma \ref{lemma: a_copy_of_the_witt_algebra_inside_the_lie_algebra_of_yangian_div_zero_vector_fields}), it makes sense to regard $\sigma_1$ and $\sigma_2$ as $2$-cocycles of $\der(\bbC[v^{\pm 1}])$ (with either values in $\bbC c_{\frakv}$ or all of $\z(\toroidal)$) after restricting their domains to $\bigwedge^2 \frakw$. We shall now attempt to show, using lemma \ref{lemma: H^2_of_witt_algebra}, that because:
            $$\sigma_1|_{\bigwedge^2 \frakw}(D_{r, -1}, D_{a, -1}) = \sigma_2|_{\bigwedge^2 \frakw}(D_{r, -1}, D_{a, -1}) = \delta_{r + a, 0} r^3 c_v$$
        they must be non-zero as elements of $H^2_{\Lie}(\frakw, \bbC c_v)$. This will be the content of lemma \ref{lemma: restrictions_of_billig_toroidal_cocycles_are_coboundary}.

        We begin with the following lemma:
        \begin{lemma}[A Virasoro $2$-coboundary] \label{lemma: a_virasoro_coboundary}
            Let:
                $$\eta': \bigwedge^2 \der(\bbC[v^{\pm 1}]) \to \bbC c_{\frakv}$$
            be the function given by:
                $$\eta'(d_r, d_a) := \delta_{r + a, 0} r c_{\frakv}$$
            This is a $2$-cocycle of $\der(\bbC[v^{\pm 1}])$ with values in $\bbC c_{\frakv}$. From this, one sees that:
                $$\eta + \eta': \bigwedge^2 \der(\bbC[v^{\pm 1}]) \to \bbC c_{\frakv}$$
            (which is given by $(\eta + \eta')(d_r, d_a) := \delta_{r + a, 0} r^3 c_{\frakv}$) is also a $2$-cocycle of $\der(\bbC[v^{\pm 1}])$ with values in $\bbC c_{\frakv}$. 

            Furthermore, we have that:
                $$\eta' \in B^2_{\Lie}( \der(\bbC[v^{\pm 1}]), \bbC c_{\frakv} )$$
            with notations as in definition \ref{def: lie_cocycles_and_coboundaries}.
        \end{lemma}
            \begin{proof}
                It is clear from the construction of $\eta'$ that it is linear and skew-symmetric; the only non-trivial thing to prove is that $\eta'$ satisfies the Jacobi identity in the sense of definition \ref{def: twisted_semi_direct_products}. To this end, simply consider the following, for all $i, j, k \in \Z$:
                    $$
                        \begin{aligned}
                            & \eta'([d_i, d_j], d_k) + \eta'([d_k, d_i], d_j) + \eta'([d_j, d_k], d_i)
                            \\
                            = & (i - j) \eta'(d_{i + j}, d_k) + (k - i) \eta'(d_{k + i}, d_j) + (j - k) \eta'(d_{j + k}, d_i)
                            \\
                            = & \delta_{i + j + k, 0} \left( (i - j) (i + j) + (k - i) (k + i) + (j - k) (j + k) \right) c_{\frakv}
                            \\
                            = & 0
                        \end{aligned}
                    $$
                    
                To show that $\eta'$ is a Lie $2$-coboundary in the sense of definition \ref{def: lie_cocycles_and_coboundaries}, we must show that there exists a linear map:
                    $$\tilde{\eta'}: \der(\bbC[v^{\pm 1}]) \to \bbC c_{\frakv}$$
                which is merely a linear map, such that:
                    $$\eta(d_i, d_j) = \tilde{\eta'}([d_i, d_j])$$
                (cf. examples \ref{example: lie_cocycles_and_coboundaries_with_trivial_coefficients} and \ref{example: low_degree_lie_cocycles_and_coboundaries_with_trivial_coefficients}). The RHS is nothing but:
                    $$\tilde{\eta'}([d_i, d_j]) = (i - j) \tilde{\eta'}(d_{i + j}) c_{\frakv}$$
                while by construction, the LHS is:
                    $$\eta(d_i, d_j) := \delta_{i + j, 0} i c_{\frakv}$$
                and hence:
                    $$\delta_{i + j, 0} i = (i - j) \tilde{\eta'}(d_{i + j})$$
                By setting $j = 0$, we then see that:
                    $$i (\tilde{\eta'}(d_i) - \delta_{i, 0}) = 0$$
                for all $i \in \Z$, which in turn implies that:
                    $$\tilde{\eta'}(d_i) = \delta_{i, 0}$$
                The sought-for linear map $\tilde{\eta}: \der(\bbC[v^{\pm 1}]) \to \bbC c_{\frakv}$ is thus defined, and hence exists.
            \end{proof}
        \begin{lemma}[Restricting $\sigma_1$ and $\sigma_2$ to the Witt algebra] \label{lemma: billig_toroidal_cocycles_on_the_witt_algebra}
            Let $\eta' \in Z^2_{\Lie}(\der(\bbC[v^{\pm 1}]), \bbC c_{\frakv})$ be as in lemma \ref{lemma: a_virasoro_coboundary}. The Lie $2$-cocycle:
                $$\eta + \eta' \in Z^2_{\Lie}(\der(\bbC[v^{\pm 1}]), \bbC c_{\frakv})$$
            given by:
                $$(\eta + \eta')(d_r, d_a) = \delta_{r + a, 0} r^3 c_{\frakv}$$
            for all $r, a \in \Z$, is not cohomologous to $0$, i.e. its image under the canonical projection:
                $$Z^2(\der(\bbC[v^{\pm 1}]), \bbC c_{\frakv}) \to H^2(\der(\bbC[v^{\pm 1}]), \bbC c_{\frakv})$$
            (cf. definition \ref{def: complexes_and_cohomology}) is non-zero.

            Furthermore, we have that:
                $$\sigma_i|_{\bigwedge^2 \frakw} = \eta + \eta'$$
            and consequently, the equivalence classes of domain restrictions $\sigma_i|_{\bigwedge^2 \frakw}$ (where $i \in \{1, 2\}$) in $H^2_{\Lie}( \frakw, \bbC c_v )$ are also non-zero.
        \end{lemma}
            \begin{proof}
                Because we know that $\dim_{\bbC} H^2(\der(\bbC[v^{\pm 1}]), \bbC c_{\frakv}) = 1$ (cf. lemma \ref{lemma: H^2_of_witt_algebra}) and that $\eta' \in B^2(\der(\bbC[v^{\pm 1}]), \bbC c_{\frakv})$ (cf. lemma \ref{lemma: a_virasoro_coboundary}), $\eta + \eta'$ must be cohomologous to $\eta$, which is not cohomologous to $0$ as shown in the proof of lemma \ref{lemma: H^2_of_witt_algebra}.

                To prove the consequential statement, simply note that:
                    $$\sigma_i(D_{r, s}, D_{a, b}) = N_i(r, s, a, b) \left( -\delta_{ (-a - r, -b - s - 2), (0, 0) } (r c_v + (s + 1) c_t) + ( r(b + 1) - a(s + 1) )K_{-r - a, -s - b - 2} \right)$$
                where:
                    $$
                        N_i(r, s, a, b) =
                        \begin{cases}
                            \text{$2 rsab - ( (sa)^2 + s a^2 ) - ( (rb)^2 + r^2 b ) + rsa + rab + ra$ if $i = 1$}
                            \\
                            \text{$ra$ if $i = 2$}
                        \end{cases}
                    $$
                Setting $s = b = -1$ then yields:
                    $$\sigma_i(D_{r, -1}, D_{a, -1}) = -N_i(r, -1, a, -1) \delta_{r + a, 0} r c_v$$
                where now, we have that:
                    $$N_i(r, -1, a, -1) = ra$$
                for both $i = 1$ and $i = 2$, and hence:
                    $$\sigma_i(D_{r, -1}, D_{a, -1}) = \delta_{r + a, 0} r^3 c_v = (\eta + \eta')(d_r, d_a)$$
            \end{proof}
        \begin{remark}[The Virasoro central element]
            Because we now know that:
                $$\sigma_i(D_{r, -1}, D_{a, -1}) = \delta_{r + a, 0} r^3 c_v$$
            it is clear that the Virasoro central element $c_{\frakv}$ is nothing but $c_v$.
        \end{remark}

        Now, even though it might be tempting to conclude right away that because of lemma \ref{lemma: billig_toroidal_cocycles_on_the_witt_algebra}, which tells us that:
            $$\sigma_i|_{\bigwedge^2 \frakw}: \bigwedge^2 \frakw \to \bbC c_v$$
        is cohomologous to $\eta + \eta' \in Z^2_{\Lie}(\frakw, \bbC c_v)$, which is not $2$-coboundary, it must then also be true that $\sigma_i \not \in B^2_{\Lie}(\divzero, \z(\toroidal))$. However, the subtlety here is that because $\z(\toroidal)$ is non-trivial as a module over $\divzero$ (and likewise, over the Lie subalgebra $\frakw \subset \divzero$), unlike $\bbC c_v$, one would have to actually check whether or not the restricted toroidal $2$-cocycle:
            $$\sigma_i|_{\bigwedge^2 \frakw}: \bigwedge^2 \divzero \to \z(\toroidal)$$
        is $2$-coboundary.

        Even though it suffices to prove the existence of an element:
            $$\tau_i \in C_1(\divzero, \z(\toroidal))$$
        (i.e. a linear map; cf. remark \ref{remark: simplified_chevalley_eilenberg_complexes}) such that:
            $$\tau_i([D, D']) = [D, \tau_i(D')]_{\extendedtoroidal} - [D', \tau_i(D)]_{\extendedtoroidal} - \sigma_i(D, D')$$
        we shall prove a stronger statement, namely the existence of $\tau_i \in C_1(\divzero, \z(\toroidal))$ that is \textit{graded} with respect to the $\Z^2$-gradings on $\divzero$ and $\z(\toroidal)$. We will do this by firstly assuming that such a graded $\tau_i$ exists, deriving an explicit closed-form formula for it, and then verifying that such a $\tau_i$ must then be $2$-coboundary. Our final result, proposition \ref{prop: cohomological_non_triviality_of_billig_toroidal_cocycles} will as such be proven firstly through the use of lemma \ref{lemma: cohomological_non_triviality_of_billig_toroidal_cocycles_explicit_formula} below.

        \begin{remark}[Some recollection]
            The proofs of lemma \ref{lemma: cohomological_non_triviality_of_billig_toroidal_cocycles_explicit_formula} below - as well as of proposition \ref{prop: cohomological_non_triviality_of_billig_toroidal_cocycles} - are somewhat computation-heavy, so for the sake of visual clarity and convenience, let us recall for the proof the following facts that have been earlier in this chapter.
            \begin{itemize}
                \item Firstly, recall from remark \ref{remark: Z^2_grading_on_toroidal_centres} and corollary \ref{coro: yangian_div_zero_vector_fields_are_graded} that:
                    $$\deg D_{-a, -b - 1} = \deg K_{a, b} = (a, b)$$
                    $$\deg D_v = \deg D_t = (0, -1)$$
                    $$\deg c_v = \deg c_t = (0, 0)$$
                \item Secondly, let us recall from lemma \ref{lemma: yangian_div_zero_vector_fields_basic_properties} that:
                    $$[D_v, D_{r, s}] = -r D_{r, s + 1}$$
                    $$[D_t, D_{r, s}] = -s D_{r, s + 1}$$
                    $$[D_{a, b}, D_{r, s}] = (br - sa) D_{a + r, b + s + 1}$$
                \item Finally, from lemma \ref{lemma: explicit_commutators_between_central_basis_elements_and_derivations}, we know that:
                    $$[D, K_{\alpha, \beta}]_{\extendedtoroidal} =
                        \begin{cases}
                            \text{$((\beta - 1)a - b\alpha) K_{\alpha - a, \beta - b - 1} + \delta_{(a, b + 1), (\alpha, \beta)} \left( a c_v + b c_t \right)$ if $D = D_{a, b}$}
                            \\
                            \text{$-\alpha K_{\alpha, \beta - 1}$ if $D_v$}
                            \\
                            \text{$-\beta K_{\alpha, \beta - 1}$ if $D_t$}
                        \end{cases}
                    $$
                    $$[D, c_v]_{\extendedtoroidal} = [D, c_t]_{\extendedtoroidal} = 0$$
                for all $D \in \divzero$.
            \end{itemize}
        \end{remark}

        \todo[inline]{I decided to split the proof of proposition \ref{prop: cohomological_non_triviality_of_billig_toroidal_cocycles} into two halves because it was getting too long. I will add the "official statement" of the main result later once I'm sure of all the contents.}
        \begin{lemma}[$\sigma_1$ is not $2$-coboundary] \label{lemma: cohomological_non_triviality_of_billig_toroidal_cocycles_explicit_formula}
            Consider the two toroidal $2$-cocycles:
                $$\sigma_1, \sigma_2 \in Z^2_{\Lie}(\divzero, \z(\toroidal))$$
            as in example \ref{example: billig_toroidal_cocycles}.
            \begin{enumerate}
                \item The toroidal $2$-cocycle $\sigma_1$ is \textit{not} $2$-coboundary, i.e. its image under the canonical projection $Z^2_{\Lie}(\divzero, \z(\toroidal) ) \to H^2_{\Lie}(\divzero, \z(\toroidal) )$ is non-zero.
                \item On the other hand, should $\sigma_2$ indeed be $2$-coboundary\footnote{... and we will see via proposition \ref{prop: cohomological_non_triviality_of_billig_toroidal_cocycles} that indeed, $\sigma_2$ is $2$-coboundary.}, then there will exist a one-parameter family:
                    $$\{\tau_2^{\kappa}\}_{\kappa \in \bbC}$$
                of graded linear maps $\tau_2^{\kappa} \in C_1(\divzero, \z(\toroidal))$ of $\sigma_2$ under $d_1^{\z(\toroidal)}: C_1(\divzero, \z(\toroidal)) \to C_2(\divzero, \z(\toroidal))$\footnote{See remark \ref{remark: simplified_chevalley_eilenberg_complexes} and definition \ref{def: lie_algebra_cohomology} for notations.}. Each $\tau_i^{\kappa}$ is given by:
                    $$\tau_2^{\kappa}(D_{r, s}) = \left( \frac12 r^2 + r\nu - \frac12 \kappa \right) K_{-r, -s - 1} + \delta_{(r, s), (0, -1)} \left( -\frac12 \kappa c_v + \nu c_t\right)$$
                    $$\tau_2^{\kappa}(D_v) = \mu_v K_{0, -1}$$
                    $$\tau_2^{\kappa}(D_t) = \mu_t K_{0, -1}$$
                where the coefficients $\nu, \mu_v, \mu_t \in \bbC$ can be chosen arbitrarily.
            \end{enumerate}
        \end{lemma}
            \begin{proof} 
                Firstly, our assumption would then imply that there are coefficients:
                    $$\lambda_{r, s}, \mu', \mu, \nu \in \bbC$$
                such that:
                    $$
                        \begin{aligned}
                            \tau_i(D_{r, s}) & = \lambda_{r, s} K_{-r, -s - 1} + \delta_{(r, s), (0, -1)} ( \mu' K_{r, 0} + \alpha c_v + \beta c_t )
                            \\
                            & = \lambda_{r, s} K_{-r, -s - 1} + \delta_{(r, s), (0, -1)} ( \mu c_v + \nu c_t )
                        \end{aligned}
                    $$
                whenever $(r, s) \not = (0, 0)$, where the second equality holds because $K_{0, 0} = 0$ (cf. example \ref{example: toroidal_lie_algebras_centres}). When $(r, s) = (0, 0)$, because $D_{0, 0} = K_{0, 0} = 0$, the assumption that $\tau_i$ is graded would then give the following tautological equation:
                    $$0 = \tau_i(D_{0, 0}) = \lambda_{0, 0} K_{0, 0} = 0$$
                and hence $\kappa$ can be taken to be a free parameter, say:
                    $$\lambda_{0, 0} := \kappa$$
                with $\kappa$ ranging over $\bbC$. 

               Likewise, there shall be coefficients:
                    $$\mu_v, \mu_t \in \bbC$$
                such that:
                    $$\tau_i(D_v) = \mu_v K_{0, -1}$$
                    $$\tau_i(D_t) = \nu_t K_{0, -1}$$
                
                \todo[inline]{Fixed the general expressions for $\tau_i(D_v)$ and for $\tau_i(D_t)$.}
                \begin{itemize}
                    \item We attempt first of all to compute $\tau_i(D_{r, s})$. For this, we refer the reader to lemma \ref{lemma: billig_toroidal_cocycles_on_yangian_div_zero_vector_fields} for the following fact:
                        $$\sigma_i(D_v, D_{r, s}) = -\delta_{i, 1} r^3 K_{-r, -s - 2}$$
                    This gives the following\footnote{Using $D_t$ in place of $D_v$ would yield the same conclusion.}:
                        $$
                            \begin{aligned}
                                & r \tau_i(D_{r, s + 1})
                                \\
                                = & \tau_i([D_v, D_{r, s}])
                                \\
                                = & [D_v, \tau_i(D_{r, s})]_{\extendedtoroidal} - [D_{r, s}, \tau_i(D_v)]_{\extendedtoroidal} - \sigma_i(D_v, D_{r, s})
                                \\
                                = & [D_v, \lambda_{r, s} K_{-r, -s - 1}]_{\extendedtoroidal} - [D_{r, s}, \mu_v c_v + \mu_t c_t]_{\extendedtoroidal} + \delta_{i, 1} r^3 K_{-r, -s - 2}
                                \\
                                = & r \lambda_{r, s} K_{-r, -s - 2} + \delta_{i, 1} r^3 K_{-r, -s - 2}
                                \\
                                = & ( r \lambda_{r, s} + \delta_{i, 1} r^3 ) K_{-r, -s - 2}
                            \end{aligned}
                        $$
                    From this, we infer that when $r \not = 0$, we have that:
                        $$\tau_i(D_{r, s + 1}) = (\lambda_{r, s} + \delta_{i, 1} r^2) K_{-r, -s - 2}$$
                    but at the same time, we have the following per our initial assumption that $\tau_i$ is graded:
                        $$\tau_i(D_{r, s + 1}) = \lambda_{r, s + 1} K_{-r, -s - 2} + \delta_{(r, s + 1), (0, -1)} ( \alpha_{r, s + 1} c_v + \beta_{r, s + 1} c_t )$$
                    and so we have:
                        $$\lambda_{r, s + 1} = \lambda_{r, s} + \delta_{i, 1} r^2$$
                    whenever $r \not = 0$ (in which case $\delta_{(r, s + 1), (0, -1)} = 0$). This recursive formula is equivalent to the following, valid for all $(r, s) \in \Z^2 \setminus ( \{0\} \x \Z )$:
                        \begin{equation} \label{equation: lambda_rs_coefficients_recursion}
                            \lambda_{r, s} = \lambda_{r, 0} + \delta_{i, 1} r^2 s
                        \end{equation}
                    From this, one infers that in order to determined $\lambda_{r, s}$ when $r \not = 0$, it suffices to only determine $\lambda_{r, 0}$.
                    
                    Next, consider the following:
                        $$
                            \begin{aligned}
                                & (br - sa) \tau_i(D_{a + r, b + s + 1})
                                \\
                                = & \tau_i( [D_{a, b}, D_{r, s}] )
                                \\
                                = & [D_{a, b}, \tau_i(D_{r, s})]_{\extendedtoroidal} - [D_{r, s}, \tau_i(D_{a, b})]_{\extendedtoroidal} - \sigma_i(D_{a, b}, D_{r, s})
                                \\
                                = & \lambda_{r, s} [D_{a, b}, K_{-r, -s - 1}]_{\extendedtoroidal} - \lambda_{a, b} [D_{r, s}, K_{-a, -b - 1}]_{\extendedtoroidal} + \sigma_i(D_{r, s}, D_{a, b})
                            \end{aligned}
                        $$
                    Using lemma \ref{lemma: explicit_commutators_between_central_basis_elements_and_derivations}, we get that:
                        $$[D_{a, b}, K_{-r, -s - 1}]_{\extendedtoroidal} = -\left( a(s + 2) - br \right) K_{-a - r, -b - s - 2} + \delta_{(a + r, b + s + 2), (0, 0)} \left( a c_v + b c_t \right)$$
                        $$[D_{r, s}, K_{-a, -b - 1}]_{\extendedtoroidal} = -\left( (b + 2) r - as \right) K_{-a - r, -b - s - 2} + \delta_{(a + r, b + s + 2), (0, 0)} \left( r c_v + s c_t \right)$$
                    and from lemma \ref{lemma: billig_toroidal_cocycles_on_yangian_div_zero_vector_fields}, we know that:
                        $$\sigma_i(D_{r, s}, D_{a, b}) = N_i(r, s, a, b) \left( ( r(b + 1) - a(s + 1) )K_{-r - a, -s - b - 2} - \delta_{ (a + r, b + s + 2), (0, 0) } (r c_v + (s + 1) c_t) \right)$$
                    where $N_i(r, s, a, b)$ is as in \textit{loc. cit.} Simultaneously, these fact imply that:
                        \begin{equation} \label{equation: coboundary_equation_D_rs_D_ab}
                            \begin{aligned}
                                & (br - sa) \tau_i(D_{a + r, b + s + 1})
                                \\
                                = & (br - sa) \left( \lambda_{a + r, b + s + 1} K_{-a - r, -b - s - 2} + \delta_{(a + r, b + s + 1), (0, -1)}( \mu c_v + \nu c_t ) \right)
                                \\
                                = &
                                \begin{aligned}
                                    & \left( -\lambda_{r, s} \left( a(s + 2) - br \right) + \lambda_{a, b} \left( (b + 2) r - as \right) + N_i(r, s, a, b)\left( r(b + 1) - a(s + 1) \right) \right) K_{-a - r, -b - s - 2}
                                    \\
                                    + & \delta_{(a + r, b + s + 2), (0, 0)} \left( \lambda_{r, s} a - \lambda_{a, b} r - N_i(r, s, a, b) r \right) c_v
                                    \\
                                    + & \delta_{(a + r, b + s + 2), (0, 0)} \left( \lambda_{r, s} b - \lambda_{a, b} s - N_i(r, s, a, b) (s + 1) \right) c_t
                                \end{aligned}
                                \\
                                = &
                                \begin{aligned}
                                    & \left( -\lambda_{r, s} \left( a(s + 2) - br \right) + \lambda_{a, b} \left( (b + 2) r - as \right) + N_i(r, s, a, b)\left( r(b + 1) - a(s + 1) \right) \right) K_{-a - r, -b - s - 2}
                                    \\
                                    - & \delta_{(a + r, b + s + 2), (0, 0)} r \left( \lambda_{r, s} + \lambda_{-r, -s - 2} + N_i(r, s, -r, -s - 2) \right) c_v
                                    \\
                                    - & \delta_{(a + r, b + s + 2), (0, 0)} \left( \lambda_{r, s} (s + 2) + \lambda_{-r, -s - 2} s + N_i(r, s, -r, -s - 2) (s + 1) \right) c_t
                                \end{aligned}
                            \end{aligned}
                        \end{equation}
                    from which it can be inferred - by comparing the coefficients of $K_{-a - r, -b - s - 2}$ - that:
                        \begin{equation} \label{equation: lambda_rs_coefficients}
                            (br - sa) \lambda_{a + r, b + s + 1}
                            =
                            -\lambda_{r, s} \left( a(s + 2) - br \right) + \lambda_{a, b} \left( (b + 2) r - as \right) + N_i(r, s, a, b)\left( r(b + 1) - a(s + 1) \right)
                        \end{equation}
                    As mentioned above (cf. equation \eqref{equation: lambda_rs_coefficients_recursion}), we would like to determine $\lambda_{r, 0}$ when $r \not = 0$. To this end, let us evaluate equation \eqref{equation: lambda_rs_coefficients} at:
                        $$b = 0, s = -1$$
                    Doing so yields:
                        $$
                            \begin{aligned}
                                & a \lambda_{a + r, 0}
                                \\
                                = & -\lambda_{r, -1} a + \lambda_{a, 0} \left( 2r + a \right) + N_i(r, -1, a, 0) r
                                \\
                                = & -(\lambda_{r, 0} - \delta_{i, 1} r^2) a + \lambda_{a, 0} \left( 2r + a \right) + N_i(r, -1, a, 0) r
                            \end{aligned}
                        $$
                    When $r + a = 0$, this will turn into:
                        $$-r \kappa = -r \lambda_{0, 0} = (\lambda_{r, 0} - \delta_{i, 1} r^2) r + \lambda_{-r, 0} r + N_i(r, -1, -r, 0) r$$
                    and since the current assumption is that $r \not = 0$, the above is equivalent to:
                        $$
                            \begin{aligned}
                                & -\kappa
                                \\
                                = & (\lambda_{r, 0} - \delta_{i, 1} r^2) + \lambda_{-r, 0} + N_i(r, -1, -r, 0)
                                \\
                                = & (\lambda_{r, 0} + \lambda_{-r, 0}) - \delta_{i, 1} r^2 + N_i(r, -1, -r, 0)
                            \end{aligned}
                        $$
                    and hence:
                        $$
                            \begin{aligned}
                                & \lambda_{r, 0} + \lambda_{-r, 0}
                                \\
                                = & \delta_{i, 1} r^2 - N_i(r, -1, -r, 0) - \kappa
                                \\
                                = & \delta_{i, 1} r^2 + \delta_{i, 2} r^2 - \kappa
                            \end{aligned}
                        $$
                    (to see why the second equality holds, see equation \eqref{equation: billig_cocycles_coefficient}, from which one sees that $N_1(r, -1, -r, 0) = 0$ and $N_2(r, -1, -r, 0) = -r^2$). Using equation \eqref{equation: lambda_rs_coefficients_recursion}, we then get that:
                        $$
                            \begin{aligned}
                                & \lambda_{r, -1} + \lambda_{-r, -1}
                                \\
                                = & \lambda_{r, 0} + \lambda_{-r, 0} - 2\delta_{i, 1} r^2
                                \\
                                = & ( \delta_{i, 1} r^2 + \delta_{i, 2} r^2 - \kappa ) - 2\delta_{i, 1} r^2
                                \\
                                = & -\delta_{i, 1} r^2 + \delta_{i, 2} r^2 - \kappa
                                \\
                                = & (-1)^{\delta_{i, 1}} r^2 - \kappa
                            \end{aligned}
                        $$
                    Next, let us compute $\lambda_{r, -1} - \lambda_{-r, -1}$ (and during the process, we will also be able to compute the coefficients $\mu, \nu$), so that we can establish a linear system in the variables $\lambda_{\pm r, -1}$. To this end, let us evaluate equation \eqref{equation: coboundary_equation_D_rs_D_ab} when $r + a = 0$ and $b = s = -1$. Doing so yields:
                        $$
                            \begin{aligned}
                                & 2r \tau_i(D_{0, -1})
                                \\
                                = & r\left( \lambda_{r, -1} + \lambda_{-r, -1} + N_i(r, -1, -r, -1) \right) c_v + \left( \lambda_{r, -1} - \lambda_{-r, -1} \right) c_t
                                \\
                                = & r\left( \lambda_{r, -1} + \lambda_{-r, -1} - r^2 \right) c_v + \left( \lambda_{r, -1} - \lambda_{-r, -1} \right) c_t
                                \\
                                = & r\left( (-1)^{\delta_{i, 1}} r^2 - \kappa - r^2 \right) c_v + \left( \lambda_{r, -1} - \lambda_{-r, -1} \right) c_t
                                \\
                                = & r\left( ((-1)^{\delta_{i, 1}} - 1) r^2 - \kappa \right) c_v + \left( \lambda_{r, -1} - \lambda_{-r, -1} \right) c_t
                            \end{aligned}
                        $$
                    (and note that from equation \eqref{equation: billig_cocycles_coefficient}, it can be inferred that $N_1(r, -1, -r, -1) = N_2(r, -1, -r, -1) = -r^2$, which gives the second equality). At the same time, we have per the graded-ness assumption on $\tau_i$ that:
                        $$\tau_i(D_{0, -1}) = \mu c_v + \nu c_t$$
                    Neither $\mu, \nu \in \bbC$ depends on $r$, so we must have the following, by simplying comparing the coefficients of $c_v$ and of $c_t$:
                        $$\lambda_{r, -1} + \lambda_{-r, -1} - r^2 = ((-1)^{\delta_{i, 1}} - 1) r^2 - \kappa = 2\mu$$
                        $$\lambda_{r, -1} - \lambda_{-r, -1} = 2r\nu$$
                    Observe now that:
                        $$(-1)^{\delta_{i, 1}} - 1 = -2\delta_{i, 1}$$
                    meaning that only when $i \not = 1$ (so $i = 2$ for us) is the expression $((-1)^{\delta_{i, 1}} - 1) r^2$ \textit{independent} of $r$, which is what we would like to have, since $\mu$ is not dependent on $r$. As such, we can immediately rule out the case $i = 1$, and therefore will only work with $i = 2$ from now on. Another consequence of the equation $(-1)^{\delta_{i, 1}} - 1 = -2\delta_{i, 1}$ is that:
                        $$\mu = -\frac12 \kappa$$
                    
                    We now have the following linear system:
                        $$
                            \begin{cases}
                                \lambda_{r, -1} + \lambda_{-r, -1} = 2\mu + r^2 
                                \\
                                \lambda_{r, -1} - \lambda_{-r, -1} = 2r\nu
                            \end{cases}
                        $$
                    which can be solved to yield:
                        $$\lambda_{r, -1} = \frac12 r^2 + r\nu + \mu = \frac12 r^2 + r\nu - \frac12 \kappa$$
                    Plugging this into equation \eqref{equation: lambda_rs_coefficients_recursion} then yields the following whenever $r \not = 0$:
                        \begin{equation} \label{equation: lambda_rs_formula}
                            \lambda_{r, s} = \frac12 r^2 + r\nu - \frac12 \kappa
                        \end{equation}
                    with no constrains on $\nu \in \bbC$.
                    
                    We have thus shown that, to check whether $\sigma_2$ is $2$-coboundary or not, we can use the following family of elements $\tau_2^{\kappa} \in C_1(\divzero, \z(\toroidal))$:
                        $$\left\{ \tau_2^{\kappa}(D_{r, s}) := \left( \frac12 r^2 + r\nu - \frac12 \kappa \right) K_{-r, -s - 1} + \delta_{(r, s), (0, -1)} \left( -\frac12 \kappa c_v + \nu c_t \right) \right\}_{\kappa \in \bbC}$$
                    where $\nu \in \bbC$ can be arbitrarily chosen.
                    \item Next, consider $\tau_i^{\kappa}(D_v)$. Because there do not exist elements $D, D' \in \divzero$ so that $D_v = [D, D']$ and because $[D_v, D_t] = 0$ (cf. lemma \ref{lemma: yangian_div_zero_vector_fields_basic_properties}), it is impossible to determine $\tau_i^{\kappa}(D_v)$ through the use of the $2$-coboundary equation:
                        $$\tau_i^{\kappa}([D, D']) = [D, \tau_i^{\kappa}(D')]_{\extendedtoroidal} - [D', \tau_i^{\kappa}(D)]_{\extendedtoroidal} - \sigma_i(D, D')$$
                    (cf. example \ref{example: low_degree_lie_coboundaries_with_non-trivial_coefficients}). As such, we can let the coefficients $\mu_v \in \bbC$ in the expression:
                        $$\tau_i(D_v) = \mu_v K_{0, -1}$$
                    be arbitrary.
                    \item Likewise, we can let the coefficients $\mu_v \in \bbC$ in the expression:
                        $$\tau_i^{\kappa}(D_t) = \mu_t K_{0, -1}$$
                    be arbitrary.
                \end{itemize}
            \end{proof}
        \begin{corollary}[A non-trivial $\gamma$-extended toroidal Lie algebra]
            The twisted semi-direct product $\toroidal \rtimes^{\sigma_1} \divzero$ is an example of a $\gamma$-extended toroidal Lie algebra that is \textit{not} isomorphic to the semi-direct product $\toroidal \rtimes \divzero$.
        \end{corollary}
            \begin{proof}
                Combine lemma \ref{lemma: cohomological_non_triviality_of_billig_toroidal_cocycles_explicit_formula} with the fact that $\sigma_1$ is $\gamma$-invariant, shown in proposition \ref{prop: invariance_of_billig_toroidal_cocycles}.
            \end{proof}

        \todo[inline]{I think everything above this line is final (validity-wise, not presentation-wise).}
        \todo[inline]{I believe I now have a correct proof of proposition \ref{prop: cohomological_non_triviality_of_billig_toroidal_cocycles}, but I will need a bit more time to type it up. Hopefully before our Wednesday meeting.}

        The following result is our final conclusion regarding whether or not the toroidal $2$-cocycles $\sigma_1$ and $\sigma_2$ are cohomologous to $0$.
        \begin{proposition}[$\sigma_2$ is $2$-coboundary] \label{prop: cohomological_non_triviality_of_billig_toroidal_cocycles}
            Let:
                $$\sigma_2 \in Z^2_{\Lie}(\divzero, \z(\toroidal) )$$
            be as in example \ref{example: billig_toroidal_cocycles}. Then, in fact, we have that:
                $$\sigma_2 \in B^2_{\Lie}(\divzero, \z(\toroidal))$$
        \end{proposition}
            \begin{proof}
                As we have lemma \ref{lemma: cohomological_non_triviality_of_billig_toroidal_cocycles_explicit_formula}, it remains to check whether or not the $2$-boundary equation:
                    $$\tau_i^{\kappa}([D, D']) = [D, \tau_i^{\kappa}(D')]_{\extendedtoroidal} - [D', \tau_i^{\kappa}(D)]_{\extendedtoroidal} - \sigma_i(D, D')$$
                is satisfied for all $D, D' \in \divzero$. Without loss of generality, we can assume again that $D, D' \in \divzero$ are basis elements.
                \begin{enumerate}
                    \item Firstly, let us verify if it is true that:
                        \begin{equation} \label{equation: coboundary_equation_D_rs_D_ab_verification}
                            \tau_i^{\kappa}([D_{a, b}, D_{r, s}]) = [D_{a, b}, \tau_i^{\kappa}(D_{r, s})]_{\extendedtoroidal} - [D_{r, s}, \tau_i^{\kappa}(D_{a, b})]_{\extendedtoroidal} - \sigma_i(D_{a, b}, D_{r, s})
                        \end{equation}
                    \begin{enumerate}
                        \item To begin, not that when $a = r = 0$, the LHS and the RHS of equation \eqref{equation: coboundary_equation_D_rs_D_ab_verification} will both be $0$ (cf. equation \eqref{equation: coboundary_equation_D_rs_D_ab}), and hence equal to one another.
                        \item When $a + r \not = 0$ and $a, r \not = 0$, 
                        \todo[inline]{I've checked this to be true, but haven't had the time to type it up.}
                        \item When $r + a = 0$ but $b + s + 2 \not = 0$, the LHS will be equal to:
                            $$
                                \begin{aligned}
                                    & \tau_2^{\kappa}([D_{-r, b}, D_{r, s}])
                                    \\
                                    = & \left( \lambda_{r, s} \left( r(s + 2) + br \right) + \lambda_{-r, b} \left( (b + 2) r + rs \right) + N_2(r, s, -r, b)\left( r(b + 1) + r(s + 1) \right) \right) K_{0, -b - s - 2}
                                    \\
                                    = & r (b + s + 2) \left( (\lambda_{r, s} + \lambda_{r, b}) - r^2 \right) K_{0, -b - s - 2}
                                \end{aligned}
                            $$
                        while the RHS will become:
                            $$
                                \begin{aligned}
                                    & [D_{-r, b}, \tau_i^{\kappa}(D_{r, s})]_{\extendedtoroidal} - [D_{r, s}, \tau_i^{\kappa}(D_{-r, b})]_{\extendedtoroidal} - \sigma_i(D_{-r, b}, D_{r, s})
                                    \\
                                    = & r (b + s + 2) \left( r^2 - \kappa + N_2(r, s, -r, b) \right) K_{0, -b - s - 2}
                                    \\
                                    = & r (b + s + 2) \left( r^2 - \kappa - r^2 \right) K_{0, -b - s - 2}
                                    \\
                                    = & -r (b + s + 2) \kappa K_{0, -b - s - 2}
                                \end{aligned}
                            $$
                        For equation \eqref{equation: coboundary_equation_D_rs_D_ab_verification} to be true, we must then have that:
                            $$\lambda_{r, s} + \lambda_{r, b} - r^2 = -\kappa$$
                        Because we have covered the case $a = r = 0$ (which does give $a + r = 0$), let us assume now that $r \not = 0$. In that case, the equation above will become:
                            $$r^2 - \kappa - r^2 = -\kappa$$
                        which is clearly true.
                        \item 
                    \end{enumerate}
                    \item \todo[inline]{The coboundary equation works for the pairs $D_v, D_{r, s}$ and $D_t, D_{r, s}$. Needs to be typed out. The proofs are more-or-less identical}
                    \item \todo[inline]{The pair $D_v, D_t$ works too.}
                \end{enumerate}
            \end{proof}
        \begin{remark}[$\gamma$-invariance is \textit{not} a cohomological property]
            Through proposition \ref{prop: invariance_of_billig_toroidal_cocycles}, we have seen that the toroidal $2$-cocycle:
                $$\sigma_2$$
            is \textit{not} $\gamma$-invariant in the sense of definition \ref{def: yangian_toroidal_cocycles}. However, in light of proposition \ref{prop: cohomological_non_triviality_of_billig_toroidal_cocycles}, we now see that this non-$\gamma$-invariance of $\sigma_2$ holds \textit{despite} the fact that it is cohomologous to $0$. Therefore, one can not conclude from observing that a toroidal $2$-cocycle $\sigma$ is cohomologous to $0$ that $\sigma$ is $\gamma$-invariant, but when $\sigma = 0$ as elements of $Z^2_{\Lie}(\divzero, \z(\toroidal))$, then it will indeed be true that $\sigma$ is $\gamma$-invariant, as we know that the semi-direct product $\toroidal \rtimes \divzero$ is an instance of a $\gamma$-extended toroidal Lie algebra.
        \end{remark}