\section{Examples of \texorpdfstring{$\gamma$}{}-extended toroidal Lie algebras}
    \subsection{Explicit examples of $\gamma$-invariant toroidal cocycles}
        For convenience, let us fix the following terminologies.
        \begin{definition}[$\gamma$-invariant toroidal $2$-cocycles] \label{def: yangian_toroidal_cocycles}
            Any $2$-cocyle $\sigma \in Z^2_{\Lie}(\divzero, \toroidal)$ shall be referred to as a \textbf{toroidal $2$-cocycle}.
            
            Any toroidal $2$-cocycle $\sigma$ such that $\toroidal \rtimes^{\sigma} \divzero$ is a $\gamma$-extended toroidal Lie algebra shall be called a \textbf{$\gamma$-invariant toroidal $2$-cocycle}.
        \end{definition}
        
        Now that we have a criterion for a given toroidal $2$-cocycle to be Yangian, let us apply it to some known toroidal $2$-cocycles from \cite{billig_energy_momentum_tensor} to check whether or not they are Yangian. 
        \begin{example} \label{example: yangian_cocycles_(counter)_examples}
            From proposition \ref{prop: lie_bracket_on_orthogonal_complement_of_toroidal_centre}, we know that:
                $$[\divzero, \divzero]_{\extendedtoroidal} \subset \z(\toroidal) \oplus \divzero$$
            we can obtain some toroidal $2$-cocyles:
                $$\sigma: \bigwedge^2 \divzero \to \z(\toroidal)$$
            by restricting $2$-cocycles of $\der(A)$ with values in $\z(\toroidal)$.

            It pays to abstract the situation out to the $n$-variable case momentarily. In \cite[p. 5, below Equation 1.3]{billig_energy_momentum_tensor}, it was noted that there are at least $2$-cocyles that we shall denote by:
                $$\sigma_1, \sigma_2: \bigwedge^2 \der(\bbC[v_1^{\pm 1}, ..., v_n^{\pm 1}]) \to \bar{\Omega}^1_{\bbC[v_1^{\pm 1}, ..., v_n^{\pm 1}]/\bbC}$$
            which are given in terms of the basis:
                $$\{ v_1^{m_1} ... v_n^{m_n} \cdot v_p \del_{v_p} \}_{(m_1, ..., m_n, p) \in \Z^n \x \Z}$$
            of $\der(\bbC[v_1^{\pm 1}, ..., v_n^{\pm 1}])$ by:
                $$\sigma_1(v_1^{m_1} ... v_n^{m_n} \cdot v_a \del_{v_a}, v_1^{r_1} ... v_n^{r_n} \cdot v_b \del_{v_b}) := r_a m_b \cdot v_1^{m_1} ... v_n^{m_n} \bar{d}( v_1^{r_1} ... v_n^{r_n} )$$
                $$\sigma_2(v_1^{m_1} ... v_n^{m_n} \cdot v_a \del_{v_a}, v_1^{r_1} ... v_n^{r_n} \cdot v_b \del_{v_b}) := r_b m_a \cdot v_1^{m_1} ... v_n^{m_n} \bar{d}( v_1^{r_1} ... v_n^{r_n} )$$
            given for every $1 \leq a, b \leq n$ and every $(m_1, ..., m_n), (r_1, ..., r_n) \in \Z^n$. 

            Now, back to the $2$-variable case, i.e. $n = 2$, where we have:
                $$v_1 := v, v_2 := t$$
            In this setting, we know how the basis elements $D_{r, s}, D_v, D_t$ of $\divzero$ are given in terms of $A$-multiples of the partial derivatives $\del_v, \del_t$ (cf. lemma \ref{lemma: yangian_div_zero_vector_fields_basic_properties}), so we can exploit the bilinearity of $2$-cocycles in order to see how $\sigma_1, \sigma_2$ act on elements of $\divzero$. Recall from lemma \ref{lemma: yangian_div_zero_vector_fields_basic_properties} that, in terms of the partial derivatives $\del_v, \del_t$, the basis elements of $\divzero$ are given by:
                $$\forall (r, s) \in \Z^2: D_{r, s} = s v^{-r + 1} t^{-s - 1} \del_v - r v^{-r} t^{-s} \del_t$$
                $$D_v = -v t^{-1} \del_v$$
                $$D_t = -\del_t$$
            Knowing this allows us to perform the following computations, where $i \in \{1, 2\}$:
            \begin{enumerate}
                \item 
                    $$
                        \begin{aligned}
                            & \sigma_i(D_{r, s}, D_{a, b})
                            \\
                            = & \sigma_i( -s v^{-r + 1} t^{-s - 1} \del_v + r v^{-r} t^{-s} \del_t, -b v^{-a + 1} t^{-b - 1} \del_v + a v^{-a} t^{-b} \del_t )
                            \\
                            = & \sigma_i( s v^{-r} t^{-s - 1} \cdot v\del_v - r v^{-r} t^{-s - 1} \cdot t \del_t, b v^{-a} t^{-b - 1} \cdot v\del_v - a v^{-a} t^{-b - 1} \cdot t \del_t )
                            \\
                            = & s \sigma_i( v^{-r} t^{-s - 1} \cdot v\del_v, b v^{-a} t^{-b - 1} \cdot v\del_v - a v^{-a} t^{-b - 1} \cdot t \del_t ) - r \sigma_i( v^{-r} t^{-s - 1} \cdot t \del_t, b v^{-a} t^{-b - 1} \cdot v\del_v - a v^{-a} t^{-b - 1} \cdot t \del_t )
                            \\
                            = &
                            \begin{aligned}
                                & s b \cdot \sigma_i( v^{-r} t^{-s - 1} \cdot v\del_v, v^{-a} t^{-b - 1} \cdot v\del_v )
                                \\
                                - & s a \cdot \sigma_i( v^{-r} t^{-s - 1} \cdot v\del_v, v^{-a} t^{-b - 1} \cdot t \del_t )
                                \\
                                - & r b \cdot \sigma_i( v^{-r} t^{-s - 1} \cdot t \del_t, v^{-a} t^{-b - 1} \cdot v\del_v )
                                \\
                                + & r a \cdot \sigma_i( v^{-r} t^{-s - 1} \cdot t \del_t, v^{-a} t^{-b - 1} \cdot t \del_t )
                            \end{aligned}
                            \\
                            = & N_i(r, s, a, b) v^{-r} t^{-s - 1} \bar{d}( v^{-a} t^{-b - 1} )
                        \end{aligned}
                    $$
                where:
                    $$
                        \begin{aligned}
                            N_i(r, s, a, b) & = 
                            sbra
                            - sa \left( \delta_{i, 1} a(s + 1) + \delta_{i, 2} (b + 1) r \right) 
                            - rb \left( \delta_{i, 1} (b + 1) r + \delta_{i, 2} a (s + 1) \right)
                            + r a (s + 1) (b + 1)
                            \\
                            & = 
                            \begin{cases}
                                \text{$
                                    sbra
                                    - s a^2 (s + 1) 
                                    - r^2 b (b + 1)
                                    + r a (s + 1) (b + 1)
                                $if $i = 1$}
                                \\
                                \text{$
                                    sbra
                                    - sa (b + 1) r
                                    - rb a (s + 1)
                                    + r a (s + 1) (b + 1)
                                $ if $i = 2$}
                            \end{cases}
                            \\
                            & = 
                            \begin{cases}
                                \text{$
                                    sbra
                                    - ( (sa)^2 + s a^2 ) 
                                    - ( (rb)^2 + r^2 b ) 
                                    + rasb + rsa + rab + ra
                                $ if $i = 1$}
                                \\
                                \text{$
                                    sbra
                                    - (sabr + sar)
                                    - (rbas + rba)
                                    + rasb + rsa + rab + ra
                                $ if $i = 2$}
                            \end{cases}
                            \\
                            & = 
                            \begin{cases}
                                \text{$2 rsab - ( (sa)^2 + s a^2 ) - ( (rb)^2 + r^2 b ) + rsa + rab + ra$ if $i = 1$}
                                \\
                                \text{$ra$ if $i = 2$}
                            \end{cases}
                        \end{aligned}
                    $$
                
                Now, recall from example \ref{example: toroidal_lie_algebras_centres} that any element:
                    $$v^n t^q \bar{d}(v^m t^p) \in \z(\toroidal)$$
                can be written in terms of the basis elements of $\z(\toroidal)$ in the following manner:
                    $$v^n t^q \bar{d}(v^m t^p) = \delta_{(m, p) + (n, q), (0, 0)} ( n c_v + q c_t ) + (np - mq) K_{m + n, p + q}$$
                Using this, we shall be able to conclude that:
                    $$\sigma_i(D_{r, s}, D_{a, b}) = N_i(r, s, a, b) \left( -\delta_{(r, s), -(a, b)} (r c_v + (s + 1) c_t) + ( r(b + 1) - a(s + 1) )K_{-r - a, -s - b - 2} \right)$$
                \item
                \todo[inline]{Checked $\sigma_i(D_{r, s}, D_v)$ and $\sigma_i(D_{r, s}, D_t)$. No more $\delta_{i, 1}$ nor $\delta_{i, 2}$ terms. Fixed some sign typos.}
                    $$
                        \begin{aligned}
                            & \sigma_i(D_{r, s}, D_v)
                            \\
                            = & \sigma_i( -s v^{-r + 1} t^{-s - 1} \del_v + r v^{-r} t^{-s} \del_t, -v t^{-1} \del_v )
                            \\
                            = & \sigma_i( s v^{-r} t^{-s - 1} \cdot v \del_v - r v^{-r} t^{-s - 1} t \del_t, t^{-1} \cdot v \del_v )
                            \\
                            = & s \sigma_i( v^{-r} t^{-s - 1} \cdot v \del_v, t^{-1} \cdot v \del_v ) - r\sigma_i( v^{-r} t^{-s - 1} \cdot t \del_t, t^{-1} \cdot v \del_v )
                            \\
                            = &
                            \begin{cases}
                                \text{$s \left( 0 \cdot (-r) \cdot v^{-r} t^{-s - 1} \bar{d}(t^{-1}) \right) - r \left( (-1) \cdot (-r) \cdot v^{-r} t^{-s - 1} \bar{d}(t^{-1}) \right)$ if $i = 1$}
                                \\
                                \text{$s \left( (-r) \cdot 0 \cdot v^{-r} t^{-s - 1} \bar{d}(t^{-1}) \right) - r \left( (-r) \cdot (-1) \cdot v^{-r} t^{-s - 1} \bar{d}(t^{-1}) \right)$ if $i = 2$}
                            \end{cases}
                            \\
                            = & -r^2 v^{-r} t^{-s - 1} \bar{d}(t^{-1})
                        \end{aligned}
                    $$
                Now, recall from example \ref{example: toroidal_lie_algebras_centres} that any element:
                    $$v^n t^q \bar{d}(v^m t^p) \in \z(\toroidal)$$
                can be written in terms of the basis elements of $\z(\toroidal)$ in the following manner:
                    $$v^n t^q \bar{d}(v^m t^p) = \delta_{(m, p) + (n, q), (0, 0)} ( n c_v + q c_t ) + (np - mq) K_{m + n, p + q}$$
                Using this, we shall get that:
                    $$
                        \begin{aligned}
                            & \sigma_i(D_{r, s}, D_v)
                            \\
                            = & -r^2 \left( -\delta_{(r, s), (0, -2)} ( r c_v + (s + 3) c_t ) - r K_{-r, -s - 2} \right)
                            \\
                            = &
                            \begin{cases}
                                \text{$0$ if $(r, s) \in \{0\} \x \Z$}
                                \\
                                \text{$r^3 K_{-r, -s - 2}$ if $(r, s) \in (\Z \setminus \{0\}) \x \Z$}
                            \end{cases}
                            \\
                            = & r^3 K_{-r, -s - 2}
                        \end{aligned}
                    $$
                \item Using similar methods, we shall get that:
                    $$\sigma_i(D_{r, s}, D_t) = r^2s K_{-r, -s - 2}$$
                \item
                    $$
                        \begin{aligned}
                            & \sigma_i(D_v, D_t)
                            \\
                            = & \sigma_i(-v t^{-1} \del_v, -\del_t)
                            \\
                            = & \sigma_i(t^{-1} \cdot v \del_v, t^{-1} t \del_t)
                            \\
                            = & 0
                        \end{aligned}
                    $$
            \end{enumerate}
            One can now use the criterion given in lemma \ref{lemma: yangian_criterion_for_toroidal_cocycles} to verify whether or not the cocycles $\sigma_1, \sigma_2$ are $\gamma$-invariant in the sense of definition \ref{def: yangian_toroidal_cocycles}.
        \end{example}
        \todo[inline]{Moved the proof of Yangian-ness into a proposition.}
        \begin{proposition}[Yangian-ness of Billig's toroidal $2$-cocycles] \label{prop: yangianness_of_billig_toroidal_cocycles}
            Regarding the toroidal $2$-cocycles:
                $$\sigma_1, \sigma_2 \in Z^2_{\Lie}(\divzero, \z(\toroidal))$$
            given as in example \ref{example: yangian_cocycles_(counter)_examples}, the former is Yangian, while the latter is not.
        \end{proposition}
            \begin{proof}
                \begin{enumerate}
                    \item Firstly, using the fact that:
                        $$\sigma_i(D_{r, s}, D_{a, b}) = N_i(r, s, a, b) \left( -\delta_{(r, s), -(a, b)} (r c_v + (s + 1) c_t) + ( r(b + 1) - a(s + 1) )K_{-r - a, -s - b - 2} \right)$$
                    we shall get that:
                        $$
                            \left( \sigma_i(D_{r, s}, D_{a, b}), D \right)_{\extendedtoroidal} =
                            \begin{cases}
                                \text{$N_i(r, s, a, b) ( r(b + 1) - a(s + 1) ) \delta_{(-r - a, -s - b - 2), (\alpha, \beta)}$ if $D = D_{\alpha, \beta}$}
                                \\
                                \text{$-N_i(r, s, a, b) \delta_{(r, s), -(a, b)} r$ if $D = D_v$}
                                \\
                                \text{$-N_i(r, s, a, b) \delta_{(r, s), -(a, b)} (s + 1)$ if $D = D_t$}
                            \end{cases}
                        $$
                    At the same time, using the fact that:
                        $$\sigma_i(D_{a, b}, D_v) = a^3 K_{-a, -b - 2}$$
                        $$\sigma_i(D_{a, b}, D_t) = a^2b K_{-a, -b - 2}$$
                    we have that:
                        $$
                            \begin{aligned}
                                \left( D_{r, s}, \sigma_i(D_{a, b}, D) \right)_{\extendedtoroidal} =
                                \begin{cases}
                                    \text{$N_i(r, s, \alpha, \beta) ( r(\beta + 1) - \alpha(s + 1) ) \delta_{(r, s), (-a - \alpha, -b - \beta - 2)}$ if $D = D_{\alpha, \beta}$}
                                    \\
                                    \text{$a^3 \delta_{(r, s), (-a, -b - 2)}$ if $D = D_v$}
                                    \\
                                    \text{$a^2 b \delta_{(r, s), (-a, -b - 2)}$ if $D = D_t$}
                                \end{cases}
                            \end{aligned}
                        $$
                    We can thus conclude immediately that $\sigma_2$ is \textit{not} invariant, as:
                        $$\left( \sigma_2(D_{r, s}, D_{a, b}), D \right)_{\extendedtoroidal} \not = \left( D_{r, s}, \sigma_2(D_{a, b}, D) \right)_{\extendedtoroidal}$$
                    when $D \in \{D_v, D_t\}$. As such, let us focus on $\sigma_1$ from now on, for which we now have:
                        $$\left( \sigma_1(D_{r, s}, D_{a, b}), D \right)_{\extendedtoroidal} \not = \left( D_{r, s}, \sigma_1(D_{a, b}, D) \right)_{\extendedtoroidal}$$
                    for all $D \in \divzero$.
                    \item Secondly, using the fact that:
                        $$\sigma_1(D_{r, s}, D_v) = r^3 K_{-r, -s - 2}$$
                    we shall get that:
                        $$
                            \left( \sigma_1(D_{r, s}, D_v), D \right)_{\extendedtoroidal} =
                            \begin{cases}
                                \text{$r^3 \delta_{(-r, -s - 2), (\alpha, \beta)}$ if $D = D_{\alpha, \beta}$}
                                \\
                                \text{$0$ if $D = D_v$}
                                \\
                                \text{$0$ if $D = D_t$}
                            \end{cases}
                        $$
                    At the same time, knowing that:
                        $$\sigma_1(D_v, D_t) = 0$$
                    we see that:
                        $$
                            \begin{aligned}
                                \left( D_{r, s}, \sigma_1(D_v, D) \right)_{\extendedtoroidal} =
                                \begin{cases}
                                    \text{$-\alpha^3 \delta_{(r, s), (-\alpha, -\beta - 2)}$ if $D = D_{\alpha, \beta}$}
                                    \\
                                    \text{$0$ if $D = D_v$}
                                    \\
                                    \text{$0$ if $D = D_t$}
                                \end{cases}
                            \end{aligned}
                        $$
                    We thus have:
                        $$\left( \sigma_1(D_{r, s}, D_v), D \right)_{\extendedtoroidal} = \left( D_{r, s}, \sigma_1(D_v, D) \right)_{\extendedtoroidal}$$
                    for all $D \in \divzero$.
                    \item Next, by using the fact that:
                        $$\sigma_1(D_{r, s}, D_t) = r^2 s K_{-r, -s - 2}$$
                    we shall get that:
                        $$
                            \left( \sigma_1(D_{r, s}, D_t), D \right)_{\extendedtoroidal} =
                            \begin{cases}
                                \text{$r^2 s \delta_{(-r, -s - 2), (\alpha, \beta)}$ if $D = D_{\alpha, \beta}$}
                                \\
                                \text{$0$ if $D = D_v$}
                                \\
                                \text{$0$ if $D = D_t$}
                            \end{cases}
                        $$
                    At the same time, we have that:
                        $$
                            \begin{aligned}
                                \left( D_{r, s}, \sigma_1(D_t, D) \right)_{\extendedtoroidal} =
                                \begin{cases}
                                    \text{$-\alpha^2 \beta \delta_{(r, s), (-\alpha, -\beta - 2)}$ if $D = D_{\alpha, \beta}$}
                                    \\
                                    \text{$0$ if $D = D_v$}
                                    \\
                                    \text{$0$ if $D = D_t$}
                                \end{cases}
                            \end{aligned}
                        $$
                    By combining these two observations, one is able to conclude furthermore that:
                        $$\left( \sigma_1(D_{r, s}, D_t), D \right)_{\extendedtoroidal} = \left( D_{r, s}, \sigma_1(D_t, D) \right)_{\extendedtoroidal}$$
                    for all $D \in \divzero$.
                    \item Lastly, since:
                        $$\sigma_1(D_v, D_t) = 0$$
                    we automatically have that:
                        $$( \sigma_1(D_v, D_t), D )_{\extendedtoroidal} = ( D_v, \sigma_1(D_t, D) )_{\extendedtoroidal}$$
                    for all $D \in \divzero$.
                \end{enumerate}
                We have therefore shown that $\sigma_1$ is $\gamma$-invariant in the sense of definition \ref{def: yangian_toroidal_cocycles}. 
            \end{proof}

    \subsection{Cohomological (non-)triviality}
        Whether or not these cocycles might be cohomologous to $0$ (cf. definition \ref{def: lie_algebra_cohomology}) - and hence whether or not they might give rise to extensions that are isomorphic to the semi-direct product $\toroidal \rtimes \divzero$ - is a much subtler issue. One way to tackle this problem is to \textit{firstly} check whether or not their restrictions to a particular Lie subalgebra of $\extendedtoroidal$ is cohomologous to $0$. We remark right away that simply checking that these restrictions are non-cohomologous to $0$ is \textit{not} sufficient for concluding that $\sigma_1$ and $\sigma_2$ are non-zero elements of $H^2_{\Lie}(\divzero, \z(\toroidal))$.
        
        Recall from lemma \ref{lemma: yangian_div_zero_vector_fields_basic_properties} that:
            $$\forall (r, s) \in \Z^2: D_{r, s} = s v^{-r + 1} t^{-s - 1} \del_v - r v^{-r} t^{-s} \del_t$$
            $$D_v = -v t^{-1} \del_v$$
            $$D_t = -\del_t$$
        and from lemma \ref{lemma: yangian_div_zero_vector_fields_basic_properties} that the commutation relations that these basis elements of $\divzero$ satisfy are:
            $$[D_v, D_t] = 0$$
            $$[D_v, D_{r, s}] = r D_{r, s + 1}$$
            $$[D_t, D_{r, s}] = D_{r, s + 1}$$
            $$[D_{a, b}, D_{r, s}] = (br - sa) D_{a + r, b + s + 1}$$
        (given for all $(r, s), (a, b) \in \Z^2$). With these information in mind, one sees that the following vector subspace of $\divzero$:
            $$\frakw := \bigoplus_{r \in \Z} \bbC D_{r, -1}$$
        is actually a Lie subalgebra, as the basis elements satisfy the following commutators:
            $$[D_{a, -1}, D_{r, -1}] = (a - r) D_{a + r, -1}$$
        given for all $a, r \in \Z$; note also that we have $D_v, D_t \not \in \frakw$ because:
            $$[D_v, D_{r, -1}] = r D_{r, 0} \not \in \frakw$$
            $$[D_t, D_{r, -1}] = D_{r, 0} \not \in \frakw$$    
        for all $r \in \Z$. Interestingly, these are precisely the commutation relations satisfied by the elements of the following basis of the Lie algebra $\der(\bbC[v^{\pm 1}])$:
            $$\{ d_r := -v^r D_{\aff} \}_{r \in \Z}$$
        (where $D_{\aff} := v \frac{d}{dv}$ is the \say{untwisted affine Kac-Moody derivation} as in subsection \ref{subsection: a_fixed_untwisted_affine_kac_moody_algebra}) and in light of this, we make the following observation:
        \begin{lemma}[A copy of the Witt algebra inside $\divzero$] \label{lemma: a_copy_of_the_witt_algebra_inside_the_lie_algebra_of_yangian_div_zero_vector_fields}
            There is an isomorphism of Lie algebras:
                $$\der(\bbC[v^{\pm 1}]) \xrightarrow[]{\cong} \frakw$$
            given by:
                $$d_r \mapsto D_{r, -1}$$
            This identifies a copy of $\der(\bbC[v^{\pm 1}])$ inside $\divzero$ as a Lie subalgebra. 
        \end{lemma}

        The Lie algebra $\der(\bbC[v^{\pm 1}])$ - commonly called the \textbf{Witt algebra} - is known to possess a \textit{non-trivial} UCE:
            $$\frakv := \der(\bbC[v^{\pm 1}]) \oplus^{\eta} \bbC c_{\frakv}$$
        called the \textbf{Virasoro algebra} (for more details, see subsection \ref{subsection: virasoro_algebra}), whose corresponding $2$-cocycle, which will often be referred to as the \textbf{Virasoro $2$-cocycle}:
            $$\eta: \bigwedge^2 \der(\bbC[v^{\pm 1}]) \to \bbC c_{\frakv}$$
        is given by:
            $$\eta(d_r, d_a) := \delta_{r + a, 0} (r^3 - r) c_{\frakv}$$
        for all $r, a \in \Z$; in particular, this means this \textit{$\eta$ is non-cohomologous to $0$}. The $2$-cocycle $\eta$ is cohomologically unique (in the sense that any $2$-cocycle $\eta': \bigwedge^2 \der(\bbC[v^{\pm 1}]) \to \bbC c_{\frakv}$ is cohomologous to $\eta$ itself; lemma \ref{lemma: H^2_of_witt_algebra}), and so should either $\sigma_1$ or $\sigma_2$ become cohomologous to $\eta$ after being restricted down to $\bigwedge^2 \frakw$, they would have to be non-cohomologous to $0$. We claim that this is indeed true.

        The answer to the following question turns out to be negative, but in answering it, we will have gained some insight into how we might show that $\sigma_1$ and $\sigma_2$ are actually cohomologous to $\eta$ after having their domains restricted to $\bigwedge^2 \frakw$.
        \begin{question}
            Is it true that:
                $$\sigma_i|_{ \bigwedge^2 \frakw } = \eta$$
            as elements of $Z^2_{\Lie}(\frakw, \bbC c_{\frakv}$ for either $i = 1$ or $i = 2$ ?
        \end{question}
        Using the computations in example \ref{example: yangian_cocycles_(counter)_examples}, we see that:
            $$\sigma_i(D_{r, -1}, D_{a, -1}) = -N_i(r, -1, a, -1) \delta_{r + a, 0} r c_v$$
        where:
            $$
                N_i(r, -1, a, -1) =
                \begin{cases}
                    \text{$2 ra - ra - ra + ra$ if $i = 1$}
                    \\
                    \text{$ra$ if $i = 2$}
                \end{cases}
                = ra
            $$
        and hence, more succinctly, we have that:
            $$\sigma_i(D_{r, -1}, D_{a, -1}) = -\delta_{r + a, 0} r^3 c_v$$
        regardless of whether $i = 1$ or $i = 2$, and for all $r, a \in \Z$. If it was true that:
            $$\sigma_i(D_{r, -1}, D_{a, -1}) = \eta(d_r, d_a)$$
        then we must have that:
            $$-\delta_{r + a, 0} r^3 c_v = \delta_{r + a, 0} (r^3 - r) c_{\frakv}$$
        which implies that:
            $$-r^3 c_v = (r^3 - r) c_{\frakv}$$
        for all $r \in \Z$. But this is certainly not true for all $r \in \Z$: e.g. if $r = 1$ then we will get that:
            $$-c_v = -v^{-1} \bar{d}v = 0$$
        which is absurd! As such, \textit{neither} of the $2$-cocycles $\sigma_1$ and $\sigma_2$ coincide with $\eta$ when restricted down to $\bigwedge^2 \frakv$. In particular, this means that it is still inconclusive as to whether or not the toroidal $2$-cocycles $\sigma_1, \sigma_2$ are cohomologous to $0$. However, this does not necessarily imply that $\eta$ and $\sigma_1, \sigma_2$ are \textit{not} cohomologous. 

        \begin{proposition}[A Virasoro $2$-coboundary] \label{prop: a_virasoro_coboundary}
            Let:
                $$\eta': \bigwedge^2 \der(\bbC[v^{\pm 1}]) \to \bbC c_{\frakv}$$
            be the function given by:
                $$\eta'(d_r, d_a) := \delta_{r + a, 0} r c_{\frakv}$$
            This is a $2$-cocycle of $\der(\bbC[v^{\pm 1}])$ with values in $\bbC c_{\frakv}$. From this, one sees that:
                $$\eta + \eta': \bigwedge^2 \der(\bbC[v^{\pm 1}]) \to \bbC c_{\frakv}$$
            (which is given by $(\eta + \eta')(d_r, d_a) := \delta_{r + a, 0} r^3 c_{\frakv}$) is also a $2$-cocycle of $\der(\bbC[v^{\pm 1}])$ with values in $\bbC c_{\frakv}$. 

            Furthermore, we have that:
                $$\eta' \in B^2_{\Lie}( \der(\bbC[v^{\pm 1}]), \bbC c_{\frakv} )$$
            with notations as in definition \ref{def: lie_cocycles_and_coboundaries}.
        \end{proposition}
            \begin{proof}
                It is clear from the construction of $\eta'$ that it is linear and skew-symmetric; the only non-trivial thing to prove is that $\eta'$ satisfies the Jacobi identity in the sense of definition \ref{def: twisted_semi_direct_products}. To this end, simply consider the following, for all $i, j, k \in \Z$:
                    $$
                        \begin{aligned}
                            & \eta'([d_i, d_j], d_k) + \eta'([d_k, d_i], d_j) + \eta'([d_j, d_k], d_i)
                            \\
                            = & (i - j) \eta'(d_{i + j}, d_k) + (k - i) \eta'(d_{k + i}, d_j) + (j - k) \eta'(d_{j + k}, d_i)
                            \\
                            = & \delta_{i + j + k, 0} \left( (i - j) (i + j) + (k - i) (k + i) + (j - k) (j + k) \right) c_{\frakv}
                            \\
                            = & 0
                        \end{aligned}
                    $$
                    
                To show that $\eta'$ is a Lie $2$-coboundary in the sense of definition \ref{def: lie_cocycles_and_coboundaries}, we must show that there exists a linear map:
                    $$\tilde{\eta'}: \der(\bbC[v^{\pm 1}]) \to \bbC c_{\frakv}$$
                which is merely a linear map, such that:
                    $$\eta(d_i, d_j) = \tilde{\eta'}([d_i, d_j])$$
                (cf. examples \ref{example: lie_cocycles_and_coboundaries_with_trivial_coefficients} and \ref{example: low_degree_lie_cocycles_and_coboundaries_with_trivial_coefficients}). The RHS is nothing but:
                    $$\tilde{\eta'}([d_i, d_j]) = (i - j) \tilde{\eta'}(d_{i + j}) c_{\frakv}$$
                while by construction, the LHS is:
                    $$\eta(d_i, d_j) := \delta_{i + j, 0} i c_{\frakv}$$
                and hence:
                    $$\delta_{i + j, 0} i = (i - j) \tilde{\eta'}(d_{i + j})$$
                By setting $j = 0$, we then see that:
                    $$i (\tilde{\eta'}(d_i) - \delta_{i, 0}) = 0$$
                for all $i \in \Z$, which in turn implies that:
                    $$\tilde{\eta'}(d_i) = \delta_{i, 0}$$
                The sought-for linear map $\tilde{\eta}: \der(\bbC[v^{\pm 1}]) \to \bbC c_{\frakv}$ is thus defined, and hence exists.
            \end{proof}
        Using the well-known fact that:
            $$H^2_{\Lie}(\der(\bbC[v^{\pm 1}]), \bbC c_{\frakv}) \cong \bbC \eta$$
        (see lemma \ref{lemma: H^2_of_witt_algebra}) meaning that up to isomorphisms of extensions (cf. definition \ref{def: lie_algebra_extensions}), the Virasoro algebra corresponding to the $2$-cocycle $\eta: \der(\bbC[v^{\pm 1}]) \to \bbC c_{\frakv}$ is unique, one can infer from proposition \ref{prop: a_virasoro_coboundary} that, precisely because we have that:
            $$\sigma_i|_{ \bigwedge^2 \frakv } = \eta + \eta'$$
        the domain restrictions $\sigma_1|_{ \bigwedge^2 \frakv }$ must be cohomologous to the $2$-cocycle $\eta$, which is known to be non-cohomologous to $0$ (cf. proposition \ref{prop: non_zero_yangian_cocycles_on_witt_algebra}). 
        \begin{proposition} \label{prop: non_zero_yangian_cocycles_on_witt_algebra}
            The Lie $2$-cocycle:
                $$\eta + \eta' \in Z^2(\der(\bbC[v^{\pm 1}]), \bbC c_{\frakv})$$
            which given by:
                $$(\eta + \eta')(d_r, d_a) = \delta_{r + a, 0} r^3 c_{\frakv}$$
            - for all $r, a \in \Z$ and with $\eta'$ as in proposition \ref{prop: a_virasoro_coboundary} - is not cohomologous to $0$, i.e. its image under the canonical projection $Z^2(\der(\bbC[v^{\pm 1}]), \bbC c_{\frakv}) \to H^2(\der(\bbC[v^{\pm 1}]), \bbC c_{\frakv})$ (cf. definition \ref{def: complexes_and_cohomology}) is non-zero.
        \end{proposition}
            \begin{proof}
                Because we know that $\dim_{\bbC} H^2(\der(\bbC[v^{\pm 1}]), \bbC c_{\frakv}) = 1$ (cf. lemma \ref{lemma: H^2_of_witt_algebra}) and that $\eta' \in B^2(\der(\bbC[v^{\pm 1}]), \bbC c_{\frakv})$ (cf. proposition \ref{prop: a_virasoro_coboundary}), $\eta + \eta'$ must be cohomologous to $\eta$, which is not cohomologous to $0$ as shown in the proof of lemma \ref{lemma: H^2_of_witt_algebra}.
            \end{proof}
        \begin{remark}[The Virasoro central element]
            Thanks to proposition \ref{prop: non_zero_yangian_cocycles_on_witt_algebra}, we can also identify:
                $$c_{\frakv} = -c_v$$
            This tells us that:
                $$\frakv \cong \bbC \cdot (-c_v) \rtimes^{\sigma_i} \frakw$$
            for either $i = 1$ or $i = 2$, thereby identifying a copy of the Virasoro algebra as a Lie subalgebra inside $\divzero$.
        \end{remark}

        Now, even though it might be tempting to conclude right away that because of proposition \ref{prop: non_zero_yangian_cocycles_on_witt_algebra}, which tells us that:
            $$\sigma_i|_{\bigwedge^2 \frakw}: \bigwedge^2 \frakw \to \bbC c_{\frakv}$$
        is cohomologous to $\eta + \eta' \in Z^2_{\Lie}(\frakw, \bbC c_{\frakv})$, which is not $2$-coboundary, it must then also be true that $\sigma_i \not \in B^2_{\Lie}(\divzero, \z(\toroidal))$. However, the subtlety here is that because $\z(\toroidal)$ is non-trivial as a module over $\divzero$ (and likewise, over the Lie subalgebra $\frakw \subset \divzero$), unlike $\bbC c_{\frakv}$, one would have to actually check whether or not the restricted toroidal $2$-cocycle:
            $$\sigma_i|_{\bigwedge^2 \frakw}: \bigwedge^2 \divzero \to \z(\toroidal)$$
        is $2$-coboundary. Our claim is that, somewhat surprisingly, it is in fact $2$-coboundary, for both $i = 1$ and $i = 2$.
        \todo[inline]{$\sigma_1|_{\bigwedge^2 \frakw}$ and $\sigma_2|_{\bigwedge^2 \frakw}$ are $2$-coboundary.}
        \todo[inline]{Changed $\tilde{\sigma}_i$ to $\tau$, as I did not want the notation to suggest that $\tilde{\sigma}_i$ is somehow a canonical lift of $\sigma_i|_{\bigwedge^2 \frakw}$.}
        \todo[inline]{Revised proof. There were some sign errors, and I forgot to include a $c_v$ term in the expression for $\tau(D_{r, -1})$.}
        \begin{proposition} \label{prop: non_trivial_yangian_restricted_coboundaries_examples}
            Let $i \in \{1, 2\}$. As $\z(\toroidal)$ is a (\textit{non-trivial}) $\divzero$-module via Lie derivatives (cf. lemma \ref{lemma: derivation_action_on_toroidal_centres}), it is also a module over the Lie subalgebra $\frakw$ of $\divzero$, determined by the same action. Then:
                $$\sigma_i|_{ \bigwedge^2 \frakw } \in B^2_{\Lie}(\frakw, \z(\toroidal))$$
        \end{proposition}
            \begin{proof}
                Let $i \in \{1, 2\}$. Also, let us regard the $\divzero$-action on $\z(\toroidal)$ as a Lie algebra homomorphism:
                    $$\rho: \divzero \to \gl( \z(\toroidal) )$$
                Recall from lemma \ref{lemma: derivation_action_on_toroidal_centres} that this is given by:
                    $$\rho(D) := [D, -]_{\extendedtoroidal}$$
                for all $D \in \divzero$, which is well-defined because we know that $[\divzero, \z(\toroidal)]_{\extendedtoroidal} \subseteq \z(\toroidal)$ (cf. \textit{loc. cit.}).
            
                Per example \ref{example: low_degree_lie_coboundaries_with_non-trivial_coefficients}, it suffices to prove the existence of a linear map:
                    $$\tau: \divzero \to \z(\toroidal)$$
                (i.e. an element of $C_1(\divzero, \z(\toroidal))$ in the notations of remark \ref{remark: simplified_chevalley_eilenberg_complexes}) such that:
                    $$
                        \begin{aligned}
                            \sigma_i|_{\bigwedge^2 \frakw}(D, D') & = \rho(D) \cdot \tau(D') - \rho(D') \cdot \tau(D) - \tau([D, D'])
                            \\
                            & = [D, \tau(D')]_{\extendedtoroidal} - [D', \tau(D)]_{\extendedtoroidal} - \tau([D, D'])
                        \end{aligned}
                    $$
                We note that since $\sigma_1|_{\bigwedge^2 \frakw} = \sigma_2|_{\bigwedge^2 \frakw}$, the sought-for map $\tau$ should be independent of $i$. 
                    
                For what follows, let us recall:
                \begin{itemize}
                    \item from example \ref{example: yangian_cocycles_(counter)_examples} that:
                        $$\sigma_i|_{ \bigwedge^2 \frakw }(D_{r, -1}, D_{a, -1}) = -\delta_{r, -a} r^3 c_v$$
                    \item from lemma \ref{lemma: yangian_div_zero_vector_fields_basic_properties} that:
                        $$[D_{a, b}, D_{r, s}] = (br - sa) D_{a + r, b + s + 1}$$
                    \item and from lemma \ref{lemma: explicit_commutators_between_central_basis_elements_and_derivations} that:
                        $$[D_{r, s}, K_{\alpha, \beta}]_{\extendedtoroidal} = ((\beta - 1)r - s\alpha) K_{\alpha - r, \beta - s - 1} + \delta_{(r, s + 1), (\alpha, \beta)} \left( r c_v + s c_t \right)$$
                        $$[D, c_v]_{\extendedtoroidal} = [D, c_t]_{\extendedtoroidal} = 0$$
                    for all $D \in \divzero$.
                \end{itemize}
                Now, since it suffices to prove only that $\tau$ exists, let us assume that it is graded, which means that for every $\alpha \in \Z$, the following is true:
                    $$\tau(D_{\alpha, -1}) = \lambda_{\alpha} K_{-\alpha, 0} + \delta_{\alpha, 0}( \mu_{\alpha} c_v + \nu_{\alpha} c_t )$$
                for some $\lambda_{\alpha}, \mu_{\alpha}, \nu_{\alpha} \in \bbC$, which is because:
                    $$\deg D_{-\alpha, -\beta - 1} = \deg K_{\alpha, \beta} = (\alpha, \beta)$$
                    $$\deg c_v = \deg c_t = (0, 0)$$
                (cf. remark \ref{remark: Z^2_grading_on_toroidal_centres} and corollary \ref{coro: yangian_div_zero_vector_fields_are_graded}). When $D := D_{r, -1}$ and $D' := D_{a, -1}$, the above shall imply that:
                    $$
                        \begin{aligned}
                            & (r - a) \left( \lambda_{a + r} K_{-a - r, 0} + \delta_{a + r, 0}( \mu_{a + r} c_v + \nu_{a + r} c_t ) \right)
                            \\
                            = & (r - a) \tau( D_{r + a, -1} )
                            \\
                            = & \tau( [D_{r, -1}, D_{a, -1}] )
                            \\
                            = & [ D_{r, -1}, \tau(D_{a, -1}) ]_{\extendedtoroidal} - [ D_{a, -1}, \tau(D_{r, -1}) ]_{\extendedtoroidal} - \sigma_i|_{\bigwedge^2 \frakw}(D_{r, -1}, D_{a, -1})
                            \\
                            = & \lambda_a [ D_{r, -1}, K_{-a, 0} ]_{\extendedtoroidal} -  \lambda_r [ D_{a, -1}, K_{-r, 0} ]_{\extendedtoroidal} + \delta_{r + a, 0} r^3 c_v
                            \\
                            = &
                            \lambda_a \left(
                                -(a + r) K_{-a - r, 0} + \delta_{(r, 0), (-a, 0)} \left( r c_v - c_t \right)
                            \right)
                            +
                            \lambda_r \left(
                                -(a + r) K_{-a - r, 0} + \delta_{(a, 0), (-r, 0)} \left( a c_v - c_t \right)
                            +
                            \delta_{r + a, 0} r^3 c_v
                            \right)
                            \\
                            = &
                            -(a + r) ( \lambda_a + \lambda_r ) K_{-a - r, 0}
                            +
                            \delta_{a + r, 0} \left( (r \lambda_a + a \lambda_r + r^3) c_v - (\lambda_a + \lambda_r)c_t \right)
                        \end{aligned}
                    $$
                from which we gather that:
                    $$(r - a) \lambda_{a + r} = -(a + r) ( \lambda_a + \lambda_r )$$
                    $$\delta_{r + a, 0} (r - a) \mu_{a + r} = \delta_{r + a, 0} (r \lambda_a + a \lambda_r + r^3)$$
                    $$\delta_{r + a, 0} (r - a) \nu_{a + r} = -\delta_{r + a, 0}(\lambda_a + \lambda_r)$$
                When $r + a = 0$ and $r \not = 0$ (which together imply in particular that $r \not = a$), the second and third shall become:
                    $$2r \mu_0 = r \lambda_{-r} - r\lambda_r + r^3$$
                    $$2r \nu_0 = -\lambda_{-r} - \lambda_r$$
                from which we get the following system of linear equations:
                    $$
                        \begin{cases}
                            \lambda_{-r} - \lambda_r = -r^2 + 2\mu_0
                            \\
                            \lambda_{-r} + \lambda_r = -2\nu_0 r
                        \end{cases}
                    $$
                Solving this system yields:
                    $$\lambda_r = \frac12 r^2 - \nu_0 r - \mu_0$$
                As such, we have arrived at:
                    $$
                        \begin{aligned}
                            & \tau( D_{r, -1} )
                            \\
                            = & \left( \frac12 r^2 - \nu_0 r - \mu_0 \right) K_{-r, 0} + \delta_{r, 0} ( \mu_r c_v + \nu_r c_t )
                            \\
                            = & \left( \frac12 r^2 - \nu_0 r - \mu_0 \right) K_{-r, 0} + \delta_{r, 0} ( \mu_0 c_v + \nu_0 c_t )
                        \end{aligned}
                    $$
                where $\mu_0, \nu_0 \in \bbC$ are undetermined, and we have thus shown that linear maps $\tau \in C_1(\frakw, \z(\toroidal))$ such that $d_1^{\z(\toroidal)}(\tau) = \sigma_i|_{\bigwedge^2 \frakw}$ exist.
            \end{proof}

        The following result is our final conclusion regarding whether or not the toroidal $2$-cocycles $\sigma_1$ and $\sigma_2$ are cohomologous to $0$.
        \todo[inline]{After redoing the proof, it seems that both $\sigma_1$ and $\sigma_2$ are coboundary. What's a bit curious to me is that, $\tau_i$ is very similar to $\tau$ in proposition \ref{prop: non_trivial_yangian_restricted_coboundaries_examples}, but I'm not sure if anything important can be said about this.}
        \begin{theorem}[Are the toroidal $2$-cocycles $\sigma_1, \sigma_2$ cohomologous to $0$ or not ?] \label{theorem: non_trivial_yangian_cocycles_examples}
            Let $i \in \{1, 2\}$ and $\sigma_i \in Z^2_{\Lie}(\divzero, \z(\toroidal))$ be as in example \ref{example: yangian_cocycles_(counter)_examples}. Then:
                $$\sigma_i \in B^2_{\Lie}(\divzero, \z(\toroidal))$$
            i.e. both the toroidal $2$-cocycles $\sigma_1, \sigma_2$ are cohomologous to $0$. A particular pre-image $\tau_i \in C_1(\divzero, \z(\toroidal))$ of $\sigma_i$ under $d_1^{\z(\toroidal)}: C_1(\divzero, \z(\toroidal)) \to C_2(\divzero, \z(\toroidal))$ (with notations as in remark \ref{remark: simplified_chevalley_eilenberg_complexes} and definition \ref{def: lie_cocycles_and_coboundaries}) is given by:
                $$\tau_i(D_{r, s}) = ( r^2 + 2r ) K_{-r, -s - 2} + \delta_{(r, s), (0, -1)} ( r c_v + s c_t )$$
                $$\tau_i(D_v) = \mu_v c_v + \mu_t c_t$$
                $$\tau_i(D_t) = \nu_v c_v + \nu_t c_t$$
            wherein the coefficients $\mu_v, \mu_t, \nu_v, \nu_t \in \bbC$ can be chosen arbitrarily.
        \end{theorem}
            \begin{proof}
                Even though it suffices to prove the existence of an element:
                    $$\tau_i \in C_1(\divzero, \z(\toroidal))$$
                (i.e. a linear map; cf. remark \ref{remark: simplified_chevalley_eilenberg_complexes}) such that:
                    $$\tau_i([D, D']) = [D, \tau_i(D')]_{\extendedtoroidal} - [D', \tau_i(D)]_{\extendedtoroidal} - \sigma_i(D, D')$$
                we shall prove a stronger statement, namely the existence of $\tau_i \in C_1(\divzero, \z(\toroidal))$ that is \textit{graded}. Should this be true, then through the fact that:
                    $$\deg D_{-a, -b - 1} = \deg K_{a, b} = (a, b)$$
                    $$\deg c_v = \deg c_t = (0, 0)$$
                (cf. remark \ref{remark: Z^2_grading_on_toroidal_centres} and corollary \ref{coro: yangian_div_zero_vector_fields_are_graded}), we would have coefficients $\lambda_{r, s}, \alpha'_{r, s}, \alpha_{r, s}, \beta_{r, s} \in \bbC$ such that:
                    $$
                        \begin{aligned}
                            \tau_i(D_{r, s}) & = \lambda_{r, s} K_{-r, -s - 1} + \delta_{(r, s), (0, -1)} ( \alpha'_{r, s} K_{r, 0} + \alpha_{r, s} c_v + \beta_{r, s} c_t )
                            \\
                            & = \lambda_{r, s} K_{-r, -s - 1} + \delta_{(r, s), (0, -1)} ( \alpha_{r, s} c_v + \beta_{r, s} c_t )
                        \end{aligned}
                    $$
                where the second equality holds because $K_{0, 0} = 0$ (cf. example \ref{example: toroidal_lie_algebras_centres}), as well as $\mu_v, \mu_t, \nu_v, \nu_t \in \bbC$ such that:
                    $$\tau_i(D_v) = \mu_v c_v + \mu_t c_t$$
                    $$\tau_i(D_t) = \nu_v c_v + \nu_t c_t$$
                (again, there are no $K_{r, 0}$-component because $K_{0, 0} = 0$ and because the LHS, i.e. $\tau_i(D_v)$ and $ \tau_i(D_t)$, do not depend on $r$).
                
                \begin{enumerate}
                    \item We attempt first of all to compute $\tau_i(D_{r, s})$. For what follows, let us recall from lemma \ref{lemma: yangian_div_zero_vector_fields_basic_properties} that:
                        $$[D_v, D_{r, s}] = -r D_{r, s + 1}$$
                        $$[D_t, D_{r, s}] = -s D_{r, s + 1}$$
                        $$[D_{a, b}, D_{r, s}] = (br - sa) D_{a + r, b + s + 1}$$
                    and from lemma \ref{lemma: explicit_commutators_between_central_basis_elements_and_derivations}, that:
                        $$[D, K_{a, b}]_{\extendedtoroidal} =
                            \begin{cases}
                                \text{$((b - 1)r - sa) K_{a - r, b - s - 1} + \delta_{(r, s + 1), (a, b)} \left( r c_v + s c_t \right)$ if $D = D_{r, s}$}
                                \\
                                \text{$-a K_{a, b - 1}$ if $D_v$}
                                \\
                                \text{$-b K_{a, b - 1}$ if $D_t$}
                            \end{cases}
                        $$
                        $$[D, c_v]_{\extendedtoroidal} = [D, c_t]_{\extendedtoroidal} = 0$$
                    for all $D \in \divzero$.
                    
                    Consider, then, the following\footnote{Using $D_t$ in place of $D_v$ would yield the same conclusion.}:
                        $$
                            \begin{aligned}
                                & r \tau_i(D_{r, s + 1})
                                \\
                                = & \tau_i([D_v, D_{r, s}])
                                \\
                                = & [D_v, \tau_i(D_{r, s})]_{\extendedtoroidal} - [D_{r, s}, \tau_i(D_v)]_{\extendedtoroidal} - \sigma_i(D_v, D_{r, s})
                                \\
                                = & [D_v, \lambda_{r, s} K_{-r, -s - 1}]_{\extendedtoroidal} - [D_{r, s}, \mu_v c_v + \mu_t c_t]_{\extendedtoroidal} - r^3 K_{-r, -s - 2}
                                \\
                                = & r \lambda_{r, s} K_{-r, -s - 2} - r^3 K_{-r, -s - 2}
                            \end{aligned}
                        $$
                    From this, we infer that:
                        $$\tau_i(D_{r, s + 1}) = (\lambda_{r, s} - r^2) K_{-r, -s - 2}$$
                    but at the same time, we have the following per our initial assumption that $\tau_i$ is graded:
                        $$\tau_i(D_{r, s + 1}) = \lambda_{r, s + 1} K_{-r, -s - 2} + \delta_{(r, s + 1), (0, 0)} ( \alpha_{r, s + 1} c_v + \beta_{r, s + 1} c_t )$$
                    and so we have:
                        $$\lambda_{r, s + 1} = \lambda_{r, s} - r^2$$
                    Next, consider the following:
                        $$
                            \begin{aligned}
                                & (br - sa) \tau_i(D_{r + a, b + s + 1})
                                \\
                                = & \tau_i( [D_{a, b}, D_{r, s}] )
                                \\
                                = & [D_{a, b}, \tau_i(D_{r, s})]_{\extendedtoroidal} - [D_{r, s}, \tau_i(D_{a, b})]_{\extendedtoroidal} - \sigma_i(D_{a, b}, D_{r, s})
                                \\
                                = & \lambda_{r, s} [D_{a, b}, K_{-r, -s - 1}]_{\extendedtoroidal} - \lambda_{a, b} [D_{r, s}, K_{-a, -b - 1}]_{\extendedtoroidal} + \sigma_i(D_{a, b}, D_{r, s})
                            \end{aligned}
                        $$
                    Using lemma \ref{lemma: explicit_commutators_between_central_basis_elements_and_derivations}, we get that:
                        $$[D_{a, b}, K_{-r, -s - 1}]_{\extendedtoroidal} = -\left( a(s + 2) - br \right) K_{-a - r, -b - s - 2} + \delta_{(a + r, b + s + 2), (0, 0)} \left( a c_v + b c_t \right)$$
                        $$[D_{r, s}, K_{-a, -b - 1}]_{\extendedtoroidal} = -\left( (b + 2) r - as \right) K_{-a - r, -b - s - 2} + \delta_{(a + r, b + s + 2), (0, 0)} \left( r c_v + s c_t \right)$$
                    and from example \ref{example: yangian_cocycles_(counter)_examples}, we know that:
                        $$\sigma_i(D_{a, b}, D_{r, s}) = -N_i(r, s, a, b) \left( -\delta_{(r, s), -(a, b)} (r c_v + (s + 1) c_t) + ( r(b + 1) - a(s + 1) )K_{-r - a, -s - b - 2} \right)$$
                    where $N_i(r, s, a, b)$ is as in \textit{loc. cit.} Simultaneously, these imply that:
                        $$
                            \begin{aligned}
                                & (br - sa) \tau_i(D_{r + a, b + s + 1})
                                \\
                                = &
                                \begin{aligned}
                                    & \left( -\left( a(s + 2) - br \right) + \left( (b + 2) r - as \right) - N_i(r, s, a, b)( r(b + 1) - a(s + 1) ) \right) K_{-a - r, -b - s - 2}
                                    \\
                                    + & \left( \delta_{(a + r, b + s + 2), (0, 0)} (a + r) + \delta_{(a + r, b + s), (0, 0)} r N_i(r, s, a, b) \right) c_v
                                    \\
                                    + & \left( \delta_{(a + r, b + s + 2), (0, 0)} (b + s) + \delta_{(a + r, b + s), (0, 0)} (s + 1)N_i(r, s, a, b) \right) c_t
                                \end{aligned}
                                \\
                                = & (br - sa) \left( \lambda_{a + r, b + s + 1} K_{-a - r, -b - s - 2} + \delta_{(a + r, b + s + 1), (0, -1)}( \alpha_{a + r, b + s + 1} c_v + \beta_{a + r, b + s + 1} c_t ) \right)
                            \end{aligned}
                        $$
                    from which it can be inferred that:
                        $$
                            \begin{aligned}
                                \lambda_{a + r, b + s + 1} & = -\left( a(s + 2) - br \right) + \left( (b + 2) r - as \right) - N_i(r, s, a, b)( r(b + 1) - a(s + 1) )
                                \\
                                & = 2( r(b + 1) - a(s + 1) ) - N_i(r, s, a, b)( r(b + 1) - a(s + 1) )
                                \\
                                & = ( 2 - N_i(r, s, a, b) ) ( r(b + 1) - a(s + 1) )
                            \end{aligned}
                        $$
                        $$\delta_{(a + r, b + s + 1), (0, -1)} \alpha_{a + r, b + s + 1} = \delta_{(a + r, b + s + 2), (0, 0)} (a + r) + \delta_{(a + r, b + s), (0, 0)} r N_i(r, s, a, b)$$
                        $$\delta_{(a + r, b + s + 1), (0, -1)} \beta_{a + r, b + s + 1} = \delta_{(a + r, b + s + 2), (0, 0)} (b + s) + \delta_{(a + r, b + s), (0, 0)} (s + 1) N_i(r, s, a, b)$$
                    By combining the first equation with the fact that $\lambda_{r, s + 1} = \lambda_{r, s} - r^2$ (as shown above), we shall get that:
                        $$\lambda_{r, s} - r^2 = \lambda_{r, s + 1} = (2 - N_i(r, s, 0, 0)) r = 2r$$
                    (note that $N_i(r, s, 0, 0) = 0$), which in turn implies that:
                        $$\lambda_{r, s} = r^2 + 2r$$
                    When $(a, b) = (0, -1)$, the second and third equations shall become:
                        $$\delta_{(r, s), (0, -1)} \alpha_{r, s} = \delta_{(r, s), (0, -1)} r$$
                        $$\delta_{(r, s), (0, -1)} \beta_{r, s} = \delta_{(r, s), (0, -1)} (-1 + s)$$
                    (note that $N_i(r, s, 0, -1) = 0$). The first equation is equivalent to the tautology $0 = 0$, so the coefficient $\alpha_{r, s} \in \bbC$ remains undetermined, but when $(r, s) = (0, -1)$, the second equation becomes:
                        $$\beta_{0, -1} = -2$$
                    By putting everything together, we shall obtain:
                        $$\tau_i(D_{r, s}) = ( r^2 + 2r ) K_{-r, -s - 1} + \delta_{(r, s), (0, -1)} ( \alpha_{r, s} c_v - 2 c_t )$$
                    We remark that the expression above for $\tau_i(D_{r, s})$ - more specifically, the coefficient $\alpha_{r, s}$ - indeed depends on $i$, and so what we have obtained does not violate the fact that $\sigma_1 \not = \sigma_2$ (cf. example \ref{example: yangian_cocycles_(counter)_examples}).
                    \item Next, consider $\tau_i(D_v)$. Because there do not exist elements $D, D' \in \divzero$ so that $D_v = [D, D']$ and because $[D_v, D_t] = 0$ (cf. lemma \ref{lemma: yangian_div_zero_vector_fields_basic_properties}), and because $\tau_i(D_v) \in \bbC c_v \oplus \bbC c_t$ by hypothesis, meaning that:
                        $$[D, \tau_i(D_v)]_{\extendedtoroidal} = 0$$
                    for all $D \in \divzero$, since $[D, c_v]_{\extendedtoroidal} = [D, c_t]_{\extendedtoroidal} = 0$ per lemma \ref{lemma: explicit_commutators_between_central_basis_elements_and_derivations}, it is impossible to determine $\tau_i(D_v)$ through the use of the $2$-coboundary equation:
                        $$\tau_i([D, D']) = [D, \tau_i(D')]_{\extendedtoroidal} - [D', \tau_i(D)]_{\extendedtoroidal} - \sigma_i(D, D')$$
                    (cf. example \ref{example: low_degree_lie_coboundaries_with_non-trivial_coefficients}). As such, we can let the coefficients $\mu_v, \mu_t \in \bbC$ in the expression:
                        $$\tau_i(D_v) = \mu_v c_v + \mu_t c_t$$
                    be arbitrary.
                    \item Likewise, we can let the coefficients $\nu_v, \nu_t \in \bbC$ in the expression:
                        $$\tau_i(D_t) = \nu_v c_v + \nu_t c_t$$
                    be arbitrary.
                \end{enumerate}
            \end{proof}
        \begin{remark}
            The linear map $\tau_i: \divzero \to \z(\toroidal)$ happens to be graded, but pre-images of $\sigma_i$ under $d_1^{\z(\toroidal)}$ are not guaranteed to be graded. Perhaps it might be possible to explicitly determine $\tau_i(D_v)$ and $\tau_i(D_t)$ when $\tau_i$ is not graded.
        \end{remark}
        \begin{remark}
            Through proposition \ref{prop: yangianness_of_billig_toroidal_cocycles}, we have seen that the toroidal $2$-cocycle:
                $$\sigma_2$$
            is \textit{not} $\gamma$-invariant in the sense of definition \ref{def: yangian_toroidal_cocycles}. However, in light of theorem \ref{theorem: non_trivial_yangian_cocycles_examples}, we now see that this non-Yangian-ness of $\sigma_2$ holds \textit{despite} the fact that it is cohomologous to $0$. Therefore, one can not conclude from observing that a toroidal $2$-cocycle $\sigma$ is cohomologous to $0$ that $\sigma$ is Yangian, but when $\sigma = 0$ as elements of $Z^2_{\Lie}(\divzero, \z(\toroidal))$, then it will indeed be true that $\sigma$ is Yangian, as we know that the semi-direct product $\toroidal \rtimes \divzero$ is an instance of a $\gamma$-extended toroidal Lie algebra.
        \end{remark}