\documentclass[a4paper, 11pt]{article}

%\usepackage[center]{titlesec}

\usepackage{amsfonts, amssymb, amsmath, amsthm, amsxtra}

\usepackage{foekfont}

\usepackage{MnSymbol}

\usepackage{pdfrender, xcolor}
%\pdfrender{StrokeColor=black,LineWidth=.4pt,TextRenderingMode=2}

%\usepackage{minitoc}
%\setcounter{section}{-1}
%\setcounter{tocdepth}{}
%\setcounter{minitocdepth}{}
%\setcounter{secnumdepth}{}

\usepackage{graphicx}

\usepackage[english]{babel}
\usepackage[utf8]{inputenc}
%\usepackage{mathpazo}
%\usepackage{eucal}
\usepackage{eufrak}
\usepackage{bbm}
\usepackage{bm}
\usepackage{csquotes}
\usepackage[nottoc]{tocbibind}
\usepackage{appendix}
\usepackage{float}
\usepackage[T1]{fontenc}
\usepackage[
    left = \flqq{},% 
    right = \frqq{},% 
    leftsub = \flq{},% 
    rightsub = \frq{} %
]{dirtytalk}

\usepackage{imakeidx}
\makeindex

%\usepackage[dvipsnames]{xcolor}
\usepackage{hyperref}
    \hypersetup{
        colorlinks=true,
        linkcolor=teal,
        filecolor=pink,      
        urlcolor=teal,
        citecolor=magenta
    }
\usepackage{comment}

% You would set the PDF title, author, etc. with package options or
% \hypersetup.

\usepackage[backend=biber, style=alphabetic, sorting=nty]{biblatex}
    \addbibresource{bibliography.bib}
\renewbibmacro{in:}{}

\raggedbottom

\usepackage{mathrsfs}
\usepackage{mathtools} 
\mathtoolsset{showonlyrefs} 
%\usepackage{amsthm}
\renewcommand\qedsymbol{$\blacksquare$}

\usepackage{tikz-cd}
\tikzcdset{scale cd/.style={every label/.append style={scale=#1},
    cells={nodes={scale=#1}}}}
\usepackage{tikz}
\usepackage{setspace}
\usepackage[version=3]{mhchem}
\parskip=0.1in
\usepackage[margin=25mm]{geometry}

\usepackage{listings, lstautogobble}
\lstset{
	language=matlab,
	basicstyle=\scriptsize\ttfamily,
	commentstyle=\ttfamily\itshape\color{gray},
	stringstyle=\ttfamily,
	showstringspaces=false,
	breaklines=true,
	frameround=ffff,
	frame=single,
	rulecolor=\color{black},
	autogobble=true
}

\usepackage{todonotes,tocloft,xpatch,hyperref}

% This is based on classicthesis chapter definition
\let\oldsec=\section
\renewcommand*{\section}{\secdef{\Sec}{\SecS}}
\newcommand\SecS[1]{\oldsec*{#1}}%
\newcommand\Sec[2][]{\oldsec[\texorpdfstring{#1}{#1}]{#2}}%

\newcounter{istodo}[section]

% http://tex.stackexchange.com/a/61267/11984
\makeatletter
%\xapptocmd{\Sec}{\addtocontents{tdo}{\protect\todoline{\thesection}{#1}{}}}{}{}
\newcommand{\todoline}[1]{\@ifnextchar\Endoftdo{}{\@todoline{#1}}}
\newcommand{\@todoline}[3]{%
	\@ifnextchar\todoline{}
	{\contentsline{section}{\numberline{#1}#2}{#3}{}{}}%
}
\let\l@todo\l@subsection
\newcommand{\Endoftdo}{}

\AtEndDocument{\addtocontents{tdo}{\string\Endoftdo}}
\makeatother

\usepackage{lipsum}

%   Reduce the margin of the summary:
\def\changemargin#1#2{\list{}{\rightmargin#2\leftmargin#1}\item[]}
\let\endchangemargin=\endlist 

%   Generate the environment for the abstract:
%\newcommand\summaryname{Abstract}
%\newenvironment{abstract}%
    %{\small\begin{center}%
    %\bfseries{\summaryname} \end{center}}

\newtheorem{theorem}{Theorem}[section]
    \numberwithin{theorem}{subsection}
\newtheorem{proposition}{Proposition}[section]
    \numberwithin{proposition}{subsection}
\newtheorem{lemma}{Lemma}[section]
    \numberwithin{lemma}{subsection}
\newtheorem{claim}{Claim}[section]
    \numberwithin{claim}{subsection}
\newtheorem{question}{Question}[section]
    \numberwithin{question}{subsection}

\theoremstyle{definition}
    \newtheorem{definition}{Definition}[section]
        \numberwithin{definition}{subsection}

\theoremstyle{remark}
    \newtheorem{remark}{Remark}[section]
        \numberwithin{remark}{subsection}
    \newtheorem{example}{Example}[section]
        \numberwithin{example}{subsection}    
    \newtheorem{convention}{Convention}[section]
        \numberwithin{convention}{subsection}
    \newtheorem{corollary}{Corollary}[section]
        \numberwithin{corollary}{subsection}

\numberwithin{equation}{section}

\setcounter{section}{-1}

\renewcommand{\cong}{\simeq}
\newcommand{\ladjoint}{\dashv}
\newcommand{\radjoint}{\vdash}
\newcommand{\<}{\langle}
\renewcommand{\>}{\rangle}
\newcommand{\ndiv}{\hspace{-2pt}\not|\hspace{5pt}}
\newcommand{\cond}{\blacktriangle}
\newcommand{\decond}{\triangle}
\newcommand{\solid}{\blacksquare}
\newcommand{\ot}{\leftarrow}
\renewcommand{\-}{\text{-}}
\renewcommand{\mapsto}{\leadsto}
\renewcommand{\leq}{\leqslant}
\renewcommand{\geq}{\geqslant}
\renewcommand{\setminus}{\smallsetminus}
\makeatletter
\DeclareRobustCommand{\cev}[1]{%
  {\mathpalette\do@cev{#1}}%
}
\newcommand{\do@cev}[2]{%
  \vbox{\offinterlineskip
    \sbox\z@{$\m@th#1 x$}%
    \ialign{##\cr
      \hidewidth\reflectbox{$\m@th#1\vec{}\mkern4mu$}\hidewidth\cr
      \noalign{\kern-\ht\z@}
      $\m@th#1#2$\cr
    }%
  }%
}
\makeatother

\newcommand{\N}{\mathbb{N}}
\newcommand{\Z}{\mathbb{Z}}
\newcommand{\Q}{\mathbb{Q}}
\newcommand{\R}{\mathbb{R}}
\newcommand{\bbC}{\mathbb{C}}
\NewDocumentCommand{\x}{e{_^}}{%
  \mathbin{\mathop{\times}\displaylimits
    \IfValueT{#1}{_{#1}}
    \IfValueT{#2}{^{#2}}
  }%
}
\NewDocumentCommand{\pushout}{e{_^}}{%
  \mathbin{\mathop{\sqcup}\displaylimits
    \IfValueT{#1}{_{#1}}
    \IfValueT{#2}{^{#2}}
  }%
}
\newcommand{\supp}{\operatorname{supp}}
\newcommand{\im}{\operatorname{im}}
\newcommand{\coker}{\operatorname{coker}}
\newcommand{\id}{\mathrm{id}}
\newcommand{\chara}{\operatorname{char}}
\newcommand{\trdeg}{\operatorname{trdeg}}
\newcommand{\rank}{\operatorname{rank}}
\newcommand{\trace}{\operatorname{tr}}
\newcommand{\length}{\operatorname{length}}
\newcommand{\height}{\operatorname{ht}}
\renewcommand{\span}{\operatorname{span}}
\newcommand{\e}{\epsilon}
\newcommand{\p}{\mathfrak{p}}
\newcommand{\q}{\mathfrak{q}}
\newcommand{\m}{\mathfrak{m}}
\newcommand{\n}{\mathfrak{n}}
\newcommand{\calF}{\mathcal{F}}
\newcommand{\calG}{\mathcal{G}}
\newcommand{\calO}{\mathcal{O}}
\newcommand{\F}{\mathbb{F}}
\DeclareMathOperator{\lcm}{lcm}
\newcommand{\gr}{\operatorname{gr}}
\newcommand{\vol}{\mathrm{vol}}
\newcommand{\ord}{\operatorname{ord}}
\newcommand{\projdim}{\operatorname{proj.dim}}
\newcommand{\injdim}{\operatorname{inj.dim}}
\newcommand{\flatdim}{\operatorname{flat.dim}}
\newcommand{\globdim}{\operatorname{glob.dim}}
\renewcommand{\Re}{\operatorname{Re}}
\renewcommand{\Im}{\operatorname{Im}}
\newcommand{\sgn}{\operatorname{sgn}}
\newcommand{\coad}{\operatorname{coad}}

\newcommand{\Ad}{\mathrm{Ad}}
\newcommand{\GL}{\mathrm{GL}}
\newcommand{\SL}{\mathrm{SL}}
\newcommand{\PGL}{\mathrm{PGL}}
\newcommand{\PSL}{\mathrm{PSL}}
\newcommand{\Sp}{\mathrm{Sp}}
\newcommand{\GSp}{\mathrm{GSp}}
\newcommand{\GSpin}{\mathrm{GSpin}}
\newcommand{\rmO}{\mathrm{O}}
\newcommand{\SO}{\mathrm{SO}}
\newcommand{\SU}{\mathrm{SU}}
\newcommand{\rmU}{\mathrm{U}}
\newcommand{\rmu}{\mathrm{u}}
\newcommand{\rmV}{\mathrm{V}}
\newcommand{\gl}{\mathfrak{gl}}
\renewcommand{\sl}{\mathfrak{sl}}
\newcommand{\diag}{\mathfrak{diag}}
\newcommand{\pgl}{\mathfrak{pgl}}
\newcommand{\psl}{\mathfrak{psl}}
\newcommand{\fraksp}{\mathfrak{sp}}
\newcommand{\gsp}{\mathfrak{gsp}}
\newcommand{\gspin}{\mathfrak{gspin}}
\newcommand{\frako}{\mathfrak{o}}
\newcommand{\so}{\mathfrak{so}}
\newcommand{\su}{\mathfrak{su}}
%\newcommand{\fraku}{\mathfrak{u}}
\newcommand{\Spec}{\operatorname{Spec}}
\newcommand{\Spf}{\operatorname{Spf}}
\newcommand{\Spm}{\operatorname{Spm}}
\newcommand{\Spv}{\operatorname{Spv}}
\newcommand{\Spa}{\operatorname{Spa}}
\newcommand{\Spd}{\operatorname{Spd}}
\newcommand{\Proj}{\operatorname{Proj}}
\newcommand{\Gr}{\mathrm{Gr}}
\newcommand{\Hecke}{\mathrm{Hecke}}
\newcommand{\Sht}{\mathrm{Sht}}
\newcommand{\Quot}{\mathrm{Quot}}
\newcommand{\Hilb}{\mathrm{Hilb}}
\newcommand{\Pic}{\mathrm{Pic}}
\newcommand{\Div}{\mathrm{Div}}
\newcommand{\Jac}{\mathrm{Jac}}
\newcommand{\Alb}{\mathrm{Alb}} %albanese variety
\newcommand{\Bun}{\mathrm{Bun}}
\newcommand{\loopspace}{\mathbf{\Omega}}
\newcommand{\suspension}{\mathbf{\Sigma}}
\newcommand{\tangent}{\mathrm{T}} %tangent space
\newcommand{\Eig}{\mathrm{Eig}}
\newcommand{\Cox}{\mathrm{Cox}} %coxeter functors
\newcommand{\rmK}{\mathrm{K}} %Killing form
\newcommand{\km}{\mathfrak{km}} %kac-moody algebras
\newcommand{\Dyn}{\mathrm{Dyn}} %associated Dynkin quivers
\newcommand{\Car}{\mathrm{Car}} %cartan matrices of finite quivers

\newcommand{\Ring}{\mathrm{Ring}}
\newcommand{\Cring}{\mathrm{CRing}}
\newcommand{\Alg}{\mathrm{Alg}}
\newcommand{\Leib}{\mathrm{Leib}} %leibniz algebras
\newcommand{\Fld}{\mathrm{Fld}}
\newcommand{\Sets}{\mathrm{Sets}}
\newcommand{\Equiv}{\mathrm{Equiv}} %equivalence relations
\newcommand{\Cat}{\mathrm{Cat}}
\newcommand{\Grp}{\mathrm{Grp}}
\newcommand{\Ab}{\mathrm{Ab}}
\newcommand{\Sch}{\mathrm{Sch}}
\newcommand{\Coh}{\mathrm{Coh}}
\newcommand{\QCoh}{\mathrm{QCoh}}
\newcommand{\Perf}{\mathrm{Perf}} %perfect complexes
\newcommand{\Sing}{\mathrm{Sing}} %singularity categories
\newcommand{\Desc}{\mathrm{Desc}}
\newcommand{\Sh}{\mathrm{Sh}}
\newcommand{\Psh}{\mathrm{PSh}}
\newcommand{\Fib}{\mathrm{Fib}}
\renewcommand{\mod}{\-\mathrm{mod}}
\newcommand{\comod}{\-\mathrm{comod}}
\newcommand{\bimod}{\-\mathrm{bimod}}
\newcommand{\Vect}{\mathrm{Vect}}
\newcommand{\Rep}{\mathrm{Rep}}
\newcommand{\Grpd}{\mathrm{Grpd}}
\newcommand{\Arr}{\mathrm{Arr}}
\newcommand{\Esp}{\mathrm{Esp}}
\newcommand{\Ob}{\mathrm{Ob}}
\newcommand{\Mor}{\mathrm{Mor}}
\newcommand{\Mfd}{\mathrm{Mfd}}
\newcommand{\Riem}{\mathrm{Riem}}
\newcommand{\RS}{\mathrm{RS}}
\newcommand{\LRS}{\mathrm{LRS}}
\newcommand{\TRS}{\mathrm{TRS}}
\newcommand{\TLRS}{\mathrm{TLRS}}
\newcommand{\LVRS}{\mathrm{LVRS}}
\newcommand{\LBRS}{\mathrm{LBRS}}
\newcommand{\Spc}{\mathrm{Spc}}
\newcommand{\Top}{\mathrm{Top}}
\newcommand{\Topos}{\mathrm{Topos}}
\newcommand{\Nil}{\mathfrak{nil}}
\newcommand{\J}{\mathfrak{J}}
\newcommand{\Stk}{\mathrm{Stk}}
\newcommand{\Pre}{\mathrm{Pre}}
\newcommand{\simp}{\mathbf{\Delta}}
\newcommand{\Res}{\mathrm{Res}}
\newcommand{\Ind}{\mathrm{Ind}}
\newcommand{\Pro}{\mathrm{Pro}}
\newcommand{\Mon}{\mathrm{Mon}}
\newcommand{\Comm}{\mathrm{Comm}}
\newcommand{\Fin}{\mathrm{Fin}}
\newcommand{\Assoc}{\mathrm{Assoc}}
\newcommand{\Semi}{\mathrm{Semi}}
\newcommand{\Co}{\mathrm{Co}}
\newcommand{\Loc}{\mathrm{Loc}}
\newcommand{\Ringed}{\mathrm{Ringed}}
\newcommand{\Haus}{\mathrm{Haus}} %hausdorff spaces
\newcommand{\Comp}{\mathrm{Comp}} %compact hausdorff spaces
\newcommand{\Stone}{\mathrm{Stone}} %stone spaces
\newcommand{\Extr}{\mathrm{Extr}} %extremely disconnected spaces
\newcommand{\Ouv}{\mathrm{Ouv}}
\newcommand{\Str}{\mathrm{Str}}
\newcommand{\Func}{\mathrm{Func}}
\newcommand{\Crys}{\mathrm{Crys}}
\newcommand{\LocSys}{\mathrm{LocSys}}
\newcommand{\Sieves}{\mathrm{Sieves}}
\newcommand{\pt}{\mathrm{pt}}
\newcommand{\Graphs}{\mathrm{Graphs}}
\newcommand{\Lie}{\mathrm{Lie}}
\newcommand{\Env}{\mathrm{Env}}
\newcommand{\Ho}{\mathrm{Ho}}
\newcommand{\rmD}{\mathrm{D}}
\newcommand{\Cov}{\mathrm{Cov}}
\newcommand{\Frames}{\mathrm{Frames}}
\newcommand{\Locales}{\mathrm{Locales}}
\newcommand{\Span}{\mathrm{Span}}
\newcommand{\Corr}{\mathrm{Corr}}
\newcommand{\Monad}{\mathrm{Monad}}
\newcommand{\Var}{\mathrm{Var}}
\newcommand{\sfN}{\mathrm{N}} %nerve
\newcommand{\Diam}{\mathrm{Diam}} %diamonds
\newcommand{\co}{\mathrm{co}}
\newcommand{\ev}{\mathrm{ev}}
\newcommand{\bi}{\mathrm{bi}}
\newcommand{\Nat}{\mathrm{Nat}}
\newcommand{\Hopf}{\mathrm{Hopf}}
\newcommand{\Dmod}{\mathrm{D}\mod}
\newcommand{\Perv}{\mathrm{Perv}}
\newcommand{\Sph}{\mathrm{Sph}}
\newcommand{\Moduli}{\mathrm{Moduli}}
\newcommand{\Pseudo}{\mathrm{Pseudo}}
\newcommand{\Lax}{\mathrm{Lax}}
\newcommand{\Strict}{\mathrm{Strict}}
\newcommand{\Opd}{\mathrm{Opd}} %operads
\newcommand{\Shv}{\mathrm{Shv}}
\newcommand{\Char}{\mathrm{Char}} %CharShv = character sheaves
\newcommand{\Huber}{\mathrm{Huber}}
\newcommand{\Tate}{\mathrm{Tate}}
\newcommand{\Affd}{\mathrm{Affd}} %affinoid algebras
\newcommand{\Adic}{\mathrm{Adic}} %adic spaces
\newcommand{\Rig}{\mathrm{Rig}}
\newcommand{\An}{\mathrm{An}}
\newcommand{\Perfd}{\mathrm{Perfd}} %perfectoid spaces
\newcommand{\Sub}{\mathrm{Sub}} %subobjects
\newcommand{\Ideals}{\mathrm{Ideals}}
\newcommand{\Isoc}{\mathrm{Isoc}} %isocrystals
\newcommand{\Ban}{\-\mathrm{Ban}} %Banach spaces
\newcommand{\Fre}{\-\mathrm{Fr\acute{e}}} %Frechet spaces
\newcommand{\Ch}{\mathrm{Ch}} %chain complexes
\newcommand{\Pure}{\mathrm{Pure}}
\newcommand{\Mixed}{\mathrm{Mixed}}
\newcommand{\Hodge}{\mathrm{Hodge}} %Hodge structures
\newcommand{\Mot}{\mathrm{Mot}} %motives
\newcommand{\KL}{\mathrm{KL}} %category of Kazhdan-Lusztig modules
\newcommand{\Pres}{\mathrm{Pres}} %presentable categories
\newcommand{\Noohi}{\mathrm{Noohi}} %category of Noohi groups
\newcommand{\Inf}{\mathrm{Inf}}
\newcommand{\LPar}{\mathrm{LPar}} %Langlands parameters
\newcommand{\ORig}{\mathrm{ORig}} %overconvergent sites
\newcommand{\Quiv}{\mathrm{Quiv}} %quivers
\newcommand{\Def}{\mathrm{Def}} %deformation functors
\newcommand{\Root}{\mathrm{Root}}
\newcommand{\gRep}{\mathrm{gRep}}
\newcommand{\Higgs}{\mathrm{Higgs}}
\newcommand{\BGG}{\mathrm{BGG}}

\newcommand{\Aut}{\mathrm{Aut}}
\newcommand{\Inn}{\mathrm{Inn}}
\newcommand{\Out}{\mathrm{Out}}
\newcommand{\der}{\mathfrak{der}} %derivations on Lie algebras
\newcommand{\frakend}{\mathfrak{end}}
\newcommand{\aut}{\mathfrak{aut}}
\newcommand{\inn}{\mathfrak{inn}} %inner derivations
\newcommand{\out}{\mathfrak{out}} %outer derivations
\newcommand{\Stab}{\mathrm{Stab}}
\newcommand{\Cent}{\mathrm{Cent}}
\newcommand{\Norm}{\mathrm{Norm}}
\newcommand{\stab}{\mathfrak{stab}}
\newcommand{\cent}{\mathfrak{cent}}
\newcommand{\norm}{\mathfrak{norm}}
\newcommand{\Rad}{\operatorname{Rad}}
\newcommand{\Transporter}{\mathrm{Transp}} %transporter between two subsets of a group
\newcommand{\Conj}{\mathrm{Conj}}
\newcommand{\Diag}{\mathrm{Diag}}
\newcommand{\Gal}{\mathrm{Gal}}
\newcommand{\bfG}{\mathbf{G}} %absolute Galois group
\newcommand{\Frac}{\mathrm{Frac}}
\newcommand{\Ann}{\mathrm{Ann}}
\newcommand{\Val}{\mathrm{Val}}
\newcommand{\Chow}{\mathrm{Chow}}
\newcommand{\Sym}{\mathrm{Sym}}
\newcommand{\End}{\mathrm{End}}
\newcommand{\Mat}{\mathrm{Mat}}
\newcommand{\Diff}{\mathrm{Diff}}
\newcommand{\Autom}{\mathrm{Autom}}
\newcommand{\Artin}{\mathrm{Artin}} %artin maps
\newcommand{\sk}{\mathrm{sk}} %skeleton of a category
\newcommand{\eqv}{\mathrm{eqv}} %functor that maps groups $G$ to $G$-sets
\newcommand{\Inert}{\mathrm{Inert}}
\newcommand{\Fil}{\mathrm{Fil}}
\newcommand{\Prim}{\mathfrak{Prim}}
\newcommand{\Nerve}{\mathrm{N}}
\newcommand{\Hol}{\mathrm{Hol}} %holomorphic functions %holonomy groups
\newcommand{\Bi}{\mathrm{Bi}} %Bi for biholomorphic functions
\newcommand{\chev}{\mathfrak{chev}} %chevalley relations
\newcommand{\bfLie}{\mathbf{Lie}} %non-reduced lie algebra associated to generalised cartan matrices
\newcommand{\frakLie}{\mathfrak{Lie}} %reduced lie algebra associated to generalised cartan matrices
\newcommand{\frakChev}{\mathfrak{Chev}} 
\newcommand{\Rees}{\operatorname{Rees}}
\newcommand{\Dr}{\mathrm{Dr}} %Drinfeld's quantum double 

\renewcommand{\projlim}{\varprojlim}
\newcommand{\indlim}{\varinjlim}
\newcommand{\colim}{\operatorname{colim}}
\renewcommand{\lim}{\operatorname{lim}}
\newcommand{\toto}{\rightrightarrows}
%\newcommand{\tensor}{\otimes}
\NewDocumentCommand{\tensor}{e{_^}}{%
  \mathbin{\mathop{\otimes}\displaylimits
    \IfValueT{#1}{_{#1}}
    \IfValueT{#2}{^{#2}}
  }%
}
\NewDocumentCommand{\singtensor}{e{_^}}{%
  \mathbin{\mathop{\odot}\displaylimits
    \IfValueT{#1}{_{#1}}
    \IfValueT{#2}{^{#2}}
  }%
}
\NewDocumentCommand{\hattensor}{e{_^}}{%
  \mathbin{\mathop{\hat{\otimes}}\displaylimits
    \IfValueT{#1}{_{#1}}
    \IfValueT{#2}{^{#2}}
  }%
}
\NewDocumentCommand{\semidirect}{e{_^}}{%
  \mathbin{\mathop{\rtimes}\displaylimits
    \IfValueT{#1}{_{#1}}
    \IfValueT{#2}{^{#2}}
  }%
}
\newcommand{\eq}{\operatorname{eq}}
\newcommand{\coeq}{\operatorname{coeq}}
\newcommand{\Hom}{\mathrm{Hom}}
\newcommand{\Maps}{\mathrm{Maps}}
\newcommand{\Tor}{\mathrm{Tor}}
\newcommand{\Ext}{\mathrm{Ext}}
\newcommand{\Isom}{\mathrm{Isom}}
\newcommand{\stalk}{\mathrm{stalk}}
\newcommand{\RKE}{\operatorname{RKE}}
\newcommand{\LKE}{\operatorname{LKE}}
\newcommand{\oblv}{\mathrm{oblv}}
\newcommand{\const}{\mathrm{const}}
\newcommand{\free}{\mathrm{free}}
\newcommand{\adrep}{\mathrm{ad}} %adjoint representation
\newcommand{\NL}{\mathbb{NL}} %naive cotangent complex
\newcommand{\pr}{\operatorname{pr}}
\newcommand{\Der}{\mathrm{Der}}
\newcommand{\Frob}{\mathrm{Fr}} %Frobenius
\newcommand{\frob}{\mathrm{f}} %trace of Frobenius
\newcommand{\bfpt}{\mathbf{pt}}
\newcommand{\bfloc}{\mathbf{loc}}
\DeclareMathAlphabet{\mymathbb}{U}{BOONDOX-ds}{m}{n}
\newcommand{\0}{\mymathbb{0}}
\newcommand{\1}{\mathbbm{1}}
\newcommand{\2}{\mathbbm{2}}
\newcommand{\Jet}{\mathrm{Jet}}
\newcommand{\Split}{\mathrm{Split}}
\newcommand{\Sq}{\mathrm{Sq}}
\newcommand{\Zero}{\mathrm{Z}}
\newcommand{\SqZ}{\Sq\Zero}
\newcommand{\lie}{\mathfrak{lie}}
\newcommand{\y}{\mathrm{y}} %yoneda
\newcommand{\Sm}{\mathrm{Sm}}
\newcommand{\AJ}{\phi} %abel-jacobi map
\newcommand{\act}{\mathrm{act}}
\newcommand{\ram}{\mathrm{ram}} %ramification index
\newcommand{\inv}{\mathrm{inv}}
\newcommand{\Spr}{\mathrm{Spr}} %the Springer map/sheaf
\newcommand{\Refl}{\mathrm{Refl}} %reflection functor]
\newcommand{\HH}{\mathrm{HH}} %Hochschild (co)homology
\newcommand{\Poinc}{\mathrm{Poinc}}
\newcommand{\Simpson}{\mathrm{Simpson}}

\newcommand{\bbU}{\mathbb{U}}
\newcommand{\V}{\mathbb{V}}
\newcommand{\calU}{\mathcal{U}}
\newcommand{\calW}{\mathcal{W}}
\newcommand{\rmI}{\mathrm{I}} %augmentation ideal
\newcommand{\bfV}{\mathbf{V}}
\newcommand{\C}{\mathcal{C}}
\newcommand{\D}{\mathcal{D}}
\newcommand{\T}{\mathscr{T}} %Tate modules
\newcommand{\calM}{\mathcal{M}}
\newcommand{\calN}{\mathcal{N}}
\newcommand{\calP}{\mathcal{P}}
\newcommand{\calQ}{\mathcal{Q}}
\newcommand{\A}{\mathbb{A}}
\renewcommand{\P}{\mathbb{P}}
\newcommand{\calL}{\mathcal{L}}
\newcommand{\E}{\mathcal{E}}
\renewcommand{\H}{\mathbf{H}}
\newcommand{\scrS}{\mathscr{S}}
\newcommand{\calX}{\mathcal{X}}
\newcommand{\calY}{\mathcal{Y}}
\newcommand{\calZ}{\mathcal{Z}}
\newcommand{\calS}{\mathcal{S}}
\newcommand{\calR}{\mathcal{R}}
\newcommand{\scrX}{\mathscr{X}}
\newcommand{\scrY}{\mathscr{Y}}
\newcommand{\scrZ}{\mathscr{Z}}
\newcommand{\calA}{\mathcal{A}}
\newcommand{\calB}{\mathcal{B}}
\renewcommand{\S}{\mathcal{S}}
\newcommand{\B}{\mathbb{B}}
\newcommand{\bbD}{\mathbb{D}}
\newcommand{\G}{\mathbb{G}}
\newcommand{\horn}{\mathbf{\Lambda}}
\renewcommand{\L}{\mathbb{L}}
\renewcommand{\a}{\mathfrak{a}}
\renewcommand{\b}{\mathfrak{b}}
\renewcommand{\c}{\mathfrak{c}}
\renewcommand{\t}{\mathfrak{t}}
\renewcommand{\r}{\mathfrak{r}}
\newcommand{\fraku}{\mathfrak{u}}
\newcommand{\bbX}{\mathbb{X}}
\newcommand{\frakw}{\mathfrak{w}}
\newcommand{\frakG}{\mathfrak{G}}
\newcommand{\frakH}{\mathfrak{H}}
\newcommand{\frakE}{\mathfrak{E}}
\newcommand{\frakF}{\mathfrak{F}}
\newcommand{\g}{\mathfrak{g}}
\newcommand{\h}{\mathfrak{h}}
\renewcommand{\k}{\mathfrak{k}}
\newcommand{\z}{\mathfrak{z}}
\newcommand{\fraki}{\mathfrak{i}}
\newcommand{\frakj}{\mathfrak{j}}
\newcommand{\del}{\partial}
\newcommand{\bbE}{\mathbb{E}}
\newcommand{\scrO}{\mathscr{O}}
\newcommand{\bbO}{\mathbb{O}}
\newcommand{\scrA}{\mathscr{A}}
\newcommand{\scrB}{\mathscr{B}}
\newcommand{\scrF}{\mathscr{F}}
\newcommand{\scrG}{\mathscr{G}}
\newcommand{\scrM}{\mathscr{M}}
\newcommand{\scrN}{\mathscr{N}}
\newcommand{\scrP}{\mathscr{P}}
\newcommand{\frakS}{\mathfrak{S}}
\newcommand{\frakT}{\mathfrak{T}}
\newcommand{\calI}{\mathcal{I}}
\newcommand{\calJ}{\mathcal{J}}
\newcommand{\scrI}{\mathscr{I}}
\newcommand{\scrJ}{\mathscr{J}}
\newcommand{\scrK}{\mathscr{K}}
\newcommand{\calK}{\mathcal{K}}
\newcommand{\scrV}{\mathscr{V}}
\newcommand{\scrW}{\mathscr{W}}
\newcommand{\bbS}{\mathbb{S}}
\newcommand{\scrH}{\mathscr{H}}
\newcommand{\bfA}{\mathbf{A}}
\newcommand{\bfB}{\mathbf{B}}
\newcommand{\bfC}{\mathbf{C}}
\renewcommand{\O}{\mathbb{O}}
\newcommand{\calV}{\mathcal{V}}
\newcommand{\scrR}{\mathscr{R}} %radical
\newcommand{\rmZ}{\mathrm{Z}} %centre of algebra
\newcommand{\rmC}{\mathrm{C}} %centralisers in algebras
\newcommand{\bfGamma}{\mathbf{\Gamma}}
\newcommand{\scrU}{\mathscr{U}}
\newcommand{\rmW}{\mathrm{W}} %Weil group
\newcommand{\frakM}{\mathfrak{M}}
\newcommand{\frakN}{\mathfrak{N}}
\newcommand{\frakB}{\mathfrak{B}}
\newcommand{\frakX}{\mathfrak{X}}
\newcommand{\frakY}{\mathfrak{Y}}
\newcommand{\frakZ}{\mathfrak{Z}}
\newcommand{\frakU}{\mathfrak{U}}
\newcommand{\frakR}{\mathfrak{R}}
\newcommand{\frakP}{\mathfrak{P}}
\newcommand{\frakQ}{\mathfrak{Q}}
\newcommand{\sfX}{\mathsf{X}}
\newcommand{\sfY}{\mathsf{Y}}
\newcommand{\sfZ}{\mathsf{Z}}
\newcommand{\sfS}{\mathsf{S}}
\newcommand{\sfT}{\mathsf{T}}
\newcommand{\sfOmega}{\mathsf{\Omega}} %drinfeld p-adic upper-half plane
\newcommand{\rmA}{\mathrm{A}}
\newcommand{\rmB}{\mathrm{B}}
\newcommand{\calT}{\mathcal{T}}
\newcommand{\sfA}{\mathsf{A}}
\newcommand{\sfD}{\mathsf{D}}
\newcommand{\sfE}{\mathsf{E}}
\newcommand{\frakL}{\mathfrak{L}}
\newcommand{\K}{\mathrm{K}}
\newcommand{\rmT}{\mathrm{T}}
\newcommand{\bfv}{\mathbf{v}}
\newcommand{\bfg}{\mathbf{g}}
\newcommand{\frakV}{\mathfrak{V}}
\newcommand{\frakv}{\mathfrak{v}}
\newcommand{\bfn}{\mathbf{n}}
\renewcommand{\o}{\mathfrak{o}}

\newcommand{\aff}{\mathrm{aff}}
\newcommand{\ft}{\mathrm{ft}} %finite type
\newcommand{\fp}{\mathrm{fp}} %finite presentation
\newcommand{\fr}{\mathrm{fr}} %free
\newcommand{\tft}{\mathrm{tft}} %topologically finite type
\newcommand{\tfp}{\mathrm{tfp}} %topologically finite presentation
\newcommand{\tfr}{\mathrm{tfr}} %topologically free
\newcommand{\aft}{\mathrm{aft}}
\newcommand{\lft}{\mathrm{lft}}
\newcommand{\laft}{\mathrm{laft}}
\newcommand{\cpt}{\mathrm{cpt}}
\newcommand{\cproj}{\mathrm{cproj}}
\newcommand{\qc}{\mathrm{qc}}
\newcommand{\qs}{\mathrm{qs}}
\newcommand{\lcmpt}{\mathrm{lcmpt}}
\newcommand{\red}{\mathrm{red}}
\newcommand{\fin}{\mathrm{fin}}
\newcommand{\fd}{\mathrm{fd}} %finite-dimensional
\newcommand{\gen}{\mathrm{gen}}
\newcommand{\petit}{\mathrm{petit}}
\newcommand{\gros}{\mathrm{gros}}
\newcommand{\loc}{\mathrm{loc}}
\newcommand{\glob}{\mathrm{glob}}
%\newcommand{\ringed}{\mathrm{ringed}}
%\newcommand{\qcoh}{\mathrm{qcoh}}
\newcommand{\cl}{\mathrm{cl}}
\newcommand{\et}{\mathrm{\acute{e}t}}
\newcommand{\fet}{\mathrm{f\acute{e}t}}
\newcommand{\profet}{\mathrm{prof\acute{e}t}}
\newcommand{\proet}{\mathrm{pro\acute{e}t}}
\newcommand{\Zar}{\mathrm{Zar}}
\newcommand{\fppf}{\mathrm{fppf}}
\newcommand{\fpqc}{\mathrm{fpqc}}
\newcommand{\orig}{\mathrm{orig}} %overconvergent topology
\newcommand{\smooth}{\mathrm{sm}}
\newcommand{\sh}{\mathrm{sh}}
\newcommand{\op}{\mathrm{op}}
\newcommand{\cop}{\mathrm{cop}}
\newcommand{\open}{\mathrm{open}}
\newcommand{\closed}{\mathrm{closed}}
\newcommand{\geom}{\mathrm{geom}}
\newcommand{\alg}{\mathrm{alg}}
\newcommand{\sober}{\mathrm{sober}}
\newcommand{\dR}{\mathrm{dR}}
\newcommand{\rad}{\mathfrak{rad}}
\newcommand{\discrete}{\mathrm{discrete}}
%\newcommand{\add}{\mathrm{add}}
%\newcommand{\lin}{\mathrm{lin}}
\newcommand{\Krull}{\mathrm{Krull}}
\newcommand{\qis}{\mathrm{qis}} %quasi-isomorphism
\newcommand{\ho}{\mathrm{ho}} %homotopy equivalence
\newcommand{\sep}{\mathrm{sep}}
\newcommand{\unr}{\mathrm{unr}}
\newcommand{\tame}{\mathrm{tame}}
\newcommand{\wild}{\mathrm{wild}}
\newcommand{\nil}{\mathrm{nil}}
\newcommand{\defm}{\mathrm{defm}}
\newcommand{\Art}{\mathrm{Art}}
\newcommand{\Noeth}{\mathrm{Noeth}}
\newcommand{\affd}{\mathrm{affd}}
%\newcommand{\adic}{\mathrm{adic}}
\newcommand{\pre}{\mathrm{pre}}
\newcommand{\coperf}{\mathrm{coperf}}
\newcommand{\perf}{\mathrm{perf}}
\newcommand{\perfd}{\mathrm{perfd}}
\newcommand{\rat}{\mathrm{rat}}
\newcommand{\cont}{\mathrm{cont}}
\newcommand{\dg}{\mathrm{dg}}
\newcommand{\almost}{\mathrm{a}}
%\newcommand{\stab}{\mathrm{stab}}
\newcommand{\heart}{\heartsuit}
\newcommand{\proj}{\mathrm{proj}}
\newcommand{\qproj}{\mathrm{qproj}}
\newcommand{\pd}{\mathrm{pd}}
\newcommand{\crys}{\mathrm{crys}}
\newcommand{\prisma}{\mathrm{prisma}}
\newcommand{\FF}{\mathrm{FF}}
\newcommand{\sph}{\mathrm{sph}}
\newcommand{\lax}{\mathrm{lax}}
\newcommand{\weak}{\mathrm{weak}}
\newcommand{\strict}{\mathrm{strict}}
\newcommand{\mon}{\mathrm{mon}}
\newcommand{\sym}{\mathrm{sym}}
\newcommand{\lisse}{\mathrm{lisse}}
\newcommand{\an}{\mathrm{an}}
\newcommand{\ad}{\mathrm{ad}}
\newcommand{\sch}{\mathrm{sch}}
\newcommand{\rig}{\mathrm{rig}}
\newcommand{\pol}{\mathrm{pol}}
\newcommand{\plat}{\mathrm{flat}}
\newcommand{\proper}{\mathrm{proper}}
\newcommand{\compl}{\mathrm{compl}}
\newcommand{\non}{\mathrm{non}}
\newcommand{\access}{\mathrm{access}}
\newcommand{\comp}{\mathrm{comp}}
\newcommand{\tstructure}{\mathrm{t}} %t-structures
\newcommand{\pure}{\mathrm{pure}} %pure motives
\newcommand{\mixed}{\mathrm{mixed}} %mixed motives
\newcommand{\num}{\mathrm{num}} %numerical motives
\newcommand{\ess}{\mathrm{ess}}
\newcommand{\topological}{\mathrm{top}}
\newcommand{\convex}{\mathrm{cvx}}
\newcommand{\locconvex}{\mathrm{lcvx}}
\newcommand{\ab}{\mathrm{ab}} %abelian extensions
\newcommand{\inj}{\mathrm{inj}}
\newcommand{\surj}{\mathrm{surj}} %coverage on sets generated by surjections
\newcommand{\eff}{\mathrm{eff}} %effective Cartier divisors
\newcommand{\Weil}{\mathrm{Weil}} %weil divisors
\newcommand{\lex}{\mathrm{lex}}
\newcommand{\rex}{\mathrm{rex}}
\newcommand{\AR}{\mathrm{A\-R}}
\newcommand{\cons}{\mathrm{c}} %constructible sheaves
\newcommand{\tor}{\mathrm{tor}} %tor dimension
\newcommand{\semisimple}{\mathrm{ss}}
\newcommand{\connected}{\mathrm{connected}}
\newcommand{\cg}{\mathrm{cg}} %compactly generated
\newcommand{\nilp}{\mathrm{nilp}}
\newcommand{\isg}{\mathrm{isg}} %isogenous
\newcommand{\qisg}{\mathrm{qisg}} %quasi-isogenous
\newcommand{\irr}{\mathrm{irr}} %irreducible represenations
\newcommand{\simple}{\mathrm{simple}} %simple objects
\newcommand{\indecomp}{\mathrm{indecomp}}
\newcommand{\preproj}{\mathrm{preproj}}
\newcommand{\preinj}{\mathrm{preinj}}
\newcommand{\reg}{\mathrm{reg}}
\renewcommand{\ss}{\mathrm{ss}}

%prism custom command
\usepackage{relsize}
\usepackage[bbgreekl]{mathbbol}
\usepackage{amsfonts}
\DeclareSymbolFontAlphabet{\mathbb}{AMSb} %to ensure that the meaning of \mathbb does not change
\DeclareSymbolFontAlphabet{\mathbbl}{bbold}
\newcommand{\prism}{{\mathlarger{\mathbbl{\Delta}}}}
\newcommand{\simpleroots}{\mathbb{I}}

\begin{document}

    \title{Vertex representations of affine Lie algebras}
    
    \author{Dat Minh Ha}
    \maketitle
    
    \begin{abstract}
    
    \end{abstract}
    
    {
    \hypersetup{} 
    %\dominitoc
    \tableofcontents %sort sections alphabetically
    }

    \section{Introduction}
        \subsection{Notations and conventions}
            All throughout, we work over the field of complex numbers $\bbC$, although any algebraically closed field of characteristic $0$ will suffice, as we only need this assumption so that certain operators will be diagonalisable. 
    
            We fix once and for all a finite-dimensional simple Lie algebra $\g$. Its Cartan matrix will be denoted by $C$, and we will use standard notations (cf. e.g. \cite{humphreys_lie_algebras} and \cite{kac_infinite_dimensional_lie_algebras}) for all of the other data that usually accompanies $\g$, e.g. a choice of simple roots, a subsequently defined root system, etc. We note that to specify these data, we will have to choose a non-degenerate and invariant symmetric bilinear form on $\g$, which shall be denoted by $\kappa$.
    
            The \textbf{affine Lie algebra} attached to the Lie $2$-cocycle $\kappa \in H^2_{\Lie}(\g[v^{\pm 1}], \bbC)$ shall be:
                $$\tilde{\g}_{\kappa} := \uce(\g[v^{\pm 1}])$$
            on which the Lie bracket is given by:
                $$[xf, yg]_{\tilde{\g}_{\kappa}} := [x, y]_{\g} fg + \kappa(x, y) g df$$
            Here, we have used a result by Kassel to identify:
                $$\tilde{\g}_{\kappa} \cong \g[v^{\pm 1}] \oplus \Omega^1_{\bbC[v^{\pm 1}]/\bbC}/d\bbC[v^{\pm 1}] \cong \g[v^{\pm 1}] \oplus \bbC c_{\kappa}$$
            (see \cite{kassel_universal_central_extensions_of_lie_algebras}). We also know, per a result of Garland (see \cite{garland_arithmetics_of_loop_groups}), that $\tilde{\g}_{\kappa}$ is - up to isomorphisms - the only central extension of $\g[v^{\pm 1}]$ (i.e. $H^2_{\Lie}(\g[v^{\pm 1}], \bbC)$); isomorphisms are given by rescaling $\kappa$. One also says that $\tilde{\g}_{\kappa}$ is the affine Lie algebra at \textbf{level} $\kappa$.
    
            The \textbf{untwisted affine Kac-Moody algebra} at level $\kappa$ (as in \cite[Chapter 7]{kac_infinite_dimensional_lie_algebras}) will then be:
                $$\hat{\g}_{\kappa} := \tilde{\g}_{\kappa} \rtimes \bbC D_{\kappa}$$
            where $D_{\kappa} \in \der(\tilde{\g}_{\kappa})$ can be shown to be equal to $v\frac{d}{dv}$. This algebra will be endowed with the non-degenerate and invariant symmetric bilinear form given by:
                $$(xf, yg)_{\hat{\g}_{\kappa}} := \kappa(x, y) \Res(g df)$$
                $$(c_{\kappa}, D_{\kappa})_{\hat{\g}_{\kappa}} := 1$$
            Using this bilinear form, one can construct the Cartan matrix $\hat{C}$ of $\hat{\g}_{\kappa}$ and related combinatorial data, like the affinisation of the root system of $\g$ and so on.
                
            Recall in particular that for $\hat{\g}$ (or indeed, for any Kac-Moody algebra) there is the \textbf{principal $\Z$-grading} of type $\vec{s} := (s_i)_{i \in \hat{\simpleroots}} \in \Z_{\geq 0} \cdot \hat{\simpleroots}$, given on the Chevalley-Serre generators by:
                $$\deg x_i^{\pm} := \pm s_i$$
                $$\deg h_i := 0$$
            for all $i \in \hat{\simpleroots}$. When $\vec{s} := \vec{1} := (1, ..., 1)$, we recover the usual root height grading. 
    
            Suppose also that $c_{\kappa}$ acts as $1$ on standard modules $\standard^{\lambda}$ (and hence also as $1$ on finite-dimensional simple modules and in particular, fundamental modules). In other words, we are stipulating that $\standard^{\lambda}$ carries also the structure of a left-module over $\bar{\rmU}_{\kappa} := \rmU(\tilde{\g}_{\kappa})/\rmU(\tilde{\g}_{\kappa}) \cdot (c_{\kappa} - 1)$.

            There exists also an analogue of $\tilde{\g}_{\kappa}$ constructed using $\bbC(\!(v)\!)$ in place of $\bbC[v^{\pm 1}]$. For a moment, let us denote the affine Kac-Moody algebra that we have been considering up until this point by $\tilde{\g}_{\kappa}^{\pol}$ and the other version by $\tilde{\g}_{\kappa}^{\loc}$. It can be easily shown that the categorry:
                $$\tilde{\g}_{\kappa}^{\pol}\mod$$
            is equivalent to the category:
                $$\tilde{\g}_{\kappa}^{\loc}\mod^{\smooth}$$
            of so-called \textbf{smooth modules} over $\tilde{\g}_{\kappa}^{\loc}$; said smooth modules are those $\tilde{\g}_{\kappa}^{\loc}$-modules on which the Lie subalgebra $\g[\![v]\!] \subset \tilde{\g}_{\kappa}^{\loc}$ acts locally nilpotently. Moreover, this equivalence maps standard modules to standard modules, and hence preserves simplicity of objects, etc. Furthermore, non-smooth modules will never be considered, so we will be using the same notation $\tilde{\g}_{\kappa}$ for both version. Clarifications will be provided if necessary.

            Finally, we will be writing:
                $$\tilde{\rmU}_{\kappa} := \projlim_{n \geq 1} \bar{\rmU}_{\kappa}/\bar{\rmU}_{\kappa} \cdot v^n$$
            Observe that:
                $$\tilde{\rmU}_{\kappa}\mod$$
            is equivalent to both the category of $\tilde{\g}_{\kappa}^{\pol}$-modules on which $c_{\kappa}$ acts as $1$, as well as the category of smooth $\tilde{\g}_{\kappa}^{\loc}$-modules on which $c_{\kappa}$ acts as $1$.

        \subsection{Why vertex operators ?}
            From a representation-theoretic point of view, one possible motivation for the use of vertex operator algebras (VOAs) comes from simply trying to write down some kind of \say{affine Casimir element}. There are many reasons to care about such elements, though one prominent reason is that they generate the centre of some completion of $\rmU(\hat{\g}_{\kappa})$ (what we will denote by $\tilde{\rmU}_{\kappa}$ down below), and hence are useful understanding affine central characters. 
            
            Firstly, recall that for the affine Kac-Moody algebra $\hat{\g}_{\kappa}$, one shall obtain a root space decomposition of the following form, after choosing a Cartan subalgebra $\hat{\h}_{\kappa}$:
                $$\hat{\g}_{\kappa} \cong \hat{\h}_{\kappa} \oplus \bigoplus_{\alpha \in \hat{\Phi}} \hat{\g}_{\kappa, \alpha}$$
            in which:
            \begin{itemize}
                \item the real roots - of which there are finitely many - are all equally of multiplicity $1$, while
                \item the imaginary roots - of which there are infinitely many, as their root spaces are identified with $\h \tensor v^n$ for all $n \in \Z \setminus \{0\}$ - do not necessarily have multiplicity $1$ (indeed, when $\g \not \cong \sl_2$, $\dim \h > 1$).
            \end{itemize}
            Because of this, if we were to insist on writing down the \textit{na\"ive} guess for an affine (quadratic) Casimir element, i.e. something of the form:
                $$\hat{\sfr}_{\text{na\"ive}} := \sum_{(i, n) \in \simpleroots \x \Z} h_i v^n \tensor h^i v^{-n} + \sum_{\alpha \in \Re \hat{\Phi}^+} (x_{\alpha}^+ \tensor x_{\alpha}^- + x_{\alpha}^- \tensor x_{\alpha}^+)$$
            then we would have a hard time trying to make sense of the summand:
                $$\hat{\sfr}_{\text{na\"ive}}^{\Im} := \sum_{(i, n) \in \simpleroots \x (\Z \setminus \{0\})} h_i v^n \tensor h^i v^{-n}$$
            which is a sum over basis-dual basis vectors of the direct sum $\bigoplus_{n \in \Z \setminus \{0\}} \h \tensor v^n$ of all the imaginary root spaces. Worse still, this is a \say{bad} infinite sum, in the sense that the infinite in both the positive and negative directions (recall that formal Laurent series contain infinitely many terms only in either direction, not both). In other words, the expression $\hat{\sfr}_{\text{na\"ive}}^{\Im}$ is properly an element of $\hat{\g}_{\kappa}[\![v^{\pm 1}]\!]$, and this very problematic because $\bbC[\![v^{\pm 1}]\!]$ - the vector space spanned by so-called \say{formal distributions} $\sum_{n \in \Z} a_n v^n$ where, crucially, there can be infinitely many $a_n \not = 0$ - is \textit{not} an associative $\bbC$-algebra at all, so we can not regard $\hat{\g}_{\kappa}[\![v^{\pm 1}]\!]$ as a current algebra!

            If, instead, we were to regard the elements $h_i, h^i$ as operators on some $\hat{\g}_{\kappa}$-module, say $V$, then we can consider:
                $$\hat{\sfr}_{V, \text{na\"ive}}^{\Im}(z) := \sum_{(i, n) \in \simpleroots \x (\Z \setminus \{0\})} h_i z^n \tensor h^i z^{-n} \in \End(V)[\![z^{\pm 1}]\!]$$
            Even though this is still a formal distribution (this time with coefficients in the associative algebra $\End(V)$), it is known that $\End(V)[\![z^{\pm 1}]\!]$ can be used for constructing a VOA structure on the underlying vector space of $V$, and hence expressions like $\hat{\sfr}_V^{\Im}(z)$ can be dealt with as vertex operators.

        \subsection{Generating series}
            To have a better sense of how to proceed from here, let us firstly introduce the physicist's perspective on infinite-dimensional Lie algebras with loop realisations, namely the apprroach via \textbf{generating series}. For untwisted affine Kac-Moody algebras, this ultimately originates from the fact that the extra \say{affinising} Chevalley-Serre generators:
                $$x_0^{\pm} := x_{\theta}^{\pm} v^{\mp 1}, h_0 := [x_{\theta}^+, x_{\theta}^-]$$
            are given in terms of the finite-type generators (elements of $\{x_i^{\pm}, h_i\}_{i \in \simpleroots}$); namely, the highest/lowest root vectors $x_{\theta}^{\pm}$ can be written as ordered monomials in $x_i^{\pm}$. The physical perspective is then, that $x_0^{\pm}, h_0$ ought be obtained as certain \say{residues} of formal distributions:
                $$J(z) := \sum_{n \in \Z} J^{(n)} z^{-n - 1}, J^{(n)} := J v^n$$
            associated to elements $J \in \g$ (and why we have decided to write $z^{-n - 1}$ instead of $z^n$ will become clear shortly); these formal distributions are usually called \textbf{associated fields} or \textbf{generating series} when $J$ is a Chevalley-Serre generator of $\g$.

            Now, to see how commutator relations are given, let us firstly fix a basis $\{J_{\alpha}\}_{\alpha = 1}^{\dim \g}$ for $\g$ and then consider the following:
                $$[J_{\alpha}^{(m)}, J_{\beta}^{(n)}] = \sum_{r = 1}^{\dim \g} C_{\alpha, \beta}^{\gamma} J_{\gamma}^{(m + n)} + m \delta_{m + n, 0} \kappa(J_{\alpha}, J_{\beta}) c_{\kappa}$$
            wherein $C_{\alpha, \beta}^{\gamma} \in \bbC$ are the structural constants. Next, let us recall that the formal Dirac distribution in a formal variable $w$ centered at $z$ is the element of $\bbC[\![w^{\pm 1}, z^{\pm 1}]\!]$ that is given by:
                $$\1(w - z) := \sum_{n \in \Z} w^n z^{-n - 1}$$
            and crucially, this formal distribution satisfies the following identity for all $J(w) \in \g[\![w^{\pm 1}]\!]$:
                $$f(z) = \Res_{w = 0} f(w) \1(w - z)$$
            We can then perform the following computations, which allows us to pass to associated fields:
                $$
                    \begin{aligned}
                        & [J_{\alpha}(w), J_{\beta}(z)]
                        \\
                        := & \sum_{m \in \Z} \sum_{n \in \Z} [J_{\alpha}^{(m)} w^{-m - 1}, J_{\beta}^{(n)} z^{-n - 1}]
                        \\
                        = & \sum_{m \in \Z} \sum_{n \in \Z} \sum_{\gamma = 1}^{\dim \g} C_{\alpha, \beta}^{\gamma} J_{\gamma}^{(m + n)} w^{-m - 1} z^{-n - 1} + \kappa(J_{\alpha}, J_{\beta}) c_{\kappa} \sum_{m \in \Z} \sum_{n \in \Z} m \delta_{m + n, 0} w^{-m - 1} z^{-n - 1}
                        \\
                        = & \sum_{m \in \Z} \sum_{n \in \Z} \sum_{\gamma = 1}^{\dim \g} C_{\alpha, \beta}^{\gamma} J_{\gamma}^{(m + n)} w^{-m - n - 1} w^n z^{-n - 1} + \kappa(J_{\alpha}, J_{\beta}) c_{\kappa} \sum_{m \in \Z} m w^{-m - 1} z^{m - 1}
                        \\
                        = & \sum_{\gamma = 1}^{\dim \g} C_{\alpha, \beta}^{\gamma} J_{\gamma}(w) \1(w - z) + \kappa(J_{\alpha}, J_{\beta}) c_{\kappa} \del_z \sum_{m \in \Z} w^{-m - 1} z^m
                        \\
                        = & [J_{\alpha}, J_{\beta}](w) \1(w - z) + \kappa(J_{\alpha}, J_{\beta}) c_{\kappa} \del_z \1(z - w)
                    \end{aligned}
                $$
            where $\del_z \in \End( \bbC[\![z^{\pm 1}]\!] )$ is the formal partial derivative, given by $\del_z \sum_{m \in \Z} a_m z^m := \sum_{m \in \Z} m a_m z^{m - 1}$. By applying $\Res_{w = 0}$ to both sides, we then get:
                $$
                    \begin{aligned}
                        & \Res_{w = 0} [J_{\alpha}(w), J_{\beta}(z)]
                        \\
                        = & \Res_{w = 0} [J_{\alpha}, J_{\beta}](w) \1(w - z) + \kappa(J_{\alpha}, J_{\beta}) c_{\kappa} \Res_{w = 0} \del_z \1(z - w)
                        \\
                        = & [J_{\alpha}, J_{\beta}](z) + \kappa(J_{\alpha}, J_{\beta}) c_{\kappa} 
                    \end{aligned}
                $$
            One can then choose the basis vectors $J_{\alpha}$ to be orthonormal to one another, and then evaluate at $z = 0$ to get $c_{\kappa}$. 

            Now, observe in particular, that for all $x, y \in \g$, the commutators of fields:
                $$[x(w), y(z)] = [x, y](w) \1(w, z) + \kappa(x, y) \del_w \1(w - z)$$
            are all \textbf{mutually local} as the formal Dirac distribution satisfies:
                $$\forall N \in \Z_{\geq 0}: N > n \implies (w - z)^N \del_w^n \1(w - z) = 0$$
            This suggests to us that VOA structures on $\tilde{\g}_{\kappa}$-modules $V$ - say, specified by a Lie algebra homomorphism $\rho: \tilde{\g}_{\kappa} \to \End(V)$ - can be specified by linear maps:
                $$\rho(z): \g[\![z^{\pm 1}]\!] \to \End(V)[\![z^{\pm 1}]\!]$$
            preserving field commutators, as the mutual locality between fields that is required to happen on the $\End(V)[\![z^{\pm 1}]\!]$ side as a part of the VOA definition is already satisfied on the $\tilde{\g}_{\kappa}[\![z^{\pm 1}]\!]$ side. There is also the so-called \textbf{Vacuum Axiom}, which for our purposes can be simply understood as requiring that $V$ is a highest-weight module on which the central charge $c_{\kappa}$ acts as $1$ (this is more-or-less a normalising condition). Finally, the \textbf{Translation Axiom}, which requires that there is a formal derivation $T$ acting on fields:
                $$J(z) := \sum_{m \in \Z} J^{(m)} z^{-m - 1} \in \End(V)[\![z^{\pm 1}]\!]$$
            via commutators, i.e.:
                $$[T, J(z)] := [T, J^{(m)}](z)$$
            where on the RHS, we regard $T$ as a derivation on $\g[v^{\pm 1}]$ or on $\g(\!(v)\!)$, depending on the situation at hand; typically, one takes $T := \id_{\g} \tensor D$ for some derivation $D$ on $\bbC[v^{\pm 1}]$ or $\bbC(\!(v)\!)$. In combination, these requirements suggest to us that a good candidates for a $\tilde{\g}_{\kappa}$-module that can support VOA structures is the induced module:
                $$\V_{\kappa} := \tilde{\rmU}_{\kappa} \tensor_{\tilde{\rmU}_{\kappa}^+} \bbC c_{\kappa}$$
            Down below, we shall see that, indeed, this module does carry a somewhat canonical VOA structure.

    \section{Vertex representations of the Heisenberg algebra}
        \subsection{The Fock space representations of the Heisenberg algebra}
            If one is willing to accept the idea that attributes of particles in physical systems ought to obey Lie algebra symmetries and hence are to be thought of as certain operators acting on certain vector spaces, then the prototypical case study for this phenomenon in the context of QFT are the \say{Fock space representations} of the \say{Heisenberg algebra} (actually, the Fock spaces are acted on by a slight modification of the universal enveloping algebra of the Heisenberg algebra, but this is a technicality).

            Let us firstly recall the definition of the latter. This is the central extension - which shall be denoted by $\tilde{\g}_0$ - of the abelian Lie algebra $\bbC[v^{\pm 1}]$ by a $1$-dimensional centre $\bbC c_0$ (in physics, one takes $c_0 := -i\hbar$):
                $$0 \to \bbC c_0 \to \tilde{\g}_0 \to \bbC[v^{\pm 1}] \to 0$$
            This Lie algebra admits the subset:
                $$\{q_m\}_{m \in \Z} \cup \{c_0\}$$
            as a basis (physicists tend to write $p_m := q_{-m}$ for $m \geq 0$, as such operators are \say{canonical momentum} operators; the way to remember this is to keep in mind the fact that such operators acts certain partial derivatives, hence lower degrees and as such are given negative indices) whose elements are subjected to the relations:
                $$[q_m, q_n] = m \delta_{m + n, 0} c_0$$
            or in other words, $\tilde{\g}_0$ models countable systems of particles obeying Heisenberg's Uncertainty principal. Notice also, that $\tilde{\g}_0$ corresponds to the Lie $2$-cocycle $\eta_0 \in Z^2_{\Lie}(\bbC[v^{\pm 1}], \bbC)$ given by:
                $$\eta_0(f, g) := \Res(gdf)$$

            Secondly, recall that Fock spaces come in two variations, bosonic and fermionic, whose constructions are experimentally motivated. Regardless, they are to be vector spaces on which so-called \say{creation operators} and \say{annihilation operators} are to act; physically, these are excitations of quantum fields, be they bosonic or fermionic, giving rise to particle creation or annihilation. If we accept the idea that each particle can be represented by a vector space, say:
                $$V$$
            of \say{quantum states} (also called \say{superpositions}), which are vectors that the aforementioned operators act on, and that a system of $n$ particles can be represented by:
                $$V^{\tensor n}$$
            then for the Fock spaces to be stable under these creation and annihilation actions, it must be the quotient of the tensor algebra $\rmT(V) := \bigoplus_{n \geq 0} V^{\tensor n}$ by some finitely generated homogeneous two-sided ideal. Then, via experimentation, we have found the \textbf{bosonic Fock space} and the \textbf{fermionic Fock spaces} attached to such a space of quantum states $V$ to respectively be:
                $$V^{\N} := \Sym(V)$$
                $$V^{\frac12 \N} := \Lambda(V)$$

            Again, from physics, we know that $V$ ought to admit a basis of the form:
                $$\{\xi_m\}_{m \in \N}$$
            (ultimately, this is due to the existence of Fourier bases of $L^2$-spaces). By Heisenberg's Uncertainty Principle, we know that then that there are canonical representations of $\tilde{\g}_0$ on both the bosonic and fermionic Fock spaces $V^{\N}$ and $V^{\frac12 \N}$ that are given by:
                $$
                    q_m \mapsto
                    \begin{cases}
                        \text{$c_0 \cdot$ if $m = 0$}
                        \\
                        \text{$\xi_m \cdot$ if $m \leq 0$}
                        \\
                        \text{$c_0 \frac{\del}{\del \xi_m}$ if $m > 0$}
                    \end{cases}
                $$
            \textit{It is from the existence of such representations that one thinks to endow Fock spaces with vertex algebra structures, as such structures formalise the idea of algebras of differential operators in infinitely many variables.} 

            Before we proceed towards representations of $\tilde{\rmU}_0$, let us also recall how $\tilde{\g}_0$ is realisable as the affine Lie algebra associated to the zero Cartan matrix (and hence the subscript $0$). Firstly, let us recall that a general affine Lie algebra $\tilde{\g}_{\kappa}$ is isomorphic to the Lie algebra generated by the set:
                $$\{x_i^{\pm}\}_{i \in \hat{\simpleroots}}$$
            whose elements are subjected to the usual Kac-Moody relations (see \cite[Chapter 1]{kac_infinite_dimensional_lie_algebras}). By degenerating the Cartan matrix $\hat{C}$ to the $|\hat{\simpleroots}| \x |\hat{\simpleroots}|$ zero matrix (cf. \cite[Section 2.9]{kac_infinite_dimensional_lie_algebras}), i.e. by setting the entries $c_{ij}$ of the affine Cartan matrix $\hat{C}$ to $0$ in the Kac-Moody relations, one shall obtain the Lie algebra generated by the same set, but whose elements are subjected to the following degenerated relations:
                $$[x_i^{\pm}, x_j^{\pm}] = 0$$
                $$[x_i^+, x_j^-] = \delta_{i, j} \alpha_i^{\vee}$$
            given for all $i, j \in \hat{\simpleroots}$. In fact, for this purpose, one can assume without any loss of generality that $\hat{C}$ is symmetric so that $(\theta, \theta)_{\hat{\g}_{\kappa}} = 2$ and hence:
                $$\alpha_0^{\vee} = c_0$$
            Then, by combining this with the fact that affine Lie algebras in general have $1$-dimensional centres, we shall get the desired characterisation of the Heisenberg algebra. In fact, one can write down a Lie algebra embedding $\tilde{\g}_0 \subset \tilde{\g}_{\kappa}$ identifying:
                $$\tilde{\g}_0 \cong \bbC c_0 \oplus \bigoplus_{m \in \Z} \hat{\g}_{\kappa, m\delta} \cong \bbC c_0 \oplus \h \oplus \bigoplus_{m \in \Z \setminus \{0\}} \hat{\g}_{\kappa, m\delta}$$
            wherein $\delta$ is the lowest positive imaginary root of $\hat{\g}_{\kappa}$.
            
            We remark that unlike when $\kappa \not = 0$, the Lie subalgebras $\tilde{\g}_0^{\pm}$ are abelian, not just nilpotent.

            We are now ready to define the \say{Heisenberg vacuum module}, which shall play the role of standard modules in this setting:
            \begin{definition}[The Heisenberg vacuum module] \label{def: heisenberg_vacuum_module}
                The \textbf{Heisenberg vacuum module} is given by:
                    $$\V_0 := \tilde{\rmU}_0 \tensor_{\tilde{\rmU}_0^{\geq 0}} \bbC c_0$$
            \end{definition}
            By PBW, one sees that as a left-$\tilde{\rmU}_0$-module, $\V_0$ is isomorphic to $\tilde{\rmU}_0^-$. Furthermore, we have can identify the abstractly defined $\tilde{\rmU}_0$-module $\V_0$ with the bosonic Fock space of $V := \bbC \cdot \{q_{-m}\}_{m \in \N}$ in a sense. As the latter is isomorphic to $\bbC[\{q_{-m}\}_{m \in \N}]$, it is tempting to say that after completing the bosonic Fock space, both it and $\V_0$ will be isomorphic to $\bbC[\![  \{q_{-m}\}_{m \in \N} ]\!]$. However, the problem is that the algebra $\bbC[\![  \{q_{-m}\}_{m \in \N} ]\!]$ is actually \textit{not} topologically complete, a somewhat strange non-Noetherian phenomenon (see \cite[\href{https://stacks.math.columbia.edu/tag/05JA}{Tag 05JA}]{stacks} for more details).  
            \begin{remark}[An etymological comment]
                The vacuum module is named as such because highest-weight vectors therein, which are generators for this cyclic module, live in the degree-$0$ component of the Fock space, and thus represent vacuum states. 
            \end{remark}

            Let us now construct a VOA structure on $\V_0$, following \cite[Section 2.2]{frenkel_ben_zvi_vertex_algebras_and_algebraic_curves}. 
    
        \subsection{The principal and homogeneous vertex constructions}

    \section{Vertex representations of the Virasoro algebra}

    \section{Vertex representations of affine Lie algebras and the Feigin-Frenkel centre}

    \section{\texorpdfstring{$\hat{\gl}_{\infty}$}{} and the Boson-Fermion Correspondence}
    
    \addcontentsline{toc}{section}{References}
    \printbibliography

\end{document}