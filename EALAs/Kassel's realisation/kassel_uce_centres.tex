\section{The Kassel realisation of UCEs of current Lie algebras}
    \subsection{A recollection of K\"ahler differentials}
        In \cite{kassel_universal_central_extensions_of_lie_algebras}, Kassel showed that centres of UCEs of current algebras (i.e. Lie algebras of the form $\g \tensor_k A$ for some commutative $k$-algebra $A$) are identified with spaces of algebraic differential $1$-forms modulo exact forms, so before proceeding, let us take the time to recall some relevant details about algebraic differential forms. 

        Ultimately, we will be relying on the description of modules of differential forms by generators and relations, but to begin, let us define algebraic differential forms as follows.
        \begin{definition}[Modules of K\"ahler differentials] \label{def: kahler_differentials}
            Let $R$ be a base commutative ring and let $S$ be a commutative $R$-algebra, defined by a multiplication map:
                $$\mu_{S/R}: S \tensor_R S \to S$$
            The $S$-module of K\"ahler differentials $\Omega^1_{S/R}$ relative to the ring map $R \to S$ is then given by:
                $$\Omega^1_{S/R} := I/I^2$$
            where $I := \ker \mu_{S/R}$. Elements of $\Omega^1_{S/R}$ are typically referred to as (differential) $1$-forms.
        \end{definition}
        \begin{remark}[Diffentials and derivations] \label{remark: differentials_and_derivations}
            Observe that the $S \tensor_R S$-ideal:
                $$I := \ker \mu_{S/R}$$
            is generated by elements of the form $1 \tensor f - f \tensor 1$, for all $f \in A$. We then see that:
                $$\Omega^1_{S/R} := I/I^2 \cong I \tensor_{S \tensor_R S} ( (S \tensor_R S)/I ) \cong I \tensor_{S \tensor_R S} S$$
            should we regard $S$ as a commutative $S \tensor_R S$-algebra; note that the last isomorphism holds thanks to the fact that the multiplication map $\mu_{S/R}: S \tensor_R S \to A$ is \textit{a priori} surjective. From this, one infers that there exists a canonical $k$-linear derivation:
                $$d: S \to \Omega^1_{S/R}$$
                $$f \mapsto 1 \tensor f - f \tensor 1$$
            Indeed, for every $f, g \in S$, the Leibniz rule is satisfied:
                $$g df + f dg = g(1 \tensor f - f \tensor 1) + f(1 \tensor g - g \tensor 1) = gf \tensor 1 - fg \tensor 1 = d(fg)$$

            This suggests to us that K\"ahler differentials might have something to do with derivations, and indeed they do. In fact, this relationship between algebraic $1$-forms and derivations comes from a universal property that the $S$-module $\Omega^1_{S/R}$ enjoys. Namely, for any $R$-module $M$, there exists a natural isomorphism of $S$-modules\footnote{The LHS is the $S$-module of $R$-linear derivations from $S$ into $M$.}:
                $$\Der_R(S, M) \cong \Hom_S(\Omega^1_{S/R}, M)$$
            (cf. \cite[\href{https://stacks.math.columbia.edu/tag/00RO}{Tag 00RO}]{stacks}). A particular instance of this phenomenon is that $R$-linear derivations from $S$ to itself are dual to differential $1$-forms relative to $R \to S$, in the sense that there is an $S$-module isomorphism:
                $$\Der_R(S) := \Der_R(S, S) \cong \Hom_A(\Omega^1_{S/R}, A)$$
            By definition, elements of $D \in \Der_R(S)$ satisfy the Leibniz rule (i.e. $D(ab) = b D(a) + a D(b)$ for every $a, b \in A$), so the above tells us also that $\Omega^1_{S/R}$ is isomorphic to the $S$-module generated by the set:
                $$\{ds\}_{s \in S}$$
            whose elements are constrained by the relations:
                $$\forall s, s' \in S: d(ss') = s' ds + s ds'$$
        \end{remark}
        \begin{remark}
            If $\Omega^1_{S/R}$ is finite free of rank $n$ over $A$, e.g.:
                $$\Omega^1_{S/R} \cong \bigoplus_{1 \leq i \leq n} A dv_i$$
            then we can identify:
                $$\Der_R(S) \cong \bigoplus_{1 \leq i \leq n} A \del_{v_i}$$
            where $\del_{v_i} \in \Der_R(S)$ are the preimages under the isomorphism $\Der_R(S) \xrightarrow[]{\cong} \Hom_S(\Omega^1_{S/R}, S)$ of the $S$-linear duals of the generators $dv_i \in \Omega^1_{S/R}$. 
        \end{remark}
        
        The following well-known lemmas are very useful. Proofs can be be found in any standard reference on general commutative algebra (e.g. \cite[\href{https://stacks.math.columbia.edu/tag/00AO}{Tag 00AO}]{stacks}).
        \begin{lemma}[$1$-forms over polynomial algebras] \label{lemma: 1_forms_over_polynomial_algebras}
            \cite[\href{https://stacks.math.columbia.edu/tag/00RX}{Tag 00RX}]{stacks} Let $R$ be a commutative ring and fix some $n \in \Z_{\geq 0}$. In this case, $\Omega^1_{R[v_1, ..., v_n]/R}$ will be free and of finite rank $n$ as an $R[v_1, ..., v_n]$-module; in particular, it admits the set $\{dv_1, ..., dv_n\}$ as a $R[v_1, ..., v_n]$-linear basis.
        \end{lemma}
        \begin{lemma}[Localisation of $1$-forms] \label{lemma: localisation_1_forms}
            \cite[\href{https://stacks.math.columbia.edu/tag/031G}{Tag 031G}]{stacks} Let $k$ be a field\footnote{... so that the only prime ideal of $k$ would be $(0)$.} and fix some $n \in \Z_{\geq 0}$, and consider the canonical ring homomorphism $k \to k[v_1, ..., v_n]$. Then, for any $1 \leq i \leq n$, there will be a $k[v_1, ..., v_n][v_i^{-1}]$-module isomorphism:
                $$\Omega^1_{k[v_1, ..., v_n][v_i^{-1}]/k} \cong \Omega^1_{k[v_1, ..., v_n]/k}[v_i^{-1}]$$
        \end{lemma}

        Let us end this subsection with the following examples, which will be useful for what comes later on.
        \begin{example}
            Let $k$ be a field.
        
            Per lemma \ref{lemma: 1_forms_over_polynomial_algebras}, we know that:
                $$\Omega^1_{k[v, t]/k} \cong k[v, t] dv \oplus k[v, t] dv$$
            Using lemma \ref{lemma: localisation_1_forms}, we then see that:
                $$\Omega^1_{k[v^{\pm 1}, t^{\pm 1}]/k} \cong k[v^{\pm 1}, t^{\pm 1}] dv \oplus k[v^{\pm 1}, t^{\pm 1}] dt$$
                $$\Omega^1_{k[v^{\pm 1}, t]/k} \cong k[v^{\pm 1}, t] dv \oplus k[v^{\pm 1}, t] dt$$
        \end{example}

    \subsection{UCEs of current Lie algebras}
        \begin{convention} \label{conv: a_fixed_finite_dimensional_simple_lie_algebra}
            From now on, we fix a finite-dimensional simple Lie algebra $\g$ over an algebraically closed field $k$ of characteristic $0$, equipped with a symmetric and non-degenerate invariant $k$-bilinear form $(-, -)_{\g}$. We fix also a Cartan subalgebra $\h$ of $\g$, along with all the accompanying data (e.g. root system, Cartan matrix, etc.) as in subsection \ref{subsection: finite_dimensional_simple_lie_algebras}. 
        \end{convention}

        For the sake of fixing terminologies, let us make the following definition:
        \begin{definition}[Current algebras]
            Let $A$ be a commutative algebra over $k$. The vector space:
                $$\g \tensor_k A$$
            with the following Lie bracket:
                $$[x f, y g]_{\g \tensor_k A} := [x, y]_{\g} \tensor fg$$
            (given for all $x, y \in \g$ and all $f, g \in A$) shall then be referred to as a \textbf{current algebra}. 
                
            Also, we will be abbreviating:
                $$xf := x \tensor f$$
            for $x \in \g$ and $f \in A$.
        \end{definition}
        \begin{remark}[(Multi)loop algebras]
            When $A \cong k[v_1^{\pm 1}, ..., v_n^{\pm 1}]$, it is common to refer to $\g \tensor_k A$ as a \textbf{multiloop algebra}. When $n = 1$, we will only be saying \textbf{loop algebra}.
        \end{remark}

        \begin{convention}
            Let $R \to S$ be a homomorphism of commutative rings. Then, let us write:
                $$\bar{\Omega}^1_{S/R} := \Omega^1_{S/R}/dS$$
            Note that this is only an $R$-module, not an $S$-module. Let us also write $\bar{d}: S \to \bar{\Omega}^1_{S/R}$ for the canonical composition:
                $$
                    \begin{tikzcd}
                	S & {\Omega^1_{S/R}} \\
                	& {\bar{\Omega}^1_{S/R}}
                	\arrow["d", from=1-1, to=1-2]
                	\arrow[two heads, from=1-2, to=2-2]
                	\arrow["{\bar{d}}"', from=1-1, to=2-2]
                    \end{tikzcd}
                $$
        \end{convention}

        In order to characterise centres of UCE Kassel constructed in the proof of \cite[Theorem 3.3(iii)]{kassel_universal_central_extensions_of_lie_algebras} a $k$-linear map:
            $$\e: \bigwedge^2 (\g \tensor_k A) \to \bar{\Omega}^1_{A/k}$$
        by the formula:
            \todo[inline]{The formula in \cite{kassel_universal_central_extensions_of_lie_algebras} says $f \bar{d}g$, but in \cite{moody_rao_yokonuma_vertex_representations_of_toroidal_lie_algebras} and \cite{wendlandt_formal_shift_operators_on_yangian_doubles}, it is $g \bar{d}f$. Should there not be an extra minus sign for the latter ?}
            $$\e(x f, y g) := (x, y)_{\g} f \bar{d}g$$
        for all $x, y \in \g$ and for all $f, g \in A$. This can be shown - relying on the $\g$-invariance of the bilinear form $(-, -)_{\g}$ - to be an element of $H^2_{\Lie}(\g \tensor_k A, k)$ and hence gives a central extension $\fraku$ of $\g \tensor_k A$ by $\bar{\Omega}_{A/k}^1$ (cf. proposition \ref{prop: lie_brackets_on_extensions}), whose underlying $k$-vector space is:
            $$(\g \tensor_k A) \oplus \bar{\Omega}_{A/k}^1$$
        and whose Lie bracket is:
            $$[-, -]_{\fraku} := [-, -]_{\g \tensor_k A} + \e$$
        \begin{lemma} \label{lemma: lie_brackets_on_UCEs_of_current_algebras}
            $[-, -]_{\fraku}$ as constructed above is a well-defined Lie bracket.
        \end{lemma}
            \begin{proof}
                By construction, it is already bilinear and skew-symmetric, so the only thing to show is that it satisfies the Jacobi identity. To this end, pick $x, y, z \in \g$ and $f, g, h \in A$ and then consider the following:
                    $$
                        \begin{aligned}
                            & [xf, [yg, zh]_{\fraku}]_{\fraku} + [yg, [zh, xf]_{\fraku}]_{\fraku} + [zh, [xf, yg]_{\fraku}]_{\fraku}
                            \\
                            & = [xf, [y, z]_{\g} gh + \e(yg, zh)]_{\fraku} + [yg, [z, x]_{\g} hf + \e(zh, xf)]_{\fraku} + [zh, [x, y]_{\g} fg + \e(xf, yg)]_{\fraku}
                            \\
                            & = [xf, [y, z]_{\g} gh]_{\fraku} + [yg, [z, x]_{\g} hf]_{\fraku} + [zh, [x, y]_{\g} fg]_{\fraku}
                            \\
                            & = 
                            \begin{aligned}
                                & \left( [x, [y, z]_{\g}]_{\g} fgh + \e(xf, [y, z]_{\g} gh) \right)
                                \\
                                + & \left( [y, [z, x]_{\g}]_{\g} ghf + \e(yg, [z, x]_{\g} hf) \right)
                                \\
                                + & \left( [z, [x, y]_{\g}]_{\g} hfg + \e(zh, [x, y]_{\g} fg) \right)
                            \end{aligned}
                            \\
                            & = \e(xf, [y, z]_{\g} gh) + \e(yg, [z, x]_{\g} hf) + \e(zh, [x, y]_{\g} fg)
                            \\
                            & = (x, [y, z]_{\g})_{\g} f \bar{d}(gh) + (y, [z, x]_{\g})_{\g} g \bar{d}(hf) + (z, [x, y]_{\g})_{\g} h \bar{d}(fg)
                            \\
                            & = (x, [y, z]_{\g})_{\g} ( f \bar{d}(gh) + g \bar{d}(hf) + h \bar{d}(fg) )
                            \\
                            & = 0
                        \end{aligned}
                    $$
            \end{proof}
        Kassel then showed that the Lie algebra $\fraku$ as above is a UCE of $\g \tensor_k A$. 
        \begin{theorem}[The Kassel realisation] \label{theorem: kassel_realisation}
            \cite[Corollary 3.5]{kassel_universal_central_extensions_of_lie_algebras} For the (perfect) Lie $k$-algebra $\g \tensor_k A$, we have that:
                $$\uce(\g \tensor_k A) \cong (\g \tensor_k A) \oplus \bar{\Omega}^1_{A/k}$$
            with Lie bracket as in lemma \ref{lemma: lie_brackets_on_UCEs_of_current_algebras}.
        \end{theorem}
        
        \begin{example}
            It is trivial to see that:
                $$\dim_k \bar{\Omega}^1_{k/k} \cong 0$$
            from which one sees that:
                $$\uce(\g) \cong \g$$
            i.e. $\g$ is its own universal central extension, and hence every central extension of $\g$ is trivial. Of course, there are other more conventional ways to see that $\g$ admits no non-trivial central extensions, but we thought we would include this example as a particularly degenerate case of Kassel's Theorem.
        \end{example}
        \begin{example}[Affine Lie algebras] \label{example: affine_lie_algebras_centres}
            Let us compute the UCE of $\g[v^{\pm 1}]$. From this, we can construct the so-called \say{untwisted affine Kac-Moody algebra} attached to $\g$ (cf. \cite[Chapter 7]{kac_infinite_dimensional_lie_algebras}). 

            To this end, let us firstly compute the underlying vector space of the centre of $\uce(\g[v^{\pm 1}])$. Abstractly, we know that it is isomorphic to $\bar{\Omega}^1_{k[v^{\pm 1}]/k}$, and it is also known that:
                $$\Omega^1_{k[v^{\pm 1}]/k} \cong k[v^{\pm 1}] dv \cong \bigoplus_{m \in \Z} k \cdot v^m dv$$
            so the only non-trivial computation to make is that of $d k[v^{\pm 1}]$. For this, let us consider how $d$ acts on the basis elements $v^m \in k[v]$:
                $$d(v^m) = m v^{m - 1} dv$$
            We see that $d(v^m) = 0$ if and only if $m = 0$, and since the set $\{v^m\}_{m \in \Z}$ is a $k$-linear basis for $k[v^{\pm 1}]$, the set:
                $$\{m v^{m - 1} dv\}_{m \in \Z \setminus \{0\}} \cong \{v^m dv\}_{m \in \Z \setminus \{-1\}}$$
            therefore spans $d(k[v^{\pm 1}])$. It is also easy to see this subset of $d(k[v^{\pm 1}])$ is $k$-linearly independent and hence is a basis for $d(k[v^{\pm 1}])$. This then tells us that:
                $$\bar{\Omega}^1_{k[v^{\pm 1}]/k} \cong k \cdot v^{-1} \bar{d}v$$
            The underlying vector space of $\uce(\g[v^{\pm 1}])$ is thus isomorphic to:
                $$\g[v^{\pm 1}] \oplus k \cdot v^{-1} \bar{d}v$$

            We know that the Lie bracket on $\uce(\g[v^{\pm 1}])$ is given by:
                $$[x f, y g]_{\g[v^{\pm 1}]} = [x, y]_{\g} fg + (x, y)_{\g} f \bar{d}g$$
            for all $x, y \in \g$ and all $f, g \in k[v^{\pm 1}]$. Since:
                $$f \bar{d}g \in k \cdot v^{-1} \bar{d}v$$
            necessarily, the bracket can be given simplier as:
                $$[x f, y g]_{\g[v^{\pm 1}]} = [x, y]_{\g} fg + (x, y)_{\g} c(f, g) v^{-1} \bar{d}v$$
            for a uniquely determined scalar $c(f, g) \in k$, which can be computed explicitly. To do this, if suffices to perform the computation for basis elements of $k[v^{\pm 1}]$, i.e. we can pick $f := v^m, g := v^n$ for some $m, n \in \Z$ and then consider the following:
                $$f \bar{d}g = v^m \bar{d}(v^n) = n v^{m + n} v^{-1} \bar{d}v$$
            This expression vanishes if and only if $m + n = 0$, so we have that:
                $$c(v^m, v^n) = n \delta_{m + n, 0}$$
            For general $f, g \in k[v^{\pm 1}]$, we can write this more succinctly as:
                $$c(f, g) = \Res_{v = 0} fdg$$
            The Lie bracket on $\uce(\g[v^{\pm 1}])$ then takes the form:
                $$[x f, y g]_{\g[v^{\pm 1}]} = [x, y]_{\g} fg + (x, y)_{\g} \Res_{v = 0} f dg v^{-1} \bar{d}v$$
                
            Note also, that there is a non-degenerate and invariant\footnote{Because $(-, -)_{\g}$ is invariant.} symmetric bilinear form on $\g[v^{\pm 1}]$ given by:
                $$(xf, yg)_{\g[v^{\pm 1}]} := (x, y)_{\g} \Res_{v = 0} f dg$$
            for all $x, y \in \g$ and all $f, g \in k[v^{\pm 1}]$. By invariance, the extension of this bilinear form to $\uce(\g[v^{\pm 1}])$ is necessarily degenerate.

            Let us also note that unlike $\uce(\g[v^{\pm 1}])$, the UCE of the perfect Lie algebra $\g[v]$ is trivial, since:
                $$\forall f, g \in k[v]: \Res_{v = 0}(fg) = 0$$
        \end{example}
        \begin{example}[Toroidal Lie algebras] \label{example: toroidal_lie_algebras_centres}
            Next, let us compute the UCE of $\g[v^{\pm 1}, t^{\pm 1}]$. 
            
            Firstly, let us compute its underlying vector space, for which the only non-trivial computation to make is that of the $k$-vector space $\bar{\Omega}^1_{k[v^{\pm 1}, t^{\pm 1}]/k}$, which we know to be isomorphic to the centre of $\uce(\g[v^{\pm 1}, t^{\pm 1}])$. We know that:
                $$\Omega^1_{k[v^{\pm 1}, t^{\pm 1}]/k} \cong k[v^{\pm 1}, t^{\pm 1}] dv \oplus k[v^{\pm 1}, t^{\pm 1}] dt$$
            and since:
                $$k[v^{\pm 1}, t^{\pm 1}] \cong \bigoplus_{(m, p) \in \Z^2} k \cdot v^m t^p$$
            we consequently have that:
                $$\Omega^1_{k[v^{\pm 1}, t^{\pm 1}]/k} \cong \bigoplus_{(m, p) \in \Z^2} (k \cdot v^m t^p dv \oplus k \cdot v^m t^p dt)$$
            It can also be seen that:
                $$d(k[v^{\pm 1}, t^{\pm 1}]) \cong \bigoplus_{(m, p) \in \Z^2} k \cdot d(v^m t^p) \cong \bigoplus_{(m, p) \in \Z^2 \setminus \{(0, 0)\}} ( k \cdot m v^{m - 1} t^p dv \oplus k \cdot p v^m t^{p - 1} dt )$$
            since:
                $$m v^{m - 1} t^p dv + p v^m t^{p - 1} dt = 0$$
            if and only if $(m, p) = (0, 0)$. As such, the $k$-vector space:
                $$\bar{\Omega}^1_{k[v^{\pm 1}, t^{\pm 1}]/k} := \Omega^1_{k[v^{\pm 1}, t^{\pm 1}]/k}/d k[v^{\pm 1}, t^{\pm 1}]$$
            decomposes in the following manner:
                $$\bar{\Omega}^1_{k[v^{\pm 1}, t^{\pm 1}]/k} \cong ( \bigoplus_{(r, s) \in \Z^2} k K_{r, s}) \oplus k c_v \oplus k c_t$$
            wherein:
                $$
                    K_{r, s} :=
                    \begin{cases}
                        \text{$\frac1s v^{r - 1} t^s \bar{d}v$ if $(r, s) \in \Z \x (\Z \setminus \{0\})$}
                        \\
                        \text{$-\frac1r v^r t^{-1} \bar{d}t$ if $(r, s) \in (\Z \setminus \{0\}) \x \{0\}$}
                        \\
                        \text{$0$ if $(r, s) = (0, 0)$}
                    \end{cases}
                $$
                $$c_v := v^{-1} \bar{d}v, c_t := t^{-1} \bar{d}t$$
            In fact, any element of the form:
                $$v^m t^p \bar{d}(v^n t^q) \in \bar{\Omega}^1_{k[v^{\pm 1}, t^{\pm 1}]/k}$$
            can be written in terms of the basis vectors $K_{r, s}, c_v, c_t$ in the following manner:
                $$v^m t^p \bar{d}(v^n t^q) = \delta_{(m, p) + (n, q), (0, 0)} ( n c_v + q c_t ) + (np - mq) K_{m + n, p + q}$$
            (cf. \cite[pp. 35]{wendlandt_formal_shift_operators_on_yangian_doubles}).

            Finally, let us note that the Lie bracket on $\uce(\g[v^{\pm 1}, t^{\pm 1}])$ is given by:
                $$
                    \begin{aligned}
                        [x v^m t^p, y v^n t^q]_{\uce(\g[v^{\pm 1}, t^{\pm 1}])} & = [x, y]_{\g} v^{m + n} t^{p + q} + (x, y)_{\g} v^m t^p \bar{d}(v^n t^q)
                        \\
                        & = [x, y]_{\g} v^{m + n} t^{p + q} + ( \delta_{(m, p) + (n, q), (0, 0)} ( n c_v + q c_t ) + (np - mq) K_{m + n, p + q} )
                    \end{aligned}
                $$
            for all $x, y \in \g$ and all $(m, p), (n, q) \in \Z^2$. As a side note, let us note that interestingly, unlike how $\g[v]$ admits only the trivial UCE, $\g[v^{\pm 1}, t]$ admits a non-trivial UCE, on which the Lie bracket is given by:
                $$[x v^m t^p, y v^n t^q]_{\uce(\g[v^{\pm 1}, t^{\pm 1}])} = [x, y]_{\g} v^{m + n} t^{p + q} + (np - mq) K_{m + n, p + q}$$
            for all $x, y \in \g$ and all $(m, p), (n, q) \in \Z \x \Z_{\geq 0}$.
        \end{example}
        \begin{remark}[The $\Z$-grading on toroidal Lie algebras] \label{remark: Z_gradings_on_toroidal_lie_algebras}
            If $k$ is a field and $A$ is a commutative $k$-algebra graded by an abelian group $Z$, say:
                $$A := \bigoplus_{n \in Z} A_n$$
            and if $\a$ is a Lie algebra over $k$, then $\a \tensor_k A$ will also be $\Z$-graded, namely in the following manner:
                $$\a \tensor_k A \cong \bigoplus_{n \in \Z} \a \tensor_k A_n$$
            This grading on $\a \tensor_k A$ actually extends to the whole of $\uce(\a_A)$. To see why this is the case, recall firstly that the $A$-module $\Omega^1_{A/k}$ is isomorphic to the quotient of $A \tensor_k A$ (which carries a grading induced by the one on $A$, with graded components given by $(A \tensor_k A)_d \cong \bigoplus_{m + n = d} (A_m \tensor_k A_n)$ for all $d \in Z$) by the homogeneous $A$-submodule generated by the relations:
                $$fg \tensor 1 - g \tensor f - f \tensor g$$
            given for all $f, g \in A$. This implies that $\Omega^1_{A/k}$ inherits a $Z$-grading from the one on $A \tensor_k A$, given by:
                $$\deg f dg = \deg f \tensor g = \deg f + \deg g$$
            Since the vector subspace $d(A)$ is also $Z$-graded ($\deg df = \deg f \tensor 1 = \deg f$ for all $f \in A$, as $d(A) \cong A \tensor 1$), the quotient vector space:
                $$\bar{\Omega}^1_{A/k} := \Omega^1_{A/k}/d(A)$$
            is also naturally graded by the abelian group $Z$, with graded components given by:
                $$(\bar{\Omega}^1_{A/k})_n := (\Omega^1_{A/k})_n/d(A_n) \cong (\Omega^1_{A/k})_n/(A_n \tensor 1)$$
            for all $n \in Z$. As we have that:
                $$\uce(\g \tensor_k A) \cong (\g \tensor_k A) \oplus \bar{\Omega}^1_{A/k}$$
            (cf. theorem \ref{theorem: kassel_realisation}), the $Z$-grading on $\g \tensor_k A$ thus extends to the larger Lie algebra $\uce(\g \tensor_k A)$, as claimed: namely, the graded components are given by:
                $$\uce(\g \tensor_k A)_n := (\g \tensor_k A_n) \oplus (\bar{\Omega}^1_{A/k})_n$$
            for all $n \in Z$. 

            When:
                $$A := k[v^{\pm 1}, t^{\pm 1}]$$
            consider the $\Z$-grading given by:
                $$\deg v := 0, \deg t := 1$$
            The induced $\Z$-grading on $\bar{\Omega}^1_{k[v^{\pm 1}, t^{\pm 1}]/k}$ is therefore given by:
                $$
                    \deg K_{r, s} =
                    \begin{cases}
                        \text{$s$ if $(r, s) \in \Z \x (\Z \setminus \{0\})$}
                        \\
                        \text{$0$ if $(r, s) \in (\Z \setminus \{0\}) \x \{0\}$}
                        \\
                        \text{$0$ if $(r, s) = (0, 0)$}
                    \end{cases}
                $$
                $$\deg c_v = 0, \deg c_t = 0$$
        \end{remark}

    \subsection{Root gradings on UCEs of current Lie algebras}
        Finally, let us demonstrate how Kassel's realisation is useful for endowing the Lie algebras of the kind $\g \tensor_k A$ with a grading coming from the root lattice $Q$ of $\g$. We will be focusing only on the cases $A \cong k[v^{\pm 1}]$ and $A \cong k[v^{\pm 1}, t^{\pm 1}]$, on which one obtains gradings by the abelian group $Q \x \Z$.

        A construction that we will be making use of frequently is the untwisted affine Kac-Moody algebra associated to a finite-dimensional simple Lie algebra. Let us briefly recall how it is constructed. For details, we refer the reader to \cite[Chapter 7]{kac_infinite_dimensional_lie_algebras}.

        Recall from example \ref{example: affine_lie_algebras_centres} that on the Lie algebra:
            $$\uce(\g[v^{\pm 1}]) \cong \g[v^{\pm 1}] \oplus k c_v$$
        (with $c_v := v^{-1} \bar{d}v$ as in example \ref{example: toroidal_lie_algebras_centres}) carries a uniquely determined invariant symmetric bilinear form $(-, -)_{\uce(\g[v^{\pm 1}])}$ which is necessarily degenerate. However, by adding an extra element $\hat{D}$ to the Lie algebra $\uce(\g[v^{\pm 1}])$ and by requiring that:
            $$(c_v, \hat{D})_{\uce(\g[v^{\pm 1}]) \oplus k \hat{D}} = 1$$
            $$(c_v, c_v)_{\uce(\g[v^{\pm 1}]) \oplus k \hat{D}} = (\hat{D}, \hat{D})_{\uce(\g[v^{\pm 1}]) \oplus k \hat{D}} = 0$$
        one obtains the \textbf{untwisted affine Kac-Moody algebra} (in the sense of \cite[Chapter 7]{kac_infinite_dimensional_lie_algebras}):
            $$\hat{\g} := \uce(\g[v^{\pm 1}]) \rtimes k \hat{D}$$
        The element $\hat{D}$ can be shown to be the derivation on $\g[v^{\pm 1}]$ given by $\id_{\g} \tensor v \frac{d}{dv}$. It can also be shown that:
            $$[\hat{D}, c_v]_{\hat{\g}} = 0$$
        
        \begin{convention} \label{conv: a_fixed_untwisted_affine_kac_moody_algebra}
            We shall also fix once and for all a Cartan subalgebra:
                $$\hat{\h}$$
            of $\hat{\g}$, along with a non-degenerate, invariant, and symmetric $k$-bilinear form:
                $$(-, -)_{\hat{\g}}$$
            on $\hat{\g}$ (cf. \cite[Chapter 2]{kac_infinite_dimensional_lie_algebras}). The set of simple roots of $\hat{\g}$ with respect to the aforementioned choices of Cartan subalgebra and non-degenerate invariant bilinear form will be denoted by:
                $$\hat{\simpleroots}$$
        \end{convention}
        
        \begin{example}
            For what follows, let us recall from \cite[Chapter 7]{kac_infinite_dimensional_lie_algebras} that the root space decomposition of the untwisted affine Kac-Moody algebra $\hat{\g}$ takes the form:
                $$\hat{\g} \cong \hat{\h} \oplus \bigoplus_{\beta \in \Re(\hat{\Phi})} \hat{\g}_{\beta} \oplus \bigoplus_{\beta \in \Im(\hat{\Phi})} \hat{\g}_{\beta}$$
            in which the untwisted affine root system $\hat{\Phi}$ decomposes into a disjoint union of the subsets of real and imaginary roots:
                $$\hat{\Phi} \cong \Re(\hat{\Phi}) \cup \Im(\hat{\Phi})$$
            where:
                $$\Re(\hat{\Phi}) \cong \Phi + \Z\delta \cong \Phi \x \Z$$
                $$\Im(\hat{\Phi}) \cong (\Z \setminus \{0\})\delta$$
            and the corresponding root spaces are given by:
                $$\forall \alpha + m\delta \in \Re(\hat{\Phi}): \hat{\g}[\alpha + m\delta] \cong \g_{\alpha} v^m$$
                $$\forall r\delta \in \Im(\hat{\Phi}): \hat{\g}[r\delta] \cong \h v^r$$
            Recall also - from \cite[Chapter 5]{kac_infinite_dimensional_lie_algebras} - that:
                $$\forall \alpha + m \delta \in \Re(\hat{\Phi}): \dim_k \hat{\g}[\alpha + m\delta] = 1$$
            entirely as a consequence of the fact that $\dim_k \g_{\alpha} = 1$. 
        
            Now, consider the Lie algebra $\uce(\g[v^{\pm 1}])$, which is nothing but the derived subalgebra of $\hat{\g}$. It is isomorphic to the Lie algebra generated by the set:
                $$\{ X_{i, r}^{\pm} \}_{(i, r) \in \hat{\simpleroots} \x \Z}$$
            whose elements are subjected to the following relations:
                $$[X_{i, r}^{\pm}, X_{j, s}^{\pm}] = \pm N_{\alpha_i, \alpha_j} X_{i + j, r + s}^{\pm}$$
            The isomorphism in question is given by:
                $$\forall (i, r) \in \simpleroots \x \Z: X_{i, r}^{\pm} \mapsto x_i^{\pm} v^r$$
                $$X_{\theta, r}^{\pm} \mapsto x_{\theta}^{\mp} v^r + c_v$$
            where $x_{\theta}^{\mp} \in \g_{\mp \theta}$ are choices of root vectors (well-defined since $\dim_k \g_{\mp \theta} = 1$).
        \end{example}

        \begin{convention}[Toroidal and positive-toroidal Lie algebras]
            From now on, we will be writing:
                $$A := k[v^{\pm 1}, t^{\pm 1}], A^{\positive} := k[v^{\pm 1}, t]$$
                $$\g_{[2]} := \g \tensor_k A, \g_{[2]}^{\positive} := \g \tensor_k A^{\positive}$$
                $$\bar{\Omega}_{[2]} := \bar{\Omega}^1_{A/k}, \bar{\Omega}^+_{[2]} := \bar{\Omega}^1_{A_{[2]^+}/k}$$

            We will be considering the Lie algebras:
                $$\toroidal := \uce(\g_{[2]})$$
                $$\toroidal^{\positive} := \uce(\g_{[2]}^{\positive})$$
            As the Lie algebras $\g_{[2]}$ and $\g_{[2]}^{\positive}$ are perfect, the Lie algebras $\toroidal$ and $\toroidal^{\positive}$ are well-defined and per theorem \ref{theorem: kassel_realisation}, their centres are respectively given by:
                $$\z_{[2]} := \z(\uce(\toroidal)) \cong \bar{\Omega}_{[2]}$$
                $$\z_{[2]}^{\positive} := \z(\uce(\toroidal^{\positive})) \cong \bar{\Omega}_{[2]}^{\positive}$$
        \end{convention}
        For $\toroidal$ and $\toroidal^{\positive}$, Kassel's presentation from theorem \ref{theorem: kassel_presentations_for_UCEs} can be futhermore refined into Chevalley-Serre-style presentations from which one can obtain $\hat{Q}$-gradings for these two Lie algebras. These presentations were originally due to Moody-Rao-Yokonuma (see \cite{moody_rao_yokonuma_vertex_representations_of_toroidal_lie_algebras}). 
        \begin{lemma}[Chevalley-Serre presentation for $\toroidal$] \label{lemma: chevalley_serre_presentation_for_central_extensions_of_multiloop_algebras}
            The Lie algebra $\toroidal$ is isomorphic to the Lie algebra generated by the set:
                $$\{ X_{i, r}^{\pm}, H_{i, r} \}_{(i, r) \in \hat{\simpleroots} \x \Z} \cup \{ K \}$$
            whose elements are subjected to the following relations, given for all $(i, r), (j, s) \in \hat{\simpleroots} \x \Z$:
                $$[ H_{i, r}, H_{j, s} ] = 0$$
                $$[ H_{i, r}, X_{j, s}^{\pm} ] = \pm (\alpha_j, \check{\alpha}_i) X_{j, r + s}^{\pm}$$
                $$[ X_{i, r}^+, X_{j, s}^- ] = \delta_{ij} H_{i, r + s}$$
                $$[ X_{i, r + 1}^{\pm}, X_{j, s}^{\pm} ] - [ X_{i, r}^{\pm}, X_{j, s + 1}^{\pm} ] = 0$$
                $$\ad(X_{i, 0}^{\pm})^{1 - c_{ij}}( X_{j, s}^{\pm} ) = 0$$
                $$[K, \toroidal] = 0$$
            The isomorphism in question is given as follows, for all $(i, r) \in \hat{\simpleroots} \x \Z$:
                $$\forall (i, r) \in \simpleroots \x \Z: X_{i, r}^{\pm} \mapsto x_i^{\pm} t^r, H_{i, r} \mapsto h_i t^r$$
                $$\forall (i, r) \in \{\theta\} \x \Z: X_{\theta, r}^{\pm} \mapsto x_{\theta}^{\mp} v^{\pm 1} t^r, H_{\theta, r} \mapsto h_{\theta} t^r + c_v t^r$$
                $$K \mapsto c_t$$
        \end{lemma}
        \begin{corollary}[Chevalley-Serre presentation for $\toroidal^{\positive}$]
            The Lie algebra $\toroidal^{\positive}$ is isomorphic to the Lie algebra generated by the set:
                $$\{ X_{i, r}^{\pm}, H_{i, r} \}_{(i, r) \in \hat{\simpleroots} \x \Z_{\geq 0}}$$
            whose elements are subjected to the following relations lemma \ref{lemma: chevalley_serre_presentation_for_central_extensions_of_multiloop_algebras}. The isomorphism in question is given also as in \textit{loc. cit.}
        \end{corollary}
            \begin{proof}
                Apply the computations in example \ref{example: toroidal_lie_algebras_centres} to lemma \ref{lemma: chevalley_serre_presentation_for_central_extensions_of_multiloop_algebras}.
            \end{proof}