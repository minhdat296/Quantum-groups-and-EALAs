\section{Classification of Yangian extended toroidal Lie algebras}
    \subsection{Yangian toroidal cocycles}
        Let:
            $$0 \to \p \to \fraky \xrightarrow[]{\pi} \d \to 0$$
        be a Lie algebra extension such that $\p$ is an $\d$-module, i.e. let $\fraky$ be a twisted semi-direct product $\p \rtimes^{\sigma} \d$ (where $\sigma: \bigwedge^2 \d \to \p$ is some $2$-cocycle). In proposition \ref{prop: twisted_semi_direct_product_criterion}, we have seen that any such cocycle arises as the difference:
            $$\sigma(D, D') := [\gamma(D), \gamma(D')]_{\fraky} - \gamma( [D, D']_{\d} )$$
        for arbitrary elements $D, D' \in \d$ and for some choice of linear section:
            $$\gamma: \d \to \fraky$$
        We therefore see that to give such a linear section $\gamma: \d \to \fraky$ is the same as to give a $2$-cocycle $\sigma: \bigwedge^2 \d \to \p$, i.e. the cocycle $\sigma$ measures how far the extension $(\fraky, \pi)$ is away from splitting, i.e. from being the semi-direct product $\p \rtimes \d$ (which is more-or-less the definition of $2$-cocycles of Lie algebras; cf. definition \ref{def: twisted_semi_direct_products}).

        Keeping the above in mind, let us return to the setting of definition \ref{def: yangian_extended_toroidal_lie_algebras}. A natural question to ask, given this definition, is as follows:
        \begin{question}
            Amongst the twisted semi-direct products $\fraky(\sigma) := \toroidal \rtimes^{\sigma} \d_{[2]}$, which ones are Yangian extended toroidal Lie algebras ? 
        \end{question}
        Since $\toroidal$ automatically embeds into any twited semi-direct product $\fraky(\sigma)$ as a Lie subalgebra, and since the underlying vector space of $\fraky(\sigma)$ is $\toroidal \oplus \d_{[2]}$ by definition, in order to answer this question, it shall suffice to give a criterion on the cocycle:
            $$\sigma: \bigwedge^2 \d_{[2]} \to \toroidal$$
        so that there would exist an \textit{invariant} and \textit{non-degenerate} symmetric bilinear form $(-, -)_{\sigma}$ on $\fraky(\sigma)$.

        For convenience, let us fix the following terminologies.
        \begin{definition}[General extended toroidal Lie algebras] \label{def: general_extended_toroidal_lie_algebras}
            Any twisted semi-direct product:
                $$\fraky(\sigma) := \toroidal \rtimes^{\sigma} \d_{[2]}$$
            shall be called an \textbf{extended toroidal Lie algebra}.
        \end{definition}
        \begin{definition}[Yangian toroidal $2$-cocycles] \label{def: yangian_toroidal_cocycles}
            Any $2$-cocyle $\sigma: \bigwedge^2 \d_{[2]} \to \toroidal$ shall be referred to as a \textbf{toroidal $2$-cocycle}.
            
            Any toroidal $2$-cocycle $\sigma$ such that $\fraky(\sigma)$ is a Yangian extended toroidal Lie algebra (in the sense of definition \ref{def: yangian_extended_toroidal_lie_algebras}) shall be called a \textbf{Yangian toroidal $2$-cocycle}.
        \end{definition}
        \begin{remark}
            Since we now know that Yangian extended toroidal Lie algebras are necessarily isomorphic to some twisted semi-direct product $\fraky(\sigma)$ (cf. theorem \ref{theorem: yangian_extended_toroidal_lie_algebras_preliminary_version} and corollary \ref{coro: yangian_extended_toroidal_lie_algebras_are_twisted_semi_direct_products}), definitions \ref{def: general_extended_toroidal_lie_algebras} and \ref{def: yangian_toroidal_cocycles} as above make sense.
        \end{remark}
        \begin{remark}[Yangian toroidal $2$-cocycles are central]
            Thanks to proposition \ref{prop: lie_bracket_on_orthogonal_complement_of_toroidal_centre}, we know that the codomain of any Yangian toroidal $2$-cocycle necessarily lies inside $\z_{[2]}$. See corollary \ref{coro: lie_brackets_on_central_extensions} as well.
        \end{remark}

    \subsection{Which extended toroidal Lie algebras are Yangian ?}
        Fix a toroidal $2$-cocyle:
            $$\sigma: \bigwedge^2 \d_{[2]} \to \toroidal$$
        along with a \textit{non-degenerate} symmetric bilinear form:
            $$(-, -)_{\sigma}: \Sym^2_k( \fraky(\sigma) ) \to k$$
        such that:
        \begin{itemize}
            \item the restriction of $(-, -)_{\sigma}$ down to the vector subspace $\g_{[2]} \oplus \z_{[2]}$ coincides with $(-, -)_{\toroidal}$, and
            \item $(\z_{[2]}, \d_{[2]})_{\sigma} \not = 0$ and $(\g_{[2]}, \d_{[2]})_{\sigma} = 0$ and $(\d_{[2]}, \d_{[2]})_{\sigma} = 0$.
        \end{itemize}
        Our task now, as indicated above, is to find a criterion on $\sigma$ so that $(-, -)_{\sigma}$ would be invariant with respect to the Lie bracket on $\fraky(\sigma)$.
        
        Firstly, observe that, per theorem \ref{theorem: yangian_extended_toroidal_lie_algebras} and corollary \ref{coro: yangian_extended_toroidal_lie_algebras_are_twisted_semi_direct_products}, we can infer particularly that:
            $$(-, -)_{\sigma}$$
        for any $\sigma$, is invariant with respect to $[-, -]_0$, in the sense that:
            $$([X, Y]_0, Z)_{\sigma} = (X, [Y, Z]_0)_{\sigma}$$
        for any $X + D, Y + D', Z + D'' \in \toroidal \oplus \d_{[2]}$. Next, recall that by definition \ref{def: twisted_semi_direct_products} (see also: example \ref{example: lie_algebra_semi_direct_products}), the Lie bracket on the twisted semi-direct product $\fraky(\sigma)$, which henceforth will be denoted by $[-, -]_{\sigma}$, is given by:
            $$[-, -]_{\sigma} = [-, -]_0 + \sigma \circ \pi$$
        where $[-, -]_0$ denotes the Lie bracket on the semi-direct product $\fraky(0) \cong \toroidal \rtimes \d_{[2]}$ and $\pi: \fraky(\sigma) \to \d_{[2]}$ is the canonical projection. Then, consider the following:
            $$
                \begin{aligned}
                    ([X + D', Y + D'']_{\sigma}, Z + D'')_{\sigma} & = ([X + D, Y + D']_0 + \sigma(D, D'), Z)_{\sigma}
                    \\
                    & = ([X + D, Y + D']_0, Z + D'')_{\sigma} + (\sigma(D, D'), Z + D'')_{\sigma}
                    \\
                    & = (X + D, [Y + D', Z + D'']_0)_{\sigma} + (\sigma(D, D'), Z + D'')_{\sigma}
                \end{aligned}
            $$
        We see then, that it shall suffices to find conditions on $\sigma$ so that:
            $$(\sigma(D, D'), Z + D'')_{\sigma} = (X + D, \sigma(D', D''))_{\sigma}$$
        For any $\zeta \in \g_{[2]} \oplus \z_{[2]} \oplus \d_{[2]}$, let us write:
            $$\zeta := \zeta_{\g_{[2]}} + \zeta_{\z_{[2]}} + \zeta_{\d_{[2]}}$$
        for its decomposition into its $\g_{[2]}$, $\z_{[2]}$, and $\d_{[2]}$-summands. We see then, that the equation $(\sigma(D, D'), Z + D'')_{\sigma} = (X + D, \sigma(D', D''))_{\sigma}$ is equivalently to:
            $$(\sigma(D, D')_{\g_{[2]}}, Z_{\g_{[2]}})_{\sigma} + (\sigma(D, D')_{\z_{[2]}}, D'')_{\sigma} = (X_{\g_{[2]}}, \sigma(D', D'')_{\g_{[2]}})_{\sigma} + (D, \sigma(D', D'')_{\z_{[2]}})_{\sigma}$$
        As $X, Y, Z \in \toroidal$ were chosen arbitrarily, the problem can thus be rephrased as follows.
        \begin{question}
            What are the necessarily conditions on a given toroidal $2$-cocycle:
                $$\sigma: \bigwedge^2 \d_{[2]} \to \toroidal$$
            so that:
                $$(\sigma(D, D')_{\g_{[2]}}, yg)_{\sigma} + (\sigma(D, D')_{\z_{[2]}}, D'')_{\sigma} = (xf, \sigma(D', D'')_{\g_{[2]}})_{\sigma} + (D, \sigma(D', D'')_{\z_{[2]}})_{\sigma}$$
            for all $D, D', D'' \in \d_{[2]}$, and for all $x, y \in \g$ and all $f, g \in A$.
        \end{question}
        \begin{remark} \label{remark: yangian_criterion_for_toroidal_cocycles}
            We know that for a toroidal $2$-cocycle:
                $$\sigma: \bigwedge^2 \d_{[2]} \to \toroidal$$
            to be Yangian is $\z_{[2]}$, its codomain must be $\z_{[2]}$. As such, it shall suffice to find a condition that it must satisfy so that:
                $$(\sigma(D, D')_{\z_{[2]}}, D'')_{\sigma} = (D, \sigma(D', D'')_{\z_{[2]}})_{\sigma}$$
            in order to make sure that the bilinear form $(-, -)_{\sigma}$ would be invariant with respect to $[-, -]_{\sigma}$. Without nay loss of generality, we can take $D, D', D'' \in \d_{[2]}$ to be basis elements. Suppose also, that $\sigma(D, D'), \sigma(D', D'') \in \z_{[2]}$ can be written as a linear combination of the basis elements of $\z_{[2]}$ as follows:
                $$\sigma(D, D') := \sum_{(m, p) \in \Z^2} \lambda_{m, p}(D, D') K_{m, p} + \lambda_v(D, D') c_v + \lambda_t(D, D') c_t$$
                $$\sigma(D', D'') := \sum_{(m, p) \in \Z^2} \mu_{m, p}(D', D'') K_{m, p} + \mu_v(D', D'') c_v + \mu_t(D', D'') c_t$$
            with the coefficients depending on the choices of elements $D, D', D'' \in \d_{[2]}$. We then have that:
                $$
                    (\sigma(D, D')_{\z_{[2]}}, D'')_{\sigma} =
                    \begin{cases}
                        \text{$\lambda_{\alpha, \beta}(D, D')$ if $D'' = D_{\alpha, \beta}$, for any $(\alpha, \beta) \in \Z^2$}
                        \\
                        \text{$\lambda_v(D, D')$ if $D'' = D_v$}
                        \\
                        \text{$\lambda_t(D, D')$ if $D'' = D_t$}
                    \end{cases}
                $$
                $$
                    (D, \sigma(D', D'')_{\z_{[2]}})_{\sigma} =
                    \begin{cases}
                        \text{$\mu_{\e, \eta}(D', D'')$ if $D = D_{\e, \eta}$, for any $(\e, \eta) \in \Z^2$}
                        \\
                        \text{$\mu_v(D', D'')$ if $D = D_v$}
                        \\
                        \text{$\mu_t(D', D'')$ if $D = D_t$}
                    \end{cases}
                $$
            This means that in order to see if $(-, -)_{\sigma}$ is an invariant bilinear form, it shall suffice to compute:
                $$\sigma(D, D'), \sigma(D', D'')$$
            as linear combinations of the basis elements of $\z_{[2]}$, record the coefficients as entries of two $4 \x 3$ matrices, namely:
                $$
                    \bfA(\sigma) :=
                    \begin{pmatrix}
                        \lambda_{\alpha, \beta}(D_{r, s}, D_{a, b}) & \lambda_v(D_{r, s}, D_{a, b}) & \lambda_t(D_{r, s}, D_{a, b})
                        \\
                        \lambda_{\alpha, \beta}(D_{r, s}, D_v) & \lambda_v(D_{r, s}, D_v) & \lambda_t(D_{r, s}, D_v)
                        \\
                        \lambda_{\alpha, \beta}(D_{r, s}, D_t) & \lambda_v(D_{r, s}, D_t) & \lambda_t(D_{r, s}, D_t)
                        \\
                        \lambda_{\alpha, \beta}(D_v, D_t) & \lambda_v(D_v, D_t) & \lambda_t(D_v, D_t)
                    \end{pmatrix}
                $$
                $$
                    \bfB(\sigma) :=
                    \begin{pmatrix}
                       \mu_{r, s}(D_{a, b}, D_{\alpha, \beta}) & \mu_v(D_{a, b}, D_{\alpha, \beta}) & \mu_t(D_{a, b}, D_{\alpha, \beta})
                        \\
                       \mu_{r, s}(D_{a, b}, D_v) & \mu_v(D_{a, b}, D_v) & \mu_t(D_{a, b}, D_v)
                        \\
                       \mu_{r, s}(D_{a, b}, D_t) & \mu_v(D_{a, b}, D_t) & \mu_t(D_{a, b}, D_t)
                        \\
                       \mu_{r, s}(D_v, D_t) & \mu_v(D_v, D_t) & \mu_t(D_v, D_t)
                    \end{pmatrix}
                $$
            and then compare said matrices. If these matrices are equal, then we will have that:
                $$(\sigma(D, D')_{\z_{[2]}}, D'')_{\sigma} = (D, \sigma(D', D'')_{\z_{[2]}})_{\sigma}$$
            i.e. $\sigma$ will be Yangian.
        \end{remark}
        By putting everything together, one obtains the following result characteristing Yangian $2$-cocycles amongst all the toroidal $2$-cocycles.
        \begin{theorem}[A Yangian-ness criterion for toroidal $2$-cocycles] \label{theorem: yangian_criterion_for_toroidal_cocycles}
            A toroidal $2$-cocycle:
                $$\sigma: \bigwedge^2 \d_{[2]} \to \z_{[2]}$$
            is Yangian if and only if:
                $$(\sigma(D, D')_{\z_{[2]}}, D'')_{\sigma} = (D, \sigma(D', D'')_{\z_{[2]}})_{\sigma}$$
            for all $D, D', D'' \in \d_{[2]}$, which can be taken to be basis elements without loss of generality. Equivalently, by letting:
                $$\sigma(D, D') := \sum_{(m, p) \in \Z^2} \lambda_{m, p}(D, D') K_{m, p} + \lambda_v(D, D') c_v + \lambda_t(D, D') c_t$$
                $$\sigma(D', D'') := \sum_{(m, p) \in \Z^2} \mu_{m, p}(D', D'') K_{m, p} + \mu_v(D', D'') c_v + \mu_t(D', D'') c_t$$
            be linear combinations in terms of the basis elements of $\z_{[2]}$, where the coefficients depend on the elements $D, D', D'' \in \z_{[2]}$, one sees that $\sigma$ is Yangian if and only if:
                $$\bfA(\sigma) = \bfB(\sigma)$$
            with $\bfA(\sigma), \bfB(\sigma)$ being the $4 \x 3$ matrices as in remark \ref{remark: yangian_criterion_for_toroidal_cocycles}.
        \end{theorem}

        Now that we have a criterion for a given toroidal $2$-cocycle to be Yangian, let us apply it to some known toroidal $2$-cocycles from \cite{billig_energy_momentum_tensor} to check whether or not they are Yangian. 
        \begin{example} \label{example: yangian_cocycles_(counter)_examples}
            From proposition \ref{prop: lie_bracket_on_orthogonal_complement_of_toroidal_centre}, we know that:
                $$[\d_{[2]}, \d_{[2]}]_{\extendedtoroidal} \subset \z_{[2]} \oplus \d_{[2]}$$
            we can obtain some toroidal $2$-cocyles:
                $$\sigma: \bigwedge^2 \d_{[2]} \to \z_{[2]}$$
            by restricting $2$-cocycles of $\der(A)$ with values in $\z_{[2]}$.

            It pays to abstract the situation out to the $n$-variable case momentarily. In \cite[p. 5, below Equation 1.3]{billig_energy_momentum_tensor}, it was noted that there are at least $2$-cocyles that we shall denote by:
                $$\sigma_1, \sigma_2: \bigwedge^2 \der(\bbC[v_1^{\pm 1}, ..., v_n^{\pm 1}]) \to \bar{\Omega}^1_{\bbC[v_1^{\pm 1}, ..., v_n^{\pm 1}]/\bbC}$$
            which are given in terms of the basis:
                $$\{ v_1^{m_1} ... v_n^{m_n} \cdot v_p \del_{v_p} \}_{(m_1, ..., m_n, p) \in \Z^n \x \Z}$$
            of $\der(\bbC[v_1^{\pm 1}, ..., v_n^{\pm 1}])$ by:
                $$\sigma_1(v_1^{m_1} ... v_n^{m_n} \cdot v_a \del_{v_a}, v_1^{r_1} ... v_n^{r_n} \cdot v_b \del_{v_b}) := r_a m_b \cdot v_1^{m_1} ... v_n^{m_n} \bar{d}( v_1^{r_1} ... v_n^{r_n} )$$
                $$\sigma_2(v_1^{m_1} ... v_n^{m_n} \cdot v_a \del_{v_a}, v_1^{r_1} ... v_n^{r_n} \cdot v_b \del_{v_b}) := r_b m_a \cdot v_1^{m_1} ... v_n^{m_n} \bar{d}( v_1^{r_1} ... v_n^{r_n} )$$
            given for every $1 \leq a, b \leq n$ and every $(m_1, ..., m_n), (r_1, ..., r_n) \in \Z^n$. 

            Now, back to the $2$-variable case, i.e. $n = 2$, where we have:
                $$v_1 := v, v_2 := t$$
            In this setting, we know how the basis elements $D_{r, s}, D_v, D_t$ of $\d_{[2]}$ are given in terms of $A$-multiples of the partial derivatives $\del_v, \del_t$ (cf. lemma \ref{lemma: derivation_action_on_multiloop_algebras}), so we can exploit the bilinearity of $2$-cocycles in order to see how $\sigma_1, \sigma_2$ act on elements of $\d_{[2]}$. Recall from lemma \ref{lemma: derivation_action_on_multiloop_algebras} that, in terms of the partial derivatives $\del_v, \del_t$, the basis elements of $\d_{[2]}$ are given by:
                $$\forall (r, s) \in \Z^2: D_{r, s} = s v^{-r + 1} t^{-s - 1} \del_v - r v^{-r} t^{-s} \del_t$$
                $$D_v = -v t^{-1} \del_v$$
                $$D_t = -\del_t$$
            Knowing this allows us to perform the following computations, where $i \in \{1, 2\}$:
            \begin{itemize}
                \item 
                    $$
                        \begin{aligned}
                            & \sigma_i(D_{r, s}, D_{a, b})
                            \\
                            = & \sigma_i( s v^{-r + 1} t^{-s - 1} \del_v - r v^{-r} t^{-s} \del_t, b v^{-a + 1} t^{-b - 1} \del_v - a v^{-a} t^{-b} \del_t )
                            \\
                            = & \sigma_i( s v^{-r} t^{-s - 1} \cdot v\del_v - r v^{-r} t^{-s - 1} \cdot t \del_t, b v^{-a} t^{-b - 1} \cdot v\del_v - a v^{-a} t^{-b - 1} \cdot t \del_t )
                            \\
                            = & s \sigma_i( v^{-r} t^{-s - 1} \cdot v\del_v, b v^{-a} t^{-b - 1} \cdot v\del_v - a v^{-a} t^{-b - 1} \cdot t \del_t ) - r \sigma_i( v^{-r} t^{-s - 1} \cdot t \del_t, b v^{-a} t^{-b - 1} \cdot v\del_v - a v^{-a} t^{-b - 1} \cdot t \del_t )
                            \\
                            = &
                            \begin{aligned}
                                & s b \cdot \sigma_i( v^{-r} t^{-s - 1} \cdot v\del_v, v^{-a} t^{-b - 1} \cdot v\del_v )
                                \\
                                - & s a \cdot \sigma_i( v^{-r} t^{-s - 1} \cdot v\del_v, v^{-a} t^{-b - 1} \cdot t \del_t )
                                \\
                                - & r b \cdot \sigma_i( v^{-r} t^{-s - 1} \cdot t \del_t, v^{-a} t^{-b - 1} \cdot v\del_v )
                                \\
                                + & r a \cdot \sigma_i( v^{-r} t^{-s - 1} \cdot t \del_t, v^{-a} t^{-b - 1} \cdot t \del_t )
                            \end{aligned}
                            \\
                            = & N_i(r, s, a, b) v^{-r} t^{-s - 1} \bar{d}( v^{-a} t^{-b - 1} )
                        \end{aligned}
                    $$
                where:
                    $$
                        \begin{aligned}
                            N_i(r, s, a, b) & = 
                            sbra
                            - sa \left( \delta_{i, 1} a(s + 1) + \delta_{i, 2} (b + 1) r \right) 
                            - rb \left( \delta_{i, 1} (b + 1) r + \delta_{i, 2} a (s + 1) \right)
                            + r a (s + 1) (b + 1)
                            \\
                            & = 
                            \begin{cases}
                                \text{$
                                    sbra
                                    - s a^2 (s + 1) 
                                    - r^2 b (b + 1)
                                    + r a (s + 1) (b + 1)
                                $if $i = 1$}
                                \\
                                \text{$
                                    sbra
                                    - sa (b + 1) r
                                    - rb a (s + 1)
                                    + r a (s + 1) (b + 1)
                                $ if $i = 2$}
                            \end{cases}
                            \\
                            & = 
                            \begin{cases}
                                \text{$
                                    sbra
                                    - ( (sa)^2 + s a^2 ) 
                                    - ( (rb)^2 + r^2 b ) 
                                    + rasb + rsa + rab + ra
                                $ if $i = 1$}
                                \\
                                \text{$
                                    sbra
                                    - (sabr + sar)
                                    - (rbas + rba)
                                    + rasb + rsa + rab + ra
                                $ if $i = 2$}
                            \end{cases}
                            \\
                            & = 
                            \begin{cases}
                                \text{$2 rsab - ( (sa)^2 + s a^2 ) - ( (rb)^2 + r^2 b ) + rsa + rab + ra$ if $i = 1$}
                                \\
                                \text{$ra$ if $i = 2$}
                            \end{cases}
                        \end{aligned}
                    $$
                
                Now, recall from example \ref{example: toroidal_lie_algebras_centres} that any element:
                    $$v^n t^q \bar{d}(v^m t^p) \in \z_{[2]}$$
                can be written in terms of the basis elements of $\z_{[2]}$ in the following manner:
                    $$v^n t^q \bar{d}(v^m t^p) = \delta_{(m, p) + (n, q), (0, 0)} ( n c_v + q c_t ) + (np - mq) K_{m + n, p + q}$$
                Using this, we shall be able to conclude that:
                    $$\sigma_i(D_{r, s}, D_{a, b}) = N_i(r, s, a, b) \left( -\delta_{(r, s), -(a, b)} (r c_v + (s + 1) c_t) + ( r(b + 1) - a(s + 1) )K_{-r - a, -s - b - 2} \right)$$
                \item
                    $$
                        \begin{aligned}
                            & \sigma_i(D_{r, s}, D_v)
                            \\
                            = & \sigma_i( s v^{-r + 1} t^{-s - 1} \del_v - r v^{-r} t^{-s} \del_t, -v t^{-1} \del_v )
                            \\
                            = & \sigma_i( s v^{-r} t^{-s - 1} \cdot v \del_v - r v^{-r} t^{-s - 1} t \del_t, -t^{-1} \cdot v \del_v )
                            \\
                            = & s \sigma_i( v^{-r} t^{-s - 1} \cdot v \del_v, t^{-1} \cdot v \del_v ) - r\sigma_i( v^{-r} t^{-s - 1} \cdot t \del_t, t^{-1} \cdot v \del_v )
                            \\
                            = & 0 - r \cdot \delta_{i, 1}r v^{-r} t^{-s - 1} \bar{d}(t^{-1})
                            \\
                            = & \delta_{i, 1} r^2 v^{-r} t^{-s - 1} \bar{d}(t^{-1})
                        \end{aligned}
                    $$
                Immediately, we see that:
                    $$\sigma_2(D_{r, s}, D_v) = 0$$
                for all $(r, s) \in \Z^2$, so from now on we will only be concerned with $\sigma_1(D_{r, s}, D_v)$, which by now we know to be given by:
                    $$\sigma_1(D_{r, s}, D_v) = r^2 v^{-r} t^{-s - 1} \bar{d}(t^{-1})$$
                Now, recall from example \ref{example: toroidal_lie_algebras_centres} that any element:
                    $$v^n t^q \bar{d}(v^m t^p) \in \z_{[2]}$$
                can be written in terms of the basis elements of $\z_{[2]}$ in the following manner:
                    $$v^n t^q \bar{d}(v^m t^p) = \delta_{(m, p) + (n, q), (0, 0)} ( n c_v + q c_t ) + (np - mq) K_{m + n, p + q}$$
                Using this, we shall get that:
                    $$
                        \begin{aligned}
                            & \sigma_1(D_{r, s}, D_v)
                            \\
                            = & r^2 \left( -\delta_{(r, s), (0, -2)} ( r c_v + (s + 3) c_t ) - r K_{-r, -s - 2} \right)
                            \\
                            = &
                            \begin{cases}
                                \text{$0$ if $(r, s) \in \{0\} \x \Z$}
                                \\
                                \text{$r^3 K_{-r, -s - 2}$ if $(r, s) \in (\Z \setminus \{0\}) \x \Z$}
                            \end{cases}
                        \end{aligned}
                    $$
                \item
                    $$
                        \begin{aligned}
                            & \sigma_i(D_{r, s}, D_t)
                            \\
                            = & \sigma_i( s v^{-r + 1} t^{-s - 1} \del_v - r v^{-r} t^{-s} \del_t, -\del_t )
                            \\
                            = & \sigma_i( s v^{-r} t^{-s - 1} \cdot v \del_v - r v^{-r} t^{-s - 1} \cdot t \del_t, -t^{-1} \cdot t\del_t )
                            \\
                            = & s \sigma_i( v^{-r} t^{-s - 1} \cdot v \del_v, t^{-1} \cdot t\del_t ) - r\sigma_i( v^{-r} t^{-s- 1} \cdot t \del_t, t^{-1} \cdot t\del_t )
                            \\
                            = & -s \cdot \left( \delta_{i, 1} \cdot 0 + \delta_{i, 2} r v^{-r} t^{-s - 1} \bar{d}(t^{-1}) \right) + r \cdot s v^{-r} t^{-s - 1} \bar{d}(t^{-1})
                            \\
                            = & (-\delta_{i, 2} + 1) rs v^{-r} t^{-s - 1} \bar{d}(t^{-1})
                        \end{aligned}
                    $$
                Immediately, we see that:
                    $$\sigma_2(D_{r, s}, D_t) = 0$$
                for all $(r, s) \in \Z^2$, so from now on we will only be concerned with $\sigma_1(D_{r, s}, D_t)$, which by now we know to be given by:
                    $$\sigma_1(D_{r, s}, D_t) = rs v^{-r} t^{-s - 1} \bar{d}(t^{-1})$$
                Now, recall from example \ref{example: toroidal_lie_algebras_centres} that any element:
                    $$v^n t^q \bar{d}(v^m t^p) \in \z_{[2]}$$
                can be written in terms of the basis elements of $\z_{[2]}$ in the following manner:
                    $$v^n t^q \bar{d}(v^m t^p) = \delta_{(m, p) + (n, q), (0, 0)} ( n c_v + q c_t ) + (np - mq) K_{m + n, p + q}$$
                Using this, we shall get that:
                    $$
                        \begin{aligned}
                            & \sigma_1(D_{r, s}, D_t)
                            \\
                            = & rs \left( -\delta_{(r, s), (0, -2)} ( r c_v + (s + 1) c_t ) + r K_{-r, -s - 2} \right)
                            \\
                            = & 
                            \begin{cases}
                                \text{$0$ if $(r, s) \in \{0\} \x \Z$}
                                \\
                                \text{$r^2s K_{-r, -s - 2}$ if $(r, s) \in (\Z \setminus \{0\}) \x \Z$}
                            \end{cases}
                        \end{aligned}
                    $$
                \item
                    $$
                        \begin{aligned}
                            & \sigma_i(D_v, D_t)
                            \\
                            = & \sigma_i(-v t^{-1} \del_v, -\del_t)
                            \\
                            = & \sigma_i(t^{-1} \cdot v \del_v, t^{-1} t \del_t)
                            \\
                            = & 0
                        \end{aligned}
                    $$
            \end{itemize}

            One can now use the criterion given in theorem \ref{theorem: yangian_criterion_for_toroidal_cocycles} to verify whether or not the cocycles $\sigma_1, \sigma_2$ are Yangian in the sense of definition \ref{def: yangian_toroidal_cocycles}.
        \end{example}
        Whether or not these cocycles might be cohomologous to $0$ is a much subtler issue. One way to tackle this problem is to check whether or not their restriction to a particular Lie subalgebra of $\extendedtoroidal$ is cohomologous to $0$. 

        Recall from lemma \ref{lemma: derivation_action_on_multiloop_algebras} that:
            $$\forall (r, s) \in \Z^2: D_{r, s} = s v^{-r + 1} t^{-s - 1} \del_v - r v^{-r} t^{-s} \del_t$$
            $$D_v = -v t^{-1} \del_v$$
            $$D_t = -\del_t$$
        and from lemma \ref{lemma: explicit_commutators_between_basis_elements_of_toroidal_central_orthogonal_complement} that the commutation relations that these basis elements of $\d_{[2]}$ satisfy are:
            $$[D_v, D_t] = 0$$
            $$[D_v, D_{r, s}] = r D_{r, s + 1}$$
            $$[D_t, D_{r, s}] = D_{r, s + 1}$$
            $$[D_{a, b}, D_{r, s}] = (br - sa) D_{a + r, b + s + 1}$$
        (given for all $(r, s), (a, b) \in \Z^2$). With these information in mind, one sees that the following vector subspace of $\d_{[2]}$:
            $$\frakw := \bigoplus_{r \in \Z} \bbC D_{r, -1}$$
        is actually a Lie subalgebra, as the basis elements satisfy the following commutators:
            $$[D_{a, -1}, D_{r, -1}] = (a - r) D_{a + r, -1}$$
        given for all $a, r \in \Z$; note also that we have $D_v, D_t \not \in \frakw$ because:
            $$[D_v, D_{r, -1}] = r D_{r, 0} \not \in \frakw$$
            $$[D_t, D_{r, -1}] = D_{r, 0} \not \in \frakw$$    
        for all $r \in \Z$. Interestingly, these are precisely the commutation relations satisfied by the elements of the following basis of the Lie algebra $\der(\bbC[v^{\pm 1}])$:
            $$\{ d_r := -v^r D_{\aff} \}_{r \in \Z}$$
        (where $D_{\aff} := v \frac{d}{dv}$ is the \say{untwisted affine Kac-Moody derivation} as in subsection \ref{subsection: a_fixed_untwisted_affine_kac_moody_algebra}) and in light of this, we make the following observation:
        \begin{lemma}[A copy of the Witt algebra inside $\d_{[2]}$]
            There is an isomorphism of Lie algebras:
                $$\der(\bbC[v^{\pm 1}]) \xrightarrow[]{\cong} \frakw$$
            given by:
                $$d_r \mapsto D_{r, -1}$$
            This identifies a copy of $\der(\bbC[v^{\pm 1}])$ inside $\d_{[2]}$ as a Lie subalgebra. 
        \end{lemma}

        The Lie algebra $\der(\bbC[v^{\pm 1}])$ is known to possess a \textit{non-trivial} UCE:
            $$\frakv := \der(\bbC[v^{\pm 1}]) \oplus^{\eta} \bbC c_{\frakv}$$
        called the \textbf{Virasoro algebra}, whose corresponding $2$-cocycle:
            $$\eta: \bigwedge^2 \der(\bbC[v^{\pm 1}]) \to \bbC c_{\frakv}$$
        is given by:
            $$\eta(d_r, d_a) := \delta_{r + a, 0} (r^3 - r) c_{\frakv}$$
        for all $r, a \in \Z$; in particular, this means this \textit{$\eta$ is non-cohomologous to $0$}. The $2$-cocycle $\eta$ is cohomologically unique (in the sense that any $2$-cocycle $\eta': \bigwedge^2 \der(\bbC[v^{\pm 1}]) \to \bbC c_{\frakv}$ is cohomologous to $\eta$ itself; lemma \ref{lemma: uniqueness_of_virasoro_cocycle}), and so should either $\sigma_1$ or $\sigma_2$ become cohomologous to $\eta$ after being restricted down to $\bigwedge^2 \frakv$, they would have to be non-cohomologous to $0$. We claim that this is indeed true.

        The answer to the following question turns out to be negative, but in answering it, we will have gained some insight into how we might show that $\sigma_1$ and $\sigma_2$ are actually cohomologous to $\eta$.
        \begin{question}
            Is it true that:
                $$\sigma_i|_{ \bigwedge^2 \frakv } = \eta$$
            for either $i = 1$ or $i = 2$ ?
        \end{question}
        Using the computations in example \ref{example: yangian_cocycles_(counter)_examples}, we see that:
            $$\sigma_i(D_{r, -1}, D_{a, -1}) = -N_i(r, -1, a, -1) \delta_{r + a, 0} r c_v$$
        where:
            $$
                N_i(r, -1, a, -1) =
                \begin{cases}
                    \text{$2 ra - ra - ra + ra$ if $i = 1$}
                    \\
                    \text{$ra$ if $i = 2$}
                \end{cases}
                = ra
            $$
        and hence, more succinctly, we have that:
            $$\sigma_i(D_{r, -1}, D_{a, -1}) = -\delta_{r + a, 0} r^3 c_v$$
        regardless of whether $i = 1$ or $i = 2$, and for all $r, a \in \Z$. If it was true that:
            $$\sigma_i(D_{r, -1}, D_{a, -1}) = \eta(d_r, d_a)$$
        then we must have that:
            $$-\delta_{r + a, 0} r^3 c_v = \delta_{r + a, 0} (r^3 - r) c_{\frakv}$$
        which implies that:
            $$-r^3 c_v = (r^3 - r) c_{\frakv}$$
        for all $r \in \Z$. But this is certainly not true for all $r \in \Z$: e.g. if $r = 1$ then we will get that:
            $$-c_v = -v^{-1} \bar{d}v = 0$$
        which is absurd! As such, \textit{neither} of the $2$-cocycles $\sigma_1$ and $\sigma_2$ coincide with $\eta$ when restricted down to $\bigwedge^2 \frakv$. In particular, this means that it is still inconclusive as to whether or not the toroidal $2$-cocycles $\sigma_1, \sigma_2$ are cohomologous to $0$. However, this does not necessarily imply that $\eta$ and $\sigma_1, \sigma_2$ are \textit{not} cohomologous. 

        \begin{proposition}[Some other Virasoro $2$-cocycles] \label{prop: auxiliary_virasoro_cocycles}
            Let:
                $$\eta': \bigwedge^2 \der(\bbC[v^{\pm 1}]) \to \bbC c_{\frakv}$$
            be the function given by:
                $$\eta'(d_r, d_a) := \delta_{r + a, 0} r c_{\frakv}$$
            This is a $2$-cocycle of $\der(\bbC[v^{\pm 1}])$ with values in $\bbC c_{\frakv}$. From this, one sees that:
                $$\eta + \eta': \bigwedge^2 \der(\bbC[v^{\pm 1}]) \to \bbC c_{\frakv}$$
            (which is given by $(\eta + \eta')(d_r, d_a) := \delta_{r + a, 0} r^3 c_{\frakv}$) is also a $2$-cocycle of $\der(\bbC[v^{\pm 1}])$ with values in $\bbC c_{\frakv}$.
        \end{proposition}
            \begin{proof}
                It is clear from the construction of $\eta'$ that it is linear and skew-symmetric; the only non-trivial thing to prove is that $\eta'$ satisfies the Jacobi identity in the sense of definition \ref{def: twisted_semi_direct_products}. To this end, simply consider the following, for all $i, j, k \in \Z$:
                    $$
                        \begin{aligned}
                            & \eta'([d_i, d_j], d_k) + \eta'([d_k, d_i], d_j) + \eta'([d_j, d_k], d_i)
                            \\
                            = & (i - j) \eta'(d_{i + j}, d_k) + (k - i) \eta'(d_{k + i}, d_j) + (j - k) \eta'(d_{j + k}, d_i)
                            \\
                            = & \delta_{i + j + k, 0} \left( (i - j) (i + j) + (k - i) (k + i) + (j - k) (j + k) \right) c_{\frakv}
                            \\
                            = & 0
                        \end{aligned}
                    $$
            \end{proof}
        In conjunction with the following well-known fact about central extensions of $\der(\bbC[v^{\pm 1}])$, one can infer from proposition \ref{prop: auxiliary_virasoro_cocycles} that, precisely because we have that:
            $$\sigma_i|_{ \bigwedge^2 \frakv } = \eta + \eta'$$
        the domain restrictions $\sigma_1|_{ \bigwedge^2 \frakv }$ must be cohomologous to the $2$-cocycle $\eta$, which is known to be non-cohomologous to $0$. The toroidal $2$-cocycles $\sigma_i: \bigwedge^2 \toroidal \to \z_{[2]}$ are therefore non-cohomologous to $0$.
        \begin{lemma} \label{lemma: uniqueness_of_virasoro_cocycle}
            \cite[Proposition 1.3]{kac_raina_rozhkovskaya_bombay_lectures_on_highest_weight_modules_of_infinite_dimensional_lie_algebras} Let $\bbC c_{\frakv}$ be viewed as a trivial $\der(\bbC[v^{\pm 1}])$-module. Then:
                $$\dim_{\bbC} H^2_{\Lie}( \der(\bbC[v^{\pm 1}]), \bbC c_{\frakv} ) = 1$$
            meaning in particular, that $\eta$ and $\eta + \eta'$ are cohomologous to each other. As the former is not cohomologous to $0$ (as $\frakv$ is a non-trivial central extension), neither is the latter.
        \end{lemma}
        Our final conclusion is thus as follows:
        \begin{theorem}[Non-triviality of the toroidal $2$-cocycles $\sigma_1, \sigma_2$] \label{theorem: non_trivial_yangian_cocycles_examples}
            Let $i \in \{1, 2\}$ and $\sigma_i$ be as in example \ref{example: yangian_cocycles_(counter)_examples}.
        
            $\sigma_i|_{ \bigwedge^2 \frakv }$ is cohomologous to the Virasoro $2$-cocycle $\eta$, which is \textit{not} cohomologous to $0$. Therefore, $\sigma_i|_{ \bigwedge^2 \frakv }$ is \textit{not} cohomologous to $0$ as well, and consequently, $\sigma_i$ is \textit{not} cohomologous to $0$ as $2$-cocycles of $\toroidal$ with values in $\z_{[2]}$.
        \end{theorem}