\section{Classification of Yangian extended toroidal Lie algebras}
    \subsection{Yangian toroidal cocycles}
        Let:
            $$0 \to \p \to \fraky \xrightarrow[]{\pi} \d \to 0$$
        be a Lie algebra extension such that $\p$ is an $\d$-module, i.e. let $\fraky$ be a twisted semi-direct product $\p \rtimes^{\sigma} \d$ (where $\sigma: \bigwedge^2 \d \to \p$ is some $2$-cocycle). In proposition \ref{prop: twisted_semi_direct_product_criterion}, we have seen that any such cocycle arises as the difference:
            $$\sigma(D, D') := [\gamma(D), \gamma(D')]_{\fraky} - \gamma( [D, D']_{\d} )$$
        for arbitrary elements $D, D' \in \d$ and for some choice of linear section:
            $$\gamma: \d \to \fraky$$
        We therefore see that to give such a linear section $\gamma: \d \to \fraky$ is the same as to give a $2$-cocycle $\sigma: \bigwedge^2 \d \to \p$, i.e. the cocycle $\sigma$ measures how far the extension $(\fraky, \pi)$ is away from splitting, i.e. from being the semi-direct product $\p \rtimes \d$ (which is more-or-less the definition of $2$-cocycles of Lie algebras; cf. definition \ref{def: twisted_semi_direct_products}).

        Keeping the above in mind, let us return to the setting of definition \ref{def: yangian_extended_toroidal_lie_algebras}. An natural question to ask, given this definition, is as follows:
        \begin{question}
            Amongst the twisted semi-direct products $\fraky(\sigma) := \toroidal \rtimes^{\sigma} \d_{[2]}$, which ones are Yangian extended toroidal Lie algebras ? 
        \end{question}
        Since $\toroidal$ automatically embeds into any twited semi-direct product $\fraky(\sigma)$ as a Lie subalgebra, and since the underlying vector space of $\fraky(\sigma)$ is $\toroidal \oplus \d_{[2]}$ by definition, in order to answer this question, it shall suffice to give a criterion on the cocycle:
            $$\sigma: \bigwedge^2 \d_{[2]} \to \toroidal$$
        so that there would exist an \textit{invariant} and \textit{non-degenerate} symmetric bilinear form $(-, -)_{\sigma}$ on $\fraky(\sigma)$. Equivalently, one can give a criterion on the corresponding linear section:
            $$\gamma: \d_{[2]} \to \fraky(\sigma)$$
        which, as mentioned above, is such that:
            $$\sigma(D, D') = [\gamma(D), \gamma(D')]_{\fraky(\sigma)} - \gamma( [D, D']_{\d_{[2]}} )$$

        For convenience, let us fix the following terminologies.
        \begin{definition}[General extended toroidal Lie algebras] \label{def: general_extended_toroidal_lie_algebras}
            Any twisted semi-direct product:
                $$\fraky(\sigma) := \toroidal \rtimes^{\sigma} \d_{[2]}$$
            shall be called an \textbf{extended toroidal Lie algebra}.
        \end{definition}
        \begin{definition}[Yangian toroidal $2$-cocycles] \label{def: yangian_toroidal_cocycles}
            Any $2$-cocyle $\sigma: \bigwedge^2 \d_{[2]} \to \toroidal$ shall be referred to as a \textbf{toroidal $2$-cocycle}.
            
            Any toroidal $2$-cocycle $\sigma$ such that $\fraky(\sigma)$ is a Yangian extended toroidal Lie algebra (in the sense of definition \ref{def: yangian_extended_toroidal_lie_algebras}) shall be called a \textbf{Yangian toroidal $2$-cocycle}.
        \end{definition}
        \begin{remark}
            Since we now know that Yangian extended toroidal Lie algebras are necessarily isomorphic to some twisted semi-direct product $\fraky(\sigma)$ (cf. theorem \ref{theorem: yangian_extended_toroidal_lie_algebras_preliminary_version} and corollary \ref{coro: yangian_extended_toroidal_lie_algebras_are_twisted_semi_direct_products}), definitions \ref{def: general_extended_toroidal_lie_algebras} and \ref{def: yangian_toroidal_cocycles} as above make sense.
        \end{remark}
        \begin{remark}[Yangian toroidal $2$-cocycles are central]
            Thanks to proposition \ref{prop: lie_bracket_on_orthogonal_complement_of_toroidal_centre}, we know that the codomain of any Yangian toroidal $2$-cocycle necessarily lies inside $\z_{[2]}$. 
        \end{remark}

    \subsection{Which extended toroidal Lie algebras are Yangian ?}
        Fix a toroidal $2$-cocyle:
            $$\sigma: \bigwedge^2 \d_{[2]} \to \toroidal$$
        along with a \textit{non-degenerate} symmetric bilinear form:
            $$(-, -)_{\sigma}: \Sym^2_k( \fraky(\sigma) ) \to k$$
        such that:
        \begin{itemize}
            \item the restriction of $(-, -)_{\sigma}$ down to the vector subspace $\g_{[2]} \oplus \z_{[2]}$ coincides with $(-, -)_{\toroidal}$, and
            \item $(\z_{[2]}, \d_{[2]})_{\sigma} \not = 0$ and $(\g_{[2]}, \d_{[2]})_{\sigma} = 0$ and $(\d_{[2]}, \d_{[2]})_{\sigma} = 0$.
        \end{itemize}
        Our task now, as indicated above, is to find a criterion on $\sigma$ so that $(-, -)_{\sigma}$ would be invariant with respect to the Lie bracket on $\fraky(\sigma)$.
        
        Firstly, observe that, per theorem \ref{theorem: yangian_extended_toroidal_lie_algebras} and corollary \ref{coro: yangian_extended_toroidal_lie_algebras_are_twisted_semi_direct_products}, we can infer particularly that:
            $$(-, -)_{\sigma}$$
        for any $\sigma$, is invariant with respect to $[-, -]_0$, in the sense that:
            $$([X, Y]_0, Z)_{\sigma} = (X, [Y, Z]_0)_{\sigma}$$
        for any $X + D, Y + D', Z + D'' \in \toroidal \oplus \d_{[2]}$. Next, recall that by definition \ref{def: twisted_semi_direct_products} (see also: example \ref{example: lie_algebra_semi_direct_products}), the Lie bracket on the twisted semi-direct product $\fraky(\sigma)$, which henceforth will be denoted by $[-, -]_{\sigma}$, is given by:
            $$[-, -]_{\sigma} = [-, -]_0 + \sigma \circ \pi$$
        where $[-, -]_0$ denotes the Lie bracket on the semi-direct product $\fraky(0) \cong \toroidal \rtimes \d_{[2]}$ and $\pi: \fraky(\sigma) \to \d_{[2]}$ is the canonical projection. Then, consider the following:
            $$
                \begin{aligned}
                    ([X + D', Y + D'']_{\sigma}, Z + D'')_{\sigma} & = ([X + D, Y + D']_0 + \sigma(D, D'), Z)_{\sigma}
                    \\
                    & = ([X + D, Y + D']_0, Z + D'')_{\sigma} + (\sigma(D, D'), Z + D'')_{\sigma}
                    \\
                    & = (X + D, [Y + D', Z + D'']_0)_{\sigma} + (\sigma(D, D'), Z + D'')_{\sigma}
                \end{aligned}
            $$
        We see then, that it shall suffices to find conditions on $\sigma$ so that:
            $$(\sigma(D, D'), Z + D'')_{\sigma} = (X + D, \sigma(D', D''))_{\sigma}$$
        For any $\zeta \in \g_{[2]} \oplus \z_{[2]} \oplus \d_{[2]}$, let us write:
            $$\zeta := \zeta_{\g_{[2]}} + \zeta_{\z_{[2]}} + \zeta_{\d_{[2]}}$$
        for its decomposition into its $\g_{[2]}$, $\z_{[2]}$, and $\d_{[2]}$-summands. We see then, that the equation $(\sigma(D, D'), Z + D'')_{\sigma} = (X + D, \sigma(D', D''))_{\sigma}$ is equivalently to:
            $$(\sigma(D, D')_{\g_{[2]}}, Z_{\g_{[2]}})_{\sigma} + (\sigma(D, D')_{\z_{[2]}}, D'')_{\sigma} = (X_{\g_{[2]}}, \sigma(D', D'')_{\g_{[2]}})_{\sigma} + (D, \sigma(D', D'')_{\z_{[2]}})_{\sigma}$$
        As $X, Y, Z \in \toroidal$ were chosen arbitrarily, the problem can thus be rephrased as follows.
        \begin{question}
            What are the necessarily conditions on a given toroidal $2$-cocycle:
                $$\sigma: \bigwedge^2 \d_{[2]} \to \toroidal$$
            so that:
                $$(\sigma(D, D')_{\g_{[2]}}, yg)_{\sigma} + (\sigma(D, D')_{\z_{[2]}}, D'')_{\sigma} = (xf, \sigma(D', D'')_{\g_{[2]}})_{\sigma} + (D, \sigma(D', D'')_{\z_{[2]}})_{\sigma}$$
            for all $D, D', D'' \in \d_{[2]}$, and for all $x, y \in \g$ and all $f, g \in A$.
        \end{question}
        \begin{remark}
            We know that for a toroidal $2$-cocycle:
                $$\sigma: \bigwedge^2 \d_{[2]} \to \toroidal$$
            to be Yangian is $\z_{[2]}$, its codomain must be $\z_{[2]}$. As such, it shall suffice to find a condition that it must satisfy so that:
                $$(\sigma(D, D')_{\z_{[2]}}, D'')_{\sigma} = (D, \sigma(D', D'')_{\z_{[2]}})_{\sigma}$$
            in order to make sure that the bilinear form $(-, -)_{\sigma}$ would be invariant with respect to $[-, -]_{\sigma}$. Without nay loss of generality, we can take $D, D', D'' \in \d_{[2]}$ to be basis elements. Suppose also, that $\sigma(D, D'), \sigma(D', D'') \in \z_{[2]}$ can be written as a linear combination of the basis elements of $\z_{[2]}$ as follows:
                $$\sigma(D, D') := \sum_{(m, p) \in \Z^2} \lambda_{m, p}(D, D') K_{m, p} + \lambda_v(D, D') c_v + \lambda_t(D, D') c_t$$
                $$\sigma(D', D'') := \sum_{(m, p) \in \Z^2} \mu_{m, p}(D', D'') K_{m, p} + \mu_v(D', D'') c_v + \mu_t(D', D'') c_t$$
            with the coefficients depending on the choices of elements $D, D', D'' \in \d_{[2]}$. We then have that:
                $$
                    (\sigma(D, D')_{\z_{[2]}}, D'')_{\sigma} =
                    \begin{cases}
                        \text{$\lambda_{\alpha, \beta}(D, D')$ if $D'' = D_{\alpha, \beta}$, for any $(\alpha, \beta) \in \Z^2$}
                        \\
                        \text{$\lambda_v(D, D')$ if $D'' = D_v$}
                        \\
                        \text{$\lambda_t(D, D')$ if $D'' = D_t$}
                    \end{cases}
                $$
                $$
                    (D, \sigma(D', D'')_{\z_{[2]}})_{\sigma} =
                    \begin{cases}
                        \text{$\mu_{\e, \eta}(D', D'')$ if $D = D_{\e, \eta}$, for any $(\e, \eta) \in \Z^2$}
                        \\
                        \text{$\mu_v(D', D'')$ if $D = D_v$}
                        \\
                        \text{$\mu_t(D', D'')$ if $D = D_t$}
                    \end{cases}
                $$
        \end{remark}
        \begin{theorem}[A Yangian-ness criterion for toroidal $2$-cocycles] \label{theorem: yangian_criterion_for_toroidal_cocycles}
            \todo[inline]{I'm not sure how to phrase this as a condition on $\sigma$.}
        \end{theorem}
            \begin{proof}
                
            \end{proof}

        Now that we have a criterion for a given toroidal $2$-cocycle to be Yangian, let us apply it to some known toroidal $2$-cocycles. 
        \begin{example}[Some explicit toroidal $2$-cocycles] \label{example: non_uniqueness_of_yangian_extended_lie_algebras}
            From proposition \ref{prop: lie_bracket_on_orthogonal_complement_of_toroidal_centre}, we know that:
                $$[\d_{[2]}, \d_{[2]}]_{\extendedtoroidal} \subset \z_{[2]} \oplus \d_{[2]}$$
            we can obtain some $2$-cocyles $\bar{\sigma} \in H^2_{\Lie}(\d_{[2]}, \z_{[2]})$ by restricting elements $\sigma \in H^2_{\Lie}(\der(A), \z_{[2]})$, some of which are known.

            It pays to abstract the situation out to the $n$-variable case for a moment, mostly for us to make the point that the dimension of the vector space $H^2_{\Lie}(\der(\bbC[v_1^{\pm 1}, ..., v_n^{\pm 1}]), \bar{\Omega}^1_{\bbC[v_1^{\pm 1}, ..., v_n^{\pm 1}]/k})$ depends not on the number of variables. In \cite[p. 5, below Equation 1.3]{billig_energy_momentum_tensor}, it was noted that the cohomology group:
                $$H^2_{\Lie}(\der(\bbC[v_1^{\pm 1}, ..., v_n^{\pm 1}]), \bar{\Omega}^1_{\bbC[v_1^{\pm 1}, ..., v_n^{\pm 1}]/k})$$
            contains at least $2$-cocyles that we shall denote by:
                $$\sigma_1, \sigma_2$$
            These \say{twist} the Lie brackets (in the sense of proposition \ref{prop: twisted_semi_direct_product_criterion}):
                $$[v_1^{m_1} ... v_n^{m_n} v_p \del_{v_p}, v_1^{r_1} ... v_n^{r_n} v_q \del_{v_q}] \in \der(\bbC[v_1^{\pm 1}, ..., v_n^{\pm 1}])$$
            by the following formulae:
                $$\sigma_1(v_1^{m_1} ... v_n^{m_n} v_p \del_{v_p}, v_1^{r_1} ... v_n^{r_n} v_q \del_{v_q}) = r_p m_q \sum_{1 \leq i \leq n} r_i v_1^{m_1 + r_1} ... v_n^{m_n + r_n} v_i^{-1} \bar{d}(v_i)$$
                $$\sigma_2(v_1^{m_1} ... v_n^{m_n} v_p \del_{v_p}, v_1^{r_1} ... v_n^{r_n} v_q \del_{v_q}) = m_p r_q \sum_{1 \leq i \leq n} r_i v_1^{m_1 + r_1} ... v_n^{m_n + r_n} v_i^{-1} \bar{d}(v_i)$$
            given for every $1 \leq p, q \leq n$ and every $(m_1, ..., m_n), (r_1, ..., r_n) \in \Z^n$. 

            Now, back to the $2$-variable case. Here, we know how the basis elements $D_{r, s}, D_v, D_t$ of $\d_{[2]}$ are given in terms of $A$-multiples of the partial derivatives $\del_v, \del_t$ (cf. lemma \ref{lemma: derivation_action_on_multiloop_algebras}), so we can exploit the bilinearity of $2$-cocycles in order to see how $\sigma_1, \sigma_2$ act on elements of $\d_{[2]}$. Knowing that the aforementioned basis elements are given by:
                $$D_{r, s} = s v^{-r + 1} t^{-s - 1} \del_v - r v^{-r} t^{-s} \del_t$$
                $$D_v = -v t^{-1} \del_v$$
                $$D_t = -\del_t$$
            we see thus that:
                $$\sigma_a(D_t, -) = 0$$
                $$\sigma_a(D_{r, s}, -), \sigma_a(D_v, -) \not = 0$$
            as elements of $\Hom_{\bbC}(\d_{[2]}, \z_{[2]})$, where $a \in \{1, 2\}$; one proves this by looking at the powers of $v, t$ in the multiples of $\del_v, \del_t$ in the expressions for $D_{r, s}, D_v, D_t$. Subsequently, we can conclude that neither of the $2$-cocycles $\sigma_1, \sigma_2$ vanish entirely on the Lie subalgebra $\d_{[2]}$ of $\der(A)$, and hence:
                $$\sigma_1, \sigma_2 \in H^2_{\Lie}(\d_{[2]}, \z_{[2]})$$
            We delegate the relevant detailed computations to the proof of theorem \ref{theorem: non_uniqueness_of_yangian_extended_toroidal_lie_algebras} down below.  

            We note also, that it should not be expected that:
                $$H^2_{\Lie}(\der(A), \toroidal) \cong H^2_{\Lie}(\d_{[2]}, \toroidal)$$
            because in general, if $\a$ is a Lie algebra and $\a'$ is a Lie subalgebra thereof, and $M$ is an $\a$-module (and hence also an $\a'$-module), then should be no reason to expect that we would have:
                $$H^i_{\Lie}(\a, M) \cong H^i_{\Lie}(\a', M)$$
            without further hypotheses. 
        \end{example}
        \begin{theorem}[Non-uniqueness of Yangian extended toroidal Lie algebras] \label{theorem: non_uniqueness_of_yangian_extended_toroidal_lie_algebras}
            Let $\d_{[2]}$ be regarded as a Lie subalgebra of $\der(A)$, with the usual commutator bracket. There are \textit{at least} two isomorphism classes of Lie algebra extensions \textit{other than the semi-direct product}:
                $$0 \to \toroidal \to \extendedtoroidal \to \d_{[2]} \to 0$$
            corresponding to two non-zero isomorphism classes $2$-cocycles:
                $$\sigma_1, \sigma_2 \in H^2_{\Lie}(\d_{[2]}, \toroidal)$$
            defined as in remark \ref{remark: non_uniqueness_of_yangian_extended_lie_algebras}.
        \end{theorem}
            \begin{proof}
                As indicated in example \ref{example: non_uniqueness_of_yangian_extended_lie_algebras}, it only remains to prove that:
                    $$\sigma_a(D_t, -) = 0$$
                    $$\sigma_a(D_v, -), \sigma_a(D_{r, s}, -) \not = 0$$
                (where $a \in \{1, 2\}$) via explicit computations. For our own convenience, let us temporarily relabel the variables as:
                    $$v := v_1, t := v_2$$
                \begin{itemize}
                    \item Firstly, note that:
                        $$\sigma_1(\del_{v_2}, v_1^{r_1} v_2^{r_2} \del_{v_q}) = \sigma_1(v_1^0 v_2^0 \del_{v_2}, -) = r_1 \cdot 0 \cdot \sum_{1 \leq i \leq 2} (...) = 0$$
                        $$\sigma_2(\del_{v_2}, v_1^{r_1} v_2^{r_2} \del_{v_q}) = \sigma_2(v_1^0 v_2^0 \del_{v_2}, -) = 0 \cdot r_2 \cdot \sum_{1 \leq i \leq 2} (...) = 0$$
                    and hence we indeed have that:
                        $$\sigma_a(D_t, -) = \sigma_a(-\del_{v_2}, -) = -\sigma_a(\del_{v_2}, -) = 0$$
                    \item Secondly, to show that:
                        $$\sigma_a(D_{r, s}, -) \not = 0$$
                    it suffices to only show that:
                        $$\sigma_a(D_{r, s}, D_v) \not = 0$$
                    To this end, recall that:
                        $$D_{r, s} = s v_1^{-r + 1} v_2^{-s - 1} \del_{v_1} - r v_1^{-r} v_2^{-s} \del_{v_2}$$
                        $$D_v = -v_1 v_2^{-1} \del_{v_1}$$
                    which tells us that:
                        $$
                            \begin{aligned}
                                \sigma_1(D_{r, s}, D_v) & = ( s \cdot 1 \cdot (-r + 1) - r \cdot (-1) \cdot (-s) ) \sum_{1 \leq i \leq 2} r_i v_1^{(-r + 1) + 1} v_2^{(-s - 1) - 1} v_i^{-1} \bar{d}(v_i)
                                \\
                                & = (-2sr + 1) \sum_{1 \leq i \leq 2} r_i v_1^{-r} v_2^{-s - 2} v_i^{-1} \bar{d}(v_i)
                            \end{aligned}
                        $$
                    \item Lastly, to show that:
                        $$\sigma_a(D_v, -) \not = 0$$
                    simply note that:
                        $$\sigma_a(D_{r, s}, D_v) = -\sigma_a(D_v, D_{r, s})$$
                \end{itemize}
            \end{proof}