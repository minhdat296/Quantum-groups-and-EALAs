\section{Examples of Yangian extended toroidal Lie algebras}
    \subsection{Yangian toroidal cocycles}
        A natural question to ask, now that we know that all Yangian extended toroidal Lie algebras in the sense of definition \ref{def: yangian_extended_toroidal_lie_algebras} are twisted semi-direct products of the form:
            $$\toroidal \rtimes^{\sigma} \d_{[2]}$$
        is as follows.
        \begin{question}
            Amongst the twisted semi-direct products $\fraky(\sigma) := \toroidal \rtimes^{\sigma} \d_{[2]}$, which ones are Yangian extended toroidal Lie algebras ? 
        \end{question}
        Since $\toroidal$ automatically embeds into any twited semi-direct product $\fraky(\sigma)$ as a Lie subalgebra, and since the underlying vector space of $\fraky(\sigma)$ is $\toroidal \oplus \d_{[2]}$ by definition, in order to answer this question, it shall suffice to give a criterion on the cocycle:
            $$\sigma: \bigwedge^2 \d_{[2]} \to \toroidal$$
        so that there would exist an \textit{invariant} and \textit{non-degenerate} symmetric bilinear form $(-, -)_{\sigma}$ on $\fraky(\sigma)$.

        For convenience, let us fix the following terminologies.
        \begin{definition}[General extended toroidal Lie algebras] \label{def: general_extended_toroidal_lie_algebras}
            Any twisted semi-direct product:
                $$\fraky(\sigma) := \toroidal \rtimes^{\sigma} \d_{[2]}$$
            shall be called an \textbf{extended toroidal Lie algebra}.
        \end{definition}
        \begin{definition}[Yangian toroidal $2$-cocycles] \label{def: yangian_toroidal_cocycles}
            Any $2$-cocyle $\sigma: \bigwedge^2 \d_{[2]} \to \toroidal$ shall be referred to as a \textbf{toroidal $2$-cocycle}.
            
            Any toroidal $2$-cocycle $\sigma$ such that $\fraky(\sigma)$ is a Yangian extended toroidal Lie algebra (in the sense of definition \ref{def: yangian_extended_toroidal_lie_algebras}) shall be called a \textbf{Yangian toroidal $2$-cocycle}.
        \end{definition}
        \begin{remark}
            Since we now know that Yangian extended toroidal Lie algebras are necessarily isomorphic to some twisted semi-direct product $\fraky(\sigma)$ (cf. theorem \ref{theorem: yangian_extended_toroidal_lie_algebras_preliminary_version} and corollary \ref{coro: yangian_extended_toroidal_lie_algebras_are_twisted_semi_direct_products}), definitions \ref{def: general_extended_toroidal_lie_algebras} and \ref{def: yangian_toroidal_cocycles} as above make sense.
        \end{remark}
        \begin{remark}[Yangian toroidal $2$-cocycles are central]
            Thanks to proposition \ref{prop: lie_bracket_on_orthogonal_complement_of_toroidal_centre}, we know that the codomain of any Yangian toroidal $2$-cocycle necessarily lies inside $\z_{[2]}$. See corollary \ref{coro: lie_brackets_on_central_extensions} as well.
        \end{remark}

    \subsection{Which extended toroidal Lie algebras are Yangian ?}
        Fix a toroidal $2$-cocyle:
            $$\sigma: \bigwedge^2 \d_{[2]} \to \toroidal$$
        along with a \textit{non-degenerate} symmetric bilinear form:
            $$(-, -)_{\sigma}: \Sym^2_k( \fraky(\sigma) ) \to k$$
        such that:
        \begin{itemize}
            \item the restriction of $(-, -)_{\sigma}$ down to the vector subspace $\g_{[2]} \oplus \z_{[2]}$ coincides with $(-, -)_{\toroidal}$, and
            \item $(\z_{[2]}, \d_{[2]})_{\sigma} \not = 0$ and $(\g_{[2]}, \d_{[2]})_{\sigma} = 0$ and $(\d_{[2]}, \d_{[2]})_{\sigma} = 0$.
        \end{itemize}
        Our task now, as indicated above, is to find a criterion on $\sigma$ so that $(-, -)_{\sigma}$ would be invariant with respect to the Lie bracket on $\fraky(\sigma)$.
        
        Firstly, observe that, per theorem \ref{theorem: yangian_extended_toroidal_lie_algebras} and corollary \ref{coro: yangian_extended_toroidal_lie_algebras_are_twisted_semi_direct_products}, we can infer particularly that:
            $$(-, -)_{\sigma}$$
        for any $\sigma$, is invariant with respect to $[-, -]_0$, in the sense that:
            $$([X, Y]_0, Z)_{\sigma} = (X, [Y, Z]_0)_{\sigma}$$
        for any $X + D, Y + D', Z + D'' \in \toroidal \oplus \d_{[2]}$. Next, recall that by definition \ref{def: twisted_semi_direct_products} (see also: example \ref{example: lie_algebra_semi_direct_products}), the Lie bracket on the twisted semi-direct product $\fraky(\sigma)$, which henceforth will be denoted by $[-, -]_{\sigma}$, is given by:
            $$[-, -]_{\sigma} = [-, -]_0 + \sigma \circ \pi$$
        where $[-, -]_0$ denotes the Lie bracket on the semi-direct product $\fraky(0) \cong \toroidal \rtimes \d_{[2]}$ and $\pi: \fraky(\sigma) \to \d_{[2]}$ is the canonical projection. Then, consider the following:
            $$
                \begin{aligned}
                    ([X + D', Y + D'']_{\sigma}, Z + D'')_{\sigma} & = ([X + D, Y + D']_0 + \sigma(D, D'), Z)_{\sigma}
                    \\
                    & = ([X + D, Y + D']_0, Z + D'')_{\sigma} + (\sigma(D, D'), Z + D'')_{\sigma}
                    \\
                    & = (X + D, [Y + D', Z + D'']_0)_{\sigma} + (\sigma(D, D'), Z + D'')_{\sigma}
                \end{aligned}
            $$
        We see then, that it shall suffices to find conditions on $\sigma$ so that:
            $$(\sigma(D, D'), Z + D'')_{\sigma} = (X + D, \sigma(D', D''))_{\sigma}$$
        For any $\zeta \in \g_{[2]} \oplus \z_{[2]} \oplus \d_{[2]}$, let us write:
            $$\zeta := \zeta_{\g_{[2]}} + \zeta_{\z_{[2]}} + \zeta_{\d_{[2]}}$$
        for its decomposition into its $\g_{[2]}$, $\z_{[2]}$, and $\d_{[2]}$-summands. We see then, that the equation $(\sigma(D, D'), Z + D'')_{\sigma} = (X + D, \sigma(D', D''))_{\sigma}$ is equivalently to:
            $$(\sigma(D, D')_{\g_{[2]}}, Z_{\g_{[2]}})_{\sigma} + (\sigma(D, D')_{\z_{[2]}}, D'')_{\sigma} = (X_{\g_{[2]}}, \sigma(D', D'')_{\g_{[2]}})_{\sigma} + (D, \sigma(D', D'')_{\z_{[2]}})_{\sigma}$$
        As $X, Y, Z \in \toroidal$ were chosen arbitrarily, the problem can thus be rephrased as follows.
        \begin{question}
            What are the necessarily conditions on a given toroidal $2$-cocycle:
                $$\sigma: \bigwedge^2 \d_{[2]} \to \toroidal$$
            so that:
                $$(\sigma(D, D')_{\g_{[2]}}, yg)_{\sigma} + (\sigma(D, D')_{\z_{[2]}}, D'')_{\sigma} = (xf, \sigma(D', D'')_{\g_{[2]}})_{\sigma} + (D, \sigma(D', D'')_{\z_{[2]}})_{\sigma}$$
            for all $D, D', D'' \in \d_{[2]}$, and for all $x, y \in \g$ and all $f, g \in A$.
        \end{question}
        \begin{remark} \label{remark: yangian_criterion_for_toroidal_cocycles}
            We know that for a toroidal $2$-cocycle:
                $$\sigma: \bigwedge^2 \d_{[2]} \to \toroidal$$
            to be Yangian is $\z_{[2]}$, its codomain must be $\z_{[2]}$. As such, it shall suffice to find a condition that it must satisfy so that:
                $$(\sigma(D, D')_{\z_{[2]}}, D'')_{\sigma} = (D, \sigma(D', D'')_{\z_{[2]}})_{\sigma}$$
            in order to make sure that the bilinear form $(-, -)_{\sigma}$ would be invariant with respect to $[-, -]_{\sigma}$. Without nay loss of generality, we can take $D, D', D'' \in \d_{[2]}$ to be basis elements. Suppose also, that $\sigma(D, D'), \sigma(D', D'') \in \z_{[2]}$ can be written as a linear combination of the basis elements of $\z_{[2]}$ as follows:
                $$\sigma(D, D') := \sum_{(m, p) \in \Z^2} \lambda_{m, p}(D, D') K_{m, p} + \lambda_v(D, D') c_v + \lambda_t(D, D') c_t$$
                $$\sigma(D', D'') := \sum_{(m, p) \in \Z^2} \mu_{m, p}(D', D'') K_{m, p} + \mu_v(D', D'') c_v + \mu_t(D', D'') c_t$$
            with the coefficients depending on the choices of elements $D, D', D'' \in \d_{[2]}$. We then have that:
                $$
                    (\sigma(D, D')_{\z_{[2]}}, D'')_{\sigma} =
                    \begin{cases}
                        \text{$\lambda_{\alpha, \beta}(D, D')$ if $D'' = D_{\alpha, \beta}$, for any $(\alpha, \beta) \in \Z^2$}
                        \\
                        \text{$\lambda_v(D, D')$ if $D'' = D_v$}
                        \\
                        \text{$\lambda_t(D, D')$ if $D'' = D_t$}
                    \end{cases}
                $$
                $$
                    (D, \sigma(D', D'')_{\z_{[2]}})_{\sigma} =
                    \begin{cases}
                        \text{$\mu_{\e, \eta}(D', D'')$ if $D = D_{\e, \eta}$, for any $(\e, \eta) \in \Z^2$}
                        \\
                        \text{$\mu_v(D', D'')$ if $D = D_v$}
                        \\
                        \text{$\mu_t(D', D'')$ if $D = D_t$}
                    \end{cases}
                $$
            This means that in order to see if $(-, -)_{\sigma}$ is an invariant bilinear form, it shall suffice to compute:
                $$\sigma(D, D'), \sigma(D', D'')$$
            as linear combinations of the basis elements of $\z_{[2]}$, record the coefficients as entries of two $4 \x 3$ matrices, namely:
                $$
                    \begin{pmatrix}
                        \lambda_{\alpha, \beta}(D_{r, s}, D_{a, b}) & \lambda_v(D_{r, s}, D_{a, b}) & \lambda_t(D_{r, s}, D_{a, b})
                        \\
                        \lambda_{\alpha, \beta}(D_{r, s}, D_v) & \lambda_v(D_{r, s}, D_v) & \lambda_t(D_{r, s}, D_v)
                        \\
                        \lambda_{\alpha, \beta}(D_{r, s}, D_t) & \lambda_v(D_{r, s}, D_t) & \lambda_t(D_{r, s}, D_t)
                        \\
                        \lambda_{\alpha, \beta}(D_v, D_t) & \lambda_v(D_v, D_t) & \lambda_t(D_v, D_t)
                    \end{pmatrix}
                $$
                $$
                    \begin{pmatrix}
                       \mu_{r, s}(D_{a, b}, D_{\alpha, \beta}) & \mu_v(D_{a, b}, D_{\alpha, \beta}) & \mu_t(D_{a, b}, D_{\alpha, \beta})
                        \\
                       \mu_{r, s}(D_{a, b}, D_v) & \mu_v(D_{a, b}, D_v) & \mu_t(D_{a, b}, D_v)
                        \\
                       \mu_{r, s}(D_{a, b}, D_t) & \mu_v(D_{a, b}, D_t) & \mu_t(D_{a, b}, D_t)
                        \\
                       \mu_{r, s}(D_v, D_t) & \mu_v(D_v, D_t) & \mu_t(D_v, D_t)
                    \end{pmatrix}
                $$
            and then compare said matrices. If these matrices are equal, then we will have that:
                $$(\sigma(D, D')_{\z_{[2]}}, D'')_{\sigma} = (D, \sigma(D', D'')_{\z_{[2]}})_{\sigma}$$
            i.e. $\sigma$ will be Yangian.
        \end{remark}
        By putting everything together, one obtains the following result characteristing Yangian $2$-cocycles amongst all the toroidal $2$-cocycles. Let us also keep in mind the fact that Lie $2$-cocycles are necessarily central (in the sense of corollary \ref{coro: 2_cocycles_are_central}), which in this case means that the codomains of toroidal $2$-cocycles lie inside $\z_{[2]} := \z(\toroidal)$, not just inside $\toroidal$.
        \begin{theorem}[A Yangian-ness criterion for toroidal $2$-cocycles] \label{theorem: yangian_criterion_for_toroidal_cocycles}
            A toroidal $2$-cocycle:
                $$\sigma: \bigwedge^2 \d_{[2]} \to \z_{[2]}$$
            is Yangian if and only if:
                $$(\sigma(D, D'), D'')_{\sigma} = (D, \sigma(D', D''))_{\sigma}$$
            for all $D, D', D'' \in \d_{[2]}$, which can be taken to be basis elements without loss of generality. 
        \end{theorem}

        Now that we have a criterion for a given toroidal $2$-cocycle to be Yangian, let us apply it to some known toroidal $2$-cocycles from \cite{billig_energy_momentum_tensor} to check whether or not they are Yangian. 
        \begin{example} \label{example: yangian_cocycles_(counter)_examples}
            From proposition \ref{prop: lie_bracket_on_orthogonal_complement_of_toroidal_centre}, we know that:
                $$[\d_{[2]}, \d_{[2]}]_{\extendedtoroidal} \subset \z_{[2]} \oplus \d_{[2]}$$
            we can obtain some toroidal $2$-cocyles:
                $$\sigma: \bigwedge^2 \d_{[2]} \to \z_{[2]}$$
            by restricting $2$-cocycles of $\der(A)$ with values in $\z_{[2]}$.

            It pays to abstract the situation out to the $n$-variable case momentarily. In \cite[p. 5, below Equation 1.3]{billig_energy_momentum_tensor}, it was noted that there are at least $2$-cocyles that we shall denote by:
                $$\sigma_1, \sigma_2: \bigwedge^2 \der(\bbC[v_1^{\pm 1}, ..., v_n^{\pm 1}]) \to \bar{\Omega}^1_{\bbC[v_1^{\pm 1}, ..., v_n^{\pm 1}]/\bbC}$$
            which are given in terms of the basis:
                $$\{ v_1^{m_1} ... v_n^{m_n} \cdot v_p \del_{v_p} \}_{(m_1, ..., m_n, p) \in \Z^n \x \Z}$$
            of $\der(\bbC[v_1^{\pm 1}, ..., v_n^{\pm 1}])$ by:
                $$\sigma_1(v_1^{m_1} ... v_n^{m_n} \cdot v_a \del_{v_a}, v_1^{r_1} ... v_n^{r_n} \cdot v_b \del_{v_b}) := r_a m_b \cdot v_1^{m_1} ... v_n^{m_n} \bar{d}( v_1^{r_1} ... v_n^{r_n} )$$
                $$\sigma_2(v_1^{m_1} ... v_n^{m_n} \cdot v_a \del_{v_a}, v_1^{r_1} ... v_n^{r_n} \cdot v_b \del_{v_b}) := r_b m_a \cdot v_1^{m_1} ... v_n^{m_n} \bar{d}( v_1^{r_1} ... v_n^{r_n} )$$
            given for every $1 \leq a, b \leq n$ and every $(m_1, ..., m_n), (r_1, ..., r_n) \in \Z^n$. 

            Now, back to the $2$-variable case, i.e. $n = 2$, where we have:
                $$v_1 := v, v_2 := t$$
            In this setting, we know how the basis elements $D_{r, s}, D_v, D_t$ of $\d_{[2]}$ are given in terms of $A$-multiples of the partial derivatives $\del_v, \del_t$ (cf. lemma \ref{lemma: derivation_action_on_multiloop_algebras}), so we can exploit the bilinearity of $2$-cocycles in order to see how $\sigma_1, \sigma_2$ act on elements of $\d_{[2]}$. Recall from lemma \ref{lemma: derivation_action_on_multiloop_algebras} that, in terms of the partial derivatives $\del_v, \del_t$, the basis elements of $\d_{[2]}$ are given by:
                $$\forall (r, s) \in \Z^2: D_{r, s} = s v^{-r + 1} t^{-s - 1} \del_v - r v^{-r} t^{-s} \del_t$$
                $$D_v = -v t^{-1} \del_v$$
                $$D_t = -\del_t$$
            Knowing this allows us to perform the following computations, where $i \in \{1, 2\}$:
            \begin{enumerate}
                \item 
                    $$
                        \begin{aligned}
                            & \sigma_i(D_{r, s}, D_{a, b})
                            \\
                            = & \sigma_i( s v^{-r + 1} t^{-s - 1} \del_v - r v^{-r} t^{-s} \del_t, b v^{-a + 1} t^{-b - 1} \del_v - a v^{-a} t^{-b} \del_t )
                            \\
                            = & \sigma_i( s v^{-r} t^{-s - 1} \cdot v\del_v - r v^{-r} t^{-s - 1} \cdot t \del_t, b v^{-a} t^{-b - 1} \cdot v\del_v - a v^{-a} t^{-b - 1} \cdot t \del_t )
                            \\
                            = & s \sigma_i( v^{-r} t^{-s - 1} \cdot v\del_v, b v^{-a} t^{-b - 1} \cdot v\del_v - a v^{-a} t^{-b - 1} \cdot t \del_t ) - r \sigma_i( v^{-r} t^{-s - 1} \cdot t \del_t, b v^{-a} t^{-b - 1} \cdot v\del_v - a v^{-a} t^{-b - 1} \cdot t \del_t )
                            \\
                            = &
                            \begin{aligned}
                                & s b \cdot \sigma_i( v^{-r} t^{-s - 1} \cdot v\del_v, v^{-a} t^{-b - 1} \cdot v\del_v )
                                \\
                                - & s a \cdot \sigma_i( v^{-r} t^{-s - 1} \cdot v\del_v, v^{-a} t^{-b - 1} \cdot t \del_t )
                                \\
                                - & r b \cdot \sigma_i( v^{-r} t^{-s - 1} \cdot t \del_t, v^{-a} t^{-b - 1} \cdot v\del_v )
                                \\
                                + & r a \cdot \sigma_i( v^{-r} t^{-s - 1} \cdot t \del_t, v^{-a} t^{-b - 1} \cdot t \del_t )
                            \end{aligned}
                            \\
                            = & N_i(r, s, a, b) v^{-r} t^{-s - 1} \bar{d}( v^{-a} t^{-b - 1} )
                        \end{aligned}
                    $$
                where:
                    $$
                        \begin{aligned}
                            N_i(r, s, a, b) & = 
                            sbra
                            - sa \left( \delta_{i, 1} a(s + 1) + \delta_{i, 2} (b + 1) r \right) 
                            - rb \left( \delta_{i, 1} (b + 1) r + \delta_{i, 2} a (s + 1) \right)
                            + r a (s + 1) (b + 1)
                            \\
                            & = 
                            \begin{cases}
                                \text{$
                                    sbra
                                    - s a^2 (s + 1) 
                                    - r^2 b (b + 1)
                                    + r a (s + 1) (b + 1)
                                $if $i = 1$}
                                \\
                                \text{$
                                    sbra
                                    - sa (b + 1) r
                                    - rb a (s + 1)
                                    + r a (s + 1) (b + 1)
                                $ if $i = 2$}
                            \end{cases}
                            \\
                            & = 
                            \begin{cases}
                                \text{$
                                    sbra
                                    - ( (sa)^2 + s a^2 ) 
                                    - ( (rb)^2 + r^2 b ) 
                                    + rasb + rsa + rab + ra
                                $ if $i = 1$}
                                \\
                                \text{$
                                    sbra
                                    - (sabr + sar)
                                    - (rbas + rba)
                                    + rasb + rsa + rab + ra
                                $ if $i = 2$}
                            \end{cases}
                            \\
                            & = 
                            \begin{cases}
                                \text{$2 rsab - ( (sa)^2 + s a^2 ) - ( (rb)^2 + r^2 b ) + rsa + rab + ra$ if $i = 1$}
                                \\
                                \text{$ra$ if $i = 2$}
                            \end{cases}
                        \end{aligned}
                    $$
                
                Now, recall from example \ref{example: toroidal_lie_algebras_centres} that any element:
                    $$v^n t^q \bar{d}(v^m t^p) \in \z_{[2]}$$
                can be written in terms of the basis elements of $\z_{[2]}$ in the following manner:
                    $$v^n t^q \bar{d}(v^m t^p) = \delta_{(m, p) + (n, q), (0, 0)} ( n c_v + q c_t ) + (np - mq) K_{m + n, p + q}$$
                Using this, we shall be able to conclude that:
                    $$\sigma_i(D_{r, s}, D_{a, b}) = N_i(r, s, a, b) \left( -\delta_{(r, s), -(a, b)} (r c_v + (s + 1) c_t) + ( r(b + 1) - a(s + 1) )K_{-r - a, -s - b - 2} \right)$$
                \item
                    $$
                        \begin{aligned}
                            & \sigma_i(D_{r, s}, D_v)
                            \\
                            = & \sigma_i( s v^{-r + 1} t^{-s - 1} \del_v - r v^{-r} t^{-s} \del_t, -v t^{-1} \del_v )
                            \\
                            = & \sigma_i( s v^{-r} t^{-s - 1} \cdot v \del_v - r v^{-r} t^{-s - 1} t \del_t, -t^{-1} \cdot v \del_v )
                            \\
                            = & s \sigma_i( v^{-r} t^{-s - 1} \cdot v \del_v, t^{-1} \cdot v \del_v ) - r\sigma_i( v^{-r} t^{-s - 1} \cdot t \del_t, t^{-1} \cdot v \del_v )
                            \\
                            = & 0 - r \cdot \delta_{i, 1}r v^{-r} t^{-s - 1} \bar{d}(t^{-1})
                            \\
                            = & \delta_{i, 1} r^2 v^{-r} t^{-s - 1} \bar{d}(t^{-1})
                        \end{aligned}
                    $$
                Immediately, we see that:
                    $$\sigma_2(D_{r, s}, D_v) = 0$$
                for all $(r, s) \in \Z^2$, so from now on we will only be concerned with $\sigma_1(D_{r, s}, D_v)$, which by now we know to be given by:
                    $$\sigma_1(D_{r, s}, D_v) = r^2 v^{-r} t^{-s - 1} \bar{d}(t^{-1})$$
                Now, recall from example \ref{example: toroidal_lie_algebras_centres} that any element:
                    $$v^n t^q \bar{d}(v^m t^p) \in \z_{[2]}$$
                can be written in terms of the basis elements of $\z_{[2]}$ in the following manner:
                    $$v^n t^q \bar{d}(v^m t^p) = \delta_{(m, p) + (n, q), (0, 0)} ( n c_v + q c_t ) + (np - mq) K_{m + n, p + q}$$
                Using this, we shall get that:
                    $$
                        \begin{aligned}
                            & \sigma_1(D_{r, s}, D_v)
                            \\
                            = & r^2 \left( -\delta_{(r, s), (0, -2)} ( r c_v + (s + 3) c_t ) - r K_{-r, -s - 2} \right)
                            \\
                            = &
                            \begin{cases}
                                \text{$0$ if $(r, s) \in \{0\} \x \Z$}
                                \\
                                \text{$r^3 K_{-r, -s - 2}$ if $(r, s) \in (\Z \setminus \{0\}) \x \Z$}
                            \end{cases}
                            \\
                            = & r^3 K_{-r, -s - 2}
                        \end{aligned}
                    $$
                \item
                    $$
                        \begin{aligned}
                            & \sigma_i(D_{r, s}, D_t)
                            \\
                            = & \sigma_i( s v^{-r + 1} t^{-s - 1} \del_v - r v^{-r} t^{-s} \del_t, -\del_t )
                            \\
                            = & \sigma_i( s v^{-r} t^{-s - 1} \cdot v \del_v - r v^{-r} t^{-s - 1} \cdot t \del_t, -t^{-1} \cdot t\del_t )
                            \\
                            = & s \sigma_i( v^{-r} t^{-s - 1} \cdot v \del_v, t^{-1} \cdot t\del_t ) - r\sigma_i( v^{-r} t^{-s- 1} \cdot t \del_t, t^{-1} \cdot t\del_t )
                            \\
                            = & -s \cdot \left( \delta_{i, 1} \cdot 0 + \delta_{i, 2} r v^{-r} t^{-s - 1} \bar{d}(t^{-1}) \right) + r \cdot s v^{-r} t^{-s - 1} \bar{d}(t^{-1})
                            \\
                            = & (-\delta_{i, 2} + 1) rs v^{-r} t^{-s - 1} \bar{d}(t^{-1})
                        \end{aligned}
                    $$
                Immediately, we see that:
                    $$\sigma_2(D_{r, s}, D_t) = 0$$
                for all $(r, s) \in \Z^2$, so from now on we will only be concerned with $\sigma_1(D_{r, s}, D_t)$, which by now we know to be given by:
                    $$\sigma_1(D_{r, s}, D_t) = rs v^{-r} t^{-s - 1} \bar{d}(t^{-1})$$
                Now, recall from example \ref{example: toroidal_lie_algebras_centres} that any element:
                    $$v^n t^q \bar{d}(v^m t^p) \in \z_{[2]}$$
                can be written in terms of the basis elements of $\z_{[2]}$ in the following manner:
                    $$v^n t^q \bar{d}(v^m t^p) = \delta_{(m, p) + (n, q), (0, 0)} ( n c_v + q c_t ) + (np - mq) K_{m + n, p + q}$$
                Using this, we shall get that:
                    $$
                        \begin{aligned}
                            & \sigma_1(D_{r, s}, D_t)
                            \\
                            = & rs \left( -\delta_{(r, s), (0, -2)} ( r c_v + (s + 1) c_t ) + r K_{-r, -s - 2} \right)
                            \\
                            = & 
                            \begin{cases}
                                \text{$0$ if $(r, s) \in \{0\} \x \Z$}
                                \\
                                \text{$r^2s K_{-r, -s - 2}$ if $(r, s) \in (\Z \setminus \{0\}) \x \Z$}
                            \end{cases}
                            \\
                            = & r^2s K_{-r, -s - 2}
                        \end{aligned}
                    $$
                \item
                    $$
                        \begin{aligned}
                            & \sigma_i(D_v, D_t)
                            \\
                            = & \sigma_i(-v t^{-1} \del_v, -\del_t)
                            \\
                            = & \sigma_i(t^{-1} \cdot v \del_v, t^{-1} t \del_t)
                            \\
                            = & 0
                        \end{aligned}
                    $$
            \end{enumerate}

            \todo[inline]{Showed that $\sigma_1$ is Yangian while $\sigma_2$ is not.}
            One can now use the criterion given in theorem \ref{theorem: yangian_criterion_for_toroidal_cocycles} to verify whether or not the cocycles $\sigma_1, \sigma_2$ are Yangian in the sense of definition \ref{def: yangian_toroidal_cocycles}.
            \begin{enumerate}
                \item Firstly, using the fact that:
                    $$\sigma_i(D_{r, s}, D_{a, b}) = N_i(r, s, a, b) \left( -\delta_{(r, s), -(a, b)} (r c_v + (s + 1) c_t) + ( r(b + 1) - a(s + 1) )K_{-r - a, -s - b - 2} \right)$$
                we shall get that:
                    $$
                        \left( \sigma_i(D_{r, s}, D_{a, b}), D \right)_{\extendedtoroidal} =
                        \begin{cases}
                            \text{$N_i(r, s, a, b) ( r(b + 1) - a(s + 1) ) \delta_{(-r - a, -s - b - 2), (\alpha, \beta)}$ if $D = D_{\alpha, \beta}$}
                            \\
                            \text{$-N_i(r, s, a, b) \delta_{(r, s), -(a, b)} r$ if $D = D_v$}
                            \\
                            \text{$-N_i(r, s, a, b) \delta_{(r, s), -(a, b)} (s + 1)$ if $D = D_t$}
                        \end{cases}
                    $$
                At the same time, using the fact that:
                    $$\sigma_i(D_{a, b}, D_v) = \delta_{i, 1} a^3 K_{-a, -b - 2}$$
                    $$\sigma_i(D_{a, b}, D_t) = \delta_{i, 1} a^2b K_{-a, -b - 2}$$
                we have that:
                    $$
                        \begin{aligned}
                            \left( D_{r, s}, \sigma_i(D_{a, b}, D) \right)_{\extendedtoroidal} =
                            \begin{cases}
                                \text{$N_i(r, s, \alpha, \beta) ( r(\beta + 1) - \alpha(s + 1) ) \delta_{(r, s), (-a - \alpha, -b - \beta - 2)}$ if $D = D_{\alpha, \beta}$}
                                \\
                                \text{$\delta_{i, 1} a^3 \delta_{(r, s), (-a, -b - 2)}$ if $D = D_v$}
                                \\
                                \text{$\delta_{i, 1} a^2 b \delta_{(r, s), (-a, -b - 2)}$ if $D = D_t$}
                            \end{cases}
                        \end{aligned}
                    $$
                We can thus conclude immediately that $\sigma_2$ is \textit{not} invariant, as:
                    $$\left( \sigma_2(D_{r, s}, D_{a, b}), D \right)_{\extendedtoroidal} \not = \left( D_{r, s}, \sigma_2(D_{a, b}, D) \right)_{\extendedtoroidal}$$
                when $D \in \{D_v, D_t\}$. As such, let us focus on $\sigma_1$ from now on, for which we now have:
                    $$\left( \sigma_1(D_{r, s}, D_{a, b}), D \right)_{\extendedtoroidal} \not = \left( D_{r, s}, \sigma_1(D_{a, b}, D) \right)_{\extendedtoroidal}$$
                for all $D \in \d_{[2]}$.
                \item Secondly, using the fact that:
                    $$\sigma_1(D_{r, s}, D_v) = r^3 K_{-r, -s - 2}$$
                we shall get that:
                    $$
                        \left( \sigma_1(D_{r, s}, D_v), D \right)_{\extendedtoroidal} =
                        \begin{cases}
                            \text{$r^3 \delta_{(-r, -s - 2), (\alpha, \beta)}$ if $D = D_{\alpha, \beta}$}
                            \\
                            \text{$0$ if $D = D_v$}
                            \\
                            \text{$0$ if $D = D_t$}
                        \end{cases}
                    $$
                At the same time, knowing that:
                    $$\sigma_1(D_v, D_t) = 0$$
                we see that:
                    $$
                        \begin{aligned}
                            \left( D_{r, s}, \sigma_1(D_v, D) \right)_{\extendedtoroidal} =
                            \begin{cases}
                                \text{$-\alpha^3 \delta_{(r, s), (-\alpha, -\beta - 2)}$ if $D = D_{\alpha, \beta}$}
                                \\
                                \text{$0$ if $D = D_v$}
                                \\
                                \text{$0$ if $D = D_t$}
                            \end{cases}
                        \end{aligned}
                    $$
                We thus have:
                    $$\left( \sigma_1(D_{r, s}, D_v), D \right)_{\extendedtoroidal} = \left( D_{r, s}, \sigma_1(D_v, D) \right)_{\extendedtoroidal}$$
                for all $D \in \d_{[2]}$.
                \item Next, by using the fact that:
                    $$\sigma_1(D_{r, s}, D_t) = r^2 s K_{-r, -s - 2}$$
                we shall get that:
                    $$
                        \left( \sigma_1(D_{r, s}, D_t), D \right)_{\extendedtoroidal} =
                        \begin{cases}
                            \text{$r^2 s \delta_{(-r, -s - 2), (\alpha, \beta)}$ if $D = D_{\alpha, \beta}$}
                            \\
                            \text{$0$ if $D = D_v$}
                            \\
                            \text{$0$ if $D = D_t$}
                        \end{cases}
                    $$
                At the same time, we have that:
                    $$
                        \begin{aligned}
                            \left( D_{r, s}, \sigma_1(D_t, D) \right)_{\extendedtoroidal} =
                            \begin{cases}
                                \text{$-\alpha^2 \beta \delta_{(r, s), (-\alpha, -\beta - 2)}$ if $D = D_{\alpha, \beta}$}
                                \\
                                \text{$0$ if $D = D_v$}
                                \\
                                \text{$0$ if $D = D_t$}
                            \end{cases}
                        \end{aligned}
                    $$
                By combining these two observations, one is able to conclude furthermore that:
                    $$\left( \sigma_1(D_{r, s}, D_t), D \right)_{\extendedtoroidal} = \left( D_{r, s}, \sigma_1(D_t, D) \right)_{\extendedtoroidal}$$
                for all $D \in \d_{[2]}$.
                \item Lastly, since:
                    $$\sigma_1(D_v, D_t) = 0$$
                we automatically have that:
                    $$( \sigma_1(D_v, D_t), D )_{\extendedtoroidal} = ( D_v, \sigma_1(D_t, D) )_{\extendedtoroidal}$$
                for all $D \in \d_{[2]}$.
            \end{enumerate}
            We have therefore shown that $\sigma_1$ is Yangian in the sense of definition \ref{def: yangian_toroidal_cocycles}. 
        \end{example}
        Whether or not these cocycles might be cohomologous to $0$ (cf. definition \ref{def: lie_algebra_cohomology}) - and hence whether or not they might give rise to extensions that are isomorphic to the semi-direct product $\toroidal \rtimes \d_{[2]}$ - is a much subtler issue. One way to tackle this problem is to \textit{firstly} check whether or not their restrictions to a particular Lie subalgebra of $\extendedtoroidal$ is cohomologous to $0$. We remark right away that simply checking that these restrictions are non-cohomologous to $0$ is \textit{not} sufficient for concluding that $\sigma_1$ and $\sigma_2$ are non-zero elements of $H^2_{\Lie}(\d_{[2]}, \z_{[2]})$.

        Recall from lemma \ref{lemma: derivation_action_on_multiloop_algebras} that:
            $$\forall (r, s) \in \Z^2: D_{r, s} = s v^{-r + 1} t^{-s - 1} \del_v - r v^{-r} t^{-s} \del_t$$
            $$D_v = -v t^{-1} \del_v$$
            $$D_t = -\del_t$$
        and from lemma \ref{lemma: explicit_commutators_between_basis_elements_of_toroidal_central_orthogonal_complement} that the commutation relations that these basis elements of $\d_{[2]}$ satisfy are:
            $$[D_v, D_t] = 0$$
            $$[D_v, D_{r, s}] = r D_{r, s + 1}$$
            $$[D_t, D_{r, s}] = D_{r, s + 1}$$
            $$[D_{a, b}, D_{r, s}] = (br - sa) D_{a + r, b + s + 1}$$
        (given for all $(r, s), (a, b) \in \Z^2$). With these information in mind, one sees that the following vector subspace of $\d_{[2]}$:
            $$\frakw := \bigoplus_{r \in \Z} \bbC D_{r, -1}$$
        is actually a Lie subalgebra, as the basis elements satisfy the following commutators:
            $$[D_{a, -1}, D_{r, -1}] = (a - r) D_{a + r, -1}$$
        given for all $a, r \in \Z$; note also that we have $D_v, D_t \not \in \frakw$ because:
            $$[D_v, D_{r, -1}] = r D_{r, 0} \not \in \frakw$$
            $$[D_t, D_{r, -1}] = D_{r, 0} \not \in \frakw$$    
        for all $r \in \Z$. Interestingly, these are precisely the commutation relations satisfied by the elements of the following basis of the Lie algebra $\der(\bbC[v^{\pm 1}])$:
            $$\{ d_r := -v^r D_{\aff} \}_{r \in \Z}$$
        (where $D_{\aff} := v \frac{d}{dv}$ is the \say{untwisted affine Kac-Moody derivation} as in subsection \ref{subsection: a_fixed_untwisted_affine_kac_moody_algebra}) and in light of this, we make the following observation:
        \begin{lemma}[A copy of the Witt algebra inside $\d_{[2]}$] \label{lemma: a_copy_of_the_witt_algebra_inside_the_lie_algebra_of_yangian_div_zero_vector_fields}
            There is an isomorphism of Lie algebras:
                $$\der(\bbC[v^{\pm 1}]) \xrightarrow[]{\cong} \frakw$$
            given by:
                $$d_r \mapsto D_{r, -1}$$
            This identifies a copy of $\der(\bbC[v^{\pm 1}])$ inside $\d_{[2]}$ as a Lie subalgebra. 
        \end{lemma}

        The Lie algebra $\der(\bbC[v^{\pm 1}])$ - commonly called the \textbf{Witt algebra} - is known to possess a \textit{non-trivial} UCE:
            $$\frakv := \der(\bbC[v^{\pm 1}]) \oplus^{\eta} \bbC c_{\frakv}$$
        called the \textbf{Virasoro algebra} (for more details, see subsection \ref{subsection: virasoro_algebra}), whose corresponding $2$-cocycle, which will often be referred to as the \textbf{Virasoro $2$-cocycle}:
            $$\eta: \bigwedge^2 \der(\bbC[v^{\pm 1}]) \to \bbC c_{\frakv}$$
        is given by:
            $$\eta(d_r, d_a) := \delta_{r + a, 0} (r^3 - r) c_{\frakv}$$
        for all $r, a \in \Z$; in particular, this means this \textit{$\eta$ is non-cohomologous to $0$}. The $2$-cocycle $\eta$ is cohomologically unique (in the sense that any $2$-cocycle $\eta': \bigwedge^2 \der(\bbC[v^{\pm 1}]) \to \bbC c_{\frakv}$ is cohomologous to $\eta$ itself; lemma \ref{lemma: H^2_of_witt_algebra}), and so should either $\sigma_1$ or $\sigma_2$ become cohomologous to $\eta$ after being restricted down to $\bigwedge^2 \frakw$, they would have to be non-cohomologous to $0$. We claim that this is indeed true.

        The answer to the following question turns out to be negative, but in answering it, we will have gained some insight into how we might show that $\sigma_1$ and $\sigma_2$ are actually cohomologous to $\eta$ after having their domains restricted to $\bigwedge^2 \frakw$.
        \begin{question}
            Is it true that:
                $$\sigma_i|_{ \bigwedge^2 \frakw } = \eta$$
            for either $i = 1$ or $i = 2$ ?
        \end{question}
        Using the computations in example \ref{example: yangian_cocycles_(counter)_examples}, we see that:
            $$\sigma_i(D_{r, -1}, D_{a, -1}) = -N_i(r, -1, a, -1) \delta_{r + a, 0} r c_v$$
        where:
            $$
                N_i(r, -1, a, -1) =
                \begin{cases}
                    \text{$2 ra - ra - ra + ra$ if $i = 1$}
                    \\
                    \text{$ra$ if $i = 2$}
                \end{cases}
                = ra
            $$
        and hence, more succinctly, we have that:
            $$\sigma_i(D_{r, -1}, D_{a, -1}) = -\delta_{r + a, 0} r^3 c_v$$
        regardless of whether $i = 1$ or $i = 2$, and for all $r, a \in \Z$. If it was true that:
            $$\sigma_i(D_{r, -1}, D_{a, -1}) = \eta(d_r, d_a)$$
        then we must have that:
            $$-\delta_{r + a, 0} r^3 c_v = \delta_{r + a, 0} (r^3 - r) c_{\frakv}$$
        which implies that:
            $$-r^3 c_v = (r^3 - r) c_{\frakv}$$
        for all $r \in \Z$. But this is certainly not true for all $r \in \Z$: e.g. if $r = 1$ then we will get that:
            $$-c_v = -v^{-1} \bar{d}v = 0$$
        which is absurd! As such, \textit{neither} of the $2$-cocycles $\sigma_1$ and $\sigma_2$ coincide with $\eta$ when restricted down to $\bigwedge^2 \frakv$. In particular, this means that it is still inconclusive as to whether or not the toroidal $2$-cocycles $\sigma_1, \sigma_2$ are cohomologous to $0$. However, this does not necessarily imply that $\eta$ and $\sigma_1, \sigma_2$ are \textit{not} cohomologous. 

        \begin{proposition}[A Virasoro $2$-coboundary] \label{prop: a_virasoro_coboundary}
            Let:
                $$\eta': \bigwedge^2 \der(\bbC[v^{\pm 1}]) \to \bbC c_{\frakv}$$
            be the function given by:
                $$\eta'(d_r, d_a) := \delta_{r + a, 0} r c_{\frakv}$$
            This is a $2$-cocycle of $\der(\bbC[v^{\pm 1}])$ with values in $\bbC c_{\frakv}$. From this, one sees that:
                $$\eta + \eta': \bigwedge^2 \der(\bbC[v^{\pm 1}]) \to \bbC c_{\frakv}$$
            (which is given by $(\eta + \eta')(d_r, d_a) := \delta_{r + a, 0} r^3 c_{\frakv}$) is also a $2$-cocycle of $\der(\bbC[v^{\pm 1}])$ with values in $\bbC c_{\frakv}$. 

            Furthermore, we have that:
                $$\eta' \in B^2_{\Lie}( \der(\bbC[v^{\pm 1}]), \bbC c_{\frakv} )$$
            with notations as in definition \ref{def: lie_cocycles_and_coboundaries}.
        \end{proposition}
            \begin{proof}
                It is clear from the construction of $\eta'$ that it is linear and skew-symmetric; the only non-trivial thing to prove is that $\eta'$ satisfies the Jacobi identity in the sense of definition \ref{def: twisted_semi_direct_products}. To this end, simply consider the following, for all $i, j, k \in \Z$:
                    $$
                        \begin{aligned}
                            & \eta'([d_i, d_j], d_k) + \eta'([d_k, d_i], d_j) + \eta'([d_j, d_k], d_i)
                            \\
                            = & (i - j) \eta'(d_{i + j}, d_k) + (k - i) \eta'(d_{k + i}, d_j) + (j - k) \eta'(d_{j + k}, d_i)
                            \\
                            = & \delta_{i + j + k, 0} \left( (i - j) (i + j) + (k - i) (k + i) + (j - k) (j + k) \right) c_{\frakv}
                            \\
                            = & 0
                        \end{aligned}
                    $$
                    
                To show that $\eta'$ is a Lie $2$-coboundary in the sense of definition \ref{def: lie_cocycles_and_coboundaries}, we must show that there exists a linear map:
                    $$\tilde{\eta'}: \der(\bbC[v^{\pm 1}]) \to \bbC c_{\frakv}$$
                which is merely a linear map, such that:
                    $$\eta(d_i, d_j) = \tilde{\eta'}([d_i, d_j])$$
                (cf. examples \ref{example: lie_cocycles_and_coboundaries_with_trivial_coefficients} and \ref{example: low_degree_lie_cocycles_and_coboundaries_with_trivial_coefficients}). The RHS is nothing but:
                    $$\tilde{\eta'}([d_i, d_j]) = (i - j) \tilde{\eta'}(d_{i + j}) c_{\frakv}$$
                while by construction, the LHS is:
                    $$\eta(d_i, d_j) := \delta_{i + j, 0} i c_{\frakv}$$
                and hence:
                    $$\delta_{i + j, 0} i = (i - j) \tilde{\eta'}(d_{i + j})$$
                By setting $j = 0$, we then see that:
                    $$i (\tilde{\eta'}(d_i) - \delta_{i, 0}) = 0$$
                for all $i \in \Z$, which in turn implies that:
                    $$\tilde{\eta'}(d_i) = \delta_{i, 0}$$
                The sought-for linear map $\tilde{\eta}: \der(\bbC[v^{\pm 1}]) \to \bbC c_{\frakv}$ is thus defined, and hence exists.
            \end{proof}
        Using the well-known fact that:
            $$H^2_{\Lie}(\der(\bbC[v^{\pm 1}]), \bbC c_{\frakv}) \cong \bbC \eta$$
        (see lemma \ref{lemma: H^2_of_witt_algebra}) meaning that up to isomorphisms of extensions (cf. definition \ref{def: lie_algebra_extensions}), the Virasoro algebra corresponding to the $2$-cocycle $\eta: \der(\bbC[v^{\pm 1}]) \to \bbC c_{\frakv}$ is unique, one can infer from proposition \ref{prop: a_virasoro_coboundary} that, precisely because we have that:
            $$\sigma_i|_{ \bigwedge^2 \frakv } = \eta + \eta'$$
        the domain restrictions $\sigma_1|_{ \bigwedge^2 \frakv }$ must be cohomologous to the $2$-cocycle $\eta$, which is known to be non-cohomologous to $0$ (cf. proposition \ref{prop: non_zero_yangian_cocycles_on_witt_algebra}). 
        \begin{proposition} \label{prop: non_zero_yangian_cocycles_on_witt_algebra}
            The Lie $2$-cocycle:
                $$\eta + \eta' \in Z^2(\der(\bbC[v^{\pm 1}]), \bbC c_{\frakv})$$
            which given by:
                $$(\eta + \eta')(d_r, d_a) = \delta_{r + a, 0} r^3 c_{\frakv}$$
            - for all $r, a \in \Z$ and with $\eta'$ as in proposition \ref{prop: a_virasoro_coboundary} - is not cohomologous to $0$, i.e. its image under the canonical projection $Z^2(\der(\bbC[v^{\pm 1}]), \bbC c_{\frakv}) \to H^2(\der(\bbC[v^{\pm 1}]), \bbC c_{\frakv})$ (cf. definition \ref{def: complexes_and_cohomology}) is non-zero.
        \end{proposition}
            \begin{proof}
                Because we know that $\dim_{\bbC} H^2(\der(\bbC[v^{\pm 1}]), \bbC c_{\frakv}) = 1$ (cf. lemma \ref{lemma: H^2_of_witt_algebra}) and that $\eta' \in B^2(\der(\bbC[v^{\pm 1}]), \bbC c_{\frakv})$ (cf. proposition \ref{prop: a_virasoro_coboundary}), $\eta + \eta'$ must be cohomologous to $\eta$, which is not cohomologous to $0$ as shown in the proof of lemma \ref{lemma: H^2_of_witt_algebra}.
            \end{proof}
        \begin{remark}[The Virasoro central element]
            Thanks to proposition \ref{prop: non_zero_yangian_cocycles_on_witt_algebra}, we can also identify:
                $$c_{\frakv} = -c_v$$
            This tells us that:
                $$\frakv \cong \bbC \cdot (-c_v) \rtimes^{\sigma_i} \frakw$$
            for either $i = 1$ or $i = 2$, thereby identifying a copy of the Virasoro algebra as a Lie subalgebra inside $\d_{[2]}$.
        \end{remark}

        Now, even though it might be tempting to conclude right away that because of proposition \ref{prop: non_zero_yangian_cocycles_on_witt_algebra}, which tells us that:
            $$\sigma_i|_{\bigwedge^2 \frakw}: \bigwedge^2 \frakw \to \bbC c_{\frakv}$$
        is cohomologous to $\eta + \eta' \in Z^2_{\Lie}(\frakw, \bbC c_{\frakv})$, which is not $2$-coboundary, it must then also be true that $\sigma_i \not \in B^2_{\Lie}(\d_{[2]}, \z_{[2]})$. However, the subtlety here is that because $\z_{[2]}$ is non-trivial as a module over $\d_{[2]}$ (and likewise, over the Lie subalgebra $\frakw \subset \d_{[2]}$), unlike $\bbC c_{\frakv}$, one would have to actually check whether or not the restricted toroidal $2$-cocycle:
            $$\sigma_i|_{\bigwedge^2 \frakw}: \bigwedge^2 \d_{[2]} \to \z_{[2]}$$
        is $2$-coboundary. Our claim is that, somewhat surprisingly, it is in fact $2$-coboundary, for both $i = 1$ and $i = 2$.
        \todo[inline]{$\sigma_1|_{\bigwedge^2 \frakw}$ and $\sigma_2|_{\bigwedge^2 \frakw}$ are $2$-coboundary.}
        \begin{proposition} \label{prop: non_trivial_yangian_restricted_coboundaries_examples}
            Let $i \in \{1, 2\}$. As $\z_{[2]}$ is a (\textit{non-trivial}) $\d_{[2]}$-module via Lie derivatives (cf. lemma \ref{lemma: derivation_action_on_toroidal_centres}), it is also a module over the Lie subalgebra $\frakw$ of $\d_{[2]}$, determined by the same action. Then:
                $$\sigma_i|_{ \bigwedge^2 \frakw } \in B^2_{\Lie}(\frakw, \z_{[2]})$$
        \end{proposition}
            \begin{proof}
                Let $i \in \{1, 2\}$. Also, let us regard the $\d_{[2]}$-action on $\z_{[2]}$ as a Lie algebra homomorphism:
                    $$\rho: \d_{[2]} \to \gl( \z_{[2]} )$$
                Recall from lemma \ref{lemma: derivation_action_on_toroidal_centres} that this is given by:
                    $$\rho(D) := [D, -]_{\extendedtoroidal}$$
                for all $D \in \d_{[2]}$, which is well-defined because we know that $[\d_{[2]}, \z_{[2]}]_{\extendedtoroidal} \subseteq \z_{[2]}$ (cf. \textit{loc. cit.}).
            
                Per example \ref{example: low_degree_lie_coboundaries_with_non-trivial_coefficients}, it suffices to prove the existence of a linear map:
                    $$\tilde{\sigma}_i: \d_{[2]} \to \z_{[2]}$$
                (i.e. an element of $C_1(\d_{[2]}, \z_{[2]})$ in the notations of remark \ref{remark: simplified_chevalley_eilenberg_complexes}) such that:
                    $$
                        \begin{aligned}
                            \sigma_i(D, D') & = \rho(D) \cdot \tilde{\sigma}_i(D') - \rho(D') \cdot \tilde{\sigma}_i(D) - \tilde{\sigma}_i([D, D'])
                            \\
                            & = [D, \tilde{\sigma}_i(D')]_{\extendedtoroidal} - [D', \tilde{\sigma}_i(D)]_{\extendedtoroidal} - \tilde{\sigma}_i([D, D'])
                        \end{aligned}
                    $$
                    
                For what follows, let us recall:
                \begin{itemize}
                    \item from example \ref{example: yangian_cocycles_(counter)_examples} that:
                        $$\sigma_i|_{ \bigwedge^2 \frakw }(D_{r, -1}, D_{a, -1}) = -\delta_{r, -a} r^3 c_v$$
                    \item from lemma \ref{lemma: explicit_commutators_between_basis_elements_of_toroidal_central_orthogonal_complement} that:
                        $$[D_{a, b}, D_{r, s}] = (br - sa) D_{a + r, b + s + 1}$$
                    \item and from lemma \ref{lemma: explicit_commutators_between_central_basis_elements_and_derivations} that:
                        $$[D_{r, s}, K_{\alpha, \beta}]_{\extendedtoroidal} = ((\beta - 1)r - s\alpha) K_{\alpha - r, \beta - s - 1} + \delta_{(r, s + 1), (\alpha, \beta)} \left( r c_v + c_t \right)$$
                        $$[D, c_v]_{\extendedtoroidal} = [D, c_t]_{\extendedtoroidal} = 0$$
                    for all $D \in \d_{[2]}$.
                \end{itemize}
                Also, let us suppose that:
                    $$\tilde{\sigma}_i(D) := \sum_{(\alpha, \beta) \in \Z^2} \lambda_{\alpha, \beta}(D) K_{\alpha, \beta} + \lambda_v(D) c_v + \lambda_t(D) c_t$$
                for all $D \in \d_{[2]}$. From this, we get that:
                    $$\sigma_i(D, D') = \sum_{(\alpha, \beta) \in \Z^2} \left( \lambda_{\alpha, \beta}(D) [D, K_{\alpha, \beta}]_{\extendedtoroidal} - \lambda_{\alpha, \beta}(D') [D', K_{\alpha, \beta}]_{\extendedtoroidal} \right) - \tilde{\sigma}_i( [D, D'] )$$
                and we might as well assume that $\lambda_v = \lambda_t = 0$.

                Consider now the following:
                    $$
                        \begin{aligned}
                            & -\delta_{r, -a} r^3 c_v
                            \\
                            = & \sigma_i(D_{r, -1}, D_{a, -1})
                            \\
                            = & \sum_{(\alpha, \beta) \in \Z^2} \left( \lambda_{\alpha, \beta}(D_{r, -1}) [D_{r, -1}, K_{\alpha, \beta}]_{\extendedtoroidal} - \lambda_{\alpha, \beta}(D_{a, -1}) [D_{a, -1}, K_{\alpha, \beta}]_{\extendedtoroidal} \right) - \tilde{\sigma}_i( [D_{r, -1}, D_{a, -1}] )
                            \\
                            = &
                            \begin{aligned}
                                & \sum_{(\alpha, \beta) \in \Z^2} \lambda_{\alpha, \beta}(D_{r, -1}) \left( ((\beta - 1)r + \alpha) K_{\alpha - r, \beta} + \delta_{(r, 0), (\alpha, \beta)} \left( r c_v + c_t \right) \right)
                                \\
                                - & \sum_{(\alpha, \beta) \in \Z^2} \lambda_{\alpha, \beta}(D_{a, -1}) \left( ((\beta - 1)a + \alpha) K_{\alpha - a, \beta} + \delta_{(a, 0), (\alpha, \beta)} \left( a c_v + c_t \right) \right)
                                \\
                                - & (r - a) \tilde{\sigma}_i( D_{r + a, -1} )
                            \end{aligned}
                        \end{aligned}
                    $$
                From this, one infers that it must be the case that:
                    $$\lambda_{\alpha, \beta} = 0$$
                unless
                    $$(\alpha, \beta) = (-a, 0)$$
                    $$(\alpha, \beta) = (-r, 0)$$
                in the first and second terms, respectively. Subsequently, the above reduces to:
                    $$
                        \begin{aligned}
                            & -\delta_{r, -a} r^3 c_v
                            \\
                            = & 
                            \begin{aligned}
                                & \lambda_{-a, 0}(D_{r, -1}) \left( -(r + a) K_{-r - a, 0} + \delta_{r, -a} \left( r c_v + c_t \right) \right)
                                \\
                                - & \lambda_{-r, 0}(D_{a, -1}) \left( -(r + a) K_{-r - a, 0} + \delta_{r, -a} \left( a c_v + c_t \right) \right)
                                \\
                                - & (r - a) \tilde{\sigma}_i( D_{r + a, -1} )
                            \end{aligned}
                            \\
                            = & 
                            \begin{aligned}
                                & -(r + a) (\lambda_{-a, 0}(D_{r, -1}) - \lambda_{-r, 0}(D_{a, -1})) K_{-r - a, 0}
                                \\
                                + & \delta_{r, -a}\left( r \lambda_{-a, 0}(D_{r, -1}) - a \lambda_{-r, 0}(D_{a, -1}) \right) c_v
                                \\
                                + & \delta_{r, -a} (\lambda_{-a, 0}(D_{r, -1}) - \lambda_{-r, 0}(D_{a, -1})) c_t
                                \\
                                - & (r - a) \tilde{\sigma}_i( D_{r + a, -1} )
                            \end{aligned}
                        \end{aligned}
                    $$
                When $r + a = 0$, we have that:
                    $$-r^3 c_v = r \left( \lambda_{r, 0}(D_{r, -1}) + \lambda_{-r, 0}(D_{-r, -1}) \right) c_v + (\lambda_{r, 0}(D_{r, -1}) - \lambda_{-r, 0}(D_{-r, -1})) c_t - 2r \tilde{\sigma}_i( D_{0, -1} )$$
                from which it can be inferred that:
                    $$-r^2 = \lambda_{r, 0}(D_{r, -1}) + \lambda_{-r, 0}(D_{-r, -1})$$
                    $$2r \tilde{\sigma}_i( D_{0, -1} ) = (\lambda_{r, 0}(D_{r, -1}) - \lambda_{-r, 0}(D_{-r, -1})) c_t$$
                In turn, this implies that:
                    $$\tilde{\sigma}_i(D_{0, -1}) = c_t$$
                and when $r \not = 0$ the coefficients $\lambda_{\pm r, 0}(D_{\pm r, -1})$ are solutions to the following linear system:
                    $$
                        \begin{cases}
                            \lambda_{-r, 0}(D_{-r, -1}) + \lambda_{r, 0}(D_{r, -1}) = -r^2
                            \\
                            -\lambda_{-r, 0}(D_{-r, -1}) + \lambda_{r, 0}(D_{r, -1}) = 2r
                        \end{cases}
                    $$
                Solving this system yields:
                    $$\lambda_{\pm r, 0}(D_{\pm r, -1}) = -\frac12 r^2 \pm r$$

                By putting everything together, we get that:
                    $$
                        \tilde{\sigma}_i(D_{r, -1}) =
                        \begin{cases}
                            \text{$c_t$ if $r = 0$}
                            \\
                            \text{$\left( -\frac12 r^2 + r \right) K_{r, 0}$ if $r \not = 0$}
                        \end{cases}
                        = \delta_{r, 0} c_t + \left( -\frac12 r^2 + r \right) K_{r, 0}
                    $$
                The desired linear map $\tilde{\sigma}_i: \frakw \to \z_{[2]}$ is thus found, proving that $\sigma_i|_{ \bigwedge^2 \frakw }$ is $2$-coboundary. 
            \end{proof}

        The following result is our final conclusion regarding whether or not the toroidal $2$-cocycles $\sigma_1$ and $\sigma_2$ are cohomologous to $0$.
        \begin{theorem}[Are the toroidal $2$-cocycles $\sigma_1, \sigma_2$ cohomologous to $0$ or not ?] \label{theorem: non_trivial_yangian_cocycles_examples}
            Let $\sigma_1, \sigma_2 \in Z^2_{\Lie}(\d_{[2]}, \z_{[2]})$ be as in example \ref{example: yangian_cocycles_(counter)_examples}. Then:
                $$\sigma_1 \in B^2_{\Lie}(\d_{[2]}, \z_{[2]})$$
            while:
                $$\sigma_2 \not \in B^2_{\Lie}(\d_{[2]}, \z_{[2]})$$
        \end{theorem}
        \todo[inline]{Fixed proof. The domain extension of $\tilde{\sigma}_i$ to $\d_{[2]}$ now depends on $i$.}
            \begin{proof}
                First of all, we remark that whatever any domain extension of $\tilde{\sigma}_i$ from $\frakw$ to all of $\d_{[2]}$ ends up being, it should depend on $i \in \{1, 2\}$: otherwise, we would have that $\sigma_1 = \sigma_2$ as elements of $Z^2_{\Lie}(\d_{[2]}, \z_{[2]})$, but we have already shown via the computations done in example \ref{example: yangian_cocycles_(counter)_examples} that this is not true! That aside, our task is similar to as in proposition \ref{prop: non_trivial_yangian_restricted_coboundaries_examples}, which is to find a linear map:
                    $$\tilde{\sigma}_i: \d_{[2]} \to \z_{[2]}$$
                (depending on $i$) such that:
                    $$\tilde{\sigma}_i([D, D']) = [D, \tilde{\sigma}_i(D')]_{\extendedtoroidal} - [D', \tilde{\sigma}_i(D)]_{\extendedtoroidal} - \sigma_i(D, D')$$
                for all $D, D' \in \d_{[2]}$, which we can take to be basis elements without any loss of generality.

                For what follows, let us recall:
                \begin{itemize}
                    \item from lemma \ref{lemma: explicit_commutators_between_basis_elements_of_toroidal_central_orthogonal_complement}, that:
                        $$[D_v, D_t] = 0$$
                        $$[D_v, D_{r, s}] = r D_{r, s + 1}$$
                        $$[D_t, D_{r, s}] = D_{r, s + 1}$$
                    \item from example \ref{example: yangian_cocycles_(counter)_examples} that:
                        $$\sigma_i(D_{r, s}, D_t) = \delta_{i, 1} r^2 s K_{-r, -s - 2}$$
                        $$\sigma_i(D_v, D_t) = 0$$
                    \item from lemma \ref{lemma: explicit_commutators_between_central_basis_elements_and_derivations} that:
                        $$
                            [D, K_{a, b}]_{\extendedtoroidal} =
                            \begin{cases}
                                \text{$((b - 1)r - sa) K_{a - r, b - s - 1} + \delta_{(r, s + 1), (a, b)} \left( r c_v + c_t \right)$ if $D = D_{r, s}$}
                                \\
                                \text{$-a K_{a, b - 1}$ if $D_v$}
                                \\
                                \text{$-K_{a, b - 1}$ if $D_t$}
                            \end{cases}
                        $$
                        $$[D, c_v]_{\extendedtoroidal} = [D, c_t]_{\extendedtoroidal} = 0$$
                    for all $D \in \d_{[2]}$.
                \end{itemize}

                Also, for what follows, suppose for all $D \in \d_{[2]}$ that:
                    $$\tilde{\sigma}_i(D) := \sum_{(a, b) \in \Z^2} \lambda_{a, b}(D, i) K_{a, b} + \lambda_v(D_v, i) c_v + \lambda_t(D, i) c_t$$
                \begin{enumerate}
                    \item To begin, let us make the following observation.
                    
                    From lemma \ref{lemma: explicit_commutators_between_basis_elements_of_toroidal_central_orthogonal_complement}, it is known that:
                        $$[D, D'] \in \bigoplus_{(\alpha, \beta) \in \Z^2} \bbC D_{\alpha, \beta}$$
                    for all $D, D' \in \d_{[2]}$, meaning in particular that there do not exist elements $D, D' \in \d_{[2]}$ such that either:
                        $$D_v = [D, D']$$
                    or:
                        $$D_t = [D, D']$$
                    and hence no such elements $D, D' \in \d_{[2]}$ so that:
                        $$\tilde{\sigma}_i(D_v) = [D, \tilde{\sigma}_i(D')]_{\extendedtoroidal} - [D', \tilde{\sigma}_i(D)]_{\extendedtoroidal} - \sigma_i(D, D')$$
                    or:
                        $$\tilde{\sigma}_i(D_t) = [D, \tilde{\sigma}_i(D')]_{\extendedtoroidal} - [D', \tilde{\sigma}_i(D)]_{\extendedtoroidal} - \sigma_i(D, D')$$
                    In conjunction with the fact that:
                        $$[D_v, D_{r, s - 1}] = r D_{r, s}, [D_t, D_{r, s - 1}] = D_{r, s}$$
                    this means that to compute $\tilde{\sigma}_i(D_v)$ and $\tilde{\sigma}_i(D_t)$, we must do so by computing:
                        $$
                            \begin{aligned}
                                & r\tilde{\sigma}_i(D_{r, s})
                                \\
                                = & r\left( [D_t, \tilde{\sigma}_i(D_{r, s - 1})]_{\extendedtoroidal} - [D_{r, s - 1}, \tilde{\sigma}_i(D_t)]_{\extendedtoroidal} - \sigma_i(D_t, D_{r, s}) \right)
                                \\
                                = & [D_v, \tilde{\sigma}_i(D_{r, s - 1})]_{\extendedtoroidal} - [D_{r, s - 1}, \tilde{\sigma}_i(D_v)]_{\extendedtoroidal} - \sigma_i(D_v, D_{r, s})
                            \end{aligned}
                        $$
                    \item We know also that:
                        $$\sigma_i(D_v, D_t) = 0, [D_v, D_t] = 0$$
                    and so:
                        $$[D_t, \tilde{\sigma}_i(D_v)]_{\extendedtoroidal} = [D_v, \tilde{\sigma}_i(D_t)]_{\extendedtoroidal}$$
                    Using the fact that $[D_v, K_{a, b}]_{\extendedtoroidal} = -a K_{-a, -b - 1}$, we then get that:
                        $$[D_t, \tilde{\sigma}_i(D_v)]_{\extendedtoroidal} = -\sum_{(a, b) \in \Z^2} a \lambda_{a, b + 1}(D_t, i) K_{a, b}$$
                    Since $[D_v, K_{a, b}]_{\extendedtoroidal} = -K_{-a, -b - 1}$, the above implies that:
                        $$\lambda_{a, b}(D_v, i) = a\lambda_{a, b}(D_t, i)$$
                    meaning that the coefficients $\lambda_{a, b}(D_v, i)$ are determined by $\lambda_{a, b}(D_t, i)$. 
                    \item In light of the observations made above, let us now attempt to compute:
                        $$\tilde{\sigma}_i(D_{r, s})$$
                    For this, let us use the fact that:
                        $$D_{r, s} = [D_t, D_{r, s - 1}]$$
                    in order to get:
                        $$
                            \begin{aligned}
                                & \tilde{\sigma}_i(D_{r, s})
                                \\
                                = & [D_t, \tilde{\sigma}_i(D_{r, s - 1})]_{\extendedtoroidal} - [D_{r, s - 1}, \tilde{\sigma}_i(D_t)]_{\extendedtoroidal} - \sigma_i(D_t, D_{r, s})
                                \\
                                = & [D_t, \tilde{\sigma}_i(D_{r, s - 1})]_{\extendedtoroidal} - [D_{r, s - 1}, \tilde{\sigma}_i(D_t)]_{\extendedtoroidal} + \delta_{i, 1} r^2 s K_{-r, -s - 2}
                            \end{aligned}
                        $$
                    For visual clarity, let us compute the two brackets individually:
                        $$
                            \begin{aligned}
                                & [D_t, \tilde{\sigma}_i(D_{r, s - 1})]_{\extendedtoroidal}
                                \\
                                = & -\sum_{(a, b) \in \Z^2} \lambda_{a, b}(D_{r, s - 1}, i) K_{a, b - 1}
                                \\
                                = & -\sum_{(a, b) \in \Z^2} \lambda_{a, b + 1}(D_{r, s - 1}, i) K_{a, b}
                            \end{aligned}
                        $$
                        $$
                            \begin{aligned}
                                & [D_{r, s - 1}, \tilde{\sigma}_i(D_t)]_{\extendedtoroidal} 
                                \\
                                = & \sum_{(a, b) \in \Z^2} \lambda_{a, b}(D_t, i) \left( (b - 1)r - (s - 1)a \right) K_{a - r, b - s} + \delta_{(r, s), (a, b)} \left( r c_v + c_t \right)
                                \\
                                = & \sum_{(a, b) \in \Z^2} \left( (b + s - 1)r - (s - 1)(a + r) \right) \lambda_{a + r, b + s}(D_t, i) K_{a, b} + \delta_{r, 0} c_t
                            \end{aligned}
                        $$
                    Now, by putting everything together, we yield:
                        $$\tilde{\sigma}_i(D_{r, s}) = -\sum_{(a, b) \in \Z^2} M_i(r, s, a, b) K_{a, b} + \delta_{i, 1} r^2 s K_{-r, -s - 2} + \delta_{r, 0} c_t$$
                    where:
                        $$M_i(r, s, a, b) := \lambda_{a, b + 1}(D_{r, s - 1}, i) + \left( (b + s - 1)r - (s - 1)(a + r) \right) \lambda_{a + r, b + s}(D_t, i)$$
                    In light of this and of the fact that:
                        $$\tilde{\sigma}_i(D_{r, -1}) = (-\frac12 r^2 + r) K_{r, 0} + \delta_{r, 0} c_t$$
                    which is known from proposition \ref{prop: non_trivial_yangian_restricted_coboundaries_examples}, and since we are only trying to prove the existence of $\tilde{\sigma}_i$, not to determine it explicitly, let us now declare that:
                        $$
                            M_i(r, s, a, b) =
                            \begin{cases}
                                \text{$\delta_{i, 1} r^2 s$ if $(a, b) = (-r, -s - 2)$}
                                \\
                                \text{$\frac12 r^2 - r$ if $(a, b) = (r, 0)$}
                                \\
                                \text{$0$ otherwise}
                            \end{cases}
                        $$
                    From this, we gather that when $(a, b) = (-r, -s - 2)$, we shall obtain the following linear system, which can be solved to obtain the coefficients $\lambda_{a, b}(D_{r, s}, i)$ and $\lambda_{2a, b}(D_t, i)$:
                        $$
                            \begin{cases}
                                \lambda_{-r, -s - 1}(D_{r, s - 1}, i) -3r\lambda_{0, -2}(D_t, i) = \delta_{i, 1} r^2 s
                                \\
                                \lambda_{r, 1}(D_{r, s - 1}, i) - (s - 1)r\lambda_{2r, s}(D_t, i) = \frac12 r^2 - r
                            \end{cases}
                        $$
                    Note, however, that the system is only consistent when $i = 1$. Indeed, when $i = 2$, we have that:
                        $$\lambda_{-r, -s - 1}(D_{r, s - 1}, i) -3r\lambda_{0, -2}(D_t, i) = 0$$
                    from which we see that, when $r = 1$, we would have that:
                        $$\lambda_{0, -2}(D_t, i) = \frac13 \lambda_{-1, -s - 1}(D_{1, s - 1}, i)$$
                    but this would mean that $\lambda_{0, -2}(D_t, i)$ depends on $s$, which is clearly nonsensical. Therefore, a domain extension of $\tilde{\sigma}_2$ from $\frakw$ to $\bigoplus_{(r, s) \in \Z^2} \bbC D_{r, s}$ (let alone to $\d_{[2]} := \bigoplus_{(r, s) \in \Z^2} \bbC D_{r, s} \oplus \bbC D_v \oplus \bbC D_t$) can not exist, i.e. $\sigma_2$ is not $2$-coboundary.
                        
                    \todo[inline]{Is it clear that the system is consistent even when $i = 1$ ?}
                    
                    As of now, we have already obtained:
                        $$\lambda_v(D_{r, s}, 1) = 0, \lambda_t(D_{r, s}, 1) = \delta_{r, 0}$$
                    We have thus shown that the values:
                        $$\tilde{\sigma}_1(D_{r, s}) = \sum_{(a, b) \in \Z^2} \lambda_{a, b}(D_{r, s}, 1) K_{a, b} + \delta_{r, 0} c_t$$
                    are well-defined (in particular, they depend on $i$), and so \textit{there is a domain extension of $\tilde{\sigma}_1$ from $\frakw := \bigoplus_{r \in \Z} \bbC D_{r, -1}$ to $\bigoplus_{(r, s) \in \Z^2} \bbC D_{r, s}$.}
                    \item Also, let us note that the above implies that:
                        $$\tilde{\sigma}_1(D_t) \not = 0$$
                    (and hence:
                        $$\tilde{\sigma}_1(D_v) \not = 0$$
                    as well) because otherwise, $\tilde{\sigma}_1(D_{r, s})$ would not have no non-zero summand lying in $\bbC c_t$, which would be a contradiction. Regardless, these values are well-defined, now that we know that:
                        $$\tilde{\sigma}_1: \bigoplus_{(r, s) \in \Z^2} \bbC D_{r, s} \to \z_{[2]}$$
                    is well-defined, again because we have that:
                        $$
                            \begin{aligned}
                                & \tilde{\sigma}_i(D_{r, s})
                                \\
                                = & r\left( [D_t, \tilde{\sigma}_i(D_{r, s - 1})]_{\extendedtoroidal} - [D_{r, s - 1}, \tilde{\sigma}_i(D_t)]_{\extendedtoroidal} - \sigma_i(D_t, D_{r, s}) \right)
                                \\
                                = & [D_v, \tilde{\sigma}_i(D_{r, s - 1})]_{\extendedtoroidal} - [D_{r, s - 1}, \tilde{\sigma}_i(D_v)]_{\extendedtoroidal} - \sigma_i(D_v, D_{r, s})
                            \end{aligned}
                        $$
                    A domain extension of $\tilde{\sigma}_1$ from $\bigoplus_{(r, s) \in \Z^2} \bbC D_{r, s}$ to $\d_{[2]}$ therefore exists.
                \end{enumerate}
            \end{proof}