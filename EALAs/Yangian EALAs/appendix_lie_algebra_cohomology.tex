\section{Lie algebra cohomology}
    \subsection{A brief account of the generalities}
        Let us work over a fixed field $k$.
    
        Even though it is true that every twisted semi-direct product:
            $$0 \to \t \to \t \rtimes^{\sigma} \d \to \d \to 0$$
        gives rise to a $2$-cocycle $\sigma: \bigwedge^2 \d \to \t$, the converse statement is slightly more subtle. Namely, it is possible that two unequal $2$-cocycles:
            $$\sigma \not = \sigma'$$
        would give rise to two twisted semi-direct products that are isomorphic as extensions of $\d$ (see e.g. proposition \ref{prop: a_virasoro_coboundary} and theorem \ref{theorem: non_trivial_yangian_cocycles_examples}), and in order to capture this fact, we will need to introduce the language of Lie algebra cohomology, which also necessitates a brief discussion of homological algebra in general. We refer the reader to \cite{hilton_stammbach_homological_algebra} as a general reference on homological algebra, and particularly, Chapter VII therein as a reference on the generalities of Lie algebra cohomology. 

        We will be adopting the following (somewhat \textit{ad hoc}) perspective on the construction of Lie algebra cohomology \say{groups}.
        \begin{convention}
            In discussing homological algebra, we will have to make use of some categorical language, but we will keep this to a minimum, only mentioning what is absolutely needed. We refer the reader to \cite{maclane}, particularly Chapter VIII therein, for more details.
        \end{convention}
        \begin{convention}
            We take for granted the fact that for any associative ring $R$, left/right-$R$-modules form an abelian category (cf. \cite[Section VIII.3, p. 198]{maclane}) - denoted, respectively, by ${}^lR\mod$ and ${}^rR\mod$ - with all limits and colimits.

            We will only be working with left-modules, but everything works identically for right-modules.
        \end{convention}

        \begin{definition}[Complexes and cohomomology] \label{def: complexes_and_cohomology}
            (Cf. \cite[Section IV.1]{hilton_stammbach_homological_algebra}) Let $R$ be a ring. A \textbf{complex} or \textbf{chain complex} of left-$R$-modules is a diagram of left-$R$-modules:
                $$\{P_i, \del_i\}_{i \in \Z} := \{ ... \xrightarrow[]{\del_{-3}} P_{-2} \xrightarrow[]{\del_{-2}} P_{-1} \xrightarrow[]{\del_{-1}} P_0 \xrightarrow[]{\del_0} P_1 \xrightarrow[]{\del_1} P_2 \xrightarrow[]{\del_2} ... \}$$
            (sometimes also abbreviated by $(P_{\bullet}, \del_{\bullet})$) such that for each $i \in \Z$, one has that:
                $$\del_i \circ \del_{i - 1} = 0$$
            The connecting maps $\del_i$ are typically called \textbf{differentials} or \textbf{coboundary maps}.
            
            For each $i \in \Z$, the \textbf{$i^{th}$ cohomology} of such a complex is given by:
                $$H^i(P_{\bullet}, \del_{\bullet}) := \frac{ \ker \del_i }{ \im \del_{i - 1} }$$
            (typically this is written as just $H^i(P_{\bullet})$, with the differentials being implicitly understood). Elements of $\ker \del_i$ are usually called \textbf{$i$-cocycles} and those of $\im \del_{i - 1}$ are usually called \textbf{$i$-coboundaries}; elements of $H^i(P_{\bullet}, \del_{\bullet})$ are usually called \textbf{$i^{th}$ cohomology classes}, and cocycles which are representatives of the same cohomology class are said to be \textbf{cohomologous}.
            
            If:
                $$H^i(P_{\bullet}, \del_{\bullet}) \cong 0$$
            for all $i \in \Z$ then we shall say that the complex $(P_{\bullet}, \del_{\bullet})$ is \textbf{exact}.
        \end{definition}
        \begin{example}
            A \textbf{short exact sequence} (cf. e.g. definition \ref{def: lie_algebra_extensions}) is nothing but an exact complex with only 3 terms. 
        \end{example}
        
        If $R$ is any associative ring and $M$ is any left-$R$-module, then the functor:
            $$\Hom_R(-, M): {}^lR\mod^{\op} \to \Z\mod$$
        is generally only left-exact, which for us shall mean that it maps kernels and finite direct sums in ${}^lR\mod^{\op}$ (i.e. cokernels and finite direct sums in ${}^lR\mod$) to kernels and finite direct sums in $\Z\mod$. Consequently, any projective resolution\footnote{It is a fact from homological algebra that any choice of projective resolution would return the same cohomologies, so the non-trivial task is to choose a resolution that would make computations convenient or in fact, even feasible in the first place. In modern terminologies, all the projective resolutions of the same module are \say{quasi-isomorphic}.} of a given left-$R$-module $P$:
            $$(P_{\bullet}, \del_{\bullet}) := \{ ... \xrightarrow[]{\del_{-3}} P_{-2} \xrightarrow[]{\del_{-2}} P_{-1} \xrightarrow[]{\del_{-1}} P_0 \xrightarrow[]{\del_0} P \to 0 \}$$
        (i.e. a chain complex of left-$R$-modules wherein each term is projective as a left-$R$-module) is mapped by $\Hom_R(-, M)$ to a diagram of $\Z$-modules as follows:
            $$\Hom_R(P_{\bullet}, M) := \{ 0 \to \Hom_R(P, M) \Hom_R(P^0, M) \to \Hom_R(P^{-1}, M) \to \Hom_R(P^{-2}, M) \to ... \}$$
        \begin{definition}[$\Ext$-groups] \label{def: Ext_groups}
            With notations as above, we can define:
                $$\Ext^i_R(P, M) := H^i(\Hom_R(P_{\bullet}, M))$$
            for each $i \in \Z_{\geq 0}$. By constructions, these are $\Z$-modules. 
        \end{definition}
        \begin{remark}
            If $R$ is an associative algebra defined over a field $k$, then the $\Z$-modules $\Ext^i_R(P, M)$ will actually carry $k$-vector space structures as well.
        \end{remark}
        \begin{definition}[Lie algebra cohomology and the Chevalley-Eilenberg resolution] \label{def: lie_algebra_cohomology}
            If $\a$ is a Lie algebra over $k$, then its \textbf{(abelian) Lie algebra cohomology} with \textbf{coefficients} in some left-$\rmU(\a)$-module $M$ (which is typically referred to simply as an $\a$-module) shall be given by:
                $$H^i_{\Lie}(\a, M) := \Ext^i_{\rmU(\a)}(k, M) := H^i( \Hom_{\rmU(\a)}(k_{\bullet}, M) )$$
            where $k$ is regarded as a trivial left-$\rmU(\a)$-module, equipped with some projective resolution of left-$\rmU(\a)$-modules $k_{\bullet}$.

            The standard projective resolution for the left-$\rmU(\a)$-module is the \textbf{Chevalley-Eilenberg resolution} (sometimes referred to as the \textbf{Chevalley-Eilenberg complex}):
                $$\left\{ k_{-i}(\a) := \rmU(\a) \tensor_k \bigwedge^i \a \right\}_{i \in \Z_{\geq 0}}$$
            and for each $i \in \Z_{\geq 0}$, the corresponding differential/coboundary map:
                $$\del_{-(i + 1)}: k_{-(i + 1)}(\a) \to k_{-i}(\a)$$
            is given on pure tensors by:
                $$
                    \begin{aligned}
                        & \del_{-(i + 1)}\left( a \tensor (x_1 \wedge ... \wedge x_{i + 1}) \right)
                        \\
                        := & \sum_{1 \leq p \leq i + 1} (-1)^p a x_p \tensor (x_1 \wedge ... \wedge \not{x_p} \wedge ... \wedge x_{i + 1}) + \sum_{1 \leq p < q \leq i + 1} (-1)^{p + q} a \tensor ( [x_p, x_q]_{\a} \wedge x_1 \wedge ... \wedge \not{x_p} \wedge ... \wedge \not{x_q} \wedge ... \wedge x_{i + 1})
                    \end{aligned}
                $$
            for all $a \in \rmU(\a)$ and all $x_1, ..., x_{i + 1} \in \a$ (cf. \cite[Section VII.4]{hilton_stammbach_homological_algebra}).
        \end{definition}
        \begin{remark}[A simplified expression of Chevalley-Eilenberg complexes] \label{remark: simplified_chevalley_eilenberg_complexes}
            Let $\a$ be a Lie algebra over $k$ and let $M$ be an arbitrary $\a$-module; let us also regard $M$ as an abelian Lie algebra. Next, consider the following complex of $k$-vector spaces\footnote{Which is sometimes also called the Chevalley-Eilenberg complex of $\a$ (with coefficients in $M$).}:
                $$\{ C_i(\a, M) := \Hom_{\rmU(\a)}(k_{-i}(\a), M) \}_{i \in \Z_{\geq 0}}$$
            and observe that its terms $C_{-i}(\a, M)$ can be described as follows:
                $$
                    \begin{aligned}
                        & C_i(\a, M)
                        \\
                        := & \Hom_{\rmU(\a)}(k_{-i}(\a), M)
                        \\
                        \cong & \Hom_{\rmU(\a)}( \rmU(\a) \tensor_k \bigwedge^i \a, M ) \ni \left( a \tensor (x_1 \wedge ... \wedge x_i) \mapsto \varphi( a \tensor (x_1 \wedge ... \wedge x_i) ) \right)
                        \\
                        \cong & \Hom_{\rmU(\a)}( \rmU(\a), \Hom_k( \bigwedge^i \a, M ) ) \ni \left( a \mapsto \varphi(a \tensor -) \right)
                        \\
                        \cong & \Hom_k( \bigwedge^i \a, M ) \ni \varphi(1 \tensor -)
                    \end{aligned}
                $$
            for all $a \in \rmU(\a)$, all $x_1, ..., x_i \in \a$, and all $\varphi \in \Hom_{\rmU(\a)}( \rmU(\a) \tensor_k \bigwedge^i \a, M )$.
        \end{remark}
        \begin{definition}[Lie cocycles and coboundaries] \label{def: lie_cocycles_and_coboundaries}
            Let us denote by:
                $$d_i^M: C_i(\a, M) \to C_{i + 1}(\a, M)$$
            the differentials of thee complex of $k$-vector spaces $\{C_i(\a, M)\}_{i \in \Z_{\geq 0}}$. These are given by:
                $$d_i^M := \Hom_{\rmU(\a)}(\del_{-(i + 1)}, M)$$
            When $M$ is understood from the context, the superscripts may be omitted.
        
            For each $i \in \Z_{\geq 0}$, \textbf{Lie algebra $i$-cocycles} of $\a$ with values in $M$ shall be elements of:
                $$Z^i_{\Lie}(\a, M) := \ker d_i^M$$
            while \textbf{Lie algebra $i$-coboundaries} of $\a$ with values in $M$ shall be elements of:
                $$B^i_{\Lie}(\a, M) := \im d_{i - 1}^M$$
        \end{definition}
        \begin{remark}
            From remark \ref{remark: simplified_chevalley_eilenberg_complexes}, one sees that - for any $i \in \Z_{\geq 0}$, a Lie $i$-cocycle of $\a$ with values in some $\a$-module $M$ is nothing but an alternating $k$-linear map:
                $$\sigma: \bigwedge^i \a \to M$$
            such that:
                $$d_i^M(\sigma) = 0$$
            In particular, using the fact that:
                $$d_i^M := \Hom_{\rmU(\a)}(\del_{-(i + 1)}, M)$$
            by construction, together with the construction of the maps $\del_{-(i + 1)}$ as in definition \ref{def: lie_algebra_cohomology}, one then sees that indeed, when $i = 2$, one recovers the notion of Lie $2$-cocycles of $\a$ with values in $M$ as in definition \ref{def: twisted_semi_direct_products} as linear maps $\sigma: \bigwedge^2 \a \to M$ satisfying the Jacobi identity in the sense stated in \textit{loc. cit.}
        \end{remark}

        Let us now see how Lie cocycles and Lie coboundaries are given explicitly, particularly in low degrees. 

        We begin with the case of cocycles and coboundaries with values in trivial modules. 
        \begin{example}[Lie cocycles and coboundaries with trivial coefficients] \label{example: lie_cocycles_and_coboundaries_with_trivial_coefficients}
            Let $\a$ be a Lie algebra over $k$ and regard $k$ itself as a trivial $\a$-module. For each $i \in \Z_{\geq 0}$, we then have - per the discussions in remark \ref{remark: simplified_chevalley_eilenberg_complexes} that:
                $$C_i(\a, k) \cong \Hom_k(\bigwedge^i \a, k) =: (\bigwedge^i \a)^*$$
            and the differentials:
                $$d_i^k: C_i(\a, k) \to C_{i + 1}(\a, k)$$
            are then nothing but:
                $$\del_{-(i + 1)}^* := \Hom_k(\del_{-(i + 1)}, k)$$
            \begin{itemize}
                \item A Lie $i$-cocycle of $\a$ with coefficients in $k$ is then - by definition - an element of $\ker \del_{-(i + 1)}^*$, i.e. a linear map:
                    $$\sigma: \bigwedge^i \a \to k$$
                such that:
                    $$
                        \begin{aligned}
                            & 0
                            \\
                            = & \del_{-(i + 1)}^*(\sigma)( (x_1 \wedge ... \wedge x_{i + 1}) )
                            \\
                            = & \sum_{1 \leq p \leq i + 1} (-1)^p \sigma(x_1 \wedge ... \wedge \not{x_p} \wedge ... \wedge x_{i + 1}) + \sum_{1 \leq p < q \leq i + 1} (-1)^{p + q} \sigma( [x_p, x_q]_{\a} \wedge x_1 \wedge ... \wedge \not{x_p} \wedge ... \wedge \not{x_q} \wedge ... \wedge x_{i + 1})
                        \end{aligned}
                    $$
                for all $x_1, ..., x_i \in \a$.
                \item A Lie $i$-coboundary is an element of $\im \del_{-i}^*$, i.e. a linear map:
                    $$\beta: \bigwedge^i \a \to k$$
                for which there exists another linear map:
                    $$\tilde{\beta}: \bigwedge^{i - 1} \a \to k$$
                i.e. a $(i - 1)$-cocycle $\tilde{\beta}$, such that:
                    $$
                        \begin{aligned}
                            & \beta(x_1 \wedge ... \wedge x_i)
                            \\
                            = & \del_{-i}^*(\tilde{\beta})(x_1 \wedge ... \wedge x_i)
                            \\
                            = & \sum_{1 \leq p \leq i} (-1)^p \sigma(x_1 \wedge ... \wedge \not{x_p} \wedge ... \wedge x_i) + \sum_{1 \leq p < q \leq i} (-1)^{p + q} \sigma( [x_p, x_q]_{\a} \wedge x_1 \wedge ... \wedge \not{x_p} \wedge ... \wedge \not{x_q} \wedge ... \wedge x_i)
                        \end{aligned}
                    $$
            \end{itemize}
        \end{example}
        \begin{example}[Lie $2$-cocycles and $2$-coboundaries with trivial coefficients] \label{example: low_degree_lie_cocycles_and_coboundaries_with_trivial_coefficients}
            Of particular usefulness to us are Lie $2$-cocycles and $2$-coboundaries with trivial coefficients. In this case where:
                $$i = 2$$
            the conditions in example \ref{example: lie_cocycles_and_coboundaries_with_trivial_coefficients} reduce to the following.
            \begin{itemize}
                \item A Lie $2$-cocycle of $\a$ with coefficients in the trivial $\a$-module $k$ is a linear map:
                    $$\sigma: \bigwedge^2 \a \to k$$
                satisfying the Jacobi identity, in the sense of definition \ref{def: twisted_semi_direct_products}.
                \item A Lie $2$-coboundary of $\a$ with coefficients in $k$ is a linear map:
                    $$\beta: \bigwedge^2 \a \to k$$
                for which there exist a linear map:
                    $$\tilde{\beta}: \a \to k$$
                such that:
                    $$\beta(x \wedge y) = \tilde{\beta}([x, y])$$
                Equivalently, this is saying that:
                    $$\beta = d_1(\tilde{\beta})$$
                Much like Lie $2$-cocycles, the value of a Lie $2$-boundaries $\beta$ at some $x \wedge y \in \bigwedge^2 \a$ is usually denotes by $\beta(x, y)$, i.e. we tend to think of Lie $2$-coboundaries as certain skew-symmetric bilinear maps $\a \x \a \to k$.
            \end{itemize}
        \end{example}
        
        Next, let us consider low-degree Lie cocycles and coboundaries with values in possibly non-trivial modules as well. Once again, by letting $i = 2$, one recovers the notion of Lie $2$-cocycles as in definition \ref{def: twisted_semi_direct_products}, but Lie $2$-coboundaries with values in possibly non-trivial modules are slightly different from those with trivial coefficients, as in example \ref{example: low_degree_lie_cocycles_and_coboundaries_with_trivial_coefficients}. We will not be needing the cases where $i > 2$.

        To make sense of Lie $2$-coboundaries with values in non-trivial modules, it shall be convenient to have a somewhat concrete description of $1$-cocycles and $1$-coboundaries.
        \begin{definition}[Lie derivations] \label{def: lie_derivations}
            (Cf. \cite[Section VII.2, Equation 2.2, p. 234]{hilton_stammbach_homological_algebra}) Let $\d$ be a Lie algebra and let $\Omega$\footnote{We choose these notations because eventually, the case where $\d := \d_{[2]}$ and $\Omega := \z_{[2]} \cong \bar{\Omega}^1_{A/\bbC}$ will be of interest to us.} be an $\d$-module defined by a Lie algebra action:
                $$\rho: \d \to \gl(\Omega)$$
            A \textbf{derivation} of $\d$ with values in $\Omega$ is a linear map:
                $$L: \d \to \Omega$$
            satisfying the following property for all $x, y \in \d$:
                $$L( [x, y] ) = \rho(x) \cdot L(y) - \rho(y) \cdot L(x)$$
            Such derivations form a vector space, for which we shall write $\der(\d, \Omega)$. For every $\omega \in \Omega$, one can define an \textbf{inner derivation} $L_{\omega}$, specified by:
                $$L_{\omega} = \rho(-) \cdot \omega$$
            Inner derivations form a vector subspace of $\der(\d, \Omega)$, which is denoted by $\inn(\d, \Omega)$. Derivations that are not inner are said to be \textbf{outer}, and the vector space of outer derivations is identified as:
                $$\out(\d, \Omega) := \der(\d, \Omega)/\inn(\d, \Omega)$$
        \end{definition}
        \begin{example}
            Let $\d$ be a Lie algebra and let $\d$ also be considered as a module over itself via the adjoint action. Then, for any $x \in \d$, the map:
                $$\ad(x): \d \to \d$$
            will be an inner derivation of $\d$ with values in itself.
        \end{example}
        \begin{theorem}[$H^1_{\Lie}$ and derivations of Lie algebras]
            \cite[Theorem 2.1 and Proposition 2.2]{hilton_stammbach_homological_algebra} Let $\d$ be a Lie algebra and $\Omega$ be an $\d$-module. Then:
                $$Z^1_{\Lie}(\d, \Omega) \cong \der(\d, \Omega)$$
                $$B^1_{\Lie}(\d, \Omega) \cong \inn(\d, \Omega)$$
            and hence:
                $$H^1_{\Lie}(\d, \Omega) \cong \out(\d, \Omega)$$
        \end{theorem}
        \begin{example}[Lie $2$-coboundaries with \textit{non-trivial} coefficients] \label{example: low_degree_lie_coboundaries_with_non-trivial_coefficients}
            Let $\Omega$ be an $\d$-module defined by a Lie algebra action
                $$\rho: \d \to \gl(\Omega)$$
            By construction, we have that:
                $$d_i^\Omega := \Hom_{\rmU(\d)}(\del_{-(i + 1)}, \Omega)$$
            for all $i \in \Z_{\geq 0}$. By using remark \ref{remark: simplified_chevalley_eilenberg_complexes}, we shall see that a Lie $2$-coboundary of $\d$ with values in $\Omega$ shall be a linear map\footnote{Again, let us regard maps out of exterior powers as alternating multi-linear maps.}:
                $$\beta: \bigwedge^2 \d \to \Omega$$
            for which there is a linear map:
                $$\tilde{\beta}: \d \to \Omega$$
            satisfying the following property:
                $$\beta(x, y) = \left( \rho(x) \cdot \tilde{\beta}(y) - \rho(y) \cdot \tilde{\beta}(x) \right) - \tilde{\beta}([x, y])$$
            (cf. \cite[Equation 1.6, p. 421]{kassel_quantum_groups}). In other words, $2$-coboundaries are error terms measuring how far a linear map $\d \to \Omega$ is from being a derivation of $\d$ with values in $\Omega$.
        \end{example}

        We will be needing these notions in the discussions leading up to theorem \ref{theorem: non_trivial_yangian_cocycles_examples}, in order to be able to equivocate certain unequal $2$-cocycles as being cohomologous to one another. More specifically, we will be making use of the fact that isomorphism classes of abelian extensions of a given Lie algebra are in bijection with its $2^{nd}$ cohomology (with suitable coefficients, of course).
        \begin{theorem}[$H^2_{\Lie}$ and abelian extensions] \label{theorem: H^2_of_lie_algebras_and_abelian_extensions}
            (Cf. \cite[Theorem VII.3.3]{hilton_stammbach_homological_algebra}) Let $\d$ be a Lie algebra over $k$ and $\Omega$ be an $\d$-module, equipped with the abelian Lie algebra structure. There is then a bijection:
                $$H^2_{\Lie}(\d, \Omega) \xrightarrow[]{\cong} \{ \text{isomorphism classes of extensions of $\d$ by $\Omega$} \}$$
                $$\sigma \mapsto \Omega \rtimes^{\sigma} \d$$
        \end{theorem}
        
        \begin{remark}[Non-abelian Lie algebra cohomology ?]
            There is also a variant of the construction given in definition \ref{def: lie_algebra_cohomology}, called \textbf{non-abelian Lie algebra cohomology}, which ostensibly is for the purpose of classifying Lie algebra extensions:
                $$0 \to \t \to \frake \to \d \to 0$$
            where the kernel $\t$ is not necessarily abelian. This is harder to define and in fact, is unnecessary for our purposes, so we will make no further mention of it.
        \end{remark}
    
    \subsection{An application: the Virasoro algebra} \label{subsection: virasoro_algebra}
        Now, let $k := \bbC$. This is not really necessary, but we make the assumption anyway, as when we apply the results of this subsection in the discussions leading up to theorem \ref{theorem: non_trivial_yangian_cocycles_examples}, the ground field there will be $\bbC$.
    
        As a particular application of Lie algebra cohomology, let us demonstrate that the Virasoro is the unique non-trivial central extension of the Lie algebra $\der(\bbC[v^{\pm 1}])$ of derivations on $\bbC[v^{\pm 1}]$ by a $1$-dimensional centre (and hence also universal).
    
        By exploiting the fact that the following commutation relation holds in the untwisted affine Kac-Moody algebra $\hat{\g}$:
            $$[D_{\aff}, c_{\aff}]_{\hat{\g}} = 0$$
        (cf. subsection \ref{subsection: a_fixed_untwisted_affine_kac_moody_algebra}) along with the fact that Lie $2$-cocycles are central \textit{a priori} (cf. corollary \ref{coro: 2_cocycles_are_central}), the classical \textbf{Virasoro algebra} can be defined to be the UCE:
            $$\frakv := \der(\bbC[v^{\pm 1}]) \oplus^{\eta} \bbC c_{\aff}$$
        of the \textbf{Witt algebra} $\der(\bbC[v^{\pm 1}])$. If we are to regard $\der(\bbC[v^{\pm 1}])$ as $\bbC[v^{\pm 1}] \cdot D_{\aff}$ then the $2$-cocycle:
            $$\eta: \bigwedge^2 \der(\bbC[v^{\pm 1}]) \to \bbC c_{\aff}$$
        can be given by\footnote{Physicists tend to renormalise this to $\frac{1}{12} \eta$.}:
            $$\eta(v^r D_{\aff}, v^a D_{\aff}) := (r^3 - r) \delta_{r + a, 0} c_{\aff}$$
        Subsequently, the bracket on $\frakv$ can be given as follows, for all $r, a \in \Z$:
            $$
                \begin{aligned}
                    \left[ v^r D_{\aff}, v^a D_{\aff} \right]_{\frakv} & = [ v^r D_{\aff}, v^a D_{\aff} ]_{\der(\bbC[v^{\pm 1}])} + \eta(v^r D_{\aff}, v^a D_{\aff})
                    \\
                    & = (a - r) v^{r + a} D_{\aff} + \eta(v^r D_{\aff}, v^a D_{\aff})
                \end{aligned}
            $$
        Typically, one endows $\der(\bbC[v^{\pm 1}])$ with the basis:
            $$\{ d_r := -v^r D_{\aff} \}_{r \in \Z}$$
        The inclusion of the extra minus sign is so that:
            $$[d_r, d_a]_{\der(\bbC[v^{\pm 1}])} = (r - a) D_{r + a}$$
        which is more or less a choice of aesthetics.
    
        
        \begin{lemma}[$H^2_{\Lie}$ of the Witt algebra] \label{lemma: H^2_of_witt_algebra}
            (Cf. \cite[Proposition 1.3]{kac_raina_rozhkovskaya_bombay_lectures_on_highest_weight_modules_of_infinite_dimensional_lie_algebras}) Let $\bbC c_{\frakv}$ be viewed as a trivial $\der(\bbC[v^{\pm 1}])$-module. Then:
                $$\dim_{\bbC} H^2_{\Lie}( \der(\bbC[v^{\pm 1}]), \bbC c_{\frakv} ) = 1$$
        \end{lemma}
            \begin{proof}
                Note first of all that any Lie $2$-cocycle of $\der(\bbC[v^{\pm 1}])$ with values in $\bbC c_{\frakv}$ is necessarily given on basis elements $d_i, d_j \in \der(\bbC[v^{\pm 1}])$ by:
                    $$\tau(d_i, d_j) = N_{i, j}(\tau) c_{\frakv}$$
                for some $N_{i, j}(\tau) \in \bbC$. By skew-symmetry, we have that:
                    $$\tau(d_i, d_j) = -\tau(d_j, d_i)$$
                for all $i, j \in \Z$, and hence:
                    $$N_{i, j}(\tau) = -N_{j, i}(\tau)$$
                By the Jacobi identity, we the following, for all $i, j, k \in \Z$:
                    $$
                        \begin{aligned}
                            & 0
                            \\
                            = & \tau([d_i, d_j], d_k) + \tau([d_k, d_i], d_j) + \tau([d_j, d_k], d_i)
                            \\
                            = & (i - j) \tau(d_{i + j}, d_k) + (k - i) \tau(d_{k + i}, d_j) + (j - k) \tau(d_{j + k}, d_i)
                            \\
                            = & \left( (i - j) N_{i + j, k}(\tau) + (k - i) N_{k + i, j}(\tau) + (j - k) N_{j + k, i}(\tau) \right) c_{\frakv}
                        \end{aligned}
                    $$
                which implies that:
                    $$(i - j) N_{i + j, k}(\tau) + (k - i) N_{k + i, j}(\tau) + (j - k) N_{j + k, i}(\tau) = 0$$
                If we let $k = 0$ then we will get:
                    $$(i - j) N_{i + j, 0}(\tau) - i N_{i, j}(\tau) + j N_{j, i}(\tau) = 0$$
                but since we have that $N_{j, i}(\tau) = -N_{i, j}(\tau)$, as shown above, the above becomes:
                    $$(i - j) N_{i + j, 0}(\tau) = (i + j) N_{i, j}(\tau)$$
                Using this, we thus get that:
                    $$
                        \begin{aligned}
                            & [[d_i, d_j]_{\frakv}, d_0]_{\frakv} = (i - j) [d_{i + j}, d_0]_{\frakv}
                            \\
                            = & (i - j) \left( (i + j) d_{i + j} + N_{i, j, 0}(\tau) c_{\frakv} \right)
                            \\
                            = & (i + j) [d_i, d_j] + (i + j) N_{i, j}(\tau) c_{\frakv}
                        \end{aligned}
                    $$
                At the same time, the Jacobi identity implies that:
                    $$
                        \begin{aligned}
                            & 0
                            \\
                            = & [[d_i, d_j]_{\frakv}, d_0]_{\frakv} + [[d_0, d_i]_{\frakv}, d_j]_{\frakv} + [[d_j, d_0]_{\frakv}, d_i]_{\frakv}
                            \\
                            = & [[d_i, d_j]_{\frakv}, d_0]_{\frakv} - i [d_i, d_j]_{\frakv} + j [d_j, d_i]_{\frakv}
                            \\
                            = & [[d_i, d_j]_{\frakv}, d_0]_{\frakv} - (i + j) [d_i, d_j]_{\frakv}
                        \end{aligned}
                    $$
                which implies that:
                    $$[[d_i, d_j]_{\frakv}, d_0]_{\frakv} = (i + j) [d_i, d_j]_{\frakv}$$
                Combining the two observations then yields:
                    $$(i + j) N_{i, j}(\tau) = 0$$
                from which one sees that:
                    $$N_{i, j}(\tau) = \delta_{i + j, 0} N_{i, -i}(\tau)$$
                Now, consider once more the Jacobi identity for $\tau$:
                    $$
                        \begin{aligned}
                            & 0
                            \\
                            = & (i - j) N_{i + j, k}(\tau) + (k - i) N_{k + i, j}(\tau) + (j - k) N_{j + k, i}(\tau)
                            \\
                            = & \delta_{i + j + k, 0} \left( (i - j)  N_{i + j, -(i + j)}(\tau) + (k - i) N_{k + i, -(k + i)}(\tau) + (j - k) N_{j + k, -(j + k)}(\tau) \right)
                            \\
                            = & (i - j)  N_{i + j, -(i + j)}(\tau) + (-(i + j) - i) N_{-(i + j) + i, -(-(i + j) + i)}(\tau) + (j + (i + j)) N_{j - (i + j), -(j -(i + j))}(\tau)
                            \\
                            = & (i - j)  N_{i + j, -(i + j)}(\tau) - (2i + j) N_{-j, j}(\tau) + (i + 2j) N_{-i, i}(\tau)
                        \end{aligned}
                    $$
                By setting $j = 1$ and each $f_r := N_{r, -r}(\tau)$ (for all $r \in \Z \geq 0$, assuming $\tau$ is fixed), we shall then get:
                    $$(i - 1) f_{i + 1} - (i + 2) f_i = -(2i + 1) f_1$$
                for all $i \in \Z_{\geq 2}$. This is a difference equation with initial condition:
                    $$f_0 = N_{0, 0} = 0$$
                coming from the skew-symmetry of $N_{i, j}$. We note also, that knowing $f_1$ and $f_2$ would help us determine $f_{i + 1}$ for all $i \in \Z_{\geq 2}$, so the equation is of order $2$. The vector space of solutions is therefore of dimension $\leq 2$, though in light of the fact that:
                    $$\eta \not = \eta' \in Z^2_{\Lie}(\der(\bbC[v^{\pm 1}]), \bbC c_{\frakv})$$
                we actually have that the solution space is $2$-dimensional, say:
                    $$Z^2_{\Lie}(\der(\bbC[v^{\pm 1}]), \bbC c_{\frakv}) \cong \bbC \eta \oplus \bbC \eta'$$
    
                If we can now show that $\eta$ is \textit{not} a Lie $2$-coboundary of $\der(\bbC[v^{\pm 1}])$ with values in $\bbC c_{\frakv}$, we will have that:
                    $$B^2_{\Lie}(\der(\bbC[v^{\pm 1}]), \bbC c_{\frakv}) \cong \bbC \eta'$$
                from propostion \ref{prop: a_virasoro_coboundary}, and hence that:
                    $$H^2_{\Lie}(\der(\bbC[v^{\pm 1}]), \bbC c_{\frakv}) \cong \bbC \eta$$
                which would give:
                    $$\dim_{\bbC} H^2_{\Lie}(\der(\bbC[v^{\pm 1}]), \bbC c_{\frakv}) = 1$$
                as desired. For this, we shall need to prove that there does \textit{not} exist a Lie linear map $\tilde{\eta}: \der(\bbC[v^{\pm 1}]) \to \bbC c_{\frakv}$ (which is merely a linear map; cf. example \ref{example: low_degree_lie_cocycles_and_coboundaries_with_trivial_coefficients}) such that:
                    $$\eta(d_i, d_j) = \tilde{\eta}([d_i, d_j])$$
                Suppose for the sake of deriving a contradiction, that there does exist such a linear map $\tilde{\eta}$. We would then have that:
                    $$\delta_{i + j, 0} (i^3 - i) c_{\frakv} = \eta(d_i, d_j) = \tilde{\eta}([d_i, d_j]) = (i - j) \tilde{\eta}(d_{i + j}) c_{\frakv}$$
                for all $i, j \in \Z$, which in turn implies that:
                    $$i^3 - i = 2i \tilde{\eta}(d_0)$$
                when $i + j = 0$. By re-arranging, we shall get:
                    $$\tilde{\eta}(d_0) = \frac12(i^2 - 1)$$
                whenever $i \not = 0$, but this implies that $\tilde{\eta}(d_0)$ depends on $i$, which is absurd. We thus have a contradiction, and the linear map $\tilde{\eta}$ therefore can not exist. $\eta$ is thus not $2$-coboundary.
            \end{proof}