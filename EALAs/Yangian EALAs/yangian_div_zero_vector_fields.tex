\section{Constructing Yangian extended toroidal Lie algebras}
    \subsection{Extending \texorpdfstring{$\toroidal$}{} by derivations to fix degeneracy}
        As we know how $\d_{[2]}$ acts on $\g_{[2]}$ (cf. lemma \ref{lemma: derivation_action_on_multiloop_algebras}), it remains to see how it acts on $\z_{[2]}$ to completely determine its action on $\toroidal$. 
        \begin{lemma}[$\d_{[2]}$ acts on $\z_{[2]}$ by Lie derivatives] \label{lemma: derivation_action_on_toroidal_centres}
            Elements of $\d_{[2]}$ act on those of $\z_{[2]}$ as Lie derivatives. This is to say that, the elements $D \in \d_{[2]}$ act on the generating elements $f \bar{d}g \in \z_{[2]}$ (for some $f, g \in A$) in the following manner:
                $$[D, f \bar{d}g]_{\extendedtoroidal} = \xi_D(f) \bar{d}g + f \bar{d}(\xi_D(g))$$
            where $\xi_D \in \der(A)$ is a derivation on $A$ determined uniquely by $A$ (well-defined because $\d_{[2]}$ is a vector subspace of $\der(A)$ per lemma \ref{lemma: derivation_action_on_multiloop_algebras}). In particular, this means that:
                $$[\d_{[2]}, \z_{[2]}]_{\extendedtoroidal} \subseteq \z_{[2]}$$
        \end{lemma}
            \begin{proof}
                Without any loss of generality, let us consider the following for any $h, h' \in \h$ so that\footnote{We can make this assumption because ultimately, elements of $\z_{[2]}$ do not depend on those of $\g$.}:
                    $$(h, h')_{\g} = 1$$
                any $f(v, t), g(v, t) \in A$, and any $D \in \d_{[2]}$:
                    $$[ D, [h f(v, t), h' g(v, t)]_{\toroidal} ]_{\extendedtoroidal} = [ D, f(v, t) \bar{d}( g(v, t) ) ]_{\extendedtoroidal}$$
                At the same time, we have via the Jacobi identity that:
                    $$
                        \begin{aligned}
                            [ D, [h f(v, t), h' g(v, t)]_{\toroidal} ]_{\extendedtoroidal} & = [ h f(v, t), [D, h' g(v, t)]_{\extendedtoroidal} ]_{\toroidal} + [ [D, h f(v, t)]_{\extendedtoroidal}, h' g(v, t) ]_{\toroidal}
                            \\
                            & = [ h f(v, t), h' D( g(v, t) ) ]_{\toroidal} + [ h D( f(v, t) ), h' g(v, t) ]_{\toroidal}
                            \\
                            & = f(v, t) \bar{d}( D( g(v, t) ) ) + D( f(v, t) ) \bar{d}(g(v, t))
                        \end{aligned}
                    $$
                One thus sees that:
                    $$[ D, f(v, t) \bar{d}( g(v, t) ) ]_{\extendedtoroidal} = f(v, t) \bar{d}( D( g(v, t) ) ) + D( f(v, t) ) \bar{d}(g(v, t))$$
                and since the element $f(v, t) \bar{d}( g(v, t) )$ is central (via the map $\e$ mentioned earlier), this gives another description of:
                    $$[ \d_{[2]}, \z_{[2]} ]_{\extendedtoroidal}$$
                With this in mind, we return quickly to lemma \ref{lemma: derivation_action_on_multiloop_algebras}; there, we previously demonstrated that:
                    $$[ \d_{[2]}, \g_{[2]} ]_{\extendedtoroidal} \subseteq \g_{[2]} \oplus \z_{[2]}$$
                but we claim now that the following stronger fact holds:
                    $$[ \d_{[2]}, \g_{[2]} ]_{\extendedtoroidal} \subseteq \g_{[2]}$$
                To see why this is the case, suppose firstly that for any $D \in \d_{[2]}$, any $X := x f(v, t) \in \g_{[2]}$ (for some $f(v, t) \in A$), there is $K(X) \in \z_{[2]}$ depending on $X$ (and indeed, such a $K(X)$ exists by lemma \ref{lemma: derivation_action_on_multiloop_algebras}) such that:
                    $$[ D, X ]_{\extendedtoroidal} = x D( f(v, t) ) + K(X)$$
                Next, pick an arbitrary element $\xi \in \d_{[2]}$ and then consider the following:
                    $$( [ D, X ]_{\extendedtoroidal}, \xi )_{\extendedtoroidal} = (D(X) + K(X), \xi)_{\extendedtoroidal} = (K(X), \xi)_{\extendedtoroidal}$$
                wherein the last equality holds as a consequence of the fact that:
                    $$( \g_{[2]}, \d_{[2]} )_{\extendedtoroidal} = 0$$
                per the construction of the bilinear form $(-, -)_{\extendedtoroidal}$ as in convention \ref{conv: orthogonal_complement_of_toroidal_centres}.
            \end{proof}
        This lemma has two important corollaries. The first one is easy.
        \begin{corollary}
            With respect to the bracket $[-, -]_{\extendedtoroidal}$, the vector subspace $\toroidal$ is actually a Lie ideal of $\extendedtoroidal$.
        \end{corollary}

        We have now reached the following conclusion, as eluded to at the beginning of the subsection:
        \begin{proposition} \label{prop: toroidal_lie_algebras_as_modules_over_vector_field_lie_algebras}
            $\toroidal$ is a $\der(A)$-module.
        \end{proposition}
        We note that even in light of this proposition and the fact that $\extendedtoroidal$ admits $\toroidal$ as a Lie ideal, we still do not yet know enough to conclude that $\extendedtoroidal$ necessarily arises as a Lie algebra extension with kernel $\toroidal$. It remains to investigate how the brackets of the form:
            $$[D, D']_{\extendedtoroidal}$$
        (for some $D, D' \in \d_{[2]}$) are given.

    \subsection{Yangian extended toroidal Lie algebras are twisted semi-direct products}
        Let us now change our perspective on $\d_{[2]}$ slightly. We begin by demonstrating (cf. proposition \ref{prop: lie_bracket_on_orthogonal_complement_of_toroidal_centre} and the corollary that follows) that for any $D, D' \in \d_{[2]}$, there exists some $K(D, D') \in \z_{[2]}$ such that:
            $$[D, D']_{\extendedtoroidal} = [D, D'] + K(D, D')$$
        with $[D, D'] := DD' - D'D$ denoting the usual commutator; this allows us to regard $\d_{[2]}$ as a Lie subalgebra of $\der(A)$. Because we now know from the previous subsection that $\toroidal$ is a $\der(A)$-module, the above implies that $\toroidal$ is also a $\d_{[2]}$-module (cf. proposition \ref{prop: toroidal_lie_algebras_as_modules_over_div_0_vector_field_lie_algebras}), and we are thus able to regard $\extendedtoroidal$ as a Lie algebra extension of $\d_{[2]}$ with kernel $\toroidal$. In fact, according to proposition \ref{prop: twisted_semi_direct_product_criterion}, we have that:
            $$\extendedtoroidal \cong \toroidal \rtimes^{\sigma} \d_{[2]}$$
        for some $2$-cocycle $\sigma: \bigwedge^2 \d_{[2]} \to \t$.
        
        This then begs the question about the (non-)uniqueness of the Lie algebra structure on $\extendedtoroidal$: namely, how many isomorphism classes of Lie algebra structures on $\extendedtoroidal$ are there, i.e. what are the possible cocycles $\sigma$ ? Of course, there is always the semi-direct product, corresponding to:
            $$\sigma = 0$$
        (cf. example \ref{example: lie_algebra_semi_direct_products}). We shall see that, actually, there are two additional isomorphism classes of Lie algebra structures different from the semi-direct product, corresponding to two non-zero $2$-cocycles:
            $$\sigma_1, \sigma_2$$
        (see example \ref{example: non_uniqueness_of_yangian_extended_lie_algebras}), which have been known since \cite{billig_energy_momentum_tensor}. 
        
        \begin{proposition}[How does $\d_{[2]}$ act on itself] \label{prop: lie_bracket_on_orthogonal_complement_of_toroidal_centre}
            Let $\d_{[2]}$ be given as in convention \ref{conv: orthogonal_complement_of_toroidal_centres}. Then:
                $$[ \d_{[2]}, \d_{[2]} ]_{\extendedtoroidal} \subset \z_{[2]} \oplus \d_{[2]}$$
            i.e. the $\g_{[2]}$-summand of any commutator of the kind $[D, D']_{\extendedtoroidal}$ (for any two $D, D' \in \d_{[2]}$) actually vanishes. Furthermore, neither the $\z_{[2]}$- nor the $\d_{[2]}$-summand of those commutators $[D, D']_{\extendedtoroidal}$ necessarily vanish in general. 
        \end{proposition}
            \begin{proof}
                For convenience, we will be abbreviating $\h_{[2]} := \h[v^{\pm}, t^{\pm 1}]$ and $\n^{\pm}_{[2]} := [v^{\pm}, t^{\pm 1}]$, with $\n^{\pm} := \bigoplus_{\alpha \in \Phi^{\pm}} \g_{\alpha}$ being the direct sums of the positive/negative roots spaces of $\g$, as usual.
            
                Pick arbitrary elements $D, D' \in \d_{[2]}$ and set:
                    $$[D, D']_{\extendedtoroidal} := X(D, D') + K(D, D') + \xi(D, D')$$
                for some $X(D, D') \in \g_{[2]}, K(D, D') \in \z_{[2]}$, and $\xi(D, D') \in \d_{[2]}$ depending on $D, D'$. Pick also a test element $y g(v, t) \in \g_{[2]}$, for some arbitrary $y \in \g$ and $g(v, t) \in A$ and set:
                    $$[D, y g(v, t)]_{\extendedtoroidal} := y D( g(v, t) ) + K_{D, Y}$$
                    $$[D', y g(v, t)]_{\extendedtoroidal} := y D'( g(v, t) ) + K_{D', Y}$$
                for some $K_{D, Y} \in \z_{[2]}$ depending on $Y$ (cf. lemma \ref{lemma: derivation_action_on_multiloop_algebras}).
                
                Via the Jacobi identity, we get that:
                    $$
                        \begin{aligned}
                            & [ [D, D']_{\extendedtoroidal}, y g(v, t) ]_{\extendedtoroidal}
                            \\
                            = & [ D, [ D', y g(v, t) ]_{\extendedtoroidal} ]_{\extendedtoroidal} + [ D', [ y g(v, t), D ]_{\extendedtoroidal} ]_{\extendedtoroidal}
                            \\
                            = & [ D, y D'( g(v, t) ) + K_{D', Y} ]_{\extendedtoroidal} - [ D', y D( g(v, t) ) + K_{D, Y} ]_{\extendedtoroidal}
                            \\
                            = & \left( y D( D'(g(v, t)) ) + K_{DD', Y} + [ D, K_{D', Y} ]_{\extendedtoroidal} \right) - \left( y D'( D(g(v, t)) ) + K_{D'D, Y} + [ D', K_{D, Y} ]_{\extendedtoroidal} \right)
                            \\
                            = & y (DD' - D'D)( g(v, t) ) + ( K_{DD', Y} - K_{D'D, Y} ) + ( [ D, K_{D', Y} ]_{\extendedtoroidal} - [ D', K_{D, Y} ]_{\extendedtoroidal} )
                        \end{aligned}
                    $$
                for some $K_{DD', Y}, K_{D'D, Y} \in \z_{[2]}$ such that:
                    $$[ D, y D'( g(v, t) ) ]_{\extendedtoroidal} := y D( D'( g(v, t) ) ) + K_{DD', Y}$$
                    $$[ D', y D( g(v, t) ) ]_{\extendedtoroidal} := y D( D'( g(v, t) ) ) + K_{D'D, Y}$$
                At the same time, we have that:
                    $$
                        \begin{aligned}
                            & [ [D, D']_{\extendedtoroidal}, y g(v, t) ]_{\extendedtoroidal}
                            \\
                            = & [ X(D, D') + K(D, D') + \xi(D, D') , y g(v, t) ]_{\extendedtoroidal}
                            \\
                            = & [ X(D, D') + \xi(D, D') , y g(v, t) ]_{\extendedtoroidal}
                            \\
                            = & [ X(D, D') , y g(v, t) ]_{\extendedtoroidal} + \left( y \xi(D, D')(g(v, t)) + K_{\xi(D, D'), Y} \right)
                        \end{aligned}
                    $$
                wherein the second equality holds thanks to the fact that $[\z_{[2]}, \g_{[2]}]_{\extendedtoroidal} = 0$, and $K_{\xi(D, D'), Y} \in \z_{[2]}$ is some element (cf. lemma \ref{lemma: derivation_action_on_multiloop_algebras}). Combining the two observations together then yields:
                    $$
                        \begin{aligned}
                            & [ X(D, D') , y g(v, t) ]_{\extendedtoroidal} + \left( y \xi(D, D')(g(v, t)) + K_{\xi(D, D'), Y} \right)
                            \\
                            = & y (DD' - D'D)( g(v, t) ) + ( K_{DD', Y} - K_{D'D, Y} ) + ( [ D, K_{D', Y} ]_{\extendedtoroidal} - [ D', K_{D, Y} ]_{\extendedtoroidal} )
                        \end{aligned}
                    $$
                There exists $K_{X(D, D'), Y} \in \z_{[2]}$ such that:
                    $$[ X(D, D') , y g(v, t) ]_{\extendedtoroidal} = [ X(D, D') , Y ]_{\extendedtoroidal} = [X(D, D'), Y]_{\g_{[2]}} + K_{X(D, D'), Y}$$
                using which we can write:
                    $$
                        \begin{aligned}
                            & [X(D, D'), Y]_{\g_{[2]}} - y \left( ( DD' - D'D) - \xi(D, D') \right)( g(v, t) )
                            \\
                            = & \left( [ D, K_{D', Y} ]_{\extendedtoroidal} - [ D', K_{D, Y} ]_{\extendedtoroidal} \right) - \left( K_{X(D, D'), Y} + K_{\xi(D, D'), Y} \right)
                        \end{aligned}
                    $$
                    
                We note right away that the LHS lies entirely in $\g_{[2]}$, whereas the RHS is an element of $\z_{[2]}$ due to the fact that $[\d_{[2]}, \z_{[2]}]_{\extendedtoroidal} \subseteq \z_{[2]}$ (cf. lemma \ref{lemma: derivation_action_on_toroidal_centres}), which tells us that $[ D, K_{D', Y} ]_{\extendedtoroidal}, [ D', K_{D, Y} ]_{\extendedtoroidal} \in \z_{[2]}$ in particular. Because $\g_{[2]}$ is centreless (as $\g$ is simple and the Lie bracket on $\g_{[2]}$ is given by extension of scalars), this observation subsequently implies that the LHS must vanish, i.e.:
                    $$[X(D, D'), Y]_{\g_{[2]}} - y \left( ( DD' - D'D) - \xi(D, D') \right)( g(v, t) ) = 0$$
                Because we have by construction that:
                    $$DD' - D'D - \xi(D, D') \in \d_{[2]}$$
                we now make the following claim: \textit{if we fix some arbitrary $E \in \g_{[2]}$ and some $P \in \d_{[2]}$ then:}
                    $$\forall H := h \varphi \in \g_{[2]}: [E, H]_{\g_{[2]}} = h P( \varphi ) \implies E = 0$$

                Using the root space decomposition for $\g$, we see that if $h \in \h$ then we then will have that $[E, H]_{\g_{[2]}} \in \n^{\pm}_{[2]}$, but at the same time, that $h P(\varphi) \in \h_{[2]}$. The only way for this to be true is that $[E, H]_{\g_{[2]}} = 0$, which is the case if and only if $E = 0$. If $h \in \n^{\pm}$, then $[E, H]_{\g_{[2]}} \in \n^{\pm}_{[2]} \oplus \h_{[2]}$ and the $\h_{[2]}$-summand will be non-zero in general; at the same time, $h P(\varphi) \in \n^{\pm}_{[2]}$ in this case, and again, the only way for these to facts to be true simultaneously is that $E = 0$ necessarily. 

                Apply the claim to the fact that:
                    $$[X(D, D'), Y]_{\g_{[2]}} = y \left( ( DD' - D'D) - \xi(D, D') \right)( g(v, t) )$$
                - and again, note that $( DD' - D'D) - \xi(D, D') \in \d_{[2]}$ - then yields:
                    $$X(D, D') = 0$$
                precisely as desired. 
            \end{proof}
        \begin{corollary}
            For any $D, D' \in \d_{[2]}$, the $\d_{[2]}$-summand of $[D, D']_{\extendedtoroidal}$ is nothing but the commutator $DD' - D'D$.
        \end{corollary} 
        \todo[inline]{Fixed statement and proof of the fact that $[\d_{[2]}, \g_{[2]}]_{\extendedtoroidal} \subseteq \g_{[2]}$.}
        \begin{corollary}[\texorpdfstring{$\z_{[2]}$}{}-summands of elements of \texorpdfstring{$[\d_{[2]}, \g_{[2]}]_{\extendedtoroidal}$}{}] \label{coro: derivation_action_on_multiloop_algebras}
            The action of $\d_{[2]}$ on $\g_{[2]}$ as in lemma \ref{lemma: derivation_action_on_multiloop_algebras} satisfies:
                $$[\d_{[2]}, \g_{[2]}]_{\extendedtoroidal} \subseteq \g_{[2]}$$
            From this, one can also infer that $\toroidal \cong \g_{[2]} \oplus \z_{[2]}$ is a direct sum of $\der(A)$-modules. 
        \end{corollary}
            \begin{proof}
                From lemma \ref{lemma: derivation_action_on_multiloop_algebras}, we know that given some $D \in \d_{[2]}$ and some $x \in \g$ and $f \in A$, there shall exist $K(D, xf) \in \z_{[2]}$ (depending on the choices of $D$ and $x, f$) such that:
                    $$[D, xf]_{\extendedtoroidal} = x D(f) + K(D, xf)$$
                Next, consider the following:
                    $$( \d_{[2]}, x D(f) + K(D, xf) )_{\extendedtoroidal} = ( \d_{[2]}, [D, xf]_{\extendedtoroidal} )_{\extendedtoroidal} = ( [\d_{[2]}, D]_{\extendedtoroidal}, xf )_{\extendedtoroidal} = 0$$
                where the second equality holds thanks to invariance, and the third equality holds due to a combination of the fact that $[\d_{[2]}, \d_{[2]}]_{\extendedtoroidal} \subset \z_{[2]} \oplus \d_{[2]}$ (cf. proposition \ref{prop: lie_bracket_on_orthogonal_complement_of_toroidal_centre}) and the fact that $(\z_{[2]} \oplus \d_{[2]}, \g_{[2]})_{\extendedtoroidal}$ per the construction of the bilinear form $(-, -)_{\extendedtoroidal}$ (cf. convention \ref{conv: orthogonal_complement_of_toroidal_centres}). We also have the following, again per the construction of the bilinear form $(-, -)_{\extendedtoroidal}$:
                    $$( \d_{[2]}, x D(f) + K(D, xf) )_{\extendedtoroidal} = ( \d_{[2]}, K(D, xf) )_{\extendedtoroidal}$$
                which means that:
                    $$( \d_{[2]}, K(D, xf) )_{\extendedtoroidal} = 0$$
                The non-degeneracy of $(-, -)_{\extendedtoroidal}$ (or more particularly, the fact that $( \d_{[2]}, \z_{[2]} )_{\extendedtoroidal} \not = 0$) then implies that:
                    $$K(D, xf) = 0$$
                necessarily. This means that, indeed, we have that:
                    $$[D, xf]_{\extendedtoroidal} = x D(f)$$
                for all $D \in \d_{[2]}$ and all $x \in \g$ and all $f \in A$. Since $\g_{[2]}$ is generated by elements of the form $xf$, this implies that:
                    $$[\d_{[2]}, \g_{[2]}]_{\extendedtoroidal} \subseteq \g_{[2]}$$
                as claimed. 
            \end{proof}
        
        In light of proposition \ref{prop: lie_bracket_on_orthogonal_complement_of_toroidal_centre}, it is also valuable to know how the $\d_{[2]}$-summand of the commutators:
            $$[D, D']_{\extendedtoroidal}$$
        are given explicitly. Later on, these computations will be used to prove that the centre of $\extendedtoroidal$ is $2$-dimensional, namely given by $\bbC c_v \oplus \bbC c_v$.
        \begin{lemma}[Explicitly commutators between basis elements of $\d_{[2]}$] \label{lemma: explicit_commutators_between_basis_elements_of_toroidal_central_orthogonal_complement}
            The usual commutator (i.e. $[D, D'] := DD' - D'D$) between the basis elements $D_{r, s}, D_v, D_t \in \d_{[2]}$ are given as follows:
                $$[D_v, D_t] = 0$$
                $$[D_v, D_{r, s}] = r D_{r, s + 1}$$
                $$[D_t, D_{r, s}] = D_{r, s + 1}$$
                $$[D_{a, b}, D_{r, s}] = (br - sa) D_{a + r, b + s + 1}$$
        \end{lemma}
            \begin{proof}
                \begin{enumerate}
                    \item Since we know that:
                        $$D_v = -vt^{-1} \del_v, D_t = -\del_t$$
                    (cf. lemma \ref{lemma: derivation_action_on_multiloop_algebras}), it is therefore trivial that:
                        $$[D_v, D_t] = 0$$
                    \item From lemma \ref{lemma: derivation_action_on_multiloop_algebras}, we know that:
                        $$D_v(v^m t^p) = -m v^m t^{p - 1}$$
                        $$D_{r, s}(v^m t^p) = ( ms - rp ) v^{m - r} t^{p - s - 1}$$
                    From this, we infer that:
                        $$
                            \begin{aligned}
                                [D_v, D_{r, s}](v^m t^p) & = D_v( D_{r, s}(v^m t^p) ) - D_{r, s}( D_v(v^m t^p) )
                                \\
                                & = (ms - rp) D_v( v^{m - r} t^{p - s - 1} ) + m D_{r, s}( v^m t^{p - 1} )
                                \\
                                & = -(m - r)(ms - rp) v^{m - r} t^{p - s - 2} + (ms - r(p - 1)) m v^{m - r} t^{p - s - 2}
                                \\
                                & = r(m(s + 1) - rp) v^{m - r} t^{p - (s + 1) - 1}
                                \\
                                & = r D_{r, s + 1}(v^m t^p)
                            \end{aligned}
                        $$
                    and hence:
                        $$[D_v, D_{r, s}] = r D_{r, s + 1}$$
                    \item Likewise, we can show that:
                        $$[D_t, D_{r, s}] = D_{r, s + 1}$$
                    \item Lastly, we can also show, using a completely analogous argument, that:
                        $$[D_{a, b}, D_{r, s}] = (br - sa) D_{a + r, b + s + 1}$$
                \end{enumerate}
            \end{proof}
        By exploiting invariance again, we are able to obtain the following corollary to the lemma above, concerning commutators between elements of $\z_{[2]}$ and those of $\d_{[2]}$ (which are already known to be elements of $\z_{[2]}$; cf. lemma \ref{lemma: derivation_action_on_toroidal_centres}). Due to its usefulness, we nevertheless grant it lemma-hood.
        \begin{lemma}[Explicit commutators between basis elements of $\d_{[2]}$ and $\z_{[2]}$] \label{lemma: explicit_commutators_between_central_basis_elements_and_derivations}
            In the Lie algebra $\extendedtoroidal$, one has the following non-trivial relations between elements of $\z_{[2]}$ and those of $\d_{[2]}$:
                $$
                    \forall (a, b) \in \Z^2: [D, K_{a, b}]_{\extendedtoroidal} =
                    \begin{cases}
                        \text{$((b - 1)r - sa) D_{a - r, b - s - 1}$ if $D = D_{r, s}$}
                        \\
                        \text{$-r K_{a, b - 1}$ if $D = D_v$}
                        \\
                        \text{$- D_{a, b - 1}$ if $D = D_t$}
                    \end{cases}
                $$
                $$[\d_{[2]}, c_v]_{\extendedtoroidal} = [\d_{[2]}, c_t]_{\extendedtoroidal} = 0 = 0$$
        \end{lemma}
            \begin{proof}
                For a moment, consider two arbitrary derivations $D', D \in \d_{[2]}$ along with an arbitrary element $K \in \z_{[2]}$. Suppose also that:
                    $$[D', D]_{\extendedtoroidal} = [D', D] + K(D', D)$$
                for some $K(D', D) \in \z_{[2]}$ (cf. proposition \ref{prop: lie_bracket_on_orthogonal_complement_of_toroidal_centre}). Using the invariance property of the bilinear form $(-, -)_{\extendedtoroidal}$ as well as the fact that:
                    $$(\z_{[2]}, \z_{[2]})_{\extendedtoroidal} = 0$$
                per the construction of $(-, -)_{\extendedtoroidal}$ (cf. convention \ref{conv: orthogonal_complement_of_toroidal_centres}), one gets:
                    $$([D', D], K)_{\extendedtoroidal} = ([D', D] + K(D', D), K)_{\extendedtoroidal} = ([D', D]_{\extendedtoroidal}, K)_{\extendedtoroidal} = (D', [D, K]_{\extendedtoroidal})_{\extendedtoroidal}$$
                Since we know how the ordinary commutators:
                    $$[D', D]$$
                are given and particularly, how if $D', D \in \d_{[2]}$ are basis elements then their commutator $[D', D]$ will be in the span of a single basis element of $\d_{[2]}$ (cf. lemma \ref{lemma: explicit_commutators_between_basis_elements_of_toroidal_central_orthogonal_complement}), as well as how $\d_{[2]}$ is the orthogonal complement of $\z_{[2]}$ with respect to $(-, -)_{\extendedtoroidal}$ by construction, it remains now to simply specialise to the case wherein $K$ is some basis element of $\z_{[2]}$. 
                \begin{enumerate}
                    \item If:
                        $$K = K_{a, b}$$
                    for some fixed $(a, b) \in \Z^2$ then:
                        $$([D', D], K_{a, b})_{\extendedtoroidal} = 1 \iff [D', D] = D_{a, b}$$
                    Hence, we have that:
                        $$1 = (D_{a, b}, K_{a, b})_{\extendedtoroidal} = ([D', D], K_{a, b})_{\extendedtoroidal} = (D', [D, K_{a, b}]_{\extendedtoroidal})_{\extendedtoroidal}$$
                    Using the commutators computed in lemma \ref{lemma: explicit_commutators_between_basis_elements_of_toroidal_central_orthogonal_complement}, we then see that:
                        $$
                            [D, K_{a, b}]_{\extendedtoroidal} =
                            \begin{cases}
                                \text{$((b - 1)r - sa) D_{a - r, b - s - 1}$ if $D = D_{r, s}$}
                                \\
                                \text{$-r K_{a, b - 1}$ if $D = D_v$}
                                \\
                                \text{$- D_{a, b - 1}$ if $D = D_t$}
                            \end{cases}
                        $$
                    \item If:
                        $$K = c_v$$
                    then we have that:
                        $$([D', D], c_v)_{\extendedtoroidal} = 1 \iff [D', D] = D_v$$
                    which in turn implies that:
                        $$1 = (D_v, K_v)_{\extendedtoroidal} = ([D', D], c_v)_{\extendedtoroidal} = (D', [D, c_v]_{\extendedtoroidal})_{\extendedtoroidal}$$
                    Once again, by using the commutators computed in lemma \ref{lemma: explicit_commutators_between_basis_elements_of_toroidal_central_orthogonal_complement}, we then see that:
                        $$\forall D \in \{D_{r, s}\}_{(r, s) \in \Z^2} \cup \{D_v, D_t\}: [D, c_v]_{\extendedtoroidal} = 0$$
                    and since the set $\{D_{r, s}\}_{(r, s) \in \Z^2} \cup \{D_v, D_t\}$ is a basis for $\d_{[2]}$, one thus has that:
                        $$[\d_{[2]}, c_v]_{\extendedtoroidal} = 0$$
                    as none of said commutators are elements of $\bbC D_v$.
                    \item Likewise, one can show that:
                        $$[\d_{[2]}, c_t]_{\extendedtoroidal} = 0$$
                \end{enumerate}
            \end{proof}
        We will also see that in fact, the centre of $\extendedtoroidal$ is actually just spanend by $c_v$ and $c_t$. See proposition \ref{prop: centres_of_yangian_extended_toroidal_lie_algebras} and the discussion preceding it for more details.
        
        \begin{proposition}[$\toroidal$ as a $\d_{[2]}$-module] \label{prop: toroidal_lie_algebras_as_modules_over_div_0_vector_field_lie_algebras}
            If we regard $\d_{[2]}$ as a Lie subalgebra of $\der(A)$ with the usual commutator bracket, then $\toroidal$ will be a $\d_{[2]}$-module, not just a $\der(A)$-module.
        \end{proposition}
        
        In summary, we have yielded the following result, which is a more explicit version of theorem \ref{theorem: yangian_extended_toroidal_lie_algebras_preliminary_version} that was stated at the beginning of the section:
        \begin{theorem}[Yangian extended toroidal Lie algebras] \label{theorem: yangian_extended_toroidal_lie_algebras}
            Let $\d_{[2]}$ be equipped with the commutator bracket inherited from $\der(A)$.
        
            When endowed with the Lie bracket $[-, -]_{\extendedtoroidal}$\footnote{... which ultimately is specified by the Lie structure on $\toroidal$ and the non-degenerate and invariant symmetric bilinear form $(-, -)_{\extendedtoroidal}$ as in convention \ref{conv: orthogonal_complement_of_toroidal_centres}.} (as determined in lemmas \ref{lemma: derivation_action_on_multiloop_algebras} and \ref{lemma: derivation_action_on_toroidal_centres}, and proposition \ref{prop: lie_bracket_on_orthogonal_complement_of_toroidal_centre}), the vector space $\extendedtoroidal := \toroidal \oplus \d_{[2]}$ becomes a Lie algebra extension:
                $$0 \to \toroidal \to \extendedtoroidal \to \d_{[2]} \to 0$$
            of $\d_{[2]}$ by $\toroidal$. 
        \end{theorem}
        \begin{corollary}[Yangian extended toroidal Lie algebras are twisted semi-direct products] \label{coro: yangian_extended_toroidal_lie_algebras_are_twisted_semi_direct_products}
            Let $\d_{[2]}$ be equipped with the commutator bracket inherited from $\der(A)$.
        
            Because the bilinear form $(-, -)_{\extendedtoroidal}$ is invariant with respect to $[-, -]_{\extendedtoroidal}$ by construction, $\extendedtoroidal$ is therefore a Yangian extended toroidal Lie algebra in the sense of definition \ref{def: yangian_extended_toroidal_lie_algebras}.

            Also, because $\toroidal$ is a $\d_{[2]}$-module, we have that:
                $$\extendedtoroidal \cong \toroidal \rtimes^{\sigma} \d_{[2]}$$
            for some $2$-cocycle:
                $$\sigma: \bigwedge^2 \d_{[2]} \to \toroidal$$
        \end{corollary}
        \begin{remark}
            Because we know that the $\d_{[2]}$-module decomposes into submodules as:
                $$\toroidal \cong \g_{[2]} \oplus \z_{[2]}$$
            and that:
                $$[\d_{[2]}, \d_{[2]}]_{\extendedtoroidal} \subseteq \d_{[2]} \oplus \z_{[2]}$$
            the codomain of the cocycle $\sigma$ from corollary \ref{coro: yangian_extended_toroidal_lie_algebras_are_twisted_semi_direct_products} can be restricted down to $\z_{[2]}$, which has the advantage of being abelian, unlike $\toroidal$. Phrased differently, the semi-direct products:
                $$\extendedtoroidal \cong \toroidal \rtimes^{\sigma} \d_{[2]}$$
            actually arise from semi-direct products $\z_{[2]} \rtimes^{\sigma} \d_{[2]}$.  
        \end{remark}
        
        \begin{remark}
            We do not yet know whether or not $\extendedtoroidal$ is unique (up to isomorphisms) as a Lie algebra extension of $\d_{[2]}$ by $\toroidal$, as we do not know the $2$-cocycles:
                $$\sigma: \bigwedge^2 \d_{[2]} \to \z_{[2]}$$
            for which we have that:
                $$\extendedtoroidal \cong \toroidal \rtimes^{\sigma} \d_{[2]}$$
            but merely that such $2$-cocycles exist. 
        \end{remark}

    \subsection{The centre of \texorpdfstring{$\extendedtoroidal$}{}}
        Let us conclude this section with the following question, which is natural now that we have a solid handle on how the Lie bracket on $\extendedtoroidal$ is given:
        \begin{question}
            What is the centre $\hat{\z}_{[2]} := \z( \extendedtoroidal )$ ? This ought to be smaller than $\z_{[2]}$ somehow, since elements of $\z_{[2]}$ need not be central in $\extendedtoroidal$. 
        \end{question}
        \begin{remark}[Computing the centre without computing all the brackets ...]
            Since $\g_{[2]}$ is centreless, we have that:
                $$\hat{\z}_{[2]} = \z( \z_{[2]} \oplus \d_{[2]} )$$
            As $\z_{[2]}$ is an abelian Lie algebra, this implies that in order to compute $\hat{\z}_{[2]}$, it suffices to explicitly compute the commutators of the form:
                $$[D, K]_{\extendedtoroidal}, [D, D']_{\extendedtoroidal}$$
            for $D, D' \in \d_{[2]}$ and $K \in \z_{[2]}$, to see which ones vanish. However, this is rather tedious and not very insightful.
            
            An alternative method is as follows: exploiting the fact that the symmetric bilinear form $(-, -)_{\extendedtoroidal}$ is both invariant and non-degenerate (by construction; cf. convention \ref{conv: orthogonal_complement_of_toroidal_centres}), we can characterise the centre $\hat{\z}_{[2]}$ as the Lie ideal of $\extendedtoroidal$ containing elements $Z$ such that:
                $$0 = ([Z, X]_{\extendedtoroidal}, Y)_{\extendedtoroidal} = (Z, [X, Y]_{\extendedtoroidal})_{\extendedtoroidal}$$
            for any $X, Y \in \extendedtoroidal$, with the first equality holding thanks to the fact that $Z$ is supposed to commute with every other element of $\extendedtoroidal$ by assumption of being central. We are thus left with the task of finding elements:
                $$Z \in \extendedtoroidal$$
            such that:
                $$(Z, [\extendedtoroidal, \extendedtoroidal]_{\extendedtoroidal})_{\extendedtoroidal} = 0$$
            Since brackets of the form:
                $$[X, Y]_{\extendedtoroidal}, [D, D']_{\extendedtoroidal}$$
            (for some $X, Y \in \g_{[2]}$ and some $D, D' \in \d_{[2]}$) are generally non-zero, their elements can not be central in $\extendedtoroidal$. As such, we have narrowed the scope of our search down to:
                $$\hat{\z}_{[2]} \subset \z_{[2]}$$

            Another way to see that:
                $$\hat{\z}_{[2]} \subset \z_{[2]}$$
            is to use the fact that $\extendedtoroidal$ is a Lie algebra extension of $\d_{[2]}$ by $\toroidal$ (cf. theorem \ref{theorem: yangian_extended_toroidal_lie_algebras}). This tells us that the centre of $\extendedtoroidal$ ought to lie inside that of $\toroidal$, i.e.:
                $$\hat{\z}_{[2]} \subset \z(\toroidal) = \z_{[2]}$$
            as per proposition \ref{prop: twisted_semi_direct_product_criterion}.
        \end{remark}
        \begin{proposition}[Centres of Yangian extended toroidal Lie algebras] \label{prop: centres_of_yangian_extended_toroidal_lie_algebras}
            The centre $\hat{\z}_{[2]}$ is a two-dimensional (abelian) Lie subalgebra of $\z_{[2]}$, spanned by $c_v$ and $c_t$. 
        \end{proposition}
            \begin{proof}
                Since we know that:
                    $$\hat{\z}_{[2]} \subset \z_{[2]}$$
                and that the only possibly non-zero bracket with elements of $\z_{[2]}$ are elements of $[\d_{[2]}, \z_{[2]}]_{\extendedtoroidal}$, and since we also know from lemma \ref{lemma: explicit_commutators_between_central_basis_elements_and_derivations} that:
                    $$[\d_{[2]}, K]_{\extendedtoroidal} = 0 \iff K \in \bbC c_v \oplus \bbC c_v$$
                we can conclude immediately that:
                    $$\hat{\z}_{[2]} = \bbC c_v \oplus \bbC c_t$$
            \end{proof}
        \begin{remark}
            It is rather interesting that:
                $$\hat{\z}_{[2]} \cong \bbC c_v \oplus \bbC c_t$$
            as this is in good analogy with the affine Kac-Moody case, where the centre of $\hat{\g}$ is $1$-dimensional, namely spanned by $c_v$ (cf. example \ref{example: affine_lie_algebras_centres}).
        \end{remark}