\section{Construction Yangian extended toroidal Lie algebras}
    \subsection{Extending \texorpdfstring{$\toroidal$}{} by derivations to fix degeneracy}
        The point to constructing the so-called \say{Yangian extended affine Lie algebras} is to fix the issue, whereby any \textit{invariant}\footnote{Let $\a$ be a Lie algebra with non-zerfo centre and $(-, -)$ be such an invariant symmetric bilinear form thereon. Then, for all $x, y \in \a$ and all $K \in \z(\a)$, we will have that $(K, [x, y]) = ([K, x], y) = (0, y) = 0$, thus proving that $(-, -)$ is degenerate.} symmetric bilinear form on:
            $$\toroidal:= \uce(\g_{[2]})$$
        is necessarily degenerate. We do this by formally introducing a \say{complementary} vector space $\d_{[2]}$ whose elements shall pair non-degenerately with those of $\z_{[2]}$. We will see (cf. lemma \ref{lemma: derivation_action_on_multiloop_algebras}) that these complementary elements are in fact certain $k$-linear derivations on $A$, which then allows us to show that $\extendedtoroidal$ is a $\der_k(A)$-module (decomposing into a direct sum of the submodules $\g_{[2]}$ and $\z_{[2]}$; see lemmas \ref{lemma: derivation_action_on_multiloop_algebras} and \ref{lemma: derivation_action_on_toroidal_centres}, respectively).
        
        \begin{convention} \label{conv: orthogonal_complement_of_toroidal_centres}
            From now on, $\d_{[2]}$ shall be the $k$-vector space:
                $$\d_{[2]} \cong ( \bigoplus_{(r, s) \in \Z^2} k D_{r, s} ) \oplus k D_v \oplus k D_t$$
            such that we can endow:
                $$\extendedtoroidal := \toroidal\oplus \d_{[2]}$$
            with a $k$-bilinear form $(-, -)_{\extendedtoroidal}$ such that:
            \begin{itemize}
                \item the elements $D_{r, s}, D_v, D_t$ are graded-dual with respect to $(-, -)_{\extendedtoroidal}$ to the elements $K_{r, s}, c_v, c_t$, respectively;
                \item $(\g_{[2]}, \z_{[2]} \oplus \d_{[2]})_{\extendedtoroidal} := 0$;
                \item $(\z_{[2]}, \z_{[2]})_{\extendedtoroidal} = (\d_{[2]}, \d_{[2]})_{\extendedtoroidal} := 0$;
                \item $(-, -)_{\extendedtoroidal}|_{\Sym^2_{k}(\g_{[2]})} := (-, -)_{\g_{[2]}}$
            \end{itemize}
        \end{convention}

        We now seek to construct a Lie bracket:
            $$[-, -]_{\extendedtoroidal}$$
        on the vector space $\extendedtoroidal := \toroidal \oplus \d_{[2]}$ such that:
        \begin{itemize}
            \item $(-, -)_{\extendedtoroidal}$ is invariant with respect to $[-, -]_{\extendedtoroidal}$, and
            \item $\toroidal := \uce(\g_{[2]})$ with its Lie bracket as in theorem \ref{theorem: kassel_realisation} embeds naturally as a Lie subalgebra into $\extendedtoroidal$ with the bracket $[-, -]_{\extendedtoroidal}$. 
        \end{itemize}
        We will do this by first identifying elements of $\d_{[2]}$ as certain derivations on $A$ (prompted by the fact that $\d_{[2]}$ is dual to $\z_{[2]}$, whose elements are differential $1$-forms of a certain kind), which will be done by exploiting the invariance of the bilinear form $(-, -)_{\extendedtoroidal}$ (see lemma \ref{lemma: derivation_action_on_multiloop_algebras}). Afterwards, we shall see that $\d_{[2]}$ is actually a Lie subalgebra of the Lie algebra $\der_k(A)$ of derivations on $A$ (with its usual commutator bracket), and we will end the subsection with the conclusion that $\toroidal$ is thus naturally a $\der_k(A)$-module (see proposition \ref{prop: toroidal_lie_algebras_as_modules_over_vector_field_lie_algebras}). In fact, it is a $\d_{[2]}$-module, and $\extendedtoroidal$ is a Lie algebra extension of $\d_{[2]}$ by $\toroidal$, but this is a subtler point and thus we will dedicate the next subsection to discussing it.  

        Let us firstly see how elements of $\d_{[2]}$ might act on those of $\g_{[2]}$, with respect to some Lie bracket $[-, -]_{\extendedtoroidal}$. \todo{Previously a remark.}
        \begin{lemma}[$\d_{[2]}$ acts on $\g_{[2]}$ by derivations] \label{lemma: derivation_action_on_multiloop_algebras}
            The elements $D \in \d_{[2]}$ act as derivations on $A$, in the sense that:
                $$[D, xf]_{\extendedtoroidal} = x \xi_D(f) + K_{D, xf}$$
            where $\xi_D \in \der_k(A)$ is a derivation on $A$ that uniquely depends on the choice of $D \in \d_{[2]}$ and $K_{D, xf} \in \z_{[2]}$ is yet to be determined explicitly (see corollary \ref{coro: derivation_action_on_multiloop_algebras}, where it is shown that in fact, these elements of $\z_{[2]}$ are necessarily $0$). In particular, the basis elements $D_{r, s}$ (for any $(r, s) \in \Z^2$) and $D_v, D_t$ of $\d_{[2]}$ act as follows on the generating elements $x v^m t^p \in \g_{[2]}$ (for some $x \in \g$ and $(m, p) \in \Z^2$), respectively:
                $$[D_{r, s}, x v^m t^p]_{\extendedtoroidal} = ( rp - ms ) x v^{m - r} t^{p - s - 1} + K_{(m, p), (r, s)}(x)$$
                $$[D_v, x v^m t^p]_{\extendedtoroidal} = -m x v^m t^{p - 1} + K_{m, p}(x)$$
                $$[D_t, x v^m t^p]_{\extendedtoroidal} = -p x v^m t^{p - 1} + K_{m, p}(x)$$
            where $K_{(m, p), (r, s)}(x), K_{m, p}(x), K_{m, p}(x) \in \z_{[2]}$ are the undetermined summands.
        \end{lemma}
            \begin{proof}
                Let us first see how the basis elements of $\d_{[2]}$ act on those of $\g_{[2]}$ before worrying about whether or not elements of the former are indeed derivations.
            
                We begin by fixing $x, y \in \g$, $(m, p), (n, q) \in \Z^2$, along with some $D \in \d_{[2]}$. Then, consider the following:
                    $$
                        \begin{aligned}
                            ( D, [x v^m t^p, y v^n t^q]_{\toroidal} )_{\extendedtoroidal} & = ( D, [x, y]_{\g} v^{m + n} t^{p + q} + (x, y)_{\g} v^n t^q \bar{d}( v^m t^p ) )_{\extendedtoroidal}
                            \\
                            & = (x, y)_{\g} ( D, v^n t^q \bar{d}( v^m t^p ) )_{\extendedtoroidal}
                            \\
                            & = (x, y)_{\g} ( D, \delta_{(m, p) + (n, q), (0, 0)} ( n c_v + q c_t ) + (np - mq) K_{m + n, p + q} )_{\extendedtoroidal}
                        \end{aligned}
                    $$
                Now, without any loss of generality, let us suppose that $D \in \d_{[2]}$ is some basis element, i.e.:
                    $$D \in \{ D_{r, s}, D_v, D_t \}$$
                and consider these cases separately, for the sake of clarity:
                \begin{enumerate}
                    \item \textbf{(Case 1: $D := D_{r, s}$):} Fix some $(r, s) \in \Z^2$ and consider the following: 
                        $$
                            \begin{aligned}
                                ( D_{r, s}, [x v^m t^p, y v^n t^q]_{\toroidal} )_{\extendedtoroidal} & = (x, y)_{\g} ( D_{r, s}, \delta_{(m, p) + (n, q), (0, 0)} ( n c_v + q c_t ) + (np - mq) K_{m + n, p + q} )_{\extendedtoroidal}
                                \\
                                & = (x, y)_{\g} (np - mq) \delta_{(r, s), (m + n, p + q)}
                            \end{aligned}
                        $$
                    The assumption that $(-, -)_{\extendedtoroidal}$ is invariant with respect to $[-, -]_{\extendedtoroidal}$ then implies that:
                        $$( [D_{r, s}, x v^m t^p]_{\extendedtoroidal}, y v^n t^q )_{\extendedtoroidal} = (x, y)_{\g} (np - mq) \delta_{(r, s), (m + n, p + q)}$$
                    Now, suppose that:
                        $$[D_{r, s}, x v^m t^p]_{\extendedtoroidal} := \sum_{(a, b) \in \Z^2} \lambda_{a, b}(x) v^a t^b + K_{(m, p), (r, s)}(x) + \xi_{(m, p), (r, s)}(x)$$
                    for some $\lambda_{a, b}(x) \in \g$, $K_{(m, p), (r, s)}(x) \in \z_{[2]}$, and $\xi_{(m, p), (r, s)}(x) \in \d_{[2]}$, depending on our choices of $x \in \g$ and $(m, p) \in \Z^2$. Next, consider the following:
                        $$
                            \begin{aligned}
                                ( [D_v, x v^m t^p]_{\extendedtoroidal}, y v^n t^q )_{\extendedtoroidal} & = \left( \sum_{(a, b) \in \Z^2} \lambda_{a, b}(x) v^a t^b + K_{(m, p), (r, s)}(x) + \xi_{(m, p), (r, s)}(x), y v^n t^q \right)_{\extendedtoroidal}
                                \\
                                & = \sum_{(a, b) \in \Z^2} \left( \lambda_{a, b}(x) v^a t^b, y v^n t^q \right)_{\g_{[2]}}
                                \\
                                & = -\sum_{(a, b) \in \Z^2} (\lambda_{a, b}(x), y)_{\g} \delta_{ (a, b) + (n, q), (0, -1) }
                                \\
                                & = -(\lambda_{-n, -q - 1}(x), y)_{\g}
                            \end{aligned}
                        $$
                    which tells us that:
                        $$(x, y)_{\g} (np - mq) \delta_{(r, s), (m + n, p + q)} = -(\lambda_{-n, -q - 1}(x), y)_{\g}$$
                    The non-degeneracy of the inner product $(-, -)_{\g}$ as well as the arbitrariness of the choices of $y \in \g$ and $(n, q) \in \Z^2$ then together imply that:
                        $$\lambda_{-n, -q - 1}(x) = -(np - mq) \delta_{(r, s), (m + n, p + q)} = (mq - np) \delta_{(r, s), (m + n, p + q)}$$
                    for any fixed choices of $x \in \g$ and $(m, p) \in \Z^2$. From this, we infer that:
                        $$
                            \begin{aligned}
                                [D_{r, s}, x v^m t^p]_{\extendedtoroidal} & = \sum_{(n, q) \in \Z^2} -(np - mq) \delta_{(r, s), (m + n, p + q)} v^{-n} t^{-q - 1} + K_{(m, p), (r, s)}(x) + \xi_{(m, p), (r, s)}(x)
                                \\
                                & = ( m(s - p) - (r - m)p ) x v^{m - r} t^{p - s - 1} + K_{(m, p), (r, s)}(x) + \xi_{(m, p), (r, s)}(x)
                                \\
                                & = ( ms - rp ) x v^{m - r} t^{p - s - 1} + K_{(m, p), (r, s)}(x) + \xi_{(m, p), (r, s)}(x)
                            \end{aligned}
                        $$
                        
                    We now claim that:
                        $$\xi_{(m, p), (r, s)}(x) = 0$$
                    To this end, consider firstly the following, wherein $Z \in \z_{[2]}$ is an arbitrary choice:
                        $$
                            \begin{aligned}
                                ( [D_{r, s}, x v^m t^p]_{\extendedtoroidal}, Z )_{\extendedtoroidal} & = ( D_{r, s}, [x v^m t^p, Z]_{\toroidal} )_{\extendedtoroidal}
                                \\
                                & = (D, 0)_{\extendedtoroidal}
                                \\
                                & = 0
                            \end{aligned}
                        $$
                    Simultaneously, consider the following:
                        $$
                            \begin{aligned}
                                ( [D_{r, s}, x v^m t^p]_{\extendedtoroidal}, Z )_{\extendedtoroidal} & = \left( \sum_{(a, b) \in \Z^2} \lambda_{a, b}(x) v^a t^b + K_{(m, p), (r, s)}(x) + \xi_{(m, p), (r, s)}(x), Z \right)_{\extendedtoroidal}
                                \\
                                & = ( \xi_{(m, p), (r, s)}(x), Z )_{\extendedtoroidal}
                            \end{aligned}
                        $$
                    The previous observation along with this one imply that:
                        $$( \xi_{(m, p), (r, s)}(x), Z )_{\extendedtoroidal} = 0$$
                    for \textit{any} $Z \in \z_{[2]}$, but since $\d_{[2]}$ is graded-dual to $\z_{[2]}$ by construction, the above implies via the non-degeneracy of the inner product $(-, -)_{\extendedtoroidal}$ that:
                        $$\xi_{(m, p), (r, s)}(x) = 0$$
                    necessarily. 
    
                    We can now conclude that:
                        $$[D_{r, s}, x v^m t^p]_{\extendedtoroidal} = ( rp - ms ) x v^{m - r} t^{p - s - 1} + K_{(m, p), (r, s)}(x)$$
                    \item \textbf{(Case 2: $D := D_v$):} In this case, it is easy to see that:
                        $$
                            \begin{aligned}
                                ( D_v, [x v^m t^p, y v^n t^q]_{\toroidal} )_{\extendedtoroidal} & = (x, y)_{\g} ( D_v, \delta_{(m, p) + (n, q), (0, 0)} ( n c_v + q c_t ) + (np - mq) K_{m + n, p + q} )_{\extendedtoroidal}
                                \\
                                & = (x, y)_{\g} \delta_{(m, p) + (n, q), (0, 0)} n
                            \end{aligned}
                        $$
                    Using invariance, we then see that:
                        $$( [D_v, x v^m t^p]_{\extendedtoroidal}, y v^n t^q )_{\extendedtoroidal} = (x, y)_{\g} \delta_{(m, p) + (n, q), (0, 0)} n$$
                    Now, suppose that:
                        $$[D_v, x v^m t^p]_{\extendedtoroidal} := \sum_{(a, b) \in \Z^2} \lambda_{a, b}(x) v^a t^b + K_{m, p}(x) + \xi_{m, p}(x)$$
                    for some $\lambda_{a, b}(x) \in \g$, $K_{m, p}(x) \in \z_{[2]}$, and $\xi_{m, p}(x) \in \d_{[2]}$, depending on our choices of $x \in \g$ and $(m, p) \in \Z^2$. Then, consider the following:
                        $$
                            \begin{aligned}
                                ( [D_v, x v^m t^p]_{\extendedtoroidal}, y v^n t^q )_{\extendedtoroidal} & = \left( \sum_{(a, b) \in \Z^2} \lambda_{a, b}(x) v^a t^b + K_{m, p}(x) + \xi_{m, p}(x), y v^n t^q \right)_{\extendedtoroidal}
                                \\
                                & = \sum_{(a, b) \in \Z^2} \left( \lambda_{a, b}(x) v^a t^b, y v^n t^q \right)_{\g_{[2]}}
                                \\
                                & = -\sum_{(a, b) \in \Z^2} (\lambda_{a, b}(x), y)_{\g} \delta_{ (a, b) + (n, q), (0, -1) }
                                \\
                                & = -(\lambda_{-n, -q - 1}(x), y)_{\g}
                            \end{aligned}
                        $$
                    From this, we are able to conclude that:
                        $$(x, y)_{\g} \delta_{(m, p) + (n, q), (0, 0)} n = -(\lambda_{-n, -q - 1}(x), y)_{\g}$$
                    As this holds for all $y \in \g$ and all $(n, q) \in \Z^2$, we can infer from the above and from the non-degeneracy of the inner product $(-, -)_{\g}$ that:
                        $$\lambda_{-n, -q - 1}(x) = \delta_{(m, p) + (n, q), (0, 0)} n x$$
                    for any $x \in \g$ and any $(m, p) \in \Z^2$ (both fixed!), and hence:
                        $$
                            \begin{aligned}
                                [D_v, x v^m t^p]_{\extendedtoroidal} & = \sum_{(n, q) \in \Z^2} \delta_{(m, p) + (n, q), (0, 0)} n x v^{-n} t^{-q - 1} + K_{m, p}(x) + \xi_{m, p}(x)
                                \\
                                & = -m x v^m t^{p - 1} + K_{m, p}(x) + \xi_{m, p}(x)
                            \end{aligned}
                        $$
    
                    Now, by arguing as in \textbf{Case 1}, we will see that:
                        $$\xi_{m, p}(x) = 0$$
                    and afterwards we will be able to conclude that:
                        $$[D_v, x v^m t^p]_{\extendedtoroidal} = -m x v^m t^{p - 1} + K_{m, p}(x)$$
                    \item \textbf{(Case 3: $D := D_t$)} Arguing as when $D = D_v$, we will obtain:
                        $$[D_t, x v^m t^p]_{\extendedtoroidal} = -p x v^m t^{p - 1} + K_{m, p}(x)$$
                    for some $K_{m, p}(x) \in \z_{[2]}$.
                \end{enumerate}

                Let us now verify that elements of $\d_{[2]}$ are indeed derivations. We can identify the derivations $D_{r, s}, D_v, D_t$ explicitly in terms of $\del_v := \frac{\del}{\del v}, \del_t := \frac{\del}{\del t}$. For this, let us firstly equip $\der_{k}(A)$ - the $k$-vector space of all $k$-linear derivations on $A$ - with the following basis:
                    $$\{ v^m t^p \del_v, v^n t^q \del_t \}_{(m, p), (n, q) \in \Z^2}$$
                \begin{enumerate}
                    \item To compute $D_{r, s}$ in terms of $\del_v, \del_t$, suppose firstly that:
                        $$D_{r, s} := f(v, t) \del_v + g(v, t) \del_t$$
                    with $f(v, t), g(v, t) \in A$. Next, fix some $(m, p) \in \Z^2$ and then consider the following:
                        $$
                            \begin{aligned}
                                D_{r, s}( v^m t^p ) & = f(v, t) \del_v( v^m t^p ) + g(v, t) \del_t( v^m t^p )
                                \\
                                & = f(v, t) m v^{m - 1} t^p + g(v, t) p v^m t^{p - 1}
                            \end{aligned}
                        $$
                    At the same time, we also have that:
                        $$D_{r, s}(v^m t^p) := ( ms - rp ) v^{m - r} t^{p - s - 1}$$
                    and hence:
                        $$f(v, t) m v^{m - 1} t^p + g(v, t) p v^m t^{p - 1} = ( ms - rp ) v^{m - r} t^{p - s - 1}$$
                    From this, one infers that:
                        $$f(v, t) = s v^{-r + 1} t^{-s - 1}, g(v, t) = -r v^{-r} t^{-s}$$
                    and therefore:
                        $$D_{r, s} = s v^{-r + 1} t^{-s - 1} \del_v - r v^{-r} t^{-s} \del_t$$
                    \item One easily checks that:
                        $$D_v = -v t^{-1} \del_v$$
                    \item Likewise:
                        $$D_t = -\del_t$$
                \end{enumerate}
                Consequently, we see that elements of $\d_{[2]}$ are derivations on $A$.
            \end{proof}
        \begin{remark}
            Now that we know that the basis elements $D_{r, s}, D_v, D_t \in \d_{[2]}$ are actually certain derivations on $A$, we can also check that the commutators of the elements $D_{r, s}, D_v, D_t$ are still elements of $\d_{[2]}$. This ensures us that we can \textit{choose} to endow $\d_{[2]}$ with the structure of a Lie subalgebra of $\der_{k}(A)$, i.e. the Lie algebra structure such that:
                $$[D, D']_{\extendedtoroidal} \in \d_{[2]}$$
            for any $D, D' \in \d_{[2]}$. In general, however, we are only guaranteed that:
                $$[D, D']_{\extendedtoroidal} = \z_{[2]} \oplus \d_{[2]}$$
            We note also that it is not even clear \textit{a priori} that the $\d_{[2]}$-summand of the commutators of the form $[D, D']_{\extendedtoroidal}$ has to be the usual commutator inherited from $\der_{k}(A)$; this turns out to be true, but follows from some non-trivial computations (cf. proposition \ref{prop: lie_bracket_on_orthogonal_complement_of_toroidal_centre}). 
        \end{remark}

        Now that we know how $\d_{[2]}$ acts on $\g_{[2]}$, it remains to see how it acts on $\z_{[2]}$ to completely determine its action on $\toroidal$. \todo{Previously a remark.}
        \begin{lemma}[$\d_{[2]}$ acts on $\z_{[2]}$ by Lie derivatives] \label{lemma: derivation_action_on_toroidal_centres}
            Elements of $\d_{[2]}$ act on those of $\z_{[2]}$ as Lie derivatives. This is to say that, the elements $D \in \d_{[2]}$ act on the generating elements $f \bar{d}g \in \z_{[2]}$ (for some $f, g \in A$) in the following manner:
                $$[D, f \bar{d}g]_{\extendedtoroidal} = \xi_D(f) \bar{d}g + f \bar{d}(\xi_D(g))$$
            where $\xi_D \in \der_k(A)$ is a derivation on $A$ determined uniquely by $A$ (well-defined because $\d_{[2]}$ is a vector subspace of $\der_k(A)$ per lemma \ref{lemma: derivation_action_on_multiloop_algebras}). In particular, this means that:
                $$[\d_{[2]}, \z_{[2]}]_{\extendedtoroidal} \subseteq \z_{[2]}$$
        \end{lemma}
            \begin{proof}
                Without any loss of generality, let us consider the following for any $h, h' \in \h$ so that\footnote{We can make this assumption because ultimately, elements of $\z_{[2]}$ do not depend on those of $\g$.}:
                    $$(h, h')_{\g} = 1$$
                any $f(v, t), g(v, t) \in A$, and any $D \in \d_{[2]}$:
                    $$[ D, [h f(v, t), h' g(v, t)]_{\toroidal} ]_{\extendedtoroidal} = [ D, f(v, t) \bar{d}( g(v, t) ) ]_{\extendedtoroidal}$$
                At the same time, we have via the Jacobi identity that:
                    $$
                        \begin{aligned}
                            [ D, [h f(v, t), h' g(v, t)]_{\toroidal} ]_{\extendedtoroidal} & = [ h f(v, t), [D, h' g(v, t)]_{\extendedtoroidal} ]_{\toroidal} + [ [D, h f(v, t)]_{\extendedtoroidal}, h' g(v, t) ]_{\toroidal}
                            \\
                            & = [ h f(v, t), h' D( g(v, t) ) ]_{\toroidal} + [ h D( f(v, t) ), h' g(v, t) ]_{\toroidal}
                            \\
                            & = f(v, t) \bar{d}( D( g(v, t) ) ) + D( f(v, t) ) \bar{d}(g(v, t))
                        \end{aligned}
                    $$
                One thus sees that:
                    $$[ D, f(v, t) \bar{d}( g(v, t) ) ]_{\extendedtoroidal} = f(v, t) \bar{d}( D( g(v, t) ) ) + D( f(v, t) ) \bar{d}(g(v, t))$$
                and since the element $f(v, t) \bar{d}( g(v, t) )$ is central (via the map $\e$ mentioned earlier), this gives another description of:
                    $$[ \d_{[2]}, \z_{[2]} ]_{\extendedtoroidal}$$
                With this in mind, we return quickly to lemma \ref{lemma: derivation_action_on_multiloop_algebras}; there, we previously demonstrated that:
                    $$[ \d_{[2]}, \g_{[2]} ]_{\extendedtoroidal} \subseteq \g_{[2]} \oplus \z_{[2]}$$
                but we claim now that the following stronger fact holds:
                    $$[ \d_{[2]}, \g_{[2]} ]_{\extendedtoroidal} \subseteq \g_{[2]}$$
                To see why this is the case, suppose firstly that for any $D \in \d_{[2]}$, any $X := x f(v, t) \in \g_{[2]}$ (for some $f(v, t) \in A$), there is $K(X) \in \z_{[2]}$ depending on $X$ (and indeed, such a $K(X)$ exists by lemma \ref{lemma: derivation_action_on_multiloop_algebras}) such that:
                    $$[ D, X ]_{\extendedtoroidal} = x D( f(v, t) ) + K(X)$$
                Next, pick an arbitrary element $\xi \in \d_{[2]}$ and then consider the following:
                    $$( [ D, X ]_{\extendedtoroidal}, \xi )_{\extendedtoroidal} = (D(X) + K(X), \xi)_{\extendedtoroidal} = (K(X), \xi)_{\extendedtoroidal}$$
                wherein the last equality holds as a consequence of the fact that:
                    $$( \g_{[2]}, \d_{[2]} )_{\extendedtoroidal} = 0$$
                per the construction of the bilinear form $(-, -)_{\extendedtoroidal}$ as in convention \ref{conv: orthogonal_complement_of_toroidal_centres}.
            \end{proof}
        This lemma has two important corollaries. The first one is easy.
        \begin{corollary}
            With respect to the bracket $[-, -]_{\extendedtoroidal}$, the vector subspace $\toroidal$ is actually a Lie ideal of $\extendedtoroidal$.
        \end{corollary}
        In light of lemma \ref{lemma: derivation_action_on_toroidal_centres}, we see also that the $\z_{[2]}$-summands of the elements of $[\d_{[2]}, \g_{[2]}]_{\extendedtoroidal}$ (cf. lemma \ref{lemma: derivation_action_on_multiloop_algebras}) necessarily vanish., yielding us the following second corollary to the lemma. \todo{Previously a remark.}
        \begin{corollary}[\texorpdfstring{$\z_{[2]}$}{}-summands of elements of \texorpdfstring{$[\d_{[2]}, \g_{[2]}]_{\extendedtoroidal}$}{}] \label{coro: derivation_action_on_multiloop_algebras}
            The action of $\d_{[2]}$ on $\g_{[2]}$ as in lemma \ref{lemma: derivation_action_on_multiloop_algebras} satisfies:
                $$[\d_{[2]}, \g_{[2]}]_{\extendedtoroidal} \subseteq \g_{[2]}$$
        \end{corollary}
            \begin{proof}
                Firstly, let us note that by using invariance, we yield:
                    $$( [ D, X ]_{\extendedtoroidal}, \xi )_{\extendedtoroidal} = (X, [\xi, D]_{\extendedtoroidal})_{\extendedtoroidal} = 0$$
                wherein the last equality is due to the fact that:
                    $$[\xi, D]_{\extendedtoroidal} \in \z_{[2]} \oplus \d_{[2]}$$
                (cf. lemma \ref{lemma: derivation_action_on_multiloop_algebras}) and the fact that:
                    $$( \g_{[2]}, \z_{[2]} \oplus \d_{[2]} )_{\extendedtoroidal} = 0$$
                per the construction of the bilnear form $(-, -)_{\extendedtoroidal}$ as in convention \ref{conv: orthogonal_complement_of_toroidal_centres}. We thus have that:
                    $$(K(X), \xi)_{\extendedtoroidal} = (X, [\xi, D]_{\extendedtoroidal})_{\extendedtoroidal} = 0$$
                for every $\xi \in \d_{[2]}$. The non-degeneracy of $(-, -)_{\extendedtoroidal}$ then implies through this fact that:
                    $$K(X) = 0$$
                As such, we have that:
                    $$[ \d_{[2]}, \g_{[2]} ]_{\extendedtoroidal} \subseteq \g_{[2]}$$
                as claimed. 
            \end{proof}

        We have now reached the following conclusion, as eluded to at the beginning of the subsection:
        \begin{proposition}[$\toroidal$ as a $\der_{k}(A)$-module] \label{prop: toroidal_lie_algebras_as_modules_over_vector_field_lie_algebras}
            $\toroidal$ is a $\der_{k}(A)$-module, decomposing into a direct sum of the submodules $\g_{[2]}$ and $\z_{[2]}$. 
        \end{proposition}
        We note that even in light of this proposition and the fact that $\extendedtoroidal$ admits $\toroidal$ as a Lie ideal, we still do not yet know enough to conclude that $\extendedtoroidal$ necessarily arises as a Lie algebra extension with kernel $\toroidal$. It remains to investigate how the brackets of the form:
            $$[D, D']_{\extendedtoroidal}$$
        (for some $D, D' \in \d_{[2]}$) are given.

    \subsection{Non-uniqueness of Yangian extended toroidal Lie brackets}
        Let us now change our perspective on $\d_{[2]}$ slightly. We begin by demonstrating (cf. proposition \ref{prop: lie_bracket_on_orthogonal_complement_of_toroidal_centre} and the corollary that follows) that for any $D, D' \in \d_{[2]}$, there exists some $K(D, D') \in \z_{[2]}$ such that:
            $$[D, D']_{\extendedtoroidal} = [D, D'] + K(D, D')$$
        with $[D, D'] := DD' - D'D$ denoting the usual commutator; this allows us to regard $\d_{[2]}$ as a Lie subalgebra of $\der_k(A)$. Because we now know from the previous subsection that $\toroidal$ is a $\der_k(A)$-module, the above implies that $\toroidal$ is also a $\d_{[2]}$-module (cf. proposition \ref{prop: toroidal_lie_algebras_as_modules_over_div_0_vector_field_lie_algebras}), and we are thus able to regard $\extendedtoroidal$ as a Lie algebra extension of $\d_{[2]}$ with kernel $\toroidal$. This then begs the question about the (non-)uniqueness of the Lie algebra structure on $\extendedtoroidal$: namely, how many isomorphism classes of Lie algebra structures on $\extendedtoroidal$ are there ? Of course, there is always the semi-direct product, corresponding to the zero $2$-cocycle of $\d_{[2]}$ with coefficients in $\z_{[2]}$ (cf. example \ref{example: lie_algebra_semi_direct_products}). Additionally, there are two isomorphism classes of Lie algebra structures different from the semi-direct product, corresponding to two non-zero isomorphism classes of $2$-cocycles of $\d_{[2]}$ with coefficients in $\z_{[2]}$ (see remark \ref{remark: non_uniqueness_of_yangian_extended_lie_algebras} and theorem \ref{theorem: non_uniqueness_of_yangian_extended_lie_algebras}); these have been known since \cite{billig_energy_momentum_tensor}. 
        
        \begin{proposition}[How does $\d_{[2]}$ act on itself] \label{prop: lie_bracket_on_orthogonal_complement_of_toroidal_centre}
            Let $\d_{[2]}$ be given as in convention \ref{conv: orthogonal_complement_of_toroidal_centres}. Then:
                $$[ \d_{[2]}, \d_{[2]} ]_{\extendedtoroidal} \subset \z_{[2]} \oplus \d_{[2]}$$
            i.e. the $\g_{[2]}$-summand of any commutator of the kind $[D, D']_{\extendedtoroidal}$ (for any two $D, D' \in \d_{[2]}$) actually vanishes. Furthermore, neither the $\z_{[2]}$- nor the $\d_{[2]}$-summand of those commutators $[D, D']_{\extendedtoroidal}$ necessarily vanish in general. 
        \end{proposition}
            \begin{proof}
                For convenience, we will be abbreviating $\h_{[2]} := \h[v^{\pm}, t^{\pm 1}]$ and $\n^{\pm}_{[2]} := [v^{\pm}, t^{\pm 1}]$, with $\n^{\pm} := \bigoplus_{\alpha \in \Phi^{\pm}} \g_{\alpha}$ being the direct sums of the positive/negative roots spaces of $\g$, as usual.
            
                Pick arbitrary elements $D, D' \in \d_{[2]}$ and set:
                    $$[D, D']_{\extendedtoroidal} := X(D, D') + K(D, D') + \xi(D, D')$$
                for some $X(D, D') \in \g_{[2]}, K(D, D') \in \z_{[2]}$, and $\xi(D, D') \in \d_{[2]}$ depending on $D, D'$. Pick also a test element $y g(v, t) \in \g_{[2]}$, for some arbitrary $y \in \g$ and $g(v, t) \in A$ and set:
                    $$[D, y g(v, t)]_{\extendedtoroidal} := y D( g(v, t) ) + K_{D, Y}$$
                    $$[D', y g(v, t)]_{\extendedtoroidal} := y D'( g(v, t) ) + K_{D', Y}$$
                for some $K_{D, Y} \in \z_{[2]}$ depending on $Y$ (cf. lemma \ref{lemma: derivation_action_on_multiloop_algebras}).
                
                Via the Jacobi identity, we get that:
                    $$
                        \begin{aligned}
                            & [ [D, D']_{\extendedtoroidal}, y g(v, t) ]_{\extendedtoroidal}
                            \\
                            = & [ D, [ D', y g(v, t) ]_{\extendedtoroidal} ]_{\extendedtoroidal} + [ D', [ y g(v, t), D ]_{\extendedtoroidal} ]_{\extendedtoroidal}
                            \\
                            = & [ D, y D'( g(v, t) ) + K_{D', Y} ]_{\extendedtoroidal} - [ D', y D( g(v, t) ) + K_{D, Y} ]_{\extendedtoroidal}
                            \\
                            = & \left( y D( D'(g(v, t)) ) + K_{DD', Y} + [ D, K_{D', Y} ]_{\extendedtoroidal} \right) - \left( y D'( D(g(v, t)) ) + K_{D'D, Y} + [ D', K_{D, Y} ]_{\extendedtoroidal} \right)
                            \\
                            = & y (DD' - D'D)( g(v, t) ) + ( K_{DD', Y} - K_{D'D, Y} ) + ( [ D, K_{D', Y} ]_{\extendedtoroidal} - [ D', K_{D, Y} ]_{\extendedtoroidal} )
                        \end{aligned}
                    $$
                for some $K_{DD', Y}, K_{D'D, Y} \in \z_{[2]}$ such that:
                    $$[ D, y D'( g(v, t) ) ]_{\extendedtoroidal} := y D( D'( g(v, t) ) ) + K_{DD', Y}$$
                    $$[ D', y D( g(v, t) ) ]_{\extendedtoroidal} := y D( D'( g(v, t) ) ) + K_{D'D, Y}$$
                At the same time, we have that:
                    $$
                        \begin{aligned}
                            & [ [D, D']_{\extendedtoroidal}, y g(v, t) ]_{\extendedtoroidal}
                            \\
                            = & [ X(D, D') + K(D, D') + \xi(D, D') , y g(v, t) ]_{\extendedtoroidal}
                            \\
                            = & [ X(D, D') + \xi(D, D') , y g(v, t) ]_{\extendedtoroidal}
                            \\
                            = & [ X(D, D') , y g(v, t) ]_{\extendedtoroidal} + \left( y \xi(D, D')(g(v, t)) + K_{\xi(D, D'), Y} \right)
                        \end{aligned}
                    $$
                wherein the second equality holds thanks to the fact that $[\z_{[2]}, \g_{[2]}]_{\extendedtoroidal} = 0$, and $K_{\xi(D, D'), Y} \in \z_{[2]}$ is some element (cf. lemma \ref{lemma: derivation_action_on_multiloop_algebras}). Combining the two observations together then yields:
                    $$
                        \begin{aligned}
                            & [ X(D, D') , y g(v, t) ]_{\extendedtoroidal} + \left( y \xi(D, D')(g(v, t)) + K_{\xi(D, D'), Y} \right)
                            \\
                            = & y (DD' - D'D)( g(v, t) ) + ( K_{DD', Y} - K_{D'D, Y} ) + ( [ D, K_{D', Y} ]_{\extendedtoroidal} - [ D', K_{D, Y} ]_{\extendedtoroidal} )
                        \end{aligned}
                    $$
                There exists $K_{X(D, D'), Y} \in \z_{[2]}$ such that:
                    $$[ X(D, D') , y g(v, t) ]_{\extendedtoroidal} = [ X(D, D') , Y ]_{\extendedtoroidal} = [X(D, D'), Y]_{\g_{[2]}} + K_{X(D, D'), Y}$$
                using which we can write:
                    $$
                        \begin{aligned}
                            & [X(D, D'), Y]_{\g_{[2]}} - y \left( ( DD' - D'D) - \xi(D, D') \right)( g(v, t) )
                            \\
                            = & \left( [ D, K_{D', Y} ]_{\extendedtoroidal} - [ D', K_{D, Y} ]_{\extendedtoroidal} \right) - \left( K_{X(D, D'), Y} + K_{\xi(D, D'), Y} \right)
                        \end{aligned}
                    $$
                    
                We note right away that the LHS lies entirely in $\g_{[2]}$, whereas the RHS is an element of $\z_{[2]}$ due to the fact that $[\d_{[2]}, \z_{[2]}]_{\extendedtoroidal} \subseteq \z_{[2]}$ (cf. lemma \ref{lemma: derivation_action_on_toroidal_centres}), which tells us that $[ D, K_{D', Y} ]_{\extendedtoroidal}, [ D', K_{D, Y} ]_{\extendedtoroidal} \in \z_{[2]}$ in particular. Because $\g_{[2]}$ is centreless (as $\g$ is simple and the Lie bracket on $\g_{[2]}$ is given by extension of scalars), this observation subsequently implies that the LHS must vanish, i.e.:
                    $$[X(D, D'), Y]_{\g_{[2]}} - y \left( ( DD' - D'D) - \xi(D, D') \right)( g(v, t) ) = 0$$
                Because we have by construction that:
                    $$DD' - D'D - \xi(D, D') \in \d_{[2]}$$
                we now make the following claim: \textit{if we fix some arbitrary $E \in \g_{[2]}$ and some $P \in \d_{[2]}$ then:}
                    $$\forall H := h \varphi \in \g_{[2]}: [E, H]_{\g_{[2]}} = h P( \varphi ) \implies E = 0$$

                Using the root space decomposition for $\g$, we see that if $h \in \h$ then we then will have that $[E, H]_{\g_{[2]}} \in \n^{\pm}_{[2]}$, but at the same time, that $h P(\varphi) \in \h_{[2]}$. The only way for this to be true is that $[E, H]_{\g_{[2]}} = 0$, which is the case if and only if $E = 0$. If $h \in \n^{\pm}$, then $[E, H]_{\g_{[2]}} \in \n^{\pm}_{[2]} \oplus \h_{[2]}$ and the $\h_{[2]}$-summand will be non-zero in general; at the same time, $h P(\varphi) \in \n^{\pm}_{[2]}$ in this case, and again, the only way for these to facts to be true simultaneously is that $E = 0$ necessarily. 

                Apply the claim to the fact that:
                    $$[X(D, D'), Y]_{\g_{[2]}} = y \left( ( DD' - D'D) - \xi(D, D') \right)( g(v, t) )$$
                - and again, note that $( DD' - D'D) - \xi(D, D') \in \d_{[2]}$ - then yields:
                    $$X(D, D') = 0$$
                precisely as desired. 
            \end{proof}
        \begin{corollary}
            For any $D, D' \in \d_{[2]}$, the $\d_{[2]}$-summand of $[D, D']_{\extendedtoroidal}$ is nothing but the commutator $DD' - D'D$.
        \end{corollary} 
        In light of the above, it is also valuable to know how the $\d_{[2]}$-summand of the commutators:
            $$[D, D']_{\extendedtoroidal}$$
        are given explicitly. Later on, these computations will be used to prove that the centre of $\extendedtoroidal$ is $2$-dimensional, namely given by $k c_v \oplus k c_v$.
        \begin{lemma}[Explicitly commutators between basis elements of $\d_{[2]}$] \label{lemma: explicit_commutators_between_basis_elements_of_toroidal_central_orthogonal_complement}
            The usual commutator (i.e. $[D, D'] := DD' - D'D$) between the basis elements $D_{r, s}, D_v, D_t \in \d_{[2]}$ are given as follows:
                $$[D_v, D_t] = 0$$
                $$[D_v, D_{r, s}] = r D_{r, s + 1}$$
                $$[D_t, D_{r, s}] = D_{r, s + 1}$$
                $$[D_{a, b}, D_{r, s}] = (br - sa) D_{a + r, b + s + 1}$$
        \end{lemma}
            \begin{proof}
                \begin{enumerate}
                    \item Since we know that:
                        $$D_v = -vt^{-1} \del_v, D_t = -\del_t$$
                    (cf. lemma \ref{lemma: derivation_action_on_multiloop_algebras}), it is therefore trivial that:
                        $$[D_v, D_t] = 0$$
                    \item From lemma \ref{lemma: derivation_action_on_multiloop_algebras}, we know that:
                        $$D_v(v^m t^p) = -m v^m t^{p - 1}$$
                        $$D_{r, s}(v^m t^p) = ( ms - rp ) v^{m - r} t^{p - s - 1}$$
                    From this, we infer that:
                        $$
                            \begin{aligned}
                                [D_v, D_{r, s}](v^m t^p) & = D_v( D_{r, s}(v^m t^p) ) - D_{r, s}( D_v(v^m t^p) )
                                \\
                                & = (ms - rp) D_v( v^{m - r} t^{p - s - 1} ) + m D_{r, s}( v^m t^{p - 1} )
                                \\
                                & = -(m - r)(ms - rp) v^{m - r} t^{p - s - 2} + (ms - r(p - 1)) m v^{m - r} t^{p - s - 2}
                                \\
                                & = r(m(s + 1) - rp) v^{m - r} t^{p - (s + 1) - 1}
                                \\
                                & = r D_{r, s + 1}(v^m t^p)
                            \end{aligned}
                        $$
                    and hence:
                        $$[D_v, D_{r, s}] = r D_{r, s + 1}$$
                    \item Likewise, we can show that:
                        $$[D_t, D_{r, s}] = D_{r, s + 1}$$
                    \item Lastly, we can also show, using a completely analogous argument, that:
                        $$[D_{a, b}, D_{r, s}] = (br - sa) D_{a + r, b + s + 1}$$
                \end{enumerate}
            \end{proof}
        By exploiting invariance again, we are able to obtain the following corollary to the lemma above, concerning commutators between elements of $\z_{[2]}$ and those of $\d_{[2]}$ (which are already known to be elements of $\z_{[2]}$; cf. lemma \ref{lemma: derivation_action_on_toroidal_centres}). Due to its usefulness, we nevertheless grant it lemma-hood.
        \begin{lemma}[Explicit commutators between basis elements of $\d_{[2]}$ and $\z_{[2]}$] \label{lemma: explicit_commutators_between_central_basis_elements_and_derivations}
            In the Lie algebra $\extendedtoroidal$, one has the following non-trivial relations between elements of $\z_{[2]}$ and those of $\d_{[2]}$:
                $$
                    \forall (a, b) \in \Z^2: [D, K_{a, b}]_{\extendedtoroidal} =
                    \begin{cases}
                        \text{$((b - 1)r - sa) D_{a - r, b - s - 1}$ if $D = D_{r, s}$}
                        \\
                        \text{$-r K_{a, b - 1}$ if $D = D_v$}
                        \\
                        \text{$- D_{a, b - 1}$ if $D = D_t$}
                    \end{cases}
                $$
                $$[\d_{[2]}, c_v]_{\extendedtoroidal} = [\d_{[2]}, c_t]_{\extendedtoroidal} = 0 = 0$$
        \end{lemma}
            \begin{proof}
                For a moment, consider two arbitrary derivations $D', D \in \d_{[2]}$ along with an arbitrary element $K \in \z_{[2]}$. Suppose also that:
                    $$[D', D]_{\extendedtoroidal} = [D', D] + K(D', D)$$
                for some $K(D', D) \in \z_{[2]}$ (cf. proposition \ref{prop: lie_bracket_on_orthogonal_complement_of_toroidal_centre}). Using the invariance property of the bilinear form $(-, -)_{\extendedtoroidal}$ as well as the fact that:
                    $$(\z_{[2]}, \z_{[2]})_{\extendedtoroidal} = 0$$
                per the construction of $(-, -)_{\extendedtoroidal}$ (cf. convention \ref{conv: orthogonal_complement_of_toroidal_centres}), one gets:
                    $$([D', D], K)_{\extendedtoroidal} = ([D', D] + K(D', D), K)_{\extendedtoroidal} = ([D', D]_{\extendedtoroidal}, K)_{\extendedtoroidal} = (D', [D, K]_{\extendedtoroidal})_{\extendedtoroidal}$$
                Since we know how the ordinary commutators:
                    $$[D', D]$$
                are given and particularly, how if $D', D \in \d_{[2]}$ are basis elements then their commutator $[D', D]$ will be in the span of a single basis element of $\d_{[2]}$ (cf. lemma \ref{lemma: explicit_commutators_between_basis_elements_of_toroidal_central_orthogonal_complement}), as well as how $\d_{[2]}$ is the orthogonal complement of $\z_{[2]}$ with respect to $(-, -)_{\extendedtoroidal}$ by construction, it remains now to simply specialise to the case wherein $K$ is some basis element of $\z_{[2]}$. 
                \begin{enumerate}
                    \item If:
                        $$K = K_{a, b}$$
                    for some fixed $(a, b) \in \Z^2$ then:
                        $$([D', D], K_{a, b})_{\extendedtoroidal} = 1 \iff [D', D] = D_{a, b}$$
                    Hence, we have that:
                        $$1 = (D_{a, b}, K_{a, b})_{\extendedtoroidal} = ([D', D], K_{a, b})_{\extendedtoroidal} = (D', [D, K_{a, b}]_{\extendedtoroidal})_{\extendedtoroidal}$$
                    Using the commutators computed in lemma \ref{lemma: explicit_commutators_between_basis_elements_of_toroidal_central_orthogonal_complement}, we then see that:
                        $$
                            [D, K_{a, b}]_{\extendedtoroidal} =
                            \begin{cases}
                                \text{$((b - 1)r - sa) D_{a - r, b - s - 1}$ if $D = D_{r, s}$}
                                \\
                                \text{$-r K_{a, b - 1}$ if $D = D_v$}
                                \\
                                \text{$- D_{a, b - 1}$ if $D = D_t$}
                            \end{cases}
                        $$
                    \item If:
                        $$K = c_v$$
                    then we have that:
                        $$([D', D], c_v)_{\extendedtoroidal} = 1 \iff [D', D] = D_v$$
                    which in turn implies that:
                        $$1 = (D_v, K_v)_{\extendedtoroidal} = ([D', D], c_v)_{\extendedtoroidal} = (D', [D, c_v]_{\extendedtoroidal})_{\extendedtoroidal}$$
                    Once again, by using the commutators computed in lemma \ref{lemma: explicit_commutators_between_basis_elements_of_toroidal_central_orthogonal_complement}, we then see that:
                        $$\forall D \in \{D_{r, s}\}_{(r, s) \in \Z^2} \cup \{D_v, D_t\}: [D, c_v]_{\extendedtoroidal} = 0$$
                    and since the set $\{D_{r, s}\}_{(r, s) \in \Z^2} \cup \{D_v, D_t\}$ is a basis for $\d_{[2]}$, one thus has that:
                        $$[\d_{[2]}, c_v]_{\extendedtoroidal} = 0$$
                    as none of said commutators are elements of $k D_v$.
                    \item Likewise, one can show that:
                        $$[\d_{[2]}, c_t]_{\extendedtoroidal} = 0$$
                \end{enumerate}
            \end{proof}
        We will also see that in fact, the centre of $\extendedtoroidal$ is actually just spanend by $c_v$ and $c_t$. See proposition \ref{prop: centres_of_extended_toroidal_lie_algebras} and the discussion preceding it for more details.
        
        \begin{proposition}[$\toroidal$ as a $\d_{[2]}$-module] \label{prop: toroidal_lie_algebras_as_modules_over_div_0_vector_field_lie_algebras}
            If we regard $\d_{[2]}$ as a Lie subalgebra of $\der_{k}(A)$ with the usual commutator bracket, then $\toroidal$ will be a $\d_{[2]}$-module, not just a $\der_k(A)$-module.
        \end{proposition}
        
        In summary, we have yielded the following result:
        \begin{theorem}[Yangian extended toroidal Lie algebras] \label{theorem: extended_toroidal_lie_algebras}
            Let $\d_{[2]}$ be equipped with the commutator bracket inherited from $\der_k(A)$.
        
            When endowed with the Lie bracket $[-, -]_{\extendedtoroidal}$\footnote{... which ultimately is specified by the Lie structure on $\toroidal$ and the non-degenerate and invariant symmetric bilinear form $(-, -)_{\extendedtoroidal}$ as in convention \ref{conv: orthogonal_complement_of_toroidal_centres}.} (as determined in lemmas \ref{lemma: derivation_action_on_multiloop_algebras} and \ref{lemma: derivation_action_on_toroidal_centres}, and proposition \ref{prop: lie_bracket_on_orthogonal_complement_of_toroidal_centre}), the vector space $\extendedtoroidal := \toroidal \oplus \d_{[2]}$ becomes a Lie algebra extension:
                $$0 \to \toroidal \to \extendedtoroidal \to \d_{[2]} \to 0$$
            of $\d_{[2]}$ by $\toroidal$. 
        \end{theorem}
        \begin{remark}
            Note that we do not yet know whether or not $\extendedtoroidal$ is unique (up to isomorphisms) as a Lie algebra extension of $\d_{[2]}$ by $\toroidal$. Due to how brackets on Lie algebra extensions are given (cf. proposition \ref{prop: lie_brackets_on_extensions}), the uniqueness or non-uniqueness (up to isomorphisms) of $\extendedtoroidal$ as a Lie algebra extension of $\d_{[2]}$ by $\toroidal$ lies in how the $\z_{[2]}$-summands of brackets of the form:
                $$[D, D']_{\extendedtoroidal} \in \z_{[2]} \oplus \d_{[2]}$$
            (for all $D, D' \in \d_{[2]}$) are given. Of course, when these summands are equally $0$, one gets $\extendedtoroidal$ as the semi-direct product of $\d_{[2]}$ by $\toroidal$:
                $$\extendedtoroidal \cong \toroidal \rtimes \d_{[2]}$$
            Otherwise, one shall need to find the number of isomorphism classes of non-zero $2$-cocycles of $\d_{[2]}$ with coefficients in $\z_{[2]}$ (again, see proposition \ref{prop: lie_brackets_on_extensions} for why), or in other words, compute the dimension of the vector space $H^2_{\Lie}(\d_{[2]}, \z_{[2]})$. Through remark \ref{remark: non_uniqueness_of_yangian_extended_lie_algebras} and theorem \ref{theorem: non_uniqueness_of_yangian_extended_lie_algebras} below, we will see that there are two non-zero isomorphism classes.
        \end{remark}
        \begin{definition}[Yangian extended toroidal Lie algebras] \label{def: extended_toroidal_lie_algebras}
            We refer to $\extendedtoroidal$ as in theorem \ref{theorem: extended_toroidal_lie_algebras} as a \textbf{Yangian extended toroidal Lie algebra}. 
        \end{definition}
        \begin{remark}[Regarding terminologies]
            Our notion of Yangian extended toroidal Lie algebras does not quite coincide with the very similar notion of an \say{toroidal extended affine Lie algebra} that appeared, for instance, in \cite{billig_representations_of_toroidal_extended_affine_lie_algebras}. Ultimately, this is because the bilinear form that we have endowed $\g_{[2]}$ with is of degree $-1$ (as opposed to $0$) in the second variable. For the latter, $\d_{[2]}$ would be the Lie algebra of divergence-free algebraic vector fields on the (smooth) affine $k$-scheme $\Spec A \cong \G_m^2$.
        \end{remark}
        
        \begin{remark}[Some explicit elements of $H^2_{\Lie}(\d_{[2]}, \z_{[2]})$] \label{remark: non_uniqueness_of_yangian_extended_lie_algebras}
            From proposition \ref{prop: lie_bracket_on_orthogonal_complement_of_toroidal_centre}, we know that:
                $$[\d_{[2]}, \d_{[2]}]_{\extendedtoroidal} \subset \z_{[2]} \oplus \d_{[2]}$$
            we can obtain some $2$-cocyles $\bar{\sigma} \in H^2_{\Lie}(\d_{[2]}, \z_{[2]})$ by restricting elements $\sigma \in H^2_{\Lie}(\der_{k}(A), \z_{[2]})$, some of which are known.

            It pays to abstract the situation out to the $n$-variable case for a moment, mostly for us to make the point that the dimension of the vector space $H^2_{\Lie}(\der_{k}(k[v_1^{\pm 1}, ..., v_n^{\pm 1}]), \bar{\Omega}^1_{k[v_1^{\pm 1}, ..., v_n^{\pm 1}]/k})$ depends not on the number of variables. In \cite[p. 5, below Equation 1.3]{billig_energy_momentum_tensor}, it was noted that\todo{Find a proper citation for this. \cite{billig_energy_momentum_tensor} does not provide one.}:
                $$H^2_{\Lie}(\der_{k}(k[v_1^{\pm 1}, ..., v_n^{\pm 1}]), \bar{\Omega}^1_{k[v_1^{\pm 1}, ..., v_n^{\pm 1}]/k}) \cong k \sigma_1 \oplus k \sigma_2$$
            with the $2$-cocyles $\sigma_1, \sigma_2$ twisting the Lie brackets:
                $$[v_1^{m_1} ... v_n^{m_n} v_p \del_{v_p}, v_1^{r_1} ... v_n^{r_n} v_q \del_{v_q}] \in \der_{k}(k[v_1^{\pm 1}, ..., v_n^{\pm 1}])$$
            being given by:
                $$\sigma_1(v_1^{m_1} ... v_n^{m_n} v_p \del_{v_p}, v_1^{r_1} ... v_n^{r_n} v_q \del_{v_q}) = r_p m_q \sum_{1 \leq i \leq n} r_i v_1^{m_1 + r_1} ... v_n^{m_n + r_n} v_i^{-1} \bar{d}(v_i)$$
                $$\sigma_2(v_1^{m_1} ... v_n^{m_n} v_p \del_{v_p}, v_1^{r_1} ... v_n^{r_n} v_q \del_{v_q}) = m_p r_q \sum_{1 \leq i \leq n} r_i v_1^{m_1 + r_1} ... v_n^{m_n + r_n} v_i^{-1} \bar{d}(v_i)$$
            for every $1 \leq p, q \leq n$ and every $(m_1, ..., m_n), (r_1, ..., r_n) \in \Z^n$. 

            Now, back to the $2$-variable case. Here, we know how the basis elements $D_{r, s}, D_v, D_t$ of $\d_{[2]}$ are given in terms of $A$-multiples of the partial derivatives $\del_v, \del_t$ (cf. lemma \ref{lemma: derivation_action_on_multiloop_algebras}), so we can exploit the bilinearity of $2$-cocycles in order to see how $\sigma_1, \sigma_2$ act on elements of $\d_{[2]}$. Knowing that the aforementioned basis elements are given by:
                $$D_{r, s} = s v^{-r + 1} t^{-s - 1} \del_v - r v^{-r} t^{-s} \del_t$$
                $$D_v = -v t^{-1} \del_v$$
                $$D_t = -\del_t$$
            we see thus that:
                $$\sigma_a(D_t, -) = 0$$
                $$\sigma_a(D_{r, s}, -), \sigma_a(D_v, -) \not = 0$$
            as elements of $\Hom_{k}(\d_{[2]}, \z_{[2]})$, where $a \in \{1, 2\}$; one proves this by looking at the powers of $v, t$ in the multiples of $\del_v, \del_t$ in the expressions for $D_{r, s}, D_v, D_t$. Subsequently, we can conclude that neither of the $2$-cocycles $\sigma_1, \sigma_2$ vanish entirely on the Lie subalgebra $\d_{[2]}$ of $\der_{k}(A)$, and hence:
                $$H^2_{\Lie}(\d_{[2]}, \z_{[2]}) \cong k \sigma_1 \oplus k \sigma_2$$
            We delegate the relevant detailed computations to the proof of theorem \ref{theorem: non_uniqueness_of_yangian_extended_lie_algebras} down below.  
        \end{remark}
        \begin{theorem}[Non-uniqueness of Yangian extended Lie algebras] \label{theorem: non_uniqueness_of_yangian_extended_lie_algebras}
            Let the vector space:
                $$\d_{[2]} := (\bigoplus_{(r, s) \in \Z^2} k D_{r, s}) \oplus k D_v \oplus k D_t$$
            as in convention \ref{conv: orthogonal_complement_of_toroidal_centres} (see lemma \ref{lemma: derivation_action_on_multiloop_algebras} also, for how the elements $D_{r, s}, D_v, D_t$ are given in terms of $\del_v, \del_t$) be viewed as a Lie subalgebra of $\der_{k}(A)$ (possible thanks to proposition \ref{prop: lie_bracket_on_orthogonal_complement_of_toroidal_centre}), which we endow with the usual commutator bracket. Then, there are two isomorphism classes of Lie algebra extensions:
                $$0 \to \toroidal \to \extendedtoroidal \to \d_{[2]} \to 0$$
            or, in cohomological terms, one has that:
                $$\dim_{k} H^2_{\Lie}(\d_{[2]}, \z_{[2]}) = \dim_{k} H^2_{\Lie}(\d_{[2]}, \toroidal) = 2$$
        \end{theorem}
            \begin{proof}
                As indicated in remark \ref{remark: non_uniqueness_of_yangian_extended_lie_algebras}, it only remains to prove that:
                    $$\sigma_a(D_t, -) = 0$$
                    $$\sigma_a(D_v, -), \sigma_a(D_{r, s}, -) \not = 0$$
                (where $a \in \{1, 2\}$) via explicit computations. For our own convenience, let us temporarily relabel the variables as:
                    $$v := v_1, t := v_2$$
                \begin{itemize}
                    \item Firstly, note that:
                        $$\sigma_1(\del_{v_2}, v_1^{r_1} v_2^{r_2} \del_{v_q}) = \sigma_1(v_1^0 v_2^0 \del_{v_2}, -) = r_1 \cdot 0 \cdot \sum_{1 \leq i \leq 2} (...) = 0$$
                        $$\sigma_2(\del_{v_2}, v_1^{r_1} v_2^{r_2} \del_{v_q}) = \sigma_2(v_1^0 v_2^0 \del_{v_2}, -) = 0 \cdot r_2 \cdot \sum_{1 \leq i \leq 2} (...) = 0$$
                    and hence we indeed have that:
                        $$\sigma_a(D_t, -) = \sigma_a(-\del_{v_2}, -) = -\sigma_a(\del_{v_2}, -) = 0$$
                    \item Secondly, to show that:
                        $$\sigma_a(D_{r, s}, -) \not = 0$$
                    it suffices to only show that:
                        $$\sigma_a(D_{r, s}, D_v) \not = 0$$
                    To this end, recall that:
                        $$D_{r, s} = s v_1^{-r + 1} v_2^{-s - 1} \del_{v_1} - r v_1^{-r} v_2^{-s} \del_{v_2}$$
                        $$D_v = -v_1 v_2^{-1} \del_{v_1}$$
                    which tells us that:
                        $$
                            \begin{aligned}
                                \sigma_1(D_{r, s}, D_v) & = ( s \cdot 1 \cdot (-r + 1) - r \cdot (-1) \cdot (-s) ) \sum_{1 \leq i \leq 2} r_i v_1^{(-r + 1) + 1} v_2^{(-s - 1) - 1} v_i^{-1} \bar{d}(v_i)
                                \\
                                & = (-2sr + 1) \sum_{1 \leq i \leq 2} r_i v_1^{-r} v_2^{-s - 2} v_i^{-1} \bar{d}(v_i)
                            \end{aligned}
                        $$
                    \item Lastly, to show that:
                        $$\sigma_a(D_v, -) \not = 0$$
                    simply note that:
                        $$\sigma_a(D_{r, s}, D_v) = -\sigma_a(D_v, D_{r, s})$$
                \end{itemize}
            \end{proof}

    \subsection{The centre of \texorpdfstring{$\extendedtoroidal$}{}}
        Let us conclude this section with the following question, which is natural now that we have a solid handle on how the Lie bracket on $\extendedtoroidal$ is given:
        \begin{question}
            What is the centre $\hat{\z}_{[2]} := \z( \extendedtoroidal )$ ? This ought to be smaller than $\z_{[2]}$ somehow, since elements of $\z_{[2]}$ need not be central in $\extendedtoroidal$. 
        \end{question}
        \begin{remark}[Computing the centre without computing all the brackets ...]
            Since $\g_{[2]}$ is centreless, we have that:
                $$\hat{\z}_{[2]} = \z( \z_{[2]} \oplus \d_{[2]} )$$
            As $\z_{[2]}$ is an abelian Lie algebra, this implies that in order to compute $\hat{\z}_{[2]}$, it suffices to explicitly compute the commutators of the form:
                $$[D, K]_{\extendedtoroidal}, [D, D']_{\extendedtoroidal}$$
            for $D, D' \in \d_{[2]}$ and $K \in \z_{[2]}$, to see which ones vanish. However, this is rather tedious and not very insightful.
            
            An alternative method is as follows: exploiting the fact that the symmetric bilinear form $(-, -)_{\extendedtoroidal}$ is both invariant and non-degenerate (by construction; cf. convention \ref{conv: orthogonal_complement_of_toroidal_centres}), we can characterise the centre $\hat{\z}_{[2]}$ as the Lie ideal of $\extendedtoroidal$ containing elements $Z$ such that:
                $$0 = ([Z, X]_{\extendedtoroidal}, Y)_{\extendedtoroidal} = (Z, [X, Y]_{\extendedtoroidal})_{\extendedtoroidal}$$
            for any $X, Y \in \extendedtoroidal$, with the first equality holding thanks to the fact that $Z$ is supposed to commute with every other element of $\extendedtoroidal$ by assumption of being central. We are thus left with the task of finding elements:
                $$Z \in \extendedtoroidal$$
            such that:
                $$(Z, [\extendedtoroidal, \extendedtoroidal]_{\extendedtoroidal})_{\extendedtoroidal} = 0$$
            Since brackets of the form:
                $$[X, Y]_{\extendedtoroidal}, [D, D']_{\extendedtoroidal}$$
            (for some $X, Y \in \g_{[2]}$ and some $D, D' \in \d_{[2]}$) are generally non-zero, their elements can not be central in $\extendedtoroidal$. As such, we have narrowed the scope of our search down to:
                $$\hat{\z}_{[2]} \subset \z_{[2]}$$

            Another way to see that:
                $$\hat{\z}_{[2]} \subset \z_{[2]}$$
            is to use the fact that $\extendedtoroidal$ is a Lie algebra extension of $\d_{[2]}$ by $\toroidal$ (cf. theorem \ref{theorem: extended_toroidal_lie_algebras}). This tells us that the centre of $\extendedtoroidal$ ought to lie inside that of $\toroidal$, i.e.:
                $$\hat{\z}_{[2]} \subset \z(\toroidal) = \z_{[2]}$$
            as per proposition \ref{prop: lie_brackets_on_extensions}.
        \end{remark}
        \begin{proposition}[Centres of extended toroidal Lie algebras] \label{prop: centres_of_extended_toroidal_lie_algebras}
            The centre $\hat{\z}_{[2]}$ is a two-dimensional (abelian) Lie subalgebra of $\z_{[2]}$, spanned by $c_v$ and $c_t$. 
        \end{proposition}
            \begin{proof}
                Since we know that:
                    $$\hat{\z}_{[2]} \subset \z_{[2]}$$
                and that the only possibly non-zero bracket with elements of $\z_{[2]}$ are elements of $[\d_{[2]}, \z_{[2]}]_{\extendedtoroidal}$, and since we also know from lemma \ref{lemma: explicit_commutators_between_central_basis_elements_and_derivations} that:
                    $$[\d_{[2]}, K]_{\extendedtoroidal} = 0 \iff K \in k c_v \oplus k c_v$$
                we can conclude immediately that:
                    $$\hat{\z}_{[2]} = k c_v \oplus k c_t$$
            \end{proof}
        \begin{remark}
            It is rather interesting that:
                $$\hat{\z}_{[2]} \cong k c_v \oplus k c_t$$
            as this is in good analogy with the affine Kac-Moody case, where the centre of $\hat{\g}$ is $1$-dimensional, namely spanned by $c_v$ (cf. example \ref{example: affine_lie_algebras_centres}).
        \end{remark}