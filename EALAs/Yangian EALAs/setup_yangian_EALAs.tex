\section{Setup and overview of results}
    Again, $\g$ is a finite-dimensional simple Lie algebra over an algebraically closed field of characteristic $0$. It is accompanied by all the data recalled in subsection \ref{subsection: finite_dimensional_simple_lie_algebras}. 

    \subsection{Conventions for toroidal Lie algebras} \label{subsection: toroidal_lie_algebra_conventions}
        \todo[inline]{All toroidal Lie algebra conventions have been moved here.}
        For our own convenience, we will also be adopting the following abbreviations:
            $$A := \bbC[v^{\pm 1}, t^{\pm 1}], \bar{\Omega}_{[2]} := \bar{\Omega}^1_{A/\bbC}$$
        and also that:
            $$\g_{[2]} := \g \tensor_{\bbC} A$$
        being understood as current algebras (in the sense of definition \ref{def: current_algebras}).
    
        We will then be interested in the Lie algebra:
            $$\toroidal := \uce(\g_{[2]})$$
        which, respectively, shall be referred to as the \textbf{toroidal Lie algebra} associated to $\g$. In example \ref{example: toroidal_lie_algebras_centres} and remark \ref{remark: Z_gradings_on_toroidal_lie_algebras}, the structures of the underlying ($\Z$-graded) vector spaces of these Lie algebras have already been described, and we refer the reader there for the details (in particular, the construction of a canonical basis for the underlying vector spaces of their centres). So that our notations would be suggestive, we shall be writing:
            $$\z_{[2]} := \z(\toroidal)$$
        from now on; often, we might refer to these as the \textbf{(positive) toroidal centres}.
        
        Now, instead of being equipped with the usual residue bilinear form of degree $(0, 0)$, the Lie algebra $\g_{[2]}$ will be equipped with the residue bilinear form of degree $(0, -1)$:
            $$(x v^m t^p, y v^n t^q)_{\g_{[2]}} := (x, y)_{\g} \delta_{(m, p) + (n, q), (0, -1)}$$
        given for all $x, y \in \g$ and all $(m, p), (n, q) \in \Z^2$. It is easy to see that the bilinear form:
            $$(-, -)_{\g_{[2]}}$$
        is symmetric, non-degenerate, and invariant. Because $\toroidal$ has a non-trivial centre, any invariant (symmetric) bilinear form thereon (so in particular, any extension of $(-, -)_{\g_{[2]}}$ to $\toroidal$) is necessarily degenerate (see remark \ref{remark: extending_bilinear_forms_to_central_extensions}). The purpose of constructing \say{Yangian extended toroidal Lie algebras} is to remedy such degeneracy.

    \subsection{Outline of the construction and classification of Yangian extended toroidal Lie algebras}
        The point of constructing the so-called \say{Yangian extended toroidal Lie algebras} is to fix the issue, whereby any \textit{invariant}\footnote{Let $\a$ be a perfect Lie algebra with non-zero centre and $(-, -)$ be such an invariant symmetric bilinear form thereon. Then, for all $x, y \in \a$ and all $K \in \z(\a)$, we will have that $(K, [x, y]) = ([K, x], y) = (0, y) = 0$, thus proving that $(-, -)$ is degenerate. The assumption that $\a$ is perfect is important here because otherwise, not every element thereof can be written as $[x, y]$, and therefore $(K, [x, y]) = 0$ for all $K \in \z(\a)$ and all $x, y \in \a$ would not imply that $(-, -)$ has a non-zero radical.} symmetric bilinear form on:
            $$\toroidal:= \uce(\g_{[2]})$$
        is necessarily degenerate. We do this by formally introducing a \say{complementary} vector space $\d_{[2]}$ whose elements shall pair non-degenerately with those of $\z_{[2]}$. We will see that these complementary elements are in fact certain $\bbC$-linear derivations on $A$, which then allows us to show that $\extendedtoroidal$ is a $\der(A)$-module. Moreover, $\d_{[2]}$ can be shown to be closed under the commutator bracket, and hence a Lie subalgebra of $\der(A)$. Consequently, $\extendedtoroidal$ is a $\d_{[2]}$-module. If we can then also show that $\extendedtoroidal$ arises as a Lie algebra extension of the form:
            $$0 \to \toroidal \to \extendedtoroidal \to \d_{[2]} \to 0$$
        then the hypotheses proposition \ref{prop: twisted_semi_direct_product_criterion} will have been met. The proposition will then tell us that:
            $$\extendedtoroidal \cong \toroidal \rtimes^{\sigma} \d_{[2]}$$
        for some $2$-cocycle:
            $$\sigma: \bigwedge^2 \d \to \toroidal$$
        We will also see that it is enough to specify a $2$-cocycle:
            $$\sigma: \bigwedge^2 \d_{[2]} \to \z_{[2]}$$
        which is better, since $\z_{[2]}$ is abelian, unlike $\toroidal$.
        
        \begin{convention} \label{conv: orthogonal_complement_of_toroidal_centres}
            From now on, let us write:
                $$\d_{[2]} := \z_{[2]}^{\star}$$
            for the $\Z$-graded dual of $\z_{[2]}$, on which the grading is as in remark \ref{remark: Z_gradings_on_toroidal_lie_algebras}. In other words, $\d_{[2]}$ shall be the $\bbC$-vector space:
                $$\d_{[2]} := ( \bigoplus_{(r, s) \in \Z^2} \bbC D_{r, s} ) \oplus \bbC D_v \oplus \bbC D_t$$
            where the basis elements $D_{r, s}, D_v, D_t$ are graded-dual to the elements $K_{r, s}, c_v, c_t$, respectively, in the sense that:
                $$D_{r, s}(K_{r, s}) = D_v(c_v) = D_t(c_t) = 1$$

            This allows us to endow the vector space:
                $$\extendedtoroidal := \toroidal \oplus \d_{[2]}$$
            with a \textit{non-degenerate} symmetric bilinear form:
                $$(-, -)_{\extendedtoroidal}$$
            such that:
                $$(X, D)_{\extendedtoroidal} = (X, K)_{\extendedtoroidal} = 0$$
                $$(D, K)_{\extendedtoroidal} := D(K)$$
            for all $X \in \g_{[2]}$, $K \in \z_{[2]}$, and $D \in \d_{[2]}$. We shall also insist that the restriction of $(-, -)_{\extendedtoroidal}$ to the vector subspace $\g_{[2]}$ coincides with the bilinear form $(-, -)_{\g_{[2]}}$.
        \end{convention}

        In order to show that there is a Lie algebra extension:
            $$0 \to \toroidal \to \extendedtoroidal \to \d_{[2]} \to 0$$
        we must first of all endow $\d_{[2]}$ with a Lie algebra structure, which we shall do by identifying $\d_{[2]}$ with a certain intrinsically defined Lie subalgebra:
            $$\der(A)_{(v, t) = (0, 0)}$$
        of the Lie algebra $\der(A)$ of derivations on $A$ (cf. lemmas \ref{lemma: yangian_div_zero_vector_fields} and \ref{lemma: derivation_action_on_multiloop_algebras}, and proposition \ref{prop: intrinsic_description_of_yangian_div_zero_vector_fields}), which shall always be equipped with the commutator bracket. After this, we shall construct Lie brackets:
            $$[-, -]_{\extendedtoroidal}$$
        on the vector space $\extendedtoroidal := \toroidal \oplus \d_{[2]}$. We shall additionally require that any Lie bracket $[-, -]_{\extendedtoroidal}$ satisfies the following properties:
        \begin{itemize}
            \item The bilinear form $(-, -)_{\extendedtoroidal}$ is invariant with respect to $[-, -]_{\extendedtoroidal}$, and
            \item $\toroidal := \uce(\g_{[2]})$ with its Lie bracket as in theorem \ref{theorem: kassel_realisation} embeds naturally as a Lie subalgebra into $\extendedtoroidal$ with the bracket $[-, -]_{\extendedtoroidal}$. 
        \end{itemize}

        In order to define the aforementioned Lie subalgebra:
            $$\der(A)_{(v, t) = (0, 0)}$$
        of $\der(A)$, we shall firstly need the notion of formal residues. 
        \begin{definition}[Formal residue] \label{def: formal_residues}
            The \textbf{formal residue} of any element $f(v_1, ..., v_n) \in \bbC[v_1^{\pm 1}, ..., v_n^{\pm 1}]$ is then given by:
                $$\Res_{(v_1, ..., v_n) = (0, ..., 0)}(f) := ( \Res_{v_n = 0} \circ ... \circ \Res_{v_1 = 0} )( f(v_1, ..., v_n) )$$
            Each map:
                $$\Res_{v_i = 0}: \bbC[v_i, ..., v_n] \to \bbC[v_{i + 1}, ..., v_n]$$
            is takes as input an element $f(v_i, ..., v_n) \in \bbC[v_{i + 1}, ..., v_n]$ and outputs the $\bbC[v_{i + 1}, ..., v_n]$-coefficient of the term with $v_i^{-1}$, or $0$ if there is no such term. 
        \end{definition}
        \begin{remark}[Taking formal residues is linear]
            It is not hard to see that:
                $$\Res_{(v_1, ..., v_n) = (0, ..., 0)}( v_1^{m_1} ... v_n^{m_n} ) = \delta_{(m_1, ..., m_n), (-1, ..., -1)}$$
            From this, one sees that:
                $$\Res_{(v_1, ..., v_n) = (0, ..., 0)}: \bbC[v_1, ..., v_n] \to k$$
            is a linear map.
        \end{remark}
        \begin{lemma}[Yangian \say{divergence-zero} vector fields] \label{lemma: yangian_div_zero_vector_fields}
            Let:
                $$\der(A)_{(v, t) = (0, 0)} := \{ D \in \der(A) \mid \forall f \in A: \Res_{(v, t) = (0, 0)}( v^{-1} D(f) ) = 0 \}$$
            This is a Lie subalgebra of $\der(A)$.
        \end{lemma}
            \begin{proof}
                Firstly, we must check that the subset $\der(A)_{(v, t) = (0, 0)}$ of $\der(A)$ is indeed a vector subspace, and secondly, we must check that it is closed under the commutator bracket $[-, -]$, qualifying it as a Lie subalgebra of the latter. To prove the first assertion, simply take any linear combination:
                    $$a D + b D'$$
                of elements $D, D' \in \der(A)_{(v, t) = (0, 0)}$, and then consider the following for any arbitrary $f \in A$:
                    $$\Res_{(v, t) = (0, 0)}( v^{-1} ( a D + b D' )(f) ) = a \Res_{(v, t) = (0, 0)}( v^{-1} D(f) ) + b\Res_{(v, t) = (0, 0)}( v^{-1} D'(f) ) = 0$$
                To prove the second assertion, pick arbitrary elements $D, D' \in \der(A)_{(v, t) = (0, 0)}$ again, and then consider the following for any $f \in A$:
                    $$\Res_{(v, t) = (0, 0)}( v^{-1} ([D, D'])(f) ) = \Res_{(v, t) = (0, 0)}( v^{-1} DD'(f) ) - \Res_{(v, t) = (0, 0)}( v^{-1} D'D(f) ) = 0$$
            \end{proof}
        Next, let us see that elements of $\d_{[2]}$ are certain derivations on $D$, and hence $\d_{[2]}$ is identifiable as a vector subspace of $\der(A)$.
        \begin{lemma}[$\d_{[2]}$ acts on $\g_{[2]}$ by derivations] \label{lemma: derivation_action_on_multiloop_algebras}
            Suppose that the vector space:
                $$\extendedtoroidal := \toroidal \oplus \d_{[2]}$$
            is endowed with some Lie bracket $[-, -]_{\extendedtoroidal}$. Then, the elements:
                $$D \in \d_{[2]}$$
            necessarily act as derivations on $A$, in the sense that:
                $$[D, xf]_{\extendedtoroidal} = x \xi_D(f) + K_{D, xf}$$
            where $\xi_D \in \der(A)$ is a derivation on $A$ that uniquely depends on the choice of $D \in \d_{[2]}$ and $K_{D, xf} \in \z_{[2]}$ is yet to be determined explicitly (see corollary \ref{coro: derivation_action_on_multiloop_algebras}, where it is shown that in fact, these elements of $\z_{[2]}$ are necessarily $0$). In particular, the basis elements $D_{r, s}$ (for any $(r, s) \in \Z^2$) and $D_v, D_t$ of $\d_{[2]}$ act as follows on the generating elements $x v^m t^p \in \g_{[2]}$ (for some $x \in \g$ and $(m, p) \in \Z^2$), respectively:
                $$[D_{r, s}, x v^m t^p]_{\extendedtoroidal} = ( rp - ms ) x v^{m - r} t^{p - s - 1} + K_{(m, p), (r, s)}(x)$$
                $$[D_v, x v^m t^p]_{\extendedtoroidal} = -m x v^m t^{p - 1} + K_{m, p}(x)$$
                $$[D_t, x v^m t^p]_{\extendedtoroidal} = -p x v^m t^{p - 1} + K_{m, p}(x)$$
            where $K_{(m, p), (r, s)}(x), K_{m, p}(x), K_{m, p}(x) \in \z_{[2]}$ are the undetermined summands.
        \end{lemma}
            \begin{proof}
                Let us first see how the basis elements of $\d_{[2]}$ act on those of $\g_{[2]}$ before worrying about whether or not elements of the former are indeed derivations.
            
                We begin by fixing $x, y \in \g$, $(m, p), (n, q) \in \Z^2$, along with some $D \in \d_{[2]}$. Then, consider the following:
                    $$
                        \begin{aligned}
                            ( D, [x v^m t^p, y v^n t^q]_{\toroidal} )_{\extendedtoroidal} & = ( D, [x, y]_{\g} v^{m + n} t^{p + q} + (x, y)_{\g} v^n t^q \bar{d}( v^m t^p ) )_{\extendedtoroidal}
                            \\
                            & = (x, y)_{\g} ( D, v^n t^q \bar{d}( v^m t^p ) )_{\extendedtoroidal}
                            \\
                            & = (x, y)_{\g} ( D, \delta_{(m, p) + (n, q), (0, 0)} ( n c_v + q c_t ) + (np - mq) K_{m + n, p + q} )_{\extendedtoroidal}
                        \end{aligned}
                    $$
                Now, without any loss of generality, let us suppose that $D \in \d_{[2]}$ is some basis element, i.e.:
                    $$D \in \{ D_{r, s}, D_v, D_t \}$$
                and consider these cases separately, for the sake of clarity:
                \begin{enumerate}
                    \item \textbf{(Case 1: $D := D_{r, s}$):} Fix some $(r, s) \in \Z^2$ and consider the following: 
                        $$
                            \begin{aligned}
                                ( D_{r, s}, [x v^m t^p, y v^n t^q]_{\toroidal} )_{\extendedtoroidal} & = (x, y)_{\g} ( D_{r, s}, \delta_{(m, p) + (n, q), (0, 0)} ( n c_v + q c_t ) + (np - mq) K_{m + n, p + q} )_{\extendedtoroidal}
                                \\
                                & = (x, y)_{\g} (np - mq) \delta_{(r, s), (m + n, p + q)}
                            \end{aligned}
                        $$
                    The assumption that $(-, -)_{\extendedtoroidal}$ is invariant with respect to $[-, -]_{\extendedtoroidal}$ then implies that:
                        $$( [D_{r, s}, x v^m t^p]_{\extendedtoroidal}, y v^n t^q )_{\extendedtoroidal} = (x, y)_{\g} (np - mq) \delta_{(r, s), (m + n, p + q)}$$
                    Now, suppose that:
                        $$[D_{r, s}, x v^m t^p]_{\extendedtoroidal} := \sum_{(a, b) \in \Z^2} \lambda_{a, b}(x) v^a t^b + K_{(m, p), (r, s)}(x) + \xi_{(m, p), (r, s)}(x)$$
                    for some $\lambda_{a, b}(x) \in \g$, $K_{(m, p), (r, s)}(x) \in \z_{[2]}$, and $\xi_{(m, p), (r, s)}(x) \in \d_{[2]}$, depending on our choices of $x \in \g$ and $(m, p) \in \Z^2$. Next, consider the following:
                        $$
                            \begin{aligned}
                                ( [D_v, x v^m t^p]_{\extendedtoroidal}, y v^n t^q )_{\extendedtoroidal} & = \left( \sum_{(a, b) \in \Z^2} \lambda_{a, b}(x) v^a t^b + K_{(m, p), (r, s)}(x) + \xi_{(m, p), (r, s)}(x), y v^n t^q \right)_{\extendedtoroidal}
                                \\
                                & = \sum_{(a, b) \in \Z^2} \left( \lambda_{a, b}(x) v^a t^b, y v^n t^q \right)_{\g_{[2]}}
                                \\
                                & = -\sum_{(a, b) \in \Z^2} (\lambda_{a, b}(x), y)_{\g} \delta_{ (a, b) + (n, q), (0, -1) }
                                \\
                                & = -(\lambda_{-n, -q - 1}(x), y)_{\g}
                            \end{aligned}
                        $$
                    which tells us that:
                        $$(x, y)_{\g} (np - mq) \delta_{(r, s), (m + n, p + q)} = -(\lambda_{-n, -q - 1}(x), y)_{\g}$$
                    The non-degeneracy of the inner product $(-, -)_{\g}$ as well as the arbitrariness of the choices of $y \in \g$ and $(n, q) \in \Z^2$ then together imply that:
                        $$\lambda_{-n, -q - 1}(x) = -(np - mq) \delta_{(r, s), (m + n, p + q)} = (mq - np) \delta_{(r, s), (m + n, p + q)}$$
                    for any fixed choices of $x \in \g$ and $(m, p) \in \Z^2$. From this, we infer that:
                        $$
                            \begin{aligned}
                                [D_{r, s}, x v^m t^p]_{\extendedtoroidal} & = \sum_{(n, q) \in \Z^2} -(np - mq) \delta_{(r, s), (m + n, p + q)} v^{-n} t^{-q - 1} + K_{(m, p), (r, s)}(x) + \xi_{(m, p), (r, s)}(x)
                                \\
                                & = ( m(s - p) - (r - m)p ) x v^{m - r} t^{p - s - 1} + K_{(m, p), (r, s)}(x) + \xi_{(m, p), (r, s)}(x)
                                \\
                                & = ( ms - rp ) x v^{m - r} t^{p - s - 1} + K_{(m, p), (r, s)}(x) + \xi_{(m, p), (r, s)}(x)
                            \end{aligned}
                        $$
                        
                    We now claim that:
                        $$\xi_{(m, p), (r, s)}(x) = 0$$
                    To this end, consider firstly the following, wherein $Z \in \z_{[2]}$ is an arbitrary choice:
                        $$
                            \begin{aligned}
                                ( [D_{r, s}, x v^m t^p]_{\extendedtoroidal}, Z )_{\extendedtoroidal} & = ( D_{r, s}, [x v^m t^p, Z]_{\toroidal} )_{\extendedtoroidal}
                                \\
                                & = (D, 0)_{\extendedtoroidal}
                                \\
                                & = 0
                            \end{aligned}
                        $$
                    Simultaneously, consider the following:
                        $$
                            \begin{aligned}
                                ( [D_{r, s}, x v^m t^p]_{\extendedtoroidal}, Z )_{\extendedtoroidal} & = \left( \sum_{(a, b) \in \Z^2} \lambda_{a, b}(x) v^a t^b + K_{(m, p), (r, s)}(x) + \xi_{(m, p), (r, s)}(x), Z \right)_{\extendedtoroidal}
                                \\
                                & = ( \xi_{(m, p), (r, s)}(x), Z )_{\extendedtoroidal}
                            \end{aligned}
                        $$
                    The previous observation along with this one imply that:
                        $$( \xi_{(m, p), (r, s)}(x), Z )_{\extendedtoroidal} = 0$$
                    for \textit{any} $Z \in \z_{[2]}$, but since $\d_{[2]}$ is graded-dual to $\z_{[2]}$ by construction, the above implies via the non-degeneracy of the inner product $(-, -)_{\extendedtoroidal}$ that:
                        $$\xi_{(m, p), (r, s)}(x) = 0$$
                    necessarily. 
    
                    We can now conclude that:
                        $$[D_{r, s}, x v^m t^p]_{\extendedtoroidal} = ( rp - ms ) x v^{m - r} t^{p - s - 1} + K_{(m, p), (r, s)}(x)$$
                    \item \textbf{(Case 2: $D := D_v$):} In this case, it is easy to see that:
                        $$
                            \begin{aligned}
                                ( D_v, [x v^m t^p, y v^n t^q]_{\toroidal} )_{\extendedtoroidal} & = (x, y)_{\g} ( D_v, \delta_{(m, p) + (n, q), (0, 0)} ( n c_v + q c_t ) + (np - mq) K_{m + n, p + q} )_{\extendedtoroidal}
                                \\
                                & = (x, y)_{\g} \delta_{(m, p) + (n, q), (0, 0)} n
                            \end{aligned}
                        $$
                    Using invariance, we then see that:
                        $$( [D_v, x v^m t^p]_{\extendedtoroidal}, y v^n t^q )_{\extendedtoroidal} = (x, y)_{\g} \delta_{(m, p) + (n, q), (0, 0)} n$$
                    Now, suppose that:
                        $$[D_v, x v^m t^p]_{\extendedtoroidal} := \sum_{(a, b) \in \Z^2} \lambda_{a, b}(x) v^a t^b + K_{m, p}(x) + \xi_{m, p}(x)$$
                    for some $\lambda_{a, b}(x) \in \g$, $K_{m, p}(x) \in \z_{[2]}$, and $\xi_{m, p}(x) \in \d_{[2]}$, depending on our choices of $x \in \g$ and $(m, p) \in \Z^2$. Then, consider the following:
                        $$
                            \begin{aligned}
                                ( [D_v, x v^m t^p]_{\extendedtoroidal}, y v^n t^q )_{\extendedtoroidal} & = \left( \sum_{(a, b) \in \Z^2} \lambda_{a, b}(x) v^a t^b + K_{m, p}(x) + \xi_{m, p}(x), y v^n t^q \right)_{\extendedtoroidal}
                                \\
                                & = \sum_{(a, b) \in \Z^2} \left( \lambda_{a, b}(x) v^a t^b, y v^n t^q \right)_{\g_{[2]}}
                                \\
                                & = -\sum_{(a, b) \in \Z^2} (\lambda_{a, b}(x), y)_{\g} \delta_{ (a, b) + (n, q), (0, -1) }
                                \\
                                & = -(\lambda_{-n, -q - 1}(x), y)_{\g}
                            \end{aligned}
                        $$
                    From this, we are able to conclude that:
                        $$(x, y)_{\g} \delta_{(m, p) + (n, q), (0, 0)} n = -(\lambda_{-n, -q - 1}(x), y)_{\g}$$
                    As this holds for all $y \in \g$ and all $(n, q) \in \Z^2$, we can infer from the above and from the non-degeneracy of the inner product $(-, -)_{\g}$ that:
                        $$\lambda_{-n, -q - 1}(x) = \delta_{(m, p) + (n, q), (0, 0)} n x$$
                    for any $x \in \g$ and any $(m, p) \in \Z^2$ (both fixed!), and hence:
                        $$
                            \begin{aligned}
                                [D_v, x v^m t^p]_{\extendedtoroidal} & = \sum_{(n, q) \in \Z^2} \delta_{(m, p) + (n, q), (0, 0)} n x v^{-n} t^{-q - 1} + K_{m, p}(x) + \xi_{m, p}(x)
                                \\
                                & = -m x v^m t^{p - 1} + K_{m, p}(x) + \xi_{m, p}(x)
                            \end{aligned}
                        $$
    
                    Now, by arguing as in \textbf{Case 1}, we will see that:
                        $$\xi_{m, p}(x) = 0$$
                    and afterwards we will be able to conclude that:
                        $$[D_v, x v^m t^p]_{\extendedtoroidal} = -m x v^m t^{p - 1} + K_{m, p}(x)$$
                    \item \textbf{(Case 3: $D := D_t$)} Arguing as when $D = D_v$, we will obtain:
                        $$[D_t, x v^m t^p]_{\extendedtoroidal} = -p x v^m t^{p - 1} + K_{m, p}(x)$$
                    for some $K_{m, p}(x) \in \z_{[2]}$.
                \end{enumerate}

                Let us now verify that elements of $\d_{[2]}$ are indeed derivations. We can identify the derivations $D_{r, s}, D_v, D_t$ explicitly in terms of $\del_v := \frac{\del}{\del v}, \del_t := \frac{\del}{\del t}$. For this, let us firstly equip $\der(A)$ - the $\bbC$-vector space of all $\bbC$-linear derivations on $A$ - with the following basis:
                    $$\{ v^m t^p \del_v, v^n t^q \del_t \}_{(m, p), (n, q) \in \Z^2}$$
                \begin{enumerate}
                    \item To compute $D_{r, s}$ in terms of $\del_v, \del_t$, suppose firstly that:
                        $$D_{r, s} := f(v, t) \del_v + g(v, t) \del_t$$
                    with $f(v, t), g(v, t) \in A$. Next, fix some $(m, p) \in \Z^2$ and then consider the following:
                        $$
                            \begin{aligned}
                                D_{r, s}( v^m t^p ) & = f(v, t) \del_v( v^m t^p ) + g(v, t) \del_t( v^m t^p )
                                \\
                                & = f(v, t) m v^{m - 1} t^p + g(v, t) p v^m t^{p - 1}
                            \end{aligned}
                        $$
                    At the same time, we also have that:
                        $$D_{r, s}(v^m t^p) := ( ms - rp ) v^{m - r} t^{p - s - 1}$$
                    and hence:
                        $$f(v, t) m v^{m - 1} t^p + g(v, t) p v^m t^{p - 1} = ( ms - rp ) v^{m - r} t^{p - s - 1}$$
                    From this, one infers that:
                        $$f(v, t) = s v^{-r + 1} t^{-s - 1}, g(v, t) = -r v^{-r} t^{-s}$$
                    and therefore:
                        $$D_{r, s} = s v^{-r + 1} t^{-s - 1} \del_v - r v^{-r} t^{-s} \del_t$$
                    \item One easily checks that:
                        $$D_v = -v t^{-1} \del_v$$
                    \item Likewise:
                        $$D_t = -\del_t$$
                \end{enumerate}
                Consequently, we see that elements of $\d_{[2]}$ are derivations on $A$.
            \end{proof}
        \begin{remark}
            Now that we know that the basis elements $D_{r, s}, D_v, D_t \in \d_{[2]}$ are actually certain derivations on $A$, we can also check that the commutators of the elements $D_{r, s}, D_v, D_t$ are still elements of $\d_{[2]}$. This ensures us that we can \textit{choose} to endow $\d_{[2]}$ with the structure of a Lie subalgebra of $\der(A)$, i.e. the Lie algebra structure such that:
                $$[D, D']_{\extendedtoroidal} \in \d_{[2]}$$
            for any $D, D' \in \d_{[2]}$. In general, however, we are only guaranteed that:
                $$[D, D']_{\extendedtoroidal} = \z_{[2]} \oplus \d_{[2]}$$
            We note also that it is not even clear \textit{a priori} that the $\d_{[2]}$-summand of the commutators of the form $[D, D']_{\extendedtoroidal}$ has to be the usual commutator inherited from $\der(A)$; this turns out to be true, but follows from some non-trivial computations (cf. proposition \ref{prop: lie_bracket_on_orthogonal_complement_of_toroidal_centre}). 
        \end{remark}

        In light of lemmas \ref{lemma: yangian_div_zero_vector_fields} and \ref{lemma: derivation_action_on_multiloop_algebras}, we can now endow $\d_{[2]}$ with the commutator bracket inherited from $\der(A)$ by identifying:
            $$\der(A)_{(v, t) = (0, 0)} = \d_{[2]}$$
        as vector subspaces of $\der(A)$.
        \begin{proposition}[Intrinsic description of $\d_{[2]}$] \label{prop: intrinsic_description_of_yangian_div_zero_vector_fields}
            As vector subspaces of $\der(A)$, one has that:
                $$\der(A)_{(v, t) = (0, 0)} = \d_{[2]}$$
            Since $\der(A)_{(v, t) = (0, 0)}$ is a Lie subalgebra of $\der(A)$, it thus makes sense to endow $\d_{[2]}$ with the commutator bracket inherited from $\der(A)$.
        \end{proposition}
            \begin{proof}
                To prove that:
                    $$\d_{[2]} = \der(A)_{(v, t) = (0, 0)}$$
                it shall suffice to prove that the underlying vector spaces are equal. Firstly, to prove that:
                    $$\d_{[2]} \subset \der(A)_{(v, t) = (0, 0)}$$
                it shall suffice to prove the basis elements $D \in \{D_{r, s}\}_{(r, s) \in \Z^2} \cup \{D_v, D_t\}$ satisfy:
                    $$\Res_{(v, t) = (0, 0)}( v^{-1} D(f) ) = 0$$
                for all $f \in A$, which we might as well take to be a basis element $v^m t^p$ (for some $(m, p) \in \Z^2$). In lemma \ref{lemma: derivation_action_on_multiloop_algebras}, we have already seen how the elements $D(v^m t^p) \in A$ are given for $D \in \d_{[2]}$ a basis vector, from which it can be easily seen that indeed, $-\Res_{(v, t) = (0, 0)}( v^{-1} D(f) ) = 0$. Conversely, to prove that:
                    $$\d_{[2]} \supset \der(A)_{(v, t) = (0, 0)}$$
                we shall need to show that any $D \in \der(A)_{(v, t) = (0, 0)}$ can be written as a linear combination of the elements of the basis $\{D_{r, s}\}_{(r, s) \in \Z^2} \cup \{D_v, D_t\}$. To that end, consider some:
                    $$D := \sum_{(a, b) \in \Z^2} \lambda_{a, b} v^a t^b \del_v + \sum_{(a', b') \in \Z^2} \mu_{a', b'} v^{a'} t^{b'} \del_t \in \der(A)_{(v, t) = (0, 0)}$$
                and then, for an arbitrary $(m, p) \in \Z^2$, consider:
                    $$
                        \begin{aligned}
                            D(v^m t^p) & = \sum_{(a, b) \in \Z^2} \lambda_{a, b} v^a t^b \del_v(v^m t^p) + \sum_{(a', b') \in \Z^2} \mu_{a', b'} v^{a'} t^{b'} \del_t(v^m t^p)
                            \\
                            & = m \sum_{(a, b) \in \Z^2} \lambda_{a, b} v^{a + m - 1} t^{b + p} + p \sum_{(a', b') \in \Z^2} \mu_{a', b'} v^{a' + m} t^{b' + p - 1}
                        \end{aligned}
                    $$
                From this, we infer that:
                    $$
                        \begin{aligned}
                            \Res_{(v, t) = (0, 0)}( v^{-1} D(v^m t^p) ) & = m \sum_{(a, b) \in \Z^2} \lambda_{a, b} \delta_{(a + m, b + p), (-1, -1)} + p \sum_{(a', b') \in \Z^2} \mu_{a', b'} \delta_{(a' + m, b' + p), (0, 0)}
                            \\
                            & = m \lambda_{-m - 1, -p - 1} + p \mu_{-m, -p}
                        \end{aligned}
                    $$
                Since we assumed that $D \in \der(A)_{(v, t) = (0, 0)}$, we must have that:
                    $$0 = \Res_{(v, t) = (0, 0)}( v^{-1} D(v^m t^p) ) = m \lambda_{-m - 1, -p - 1} + p \mu_{-m, -p}$$
                for all $(m, p) \in \Z^2$. By comparing this with how the elements $D_{r, s}, D_v, D_t$ are given in terms of the pari]tial derivatives $\del_v, \del_t$ (cf. lemma \ref{lemma: derivation_action_on_multiloop_algebras}), one sees thus that indeed, we have that:
                    $$D \in \d_{[2]}$$
                Since $D \in \der(A)_{(v, t) = (0, 0)}$ was chosen arbitrarily, this means that:
                    $$\d_{[2]} \supset \der(A)_{(v, t) = (0, 0)}$$
                We can thus conclude that:
                    $$\d_{[2]} = \der(A)_{(v, t) = (0, 0)}$$
                as needed.
            \end{proof}

        \begin{definition}[Yangian extended toroidal Lie algebras] \label{def: yangian_extended_toroidal_lie_algebras}
            A \textbf{Yangian extended toroidal Lie algebra} is a Lie algebra $\fraky$ satisfying the following properties:
            \begin{itemize}
                \item The underlying vector space of $\fraky$ is isomorphic to $\g_{[2]} \oplus \z_{[2]} \oplus \d_{[2]}$.
                \item There is an invariant and non-degenerate symmetric bilinear form $(-, -)_{\fraky}$ whose restriction down to $\toroidal$ coincides with $(-, -)_{\toroidal}$ as subsection \ref{subsection: toroidal_lie_algebra_conventions}.
                \item There exists an injective Lie algebra homomorphism $\toroidal \hookrightarrow \fraky$.
            \end{itemize}
        \end{definition}
        \begin{remark}[Regarding terminologies]
            Our notion of Yangian extended toroidal Lie algebras does not quite coincide with the very similar notion of an \say{toroidal extended affine Lie algebra} (also known as \say{extended affine Lie algebras of nullity $2$}) that appeared, for instance, in \cite{billig_representations_of_toroidal_extended_affine_lie_algebras} and \cite{neher_lectures_on_EALAs}. Ultimately, this is because the bilinear form that we have endowed $\g_{[2]}$ with is of degree $-1$ (as opposed to $0$) in the second variable. For the latter, instead of $\d_{[2]}$, one would consider the Lie algebra of divergence-free algebraic vector fields on the (smooth) affine $\bbC$-scheme $\Spec A \cong \G_m^2$.
        \end{remark}
        
        It now makes sense to speak of Lie algebra extensions:
            $$0 \to \toroidal \to \extendedtoroidal \to \d_{[2]} \to 0$$
        The main theorem of the first half of this chapter is the following result:
        \begin{theorem}[Yangian extended toroidal Lie algebras] \label{theorem: yangian_extended_toroidal_lie_algebras_preliminary_version}
            \begin{enumerate}
                \item Any Yangian toroidal Lie algebra is isomorphic to some twisted semi-direct product:
                    $$\toroidal \rtimes^{\sigma} \d_{[2]}$$
                for some $2$-cocycle of $\d_{[2]}$ with values in $\z_{[2]}$.
                \item Conversely, a given twisted semi-direct product:
                    $$\fraky(\sigma) := \toroidal \rtimes^{\sigma} \d_{[2]}$$
                is a Yangian toroidal Lie algebra if and only if:
                \begin{itemize}
                    \item the codomain of the $2$-cocycle $\sigma: \bigwedge^2 \d_{[2]} \to \toroidal$ lies inside $\z_{[2]}$, and 
                    \item $\sigma$ satisfies invariances in the sense that:
                        $$(\sigma(D, D'), D'')_{\fraky(\sigma)} = (D, \sigma(D', D''))_{\fraky(\sigma)}$$
                    for any triple of elements $D, D', D'' \in \d_{[2]}$.
                \end{itemize}
            \end{enumerate}
        \end{theorem}
        A trivial - but nevertheless very important to us - corollary of this theorem is as follows.
        \begin{corollary}
            The semi-direct product:
                $$\toroidal \rtimes \d_{[2]}$$
            corresponding to $\sigma = 0$, is a Yangian extended toroidal Lie algebra.
        \end{corollary}

        \todo[inline]{Added outline of the proof of main theorems.}
        Now, in order to prove the first part of theorem \ref{theorem: yangian_extended_toroidal_lie_algebras_preliminary_version} (which will eventually be packaged into a sub-result; cf. theorem \ref{theorem: yangian_extended_toroidal_lie_algebras}), we shall proceed in the following steps:
        \begin{enumerate}
            \item Prove that $\d_{[2]}$ is a vector subspace of $\der(A)$ (cf. lemma \ref{lemma: derivation_action_on_multiloop_algebras}).
            \item Prove that $\toroidal$ is a $\der(A)$-module (cf. lemmas \ref{lemma: derivation_action_on_multiloop_algebras} and \ref{lemma: derivation_action_on_toroidal_centres}, which together imply proposition \ref{prop: toroidal_lie_algebras_as_modules_over_vector_field_lie_algebras}).
            \item Prove that for any Lie bracket $[-, -]_{\extendedtoroidal}$ on $\extendedtoroidal$ with respect to which $(-, -)_{\extendedtoroidal}$ is invariant, one shall have for all $D, D' \in \d_{[2]}$ that:
                $$[D, D']_{\extendedtoroidal} = [D, D'] + K(D, D')$$
            where $[-, -]$ means the usual commutator bracket, and $K(D, D') \in \z_{[2]}$ is some element depending on the choices of $D, D'$ (cf. proposition \ref{prop: lie_bracket_on_orthogonal_complement_of_toroidal_centre}). This implies that the vector subspace $\d_{[2]}$ of $\der(A)$ is closed under the usual commutator bracket $[-, -]$ and hence, identifiable with a Lie subalgebra thereof, meaning that we indeed have a Lie algebra extension:
                $$0 \to \toroidal \to \extendedtoroidal \to \d_{[2]} \to 0$$
            Because $\toroidal$ is a $\der(A)$-module, it is also a $\d_{[2]}$-module, and therefore $\extendedtoroidal$ must be a twisted semi-direct product:
                $$\extendedtoroidal \cong \toroidal \rtimes^{\sigma} \d_{[2]}$$
            for some $2$-cocycle $\sigma: \bigwedge^2 \d_{[2]} \to \toroidal$, according to proposition \ref{prop: twisted_semi_direct_product_criterion}.
            \item In lemma \ref{lemma: derivation_action_on_multiloop_algebras}, we can only show that $[\d_{[2]}, \g_{[2]}]_{\extendedtoroidal} \subseteq \toroidal$ but it can now also be shown that:
                $$[\d_{[2]}, \g_{[2]}]_{\extendedtoroidal} \subseteq \g_{[2]}$$
            (cf. corollary \ref{coro: derivation_action_on_multiloop_algebras}). In lemma \ref{lemma: derivation_action_on_toroidal_centres}, it was already shown that:
                $$[\d_{[2]}, \z_{[2]}]_{\extendedtoroidal} \subseteq \z_{[2]}$$
            Thus, we have a direct sum decomposition of $\d_{[2]}$-modules:
                $$\toroidal \cong \g_{[2]} \oplus \z_{[2]}$$
            which, in combination with the fact that $[\d_{[2]}, \d_{[2]}]_{\extendedtoroidal} \subseteq \d_{[2]} \oplus \z_{[2]}$, tells us that the aforementioned $2$-cocycles $\sigma: \bigwedge^2 \d_{[2]} \to \toroidal$ actually is specified by $2$-cocycles:
                $$\sigma: \bigwedge^2 \d_{[2]} \to \z_{[2]}$$
            In other words, we have constructed twisted semi-direct products:
                $$\z_{[2]} \rtimes^{\sigma} \d_{[2]}$$
        \end{enumerate}

        For the classification result, i.e. the second part of theorem \ref{theorem: yangian_extended_toroidal_lie_algebras_preliminary_version}, we will see:
        \begin{enumerate}
            \item that the codomain of any $2$-cocycle $\sigma: \bigwedge^2 \d_{[2]} \to \toroidal$ actually lies in $\z_{[2]}$ via proposition \ref{prop: lie_bracket_on_orthogonal_complement_of_toroidal_centre}, and
            \item that the invariance property:
                $$(\sigma(D, D'), D'')_{\fraky(\sigma)} = (D, \sigma(D', D''))_{\fraky(\sigma)}$$
            is satisfied in theorem \ref{theorem: yangian_criterion_for_toroidal_cocycles}.
        \end{enumerate}
        Unfortunately, we are still not able to say more about the conditions on $\sigma$ itself so that it would be Yangian in the sense of definition \ref{def: yangian_toroidal_cocycles}, but we will be giving a specific example of a non-trivial Yangian cocycle in example \ref{example: yangian_cocycles_(counter)_examples}.