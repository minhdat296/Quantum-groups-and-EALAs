\input{article preambles}

\setcounter{section}{0}

\renewcommand{\cong}{\simeq}
\newcommand{\ladjoint}{\dashv}
\newcommand{\radjoint}{\vdash}
\newcommand{\<}{\langle}
\renewcommand{\>}{\rangle}
\newcommand{\ndiv}{\hspace{-2pt}\not|\hspace{5pt}}
\newcommand{\cond}{\blacktriangle}
\newcommand{\solid}{\blacksquare}
\newcommand{\ot}{\leftarrow}
\renewcommand{\-}{\text{-}}
\renewcommand{\mapsto}{\leadsto}
\renewcommand{\leq}{\leqslant}
\renewcommand{\geq}{\geqslant}
\renewcommand{\setminus}{\smallsetminus}
\makeatletter
\DeclareRobustCommand{\cev}[1]{%
  {\mathpalette\do@cev{#1}}%
}
\newcommand{\do@cev}[2]{%
  \vbox{\offinterlineskip
    \sbox\z@{$\m@th#1 x$}%
    \ialign{##\cr
      \hidewidth\reflectbox{$\m@th#1\vec{}\mkern4mu$}\hidewidth\cr
      \noalign{\kern-\ht\z@}
      $\m@th#1#2$\cr
    }%
  }%
}
\makeatother

\newcommand{\N}{\mathbb{N}}
\newcommand{\Z}{\mathbb{Z}}
\newcommand{\Q}{\mathbb{Q}}
\newcommand{\R}{\mathbb{R}}
\newcommand{\bbC}{\mathbb{C}}
\NewDocumentCommand{\x}{e{_^}}{%
  \mathbin{\mathop{\times}\displaylimits
    \IfValueT{#1}{_{#1}}
    \IfValueT{#2}{^{#2}}
  }%
}
\NewDocumentCommand{\pushout}{e{_^}}{%
  \mathbin{\mathop{\sqcup}\displaylimits
    \IfValueT{#1}{_{#1}}
    \IfValueT{#2}{^{#2}}
  }%
}
\newcommand{\supp}{\operatorname{supp}}
\newcommand{\im}{\operatorname{im}}
\newcommand{\coker}{\operatorname{coker}}
\newcommand{\id}{\mathrm{id}}
\newcommand{\chara}{\operatorname{char}}
\newcommand{\trdeg}{\operatorname{trdeg}}
\newcommand{\rank}{\operatorname{rank}}
\newcommand{\trace}{\operatorname{tr}}
\newcommand{\length}{\operatorname{length}}
\newcommand{\height}{\operatorname{height}}
\renewcommand{\span}{\operatorname{span}}
\newcommand{\e}{\epsilon}
\newcommand{\p}{\mathfrak{p}}
\newcommand{\q}{\mathfrak{q}}
\newcommand{\m}{\mathfrak{m}}
\newcommand{\n}{\mathfrak{n}}
\newcommand{\calF}{\mathcal{F}}
\newcommand{\calG}{\mathcal{G}}
\newcommand{\calO}{\mathcal{O}}
\newcommand{\F}{\mathbb{F}}
\DeclareMathOperator{\lcm}{lcm}
\newcommand{\gr}{\operatorname{gr}}
\newcommand{\vol}{\mathrm{vol}}
\newcommand{\ord}{\operatorname{ord}}

\newcommand{\GL}{\operatorname{GL}}
\newcommand{\SL}{\operatorname{SL}}
\newcommand{\Sp}{\operatorname{Sp}}
\newcommand{\GSp}{\operatorname{GSp}}
\newcommand{\GSpin}{\operatorname{GSpin}}
\newcommand{\opO}{\operatorname{O}}
\newcommand{\SO}{\operatorname{SO}}
\newcommand{\SU}{\operatorname{SU}}
\newcommand{\opU}{\operatorname{U}}
\newcommand{\Spec}{\mathrm{Spec}}
\newcommand{\Spf}{\mathrm{Spf}}
\newcommand{\Spm}{\mathrm{Spm}}
\newcommand{\Spv}{\mathrm{Spv}}
\newcommand{\Spa}{\mathrm{Spa}}
\newcommand{\Spd}{\mathrm{Spd}}
\newcommand{\Proj}{\mathrm{Proj}}
\newcommand{\Gr}{\mathrm{Gr}}
\newcommand{\Hecke}{\mathrm{Hecke}}
\newcommand{\Sht}{\mathrm{Sht}}
\newcommand{\Quot}{\mathrm{Quot}}
\newcommand{\Hilb}{\mathrm{Hilb}}
\newcommand{\Pic}{\mathrm{Pic}}
\newcommand{\Div}{\mathrm{Div}}
\newcommand{\Jac}{\mathrm{Jac}}
\newcommand{\Alb}{\mathrm{Alb}} %albanese variety
\newcommand{\Bun}{\mathrm{Bun}}
\newcommand{\loopspace}{\mathbf{\Omega}}
\newcommand{\suspension}{\mathbf{\Sigma}}
\newcommand{\tangent}{\mathrm{T}} %tangent space
\newcommand{\Eig}{\mathrm{Eig}}

\newcommand{\Ring}{\mathrm{Ring}}
\newcommand{\Cring}{\mathrm{CRing}}
\newcommand{\Alg}{\mathrm{Alg}}
\newcommand{\Leib}{\mathrm{Leib}} %leibniz algebras
\newcommand{\Fld}{\mathrm{Fld}}
\newcommand{\Sets}{\mathrm{Sets}}
\newcommand{\Cat}{\mathrm{Cat}}
\newcommand{\Grp}{\mathrm{Grp}}
\newcommand{\Ab}{\mathrm{Ab}}
\newcommand{\Sch}{\mathrm{Sch}}
\newcommand{\Coh}{\mathrm{Coh}}
\newcommand{\QCoh}{\mathrm{QCoh}}
\newcommand{\Desc}{\mathrm{Desc}}
\newcommand{\Sh}{\mathrm{Sh}}
\newcommand{\Psh}{\mathrm{PSh}}
\newcommand{\Fib}{\mathrm{Fib}}
\renewcommand{\mod}{\-\mathrm{mod}}
\newcommand{\comod}{\-\mathrm{comod}}
\newcommand{\bimod}{\-\mathrm{bimod}}
\newcommand{\Vect}{\mathrm{Vect}}
\newcommand{\Rep}{\mathrm{Rep}}
\newcommand{\Grpd}{\mathrm{Grpd}}
\newcommand{\Arr}{\mathrm{Arr}}
\newcommand{\Esp}{\mathrm{Esp}}
\newcommand{\Ob}{\mathrm{Ob}}
\newcommand{\Mor}{\mathrm{Mor}}
\newcommand{\Mfd}{\mathrm{Mfd}}
%\newcommand{\LR}{\mathrm{LR}}
%\newcommand{\RSpc}{\mathrm{RSpc}}
\newcommand{\Spc}{\mathrm{Spc}}
\newcommand{\Top}{\mathrm{Top}}
\newcommand{\Topos}{\mathrm{Topos}}
\newcommand{\Nil}{\mathfrak{Nil}}
\newcommand{\J}{\mathfrak{J}}
\newcommand{\Stk}{\mathrm{Stk}}
\newcommand{\Pre}{\mathrm{Pre}}
\newcommand{\simp}{\mathbf{\Delta}}
\newcommand{\Ind}{\mathrm{Ind}}
\newcommand{\Pro}{\mathrm{Pro}}
\newcommand{\Mon}{\mathrm{Mon}}
\newcommand{\Comm}{\mathrm{Comm}}
\newcommand{\Fin}{\mathrm{Fin}}
\newcommand{\Assoc}{\mathrm{Assoc}}
\newcommand{\Co}{\mathrm{Co}}
\newcommand{\Loc}{\mathrm{Loc}}
\newcommand{\Ringed}{\mathrm{Ringed}}
\newcommand{\Comp}{\mathrm{Comp}} %compact hausdorff spaces
\newcommand{\Stone}{\mathrm{Stone}} %stone spaces
\newcommand{\sfExt}{\mathrm{Ext}} %extremely disconnected spaces
\newcommand{\Ouv}{\mathrm{Ouv}}
\newcommand{\Str}{\mathrm{Str}}
\newcommand{\Func}{\mathrm{Func}}
\newcommand{\Crys}{\mathrm{Crys}}
\newcommand{\LocSys}{\mathrm{LocSys}}
\newcommand{\Sieves}{\mathrm{Sieves}}
\newcommand{\pt}{\mathrm{pt}}
\newcommand{\Graphs}{\mathrm{Graphs}}
\newcommand{\Lie}{\mathrm{Lie}}
\newcommand{\Env}{\mathrm{Env}}
\newcommand{\Ho}{\mathrm{Ho}}
\newcommand{\rmD}{\mathrm{D}}
\newcommand{\Cov}{\mathrm{Cov}}
\newcommand{\Frames}{\mathrm{Frames}}
\newcommand{\Locales}{\mathrm{Locales}}
\newcommand{\Span}{\mathrm{Span}}
\newcommand{\Corr}{\mathrm{Corr}}
\newcommand{\Monad}{\mathrm{Monad}}
\newcommand{\Var}{\mathrm{Var}}
\newcommand{\sfN}{\mathrm{N}} %nerve
\newcommand{\Dia}{\mathrm{Dia}}
\newcommand{\co}{\mathrm{co}}
\newcommand{\ev}{\mathrm{ev}}
\newcommand{\bi}{\mathrm{bi}}
\newcommand{\Nat}{\mathrm{Nat}}
\newcommand{\Hopf}{\mathrm{Hopf}}
\newcommand{\Dmod}{\mathrm{D}\mod}
\newcommand{\Perv}{\mathrm{Perv}}
\newcommand{\Sph}{\mathrm{Sph}}
\newcommand{\Moduli}{\mathrm{Moduli}}
\newcommand{\Pseudo}{\mathrm{Pseudo}}
\newcommand{\Lax}{\mathrm{Lax}}
\newcommand{\Strict}{\mathrm{Strict}}
\newcommand{\Opd}{\mathrm{Opd}} %operads
\newcommand{\Shv}{\mathrm{Shv}}
\newcommand{\Char}{\mathrm{Char}} %CharShv = character sheaves
\newcommand{\Huber}{\mathrm{Huber}}
\newcommand{\Tate}{\mathrm{Tate}}
\newcommand{\Ad}{\mathrm{Ad}} %adic spaces
\newcommand{\Perfd}{\mathrm{Perfd}} %perfectoid spaces
\newcommand{\Sub}{\mathrm{Sub}} %subobjects
\newcommand{\Ideals}{\mathrm{Ideals}}
\newcommand{\Isoc}{\mathrm{Isoc}}
\newcommand{\Ban}{\-\mathrm{Ban}} %Banach spaces
\newcommand{\Fre}{\-\mathrm{Fre}} %Frechet spaces
\newcommand{\Ch}{\mathrm{Ch}} %chain complexes
\newcommand{\Mot}{\mathrm{Mot}} %motives
\newcommand{\KL}{\mathrm{KL}} %category of Kazhdan-Lusztig modules
\newcommand{\Pres}{\mathrm{Pres}} %presentable categories
\newcommand{\Noohi}{\mathrm{Noohi}} %category of Noohi groups
\newcommand{\Inf}{\mathrm{Inf}}

\newcommand{\Aut}{\mathrm{Aut}}
\newcommand{\Inn}{\mathrm{Inn}}
\newcommand{\Out}{\mathrm{Out}}
\newcommand{\frakgl}{\mathfrak{gl}}
\newcommand{\der}{\mathfrak{der}} %derivations on Lie algebras
\newcommand{\inn}{\mathfrak{inn}} %inner derivations
\newcommand{\out}{\mathfrak{out}} %outer derivations
\newcommand{\Stab}{\mathrm{Stab}}
\newcommand{\Cent}{\mathrm{Cent}}
\newcommand{\Norm}{\mathrm{Norm}}
\newcommand{\Rad}{\mathrm{Rad}}
\newcommand{\Transporter}{\mathrm{Transp}} %transporter between two subsets of a group
\newcommand{\Conj}{\mathrm{Conj}}
\newcommand{\Diag}{\mathrm{Diag}}
\newcommand{\Gal}{\mathrm{Gal}}
\newcommand{\bfG}{\mathbf{G}} %absolute Galois group
\newcommand{\Frac}{\mathrm{Frac}}
\newcommand{\Ann}{\mathrm{Ann}}
\newcommand{\Val}{\mathrm{Val}}
\newcommand{\Chow}{\mathrm{Chow}}
\newcommand{\Sym}{\mathrm{Sym}}
\newcommand{\End}{\mathrm{End}}
\newcommand{\Mat}{\mathrm{Mat}}
\newcommand{\Diff}{\mathrm{Diff}}
\newcommand{\Autom}{\mathrm{Autom}}
\newcommand{\Artin}{\mathrm{Artin}} %artin maps
\newcommand{\sk}{\mathrm{sk}} %skeleton of a category
\newcommand{\eqv}{\mathrm{eqv}} %functor that maps groups $G$ to $G$-sets
\newcommand{\Inert}{\mathrm{Inert}}
\newcommand{\Fil}{\mathrm{Fil}}

\newcommand{\colim}{\operatorname{colim} \:}
\renewcommand{\lim}{\operatorname{lim} \:}
\newcommand{\toto}{\rightrightarrows}
%\newcommand{\tensor}{\otimes}
\NewDocumentCommand{\tensor}{e{_^}}{%
  \mathbin{\mathop{\otimes}\displaylimits
    \IfValueT{#1}{_{#1}}
    \IfValueT{#2}{^{#2}}
  }%
}
\newcommand{\eq}{\operatorname{eq}}
\newcommand{\coeq}{\operatorname{coeq}}
\newcommand{\Hom}{\mathrm{Hom}}
\newcommand{\Maps}{\mathrm{Maps}}
\newcommand{\Tor}{\mathrm{Tor}}
\newcommand{\Ext}{\mathrm{Ext}}
\newcommand{\Isom}{\mathrm{Isom}}
\newcommand{\stalk}{\mathrm{stalk}}
\newcommand{\RKE}{\operatorname{RKE}}
\newcommand{\LKE}{\operatorname{LKE}}
\newcommand{\oblv}{\mathrm{oblv}}
\newcommand{\const}{\mathrm{const}}
%\newcommand{\forget}{\mathrm{forget}}
\newcommand{\adrep}{\mathrm{ad}} %adjoint representation
\newcommand{\NL}{\mathbb{NL}} %naive cotangent complex
\newcommand{\pr}{\operatorname{pr}}
\newcommand{\Der}{\mathrm{Der}}
\newcommand{\Frob}{\mathrm{Frob}} %Frobenius
\newcommand{\frob}{\mathrm{f}} %trace of Frobenius
\newcommand{\bfpt}{\mathbf{pt}}
\newcommand{\bfloc}{\mathbf{loc}}
\DeclareMathAlphabet{\mymathbb}{U}{BOONDOX-ds}{m}{n}
\newcommand{\0}{\mymathbb{0}}
\newcommand{\1}{\mathbbm{1}}
\newcommand{\2}{\mathbbm{2}}
\newcommand{\Jet}{\mathrm{Jet}}
\newcommand{\Split}{\mathrm{Split}}
\newcommand{\Sq}{\mathrm{Sq}}
\newcommand{\Zero}{\mathrm{Z}}
\newcommand{\SqZ}{\Sq\Zero}
\newcommand{\frakLie}{\mathfrak{Lie}}
\newcommand{\y}{\mathrm{y}} %yoneda
\newcommand{\Sm}{\mathrm{Sm}}
\newcommand{\AJ}{\phi} %abel-jacobi map
\newcommand{\act}{\mathrm{act}}
\newcommand{\ram}{\mathrm{ram}} %ramification index
\newcommand{\inv}{\mathrm{inv}}

\newcommand{\bbU}{\mathbb{U}}
\newcommand{\V}{\mathbb{V}}
\newcommand{\U}{\mathrm{U}}
\newcommand{\calU}{\mathcal{U}}
\newcommand{\calW}{\mathcal{W}}
\newcommand{\rmI}{\mathrm{I}} %augmentation ideal
\newcommand{\bfV}{\mathbf{V}}
\newcommand{\C}{\mathcal{C}}
\newcommand{\D}{\mathcal{D}}
\newcommand{\T}{\mathscr{T}} %Tate modules
\newcommand{\calM}{\mathcal{M}}
\newcommand{\calN}{\mathcal{N}}
\newcommand{\calP}{\mathcal{P}}
\newcommand{\calQ}{\mathcal{Q}}
\newcommand{\A}{\mathbb{A}}
\renewcommand{\P}{\mathbb{P}}
\newcommand{\calL}{\mathcal{L}}
\newcommand{\E}{\mathcal{E}}
\renewcommand{\H}{\mathbf{H}}
\newcommand{\scrS}{\mathscr{S}}
\newcommand{\calX}{\mathcal{X}}
\newcommand{\calY}{\mathcal{Y}}
\newcommand{\calZ}{\mathcal{Z}}
\newcommand{\calS}{\mathcal{S}}
\newcommand{\calR}{\mathcal{R}}
\newcommand{\scrX}{\mathscr{X}}
\newcommand{\scrY}{\mathscr{Y}}
\newcommand{\scrZ}{\mathscr{Z}}
\newcommand{\calA}{\mathcal{A}}
\newcommand{\calB}{\mathcal{B}}
\newcommand{\sfT}{\mathrm{T}}
\renewcommand{\S}{\mathcal{S}}
\newcommand{\B}{\mathbb{B}}
\newcommand{\bbD}{\mathbb{D}}
\newcommand{\G}{\mathbb{G}}
\newcommand{\horn}{\mathbf{\Lambda}}
\renewcommand{\L}{\mathbb{L}}
\renewcommand{\a}{\mathfrak{a}}
\renewcommand{\b}{\mathfrak{b}}
\renewcommand{\t}{\mathfrak{t}}
\renewcommand{\r}{\mathfrak{r}}
\newcommand{\bbX}{\mathbb{X}}
\newcommand{\g}{\mathfrak{g}}
\newcommand{\h}{\mathfrak{h}}
\renewcommand{\k}{\mathfrak{k}}
\newcommand{\del}{\partial}
\newcommand{\bbE}{\mathbb{E}}
\newcommand{\scrO}{\mathscr{O}}
\newcommand{\bbO}{\mathbb{O}}
\newcommand{\scrA}{\mathscr{A}}
\newcommand{\scrB}{\mathscr{B}}
\newcommand{\scrF}{\mathscr{F}}
\newcommand{\scrG}{\mathscr{G}}
\newcommand{\scrM}{\mathscr{M}}
\newcommand{\scrN}{\mathscr{N}}
\newcommand{\scrP}{\mathscr{P}}
\newcommand{\frakS}{\mathfrak{S}}
\newcommand{\calI}{\mathcal{I}}
\newcommand{\calJ}{\mathcal{J}}
\newcommand{\scrK}{\mathscr{K}}
\newcommand{\calK}{\mathcal{K}}
\newcommand{\scrV}{\mathscr{V}}
\newcommand{\bbS}{\mathbb{S}}
\newcommand{\scrH}{\mathscr{H}}
\newcommand{\bfB}{\mathbf{B}}
\newcommand{\Witt}{W}
%\newcommand{\bfA}{\mathbf{A}}
\renewcommand{\O}{\mathbb{O}}
\newcommand{\calV}{\mathcal{V}}
\newcommand{\scrR}{\mathscr{R}} %radical
\newcommand{\rmZ}{\mathrm{Z}} %centre of algebra
\newcommand{\bfGamma}{\mathbf{\Gamma}}
\newcommand{\scrU}{\mathscr{U}}
\newcommand{\rmW}{\mathrm{W}} %Weil group
\newcommand{\frakM}{\mathfrak{M}}
\newcommand{\frakN}{\mathfrak{N}}
\newcommand{\frakX}{\mathfrak{X}}
\newcommand{\frakY}{\mathfrak{Y}}
\newcommand{\frakZ}{\mathfrak{Z}}

\newcommand{\aff}{\mathrm{aff}}
\newcommand{\ft}{\mathrm{ft}} %finite type
\newcommand{\fp}{\mathrm{fp}} %finite presentation
\newcommand{\aft}{\mathrm{aft}}
\newcommand{\lft}{\mathrm{lft}}
\newcommand{\laft}{\mathrm{laft}}
\newcommand{\cmpt}{\mathrm{cmpt}}
\newcommand{\qc}{\mathrm{qc}}
\newcommand{\qs}{\mathrm{qs}}
\newcommand{\lcmpt}{\mathrm{lcmpt}}
%\newcommand{\conv}{\mathrm{conv}}
\newcommand{\red}{\mathrm{red}}
\newcommand{\fin}{\mathrm{fin}}
\newcommand{\gen}{\mathrm{gen}}
\newcommand{\petit}{\mathrm{petit}}
\newcommand{\gros}{\mathrm{gros}}
\newcommand{\loc}{\mathrm{loc}}
\newcommand{\glob}{\mathrm{glob}}
%\newcommand{\ringed}{\mathrm{ringed}}
\newcommand{\qcoh}{\mathrm{qcoh}}
\newcommand{\cl}{\mathrm{cl}}
\newcommand{\et}{\mathrm{\acute{e}t}}
\newcommand{\fet}{\mathrm{f\acute{e}t}}
\newcommand{\profet}{\mathrm{prof\acute{e}t}}
\newcommand{\proet}{\mathrm{pro\acute{e}t}}
\newcommand{\Zar}{\mathrm{Zar}}
\newcommand{\fppf}{\mathrm{fppf}}
\newcommand{\fpqc}{\mathrm{fpqc}}
\newcommand{\smooth}{\mathrm{sm}}
\newcommand{\sh}{\mathrm{sh}}
\newcommand{\op}{\mathrm{op}}
\newcommand{\open}{\mathrm{open}}
\newcommand{\closed}{\mathrm{closed}}
\newcommand{\geom}{\mathrm{geom}}
\newcommand{\alg}{\mathrm{alg}}
\newcommand{\sober}{\mathrm{sober}}
\newcommand{\dR}{\mathrm{dR}}
\newcommand{\rad}{\mathrm{rad}}
\newcommand{\discrete}{\mathrm{discrete}}
%\newcommand{\add}{\mathrm{add}}
%\newcommand{\lin}{\mathrm{lin}}
\newcommand{\Krull}{\mathrm{Krull}}
\newcommand{\qis}{\mathrm{qis}} %quasi-isomorphism
\newcommand{\ho}{\mathrm{ho}} %homotopy equivalence
\newcommand{\sep}{\mathrm{sep}}
\newcommand{\unr}{\mathrm{unr}}
\newcommand{\tame}{\mathrm{tame}}
\newcommand{\wild}{\mathrm{wild}}
\newcommand{\nil}{\mathrm{nil}}
\newcommand{\defm}{\mathrm{defm}}
\newcommand{\Art}{\mathrm{Art}}
\newcommand{\Noeth}{\mathrm{Noeth}}
\newcommand{\affd}{\mathrm{affd}}
%\newcommand{\adic}{\mathrm{adic}}
\newcommand{\pre}{\mathrm{pre}}
\newcommand{\perf}{\mathrm{perf}}
\newcommand{\perfd}{\mathrm{perfd}}
\newcommand{\rat}{\mathrm{rat}}
\newcommand{\cont}{\mathrm{cont}}
\newcommand{\dg}{\mathrm{dg}}
\newcommand{\almost}{\mathrm{a}}
%\newcommand{\stab}{\mathrm{stab}}
\newcommand{\heart}{\heartsuit}
\newcommand{\proj}{\mathrm{proj}}
\newcommand{\qproj}{\mathrm{qproj}}
\newcommand{\pd}{\mathrm{pd}}
\newcommand{\crys}{\mathrm{crys}}
\newcommand{\prisma}{\mathrm{prisma}}
\newcommand{\FF}{\mathrm{FF}}
\newcommand{\sph}{\mathrm{sph}}
\newcommand{\lax}{\mathrm{lax}}
\newcommand{\weak}{\mathrm{weak}}
\newcommand{\strict}{\mathrm{strict}}
\newcommand{\mon}{\mathrm{mon}}
\newcommand{\sym}{\mathrm{sym}}
\newcommand{\lisse}{\mathrm{lisse}}
\newcommand{\an}{\mathrm{an}}
\newcommand{\ad}{\mathrm{ad}}
\newcommand{\sch}{\mathrm{sch}}
\newcommand{\rig}{\mathrm{rig}}
\newcommand{\pol}{\mathrm{pol}}
\newcommand{\plat}{\mathrm{flat}}
\newcommand{\proper}{\mathrm{proper}}
\newcommand{\compl}{\mathrm{compl}}
\newcommand{\non}{\mathrm{non}}
\newcommand{\access}{\mathrm{access}}
\newcommand{\comp}{\mathrm{comp}}
\newcommand{\tstructure}{\mathrm{t}} %t-structures
\newcommand{\pure}{\mathrm{pure}} %pure motives
\newcommand{\mixed}{\mathrm{mixed}} %mixed motives
\newcommand{\num}{\mathrm{num}} %numerical motives
\newcommand{\ess}{\mathrm{ess}}
\newcommand{\topological}{\mathrm{top}}
\newcommand{\convex}{\mathrm{cv}}
\newcommand{\ab}{\mathrm{ab}} %abelian extensions
\newcommand{\surj}{\mathrm{surj}} %coverage on sets generated by surjections
\newcommand{\eff}{\mathrm{eff}} %effective Cartier divisors
\newcommand{\Weil}{\mathrm{Weil}} %weil divisors
\newcommand{\lex}{\mathrm{lex}}
\newcommand{\rex}{\mathrm{rex}}
\newcommand{\AR}{\mathrm{A\-R}}
\newcommand{\cons}{\mathrm{c}} %constructible sheaves
\newcommand{\tor}{\mathrm{tor}} %tor dimension
\newcommand{\semisimple}{\mathrm{ss}}

%prism custom command
\usepackage{relsize}
\usepackage[bbgreekl]{mathbbol}
\usepackage{amsfonts}
\DeclareSymbolFontAlphabet{\mathbb}{AMSb} %to ensure that the meaning of \mathbb does not change
\DeclareSymbolFontAlphabet{\mathbbl}{bbold}
\newcommand{\prism}{{\mathlarger{\mathbbl{\Delta}}}}
\newcommand{\toroidal}{\t}
\newcommand{\extendedtoroidal}{\hat{\t}}
\newcommand{\simpleroots}{\mathbb{I}}

\begin{document}

    \title{Are $\sigma_1$ and/or $\sigma_2$ coboundary ?}
    
    \author{Dat Minh Ha}
    \maketitle
    
    {
    \hypersetup{} 
    %\dominitoc
    %\tableofcontents %sort sections alphabetically
    }

    \begin{theorem}[Are the toroidal $2$-cocycles $\sigma_1, \sigma_2$ cohomologous to $0$ or not ?] \label{theorem: non_trivial_yangian_cocycles_examples}
            Let $\sigma_1, \sigma_2 \in Z^2_{\Lie}(\d_{[2]}, \z_{[2]})$ be as in example \ref{example: yangian_cocycles_(counter)_examples}. Then:
                $$\sigma_1 \in B^2_{\Lie}(\d_{[2]}, \z_{[2]})$$
            while:
                $$\sigma_2 \not \in B^2_{\Lie}(\d_{[2]}, \z_{[2]})$$
        \end{theorem}
        \todo[inline]{Fixed proof. The domain extension of $\tilde{\sigma}_i$ to $\d_{[2]}$ now depends on $i$.}
            \begin{proof}
                First of all, we remark that whatever any domain extension of $\tilde{\sigma}_i$ from $\frakw$ to all of $\d_{[2]}$ ends up being, it should depend on $i \in \{1, 2\}$: otherwise, we would have that $\sigma_1 = \sigma_2$ as elements of $Z^2_{\Lie}(\d_{[2]}, \z_{[2]})$, but we have already shown via the computations done in example \ref{example: yangian_cocycles_(counter)_examples} that this is not true! That aside, our task is similar to as in proposition \ref{prop: non_trivial_yangian_restricted_coboundaries_examples}, which is to find a linear map:
                    $$\tilde{\sigma}_i: \d_{[2]} \to \z_{[2]}$$
                (depending on $i$) such that:
                    $$\tilde{\sigma}_i([D, D']) = [D, \tilde{\sigma}_i(D')]_{\extendedtoroidal} - [D', \tilde{\sigma}_i(D)]_{\extendedtoroidal} - \sigma_i(D, D')$$
                for all $D, D' \in \d_{[2]}$, which we can take to be basis elements without any loss of generality.

                For what follows, let us recall:
                \begin{itemize}
                    \item from lemma \ref{lemma: explicit_commutators_between_basis_elements_of_toroidal_central_orthogonal_complement}, that:
                        $$[D_v, D_t] = 0$$
                        $$[D_v, D_{r, s}] = r D_{r, s + 1}$$
                        $$[D_t, D_{r, s}] = D_{r, s + 1}$$
                    \item from example \ref{example: yangian_cocycles_(counter)_examples} that:
                        $$\sigma_i(D_{r, s}, D_t) = \delta_{i, 1} r^2 s K_{-r, -s - 2}$$
                        $$\sigma_i(D_v, D_t) = 0$$
                    \item from lemma \ref{lemma: explicit_commutators_between_central_basis_elements_and_derivations} that:
                        $$
                            [D, K_{a, b}]_{\extendedtoroidal} =
                            \begin{cases}
                                \text{$((b - 1)r - sa) K_{a - r, b - s - 1} + \delta_{(r, s + 1), (a, b)} \left( r c_v + c_t \right)$ if $D = D_{r, s}$}
                                \\
                                \text{$-a K_{a, b - 1}$ if $D_v$}
                                \\
                                \text{$-K_{a, b - 1}$ if $D_t$}
                            \end{cases}
                        $$
                        $$[D, c_v]_{\extendedtoroidal} = [D, c_t]_{\extendedtoroidal} = 0$$
                    for all $D \in \d_{[2]}$.
                \end{itemize}

                Also, for what follows, suppose for all $D \in \d_{[2]}$ that:
                    $$\tilde{\sigma}_i(D) := \sum_{(a, b) \in \Z^2} \lambda_{a, b}(D, i) K_{a, b} + \lambda_v(D_v, i) c_v + \lambda_t(D, i) c_t$$
                \begin{enumerate}
                    \item To begin, let us make the following observation.
                    
                    From lemma \ref{lemma: explicit_commutators_between_basis_elements_of_toroidal_central_orthogonal_complement}, it is known that:
                        $$[D, D'] \in \bigoplus_{(\alpha, \beta) \in \Z^2} \bbC D_{\alpha, \beta}$$
                    for all $D, D' \in \d_{[2]}$, meaning in particular that there do not exist elements $D, D' \in \d_{[2]}$ such that either:
                        $$D_v = [D, D']$$
                    or:
                        $$D_t = [D, D']$$
                    and hence no such elements $D, D' \in \d_{[2]}$ so that:
                        $$\tilde{\sigma}_i(D_v) = [D, \tilde{\sigma}_i(D')]_{\extendedtoroidal} - [D', \tilde{\sigma}_i(D)]_{\extendedtoroidal} - \sigma_i(D, D')$$
                    or:
                        $$\tilde{\sigma}_i(D_t) = [D, \tilde{\sigma}_i(D')]_{\extendedtoroidal} - [D', \tilde{\sigma}_i(D)]_{\extendedtoroidal} - \sigma_i(D, D')$$
                    In conjunction with the fact that:
                        $$[D_v, D_{r, s - 1}] = r D_{r, s}, [D_t, D_{r, s - 1}] = D_{r, s}$$
                    this means that to compute $\tilde{\sigma}_i(D_v)$ and $\tilde{\sigma}_i(D_t)$, we must do so by computing:
                        $$
                            \begin{aligned}
                                & r\tilde{\sigma}_i(D_{r, s})
                                \\
                                = & r\left( [D_t, \tilde{\sigma}_i(D_{r, s - 1})]_{\extendedtoroidal} - [D_{r, s - 1}, \tilde{\sigma}_i(D_t)]_{\extendedtoroidal} - \sigma_i(D_t, D_{r, s}) \right)
                                \\
                                = & [D_v, \tilde{\sigma}_i(D_{r, s - 1})]_{\extendedtoroidal} - [D_{r, s - 1}, \tilde{\sigma}_i(D_v)]_{\extendedtoroidal} - \sigma_i(D_v, D_{r, s})
                            \end{aligned}
                        $$
                    \item We know also that:
                        $$\sigma_i(D_v, D_t) = 0, [D_v, D_t] = 0$$
                    and so:
                        $$[D_t, \tilde{\sigma}_i(D_v)]_{\extendedtoroidal} = [D_v, \tilde{\sigma}_i(D_t)]_{\extendedtoroidal}$$
                    Using the fact that $[D_v, K_{a, b}]_{\extendedtoroidal} = -a K_{-a, -b - 1}$, we then get that:
                        $$[D_t, \tilde{\sigma}_i(D_v)]_{\extendedtoroidal} = -\sum_{(a, b) \in \Z^2} a \lambda_{a, b + 1}(D_t, i) K_{a, b}$$
                    Since $[D_v, K_{a, b}]_{\extendedtoroidal} = -K_{-a, -b - 1}$, the above implies that:
                        $$\lambda_{a, b}(D_v, i) = a\lambda_{a, b}(D_t, i)$$
                    meaning that the coefficients $\lambda_{a, b}(D_v, i)$ are determined by $\lambda_{a, b}(D_t, i)$. 
                    \item In light of the observations made above, let us now attempt to compute:
                        $$\tilde{\sigma}_i(D_{r, s})$$
                    For this, let us use the fact that:
                        $$D_{r, s} = [D_t, D_{r, s - 1}]$$
                    in order to get:
                        $$
                            \begin{aligned}
                                & \tilde{\sigma}_i(D_{r, s})
                                \\
                                = & [D_t, \tilde{\sigma}_i(D_{r, s - 1})]_{\extendedtoroidal} - [D_{r, s - 1}, \tilde{\sigma}_i(D_t)]_{\extendedtoroidal} - \sigma_i(D_t, D_{r, s})
                                \\
                                = & [D_t, \tilde{\sigma}_i(D_{r, s - 1})]_{\extendedtoroidal} - [D_{r, s - 1}, \tilde{\sigma}_i(D_t)]_{\extendedtoroidal} + \delta_{i, 1} r^2 s K_{-r, -s - 2}
                            \end{aligned}
                        $$
                    For visual clarity, let us compute the two brackets individually:
                        $$
                            \begin{aligned}
                                & [D_t, \tilde{\sigma}_i(D_{r, s - 1})]_{\extendedtoroidal}
                                \\
                                = & -\sum_{(a, b) \in \Z^2} \lambda_{a, b}(D_{r, s - 1}, i) K_{a, b - 1}
                                \\
                                = & -\sum_{(a, b) \in \Z^2} \lambda_{a, b + 1}(D_{r, s - 1}, i) K_{a, b}
                            \end{aligned}
                        $$
                        $$
                            \begin{aligned}
                                & [D_{r, s - 1}, \tilde{\sigma}_i(D_t)]_{\extendedtoroidal} 
                                \\
                                = & \sum_{(a, b) \in \Z^2} \lambda_{a, b}(D_t, i) \left( (b - 1)r - (s - 1)a \right) K_{a - r, b - s} + \delta_{(r, s), (a, b)} \left( r c_v + c_t \right)
                                \\
                                = & \sum_{(a, b) \in \Z^2} \left( (b + s - 1)r - (s - 1)(a + r) \right) \lambda_{a + r, b + s}(D_t, i) K_{a, b} + \delta_{r, 0} c_t
                            \end{aligned}
                        $$
                    Now, by putting everything together, we yield:
                        $$\tilde{\sigma}_i(D_{r, s}) = -\sum_{(a, b) \in \Z^2} M_i(r, s, a, b) K_{a, b} + \delta_{i, 1} r^2 s K_{-r, -s - 2} + \delta_{r, 0} c_t$$
                    where:
                        $$M_i(r, s, a, b) := \lambda_{a, b + 1}(D_{r, s - 1}, i) + \left( (b + s - 1)r - (s - 1)(a + r) \right) \lambda_{a + r, b + s}(D_t, i)$$
                    In light of this and of the fact that:
                        $$\tilde{\sigma}_i(D_{r, -1}) = (-\frac12 r^2 + r) K_{r, 0} + \delta_{r, 0} c_t$$
                    which is known from proposition \ref{prop: non_trivial_yangian_restricted_coboundaries_examples}, and since we are only trying to prove the existence of $\tilde{\sigma}_i$, not to determine it explicitly, let us now declare that:
                        $$
                            M_i(r, s, a, b) =
                            \begin{cases}
                                \text{$\delta_{i, 1} r^2 s$ if $(a, b) = (-r, -s - 2)$}
                                \\
                                \text{$\frac12 r^2 - r$ if $(a, b) = (r, 0)$}
                                \\
                                \text{$0$ otherwise}
                            \end{cases}
                        $$
                    From this, we gather that when $(a, b) = (-r, -s - 2)$, we shall obtain the following linear system, which can be solved to obtain the coefficients $\lambda_{a, b}(D_{r, s}, i)$ and $\lambda_{2a, b}(D_t, i)$:
                        $$
                            \begin{cases}
                                \lambda_{-r, -s - 1}(D_{r, s - 1}, i) -3r\lambda_{0, -2}(D_t, i) = \delta_{i, 1} r^2 s
                                \\
                                \lambda_{r, 1}(D_{r, s - 1}, i) - (s - 1)r\lambda_{2r, s}(D_t, i) = \frac12 r^2 - r
                            \end{cases}
                        $$
                    Note, however, that the system is only consistent when $i = 1$. Indeed, when $i = 2$, we have that:
                        $$\lambda_{-r, -s - 1}(D_{r, s - 1}, i) -3r\lambda_{0, -2}(D_t, i) = 0$$
                    from which we see that, when $r = 1$, we would have that:
                        $$\lambda_{0, -2}(D_t, i) = \frac13 \lambda_{-1, -s - 1}(D_{1, s - 1}, i)$$
                    but this would mean that $\lambda_{0, -2}(D_t, i)$ depends on $s$, which is clearly nonsensical. Therefore, a domain extension of $\tilde{\sigma}_2$ from $\frakw$ to $\bigoplus_{(r, s) \in \Z^2} \bbC D_{r, s}$ (let alone to $\d_{[2]} := \bigoplus_{(r, s) \in \Z^2} \bbC D_{r, s} \oplus \bbC D_v \oplus \bbC D_t$) can not exist, i.e. $\sigma_2$ is not $2$-coboundary.
                        
                    \todo[inline]{Is it clear that the system is consistent even when $i = 1$ ?}
                    
                    As of now, we have already obtained:
                        $$\lambda_v(D_{r, s}, 1) = 0, \lambda_t(D_{r, s}, 1) = \delta_{r, 0}$$
                    We have thus shown that the values:
                        $$\tilde{\sigma}_1(D_{r, s}) = \sum_{(a, b) \in \Z^2} \lambda_{a, b}(D_{r, s}, 1) K_{a, b} + \delta_{r, 0} c_t$$
                    are well-defined (in particular, they depend on $i$), and so \textit{there is a domain extension of $\tilde{\sigma}_1$ from $\frakw := \bigoplus_{r \in \Z} \bbC D_{r, -1}$ to $\bigoplus_{(r, s) \in \Z^2} \bbC D_{r, s}$.}
                    \item Also, let us note that the above implies that:
                        $$\tilde{\sigma}_1(D_t) \not = 0$$
                    (and hence:
                        $$\tilde{\sigma}_1(D_v) \not = 0$$
                    as well) because otherwise, $\tilde{\sigma}_1(D_{r, s})$ would not have no non-zero summand lying in $\bbC c_t$, which would be a contradiction. Regardless, these values are well-defined, now that we know that:
                        $$\tilde{\sigma}_1: \bigoplus_{(r, s) \in \Z^2} \bbC D_{r, s} \to \z_{[2]}$$
                    is well-defined, again because we have that:
                        $$
                            \begin{aligned}
                                & \tilde{\sigma}_i(D_{r, s})
                                \\
                                = & r\left( [D_t, \tilde{\sigma}_i(D_{r, s - 1})]_{\extendedtoroidal} - [D_{r, s - 1}, \tilde{\sigma}_i(D_t)]_{\extendedtoroidal} - \sigma_i(D_t, D_{r, s}) \right)
                                \\
                                = & [D_v, \tilde{\sigma}_i(D_{r, s - 1})]_{\extendedtoroidal} - [D_{r, s - 1}, \tilde{\sigma}_i(D_v)]_{\extendedtoroidal} - \sigma_i(D_v, D_{r, s})
                            \end{aligned}
                        $$
                    A domain extension of $\tilde{\sigma}_1$ from $\bigoplus_{(r, s) \in \Z^2} \bbC D_{r, s}$ to $\d_{[2]}$ therefore exists.
                \end{enumerate}
            \end{proof}
    
    \addcontentsline{toc}{section}{References}
    \printbibliography

\end{document}